\documentclass[a4paper,12pt]{scrartcl}

\author{}
\title{London:\\ Sights, Food, and Shopping}
\date{}
\usepackage{enumitem} %Has to do with enumeration	
\usepackage{amsfonts}
\usepackage{amsmath}
\usepackage{amsthm} %allows for labeling of theorems
\usepackage[T1]{fontenc}
\usepackage[utf8]{inputenc}
\usepackage{blindtext}
\usepackage{graphicx}
\usepackage{hyperref} % For internal/external linking. 
				 % [hidelinks] removes boxes
\hypersetup{
    colorlinks=true,
    linkcolor=black,
    citecolor=black,
    filecolor=black,
    urlcolor=black,
}
% \usepackage{url} % allows for url display, non-clickable
%\numberwithin{equation}{section} 
   % This labels the equations in relation to the sections 
      % rather than other equations
%\numberwithin{equation}{subsection} %This labels relative to subsections
\newtheorem{thm}{Theorem}[section]
\newtheorem{lem}[thm]{Lemma}
\newtheorem{prop}[thm]{Proposition}
\newtheorem{cor}[thm]{Corollary}
\setkomafont{disposition}{\normalfont\bfseries}
\usepackage{appendix}
\usepackage{subfigure} % For plotting multiple figures at once
\usepackage{verbatim} % for including verbatim code from a file
\usepackage{natbib} % for bibliographies

\begin{document}
\maketitle

The first two sections are organized by checklists that you can
go through to make sure that you don't forget anything. Plus, if you 
open this in a pdf viewer, you'll be able to click on any of the 
places listed. This will jump you to section three for more details 
in a short profile.



\tableofcontents %adds it here


\newpage


\section{By Location}

Within each location subsection, you can walk or take the bus
anywhere.  Between two sections, you can take the bus (maybe)
if the sections are adjacent. Take the tube for anything further.
   
\subsection{East London}
   Defining very broadly as the City of London, the oldest portion
   where the original settlement was. It will be busy during
   the week, an absolute ghost-town on the weekend with lots of
   stuff closed.
   \\
   \\
   Click on any of the items below to jump to their profile and
   get more information.
   \begin{enumerate}
      \item \hyperlink{tol}{Tower of London} (Sights): Couple Hours
      \item \hyperlink{tbridge}{Tower Bridge} (Sights)
      \item \hyperlink{lead}{Leadenhall Market} 
	 (Sights, Food, Shopping)
      \item \hyperlink{camden}{Bank of England} (Sights)
      \item \hyperlink{stpaul}{St. Paul's Cathedral} (Sights) 
      \item \hyperlink{mill}{Millennium Bridge} (Sights)
      \item \hyperlink{tate}{Tate Modern} (Sights)
      \item \hyperlink{globe}{Shakespeare's Globe Theatre} (Sights)
   \end{enumerate}


\subsection{Central London}
I'm defining this roughly as west of The City until Buckingham, 
Westminster, and Hyde Park.
Again, click on sites to jump to a profile.
\begin{enumerate}
   \item \hyperlink{british}{British Museum} (Sights): Half or whole 
      day.
   \item \hyperlink{covent}{Covent Garden} (Sights, Food, Shopping):
      A couple hours.
   \item \hyperlink{traf}{Trafalgar Square} (Sights)
   \item \hyperlink{natgal}{National Gallery} (Sights, Shopping):
      Half to whole day.
   \item \hyperlink{sbank}{South Bank} (Sights, Food)
   \item \hyperlink{china}{Chinatown} (Food)
   \item \hyperlink{leic}{Leicester Square} (Sights)
   \item \hyperlink{oxst}{Oxford Street} (Shopping)
   \item \hyperlink{regend}{Regent Street} (Shopping)
   \item \hyperlink{picc}{Piccadilly} (Shopping)
\end{enumerate}


\subsection{Westminster and Kensington}
This is kind of like the Washington DC of London, where there's
a bigger concentration of government and museums.
\begin{enumerate}
   \item \hyperlink{buck}{Buckingham Palace}  (Sights): Half day inside,
      an hour or two for the guards.
   \item \hyperlink{horse}{Palace of Whitehall or Horse Guards 
      Parade}(Sights)
   \item \hyperlink{parl}{Parliament} (Sights)
   \item \hyperlink{abbey}{Westminster Abbey} (Sights): Half day.
   \item \hyperlink{hyde}{Hyde Park} (Sights): As much time as you need
      to relax.
   \item \hyperlink{harrods}{Harrods} (Food, Shopping): A half hour
      to two hours.
   \item \hyperlink{va}{Victoria and Albert Museum} (Sights): Half day
      to full day.
   \item \hyperlink{nathist}{Natural History Museum} (Sights): 
      Half day to full day.
   \item Pippa Middleton (Sights)
\end{enumerate}


\subsection{Other Places of Interest}
Here's some odds and ends. Click to jump to profiles.
\begin{enumerate}
   \item \hyperlink{camden}{Camden} (Food, Shopping)
   \item \hyperlink{oxbridge}{Cambridge and Oxford} (Sights): 
      Day trip, dinner included if you want.
   \item \hyperlink{windsor}{Windsor} (Sights): Day trip, back for
      dinner.
   \item \hyperlink{trains}{Train Stations} (Sights)
\end{enumerate}




\newpage
\section{By Type}

Now here's everything organized by type. Click on an item to 
jump to its profile.

\subsection{Sights}
This includes museums, picture spots, London icons, and pretty
much anything of note.
\begin{enumerate}
   \item \hyperlink{tol}{Tower of London} (East London): Couple hours
   \item \hyperlink{tbridge}{Tower Bridge} (East London)
   \item \hyperlink{lead}{Leadenhall Market} (East London)
   \item \hyperlink{boe}{Bank of England} (East London)
   \item \hyperlink{stpaul}{St. Paul's Cathedral} (East London) 
   \item \hyperlink{mill}{Millennium Bridge} (East London)
   \item \hyperlink{tate}{Tate Modern} (East London)
   \item \hyperlink{globe}{Shakespeare's Globe Theatre} (East London)
   \item \hyperlink{british}{British Museum} (Central London): Half
      to whole day.
   \item \hyperlink{covent}{Covent Garden} (Central London): Couple
      hours.
   \item \hyperlink{traf}{Trafalgar Square} (Central London)
   \item \hyperlink{natgal}{National Gallery} (Central London):
      Half to whole day.
   \item \hyperlink{sbank}{South Bank} (Central London)
   \item \hyperlink{leic}{Leicester Square} (Central London)
   \item \hyperlink{buck}{Buckingham Palace} (Central London): 
      Half day inside, hour or two for the guards.
   \item \hyperlink{horse}{Palace of Whitehall or Horse Guards Parade}
      (Central London)
   \item \hyperlink{parl}{Parliament} (Westminster)
   \item \hyperlink{abbey}{Westminster Abbey} (Westminster): Half day.
   \item \hyperlink{hyde}{Hyde Park} (Central London)
   \item \hyperlink{va}{Victoria and Albert Museum} (Central London):
      Half to whole day.
   \item \hyperlink{nathist}{Natural History Museum} (Central London):
      Half to whole day.
   \item \hyperlink{oxbridge}{Cambridge and Oxford} (Central London):
      Day trip, with dinner option there.
   \item \hyperlink{windsor}{Windsor} (Central London): Day tripe with
      dinner back in London.
   \item \hyperlink{trains}{Train Stations} (Everywhere)
\end{enumerate}



\subsection{Food}
This is pretty general locations, good for lunch or for dinner if you
can find some stands or a food festival. 
If you're going to go out to dinner, check Yelp for more specific 
suggestions.
\begin{enumerate}
   \item \hyperlink{lead}{Leadenhall Market} (East London)
   \item \hyperlink{covent}{Covent Garden} (Central London): Couple
      hours.
   \item \hyperlink{sbank}{South Bank} (Central London)
   \item \hyperlink{china}{Chinatown} (Central London)
   \item \hyperlink{camden}{Camden} (Other)
   \item \hyperlink{greggs}{Gregg's} (Everywhere)
\end{enumerate}



\subsection{Shopping}

This includes clothes shopping and gift shopping.
\begin{enumerate}
   \item \hyperlink{covent}{Covent Garden} (Central London)
   \item \hyperlink{natgal}{National Gallery} (Central London)
   \item \hyperlink{oxst}{Oxford Street} (Central London)
   \item \hyperlink{regent}{Regent Street} (Central London)
   \item \hyperlink{picc}{Piccadilly} (Central London)
   \item \hyperlink{harrods}{Harrods} (Kensington)
   \item \hyperlink{camden}{Camden} (Other)
\end{enumerate}



\newpage
\section{Profiles}




\paragraph{Bank of England} \hypertarget{boe}{I} made it there on
the weekend when it was closed, so I only got to see the outside.
But I think they have a tour during
the week. Given its place in monetary and financial history, it's a
really neat place to me.


\paragraph{British Museum} \hypertarget{british}{As} one lady from
King's told me, it houses ``everything the British nicked from
other countries. Better go before they have to give it back.'' 
It's absolutely amazing and a must-see. It's the best museum I've
been to, with the Rosetta stone, portions of the Parthenon, and other
remnants of Empire. You can spend a \textbf{whole day} here.


\paragraph{Buckingham Palace} \hypertarget{buck}{Changing}
Changing of the guards; happens only every two days I think, so plan
accordingly and get there early for a good spot.
The palace is only open to public for a portion of the year. 
Definitely get inside here or Windsor to see how ridiculously 
nice the royal family lives.


\paragraph{Cambridge and Oxford} \hypertarget{oxbridge}{Both}
Both are great for a weekend day trip. It's like an hour
by regional train and only about 10-20 pounds depending on
how early you book.
\begin{itemize}
   \item[-] 
      Cambridge and Oxford are both composed of constituent colleges.
      Each has its own Housing, Dining Hall, Classrooms, Libraries,
       and Church.
      Those colleges are typically ringed quads with a courtyard in the 
      middle. Access is restricted to the public, and even if it's open, 
      you typically have to pay. If you don't want to pay, do what I did:
      dress nicely and pretend you're a student. Then walk in really 
      confidently or looking at your phone. Worked out perfectly
       for me.
   \item[-] You can spend a whole day exploring the colleges, 
      walking around the town, visiting museums and chapels, eating, 
      etc. They're both great little quiet towns in their own right, 
      ancient and historic (you'll see where Newton taught). And they're 
      a great escape from the city.
   \item[-] {\sl Punting}: 
      Cambridge is built along a canal and you can get
      a gondola-like boat to travel. Plus, the college
      courtyards are typically open to the canal, meaning you can get
      views of the colleges that you can't get unless you go in.
      Punting is pretty difficult, but you get the hang of it after
      a while. Don't get a professional punter, go for it yourself
      with your friends and try not to fall in. 
\end{itemize}


\paragraph{Camden} \hypertarget{camden}{Very} 
very cool market. Has a Philly South Street vibe.
Plenty of young and weird people and odd shopping items.
Food stands all over with plenty of free samples of Chinese food and
other good stuff.  Located right along a canal, so you can sit along 
the water while you eat.
You'll might even see boats go through the canal and locks.



\paragraph{Chinatown} \hypertarget{china}{A}
little section of London, but obviously really great Chinese food.
Look for the restaurant with the \emph{least} white people, 
and you're bound to have a good meal.
Very centrally located near Leicester Square and Piccadilly.


\paragraph{Covent Garden} \hypertarget{covent}{An} open air shopping
center and market, centrally located. \\
Spend \textbf{an hour or two} here.
\begin{itemize}
   \item[-] 
      If you go for food, get the meatpies at the place on the lower 
      level (forget the name, but there's only one). 
      More generally, it
      has an okay selection of food when nothing special is going on.
   \item[-] But when stands set up, it has an AMAZING selection.
      When I was there, that happened every Wednesday for me, but
      check online and ask around. 
      There will be all kinds of cuisine plus a 
      killer American burger stand. 
   \item[-] Within Covent Garden, you'll find more 
      boutique-type shops---like tea 
      and artsy stuff. So it's a decent place for getting mom, sister,
      and aunt gifts.
    \item[-] In addition, there's 
      stores and shopping in the surrounding streets, similar
      to what you'd find on Oxford and Regent Streets. Plus,
      there's a huge Apple store.
   \item[-] Great, great
      area to wander and get lost. 
      I also remember a fair amount of restaurants within a 10 
      minute walk.
\end{itemize}


\paragraph{Gregg's} \hypertarget{greggs}{Great}
Great for breakfast and quick lunch. Good for muffins, donuts, and
traditional British pauper food (i.e. meat stuffed in pastry dough).
Pretty cheap, so you can make a habit out of it.


\paragraph{Harrods} \hypertarget{harrods}{A} bit like Saks Fifth
Avenue and Macy's in NYC put together, but it's, by no means, just a 
shopping place.
\begin{itemize}
   \item[-] There's a food market too where you should
      GET THE MEATPIES---some of the best I've had.
      The market is great for British staples more generally. 
   \item[-] Huge selection of chocolate, coffee, 
      and tea right next to the food market. 
   \item[-] While everything else is pretty expensive, 
      it's still awesome just to visit. Every room has a theme---even
      the escalators are themed (they're Egyptian).
   \item[-] Great place for family gifts that aren't crappy 
      like what you'd find in touristy places.
\end{itemize}
     

\paragraph{Hyde Park} \hypertarget{hyde}{A} great way to escape the
city. So absolutely enormous (like Central Park) that you'll
forget you're in London. In fact, there's parts where you can 
\emph{only} see grass and trees, no buildings. Watch out for
periodic events on the southeast corner.


\paragraph{Leadenhall Market} \hypertarget{lead}{Never} made it,
but looked like a neat place.


\paragraph{Leicester Square} \hypertarget{leic}{Where} movie
premieres happen in London. Also where a lot of clubs are.
Basically, it's the place where you can go at 2 or 3 am and still
see people and find food. (Although, at that time, you will
have to walk around some vomit on the sidewalk from people who
had a little too much fun.)


\paragraph{Millennium Bridge} \hypertarget{mill}{In} Harry Potter,
this was the bridge the Death Eaters sank into the Thames. 
It connects St. Paul's and the Tate Modern, so it has practical value
too. Not to be confused with the Willennium Bridge.


\paragraph{Natural History Museum} \hypertarget{nathist}{Right}
Right next to Victoria and Albert. Didn't go, but worth
checking out, especially because of it's location. \textbf{Half day
to whole day}. Maybe pair it with the Victoria and Albert Museum along
with Harrods to make a day.



\paragraph{National Gallery} \hypertarget{natgal}{Great},
great museum. Definitely see it (I think it's free); located
in Trafalgar Square. Also, the shop below has some neat art-related
gifts for people. Probably a \textbf{half-day} museum.


\paragraph{Oxford Street} \hypertarget{oxst}{Major}
international brands that require big stores, so
Uniqlo, a few hundred H\&M stores, etc.
Also, department stores along this street.
Stretches all the way to Primark at Hyde Park, which is like
Old Navy in 2000---cheap, way too crowded, and clothes thrown 
everywhere. Great if you need cheap, replacement staples like plain
t-shirts.


\paragraph{Palace of Whitehall or Horse Guards Parade} 
\hypertarget{horse}{It's} here,
along Whitehall street, that you'll be able to get up close to
the famous stoic British 
Guards in red garb and black hats. They're right on the street,
some are on horses.


\paragraph{Parliament} \hypertarget{parl}{Don't} know if you can get
in. I wasn't able to find out. But still neat to see. Get great
pictures of it from the South Bank by the London eye or from 
Westminster Bridge.


\paragraph{Piccadilly} \hypertarget{picc}{More}
expensive shopping, if I remember right.
Plus, they have these long arcades that shoot of Piccadilly. They're
covered walkways (somewhat indoor, but open at either end)
that have shops on either side. Pretty unique.


\paragraph{Regent Street} \hypertarget{regent}{Very}
Very much like 5th Avenue in New York. 
Major international brands. 


\paragraph{Shakespeare's Globe Theatre} \hypertarget{globe}{Definitely}
see a show here if you can, preferably a few. 
Granted, being a groundling kinda sucks,
so get a seat if you can. (But even if you're a groundling
like the poors of old, so very worth it.)


\paragraph{South Bank} \hypertarget{sbank}{Runs}
for miles along the south side of the Thames.
In many places, there are museums along it, restaurants and occasional
food festivals, open-air markets, and gorgeous views of London
(since the North Side is
built up with all the nice stuff).
One of the best places to get pictures of 
Parliament down by the London Eye. On a nice day, take a
\textbf{half- or one-hour} walk or so.


\paragraph{St. Paul's Cathedral}\hypertarget{stpaul}{This} is
the main Anglican church. Iconic dome can be seen from pretty much 
anywhere in the city. 


\paragraph{Tate Modern} \hypertarget{tate}{Didn't} make it
myself, but supposedly very good. They also own and operate the Globe
Theatre. Could do a \textbf{half day} here probably, and possible to
skip.


\paragraph{Tower Bridge} \hypertarget{tbridge}{Surprise}, it's right
next to the Tower of London. Not much on the south side, but a neat
place to visit, take a picture, maybe walk across.


\paragraph{Tower of London} \hypertarget{tol}{Missed} 
out because I'm an idiot, but my parents did the
guided tour and loved it. Crown Jewels are there. Weird hours of 
operation if I remember right. Tour was a \textbf{couple of hours}.


\paragraph{Trafalgar Square} \hypertarget{traf}{This} is a really
central location. It has the National Gallery one side, while
food markets and events sometimes spring up on the plaza. Two
streets shoot off the Square: The Mall, which goes straight to
Buckingham, and Whitehall, which goes straight to 
Parliament.\footnote{``Whitehall'' is colloquially the term for 
government agencies and function, since most (really nice)
government buildings lie along here---and that includes buildings
that James Bond goes to and stands on in the movies.}


\paragraph{Train Stations} \hypertarget{trains}{London}
London Train Stations are gorgeous. Every major station is 
    almost as nice or nicer than Grand Central and 30th
    Street Station.
Visit as many as you can. St. Pancras in particular is especially
    nice, and it's right next to King's Cross of Harry Potter fame.
Other nice stations include Waterloo, Charing Cross, Paddington,
and Victoria Stations.



\paragraph{Victoria and Albert Museum} \hypertarget{va}{Neat}
room of plaster casts of famous objects. Can spend a 
\textbf{half or whole day} here.
Maybe pair it with the Museum of Natural History along
with Harrods to make a day.



\paragraph{Westminster Abbey} \hypertarget{abbey}{The} audio tour
is absolutely great. Not only do you get to explore a beautiful
and ancient church, but there's old monastery grounds, British poets,
and monarchs throughout. Don't miss it. \textbf{Half day}.


\paragraph{Windsor} \hypertarget{windsor}{Short}
Short train ride (no more than 1.5 hrs).  
Windsor castle is composed of the grounds (courtyards, chapel, 
gift shops, museums) and the living quarters.
It's amazingly nice once you're inside the living quarters and very much 
worth seeing. It's also a pleasant small-town break from London.
Great for a \textbf{day trip};
Go in the morning, the make it back to London for dinner.






\end{document}



%%%% INCLUDING FIGURES %%%%%%%%%%%%%%%%%%%%%%%%%%%%

   % H indicates here 
   %\begin{figure}[h!]
   %   \centering
   %   \includegraphics[scale=1]{file.pdf}
   %\end{figure}

%   \begin{figure}[h!]
%      \centering
%      \mbox{
%	 \subfigure{
%	    \includegraphics[scale=1]{file1.pdf}
%	 }\quad
%	 \subfigure{
%	    \includegraphics[scale=1]{file2.pdf} 
%	 }
%      }
%   \end{figure}
 

%%%%% Including Code %%%%%%%%%%%%%%%%%%%%%5
% \verbatiminput{file.ext}    % Includes verbatim text from the file
% \texttt{text}	  % includes text in courier, or code-like, font
