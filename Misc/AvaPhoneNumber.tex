\documentclass[a4paper,12pt]{scrartcl}

\author{}
\title{}
\date{}
\usepackage{enumitem} %Has to do with enumeration	
\usepackage{amsfonts}
\usepackage{amsmath}
\usepackage{amsthm} %allows for labeling of theorems
\usepackage[T1]{fontenc}
\usepackage[utf8]{inputenc}
\usepackage{blindtext}
\usepackage{graphicx}
\usepackage{hyperref} % For internal/external linking. 
				 % [hidelinks] removes boxes
\hypersetup{
    colorlinks=true,
    linkcolor=black,
    citecolor=black,
    filecolor=black,
    urlcolor=black,
}
% \usepackage{url} % allows for url display, non-clickable
%\numberwithin{equation}{section} 
   % This labels the equations in relation to the sections 
      % rather than other equations
%\numberwithin{equation}{subsection} %This labels relative to subsections
\newtheorem{thm}{Theorem}[section]
\newtheorem{lem}[thm]{Lemma}
\newtheorem{prop}[thm]{Proposition}
\newtheorem{cor}[thm]{Corollary}
\setkomafont{disposition}{\normalfont\bfseries}
\usepackage{appendix}
\usepackage{subfigure} % For plotting multiple figures at once
\usepackage{verbatim} % for including verbatim code from a file
\usepackage{natbib} % for bibliographies

\begin{document}
\begin{center}
    \LARGE
    Phone Number Problem
\end{center}

\[ \text{Sample Phone Number}: \underbrace{123}_{x}-\underbrace{4567}_{y} \]

\paragraph{How to do the steps below}
\begin{itemize}
    \item Wherever you see $x$, replace it with the first three digits of your
	phone number (excluding the area code). So Step 1 below would mean 
	``Start with the first three digits of your phone number, and add 2.
	Press =.''
    \item Similarly, when you see $y$, replace it with the last four digits of your
	phone number.
    \item So if your phone number were ``123-4567'' (like above), 
	wherever you see $x$, you'd use ``123,'' and wherever you
	see $y$, you'd use ``4567.''
\end{itemize}

\paragraph{Steps to get back your phone number}
\begin{enumerate}
    \item Start with $x$, and add 2. Press =.
    \item Multiply the result by $y$. Press =.
    \item Divide by $x$. Press =.
    \item Subtract $y$. Press =.
    \item Multiply by $x$. Press =.
    \item Divide by 20,000. Press =. (Should look familiar.)
    \item Add $x$. Press =.
    \item Multiply by 10,000. Press =.
\end{enumerate}

\paragraph{Abstraction}
Finally, I used $x$ and $y$ above instead of ``the first three digits'' and ``the last four 
digits'' when I wrote this because it hints at the idea of \emph{abstraction}, 
which is probably the most important idea in math.
\\
\\
Specifically, math is great because it works \emph{no matter what} the numbers 
represent and stand for. Math ``abstracts'' from specific things. So
``2+2=4'' holds whether you're adding up chickens, two dollar bills in
a piggy bank, or planets a million miles away. 
\\
\\
Now what I did above with $x$ and $y$ takes that exact same idea one step further.  
While arithmetic lets numbers stand
for \emph{any old things} (whatever those things may be), I used $x$ and 
$y$ to stand for \emph{any old numbers} (whatever those numbers may be, or
whatever your phone number may be).
\\
\\
It turns out, that's a really good thing to do! Remember that arithmetic 
and its rules let you add up chickens, money, and planets without having them 
right there in front of you. Well, having letters 
stand for numbers allows you
to handle \emph{any} number without having it right
there in front of you.  Using $x$ and $y$, I could describe what to
do with \emph{any} phone number quickly, simply, and precisely without
ever knowing what exactly that phone number is. 
\\
\\
So if you understand that, Miss Ava, you'll understand and appreciate
much more than I did when I was in your grade.  Plus, you'll be well on your
way to algebra and the kind of fun, interesting math that
tells you how quickly gravity will push you down your bed's slide,
how fast you'd have to drive from your house to beat Uncle
Jimmy to the beach,
or even how to make up your own crazier version of the steps I
wrote above. 

    


\end{document}



%%%% INCLUDING FIGURES %%%%%%%%%%%%%%%%%%%%%%%%%%%%

   % H indicates here 
   %\begin{figure}[h!]
   %   \centering
   %   \includegraphics[scale=1]{file.pdf}
   %\end{figure}

%   \begin{figure}[h!]
%      \centering
%      \mbox{
%	 \subfigure{
%	    \includegraphics[scale=1]{file1.pdf}
%	 }\quad
%	 \subfigure{
%	    \includegraphics[scale=1]{file2.pdf} 
%	 }
%      }
%   \end{figure}
 

%%%%% Including Code %%%%%%%%%%%%%%%%%%%%%5
% \verbatiminput{file.ext}    % Includes verbatim text from the file
% \texttt{text}	  % includes text in courier, or code-like, font
