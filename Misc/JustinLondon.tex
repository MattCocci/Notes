\documentclass[a4paper,12pt]{scrartcl}

\author{}
\title{London:\\ General Tips and Advice}
\date{}
\usepackage{enumitem} %Has to do with enumeration	
\usepackage{amsfonts}
\usepackage{amsmath}
\usepackage{amsthm} %allows for labeling of theorems
\usepackage[T1]{fontenc}
\usepackage[utf8]{inputenc}
\usepackage{blindtext}
\usepackage{graphicx}
\usepackage{hyperref} % For internal/external linking. 
				 % [hidelinks] removes boxes
\hypersetup{
    colorlinks=true,
    linkcolor=black,
    citecolor=black,
    filecolor=black,
    urlcolor=black,
}
% \usepackage{url} % allows for url display, non-clickable
%\numberwithin{equation}{section} 
   % This labels the equations in relation to the sections 
      % rather than other equations
%\numberwithin{equation}{subsection} %This labels relative to subsections
\newtheorem{thm}{Theorem}[section]
\newtheorem{lem}[thm]{Lemma}
\newtheorem{prop}[thm]{Proposition}
\newtheorem{cor}[thm]{Corollary}
\setkomafont{disposition}{\normalfont\bfseries}
\usepackage{appendix}
\usepackage{subfigure} % For plotting multiple figures at once
\usepackage{verbatim} % for including verbatim code from a file
\usepackage{natbib} % for bibliographies

\begin{document}
\maketitle

\tableofcontents %adds it here

\newpage
\section{Travel and Transport}

\subsection{Flying, Airports, International Travel}

London has a bunch of airports (at least 3-4 that you'll 
use frequently). They vary by size, destination, and servicing
airlines. Some things to keep in mind:

\begin{itemize}
   \item[-] Each airport will have a train link that goes to a different 
      station in central London---e.g. Heathrow leads to Paddington
      Station, Luton to Liverpool Station, and Gatwick to Victoria.
   \item[-] 
      The rail links may be express links (only airport to station),
      local rail (stops along the way), or even Tube connections
      (like Paddington). Oftentimes, there will be a mix of these
      with buses too. So check prices and see which you prefer.
   \item[-] For your first time in, a private car might be good
      with a lot of luggage and since you won't be as familiar with
      the transport.
\end{itemize}

\subsection{Traveling within London}

Three main ways of transport: the Tube (or Underground), bus, cab. 
Avoid cab when possible, using it only as a last resort. Here's
some of the main stuff:

\subsubsection{Oyster Card}  An extra 5 pounds or so, but 100\% worth it.
   Here's some features.
   \begin{itemize}
      \item[-] Like NYC's MetroCard. Preload an amount onto the card or
	  buy a daily/weekly/monthly pass. (See how often you ride
	  before you buy passes.) 
       \item[-] On the bus, you swipe once before you get on. On the
	  Tube, you swipe in and out, so they can keep track of
	  how far you traveled (i.e. if you changed zones, see below).
       \item[-] If you make a certain amount of trips via bus or subway 
	  and  would have been better off buying the pass but didn't, 
	  your Oyster card will automatically buy a daily pass for you, 
	  and additional trips that day cost nothing.  
	  (Don't think it works for weekly/monthly)
       \item[-] You CANNOT reload at bus stops if I remember right, 
	  only in Tube stations. So make a point to do so if you 
	  plan to take the bus.

   \end{itemize}

\subsubsection{The Tube and its Zones} 
The zones are concentric circles around London's sprawl representing 
different fares. If you change zones during the course of your 
Tube ride, they deduct more.
You will mostly stay within Zone 1 (Central London). Camden might
be the only time you venture into a different zone.


\subsubsection{Local vs. Express on the Tube}
Most lines have express trains and local trains running on 
the same tracks.  Look for that on the train's front and side and
take care if your stop is a local stop only. 
For your (what looked like) fairly lengthy trips from home to class or
work,
express vs. local will make a big time difference, especially during 
rush hour. So get a practice run or two in before you start.

\subsubsection{The Tube vs. The Bus} 
Unless you have a really long trip, take the bus. A few reasons:
\begin{enumerate}
   \item The subway has a lot of walking underground, especially for 
      some transfers. Adding in the time to takes for the Tube train
      to arrive, the bus will be better for short trips.
   \item Since a lot of bus routes cover the same tracks in Central
      London (only splitting off later, outside the center), buses
      will come more often.
   \item So if you're simply going straight down a main thoroughfare or 
      if you're traveling within a neighborhood or two, the bus
      will likely be quicker.
   \item Sit on the second deck.  This is a great way to see London
      and it's much, much better than scurrying around underground.
\end{enumerate}

\subsubsection{Bus Maps}
The bus maps when you first see them are super confusing, took
 me a bit to figure them out. So check out 
 \url{http://www.runcheaptravel.com/?p=33}
 for a site which explains everything how to read them. It's worth it.  
 Now here's some important stuff:
 \begin{itemize}
   \item[-] The main thing to understand is that there are general 
	  locations where a ton of different buses for different
	  routes stop, like  "Liverpool Street" or
	  "Trafalgar Square." But these are huge, busy locations.
	  So at these general locations there are individual
	  stops for particular routes (since they don't want 20 bus
	  routes to try to load people at a single stop).
   \item[-] So to get from A to B via bus, you go roughly to A, find
	  a map that shows the individual route stops/pickups and
	  dropoffs
	  within a minute or two walk from A. You then take the bus to
	  one of the many stops for roughly location B.
    \item[-] {\sl Heritage Lines}: A few lines (the 9 and 15 
	 being two examples)
	  run old buses along them. These lines are typically
	  major lines that you would take anyway. The old buses
	  are the early and iconic double-deckers. Very cool, and
	  worth waiting for at least once.	
\end{itemize}

\subsubsection{Night Bus}
The tube shuts down at midnight, which is a real bummer.
From there, you have two options: Cab or the Night Bus.
Night buses don't stop as many places as regular buses and they
don't come as often, but they hit all the major spots.
Check the charts at the stops to get more info.


\section{General Food Tips}

Aside from specific food places that I'll give you, here's some
general tips:
\begin{enumerate}
   \item \textbf{Yelp}: 
      In addition to the places below, use Yelp extensively
	     to find good places.
   \item If you're going out to dinner, never eat in the same place
       twice.  There's so many food options, try as much as you can.
   \item If you recognize a restaurant, chain, or brand from the US, 
      AVOID IT. It will almost certainly be overpriced and crappier 
      than what you'd get here. Plus, it's no fun.
   \item {\sl Eating Cheap}: 
      Try the student union building. Also, go to 
       Tesco's and Sainsbury's for their 3 pound lunch specials where
       you get a sandwich, chips, and drink.
    \item {\sl Coffee}: If you're a coffee drinker, 
      local coffee shop $\geq$ Café Nero >>
      Starbucks. (Starbucks, like most American brands in London,
      is way overpriced. They swap \$ for \pounds $\;$ without changing
      the numbers.)
   \item {\sl Groceries}: In decreasing order of price: 
      Sainsbury's, Tesco, Iceland.
   \item {\sl Food markets}: If you're going to cook, 
      go to them for fresh bread, meat, fruit, and vegetables. 
      There's many all over town.
   \item \textbf{Food Festivals}: THE BEST THING. 
      They happen sporadically in major locations like South Bank, 
      Hyde Park, Trafalgar Square, etc. Really
      keep an eye out for them. In Covent Garden, there's also
      food stands set up every week specializing in different cuisine.
   \item Eat Internationally: London's one of the most international
	  cities out there, so it has every type of food.
	  Use that to try dishes and cuisine's you haven't had
	  before. It's likely to be better here than most other places.
\end{enumerate}


\newpage

\section{Money}

Just a few things:
\begin{itemize}
   \item[-] Use your credit card if you can to get the best conversion 
      rate.
   \item[-] Have a debit card that onto which you can load money online
      (for you or your parents to do). Then go to ATMs to withdraw
      cash from this card.  That way, it's not tied to your account
      (so no fraud issues) and it gives a reasonable exchange rate.
      I used AAA for this. They have a travel card like that.
   \item[-] If you're going abroad, this system should work well too.
\end{itemize}



\section{Walking Around}

Get a map, wander, and get lost. The non-grid streets mean you'll find
awesome stuff in unexpected places.  Learn a few main thoroughfares and
long stretches of the same street to help orient yourself.  
That way, if you get lost, you can find that street and find your way 
North, South, East or West.




\section{Lingo, Conventions, and Blending In}

\begin{enumerate}
   \item STAND ON THE RIGHT. This is perhaps the most important thing
       I write. On escalators, don't stand on the lefthand side---it's
       for passing and walking purposes only---especially in Tube 
       stations. People will get British-upset 
       with you, where they mutter under their breath and 
       get annoyed to themselves. 
       (This differs from New York upset, where people will
       directly tell you about it.)
   \item  "Quid" means "pounds." Took me a few days to realize.
   \item "Leicester" is pronounced "Lester." Any word/name
      with that kind of spelling is also said the same way. It's
      one way not to stick out as a newb. 
   \item If you wear a baseball cap and/or sweatpants, people will 
      assume you are an American. Just a note if you're going somewhere 
      that you don't want to stick out (like 
      Southern Europe where there might be pickpockets).
   \item If you see a friend or someone from the US doing 
      something embarrassing, offensive, or stupid,
      the magic words are "He's from Canada."
\end{enumerate}



\section{Phones}

Some places to look are Carphone Warehouse and Phones4U, which
have multiple service provides for you to compare plans.
Ask lots of questions about the servicers and plans because
the employees might be pretty useless and might not be
very forthcoming.



\section{Electronics}

Electronics are simple if you know the difference between converters
and adapters. I didn't and fried some stuff, so hopefully I can 
prevent that for you. 

\subsection{Adapters} 
This is the simplest type of widget. All it does is change the prongs.
Since the US prongs don't fit in a British or European socket, 
you need an adapter. You plug your us cord into the adapter
then the adapter into the wall. 

\subsection{Converters} Here's where things get a bit tricky. But
the good news is, for most electronics like laptops and phones, you
don't need one. You just need an adapter. Anyway, here's more
specifics:
\begin{enumerate}
   \item The electricity that comes out of non-US sockets is typically 
      120 volts. For non-US countries, it's typically 220 or 240 volts.
      That means you need to convert the voltage so that you don't
      fry your electronics.
   \item A converter will dial down the input voltage and so that
      you can use US power cords that don't accept
      220 or 240 volts while there.
   \item Note, sometimes a converter and adapter will be bundled
      in one, so it can get jumbled. But keep the two 
      purposes separate in your mind.
   \item Now, to determine whether you need one, pick up a laptop
      or phone power cord. Look at the little tape wrapped around
      the cord or at the really tiny lettering that talks about
      watts and voltage.
   \item For a lot of major electronics, laptop chargers and smartphone
      chargers, you'll see ``Input: 120-240V.'' That means the charger
      accepts \emph{any} voltage in that range, and congrats---you
      \emph{don't} need a converter, just an adapter. (Also, if 
      your power cord doesn't need it, 
      definitely don't run your cord through a converter. That's 
      also bad, and I learned that
      the hard way before I figured out all this stuff.)
   \item For some other electronics, you might just see ``Input: 120V''
      If that's the case, you need a converter.
\end{enumerate}



     
\newpage
\section{Studying in London}

\begin{itemize}
   \item[-] 
      LSE is part of the broader University of London (UL), so you have
	  access to LSE resources, shared UL-wide resources, and
	  resources of \emph{any} UL member university (King's, UCL,
	  etc.).
   \item[-] The main UL stuff is up by the British Museum. Use their 
	  library for books that you don't want to buy (they have
	  plenty) and to study. The library's gorgeous.
   \item[-]
      Waterstones is like Barnes and Noble. If you're buying books,
	  the Waterstones on Gower Street is the main one for UL 
	  students, it's massive, and it's quite nice.
\end{itemize}




%%%% APPPENDIX %%%%%%%%%%%

% \appendix

%\cite{LabelInSourcesFile} 
%\citep{LabelInSourcesFile} Cites in parens
%\nocite{LabelInSourceFile} includes in refs w/o specific citation
%\bibliographystyle{apalike} 
%\bibliography{sources.bib} where sources.bib is file




\end{document}



%%%% INCLUDING FIGURES %%%%%%%%%%%%%%%%%%%%%%%%%%%%

   % H indicates here 
   %\begin{figure}[h!]
   %   \centering
   %   \includegraphics[scale=1]{file.pdf}
   %\end{figure}

%   \begin{figure}[h!]
%      \centering
%      \mbox{
%	 \subfigure{
%	    \includegraphics[scale=1]{file1.pdf}
%	 }\quad
%	 \subfigure{
%	    \includegraphics[scale=1]{file2.pdf} 
%	 }
%      }
%   \end{figure}
 

%%%%% Including Code %%%%%%%%%%%%%%%%%%%%%5
% \verbatiminput{file.ext}    % Includes verbatim text from the file
% \texttt{text}	  % includes text in courier, or code-like, font
