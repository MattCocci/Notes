\documentclass[12pt]{article}

\author{Matthew D. Cocci}
\title{Title}
\date{\today}

%% Formatting & Spacing %%%%%%%%%%%%%%%%%%%%%%%%%%%%%%%%%%%%

%\usepackage[top=1in, bottom=1in, left=1in, right=1in]{geometry} % most detailed page formatting control
\usepackage{fullpage} % Simpler than using the geometry package; std effect
\usepackage{setspace}
%\onehalfspacing
\usepackage{microtype}

%% Formatting %%%%%%%%%%%%%%%%%%%%%%%%%%%%%%%%%%%%%%%%%%%%%

%\usepackage[margin=1in]{geometry}
    %   Adjust the margins with geometry package
%\usepackage{pdflscape}
    %   Allows landscape pages
%\usepackage{layout}
    %   Allows plotting of picture of formatting
\usepackage{version}
\excludeversion{solutions}
%\includeversion{solutions}



%% Header %%%%%%%%%%%%%%%%%%%%%%%%%%%%%%%%%%%%%%%%%%%%%%%%%

%\usepackage{fancyhdr}
%\pagestyle{fancy}
%\lhead{}
%\rhead{}
%\chead{}
%\setlength{\headheight}{15.2pt}
    %   Make the header bigger to avoid overlap

%\fancyhf{}
    %   Erase header settings

%\renewcommand{\headrulewidth}{0.3pt}
    %   Width of the line

%\setlength{\headsep}{0.2in}
    %   Distance from line to text


%% Mathematics Related %%%%%%%%%%%%%%%%%%%%%%%%%%%%%%%%%%%

\usepackage{amsmath}
\usepackage{amssymb}
\usepackage{amsfonts}
\usepackage{mathrsfs}
\usepackage{mathtools}
\usepackage{amsthm} %allows for labeling of theorems
%\numberwithin{equation}{section} % Number equations by section
\usepackage{bbm} % For bold numbers

\theoremstyle{plain}
\newtheorem{thm}{Theorem}[section]
\newtheorem{lem}[thm]{Lemma}
\newtheorem{prop}[thm]{Proposition}
\newtheorem{cor}[thm]{Corollary}

\theoremstyle{definition}
\newtheorem{defn}[thm]{Definition}
\newtheorem{ex}[thm]{Example}

\theoremstyle{remark}
\newtheorem*{rmk}{Remark}
\newtheorem*{note}{Note}

% Below supports left-right alignment in matrices so the negative
% signs don't look bad
\makeatletter
\renewcommand*\env@matrix[1][c]{\hskip -\arraycolsep
  \let\@ifnextchar\new@ifnextchar
  \array{*\c@MaxMatrixCols #1}}
\makeatother


%% Font Choices %%%%%%%%%%%%%%%%%%%%%%%%%%%%%%%%%%%%%%%%%

\usepackage[T1]{fontenc}
\usepackage{lmodern}
\usepackage[utf8]{inputenc}
%\usepackage{blindtext}
\usepackage{courier}


%% Figures %%%%%%%%%%%%%%%%%%%%%%%%%%%%%%%%%%%%%%%%%%%%%%

\usepackage{tikz}
\usetikzlibrary{decorations.pathreplacing}
\usetikzlibrary{arrows.meta}
\usepackage{graphicx}
\usepackage{subfigure}
    %   For plotting multiple figures at once
%\graphicspath{ {Directory/} }
    %   Set a directory for where to look for figures


%% Hyperlinks %%%%%%%%%%%%%%%%%%%%%%%%%%%%%%%%%%%%%%%%%%%%
\usepackage{hyperref}
\hypersetup{%
    colorlinks,
        %   This colors the links themselves, not boxes
    citecolor=black,
        %   Everything here and below changes link colors
    filecolor=black,
    linkcolor=black,
    urlcolor=black
}

%% Colors %%%%%%%%%%%%%%%%%%%%%%%%%%%%%%%%%%%%%%%%%%%%%%%

\usepackage{color}
\definecolor{codegreen}{RGB}{28,172,0}
\definecolor{codelilas}{RGB}{170,55,241}

% David4 color scheme
\definecolor{d4blue}{RGB}{100,191,255}
\definecolor{d4gray}{RGB}{175,175,175}
\definecolor{d4black}{RGB}{85,85,85}
\definecolor{d4orange}{RGB}{255,150,100}

%% Including Code %%%%%%%%%%%%%%%%%%%%%%%%%%%%%%%%%%%%%%%

\usepackage{verbatim}
    %   For including verbatim code from files, no colors
\usepackage{listings}
    %   For including code snippets written directly in this doc

\lstdefinestyle{bash}{%
  language=bash,%
  basicstyle=\footnotesize\ttfamily,%
  showstringspaces=false,%
  commentstyle=\color{gray},%
  keywordstyle=\color{blue},%
  xleftmargin=0.25in,%
  xrightmargin=0.25in
}
\lstdefinestyle{log}{%
  basicstyle=\scriptsize\ttfamily,%
  showstringspaces=false,%
  xleftmargin=0.25in,%
  xrightmargin=0.25in
}


\lstdefinestyle{matlab}{%
  language=Matlab,%
  basicstyle=\footnotesize\ttfamily,%
  breaklines=true,%
  morekeywords={matlab2tikz},%
  keywordstyle=\color{blue},%
  morekeywords=[2]{1}, keywordstyle=[2]{\color{black}},%
  identifierstyle=\color{black},%
  stringstyle=\color{codelilas},%
  commentstyle=\color{codegreen},%
  showstringspaces=false,%
    %   Without this there will be a symbol in
    %   the places where there is a space
  %numbers=left,%
  %numberstyle={\tiny \color{black}},%
    %   Size of the numbers
  numbersep=9pt,%
    %   Defines how far the numbers are from the text
  emph=[1]{for,end,break,switch,case},emphstyle=[1]\color{blue},%
    %   Some words to emphasise
}

\newcommand{\matlabcode}[1]{%
    \lstset{style=matlab}%
    \lstinputlisting{#1}
}
    %   For including Matlab code from .m file with colors,
    %   line numbering, etc.

\lstdefinelanguage{Julia}%
  {morekeywords={abstract,break,case,catch,const,continue,do,else,elseif,%
      end,export,false,for,function,immutable,import,importall,if,in,%
      macro,module,otherwise,quote,return,switch,true,try,type,typealias,%
      using,while},%
   sensitive=true,%
   %alsoother={$},%
   morecomment=[l]\#,%
   morecomment=[n]{\#=}{=\#},%
   morestring=[s]{"}{"},%
   morestring=[m]{'}{'},%
}[keywords,comments,strings]

\lstdefinestyle{julia}{%
    language         = Julia,
    basicstyle       = \scriptsize\ttfamily,
    keywordstyle     = \bfseries\color{blue},
    stringstyle      = \color{codegreen},
    commentstyle     = \color{codegreen},
    showstringspaces = false,
    literate         = %
      {ρ}{{$\rho$}}1
      {ℓ}{{$\ell$}}1
      {∑}{{$\Sigma$}}1
      {Σ}{{$\Sigma$}}1
      {√}{{$\sqrt{}$}}1
      {θ}{{$\theta$}}1
      {ω}{{$\omega$}}1
      {ɛ}{{$\varepsilon$}}1
      {φ}{{$\varphi$}}1
      {σ²}{{$\sigma^2$}}1
      {Φ}{{$\Phi$}}1
      {ϕ}{{$\phi$}}1
      {Dₑ}{{$D_e$}}1
      {Σ}{{$\Sigma$}}1
      {γ}{{$\gamma$}}1
      {δ}{{$\delta$}}1
      {τ}{{$\tau$}}1
      {μ}{{$\mu$}}1
      {β}{{$\beta$}}1
      {Λ}{{$\Lambda$}}1
      {λ}{{$\lambda$}}1
      {r̃}{{$\tilde{\text{r}}$}}1
      {α}{{$\alpha$}}1
      {σ}{{$\sigma$}}1
      {π}{{$\pi$}}1
      {∈}{{$\in$}}1
      {∞}{{$\infty$}}1
}


%% Bibliographies %%%%%%%%%%%%%%%%%%%%%%%%%%%%%%%%%%%%

%\usepackage{natbib}
    %---For bibliographies
%\setlength{\bibsep}{3pt} % Set how far apart bibentries are

%% Misc %%%%%%%%%%%%%%%%%%%%%%%%%%%%%%%%%%%%%%%%%%%%%%

\usepackage{enumitem}
    %   Has to do with enumeration
\usepackage{appendix}
%\usepackage{natbib}
    %   For bibliographies
\usepackage{pdfpages}
    %   For including whole pdf pages as a page in doc
\usepackage{pgffor}
    %   For easier looping


%% User Defined %%%%%%%%%%%%%%%%%%%%%%%%%%%%%%%%%%%%%%%%%%

%\newcommand{\nameofcmd}{Text to display}
\newcommand*{\Chi}{\mbox{\large$\chi$}} %big chi
    %   Bigger Chi

% In math mode, Use this instead of \munderbar, since that changes the
% font from math to regular
\makeatletter
\def\munderbar#1{\underline{\sbox\tw@{$#1$}\dp\tw@\z@\box\tw@}}
\makeatother

% Misc Math
\newcommand{\ra}{\rightarrow}
\newcommand{\diag}{\text{diag}}
\newcommand{\proj}{\operatorname{proj}}
\newcommand{\ch}{\text{ch}}
\newcommand{\dom}{\text{dom}}
\newcommand{\one}[1]{\mathbf{1}_{#1}}


% Command to generate new math commands:
% - Suppose you want to refer to \boldsymbol{x} as just \bsx, where 'x'
%   is any letter. This commands lets you generate \bsa, \bsb, etc.
%   without copy pasting \newcommand{\bsa}{\boldsymbol{a}} for each
%   letter individually. Instead, just include
%
%     \generate{bs}{\boldsymbol}{a,...,z}
%
% - Uses pgffor package to loop
% - Example with optional argument. Will generate \bshatx to represent
%   \boldsymbol{\hat{x}} for all letters x
%
%     \generate[\hat]{bshat}{\boldsymbol}{a,...,z}

\newcommand{\generate}[4][]{%
  % Takes 3 arguments (maybe four):
  % - 1   wrapcmd (optional, defaults to nothing)
  % - 2   newname
  % - 3   mathmacro
  % - 4   Names to loop over
  %
  % Will produce
  %
  %   \newcommand{\newnameX}{mathmacro{wrapcmd{X}}}
  %
  % for each X in argument 4

  \foreach \x in {#4}{%
    \expandafter\xdef\csname%
      #2\x%
    \endcsname%
    {\noexpand\ensuremath{\noexpand#3{\noexpand#1{\x}}}}
  }
}


% MATHSCR: Gen \sX to stand for \mathscr{X} for all upper case letters
\generate{s}{\mathscr}{A,...,Z}


% BOLDSYMBOL: Generate \bsX to stand for \boldsymbol{X}, all upper and
% lower case.
%
% Letters and greek letters
\generate{bs}{\boldsymbol}{a,...,z}
\generate{bs}{\boldsymbol}{A,...,Z}
\newcommand{\bstheta}{\boldsymbol{\theta}}
\newcommand{\bsmu}{\boldsymbol{\mu}}
\newcommand{\bsSigma}{\boldsymbol{\Sigma}}
\newcommand{\bsvarepsilon}{\boldsymbol{\varepsilon}}
\newcommand{\bsalpha}{\boldsymbol{\alpha}}
\newcommand{\bsbeta}{\boldsymbol{\beta}}
\newcommand{\bsOmega}{\boldsymbol{\Omega}}
\newcommand{\bshatOmega}{\boldsymbol{\hat{\Omega}}}
\newcommand{\bshatG}{\boldsymbol{\hat{G}}}
\newcommand{\bsgamma}{\boldsymbol{\gamma}}
\newcommand{\bslambda}{\boldsymbol{\lambda}}

% Special cases like \bshatb for \boldsymbol{\hat{b}}
\generate[\hat]{bshat}{\boldsymbol}{b,y,x,X,V,S,W}
\newcommand{\bshatbeta}{\boldsymbol{\hat{\beta}}}
\newcommand{\bshatmu}{\boldsymbol{\hat{\mu}}}
\newcommand{\bshattheta}{\boldsymbol{\hat{\theta}}}
\newcommand{\bshatSigma}{\boldsymbol{\hat{\Sigma}}}
\newcommand{\bstildebeta}{\boldsymbol{\tilde{\beta}}}
\newcommand{\bstildetheta}{\boldsymbol{\tilde{\theta}}}
\newcommand{\bsbarbeta}{\boldsymbol{\overline{\beta}}}
\newcommand{\bsbarg}{\boldsymbol{\overline{g}}}

% Redefine \bso to be the zero vector
\renewcommand{\bso}{\boldsymbol{0}}

% Transposes of all the boldsymbol shit
\newcommand{\bsbp}{\boldsymbol{b'}}
\newcommand{\bshatbp}{\boldsymbol{\hat{b'}}}
\newcommand{\bsdp}{\boldsymbol{d'}}
\newcommand{\bsgp}{\boldsymbol{g'}}
\newcommand{\bsGp}{\boldsymbol{G'}}
\newcommand{\bshp}{\boldsymbol{h'}}
\newcommand{\bsSp}{\boldsymbol{S'}}
\newcommand{\bsup}{\boldsymbol{u'}}
\newcommand{\bsxp}{\boldsymbol{x'}}
\newcommand{\bsyp}{\boldsymbol{y'}}
\newcommand{\bsthetap}{\boldsymbol{\theta'}}
\newcommand{\bsmup}{\boldsymbol{\mu'}}
\newcommand{\bsSigmap}{\boldsymbol{\Sigma'}}
\newcommand{\bshatmup}{\boldsymbol{\hat{\mu'}}}
\newcommand{\bshatSigmap}{\boldsymbol{\hat{\Sigma'}}}

% MATHCAL: Gen \calX to stand for \mathcal{X}, all upper case
\generate{cal}{\mathcal}{A,...,Z}

% MATHBB: Gen \X to stand for \mathbb{X} for some upper case
\generate{}{\mathbb}{R,Q,C,Z,N,Z,E}
\newcommand{\Rn}{\mathbb{R}^n}
\newcommand{\RN}{\mathbb{R}^N}
\newcommand{\Rk}{\mathbb{R}^k}
\newcommand{\RK}{\mathbb{R}^K}
\newcommand{\RL}{\mathbb{R}^L}
\newcommand{\Rl}{\mathbb{R}^\ell}
\newcommand{\Rm}{\mathbb{R}^m}
\newcommand{\Rnn}{\mathbb{R}^{n\times n}}
\newcommand{\Rmn}{\mathbb{R}^{m\times n}}
\newcommand{\Rnm}{\mathbb{R}^{n\times m}}
\newcommand{\Rkn}{\mathbb{R}^{k\times n}}
\newcommand{\Cn}{\mathbb{C}^n}
\newcommand{\Cnn}{\mathbb{C}^{n\times n}}

% Dot over
\newcommand{\dx}{\dot{x}}
\newcommand{\ddx}{\ddot{x}}
\newcommand{\dy}{\dot{y}}
\newcommand{\ddy}{\ddot{y}}

% First derivatives
\newcommand{\dydx}{\frac{dy}{dx}}
\newcommand{\dfdx}{\frac{df}{dx}}
\newcommand{\dfdy}{\frac{df}{dy}}
\newcommand{\dfdz}{\frac{df}{dz}}

% Second derivatives
\newcommand{\ddyddx}{\frac{d^2y}{dx^2}}
\newcommand{\ddydxdy}{\frac{d^2y}{dx dy}}
\newcommand{\ddydydx}{\frac{d^2y}{dy dx}}
\newcommand{\ddfddx}{\frac{d^2f}{dx^2}}
\newcommand{\ddfddy}{\frac{d^2f}{dy^2}}
\newcommand{\ddfddz}{\frac{d^2f}{dz^2}}
\newcommand{\ddfdxdy}{\frac{d^2f}{dx dy}}
\newcommand{\ddfdydx}{\frac{d^2f}{dy dx}}


% First Partial Derivatives
\newcommand{\pypx}{\frac{\partial y}{\partial x}}
\newcommand{\pfpx}{\frac{\partial f}{\partial x}}
\newcommand{\pfpy}{\frac{\partial f}{\partial y}}
\newcommand{\pfpz}{\frac{\partial f}{\partial z}}


% argmin and argmax
\DeclareMathOperator*{\argmin}{arg\;min}
\DeclareMathOperator*{\argmax}{arg\;max}


% Various probability and statistics commands
\newcommand{\iid}{\overset{iid}{\sim}}
\newcommand{\vc}{\operatorname{vec}}
\newcommand{\Cov}{\operatorname{Cov}}
\newcommand{\rank}{\operatorname{rank}}
\newcommand{\trace}{\operatorname{trace}}
\newcommand{\Corr}{\operatorname{Corr}}
\newcommand{\Var}{\operatorname{Var}}
\newcommand{\asto}{\xrightarrow{a.s.}}
\newcommand{\pto}{\xrightarrow{p}}
\newcommand{\msto}{\xrightarrow{m.s.}}
\newcommand{\dto}{\xrightarrow{d}}
\newcommand{\Lpto}{\xrightarrow{L_p}}
\newcommand{\Lqto}[1]{\xrightarrow{L_{#1}}}
\newcommand{\plim}{\text{plim}_{n\rightarrow\infty}}


% Redefine real and imaginary from fraktur to plain text
\renewcommand{\Re}{\operatorname{Re}}
\renewcommand{\Im}{\operatorname{Im}}

% Shorter sums: ``Sum from X to Y''
% - sumXY  is equivalent to \sum^Y_{X=1}
% - sumXYz is equivalent to \sum^Y_{X=0}
\newcommand{\sumnN}{\sum^N_{n=1}}
\newcommand{\sumin}{\sum^n_{i=1}}
\newcommand{\sumjn}{\sum^n_{j=1}}
\newcommand{\sumim}{\sum^m_{i=1}}
\newcommand{\sumik}{\sum^k_{i=1}}
\newcommand{\sumiN}{\sum^N_{i=1}}
\newcommand{\sumkn}{\sum^n_{k=1}}
\newcommand{\sumtT}{\sum^T_{t=1}}
\newcommand{\sumninf}{\sum^\infty_{n=1}}
\newcommand{\sumtinf}{\sum^\infty_{t=1}}
\newcommand{\sumnNz}{\sum^N_{n=0}}
\newcommand{\suminz}{\sum^n_{i=0}}
\newcommand{\sumknz}{\sum^n_{k=0}}
\newcommand{\sumtTz}{\sum^T_{t=0}}
\newcommand{\sumninfz}{\sum^\infty_{n=0}}
\newcommand{\sumtinfz}{\sum^\infty_{t=0}}

\newcommand{\prodnN}{\prod^N_{n=1}}
\newcommand{\prodin}{\prod^n_{i=1}}
\newcommand{\prodiN}{\prod^N_{i=1}}
\newcommand{\prodkn}{\prod^n_{k=1}}
\newcommand{\prodtT}{\prod^T_{t=1}}
\newcommand{\prodnNz}{\prod^N_{n=0}}
\newcommand{\prodinz}{\prod^n_{i=0}}
\newcommand{\prodknz}{\prod^n_{k=0}}
\newcommand{\prodtTz}{\prod^T_{t=0}}

% Bounds
\newcommand{\atob}{_a^b}
\newcommand{\ztoinf}{_0^\infty}
\newcommand{\kinf}{_{k=1}^\infty}
\newcommand{\ninf}{_{n=1}^\infty}
\newcommand{\minf}{_{m=1}^\infty}
\newcommand{\tinf}{_{t=1}^\infty}
\newcommand{\nN}{_{n=1}^N}
\newcommand{\tT}{_{t=1}^T}
\newcommand{\kinfz}{_{k=0}^\infty}
\newcommand{\ninfz}{_{n=0}^\infty}
\newcommand{\minfz}{_{m=0}^\infty}
\newcommand{\tinfz}{_{t=0}^\infty}
\newcommand{\nNz}{_{n=0}^N}

% Limits
\newcommand{\limN}{\lim_{N\rightarrow\infty}}
\newcommand{\limn}{\lim_{n\rightarrow\infty}}
\newcommand{\limk}{\lim_{k\rightarrow\infty}}
\newcommand{\limt}{\lim_{t\rightarrow\infty}}
\newcommand{\limT}{\lim_{T\rightarrow\infty}}
\newcommand{\limhz}{\lim_{h\rightarrow 0}}

% Shorter integrals: ``Integral from X to Y''
% - intXY is equivalent to \int^Y_X
\newcommand{\intab}{\int_a^b}
\newcommand{\intzN}{\int_0^N}


%%%%%%%%%%%%%%%%%%%%%%%%%%%%%%%%%%%%%%%%%%%%%%%%%%%%%%%%%%%%%%%%%%%%%%%%
%% BODY %%%%%%%%%%%%%%%%%%%%%%%%%%%%%%%%%%%%%%%%%%%%%%%%%%%%%%%%%%%%%%%%
%%%%%%%%%%%%%%%%%%%%%%%%%%%%%%%%%%%%%%%%%%%%%%%%%%%%%%%%%%%%%%%%%%%%%%%%


\begin{document}
%\maketitle
\lstset{style=log}

\section{SQL (Structured Query Language)}

\subsection{Introduction}



Data Modeling:
Consists of the following steps.
\begin{itemize}
  \item Define \emph{entities}/\emph{concepts} that we want to capture
    data about, e.g.  persons, employees, customers, producers, etc.
    There will be a table associated with each entity/concept
  \item Determine \emph{attributes} of each entity/concept.
    These will be the columns in the table associated with each
    entity/concept.
  \item Determine \emph{relationships} between entities/concepts, e.g.
    customers \emph{buy from} producers.
  \item Determine \emph{cardinality}/\emph{multiplicity} of the
    relationship: 1:1, 1:M, M:M
\end{itemize}
Data modeling is typically done with \emph{ER Diagrams}.
The diagrams have a box for each entity, a line for each relationship,
and specific notation for the cardinality.
There are a few choices for that notation.
\begin{itemize}
  \item Chen Notation:
    Use 1:M, M:N, 1:1 for one-to-many, many-to-many, and
    one-to-one.
  \item Crow's Foot Notation:
    Train tracks and crow's foot to represent one and many.
  \item UML Class Diagram Notation:
    1.1 and 1.* to represent one and many.
\end{itemize}
The process of data modeling is an abstract task distinct from the
question of how we might collect, store, and represent the data.
We often store data in \emph{databases}, which are containers (file or
set of files) to store organized data, a set of related information.
Our database often consists of \emph{tables}, which are structured list
of data or a specific type.
They consist of of \emph{columns} (single fields in a table) and
\emph{rows} (records in a table).

Relational database model
\begin{itemize}
  \item This is one implementation for a given data model
    that is specifically designed for querying.
  \item Each entity has a table, attributes of an entity are stored in
    columns.
  \item Primary Key:
    Column or set of columns whose values unique identify every row in a
    table.
    Lets us use these to join to another table.
  \item Foreign Key:
    One or more columns that can be used together to identify a single
    row in another table.
\end{itemize}
NoSql (Not Only SQL)
\begin{itemize}
  \item Relational data model
  \item Mechanism for storage and retrieval of unstructured
    data model by means \emph{other than} tabular relations in
    relational databases.
\end{itemize}

%Transactional database model:
%Operational database.
%Not specifically designed for querying


\clearpage
\subsection{Basic Syntax}
\begin{itemize}
  \item Terminate commands with semicolon

  \item Comments:
    \begin{lstlisting}
    - - To comment out a single line

    /*
    To comment out
    multiple lines
    */
    \end{lstlisting}

  \item \texttt{NULL} is special entry.
    Null values are distinct from just ``empty string.''
    They really are ``nothing there.''

\end{itemize}


\subsection{Select Statements}

Select statements are the core of SQL.
They are how we retreive data.
The most basic select statement retrieves specific columns from a table.
\begin{lstlisting}
SELECT col1,
       col2
FROM tableName;
\end{lstlisting}
Can use $*$ as wildcard to retrieve all columns in the table.
\\
\\
At its most complex, we can augment the select statement with many
modifiers that refine the select statement.
\begin{lstlisting}
- - Selects columns to be returned
SELECT col1, col2

- - Specifies table where columns are
FROM tableName

- - Row-level filter (prior to any grouping)
WHERE

- - Group specification if we have an aggregate function in select statement
GROUP BY

- - Group-level filter
HAVING

- - Output sort order
ORDER BY
\end{lstlisting}



\clearpage
\subsubsection{Row-wise Operations}

For numeric data, can use simple math operations in select statement:
\begin{lstlisting}
SELECT col1,
       col2,
       col1 * col2 AS newName
FROM tableName;
\end{lstlisting}
For string data, can use string functions in select statements
\begin{itemize}
  \item
    Concatenate with \texttt{||}, which is like using + in Python for
    strings.
    \begin{lstlisting}
    SELECT CompanyName,
           ContactName,
           Company Name || ',' || ContactName
    FROM Customers;
    \end{lstlisting}

  \item
    Trim with \texttt{TRIM}, \texttt{LTRIM}, \texttt{RTRIM}.
    This removes leading or trailing whitespace.
    \begin{lstlisting}
    SELECT TRIM(CompanyName) AS TrimmedName
    FROM Customers;
    \end{lstlisting}

  \item
    Use \texttt{SUBSTR} return specified number of characters from a
    particular position of a given string.
    Arguments are string name, starting position in string, number of
    characters to be returned.
    \begin{lstlisting}
    SELECT first_name, SUBSTR(first_name, 2, 3)
    FROM employees;
    \end{lstlisting}

  \item
    Use \texttt{UPPER}, \texttt{LOWER}, \texttt{UCASE} to change case.
    \begin{lstlisting}
    SELECT UPPER(column_name)
    FROM table_name;
    \end{lstlisting}
\end{itemize}
For strings that are specifically dates and/or times, there are
additional date-time functions.
\begin{itemize}
  \item
    Standard formats:
    All are encoded as strings.
    \begin{itemize}
      \item \texttt{DATE}: YYYY-MM-DD
      \item \texttt{DATETIME}: YYYY-MM-DD HH:MI:SS
      \item \texttt{TIMESTAMP}: YYYY-MM-DD HH:MI:SS
    \end{itemize}

  \item
    For converting birthdate to a year, month, day, time etc.:
    \begin{lstlisting}
    SELECT Birthdate,
           STRFTIME('%Y', Birthdate) AS Year,
           STRFTIME('%m', Birthdate) AS Month,
           STRFTIME('%d', Birthdate) AS Day
    FROM Employees
    \end{lstlisting}
    Compute current date and derivatives thereof
    by using \texttt{'now'}.
    \begin{lstlisting}
    SELECT DATE('now')

    SELECT STRFTIME('%Y %m %d', 'now')
    \end{lstlisting}
\end{itemize}
Case Statements:
Mimics if-then-else.
Produce a new column where each record/row is built from an if
statement based on the values in the other columns for that row.
\begin{lstlisting}
SELECT col1,
       col2,
       col3,
       NewColName
CASE col1
  WHEN 'SomeVal1' THEN 'SetToWhen1'
  WHEN 'SomeVal2' THEN 'SetToWhen2'
  ELSE                 'SetToOther'
END NewColName
FROM TableName;
\end{lstlisting}
Alternatively
\begin{lstlisting}
SELECT col1,
       col2,
       col3,
       NewColName
CASE
  WHEN col1='SomeVal1' THEN 'SetToWhen1'
  WHEN col1='SomeVal2' THEN 'SetToWhen2'
  ELSE                      'SetToOther'
END NewColName
FROM TableName;
\end{lstlisting}
Note, what comes after \texttt{THEN} doesn't always have to be a
word or a number, can be a field from another column.



\clearpage
\subsubsection{Filtering with \texttt{WHERE}}

Filtering reduces the amount of rows that are pulled according based
on certain rules and logical conditions.
We do so by adding a \texttt{WHERE} line to the select statement.
\begin{lstlisting}
SELECT col1,
       col2
FROM tableName
WHERE logical-condition;
\end{lstlisting}
This pulls those rows for which the logical condition is satisfied.
The logical condition can be built using any of the following
approaches or keywords.
\begin{itemize}
  \item Comparisons:
    The logical condition might be phrased
    \begin{lstlisting}
    colName someOperator someValue
    \end{lstlisting}
    Ooperators include:
    \texttt{=},
    \texttt{!=},
    \texttt{>},
    \texttt{<},
    \texttt{BETWEEN} (beteween an inclusive range),
    \texttt{IS NULL}.

  \item \texttt{IN}:
    Restrict to those rows where a given column is in some set.
    \begin{lstlisting}
    SELECT col1,
            col2
    FROM tableName
    WHERE colName IN (val1, val2, val3);
    \end{lstlisting}

  \item \texttt{OR}:
    Non-inclusive matching.
    This is like \texttt{IN} but with precedence.
    \begin{lstlisting}
    SELECT col1,
            col2
    FROM tableName
    WHERE colName = val1 OR val2;
    \end{lstlisting}
    If there are some rows matching \texttt{val1}, it will return
    those rows and \emph{will not} retreive rows matching
    \texttt{val2}.
    If there are no rows matching \texttt{val1}, it will try to
    match \texttt{val2} and return rows matching that.
    So order matters; it specifies precedence.

    Use parentheses when working with \texttt{AND} because of order
    of operations.

  \item \texttt{AND}:
    Works like you think.
    Use parentheses when used with \texttt{OR} to be safe.

  \item \texttt{NOT}:
    Place in front of condition to return rows not matching that
    condition, e.g.
    \begin{lstlisting}
    SELECT col1,
            col2
    FROM tableName
    WHERE NOT colName = val1;
    \end{lstlisting}

  \item
    \texttt{LIKE}:
    Allows use of wildcards for matching text data.
    \begin{lstlisting}
    SELECT col1,
            col2
    FROM tableName
    WHERE colName LIKE '%partofstring';
    \end{lstlisting}
    Use \texttt{\%} anywhere in the string to match any number of
    characters.
    Use \texttt{\_} to match a single character.

    NOTE: These are slower. Other operators are preferred.
\end{itemize}



\subsubsection{%
  Grouping/Aggregating with \texttt{GROUP BY} and \texttt{HAVING}
}

\begin{itemize}
  \item Aggregate Functions:
    Built in functions used to summarize data.
    Includes
    \texttt{AVG()},
    \texttt{COUNT()},
    \texttt{MIN()},
    \texttt{MAX()},
    \texttt{SUM()}.
    Not that \texttt{NULL} values are ignored.

    We use a select statement with an \texttt{AS} to give the returned
    column/info a name:
    \begin{lstlisting}
    SELECT AVG(col1) as newName
    FROM tableName;
    \end{lstlisting}
    Note, can also use math expressions in place of a single column:
    \begin{lstlisting}
    SELECT AVG(col1*col2) as newName
    FROM tableName;
    \end{lstlisting}
    Use \texttt{DISTINCT} if we just want to use distinct entries of a
    column (non-duplicates).

  \item Grouping:
    \begin{lstlisting}
    SELECT col1,
           COUNT(col2) as newName
    FROM tableName
    GROUP BY col1;
    \end{lstlisting}
    \emph{Every} column in select statement must then be present in a
    \texttt{GROUP BY} clause, except for those columns that are purely
    arguments to aggregate functions (and so are, by definition,
    aggregated by group).
    So it's clear \texttt{GROUP BY} can contain multiple columns.

  \item
    Filtering after Grouping:
    \texttt{WHERE} does not work for groups because it filters on rows
    of the table (i.e. prior to group), not on grouped data (i.e. after
    grouping).  Instead, use \texttt{HAVING} to filter the grouped data.
    \begin{lstlisting}
    SELECT CustomerID,
           COUNT(*) as orders
    FROM OrderTable
    GROUP BY CustomerID
    HAVING COUNT (*) >=2;
    \end{lstlisting}
    This groups by CustomerID, counts the number of orders, and retains
    only those CustomerID with more than two orders.

    Can use both \texttt{WHERE} and \texttt{HAVING} to filter before
    grouping, and then filter after grouping.
    Note: rows eliminated by the \texttt{WHERE} clause will not be
    included in the group.

\end{itemize}


\subsubsection{Sorting with \texttt{ORDER BY}}

Sort returned data by sorting on (potentially multiple) columns.
We do so by adding lines to a select statement.
It must always be the last clause in the select statement.
Note that we can even sort on columns not selected for returning.
\begin{lstlisting}
SELECT col1,
        col2
FROM tableName
ORDER BY colX, colY, colZ
;
\end{lstlisting}
Can also add \texttt{ASC} or \texttt{DESC} before a particular
column name to specify ascending or descending order for the sort.




\clearpage
\subsection{Joins and Unions}


Joins retrieve data from multiple tables in one query.
Doesn't store joined information or have a phsyical effect;
join persists only for the duration of the query execution.

\begin{table}[htbp!]
\scriptsize
\centering
\begin{tabular}{|ccc|ccc|c|c|c|c|}
  \hline
  \multicolumn{3}{|c|}{Table 1}
  &
  \multicolumn{3}{|c|}{Table 2}
  &
  \multicolumn{4}{|c|}{Join Type}
  \\
  \hline
  \texttt{id} & \texttt{x1} & $\cdots$ &
  \texttt{id} & \texttt{y1} & $\cdots$
  & Inner
  & Left
  & Right
  & Outer
  \\
  \hline
  1 & & &
  1 & & &
  $\checkmark$ & $\checkmark$
  & $\checkmark$
  & $\checkmark$
  \\
  2 & & &
  2 & & &
  $\checkmark$ & $\checkmark$
  & $\checkmark$
  & $\checkmark$
  \\
  3 & & &
    & & &
    & $\checkmark$ &
  & $\checkmark$
  \\
  4 & & &
    & & &
    & $\checkmark$ &
  & $\checkmark$
  \\
    & & &
  5 & & &
    & &
    $\checkmark$
  & $\checkmark$
  \\
    & & &
  6 & & &
    & &
    $\checkmark$
  & $\checkmark$
  \\
  \hline
\end{tabular}
\end{table}


\subsubsection{Aliasing and Prequalifying}

Instead of writing out the table names all the times, we can create
aliases that we use elsewhere in the query.
\begin{lstlisting}
SELECT vendor_name,
       product_name,
       product_price
FROM Vendors v, Products p
WHERE v.vendor_id = p.vendor_id;
\end{lstlisting}
When performing joins, we will often ``qualify'' by
prepending to a given column name either the full table name (e.g.
\texttt{TableName.}) or that table's alias (e.g. \texttt{T1.}) if we
defined an alias for that table.
This is always allowed, and is good practice and/or necessary if two
tables that we're merging have columns with the same name or if
we're doing a self join.


\subsubsection{Inner Join}

Match on a key (unique identifier in both tables) and throw out any
unmatched records/rows in either table; the key needs to be in both.
We specify the columns we want from the two tables, the two tables
to join, and the key that is used to establish a match.
Only returns those records for which the key is found in \emph{both}
tables.
\begin{lstlisting}
SELECT Suppliers.CompanyName,
        ProductName,
        UnitPrice
FROM Suppliers INNER JOIN Products
  ON Suppliers.supplierid = Products.supplierid;
\end{lstlisting}


\subsubsection{Left and Right Joins}

Left Join:
Returns \emph{all} records from left table and the matched records
from the right table.
The key only needs to be in the left table.
Result is \texttt{NULL} if no match, rather
than throwing out as for an inner join.
\begin{lstlisting}
SELECT C.CompanyName,
        O.OrderID
FROM Customers C LEFT JOIN Orders O
  ON C.CustomerID = O.CustomerID;
\end{lstlisting}
Right Join: Similar to left join, just flipped.
The key only needs to be in the right table.
Given the abilitity to produce left joins, this type of join is
redundant since we can always just switch the order of the tables in the
join statement.

\subsubsection{Full Outer Join}

Return all records when there is a match in either the left
\emph{or} the right table records. ``Give me everything.''
The key only needs to be in one of the tables.
\begin{lstlisting}
SELECT Customers.CustomerName,
       Orders.OrderID
FROM Customers FULL OUTER JOIN Orders
  ON Customers.CustomerID = Orders.CustomerID;
\end{lstlisting}



\subsubsection{Cartesian (Cross Joins)}

Each row from first table joins with all the rows of another table.
So if Table 1 has $N_1$ rows and Table 2 has $N_2$ rows, result will
have $N_1\times N_2$ rows.
Specify all columns that you want from either table, and use
\texttt{CROSS JOIN} to list the two tables containing those columns.
Doesn't match on anything; just weirdly mashes together the two
datasets. Unclear  why you'd ever want to do this.
\begin{lstlisting}
SELECT product_name,
       unit_price,
       company_name
FROM suppliers CROSS JOIN products;
\end{lstlisting}


\subsubsection{Self Joins}

Join table to itself. Requires using aliases.
\begin{lstlisting}
SELECT A.CustomerName AS CustomerName1,
       B.CustomerName AS CustomerName2,
       A.City
FROM Customers A, Customers B
WHERE A.CustomerID = B.CustomerID
  AND A.City = B.City
ORDER BY A.City;
\end{lstlisting}
Alias the single table \texttt{Customers} by \texttt{A} and
\texttt{B}, effectively treating as two separate tables.
Not clear what this does or why you'd ever want to do this.


\subsubsection{Unions}

Combine the result-set of two or more select statements.
Essentially just stack one on top of the other.
Need to ensure the same number of columns and data types in each
column.
\begin{lstlisting}
SELECT colnames
FROM Table1

UNION

SELECT colnames
FROM Table2;
\end{lstlisting}


\clearpage
\subsection{Subqueries, i.e. Nesting Select Statements}

Subqueries are queries embedded in other queries; no limit to amount of
nesting.
Allows merging and filtering of entries in one table using
information in another table.
Notion of ``query expansion'' where innermost query evaluated and
work to outer query.
\begin{lstlisting}
SELECT customerID,
        companyName,
        Region
FROM CustomerTable
WHERE customerID IN (
                      SELECT customerID
                      FROM OrdersTable
                      WHERE Freight > 100
                    )
;
\end{lstlisting}
Another example that does a ``calculated fill'' (calculates and
fills).
\begin{lstlisting}
SELECT customer_name,
        customer_state
        (
        SELECT COUNT (*) AS orders
        FROM OrdersTable
        WHERE OrdersTable.customer_id = CustomersTable.customer_id
        ) AS orders
FROM CustomersTable
ORDER BY Customer_name
;
\end{lstlisting}
From \texttt{OrdersTable}, count the number of entries with a
\texttt{customer\_id} that is also in \texttt{CustomersTable};
this counts the number of orders for each customer in
\texttt{CustomersTable}.
Save the column as \texttt{orders}.






\clearpage
\subsection{Creating Tables and Views}

\begin{itemize}
  \item
    \texttt{CREATE TABLE}:
    Specify the table name, the column names,
    the data types, which column is a primary key, and
    whether we allow null values.
    \begin{lstlisting}
    CREATE TABLE TableName
    ( Id      char(10)        PRIMARY KEY,
      Col1    char(250)       NOT NULL,
      Col2    decimal(8,2)    NOT NULL,
      Col3    Varchar(750)    NULL );
    \end{lstlisting}

  \item \texttt{DROP TABLE}:
    Delete tables that are no longer needed
    \begin{lstlisting}
    DROP TABLE TableName;
    \end{lstlisting}


  \item
    \texttt{INSERT INTO}:
    To add a row/record into various columns of the table,
    \begin{lstlisting}
    INSERT INTO TableName
    ( '143423',
      'Matthew',
      '3.14',
      NULL );
    \end{lstlisting}
    Better is to specify specifically the columns that the data is
    intended for:
    \begin{lstlisting}
    INSERT INTO TableName
    ( Id,
      Col1,
      Col2,
      Col3 )
    VALUES
    ( '143423',
      'Matthew',
      '3.14',
      NULL );
    \end{lstlisting}

  \item
    \texttt{UPDATE}:
    Modify the rows in a table,
    \begin{lstlisting}
    UPDATE PhoneBook
    ( phone,
      firstname,
      lastname,
      address )
    VALUES
    ( '+1 610 582 7884',
      'Matthew',
      'Cocci',
      '333 Maple Street' )
    WHERE firstname = 'Matthew'
      AND lastname  = 'Cocci';
    \end{lstlisting}

  \item \texttt{DELETE FROM}:
    Delete the rows in a table,
    \begin{lstlisting}
    DELETE FROM PhoneBook
    WHERE firstname = 'Matthew'
      AND lastname  = 'Cocci';
    \end{lstlisting}

  \item
    \texttt{CREATE TEMPORARY TABLE}:
    Retrieve data and create a tempororary table (faster than creating a
    regular table). It stays around for the duration of the client
    session, and is then deleted.
    \begin{lstlisting}
    CREATE TEMPORARY TABLE NewTempTableName AS
    (
      select-statement
    )
    \end{lstlisting}

  \item Views:
    Essentially a stored query, but it will be removed after database
    connection has ended. You're not actually reading or writing to
    database; you're just simulating it.

    Basic syntax:
    \begin{lstlisting}
    CREATE [TEMP] VIEW [IF NOT EXISTS] ViewName[column_name_list]
    AS
    select-statement;
    \end{lstlisting}
    \texttt{TEMP} is optional specification of whether a view is
    temporary or not. \texttt{IF NOT EXISTS} is an optional
    specification of what to do if the view doesn't exist.

    After creating a view, we can use the view name as a table name in
    select statements, and proceed as if the view really were a table.

    To drop the view after we're done,
    \begin{lstlisting}
    DROP VIEW ViewName;
    \end{lstlisting}

\end{itemize}



\subsection{Miscellaneous}

\texttt{LIMIT}:
To limit the amount of rows pulled to obtain a sample.
\begin{lstlisting}
SELECT col1,
      col2
FROM tableName
LIMIT 5
;
\end{lstlisting}




\end{document}


%%%%%%%%%%%%%%%%%%%%%%%%%%%%%%%%%%%%%%%%%%%%%%%%%%%%%%%%%%%%%%%%%%%%%%%%
%%%%%%%%%%%%%%%%%%%%%%%%%%%%%%%%%%%%%%%%%%%%%%%%%%%%%%%%%%%%%%%%%%%%%%%%
%%%%%%%%%%%%%%%%%%%%%%%%%%%%%%%%%%%%%%%%%%%%%%%%%%%%%%%%%%%%%%%%%%%%%%%%

%%%% SAMPLE CODE %%%%%%%%%%%%%%%%%%%%%%%%%%%%%%%%%%%%%%

    %% VIEW LAYOUT %%

        \layout

    %% LANDSCAPE PAGE %%

        \begin{landscape}
        \end{landscape}

    %% BIBLIOGRAPHIES %%

        \cite{LabelInSourcesFile}  %Use in text; cites
        \citep{LabelInSourcesFile} %Use in text; cites in parens

        \nocite{LabelInSourceFile} % Includes in refs w/o specific citation
        \bibliographystyle{apalike}  % Or some other style

        % To ditch the ``References'' header
        \begingroup
        \renewcommand{\section}[2]{}
        \endgroup

        \bibliography{sources} % where sources.bib has all the citation info

    %% SPACING %%

        \vspace{1in}
        \hspace{1in}

    %% URLS, EMAIL, AND LOCAL FILES %%

      \url{url}
      \href{url}{name}
      \href{mailto:mcocci@raidenlovessusie.com}{name}
      \href{run:/path/to/file.pdf}{name}


    %% INCLUDING PDF PAGE %%

        \includepdf{file.pdf}


    %% INCLUDING CODE %%

        %\verbatiminput{file.ext}
            %   Includes verbatim text from the file

        \texttt{text}
            %   Renders text in courier, or code-like, font

        \matlabcode{file.m}
            %   Includes Matlab code with colors and line numbers

        \lstset{style=bash}
        \begin{lstlisting}
        \end{lstlisting}
            % Inline code rendering


    %% INCLUDING FIGURES %%

        % Basic Figure with size scaling
            \begin{figure}[h!]
               \centering
               \includegraphics[scale=1]{file.pdf}
            \end{figure}

        % Basic Figure with specific height
            \begin{figure}[h!]
               \centering
               \includegraphics[height=5in, width=5in]{file.pdf}
            \end{figure}

        % Figure with cropping, where the order for trimming is  L, B, R, T
            \begin{figure}
               \centering
               \includegraphics[trim={1cm, 1cm, 1cm, 1cm}, clip]{file.pdf}
            \end{figure}

        % Side by Side figures: Use the tabular environment



