\documentclass[12pt]{article}

\author{Matthew D. Cocci}
\title{Title}
\date{\today}

%% Formatting & Spacing %%%%%%%%%%%%%%%%%%%%%%%%%%%%%%%%%%%%

%\usepackage[top=1in, bottom=1in, left=1in, right=1in]{geometry} % most detailed page formatting control
\usepackage{fullpage} % Simpler than using the geometry package; std effect
\usepackage{setspace}
%\onehalfspacing
\usepackage{microtype}

%% Formatting %%%%%%%%%%%%%%%%%%%%%%%%%%%%%%%%%%%%%%%%%%%%%

%\usepackage[margin=1in]{geometry}
    %   Adjust the margins with geometry package
%\usepackage{pdflscape}
    %   Allows landscape pages
%\usepackage{layout}
    %   Allows plotting of picture of formatting
\usepackage{version}
\excludeversion{solutions}
%\includeversion{solutions}



%% Header %%%%%%%%%%%%%%%%%%%%%%%%%%%%%%%%%%%%%%%%%%%%%%%%%

%\usepackage{fancyhdr}
%\pagestyle{fancy}
%\lhead{}
%\rhead{}
%\chead{}
%\setlength{\headheight}{15.2pt}
    %   Make the header bigger to avoid overlap

%\fancyhf{}
    %   Erase header settings

%\renewcommand{\headrulewidth}{0.3pt}
    %   Width of the line

%\setlength{\headsep}{0.2in}
    %   Distance from line to text


%% Mathematics Related %%%%%%%%%%%%%%%%%%%%%%%%%%%%%%%%%%%

\usepackage{amsmath}
\usepackage{amssymb}
\usepackage{amsfonts}
\usepackage{mathrsfs}
\usepackage{mathtools}
\usepackage{amsthm} %allows for labeling of theorems
%\numberwithin{equation}{section} % Number equations by section
\usepackage{bbm} % For bold numbers

\theoremstyle{plain}
\newtheorem{thm}{Theorem}[section]
\newtheorem{lem}[thm]{Lemma}
\newtheorem{prop}[thm]{Proposition}
\newtheorem{cor}[thm]{Corollary}

\theoremstyle{definition}
\newtheorem{defn}[thm]{Definition}
\newtheorem{ex}[thm]{Example}

\theoremstyle{remark}
\newtheorem*{rmk}{Remark}
\newtheorem*{note}{Note}

% Below supports left-right alignment in matrices so the negative
% signs don't look bad
\makeatletter
\renewcommand*\env@matrix[1][c]{\hskip -\arraycolsep
  \let\@ifnextchar\new@ifnextchar
  \array{*\c@MaxMatrixCols #1}}
\makeatother


%% Font Choices %%%%%%%%%%%%%%%%%%%%%%%%%%%%%%%%%%%%%%%%%

\usepackage[T1]{fontenc}
\usepackage{lmodern}
\usepackage[utf8]{inputenc}
%\usepackage{blindtext}
\usepackage{courier}


%% Figures %%%%%%%%%%%%%%%%%%%%%%%%%%%%%%%%%%%%%%%%%%%%%%

\usepackage{tikz}
\usetikzlibrary{decorations.pathreplacing}
\usetikzlibrary{arrows.meta}
\usepackage{graphicx}
\usepackage{subfigure}
    %   For plotting multiple figures at once
%\graphicspath{ {Directory/} }
    %   Set a directory for where to look for figures


%% Hyperlinks %%%%%%%%%%%%%%%%%%%%%%%%%%%%%%%%%%%%%%%%%%%%
\usepackage{hyperref}
\hypersetup{%
    colorlinks,
        %   This colors the links themselves, not boxes
    citecolor=black,
        %   Everything here and below changes link colors
    filecolor=black,
    linkcolor=black,
    urlcolor=black
}

%% Colors %%%%%%%%%%%%%%%%%%%%%%%%%%%%%%%%%%%%%%%%%%%%%%%

\usepackage{color}
\definecolor{codegreen}{RGB}{28,172,0}
\definecolor{codelilas}{RGB}{170,55,241}

% David4 color scheme
\definecolor{d4blue}{RGB}{100,191,255}
\definecolor{d4gray}{RGB}{175,175,175}
\definecolor{d4black}{RGB}{85,85,85}
\definecolor{d4orange}{RGB}{255,150,100}

%% Including Code %%%%%%%%%%%%%%%%%%%%%%%%%%%%%%%%%%%%%%%

\usepackage{verbatim}
    %   For including verbatim code from files, no colors
\usepackage{listings}
    %   For including code snippets written directly in this doc

\lstdefinestyle{bash}{%
  language=bash,%
  basicstyle=\footnotesize\ttfamily,%
  showstringspaces=false,%
  commentstyle=\color{gray},%
  keywordstyle=\color{blue},%
  xleftmargin=0.25in,%
  xrightmargin=0.25in
}
\lstdefinestyle{log}{%
  basicstyle=\scriptsize\ttfamily,%
  showstringspaces=false,%
  xleftmargin=0.25in,%
  xrightmargin=0.25in
}


\lstdefinestyle{matlab}{%
  language=Matlab,%
  basicstyle=\footnotesize\ttfamily,%
  breaklines=true,%
  morekeywords={matlab2tikz},%
  keywordstyle=\color{blue},%
  morekeywords=[2]{1}, keywordstyle=[2]{\color{black}},%
  identifierstyle=\color{black},%
  stringstyle=\color{codelilas},%
  commentstyle=\color{codegreen},%
  showstringspaces=false,%
    %   Without this there will be a symbol in
    %   the places where there is a space
  %numbers=left,%
  %numberstyle={\tiny \color{black}},%
    %   Size of the numbers
  numbersep=9pt,%
    %   Defines how far the numbers are from the text
  emph=[1]{for,end,break,switch,case},emphstyle=[1]\color{blue},%
    %   Some words to emphasise
}

\newcommand{\matlabcode}[1]{%
    \lstset{style=matlab}%
    \lstinputlisting{#1}
}
    %   For including Matlab code from .m file with colors,
    %   line numbering, etc.

\lstdefinelanguage{Julia}%
  {morekeywords={abstract,break,case,catch,const,continue,do,else,elseif,%
      end,export,false,for,function,immutable,import,importall,if,in,%
      macro,module,otherwise,quote,return,switch,true,try,type,typealias,%
      using,while},%
   sensitive=true,%
   %alsoother={$},%
   morecomment=[l]\#,%
   morecomment=[n]{\#=}{=\#},%
   morestring=[s]{"}{"},%
   morestring=[m]{'}{'},%
}[keywords,comments,strings]

\lstdefinestyle{julia}{%
    language         = Julia,
    basicstyle       = \scriptsize\ttfamily,
    keywordstyle     = \bfseries\color{blue},
    stringstyle      = \color{codegreen},
    commentstyle     = \color{codegreen},
    showstringspaces = false,
    literate         = %
      {ρ}{{$\rho$}}1
      {ℓ}{{$\ell$}}1
      {∑}{{$\Sigma$}}1
      {Σ}{{$\Sigma$}}1
      {√}{{$\sqrt{}$}}1
      {θ}{{$\theta$}}1
      {ω}{{$\omega$}}1
      {ɛ}{{$\varepsilon$}}1
      {φ}{{$\varphi$}}1
      {σ²}{{$\sigma^2$}}1
      {Φ}{{$\Phi$}}1
      {ϕ}{{$\phi$}}1
      {Dₑ}{{$D_e$}}1
      {Σ}{{$\Sigma$}}1
      {γ}{{$\gamma$}}1
      {δ}{{$\delta$}}1
      {τ}{{$\tau$}}1
      {μ}{{$\mu$}}1
      {β}{{$\beta$}}1
      {Λ}{{$\Lambda$}}1
      {λ}{{$\lambda$}}1
      {r̃}{{$\tilde{\text{r}}$}}1
      {α}{{$\alpha$}}1
      {σ}{{$\sigma$}}1
      {π}{{$\pi$}}1
      {∈}{{$\in$}}1
      {∞}{{$\infty$}}1
}


%% Bibliographies %%%%%%%%%%%%%%%%%%%%%%%%%%%%%%%%%%%%

%\usepackage{natbib}
    %---For bibliographies
%\setlength{\bibsep}{3pt} % Set how far apart bibentries are

%% Misc %%%%%%%%%%%%%%%%%%%%%%%%%%%%%%%%%%%%%%%%%%%%%%

\usepackage{enumitem}
    %   Has to do with enumeration
\usepackage{appendix}
%\usepackage{natbib}
    %   For bibliographies
\usepackage{pdfpages}
    %   For including whole pdf pages as a page in doc
\usepackage{pgffor}
    %   For easier looping


%% User Defined %%%%%%%%%%%%%%%%%%%%%%%%%%%%%%%%%%%%%%%%%%

%\newcommand{\nameofcmd}{Text to display}
\newcommand*{\Chi}{\mbox{\large$\chi$}} %big chi
    %   Bigger Chi

% In math mode, Use this instead of \munderbar, since that changes the
% font from math to regular
\makeatletter
\def\munderbar#1{\underline{\sbox\tw@{$#1$}\dp\tw@\z@\box\tw@}}
\makeatother

% Misc Math
\newcommand{\ra}{\rightarrow}
\newcommand{\diag}{\text{diag}}
\newcommand{\proj}{\operatorname{proj}}
\newcommand{\ch}{\text{ch}}
\newcommand{\dom}{\text{dom}}
\newcommand{\one}[1]{\mathbf{1}_{#1}}


% Command to generate new math commands:
% - Suppose you want to refer to \boldsymbol{x} as just \bsx, where 'x'
%   is any letter. This commands lets you generate \bsa, \bsb, etc.
%   without copy pasting \newcommand{\bsa}{\boldsymbol{a}} for each
%   letter individually. Instead, just include
%
%     \generate{bs}{\boldsymbol}{a,...,z}
%
% - Uses pgffor package to loop
% - Example with optional argument. Will generate \bshatx to represent
%   \boldsymbol{\hat{x}} for all letters x
%
%     \generate[\hat]{bshat}{\boldsymbol}{a,...,z}

\newcommand{\generate}[4][]{%
  % Takes 3 arguments (maybe four):
  % - 1   wrapcmd (optional, defaults to nothing)
  % - 2   newname
  % - 3   mathmacro
  % - 4   Names to loop over
  %
  % Will produce
  %
  %   \newcommand{\newnameX}{mathmacro{wrapcmd{X}}}
  %
  % for each X in argument 4

  \foreach \x in {#4}{%
    \expandafter\xdef\csname%
      #2\x%
    \endcsname%
    {\noexpand\ensuremath{\noexpand#3{\noexpand#1{\x}}}}
  }
}


% MATHSCR: Gen \sX to stand for \mathscr{X} for all upper case letters
\generate{s}{\mathscr}{A,...,Z}


% BOLDSYMBOL: Generate \bsX to stand for \boldsymbol{X}, all upper and
% lower case.
%
% Letters and greek letters
\generate{bs}{\boldsymbol}{a,...,z}
\generate{bs}{\boldsymbol}{A,...,Z}
\newcommand{\bstheta}{\boldsymbol{\theta}}
\newcommand{\bsmu}{\boldsymbol{\mu}}
\newcommand{\bsSigma}{\boldsymbol{\Sigma}}
\newcommand{\bsvarepsilon}{\boldsymbol{\varepsilon}}
\newcommand{\bsalpha}{\boldsymbol{\alpha}}
\newcommand{\bsbeta}{\boldsymbol{\beta}}
\newcommand{\bsOmega}{\boldsymbol{\Omega}}
\newcommand{\bshatOmega}{\boldsymbol{\hat{\Omega}}}
\newcommand{\bshatG}{\boldsymbol{\hat{G}}}
\newcommand{\bsgamma}{\boldsymbol{\gamma}}
\newcommand{\bslambda}{\boldsymbol{\lambda}}

% Special cases like \bshatb for \boldsymbol{\hat{b}}
\generate[\hat]{bshat}{\boldsymbol}{b,y,x,X,V,S,W}
\newcommand{\bshatbeta}{\boldsymbol{\hat{\beta}}}
\newcommand{\bshatmu}{\boldsymbol{\hat{\mu}}}
\newcommand{\bshattheta}{\boldsymbol{\hat{\theta}}}
\newcommand{\bshatSigma}{\boldsymbol{\hat{\Sigma}}}
\newcommand{\bstildebeta}{\boldsymbol{\tilde{\beta}}}
\newcommand{\bstildetheta}{\boldsymbol{\tilde{\theta}}}
\newcommand{\bsbarbeta}{\boldsymbol{\overline{\beta}}}
\newcommand{\bsbarg}{\boldsymbol{\overline{g}}}

% Redefine \bso to be the zero vector
\renewcommand{\bso}{\boldsymbol{0}}

% Transposes of all the boldsymbol shit
\newcommand{\bsbp}{\boldsymbol{b'}}
\newcommand{\bshatbp}{\boldsymbol{\hat{b'}}}
\newcommand{\bsdp}{\boldsymbol{d'}}
\newcommand{\bsgp}{\boldsymbol{g'}}
\newcommand{\bsGp}{\boldsymbol{G'}}
\newcommand{\bshp}{\boldsymbol{h'}}
\newcommand{\bsSp}{\boldsymbol{S'}}
\newcommand{\bsup}{\boldsymbol{u'}}
\newcommand{\bsxp}{\boldsymbol{x'}}
\newcommand{\bsyp}{\boldsymbol{y'}}
\newcommand{\bsthetap}{\boldsymbol{\theta'}}
\newcommand{\bsmup}{\boldsymbol{\mu'}}
\newcommand{\bsSigmap}{\boldsymbol{\Sigma'}}
\newcommand{\bshatmup}{\boldsymbol{\hat{\mu'}}}
\newcommand{\bshatSigmap}{\boldsymbol{\hat{\Sigma'}}}

% MATHCAL: Gen \calX to stand for \mathcal{X}, all upper case
\generate{cal}{\mathcal}{A,...,Z}

% MATHBB: Gen \X to stand for \mathbb{X} for some upper case
\generate{}{\mathbb}{R,Q,C,Z,N,Z,E}
\newcommand{\Rn}{\mathbb{R}^n}
\newcommand{\RN}{\mathbb{R}^N}
\newcommand{\Rk}{\mathbb{R}^k}
\newcommand{\RK}{\mathbb{R}^K}
\newcommand{\RL}{\mathbb{R}^L}
\newcommand{\Rl}{\mathbb{R}^\ell}
\newcommand{\Rm}{\mathbb{R}^m}
\newcommand{\Rnn}{\mathbb{R}^{n\times n}}
\newcommand{\Rmn}{\mathbb{R}^{m\times n}}
\newcommand{\Rnm}{\mathbb{R}^{n\times m}}
\newcommand{\Rkn}{\mathbb{R}^{k\times n}}
\newcommand{\Cn}{\mathbb{C}^n}
\newcommand{\Cnn}{\mathbb{C}^{n\times n}}

% Dot over
\newcommand{\dx}{\dot{x}}
\newcommand{\ddx}{\ddot{x}}
\newcommand{\dy}{\dot{y}}
\newcommand{\ddy}{\ddot{y}}

% First derivatives
\newcommand{\dydx}{\frac{dy}{dx}}
\newcommand{\dfdx}{\frac{df}{dx}}
\newcommand{\dfdy}{\frac{df}{dy}}
\newcommand{\dfdz}{\frac{df}{dz}}

% Second derivatives
\newcommand{\ddyddx}{\frac{d^2y}{dx^2}}
\newcommand{\ddydxdy}{\frac{d^2y}{dx dy}}
\newcommand{\ddydydx}{\frac{d^2y}{dy dx}}
\newcommand{\ddfddx}{\frac{d^2f}{dx^2}}
\newcommand{\ddfddy}{\frac{d^2f}{dy^2}}
\newcommand{\ddfddz}{\frac{d^2f}{dz^2}}
\newcommand{\ddfdxdy}{\frac{d^2f}{dx dy}}
\newcommand{\ddfdydx}{\frac{d^2f}{dy dx}}


% First Partial Derivatives
\newcommand{\pypx}{\frac{\partial y}{\partial x}}
\newcommand{\pfpx}{\frac{\partial f}{\partial x}}
\newcommand{\pfpy}{\frac{\partial f}{\partial y}}
\newcommand{\pfpz}{\frac{\partial f}{\partial z}}


% argmin and argmax
\DeclareMathOperator*{\argmin}{arg\;min}
\DeclareMathOperator*{\argmax}{arg\;max}


% Various probability and statistics commands
\newcommand{\iid}{\overset{iid}{\sim}}
\newcommand{\vc}{\operatorname{vec}}
\newcommand{\Cov}{\operatorname{Cov}}
\newcommand{\rank}{\operatorname{rank}}
\newcommand{\trace}{\operatorname{trace}}
\newcommand{\Corr}{\operatorname{Corr}}
\newcommand{\Var}{\operatorname{Var}}
\newcommand{\asto}{\xrightarrow{a.s.}}
\newcommand{\pto}{\xrightarrow{p}}
\newcommand{\msto}{\xrightarrow{m.s.}}
\newcommand{\dto}{\xrightarrow{d}}
\newcommand{\Lpto}{\xrightarrow{L_p}}
\newcommand{\Lqto}[1]{\xrightarrow{L_{#1}}}
\newcommand{\plim}{\text{plim}_{n\rightarrow\infty}}


% Redefine real and imaginary from fraktur to plain text
\renewcommand{\Re}{\operatorname{Re}}
\renewcommand{\Im}{\operatorname{Im}}

% Shorter sums: ``Sum from X to Y''
% - sumXY  is equivalent to \sum^Y_{X=1}
% - sumXYz is equivalent to \sum^Y_{X=0}
\newcommand{\sumnN}{\sum^N_{n=1}}
\newcommand{\sumin}{\sum^n_{i=1}}
\newcommand{\sumjn}{\sum^n_{j=1}}
\newcommand{\sumim}{\sum^m_{i=1}}
\newcommand{\sumik}{\sum^k_{i=1}}
\newcommand{\sumiN}{\sum^N_{i=1}}
\newcommand{\sumkn}{\sum^n_{k=1}}
\newcommand{\sumtT}{\sum^T_{t=1}}
\newcommand{\sumninf}{\sum^\infty_{n=1}}
\newcommand{\sumtinf}{\sum^\infty_{t=1}}
\newcommand{\sumnNz}{\sum^N_{n=0}}
\newcommand{\suminz}{\sum^n_{i=0}}
\newcommand{\sumknz}{\sum^n_{k=0}}
\newcommand{\sumtTz}{\sum^T_{t=0}}
\newcommand{\sumninfz}{\sum^\infty_{n=0}}
\newcommand{\sumtinfz}{\sum^\infty_{t=0}}

\newcommand{\prodnN}{\prod^N_{n=1}}
\newcommand{\prodin}{\prod^n_{i=1}}
\newcommand{\prodiN}{\prod^N_{i=1}}
\newcommand{\prodkn}{\prod^n_{k=1}}
\newcommand{\prodtT}{\prod^T_{t=1}}
\newcommand{\prodnNz}{\prod^N_{n=0}}
\newcommand{\prodinz}{\prod^n_{i=0}}
\newcommand{\prodknz}{\prod^n_{k=0}}
\newcommand{\prodtTz}{\prod^T_{t=0}}

% Bounds
\newcommand{\atob}{_a^b}
\newcommand{\ztoinf}{_0^\infty}
\newcommand{\kinf}{_{k=1}^\infty}
\newcommand{\ninf}{_{n=1}^\infty}
\newcommand{\minf}{_{m=1}^\infty}
\newcommand{\tinf}{_{t=1}^\infty}
\newcommand{\nN}{_{n=1}^N}
\newcommand{\tT}{_{t=1}^T}
\newcommand{\kinfz}{_{k=0}^\infty}
\newcommand{\ninfz}{_{n=0}^\infty}
\newcommand{\minfz}{_{m=0}^\infty}
\newcommand{\tinfz}{_{t=0}^\infty}
\newcommand{\nNz}{_{n=0}^N}

% Limits
\newcommand{\limN}{\lim_{N\rightarrow\infty}}
\newcommand{\limn}{\lim_{n\rightarrow\infty}}
\newcommand{\limk}{\lim_{k\rightarrow\infty}}
\newcommand{\limt}{\lim_{t\rightarrow\infty}}
\newcommand{\limT}{\lim_{T\rightarrow\infty}}
\newcommand{\limhz}{\lim_{h\rightarrow 0}}

% Shorter integrals: ``Integral from X to Y''
% - intXY is equivalent to \int^Y_X
\newcommand{\intab}{\int_a^b}
\newcommand{\intzN}{\int_0^N}


%%%%%%%%%%%%%%%%%%%%%%%%%%%%%%%%%%%%%%%%%%%%%%%%%%%%%%%%%%%%%%%%%%%%%%%%
%% BODY %%%%%%%%%%%%%%%%%%%%%%%%%%%%%%%%%%%%%%%%%%%%%%%%%%%%%%%%%%%%%%%%
%%%%%%%%%%%%%%%%%%%%%%%%%%%%%%%%%%%%%%%%%%%%%%%%%%%%%%%%%%%%%%%%%%%%%%%%


\begin{document}
%\maketitle
\lstset{style=log}


\section{Unix File Structure}

There is the root directory \texttt{/} with a single root user who can
edit everything under this directory.

Under this root directory, we have
\begin{itemize}
  \item \texttt{/bin}: \emph{Essential} binary executables.
    Important system programs and utilities like the bash shell are
    here.
    These are the crucial absolutely necessary things that are
    guaranteed to be mounted whenever \texttt{/} is mounted,
    irrespective of whether the \texttt{/usr} directory is mounted (as
    it can be stored on a separate partition).
    Used by the system, the system administrator, and users.

  \item \texttt{/sbin}: Like \texttt{/bin}, contains essential binaries
    that are generally intended to be run by \emph{system} and
    \emph{system administrator} for \emph{system maintenance} purposes.

  \item \texttt{/lib}:
    Libraries needed by the essential binaries in \texttt{/bin} and
    \texttt{/sbin}.

  \item \texttt{/usr}:
    Second level user binaries and read-only data.
    Non-essential applications and files used by \emph{users} as opposed
    to the system. Non-essential in the sense that if we mount
    \texttt{/}, we can still work with the system even we don't mount
    \texttt{/usr}.


    Has subdirectories similar to the subdirectories of \texttt{/}:
    \begin{itemize}
      \item \texttt{/usr/bin}:
        Binary files for user programs
      \item \texttt{/usr/lib}
      \item \texttt{/usr/sbin}
      \item \texttt{/usr/local/}:
        Locally compiled applications (i.e. built from source, not
        installed with APT or package manager) are installed here by
        default so as not to fuck up the rest of the system.
    \end{itemize}
  \item \texttt{/var}:
    Variable data files.
    Since \texttt{/usr} is read-only, log files and everything else that
    must be written are written here.
    System log files, emails, print queues, etc.


  \item \texttt{/root}:
    Home directory of the single root user.
    His home directory is \emph{not} just \texttt{/} or in
    \texttt{/home}.

  \item \texttt{/home}:
    Contains the home folders of all non-root users on the system.
    Each user of the system gets their own subfolder under
    \texttt{/home} with all their files.
  \item \texttt{/mnt}:
    Historically, where system admins mounted temporary file systems,
    although you can technically mount anywhere in the file system.
  \item \texttt{/media}:
    Contains subdirectories where removable media devices are mounted
    when inserted/connected.
  \item \texttt{/etc}:
    Configuration files required by all programs, can generally be
    edited by hand in a text editor.
  \item \texttt{/boot}:
    Contains files needed to boot the system, e.g. the GRUB files and
    the Linux kernels.
    Note, doesn't have the configuration files though. Those are all in
    \texttt{/etc}.
  \item \texttt{/tmp}:
    For storing temporary files. Deleted on restart or at any time.
  \item \texttt{/run}:
    Standard place for applications to store transient files they need.
    Not stored in \texttt{/tmp} because files there can be deleted.
  \item \texttt{/dev}:
    Note actual files, but they are ``files'' representing
    devices/hardware (like the hard drive).
    Also contains pseudo-devices like \texttt{rand} or \texttt{null}
    that generate random numbers or discard output piped to it,
    respectively.
  \item \texttt{/proc}:
    Contains special files that represent the system and process
    information.
    \texttt{/proc/(pid)} contains information about the process running
    with that pid.

  \item \texttt{/srv}: Service data containing data for services
    \emph{provided by} the system.
  \item \texttt{/opt}:
    Subdirectories for optional software packages.
    Commonly used by proprietary software that doesn't obey the standard
    Unix file system heirarchy and wants to do it's own thing.
    For example, the application ``application'' might just dumb all
    its files into \texttt{/opt/application/} rather than scatter it's
    files around in the various system directories as is standard.
\end{itemize}




\clearpage
\section{Installation and Package Management}

\subsection{From Source}

Steps
\begin{itemize}
  \item Download files, usually a zip or tarball. Extract.

  \item CD into the directory, which contains all the source files.
    Often there's a README or install file. Read that, see if there's
    anything distinctive about this install.

  \item Configure:
    Typically, you first run \texttt{./configure}, which checks the
    system for the software and packages needed to build the program.

    Resolve dependences with APT or by installing other packages
    from source. Rerun \texttt{./configure}.

    Might need to do this multiple times until \texttt{./configure} runs
    through successfully.

  \item Compile:
    Run the command \texttt{make} to compile source code into binary.

  \item Install:
    Run \texttt{sudo make install} to use the compiled files to install.
    Files are probably stored under \texttt{/usr/local/}

  \item Uninstall:
    Run \texttt{sudo make uninstall}.

  \item Updating (Lack Thereof):
    No easy way to update packages this way.
    Need to uninstall and reinstall.
    That's why it's best to use a package manager.
    See below.
\end{itemize}


\subsection{APT, DPGK, and Package Management}

APT (Advanced Packaging Tool).
This is a more-user friendly wrapper of Debian's package manager DPGK.
Comprehensive reference
\href{https://www.digitalocean.com/community/tutorials/ubuntu-and-debian-package-management-essentials}{here}.
\begin{itemize}
  \item Packages: A discrete unit of software containing user
    interfaces, modules, libraries.
    Want to install, use, and possibly uninstall these.

  \item Repositories/PPAs:
    Remote software archives where downloadable packages are listed,
    along with both download information and update information (e.g.
    when the last update was).
    This determines my visible universe of packages that are available
    to download or update with APT.\footnote{%
      There are other packages to download in non-added repositories
      that I can't see, and there lots of other packages that I could
      download source code for and install. But I can't see these or
      interact with them through APT since I haven't added them to my
      source list.
    }
    Adding a repository/PPA will expand the list of APT-downloadable
    packages (without actually downloading or installing anything).

    The configuration file \texttt{/etc/apt/sources.list}
    and files in \texttt{/etc/apt/sources.list.d/}
    have a list of all sources/repositories with available downloads.
    These files can be edited manually or using commands described
    below.

    By having a list pointing to remote sources/repositories with
    downloadable packages/updates and update information,
    it's extremely easy to download packages, query/check for updates,
    and install updates.
    This wouldn't be as easy if we had to downloaded the packages and
    install them from source in one shot, without storing
    \emph{sources}/\emph{repositories}.

    The term ``repositories'' is generally used for official/traditional
    sources, while PPAs refers to ``personal package archives'' which
    are generally smaller in scope and very targeted in the types of
    programs/software contained in them, unlike the broader traditional
    repostories.
    Although in practice, there's no difference for downloading,
    installing, etc.

  \item Local Database of Packages:
    APT keeps a local database of packages along with information about
    whether any updates for my installed packages are available.

  \item Searching all sources/repositories for packages using keywords:
    This queries all the sources/repositories listed in
    \texttt{/etc/apt/sources.list} to see if there are any packages
    whose descriptions involve keywords:
    \lstset{style=log}
    \begin{lstlisting}
apt-cache search [keywords]
    \end{lstlisting}

  \item Adding repositories:
    Can directly edit \texttt{/etc/apt/sources.list} or place a new
    list in \texttt{/etc/apt/sources.list.d/}.
    Alternatively and preferably use
    \begin{lstlisting}
add-apt-repository [repository name]
    \end{lstlisting}
    Note that the repository name might be of the form
    \texttt{ppa:ownername/ppaname}, indicating that it's a PPA rather
    than a big official/traditional repository. Although this
    distinction doesn't really matter.

  \item Updating Sources:
    After adding repositories, update the package lists (the local
    database of downloadable/updatable packages) for all of my
    repositories and check for updates in one shot with
    \begin{lstlisting}
    sudo apt-get update
    \end{lstlisting}
    This will also list out all servers where information is being
    retreived from in the course of updating.

  \item Installing a Package:
    To install a package, use
    \begin{lstlisting}
    sudo apt-get install [packagename]
    \end{lstlisting}
    Will sometimes also install related packages and dependencies
    (prerequisite packages).

    Add \texttt{-s} flag to run a ``simulated'' install.

    Add \texttt{-f} and omit \texttt{[packagename]} if the install
    failed because of dependency issues.
    This will search for any unsatisfied dependences and try to install
    them in an attempt to fix the dependency tree.

  \item Removing a Package and Configuration Files:
    To get rid of a package and its dependencies, use
    \begin{lstlisting}
    sudo apt-get remove [packagename]
    \end{lstlisting}
    Note however that an installed package also often has various
    configuration files and associated directories that won't be needed
    after the package has been uninstalled.
    To remove these \emph{also} (so you don't have to manually delete),
    modify the above command to
    \begin{lstlisting}
    sudo apt-get remove --purge [packagename]
    \end{lstlisting}

  \item Cleaning Up Unused Dependencies:
    Some installed packages were only installed because they were
    dependencies for other packages that might now be uninstalled.
    If there any unused dependencies that aren't associated with an
    installed program still hanging around, use the following command to
    remove them:
    \begin{lstlisting}
    sudo apt-get autoremove
    \end{lstlisting}
    Can also add \texttt{--purge} argument to remove their configuration
    files and associated directories.

  \item Upgrading Packages:
    First, always make sure you did an update before doing an upgrade.
    Next, to replace old packages with their updated selves, run
    \begin{lstlisting}
    sudo apt-get upgrade
    \end{lstlisting}
    The above command takes an installed package, removes it, and
    replaces it with one of the exact same name.
    No new packages installed, no packages uninstalled.
    This can be an issue.

    Sometimes an updated package requires a new dependency or requires
    removing one dependency and replacing it with a different one with a
    different name.
    The above command will not take care of these dependency issues
    since it cannot install or uninstall anything.
    So instead run the following command which can resolve these
    potential dependency issues.
    \begin{lstlisting}
    sudo apt-get dist-upgrade
    \end{lstlisting}
    Of course, this is a little more aggressive and we should use this
    with care probably.

  \item Cleaning up the Cached Downloaded Packages:
    Even after a package is downloaded and installed, the download is
    still cached in case the system needs to reference it.
    They are never automatically deleted even if a package is
    redownloaded and reinstalled after an update. The new download will
    be cached in addition with the old one still there.

    The following command deletes all cached downloads
    \begin{lstlisting}
    sudo apt-get clean
    \end{lstlisting}
    The following command deletes all cached downloads, except the most
    recent for each package. So it just ditches old versions.
    \begin{lstlisting}
    sudo apt-get autoclean
    \end{lstlisting}

  \item Querying Installed Packages:
    To get a list
    \begin{lstlisting}
    sudo dpkg --list
    \end{lstlisting}

  \item Installing \texttt{.deb} Packages:
    Sometimes a software vendor will just supply a deb file.
    After downloading to the current directory, we can install with
    \begin{lstlisting}
    sudo dpkg --install debfile.deb
    \end{lstlisting}
    Note that the \texttt{dpkg} command doesn't resolve dependencies, so
    if there are unmet dependencies, installation will typiclly fail.
    However, it will mark them, and we can use \texttt{apt-get} to fix
    them by running the following command, which will install unmet
    dependencies:
    \begin{lstlisting}
    sudo apt-get install -f
    \end{lstlisting}

  \item Show Dependencies and Reverse Dependencies:
    List all packages listed as a hard dependency, suggestion,
    recommendation, conflict to a given package,
    \begin{lstlisting}
    apt-cache depends [packagename]
    \end{lstlisting}
    List what \emph{depends on} the given package
    \begin{lstlisting}
    apt-cache rdepends [packagename]
    \end{lstlisting}

  \item Show Installed and Available Versions of a Package:
    Often many versions of a package within the repositories, with a
    single default.
    To see the version installed, the candidate to be installed unless
    otherwise specified, and a table of versions, run
    \begin{lstlisting}
    apt-cache policy [packagename]
    \end{lstlisting}


  \item Tracing Package Files:
    To see which files a package installed (not including configuration
    files),
    \begin{lstlisting}
    dpkg -L package
    \end{lstlisting}
    To see what package installed a file, run
    \begin{lstlisting}
    dpkg -S /path/to/file
    \end{lstlisting}

\end{itemize}






\end{document}


%%%%%%%%%%%%%%%%%%%%%%%%%%%%%%%%%%%%%%%%%%%%%%%%%%%%%%%%%%%%%%%%%%%%%%%%
%%%%%%%%%%%%%%%%%%%%%%%%%%%%%%%%%%%%%%%%%%%%%%%%%%%%%%%%%%%%%%%%%%%%%%%%
%%%%%%%%%%%%%%%%%%%%%%%%%%%%%%%%%%%%%%%%%%%%%%%%%%%%%%%%%%%%%%%%%%%%%%%%

%%%% SAMPLE CODE %%%%%%%%%%%%%%%%%%%%%%%%%%%%%%%%%%%%%%

    %% VIEW LAYOUT %%

        \layout

    %% LANDSCAPE PAGE %%

        \begin{landscape}
        \end{landscape}

    %% BIBLIOGRAPHIES %%

        \cite{LabelInSourcesFile}  %Use in text; cites
        \citep{LabelInSourcesFile} %Use in text; cites in parens

        \nocite{LabelInSourceFile} % Includes in refs w/o specific citation
        \bibliographystyle{apalike}  % Or some other style

        % To ditch the ``References'' header
        \begingroup
        \renewcommand{\section}[2]{}
        \endgroup

        \bibliography{sources} % where sources.bib has all the citation info

    %% SPACING %%

        \vspace{1in}
        \hspace{1in}

    %% URLS, EMAIL, AND LOCAL FILES %%

      \url{url}
      \href{url}{name}
      \href{mailto:mcocci@raidenlovessusie.com}{name}
      \href{run:/path/to/file.pdf}{name}


    %% INCLUDING PDF PAGE %%

        \includepdf{file.pdf}


    %% INCLUDING CODE %%

        %\verbatiminput{file.ext}
            %   Includes verbatim text from the file

        \texttt{text}
            %   Renders text in courier, or code-like, font

        \matlabcode{file.m}
            %   Includes Matlab code with colors and line numbers

        \lstset{style=bash}
        \begin{lstlisting}
        \end{lstlisting}
            % Inline code rendering


    %% INCLUDING FIGURES %%

        % Basic Figure with size scaling
            \begin{figure}[h!]
               \centering
               \includegraphics[scale=1]{file.pdf}
            \end{figure}

        % Basic Figure with specific height
            \begin{figure}[h!]
               \centering
               \includegraphics[height=5in, width=5in]{file.pdf}
            \end{figure}

        % Figure with cropping, where the order for trimming is  L, B, R, T
            \begin{figure}
               \centering
               \includegraphics[trim={1cm, 1cm, 1cm, 1cm}, clip]{file.pdf}
            \end{figure}

        % Side by Side figures: Use the tabular environment


