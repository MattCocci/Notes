\documentclass[a4paper,12pt]{scrartcl}

\author{Matthew Cocci}
\title{The Natural Logarithm and $e$}
\date{}
\usepackage{enumitem} %Has to do with enumeration	
\usepackage{amsfonts}
\usepackage{amsmath}
\usepackage{amsthm} %allows for labeling of theorems
\usepackage[T1]{fontenc}
\usepackage[utf8]{inputenc}
\usepackage{blindtext}
\usepackage{graphicx}
\usepackage[hidelinks]{hyperref} % For internal/external linking. 
				 % [hidelinks] removes boxes
% \usepackage{url} % allows for url display, non-clickable
%\numberwithin{equation}{section} 
   % This labels the equations in relation to the sections 
      % rather than other equations
%\numberwithin{equation}{subsection} %This labels relative to subsections
\newtheorem{thm}{Theorem}[section]
\newtheorem{lem}[thm]{Lemma}
\newtheorem{prop}[thm]{Proposition}
\newtheorem{cor}[thm]{Corollary}
\setkomafont{disposition}{\normalfont\bfseries}
\usepackage{appendix}
\usepackage{subfigure} % For plotting multiple figures at once
\usepackage{verbatim} % for including verbatim code from a file
\usepackage{natbib} % for bibliographies

\begin{document}
\maketitle


This note devotes its ink to one of the greatest and most important
notions in the entirety of mathematics: The transcendental constant,
$e$, and its close, the natural logarithm.
Beginning with its origins in real number theory and
calculus, the number $e$ finds additional application in complex
analysis and applied branches of mathematics, like economics and
finance.  

\section{Log Approximations}

First, using the Taylor Series expansion, we can write:
\begin{align*}
   \ln(1+x) &= x - \frac{x^2}{2} + \frac{x^3}{3} - \frac{x^4}{4} 
      + \cdots \\
   &= x + O(x^2)
\end{align*}
So for small values of $x$, the approximation is very good, as
the approximation error will be bounded $c x^2$ as $x \rightarrow 0$.
\\
\\
It is often convenient to use this property in
economics.  For example, suppose we want to measure percentage 
deviations of some variable $y$ from it's steady state value, $y^*$.
We'll denote this percentage deviation by $k$, and 
we can approximate it by log-differences:
\begin{align*}
   \frac{y}{y^*} &= 1 + \frac{k}{100} \\
   \ln \frac{y}{y^*} &= \ln\left(1 + \frac{k}{100}\right) \\
   \Rightarrow \quad \ln y - \ln y^* &\approx \frac{k}{100}
\end{align*}
Or changing up some subscripts and notation, we could imagine 
approximating period-over-period percentage growth in an asset or 
economic indicator, again using log-differences:
\begin{align*}
   \frac{y_t}{y_{t-1}} &= 1 + \frac{r}{100} \\
   \ln \frac{y_t}{y_{t-1}} &= \ln\left(1 + \frac{r}{100}\right) \\
   \Rightarrow \quad \ln y_t - \ln y_{t-1} &\approx \frac{r}{100}
\end{align*}




%%%% APPPENDIX %%%%%%%%%%%

% \appendix

%\cite{LabelInSourcesFile} 
%\citep{LabelInSourcesFile} Cites in parens
%\nocite{LabelInSourceFile} includes in refs w/o specific citation
%\bibliographystyle{apalike} 
%\bibliography{sources.bib} where sources.bib is file




\end{document}



%%%% INCLUDING FIGURES %%%%%%%%%%%%%%%%%%%%%%%%%%%%

   % H indicates here 
   %\begin{figure}[h!]
   %   \centering
   %   \includegraphics[scale=1]{file.pdf}
   %\end{figure}

%   \begin{figure}[h!]
%      \centering
%      \mbox{
%	 \subfigure{
%	    \includegraphics[scale=1]{file1.pdf}
%	 }\quad
%	 \subfigure{
%	    \includegraphics[scale=1]{file2.pdf} 
%	 }
%      }
%   \end{figure}
 

%%%%% Including Code %%%%%%%%%%%%%%%%%%%%%5
% \verbatiminput{file.ext}    % Includes verbatim text from the file
% \texttt{text}	  % includes text in courier, or code-like, font
