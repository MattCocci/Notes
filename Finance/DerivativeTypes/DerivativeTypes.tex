\documentclass[a4paper,12pt]{scrartcl}

\author{Matthew Cocci}
\title{Different Types of Derivatives}
\date{}
\usepackage{enumitem} %Has to do with enumeration	
\usepackage{amsfonts}
\usepackage{amsmath}
\usepackage{amsthm} %allows for labeling of theorems
\usepackage[T1]{fontenc}
\usepackage[utf8]{inputenc}
\usepackage{blindtext}
\usepackage{graphicx}
%\numberwithin{equation}{section} 
%, This labels the equations in relation to the sections rather than other equations
%\numberwithin{equation}{subsection} %This labels relative to subsections
\newtheorem{thm}{Theorem}[section]
\newtheorem{lem}[thm]{Lemma}
\newtheorem{prop}[thm]{Proposition}
\newtheorem{cor}[thm]{Corollary}
\setkomafont{disposition}{\normalfont\bfseries}



\begin{document}
\begin{center}
   \LARGE
   \textbf{Different Types of Derivatives}
\end{center}
In this note, I want to enumerate some different types of derivatives,
specifying their names and their payoff schemes. Broadly speaking
there are two classes of options: \textbf{European}, which can only
be exercised at maturity, and \textbf{American}, which permit early
exercise.  The different payoff schemes could be applied to both
classes.
\\
\\
{\sl \large Path Independent Options}

\paragraph{European Call on Stock} 
$\varphi(S,K) = \max\{ S-K, 0 \}$

\paragraph{European Put on Stock} 
$\varphi(S,K) = \max\{ K-S, 0 \}$

\paragraph{Cash-or-Nothing (CON) Call} A so-called digital or binary 
option:
\[\varphi(S,K) = \begin{cases} \$1 & S>K \\ 0 & S \leq K\end{cases}.\]


\paragraph{All-or-Nothing (AON) Call} A so-called digital or binary 
option:
   \[\varphi(S,K) = \begin{cases} \$S & S>K \\ 0 & S \leq K\end{cases}\]
{\sl \large Path Dependent Options}

\paragraph{Lookback Floating Call} $\varphi(S) = 
S(T) - \min_{s \in [t,T]} S(s)$.

\paragraph{Lookback Fixed Call} $\varphi(S,K) = \max\left\{ 0, 
   \max_{s \in [t,T]} S(s) - K \right\}$.

\paragraph{Asian Arithmetic Call} $\varphi(S,K) = \max\{0,
   \frac{1}{T-t} \int_t^T S(t) \; dt \}$.

\paragraph{Asian Geometric Call} $\varphi(S,K) = \max\left\{0,
   \exp\left(\frac{1}{T-t} \int_t^T \ln(S(t)) \; dt \right)\right\}$.




\end{document}

