\documentclass[12pt]{article}
\author{Matthew D. Cocci}
\title{Homework 8}
\date{\today}
%% Formatting & Spacing %%%%%%%%%%%%%%%%%%%%%%%%%%%%%%%%%%%%

%\usepackage[top=1in, bottom=1in, left=1in, right=1in]{geometry} % most detailed page formatting control
\usepackage{fullpage} % Simpler than using the geometry package; std effect
\usepackage{setspace}
%\onehalfspacing
\usepackage{microtype}


%% Header %%%%%%%%%%%%%%%%%%%%%%%%%%%%%%%%%%%%%%%%%%%%%%%%%

%\usepackage{fancyhdr}
%\pagestyle{fancy}
%\lhead{}
%\rhead{}
%\chead{}
%\setlength{\headheight}{15.2pt}
    %---Make the header bigger to avoid overlap

%\renewcommand{\headrulewidth}{0.3pt}
    %---Width of the line

%\setlength{\headsep}{0.2in}
    %---Distance from line to text


%% Mathematics Related %%%%%%%%%%%%%%%%%%%%%%%%%%%%%%%%%%%

\usepackage{amsmath}
\usepackage{amsfonts}
\usepackage{mathrsfs}
\usepackage{amsthm} %allows for labeling of theorems
\theoremstyle{plain}
\newtheorem{thm}{Theorem}[section]
\newtheorem{lem}[thm]{Lemma}
\newtheorem{prop}[thm]{Proposition}
\newtheorem{cor}[thm]{Corollary}

\theoremstyle{definition}
\newtheorem{defn}[thm]{Definition}
\newtheorem{ex}[thm]{Example}

\theoremstyle{remark}
\newtheorem*{rem}{Remark}
\newtheorem*{note}{Note}

% Below supports left-right alignment in matrices so the negative
% signs don't look bad
\makeatletter
\renewcommand*\env@matrix[1][c]{\hskip -\arraycolsep
  \let\@ifnextchar\new@ifnextchar
  \array{*\c@MaxMatrixCols #1}}
\makeatother


%% Font Choices %%%%%%%%%%%%%%%%%%%%%%%%%%%%%%%%%%%%%%%%%

\usepackage[T1]{fontenc}
\usepackage{lmodern}
\usepackage[utf8]{inputenc}
%\usepackage{blindtext}


%% Figures %%%%%%%%%%%%%%%%%%%%%%%%%%%%%%%%%%%%%%%%%%%%%%

\usepackage{graphicx}
\usepackage{subfigure}
    %---For plotting multiple figures at once
%\graphicspath{ {Directory/} }
    %---Set a directory for where to look for figures


%% Hyperlinks %%%%%%%%%%%%%%%%%%%%%%%%%%%%%%%%%%%%%%%%%%%%
\usepackage{hyperref}
\hypersetup{
    colorlinks,
        %---This colors the links themselves, not boxes
    citecolor=black,
        %---Everything here and below changes link colors
    filecolor=black,
    linkcolor=black,
    urlcolor=black
}

%% Including Code %%%%%%%%%%%%%%%%%%%%%%%%%%%%%%%%%%%%%%%

\usepackage{verbatim}
    %---For including verbatim code from files, no colors

\usepackage{listings}
\usepackage{color}
\definecolor{mygreen}{RGB}{28,172,0}
\definecolor{mylilas}{RGB}{170,55,241}
\newcommand{\matlabcode}[1]{%
    \lstset{language=Matlab,%
        basicstyle=\footnotesize,%
        breaklines=true,%
        morekeywords={matlab2tikz},%
        keywordstyle=\color{blue},%
        morekeywords=[2]{1}, keywordstyle=[2]{\color{black}},%
        identifierstyle=\color{black},%
        stringstyle=\color{mylilas},%
        commentstyle=\color{mygreen},%
        showstringspaces=false,%
            %---Without this there will be a symbol in
            %---the places where there is a space
        numbers=left,%
        numberstyle={\tiny \color{black}},%
            %---Size of the numbers
        numbersep=9pt,%
            %---Defines how far the numbers are from the text
        emph=[1]{for,end,break,switch,case},emphstyle=[1]\color{red},%
            %---Some words to emphasise
    }%
    \lstinputlisting{#1}
}
    %---For including Matlab code from .m file with colors,
    %---line numbering, etc.

%% Bibliographies %%%%%%%%%%%%%%%%%%%%%%%%%%%%%%%%%%%%

%\usepackage{natbib}
    %---For bibliographies
%\setlength{\bibsep}{3pt} % Set how far apart bibentries are

%% Misc %%%%%%%%%%%%%%%%%%%%%%%%%%%%%%%%%%%%%%%%%%%%%%

\usepackage{enumitem}
    %---Has to do with enumeration
\usepackage{appendix}
\usepackage{pdfpages}
    %---For including whole pdf pages as a page in doc


%% User Defined %%%%%%%%%%%%%%%%%%%%%%%%%%%%%%%%%%%%%%%%%%

%\newcommand{\nameofcmd}{Text to display}



%%%%%%%%%%%%%%%%%%%%%%%%%%%%%%%%%%%%%%%%%%%%%%%%%%%%%%%%%%%%%%%%%%%%%%%%
%% BODY %%%%%%%%%%%%%%%%%%%%%%%%%%%%%%%%%%%%%%%%%%%%%%%%%%%%%%%%%%%%%%%%
%%%%%%%%%%%%%%%%%%%%%%%%%%%%%%%%%%%%%%%%%%%%%%%%%%%%%%%%%%%%%%%%%%%%%%%%


\begin{document}
\maketitle

%\tableofcontents


\begin{enumerate}
  \item % Question 1
    \begin{enumerate}
      \item % Question 1a
        We want to convert the following into Ito form:
        \begin{align*}
          dX_t = aX_t dt + b X_t \circ dB_t
        \end{align*}
        From this, we can identify the drift $d$ and the diffusion
        coefficient $\sigma$
        \begin{align*}
          d = aX_t \qquad
          \sigma = bX_t \qquad
          \sigma'_x = b
        \end{align*}
        Using the conversion formula, the Ito form is
        \begin{align*}
          dX_t
            &= \left( aX_t + \frac{1}{2} b (bX_t) \right) dt
          + bX_t dB_t \\
          &= \left( a + \frac{1}{2} b^2 \right) X_t dt
            + bX_t dB_t
        \end{align*}

      \item % Question 1b
        We want to convert the following into Ito form:
        \begin{align*}
          dX_t = \sin X_t \cos X_t dt + (t^2 + cos X_t) \circ dB_t
        \end{align*}
        From this, we can identify the drift $d$ and the diffusion
        coefficient $\sigma$
        \begin{align*}
          d = \sin X_t \cos X_t \qquad
          \sigma = t^2 + \cos X_t \qquad
          \sigma'_x = -\sin X_t
        \end{align*}
        Using the conversion formula, the Ito form is
        \begin{align*}
          dX_t
            &= \left( \sin X_t \cos X_t
            - \frac{1}{2} \sin X_t (t^2 + \cos X_t)\right)
            + (t^2+\cos X_t) dB_t \\
            &= \frac{1}{2} \left( \sin X_t \cos X_t
            - t^2 \sin X_t \right)
            + (t^2+\cos X_t) dB_t \\
            &= \frac{1}{2} \left( \cos X_t
            - t^2  \right) \sin X_t
            + (t^2+\cos X_t) dB_t
        \end{align*}

      \item % Question 1c
        We want to convert the following into Stratonovich form:
        \begin{align*}
          dX_t = r X_t dt + \alpha X_t dB_t
        \end{align*}
        From this, we can identify the drift $d$ and the diffusion
        coefficient $\sigma$
        \begin{align*}
          d = rX_t \qquad
          \sigma = \alpha X_t \qquad
          \sigma'_x = \alpha
        \end{align*}
        Using the conversion formula, the Ito form is
        \begin{align*}
          dX_t
          &= \left( rX_t - \frac{1}{2} \alpha (\alpha X_t)\right)dt
          + \alpha X_t \circ dB_t \\
          &= \left( r - \frac{1}{2} \alpha^2 \right)X_t dt
          + \alpha X_t \circ dB_t
        \end{align*}

      \item % Question 1d
        We want to convert the following into Stratonovich form:
        \begin{align*}
          dX_t = 2e^{-X_t} dt + X^2_t dB_t
        \end{align*}
        From this, we can identify the drift $d$ and the diffusion
        coefficient $\sigma$
        \begin{align*}
          d = 2e^{-X_t} \qquad
          \sigma = X^2_t \qquad
          \sigma'_x = 2X_t
        \end{align*}
        Using the conversion formula, the Ito form is
        \begin{align*}
          dX_t
          &= \left( 2e^{-X_t} - \frac{1}{2} (2X_t) (X_t^2)\right)dt
            + X^2_t \circ dB_t \\
          &= \left( 2e^{-X_t} - X^3_t\right)dt
            + X^2_t \circ dB_t
        \end{align*}

    \end{enumerate}

  \item % Question 2
    We want to verify that
    \begin{align}
      df(X_t) = f'(X_t) \circ dX_t =
      f'(X_t) b(X_t,t) dt + f'(X_t) \sigma(X_t,t) \circ dW_t
      \label{q2.1}
    \end{align}
    for process $X_t$ that solves
    \begin{align}
      \label{q2.2}
      dX_t = b(X_t,t) dt + \sigma(X_t,t)\circ dW_t
    \end{align}
    %Since $f$ is invertible, we can also write this as
    %\begin{align}
      %\label{q2.2b}
      %dX_t
      %&= b\left(f^{-1}(f(X_t)),t\right) dt
        %+ \sigma\left(f^{-1}(f(X_t)),t\right)\circ dW_t\\
      %&= \hat{b}\left(f(X_t),t\right) dt
        %+ \hat{\sigma}\left(f(X_t),t\right)\circ dW_t
    %\end{align}
    %where $\hat{b}\left(f(X_t),t\right) =
    %b\left(f^{-1}(f(X_t)),t\right)$ and
    %$\hat{\sigma}\left(f(X_t),t\right) =
    %\sigma\left(f^{-1}(f(X_t)),t\right)$.
    Note that in the work below, I will abuse notation so that
    $b(X_t,t)$ becomes $b$ and $\sigma(X_t,t)$ becomes $\sigma$. This is
    purely for readability, and it is implicit that both $b$ and
    $\sigma$ are functions of $X_t$ and $t$. Moreover,
    $\sigma'_x:=\frac{\partial\sigma}{\partial x}$ and, since $f$ is
    only a function of $X_t$ (not~$t$), $f':=\frac{df}{dx}$
    and $f'':=\frac{d^2f}{dx^2}$.

    With that, we start by converting Equation~\ref{q2.2} to Ito form:
    \begin{align*}
      dX_t = \left( b + \frac{1}{2} \sigma_x' \sigma
        \right) dt
        + \sigma dW_t
    \end{align*}
    Next, apply Ito's lemma to the above equation:
    \begin{align}
      df(X_t) &=
      0 \cdot dt
      + f'(X_t) dX_t
      + \frac{1}{2} f''(X_t) (dX_t)^2\notag\\
      &=
      f'
        \left[
          \left(
          b + \frac{1}{2} \sigma_x' \sigma
          \right)
        dt
        + \sigma dW_t
      \right]\notag\\
      &\qquad
      + \frac{1}{2} f''
      \left[\left( b + \frac{1}{2}\cdot
        \sigma_x' \cdot \sigma
        \right) dt
      + \sigma dW_t\right]^2\notag\\
      &= \left(
      b f'
      + \frac{1}{2} f' \sigma'_x \sigma
        + \frac{1}{2} f'' \sigma^2
        \right) dt
        + f' \sigma dW_t\notag\\
      \Rightarrow \qquad
      df(X_t)
      &= f' \left[ b dt + \sigma dW_t\right]
        + \frac{1}{2}
          \left( f' \sigma'_x \sigma +  f'' \sigma^2\right) dt
          \label{q2.3}
    \end{align}
    This equation (in Ito form) looks a lot like what we want. The only
    thing left to show is that the second term in the sum above $\left(
    f' \sigma'_x \sigma +  f'' \sigma^2\right)$ is the drift correction
    we would get from converting Equation~\ref{q2.1} from Stratonovich
    to Ito form.

    So let's compute that object, using the hint that $\frac{d}{df} =
    \left(\frac{df}{dx}\right)^{-1} \frac{d}{dx}$ to simplify. We apply
    $\frac{d}{df}$ to the drift of (\ref{q2.1}) because we need to
    differentiate the drift with respect to the object on the lefthand
    side, $f(X_t)$. So now
    \begin{align*}
      \text{Drift correction for (\ref{q2.1})}
      &= \frac{1}{2}
        \left(
        \frac{d}{df} \left[f' \sigma\right] \cdot f'\sigma
        \right)\\
      &= \frac{1}{2}
        \left(
        \left(f'\right)^{-1} \frac{d}{dx}
        \left[f' \sigma\right] \cdot f'\sigma
        \right)\\
      &= \frac{1}{2}
        \left(
        \left(f'\right)^{-1}
        \left[f'' \sigma + f' \sigma'_x \right] \cdot f'\sigma
        \right)\\
      &= \frac{1}{2}
        \left(
        f'' \sigma^2 + f' \sigma'_x \sigma
        \right)
    \end{align*}
    which is exactly the extra term in Equation~\ref{q2.3}. Therefore,
    when we convert that object back into Stratonovich form, we would
    subtract off precisely that amount to get
    \begin{align*}
      df(X_t)
      &= f'(X_t) b(X_t,t) dt + f'(X_t) \sigma(X_t,t) \circ dW_t
    \end{align*}

  \item % Question 3
    Wasn't able to work this one out.
    %We have the following differential equations:
    %\begin{align*}
      %dX_1 &= \frac{1}{|X|^2}
        %\begin{pmatrix}
          %X_2^2 & -X_1X_2
        %\end{pmatrix}
        %\circ dW_t \\
      %dX_2 &= \frac{1}{|X|^2}
        %\begin{pmatrix}
          %-X_1X_2 & X_1^2
        %\end{pmatrix}
        %\circ dW_t \\
    %\end{align*}

  \item % Question 4
    We assume that $X_t$ solves
    \begin{align*}
      dX_t = -\alpha X_t dt + \sigma dB_t
    \end{align*}
    Following the example in the notes, writing this in physicists' notation:
    \begin{align}
      \frac{dX_t}{dt} &= -\alpha X_t + \sigma \eta_t \notag \\
      \Leftrightarrow\qquad
      \frac{dX_t}{dt} &+\alpha X_t =  \sigma \eta_t
      \label{q4.1}
    \end{align}
    \begin{enumerate}
      \item
        We now want to consider the spectral form of (\ref{q4.1}),
        \begin{align*}
          i\lambda \hat{X}_\lambda + \alpha \hat{X}_\lambda = \sigma \hat{\eta}
        \end{align*}
        where $\hat{X}_\lambda$ stands for the spectra measure
        $dZ_X(\lambda)$ and $\hat{\eta}$ stands for the spectral measure
        of $\eta$, which is such that $|\hat{\eta}|^2 = \frac{1}{2\pi}$
        as in the notes.

        From there, we can solve for $\hat{X}_\lambda$:
        \begin{align*}
          \hat{X}_\lambda = \frac{\sigma \hat{\eta}}{\alpha + \lambda i}
          \quad\Rightarrow\quad
          |\hat{X}_\lambda|^2 = \frac{1}{2\pi}
          \left(\frac{\sigma^2}{\alpha^2 + \lambda^2}\right)
          = S(\lambda)
        \end{align*}

      \item % Question 4b
        Not sure if this is the correct approach at all, but maybe start
        by writing
        \begin{align*}
          dW_t &= d\left( X_t\cos kt + Y_t\sin kt\right) \\
               &= d\left( X_t\cos kt\right) + d\left(Y_t\sin kt\right) \\
          &= (\cos kt) dX_t - (k (\sin kt) dt) X_t
              + (\sin kt) dY_t + (k(\cos kt) dt)dY_t
        \end{align*}
        Since $dt^2$ and $dW_t dt$ terms drop, the above reduces to
        \begin{align*}
          dW_t
          &= (\cos kt) dX_t + (\sin kt) dY_t\\
          &= (\cos kt) \left( -\alpha X_t dt + \sigma dB_{t,1} \right)
          + (\sin kt)\left( -\alpha Y_t dt + \sigma dB_{t,2}\right)
        \end{align*}
        The $\alpha$ and $\sigma$ terms are identical between equations
        because the processes are iid. Next, rearrange
        \begin{align*}
          dW_t
          &= -\alpha  \left(X_t (\cos kt) + Y_t(\sin kt) \right)dt + \sigma dB_{t,1} + \sigma dB_{t,2}\\
          &= -\alpha  W_t dt + \sigma dB_{t,1} + \sigma dB_{t,2}
        \end{align*}
        In physicists' notation:
        \begin{align*}
          \frac{dW_t}{dt} + \alpha  W_t
          = \sigma \eta_{t,1} + \sigma \eta_{t,2}
        \end{align*}
        Then, we can write this in spectral form as above:
        \begin{align*}
          i\lambda \hat{X}_\lambda + \alpha  \hat{X}_\lambda
          = \sigma \hat{\eta}_{t,1} + \sigma \hat{\eta}_{t,2}
        \end{align*}
        Then solve for $\hat{X}_\lambda$:
        \begin{align*}
          \hat{X}_\lambda =
          \frac{\sigma (\hat{\eta}_{t,1} + \hat{\eta}_{t,2})}{\alpha + i\lambda}
          \quad\Rightarrow\quad
          |\hat{X}_\lambda|^2 =
          \frac{\sigma^2 (\hat{\eta}_{t,1} + \hat{\eta}_{t,2})^2}{\alpha^2 + \lambda^2}
        \end{align*}
        Using the fact that the disturbance terms are uncorrelated and
        that
        \begin{align*}
          |\hat{\eta}_{t,1}|=|\hat{\eta}_{t,2}|=\frac{1}{2\pi}
        \end{align*}
        We get
        \begin{align*}
          (\hat{\eta}_{t,1}+\hat{\eta}_{t,2})^2 = \frac{1}{2\pi}+\frac{1}{2\pi} = \frac{1}{\pi}
        \end{align*}
        so that we get
        \begin{align*}
          |\hat{X}_\lambda|^2 =
          \frac{1}{\pi}
          \frac{\sigma^2}{\alpha^2 + \lambda^2}
        \end{align*}

    \end{enumerate}

  \item % Question 5
    \begin{enumerate}
      \item % Question 5a
        We want to solve
        \begin{align*}
          dX_t &= t dt + 2 dB_t \\
          \Rightarrow \qquad
          \int^t_0 dX_s &= \int^t_0 s ds + 2 \int^t_0 dB_s \\
          X_t - X_0 &= (t-0) + 2 (B_s - B_0) \\\\
          \Rightarrow \qquad
          X_t &= \xi + t + 2B_t
        \end{align*}
        As a result, we have
        \begin{align*}
          EX_t &= E\left[ \xi + t + 2B_t \right] \\
          &= 0 + t + 0 = t
        \end{align*}

      \item % Question 5b
        Next, we want to solve
        \begin{align}
          \label{q5b.1}
          dX_t = (\sin t) X_t dt + dB_t
          \end{align}
        In the case with no noise term, we would expect a solution of
        the form $Ce^{\cos t}$. So set
        \begin{align*}
          A_t = e^{\cos t}
          \quad \Rightarrow\quad
          dA_t = -(\sin t) e^{\cos t} dt
        \end{align*}
        From there, if we multiply through Equation~\ref{q5b.1} by $A_t$
        and rearrange, we get
        \begin{align}
          \label{q5b.2}
          e^{\cos t} dX_t - X_t (\sin t) e^{\cos t} dt = e^{\cos t} dB_t
        \end{align}
        We can then identify $A_t$, and $dA_t$ more explicitly in
        Equation~\ref{q5b.2} above:
        \begin{align}
          A_t dX_t + X_t dA_t = e^{\cos t} dB_t
          \label{q5b.3}
        \end{align}
        Next, note that if we were to compute $dA_t dX_t$ (using the
        rightand side of Expression~\ref{q5b.1} as $dX_t$), the result
        would be zero since the result would be a sum of multiples of
        $dt^2$ and $dB_t dt$ which are zero. This allows us to use the
        product rule $d(A_t X_t) = A_t dX_t + X_t dA_t + dA_t dX_t$ in
        reverse to simplify the lefthand side of Equation~\ref{q5b.3} to
        \begin{align}
          d(A_t X_t) = d\left( e^{\cos t} X_t\right)
          %= e^{\cos t} dB_t
        \end{align}
        Using this in (\ref{q5b.3}) and integrating, we have
        \begin{align*}
          \int^t_0 d\left( e^{\cos s} X_s\right) &=
          \int^t_0 e^{\cos s} dB_s \\
          e^{\cos t} X_t - e^{\cos 0} X_0 &=
          \int^t_0 e^{\cos s} dB_s \\
          \Rightarrow
          X_t &= e^{1-\cos t} X_0 +
          \int^t_0 e^{\cos s - \cos t} dB_s\\
          X_t &= e^{1-\cos t} \xi +
          \int^t_0 e^{\cos s - \cos t} dB_s
        \end{align*}
        As a result, $EX_t = e^{1-\cos t} E\left[ \xi\right] = 0$
        because $E\xi = 0$ by assumption.

      \item % Question 5c
        Next, we want to solve
        \begin{align}
          \label{q5c.1}
          dX_t = (1-X_t)dt + dB_t
        \end{align}
        To get a candidate for the integrating factor, we can calculate
        the solution to an ODE without the noise term, which would
        suggest $A_t = 1-e^{-t}$. However, this didn't work, so I tried
        other integrating factors like the one above and eventually
        settled on
        \begin{equation}
          A_t = e^t \qquad \Rightarrow \qquad dA_t = e^t dt
        \end{equation}
        Multiplying through Equation~\ref{q5c.1} by $A_t$ and
        rearranging we get
        \begin{align*}
          e^t dX_t + X_t e^t dt = e^t dt + e^t dB_t
        \end{align*}
        Identifying $A_t$ and $dA_t$ in the equation above leaves
        \begin{align*}
          A_t dX_t + X_t dA_t &= e^t dt + e^t dB_t \\
          d(A_t X_t) &= e^t dt + e^t dB_t
        \end{align*}
        Integrating, we get
        \begin{align*}
          \int^t_0  d(e^s X_s)
            &= \int^t_0 e^s ds + \int^t_0 e^s dB_s \\
          e^t X_t - e^0 X_0
            &= (e^s)^t_0 + \int^t_0 e^s dB_s \\
          X_t
            &= e^{-t} X_0 + (1 - e^{-t}) + \int^t_0 e^{s-t} dB_s\\
          X_t
            &= e^{-t} \xi + (1 - e^{-t}) + \int^t_0 e^{s-t} dB_s
        \end{align*}
        From there, we can obtain the expression
        \begin{align*}
          E[X_t]
            &= E\left[e^{-t} \xi + (1 - e^{-t})
                  + \int^t_0 e^{s-t} dB_s\right] \\
            &= e^{-t} E[\xi] + (1 - e^{-t})\\
            &= (1 - e^{-t})
        \end{align*}

    \end{enumerate}

  \item % Question 6
    \begin{enumerate}
      \item % Question 6a
        We want to find $dG_t$ where
        \begin{align*}
          G(B_t,t) = \exp\left\{ \frac{\alpha^2 t}{2} - \alpha B_t\right\}
        \end{align*}

        We will apply Ito's formula, treating $G$ as a function of $X_t
        = W_t$ which we know satisfies the SDE $dX_t = dW_t$. Therefore,
        we find the partial derivatives:
        \begin{align*}
          g(x,t) &= \exp\left\{ \frac{\alpha^2 t}{2} - \alpha x\right\}\\
          \frac{\partial g}{\partial t} &=
            \frac{\alpha^2}{2}
            \exp\left\{ \frac{\alpha^2 t}{2} - \alpha x\right\}
            = \frac{\alpha^2}{2} g(x,t)\\
          \frac{\partial g}{\partial x} &=
            -\alpha\exp\left\{ \frac{\alpha^2 t}{2} - \alpha x\right\}
            = -\alpha g(x,t)\\
          \frac{\partial^2 g}{\partial x^2} &=
            \alpha^2\exp\left\{ \frac{\alpha^2 t}{2} - \alpha x\right\}
            = \alpha^2 g(x,t)
        \end{align*}
        Substituting into Ito's formula:
        \begin{align*}
          dG_t =
            \frac{\alpha^2}{2} G_t dt
            -\alpha G_t (dX_t)
            + \frac{1}{2}\alpha^2 G_t (dX_t)^2
        \end{align*}
        Subsituting in $dX_t=dW_t$ and simplifying, we get
        \begin{align*}
          dG_t &=
            \frac{\alpha^2}{2} G_t dt
            -\alpha G_t dW_t
            + \frac{1}{2}\alpha^2 G_t dt\\
          \Rightarrow \qquad
          dG_t
          &= {\alpha^2} G_t dt
            -\alpha G_t dB_t
        \end{align*}
        Next, we find $d(X_tG_t)$ using the product rule, the definition
        of $dX_t$, and $dG_t$ which we just found:
        \begin{align*}
          d(X_t G_t)
          &= X_t dG_t + G_t dX_t +  dX_tdG_t \\
          &=
            X_t \left( {\alpha^2} G_t dt -\alpha G_t dB_t\right)
            + G_t \left( \frac{1}{X_t} dt + \alpha X_t dB_t\right) \\
          &\qquad
            + \left( \frac{1}{X_t} dt + \alpha X_t dB_t\right)
            \left( {\alpha^2} G_t dt -\alpha G_t dB_t\right) \\
          &= \left(
            \alpha^2 X_t G_t + \frac{G_t}{X_t} -\alpha^2 X_t G_t
          \right) dt
          + \left(
            -\alpha X_t G_t + \alpha G_t X_t
          \right) dB_t\\
          \Rightarrow
          d(X_t G_t)
          &= \frac{G_t}{X_t} dt
        \end{align*}

      \item % Question 6b
        We set $Y_t = xG$, and, using the result above, we have
        \begin{align*}
          \frac{dY_t}{dt} = \frac{G}{x}
        \end{align*}
        Substituting in $x=Y_t/G$, we get
        \begin{align*}
          \frac{dY_t}{dt} &= \frac{G^2}{Y_t}
          \\
          \Leftrightarrow \quad
          Y_t dY_t &= G^2 dt
        \end{align*}

      \item % Question 6c
        Since the above equation is deterministic, we can solve it like
        a usual ODE for $Y_t$:
        \begin{align*}
          \int^t_0 Y_s dY_s &= \int^t_0 G_s^2 ds \\
          \left(\frac{Y_s^2}{2}\right)^t_0
          &= \int^t_0 G_s^2 ds \\
          Y_t
          &= \left[\left(Y_0\right)^2 + 2\int^t_0 G_s^2 ds\right]^{1/2} \\
        \end{align*}
        Substituting in $Y_t = X_t G_t$ and solving for $X_t$, we get
        \begin{align*}
          X_t G_t
          &= \left[\left(X_0G_0\right)^2
            + 2\int^t_0 G_s^2 ds\right]^{1/2} \\
          X_t
          &= \frac{1}{G_t}\left[\left(X_0G_0\right)^2
            + 2\int^t_0 G_s^2 ds\right]^{1/2}
        \end{align*}
        Lastly, plugging in the formula for $G_t$ (and using $G_0=1$, we
        can simplify further:
        \begin{align*}
          X_t
          &= \frac{%
            \left[\left(X_0\right)^2
            + 2\int^t_0 e^{2(\alpha^2 s/2 -\alpha B_s) } ds\right]^{1/2}
          }{e^{ \alpha^2 t/2 - \alpha B_t} }
        \end{align*}

    \end{enumerate}

  \item % Question 7
    Start by defining
    \begin{align*}
      \int^T_0 f(t,\omega) * dW_t := \lim_{\Delta t\rightarrow 0}
        \sum_j f(t_{j+1}) \Delta W_{t_j}
    \end{align*}
    Suppose that $X_t$ solves an right-endpoint SDE of form
    \begin{align}
      \label{q7.0}
      dX_t = b(t,X_t) dt + \sigma(t,X_t) * dW_t
    \end{align}
    and an Ito SDE of form
    \begin{align}
      \label{q7.1}
      dX_t = \alpha(t,X_t)\; dt + \beta(t,X_t)\; dW_t
    \end{align}
    If we consider the sum corresponding to the integral
    \begin{align*}
      \int^T_0 \sigma(t,X_t) * dW_t
      =
      \lim_{\Delta t\rightarrow 0}
      \sum_j \sigma(t_{j+1},X_{t_{j+1}}) \Delta W_{t_j}
    \end{align*}
    we can Taylor-expand $\sigma(t_{j+1},X_{t_{j+1}})$ about each point
    $t_j$ and combine this with the approximation
    \begin{align}
      \Delta X_{t_{j}} \approx \alpha(t_j,X_{t_j})
      \Delta t_j + \beta(t_j , X_{t_j}) \Delta W_{t_j}
      \label{q7.2b}
    \end{align}
    to approximate the sum
    \begin{align*}
      \sum_j \sigma(t_{j+1},X_{t_{j+1}}) \Delta W_{t_j}
      &\approx
      \sum_j
      \left(
        \sigma(t_j,X_{t_j}) + \partial_t \sigma(t_j,X_{t_j}) \Delta t_j
        + \partial_x \sigma(t_j,X_{t_j})\Delta X_{t_j}
      \right)\Delta W_j\\
      &=
      \sum_j
      \sigma(t_j,X_{t_j})\Delta W_j
        + \partial_t \sigma(t_j,X_{t_j}) \Delta t_j\Delta W_j\\
      \text{(\ref{q7.2b}) for $\Delta X_{t_j}$} \qquad
      &\qquad + \partial_x \sigma(t_j,X_{t_j})\left(
        \alpha(t_j,X_{t_j})
          \Delta t_j + \beta(t_j, X_{t_j}) \Delta W_{t_j}
      \right)\Delta W_j\\
      &=
      \sum_j
      \sigma(t_j,X_{t_j})\Delta W_j
        + \partial_t \sigma(t_j,X_{t_j}) \Delta t_j\Delta W_j\\
      &\qquad + \partial_x \sigma(t_j,X_{t_j})
        \alpha(t_j,X_{t_j}) \Delta t_j \Delta W_j
      + \partial_x \sigma(t_j,X_{t_j})\beta(t_j , X_{t_j}) (\Delta W_{t_j})^2
    \end{align*}
    Taking the limit of this object as $\Delta t \rightarrow 0$ kills
    the $\Delta t_j \Delta W_{t_j}$ terms and turns $(\Delta W_{t_j})^2$
    into $dt$, giving
    \begin{align*}
      \lim_{\Delta t \rightarrow 0}
      \sum_j \sigma(t_{j+1},X_{t_{j+1}}) \Delta W_{t_j}
      =
      \int^t_0 \sigma(s,X_{s}) dW_s
      + \int^t_0 \partial_x \sigma(s,X_{s})\beta(s , X_{s})ds
    \end{align*}
    Using this result, we have that
    \begin{align}
      dX_t &= b(t,X_t) dt + \sigma(t,X_t) * dW_t \notag \\
      \Rightarrow\quad
      dX_t &= \left(b(t,X_t) +\partial_x \sigma(t,X_t) \beta(t,X_t)\right)dt + \sigma(t,X_t) dW_t
      \label{q7.2}
    \end{align}
    But we also assumed that $X_t$ satisfies Ito SDE (\ref{q7.1}). To
    get (\ref{q7.1}) to match up with (\ref{q7.2}), we need:
    \begin{align*}
      \beta(t,X_t) = \sigma(t,X_t) \qquad\qquad
      \alpha(t,X_t) = b(t,X_t) + \partial_x \sigma(t,X_t)\sigma(t,X_t)
    \end{align*}
    so that the Ito version of (\ref{q7.0}) is
    \begin{align*}
      dX_t &= \left(b(t,X_t) +\partial_x \sigma(t,X_t) \sigma(t,X_t)\right)dt + \sigma(t,X_t) dW_t
    \end{align*}




\end{enumerate}

\end{document}


%%%%%%%%%%%%%%%%%%%%%%%%%%%%%%%%%%%%%%%%%%%%%%%%%%%%%%%%%%%%%%%%%%%%%%%%
%%%%%%%%%%%%%%%%%%%%%%%%%%%%%%%%%%%%%%%%%%%%%%%%%%%%%%%%%%%%%%%%%%%%%%%%
%%%%%%%%%%%%%%%%%%%%%%%%%%%%%%%%%%%%%%%%%%%%%%%%%%%%%%%%%%%%%%%%%%%%%%%%

%%%% SAMPLE CODE %%%%%%%%%%%%%%%%%%%%%%%%%%%%%%%%%%%%%%

    %% BIBLIOGRAPHIES %%

        \cite{LabelInSourcesFile}  %Use in text; cites
        \citep{LabelInSourcesFile} %Use in text; cites in parens

        \nocite{LabelInSourceFile} % Includes in refs w/o specific citation
        \bibliographystyle{apalike}  % Or some other style

        % To ditch the ``References'' header
        \begingroup
        \renewcommand{\section}[2]{}
        \endgroup

        \bibliography{sources} % where sources.bib has all the citation info

    %% SPACING %%

        \vspace{1in}
        \hspace{1in}


    %% INCLUDING PDF PAGE %%

        \includepdf{file.pdf}


    %% INCLUDING CODE %%

        \verbatiminput{file.ext}
            %---Includes verbatim text from the file
        \texttt{text}
            %---Renders text in courier, or code-like, font

        \matlabcode{file.m}
            %---Includes Matlab code with colors and line numbers


    %% INCLUDING FIGURES %%

        % Basic Figure with size scaling
            \begin{figure}[h!]
               \centering
               \includegraphics[scale=1]{file.pdf}
            \end{figure}

        % Basic Figure with specific height
            \begin{figure}[h!]
               \centering
               \includegraphics[height=5in, width=5in]{file.pdf}
            \end{figure}

        % Figure with cropping, where the order for trimming is  L, B, R, T
            \begin{figure}
               \centering
               \includegraphics[trim={1cm, 1cm, 1cm, 1cm}, clip]{file.pdf}
            \end{figure}


        % Side by Side figures
            \begin{figure}[h!]
                \centering
                \mbox{\subfigure{
                    \includegraphics[scale=1]{file1.pdf}
                }\quad\subfigure{
                    \includegraphics[scale=1]{file2.pdf}
                }
                }
            \end{figure}


