\documentclass[12pt]{article}

\author{Matthew D. Cocci}
\title{Homework 5}
\date{\today}
%% Formatting & Spacing %%%%%%%%%%%%%%%%%%%%%%%%%%%%%%%%%%%%

%\usepackage[top=1in, bottom=1in, left=1in, right=1in]{geometry} % most detailed page formatting control
\usepackage{fullpage} % Simpler than using the geometry package; std effect
\usepackage{setspace}
%\onehalfspacing
\usepackage{microtype}


%% Header %%%%%%%%%%%%%%%%%%%%%%%%%%%%%%%%%%%%%%%%%%%%%%%%%

%\usepackage{fancyhdr}
%\pagestyle{fancy}
%\lhead{}
%\rhead{}
%\chead{}
%\setlength{\headheight}{15.2pt}
    %---Make the header bigger to avoid overlap

%\renewcommand{\headrulewidth}{0.3pt}
    %---Width of the line

%\setlength{\headsep}{0.2in}
    %---Distance from line to text


%% Mathematics Related %%%%%%%%%%%%%%%%%%%%%%%%%%%%%%%%%%%

\usepackage{amsmath}
\usepackage{amsfonts}
\usepackage{mathrsfs}
\usepackage{amsthm} %allows for labeling of theorems
\theoremstyle{plain}
\newtheorem{thm}{Theorem}[section]
\newtheorem{lem}[thm]{Lemma}
\newtheorem{prop}[thm]{Proposition}
\newtheorem{cor}[thm]{Corollary}

\theoremstyle{definition}
\newtheorem{defn}[thm]{Definition}
\newtheorem{ex}[thm]{Example}

\theoremstyle{remark}
\newtheorem*{rem}{Remark}
\newtheorem*{note}{Note}

% Below supports left-right alignment in matrices so the negative
% signs don't look bad
\makeatletter
\renewcommand*\env@matrix[1][c]{\hskip -\arraycolsep
  \let\@ifnextchar\new@ifnextchar
  \array{*\c@MaxMatrixCols #1}}
\makeatother


%% Font Choices %%%%%%%%%%%%%%%%%%%%%%%%%%%%%%%%%%%%%%%%%

\usepackage[T1]{fontenc}
\usepackage{lmodern}
\usepackage[utf8]{inputenc}
%\usepackage{blindtext}


%% Figures %%%%%%%%%%%%%%%%%%%%%%%%%%%%%%%%%%%%%%%%%%%%%%

\usepackage{graphicx}
\usepackage{subfigure}
    %---For plotting multiple figures at once
%\graphicspath{ {Directory/} }
    %---Set a directory for where to look for figures


%% Hyperlinks %%%%%%%%%%%%%%%%%%%%%%%%%%%%%%%%%%%%%%%%%%%%
\usepackage{hyperref}
\hypersetup{
    colorlinks,
        %---This colors the links themselves, not boxes
    citecolor=black,
        %---Everything here and below changes link colors
    filecolor=black,
    linkcolor=black,
    urlcolor=black
}

%% Including Code %%%%%%%%%%%%%%%%%%%%%%%%%%%%%%%%%%%%%%%

\usepackage{verbatim}
    %---For including verbatim code from files, no colors

\usepackage{listings}
\usepackage{color}
\definecolor{mygreen}{RGB}{28,172,0}
\definecolor{mylilas}{RGB}{170,55,241}
\newcommand{\matlabcode}[1]{%
    \lstset{language=Matlab,%
        basicstyle=\footnotesize,%
        breaklines=true,%
        morekeywords={matlab2tikz},%
        keywordstyle=\color{blue},%
        morekeywords=[2]{1}, keywordstyle=[2]{\color{black}},%
        identifierstyle=\color{black},%
        stringstyle=\color{mylilas},%
        commentstyle=\color{mygreen},%
        showstringspaces=false,%
            %---Without this there will be a symbol in
            %---the places where there is a space
        numbers=left,%
        numberstyle={\tiny \color{black}},%
            %---Size of the numbers
        numbersep=9pt,%
            %---Defines how far the numbers are from the text
        emph=[1]{for,end,break,switch,case},emphstyle=[1]\color{red},%
            %---Some words to emphasise
    }%
    \lstinputlisting{#1}
}
    %---For including Matlab code from .m file with colors,
    %---line numbering, etc.

%% Bibliographies %%%%%%%%%%%%%%%%%%%%%%%%%%%%%%%%%%%%

%\usepackage{natbib}
    %---For bibliographies
%\setlength{\bibsep}{3pt} % Set how far apart bibentries are

%% Misc %%%%%%%%%%%%%%%%%%%%%%%%%%%%%%%%%%%%%%%%%%%%%%

\usepackage{enumitem}
    %---Has to do with enumeration
\usepackage{appendix}
\usepackage{pdfpages}
    %---For including whole pdf pages as a page in doc


%% User Defined %%%%%%%%%%%%%%%%%%%%%%%%%%%%%%%%%%%%%%%%%%

%\newcommand{\nameofcmd}{Text to display}



%%%%%%%%%%%%%%%%%%%%%%%%%%%%%%%%%%%%%%%%%%%%%%%%%%%%%%%%%%%%%%%%%%%%%%%%
%% BODY %%%%%%%%%%%%%%%%%%%%%%%%%%%%%%%%%%%%%%%%%%%%%%%%%%%%%%%%%%%%%%%%
%%%%%%%%%%%%%%%%%%%%%%%%%%%%%%%%%%%%%%%%%%%%%%%%%%%%%%%%%%%%%%%%%%%%%%%%


\begin{document}
\maketitle

%\tableofcontents

\begin{enumerate}
  \item % Question 1
    \begin{enumerate}
      \item % Question 1a
        By definition, Brownian motion has normally distributed
        increments and $W_0=0$. Therefore we can write
        \begin{align*}
          X_t = W_t - t W_1 = (W_t-W_0) - t(W_1-W_0)
        \end{align*}
        Since the sums of normally distributed random variables are also
        normally distributed, the stochastic process $X_t$ will be
        normal as it is the sum of two normally distributed RV's
        (the increments $W_t-W_0$ and $-t(W_1-W_0)$).
      \item % Question 1b
        First the mean:
        \begin{align*}
          E[X_t] &= E[W_t - t W_1] = E[W_t] - tE[W_1] \\
          &= 0 - 0 \cdot 0 = 0
        \end{align*}
        Next the covariance of $X_t$:
        \begin{align*}
          E[X_tX_s]
          &= E\left[
            (W_t -tW_1)(W_s -sW_1)
            \right]\\
          &= E\left[
            W_t W_s - sW_tW_1 - t W_1 W_s + st W_1^2
            \right]\\
          &= E[ W_t W_s] - sE[W_tW_1] - tE[W_1 W_s] + stE[ W_1^2 ] \\
          &= \min(s,t) - s\min(t,1) - t\min(1,s) + st
        \end{align*}
        If $s,t\in[0,1]$, then this simplifies to
        \begin{align*}
          E[X_tX_s]
          &= \min(s,t) - st - ts + st
          = \min(s,t) - st
        \end{align*}

      \item % Question 1c
        To find the KL expansion on $[0,1]$ of $X_t$, we first find the
        eigenfunctions
        \begin{align*}
          \lambda_k e_k(s) &=
          \int^1_0 \left[ \min(s,t) - st\right]e_k(t) dt \\
          &=
          \int^s_0 \left[t - st\right]e_k(t) dt +
          \int^1_s \left[ s - st\right]e_k(t) dt \\
        \end{align*}
        Taking the first derivative $\frac{d}{ds}$ gives
        \begin{align*}
          \lambda_k e'_k(s) &=
          e_k(s)(s-s^2) - \int^s_0 te_k(t) dt
          -  e_k(s)(s-s^2) + \int^1_s (1-t) e_k(t) dt \\
          &= - \int^s_0 te_k(t) dt
           + \int^1_s (1-t) e_k(t) dt
        \end{align*}
        Taking the derivative again, we get
        \begin{align*}
          \lambda_k e''_k(s) &=  -s e_k(s) + (s-1)e_k(s)\\
          &= -e_k(s)
        \end{align*}
        Note that the above equations give boundary conditions of
        $e_k(0)=e_k(1)=0$. Solving the following Sturm-Liouville problem
        \begin{align*}
          e''_k(s) +\frac{1}{\lambda_k} e_k(s) = 0
        \end{align*}
        This gives eigenvalues and eigenfunctions
        \begin{align*}
          \lambda_k = \left(\pi k\right)^{-2}
          \qquad
          e_k(s) = \sqrt{2} \sin\left( \pi ks \right)
          \qquad k = 1,2,\ldots
        \end{align*}
        Therefore we can express the Brownian Bridge as
        \begin{align*}
          X_t = \sum_{k=1}^\infty \xi_k \frac{\sqrt{2}}{\pi k}
          \sin(\pi kt)
        \end{align*}
        To pin down the $\xi_k$ terms:
        \begin{align*}
          \xi_k
            &= \int^1_0 X_t e_k(t) dt \\
            &= \sqrt{2} \int^1_0 X_t \sin(\pi k t) dt 
        \end{align*}
        From the notes, we know that the $\xi_k$ terms are also Gaussian
        with mean zero and variance $\lambda_k$, as the process $X_t$ is
        Gaussian (from part a).


    \end{enumerate}

  \item % Question 2
    Define $X_t = e^{-\frac{1}{2}\int^1_0 W_t^2 dt}$. Now, let's write
    out $X_t$ using the KL expansion on $[0,1]$ for $W_t$ derived in the
    notes, subtituting in for $W^2_t$:
    \begin{align*}
      X_t = \exp\left\{
        -\frac{1}{2}\int^1_0
        \left[
        \sum^\infty_{k=1}\xi_k \frac{\sqrt{2}}{\pi\left(k-\frac{1}{2}\right)}
        \sin\left(\pi\left(k-\frac{1}{2}\right)t\right)
        \right]^2 dt
      \right\}
    \end{align*}
    Since the cross product terms drop, this leaves
    \begin{align*}
      X_t
      &= \exp\left\{
        -\frac{1}{2}\int^1_0
        \sum^\infty_{k=1}\xi^2_k \frac{2}{\pi^2\left(k-\frac{1}{2}\right)^2}
        \sin^2\left(\pi\left(k-\frac{1}{2}\right)t\right) dt
      \right\}\\
      &= \exp\left\{
        -\sum^\infty_{k=1}
        \xi^2_k \frac{1}{\pi^2\left(k-\frac{1}{2}\right)^2}\int^1_0
        \sin^2\left(\pi\left(k-\frac{1}{2}\right)t\right) dt
      \right\}
    \end{align*}
    Sparing the tedious calculations, the integral and expression
    evaluate to
    \begin{align*}
      X_t
      &= \exp\left\{
        -\sum^\infty_{k=1}\xi_k^2\cdot
        \frac{2(\pi(2k-1) + \sin(2\pi k) )}{\pi^3(2k-1)^3}
      \right\}
    \end{align*}
    Since the $\sin(2\pi k)$ term always evaluates at multiples of
    $2\pi$, this expression will always be zero, allowing further
    simplification
    \begin{align*}
      X_t
      &= \exp\left\{
        -\sum^\infty_{k=1}\xi_k^2\cdot
        \frac{2(\pi(2k-1) + 0 )}{\pi^3(2k-1)^3}
      \right\}\\
      &= \exp\left\{
        -\frac{1}{2}\sum^\infty_{k=1}
        \frac{4\xi_k^2}{\pi^2(2k-1)^2}
      \right\}
    \end{align*}
    That was as far as I got. I suspect I'm leaving off something from
    the cross terms from the squaring.

    %We want to compute the KL expansion of $W^2_t$. For that, we need
    %the covariance function:
    %\begin{align*}
      %C(t)
        %&= E[W_{t+s}^2 W_s^2] - (EW^2_{t+s})(EW^2_{s}) \\
        %&= E[((W_{t+s}-W_s) + W_s)^2 W_s^2] - (t+s)(s) \\
        %&= E[((W_{t+s}-W_s)^2 + 2W_s(W_{t+s}-W_s) + W^2_s) W_s^2] - (t+s)(s) \\
        %&= E[(W_{t+s}-W_s)^2W_s^2] + 2E[W_s(W_{t+s}-W_s)W_s^2] + E[W^2_s W_s^2] - (t+s)(s) \\
    %\end{align*}
    %By independent increments, we can break a lot of these up using
    %$E[XY]=E[X]E[Y]$ for independent $X,Y$:
    %\begin{align*}
      %C(t)
        %&= E[(W_{t+s}-W_s)^2W_s^2] + 2E[W_s(W_{t+s}-W_s)W_s^2] + E[W_s W_s^2] - (t+s)(s) \\
        %&= E[(W_{t+s}-W_s)^2]E[W_s^2] + 2E[W_s^3]E[(W_{t+s}-W_s)] + E[W^4_s] - (t+s)(s) \\
        %&= ts + 0 + 3s^2 - (t+s)(s) \\
        %&= 2s^2
    %\end{align*}
    %From there, we can find the 


  \item % Question 3
    To show that $Y_t = Q W_t$ is also a $d$-dimension Brownian motion,
    we need to show that each component is an independent Brownian
    motion.

    \paragraph{Normality}
    Let $Y^{(n)}_t$ be the $n$th entry in $Y_t$, which can be written
    \begin{align*}
      Y_t^{(n)} = Q_{n\cdot} W_t =
      \sum^d_{j=1} q_{nj} W^{(j)}_t
    \end{align*}
    As a sum of independent Gaussians RV's, $Y^{(n)}_t$ will also be
    Gaussian. Moreover, the process will be mean zero as it is a sum of
    mean-zero processes.

    \paragraph{Independent Elements} To show that any two entries
    $Y^{(n)}_t$ and $Y^{(m)}_t$ (for $n\neq m$ and
    $n,m\in\{1,\ldots,d\}$) in vector $Y_t$ are independent, it suffices
    to show that they are uncorrelated since each element is Gaussian
    (from above).

    With that in mind, we now compute the covariance matrix of $Y_t$
    (which is equivalent to $E[Y_tY^T_t]$ since mean zero):
    \begin{align*}
      E\left[Y_t Y^T_t\right]
      &= E\left[
        \left(Q W_t\right)
        \left(QW_t\right)^T
      \right] \\
      &= E\left[
        \left(Q W_t\right)
        \left(W^T_t Q^T\right)
      \right] \\
      &= QE\left[
        W_t
        W^T_t\right] Q^T
    \end{align*}
    Since $W_t$ was composed of independent Brownian motions, the
    covariance matrix of $W_t$ is
    \begin{align*}
      \text{Cov}(W_t,W^T_t)
        &= E[W_tW^T_t] - E[W_t] E[W^T_t]\\
        &= tI_d - 0
    \end{align*}
    Substituting into the expression above
    \begin{align*}
      E\left[Y_t Y^T_t\right]
      &= QE\left[
        W_t W^T_t
        \right] Q^T
      = Q
        \left(t I_d\right)
        Q^T \\
      \text{By orthogonality} \qquad
      &= t Q Q^T = t I_d
    \end{align*}
    Therefore all off-diagonal elements are 0, indicating that the
    elements of $Y_t$ are uncorrelated. Since they are Gaussian, this
    also implies independence.

    \paragraph{Independent Increments}
    For any $\{t_0,t_1,t_2\}$, we have
    \begin{align*}
      Y_{t_2} - Y_{t_1} &= Q (W_{t_2} - W_{t_1})\\
      Y_{t_1} - Y_{t_0} &= Q (W_{t_1} - W_{t_0})
    \end{align*}
    By definition, $(W_{t_2}-W_{t_1}) \perp (W_{t_1}-W_{t_0})$,
    therefore $Q(W_{t_2}-W_{t_1}) \perp Q(W_{t_1}-W_{t_0})$.

    \paragraph{Continuity}
    Since the sum of continuous processes is continuous, each element of
    $Y_t$ is continuous.

    Therefore $Y_t$ satisfies all properties of $d$-dimensional Brownian
    motion.

  \item % Question 4

    \begin{enumerate}
    \item 
      %\item % Question 4, Part a
        %First, start by defining a sequence of partitions
        %\begin{align*}
          %\sigma_n =
          %\left\{0,\frac{t}{n},\frac{2t}{n},\frac{3t}{n},\ldots,t\right\}
          %\Rightarrow |\sigma_n| = \frac{t}{n}
        %\end{align*}
        %To show convergence in mean square of the quadratic variation,
        %we need to show
        %\begin{align*}
          %\lim_{n\rightarrow \infty}
          %E|Q^{\sigma_n}_t(aW_t + bt) - a^2 t|^2 = 0
        %\end{align*}
        %where $X_n = Q^{\sigma_n}_t(aW_t + bt)$ is the random variable
        %that will be converging in mean square.

        %For more compact expressions, I will use the following notation
        %\begin{align*}
          %\Delta t_i = t_i-t_{i-1} \qquad
          %\Delta W_i = W_i-W_{i-1} \qquad
        %\end{align*}
        %I will also use the result from the notes that $(\Delta W_t)^2=
        %\Delta t$. As a result,
        %\begin{align*}
          %E|Q^{\sigma_n}_t(aW_t + bt) - a^2 t |^2
          %&=
          %E\left\lvert
            %\sum^{n}_{i=1}
              %\left[
                %\left(aW_{t_{i}} + bt_{i}\right)
                %-\left(aW_{t_{i-1}} + bt_{i-1}\right)
              %\right]^2
              %- a^2 t
          %\right\rvert^2\\
          %&=
          %E\left\lvert
            %\sum^{n}_{i=1}
              %\left[
                %a\left(W_{t_{i}} - W_{t_{i-1}}\right)
                %+ b\left(t_{i} - t_{i-1}\right)
              %\right]^2
              %- a^2 t
          %\right\rvert^2\\
          %&=
          %E\left\lvert
            %\sum^{n}_{i=1}
              %\left[
                %a\left(\Delta W_{t_{i}}\right)
                %+ b\left(\Delta t_i \right)
              %\right]^2
              %- a^2 \sum^n_{i=1} \Delta t_i
          %\right\rvert^2\\
          %&=
          %E\left\lvert
            %\sum^{n}_{i=1}
                %a^2\left(\Delta W_{t_{i}}\right)^2
                %+ 2ab\left(\Delta W_{t_{i}}\right)\left(\Delta t_i \right)
                %+ b^2\left(\Delta t_i \right)^2
              %- a^2 \Delta t_i
          %\right\rvert^2\\
        %\end{align*}

      \item % Question 4b
        Start by defining
        \begin{align*}
          M = \sup_{\sigma} \sum^n_{i=0} |f(t_{i+1})-f(t_i)|
        \end{align*}
        Next, let $\varepsilon>0$ be given. Since $f$ is continuous with
        finite total variation on $[0,t]$ (and $[0,t]$ is compact), $f$
        is uniformly continuous on $[0,t]$. Therefore there exists a
        $\delta>0$ such that
        \begin{align*}
          |t_{i+1}-t_i|<\delta
          \quad \Rightarrow \quad
          |f(t_{i+1}) - f(t_i)|< \frac{\varepsilon}{M}
        \end{align*}
        As result, as $|\sigma|\rightarrow0$, we will have
        $|\sigma|<\delta$, which implies
        \begin{align*}
          Q^\sigma_t(f) = \sum^n_{i=0} |f(t_{i+1})-f(t_i)|^2
          \leq
          \frac{\varepsilon}{M}
          \sum^n_{i=0} |f(t_{i+1})-f(t_i)|
          \leq \frac{\varepsilon}{M} \cdot M
          = \varepsilon
        \end{align*}
        Now since $Q^\sigma_t(f)\geq 0$ by definition and since we just
        showed that $Q^\sigma_t(f)\leq\varepsilon$ for arbitrary
        $\varepsilon$, we have $Q^\sigma_t(f)\rightarrow 0$ as
        $|\sigma|\rightarrow0$.

      \item % Question 4c

    \end{enumerate}
  \item % Question 5
    To compute the generator of $X_t$, we need to simplify
    \begin{align*}
      \mathcal{A}f(x)
        & \lim_{t\rightarrow0}
        \frac{E_xf(X_t) - f(x)}{t} \\
        &= \lim_{t\rightarrow0} \frac{1}{t}
        \int_A (f(y)-f(x))p(y,t|x,0) dy\\
    \end{align*}
    Where $A$ is the (assumed) compact support of $f$. We then break up
    this support into $A = A_{\geq\varepsilon} \cup A_{<\varepsilon}$, where
    \begin{align*}
      A_{\geq\varepsilon} = \{y \in A;\; |x-y| \geq \varepsilon\}
      \qquad
      A_{<\varepsilon} = \{y \in A:\; |x-y| < \varepsilon\}
    \end{align*}
    Splitting up the integral, we have
    \begin{align}
      \lim_{t\rightarrow0} \frac{1}{t}
      \int_A (f(y)-f(x))p(y,t|x,0) dy
      &= \lim_{t\rightarrow0} \frac{1}{t}
        \left[
        \int_{A_{\geq\varepsilon}} (f(y)-f(x))p(y,t|x,0) dy \right.
        \label{q5.1}
        \\
        &\qquad \qquad +\left.
        \int_{A_{<\varepsilon}} (f(y)-f(x))p(y,t|x,0) dy
      \right]\notag
    \end{align}
    First, let's handle the first part of the sum in
    Expression~\ref{q5.1}. Since $f$ is twice differentiable it is
    continuous.  Since $A$, the domain of $f$, is compact, we know that
    $f$ is bounded (and $f(y)-f(x)$ will be too). Then, by an assumption
    of the problem,
    \begin{align*}
      \lim_{t\rightarrow0} \frac{1}{t}
        \int_{A_{\geq\varepsilon}} (f(y)-f(x))p(y,t|x,0) dy
      = O(\varepsilon)
    \end{align*}
    Now let's handle the second part of the sum in
    Expression~\ref{q5.1}, using the fact that $f$ is twice
    differentiable so that
    \begin{align*}
      f(y) = f(x)+f'(x)(y-x)+\frac{1}{2}f''(x)(y-x)^2 + |x-y|^2R(x,y)
    \end{align*}
    So we can substitute in the Taylor expansion to get
    \begin{align*}
      \lim_{t\rightarrow0} &\frac{1}{t}
        \int_{A_{<\varepsilon}} (f(y)-f(x))p(y,t|x,0) dy
        \\
      &= \lim_{t\rightarrow0} \frac{1}{t}
      \int_{A_{<\varepsilon}} \left[
        f'(x)(y-x)+\frac{1}{2}f''(x)(y-x)^2 + |x-y|^2R(x,y)
      \right]p(y,t|x,0) dy \\
      &= \lim_{t\rightarrow0} \frac{1}{t}
        \left[
         f'(x)\int_{A_{<\varepsilon}}
         (y-x) p(y,t|x,0) dy
        + \frac{f''(x)}{2} \int_{A_{<\varepsilon}}
          (y-x)^2 p(y,t|x,0) dy \right. \\
        & \qquad \qquad \left. +\int_{A_{<\varepsilon}}
          |x-y|^2R(x,y) p(y,t|x,0) dy
        \right]
    \end{align*}
    By the assumptions of the problem, many of these limits simplify
    \begin{align*}
      \lim_{t\rightarrow0} &\frac{1}{t}
        \int_{A_{<\varepsilon}} (f(y)-f(x))p(y,t|x,0) dy \\
      &=
        f'(x)\left[a(x) + O(\varepsilon)\right]
        + \frac{f''(x)}{2} [b(x)+O(\varepsilon)] + O(\varepsilon^2)
    \end{align*}
    Letting $\varepsilon\to 0$ for everything and combining expressions,
    \begin{align*}
      (Af)(x) &= \lim_{\varepsilon\to0}
        O(\varepsilon) +
        \frac{\partial f}{\partial x}\left[a(x) + O(\varepsilon)\right]
        + \frac{1}{2}\frac{\partial^2 f}{\partial x^2} [b(x)+O(\varepsilon)]
        + O(\varepsilon^2) \\
        &=
        a(x)\frac{\partial f}{\partial x}
        + b(x)\frac{1}{2}\frac{\partial^2 f}{\partial x^2}
    \end{align*}

  \item % Question 6
    Again, break up $A$, the (assumed) compact support of $f$, into $A =
    A_{\geq\varepsilon} \cup A_{<\varepsilon}$, where
    \begin{align*}
      A_{\geq\varepsilon} = \{y \in A;\; |x-y| \geq \varepsilon\}
      \qquad
      A_{<\varepsilon} = \{y \in A:\; |x-y| < \varepsilon\}
    \end{align*}
    Again, we split up the integral, use the fact that $f$
    is twice differentiable, and take advantage of the assumed
    limits for the different components:
    \begin{align*}
      \lim_{t\rightarrow0} &\frac{1}{t}
      \int_A (f(y)-f(x))p(y,t|x,0) dy \\
      &= \lim_{t\rightarrow0} \frac{1}{t}
        \left[
        \int_{A_{\geq\varepsilon}} (f(y)-f(x))p(y,t|x,0) dy
        + \int_{A_{<\varepsilon}} (f(y)-f(x))p(y,t|x,0) dy
      \right]\notag\\
      &= \lim_{t\rightarrow0} \frac{1}{t}
        \left[
        \int_{A_{\geq\varepsilon}} (f(y)-f(x))p(y,t|x,0) dy \right]  \notag\\
      &\quad \lim_{t\rightarrow0} \frac{1}{t}
        \left[
         f'(x)\int_{A_{<\varepsilon}}
         (y-x) p(y,t|x,0) dy
        + \frac{f''(x)}{2} \int_{A_{<\varepsilon}}
          (y-x)^2 p(y,t|x,0) dy \right. \\
        & \qquad \qquad \left. +\int_{A_{<\varepsilon}}
          |x-y|^2R(x,y) p(y,t|x,0) dy
        \right]\\
      &=
        \left[
        \int_{A_{\geq\varepsilon}}
        (f(y)-f(x))W(y|x,t)dy  + O(\varepsilon)
        \right]
      + \left[ 0 + 0 + O(\varepsilon^2) \right]
    \end{align*}
    Letting $\varepsilon\to0$, $A_{\geq\varepsilon}\to A$ and we get
    \begin{align*}
      (Af)(x)
      &=
        \int_{A}
        (f(y)-f(x))W(y|x,t)dy
    \end{align*}
    From there, it's not clear to me that more simplification can be
    done without knowing more about the structure of $W(y|x,t)$.

\end{enumerate}


%% APPPENDIX %%

% \appendix




\end{document}


%%%%%%%%%%%%%%%%%%%%%%%%%%%%%%%%%%%%%%%%%%%%%%%%%%%%%%%%%%%%%%%%%%%%%%%%
%%%%%%%%%%%%%%%%%%%%%%%%%%%%%%%%%%%%%%%%%%%%%%%%%%%%%%%%%%%%%%%%%%%%%%%%
%%%%%%%%%%%%%%%%%%%%%%%%%%%%%%%%%%%%%%%%%%%%%%%%%%%%%%%%%%%%%%%%%%%%%%%%

%%%% SAMPLE CODE %%%%%%%%%%%%%%%%%%%%%%%%%%%%%%%%%%%%%%

    %% BIBLIOGRAPHIES %%

        \cite{LabelInSourcesFile}  %Use in text; cites
        \citep{LabelInSourcesFile} %Use in text; cites in parens

        \nocite{LabelInSourceFile} % Includes in refs w/o specific citation
        \bibliographystyle{apalike}  % Or some other style

        % To ditch the ``References'' header
        \begingroup
        \renewcommand{\section}[2]{}
        \endgroup

        \bibliography{sources} % where sources.bib has all the citation info

    %% SPACING %%

        \vspace{1in}
        \hspace{1in}


    %% INCLUDING PDF PAGE %%

        \includepdf{file.pdf}


    %% INCLUDING CODE %%

        \verbatiminput{file.ext}
            %---Includes verbatim text from the file
        \texttt{text}
            %---Renders text in courier, or code-like, font

        \matlabcode{file.m}
            %---Includes Matlab code with colors and line numbers


    %% INCLUDING FIGURES %%

        % Basic Figure with size scaling
            \begin{figure}[h!]
               \centering
               \includegraphics[scale=1]{file.pdf}
            \end{figure}

        % Basic Figure with specific height
            \begin{figure}[h!]
               \centering
               \includegraphics[height=5in, width=5in]{file.pdf}
            \end{figure}

        % Figure with cropping, where the order for trimming is  L, B, R, T
            \begin{figure}
               \centering
               \includegraphics[trim={1cm, 1cm, 1cm, 1cm}, clip]{file.pdf}
            \end{figure}


        % Side by Side figures
            \begin{figure}[h!]
                \centering
                \mbox{\subfigure{
                    \includegraphics[scale=1]{file1.pdf}
                }\quad\subfigure{
                    \includegraphics[scale=1]{file2.pdf}
                }
                }
            \end{figure}


