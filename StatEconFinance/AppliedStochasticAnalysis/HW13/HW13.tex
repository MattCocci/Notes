\documentclass[12pt]{article}
\author{Matthew Cocci}
\title{Homework 13}
\date{\today}
%% Formatting & Spacing %%%%%%%%%%%%%%%%%%%%%%%%%%%%%%%%%%%%

%\usepackage[top=1in, bottom=1in, left=1in, right=1in]{geometry} % most detailed page formatting control
\usepackage{fullpage} % Simpler than using the geometry package; std effect
\usepackage{setspace}
%\onehalfspacing
\usepackage{microtype}


%% Header %%%%%%%%%%%%%%%%%%%%%%%%%%%%%%%%%%%%%%%%%%%%%%%%%

%\usepackage{fancyhdr}
%\pagestyle{fancy}
%\lhead{}
%\rhead{}
%\chead{}
%\setlength{\headheight}{15.2pt}
    %---Make the header bigger to avoid overlap

%\renewcommand{\headrulewidth}{0.3pt}
    %---Width of the line

%\setlength{\headsep}{0.2in}
    %---Distance from line to text


%% Mathematics Related %%%%%%%%%%%%%%%%%%%%%%%%%%%%%%%%%%%

\usepackage{amsmath}
\usepackage{amsfonts}
\usepackage{mathrsfs}
\usepackage{amsthm} %allows for labeling of theorems
\usepackage{accents}
\theoremstyle{plain}
\newtheorem{thm}{Theorem}[section]
\newtheorem{lem}[thm]{Lemma}
\newtheorem{prop}[thm]{Proposition}
\newtheorem{cor}[thm]{Corollary}

\theoremstyle{definition}
\newtheorem{defn}[thm]{Definition}
\newtheorem{ex}[thm]{Example}

\theoremstyle{remark}
\newtheorem*{rem}{Remark}
\newtheorem*{note}{Note}

% Below supports left-right alignment in matrices so the negative
% signs don't look bad
\makeatletter
\renewcommand*\env@matrix[1][c]{\hskip -\arraycolsep
  \let\@ifnextchar\new@ifnextchar
  \array{*\c@MaxMatrixCols #1}}
\makeatother


%% Font Choices %%%%%%%%%%%%%%%%%%%%%%%%%%%%%%%%%%%%%%%%%

\usepackage[T1]{fontenc}
\usepackage{lmodern}
\usepackage[utf8]{inputenc}
%\usepackage{blindtext}


%% Figures %%%%%%%%%%%%%%%%%%%%%%%%%%%%%%%%%%%%%%%%%%%%%%

\usepackage{graphicx}
\usepackage{subfigure}
    %---For plotting multiple figures at once
%\graphicspath{ {Directory/} }
    %---Set a directory for where to look for figures


%% Hyperlinks %%%%%%%%%%%%%%%%%%%%%%%%%%%%%%%%%%%%%%%%%%%%
\usepackage{hyperref}
\hypersetup{
    colorlinks,
        %---This colors the links themselves, not boxes
    citecolor=black,
        %---Everything here and below changes link colors
    filecolor=black,
    linkcolor=black,
    urlcolor=black
}

%% Including Code %%%%%%%%%%%%%%%%%%%%%%%%%%%%%%%%%%%%%%%

\usepackage{verbatim}
    %---For including verbatim code from files, no colors

\usepackage{listings}
\usepackage{color}
\definecolor{mygreen}{RGB}{28,172,0}
\definecolor{mylilas}{RGB}{170,55,241}
\newcommand{\matlabcode}[1]{%
    \lstset{language=Matlab,%
        basicstyle=\footnotesize,%
        breaklines=true,%
        morekeywords={matlab2tikz},%
        keywordstyle=\color{blue},%
        morekeywords=[2]{1}, keywordstyle=[2]{\color{black}},%
        identifierstyle=\color{black},%
        stringstyle=\color{mylilas},%
        commentstyle=\color{mygreen},%
        showstringspaces=false,%
            %---Without this there will be a symbol in
            %---the places where there is a space
        numbers=left,%
        numberstyle={\tiny \color{black}},%
            %---Size of the numbers
        numbersep=9pt,%
            %---Defines how far the numbers are from the text
        emph=[1]{for,end,break,switch,case},emphstyle=[1]\color{red},%
            %---Some words to emphasise
    }%
    \lstinputlisting{#1}
}
    %---For including Matlab code from .m file with colors,
    %---line numbering, etc.

%% Bibliographies %%%%%%%%%%%%%%%%%%%%%%%%%%%%%%%%%%%%

%\usepackage{natbib}
    %---For bibliographies
%\setlength{\bibsep}{3pt} % Set how far apart bibentries are

%% Misc %%%%%%%%%%%%%%%%%%%%%%%%%%%%%%%%%%%%%%%%%%%%%%

\usepackage{enumitem}
    %---Has to do with enumeration
\usepackage{appendix}
\usepackage{pdfpages}
    %---For including whole pdf pages as a page in doc


%% User Defined %%%%%%%%%%%%%%%%%%%%%%%%%%%%%%%%%%%%%%%%%%

%\newcommand{\nameofcmd}{Text to display}



%%%%%%%%%%%%%%%%%%%%%%%%%%%%%%%%%%%%%%%%%%%%%%%%%%%%%%%%%%%%%%%%%%%%%%%%
%% BODY %%%%%%%%%%%%%%%%%%%%%%%%%%%%%%%%%%%%%%%%%%%%%%%%%%%%%%%%%%%%%%%%
%%%%%%%%%%%%%%%%%%%%%%%%%%%%%%%%%%%%%%%%%%%%%%%%%%%%%%%%%%%%%%%%%%%%%%%%


\begin{document}
\maketitle

%\tableofcontents


\begin{enumerate}
  \item % Question 1
    This question deals with the system
    \begin{align*}
      \frac{dx}{dt} &= \frac{x}{\varepsilon}y\\
      dy &= -\frac{y}{\varepsilon^2} dt + \frac{\sqrt{2\lambda}}{\varepsilon} dW_t
    \end{align*}
    \begin{enumerate}
      \item % Question 1a
        To find an SDE that $x_t$ will satisfy asymptotically as
        $\varepsilon\rightarrow 0$, start by writing the generator of
        $(x_t,y_t)$
        \begin{align*}
          \mathscr{L}^\varepsilon
          &=
          \frac{1}{\varepsilon^2}
          \left(
          -y\frac{\partial}{\partial y}
          + \lambda\frac{\partial^2}{\partial y^2}
          \right)
          +
          \frac{1}{\varepsilon}
          xy \frac{\partial}{\partial x}
        \end{align*}
        The backward equation is
        \begin{align*}
          \frac{\partial u^\varepsilon}{\partial t}
          &= \left(
          \frac{1}{\varepsilon^2} \mathscr{L}_0
          + \frac{1}{\varepsilon} \mathscr{L}_1
          \right)
          u^\varepsilon
        \end{align*}
        where $\mathscr{L}_0= -y\frac{\partial}{\partial y} +
        \lambda\frac{\partial^2}{\partial y^2}$
        and $\mathscr{L}_1= xy \frac{\partial}{\partial x}$.
        Expand $u^\varepsilon$ in an asymptotic expansion:
        \begin{align*}
          u^\varepsilon(x,y,t)
          &=
          u_0(x,y,t)
          + \varepsilon u_1(x,y,t)
          + \varepsilon^2 u_2(x,y,t) + \cdots
        \end{align*}
        Substitute into the backward equation and equate terms of each
        order:
        \begin{alignat*}{3}
          O(1/\varepsilon^2)&&\qquad -\mathscr{L}_0u_0 &= 0 \\
          O(1/\varepsilon) && \qquad -\mathscr{L}_0u_1 &= \mathscr{L}_1u_0 \\
          O(1) && \qquad -\mathscr{L}_0u_2 &=
          \mathscr{L}_1u_1
          - \frac{\partial u_0}{\partial t}
        \end{alignat*}
        For $O\left(\frac{1}{\varepsilon^2}\right)$, we have
        \begin{align*}
          -\mathscr{L}_0u_0 &= 0
          \qquad\Rightarrow\qquad
          -y \frac{\partial u_0}{\partial y}
          + \lambda\frac{\partial^2 u_0}{\partial y^2}
          = 0
          \qquad\Rightarrow\qquad
          u_0=u_0(x,t)
        \end{align*}
        i.e.\ a constant in $y$. Then at
        $O\left(\frac{1}{\varepsilon}\right)$,
        \begin{align*}
          -\mathscr{L}_0u_1 &= \mathscr{L}_1u_0
          \quad\Rightarrow\quad
          -y \frac{\partial u_1}{\partial y}
          + \lambda\frac{\partial^2 u_1}{\partial y^2}
          = xy \frac{\partial u_0}{\partial x}
          \quad\Rightarrow\quad
          u_1(x,y,t) = {xy} \frac{\partial u_0}{\partial x}
          +\Psi_1(x,t)
        \end{align*}
        From there, handle the $O(1)$ condition
        \begin{align*}
          \int \pi(y) \frac{\partial u_0}{\partial t} dy
          &=
          \int \pi(y)
          \left(
          \mathscr{L}_1u_1
          \right) dy\\
          \Leftrightarrow\qquad
          \frac{\partial u_0}{\partial t}
          &=
          \int \pi(y) \mathscr{L}_1u_1 dy\\
          &=
          \int \pi(y) xy \frac{\partial u_1}{\partial x} dy\\
          &=
          \int \pi(y) xy \frac{\partial }{\partial x}
          \left[
          {xy} \frac{\partial u_0}{\partial x}
          +\Psi_1(x,t)
          \right]dy\\
          &=
           x\frac{\partial }{\partial x}
          \left[ x\frac{\partial u_0}{\partial x} \right]
           \int y^2\pi(y) dy
          + x\frac{\partial \Psi_1}{\partial x}
          \int  y\pi(y) dy\\
          &=
           x\frac{\partial }{\partial x}
          \left[ x\frac{\partial u_0}{\partial x} \right]
          \mathbb{E}^\pi[y^2]
          + x\frac{\partial \Psi_1}{\partial x}
          \mathbb{E}^\pi[y]
        \end{align*}
        Since $y$ is an Ornstein-Uhlenbeck process with drift
        coefficient $-1$ and diffusion coefficient $\sqrt{2\lambda}$,
        giving
        \begin{align*}
          \mathbb{E}^\pi[y] &= 0 \\
          \mathbb{E}^\pi[y^2] &= \lambda
        \end{align*}
        Substituting in
        \begin{align*}
          \frac{\partial u_0}{\partial t}
          &=
           x\frac{\partial }{\partial x}
          \left[ x\frac{\partial u_0}{\partial x} \right]
          \left(\lambda\right)
          + x\frac{\partial \Psi_1}{\partial x}
          \cdot 0 \\
          &=
          \lambda x\frac{\partial u_0}{\partial x}
          + \lambda x^2\frac{\partial^2 u_0}{\partial x^2}
        \end{align*}
        And this righthand side is the generator of the
        Stratanovich SDE
        \begin{align*}
          dX_t &= X_t \sqrt{2\lambda} \circ dW_t
        \end{align*}
        Or in Ito form,
        \begin{align*}
          dX_t &= \lambda X_t \; dt + X_t \sqrt{2\lambda} \; dW_t
        \end{align*}

    \item % Question 1b
      To understand how trajectories of $X_t$ behave as $t\rightarrow
      \infty$, note that $X_t$ is Geometric Brownian motion (as is
      evident from the Ito representation). The drift is $\lambda$ and
      the diffusion is $\sqrt{2\lambda}$, giving a solution
      \begin{align*}
        S_t
        &= S_0
        \exp
        \left\{
          \left(\lambda - \frac{(\sqrt{2\lambda})^2}{2}\right)t
          + \sqrt{2\lambda} \; W_t
        \right\} \\
        &= S_0
        \exp
        \left\{
          \sqrt{2\lambda} \; W_t
        \right\}
      \end{align*}
      This suggests that $X_t$ will oscillate between arbitrarily large
      and small values as $t\rightarrow \infty$.

    \item % Question 1c
      We want to solve
      \begin{align*}
        \frac{dx}{dt}
        &= \frac{1}{\varepsilon} xy \\
        \Leftrightarrow \qquad
        \frac{dx}{x}
        &= \frac{1}{\varepsilon} y \; dt \\
        \ln x
          &= \frac{1}{\varepsilon} y t + C \\
        x &= C\exp\left\{ \frac{1}{\varepsilon} y t \right\}
      \end{align*}
      Note that since $y$ satisfies an Ornstein-Uhlenbeck process, $y$
      is Gaussian. That means that $x$ will have a log-normal
      distribution. This is consistent with the fact that $X_t$ above
      represented Geometric Brownian motion, which is log-normally distributed.

      Moreover, $y$ will have a stationary distribution
      \begin{align*}
        y &\sim N(0,\sigma^2) \\
        \sigma^2 &= \frac{\left(\frac{\sqrt{2\lambda}}{\varepsilon}\right)^2}{2\left(%
        \frac{1}{\varepsilon^2}\right)} = \lambda
      \end{align*}
      Therefore, $\log(x)$ will be normally distributed with variance
      that grows with $t$ at rate $\lambda$. Therefore, $x$ will
      fluctuate between very large and small values.
  \end{enumerate}

  \clearpage
  \item % Question 2
    This question concerns the Langevin equation
    \begin{align*}
      m \ddot{x} &= -\nabla U(x) - \gamma \dot{x} + \sqrt{2}\sigma \eta(t)
    \end{align*}
    \begin{enumerate}
      \item % Question 2a
        Start by rewriting this as first-order system of SDEs in
        $(x,y)$, letting $m=\varepsilon^2$ and
        $y=\varepsilon\dot{x}$
        \begin{align*}
          \varepsilon \dot{y} &= -\nabla U(x) - \frac{\gamma}{\varepsilon} y + \sqrt{2}\sigma \eta(t)\\
          \Leftrightarrow \qquad
          \dot{y} &= -\frac{1}{\varepsilon}\nabla U(x) - \frac{\gamma}{\varepsilon^2} y
          + \frac{\sqrt{2}\sigma}{\varepsilon} \eta(t)\\
        \end{align*}
        Multiplying through by $dt$, this gives the system
        \begin{align*}
          dy &= -\left(\frac{1}{\varepsilon}\nabla U(x) + \frac{\gamma}{\varepsilon^2} y\right) dt
          + \frac{\sqrt{2}\sigma}{\varepsilon} dW_t\\
          dx &= \frac{1}{\varepsilon} y \; dt
        \end{align*}

      \item % Question 2b
        The backward equation can be written
        \begin{align*}
          \frac{\partial u}{\partial t}
          =
          \mathscr{L}u
          &=
          -\left(\frac{1}{\varepsilon}\nabla U(x) + \frac{\gamma}{\varepsilon^2} y\right)
          \frac{\partial u}{\partial y}
          +\frac{1}{\varepsilon} y
          \frac{\partial u}{\partial x}
          +
          \frac{\sigma^2}{\varepsilon^2}
          \frac{\partial^2 u}{\partial y^2} \\
          &=
          \frac{1}{\varepsilon^2}
            \left(
              -\gamma y \frac{\partial u}{\partial y}
              +\sigma^2 \frac{\partial^2 u}{\partial y^2}
            \right)
          +
          \frac{1}{\varepsilon}
          \left(
            -\nabla U(x) \frac{\partial u}{\partial y}
            + y \frac{\partial u}{\partial x}
          \right)\\
          &=
          \frac{1}{\varepsilon^2} \mathscr{L}_0u
          +
          \frac{1}{\varepsilon} \mathscr{L}_1u
        \end{align*}
        Therefore, we can identify
        \begin{align*}
          \mathscr{L}_0 &=
            -\gamma y \frac{\partial }{\partial y}
            +\sigma^2 \frac{\partial^2 }{\partial y^2} \\
          \mathscr{L}_1 &=
            -\nabla U(x) \frac{\partial }{\partial y}
            + y \frac{\partial }{\partial x}
        \end{align*}

      \item % Question 2c
        For $O(\varepsilon^{-2})$ we have
        \begin{align*}
          0 &= -\mathscr{L}_0u_0 \\
          0 &= -\gamma y \frac{\partial u_0}{\partial y}
            +\sigma^2 \frac{\partial^2 u_0}{\partial y^2}
        \end{align*}
        This implies that $u_0$ is a constant in $y$
        \begin{align*}
          u_0 &= u_0(x,t)
        \end{align*}

      \item % Question 2d
        At $O(\varepsilon^{-1})$, we have
        \begin{align*}
          -\mathscr{L}_0u_1 &= \mathscr{L}_1u_0 \\
          \gamma y \frac{\partial u_1}{\partial y}
            -\sigma^2 \frac{\partial^2 u_1}{\partial y^2}
            &= -\nabla U(x) \frac{\partial u_0}{\partial y}
            + y \frac{\partial u_0}{\partial x}
        \end{align*}
        Since $u_0$ is a function of $x$ and $t$ only, we can drop one
        term to get to
        \begin{align*}
          \gamma y \frac{\partial u_1}{\partial y}
            -\sigma^2 \frac{\partial^2 u_1}{\partial y^2}
            &= y \frac{\partial u_0}{\partial x}
        \end{align*}
        %Next, separation of variables, assuming that $u_1 = f(x) g(y)$:
        %\begin{align*}
          %\gamma y f(x) \frac{\partial g'}{\partial y}
            %-\sigma^2 f(x) \frac{\partial^2 g}{\partial y^2}
            %&= y \frac{\partial u_0}{\partial x} \\
          %\gamma \frac{\partial g'}{\partial y}
            %-\sigma^2 \frac{1}{y} \frac{\partial^2 g}{\partial y^2}
            %&= \frac{1}{g(x)}\frac{\partial u_0}{\partial x} \\
        %\end{align*}
        We can use separation of variables to get
        \begin{align*}
          u_1(x,y,t) &= \frac{y}{\gamma} \frac{\partial u_0}{\partial x} + \Psi_1(x,t)
        \end{align*}

      \item % Question 2e
        First, solve for $\pi(y)$ by noting that $\pi(y)$ satisfies
        \begin{align*}
          \mathscr{L}^*_0 \pi &= 0
        \end{align*}
        But $\mathscr{L}_0$ is simply the generator of the following
        process
        \begin{align*}
          dy &= -\gamma y dt + \sqrt{2} \sigma dW_t
        \end{align*}
        This is an Ornstein-Uhlenbeck process which will have a
        Gaussian stationary distribution
        \begin{align}
          \pi(y) &= N(0, \nu^2) \notag \\
          \nu &= \frac{\sigma^2}{\gamma}
          \label{q2e.1}
        \end{align}
        Next, to solve for the evolution equation for $u_0$, we have at
        $O(1)$
        \begin{align*}
          -\mathscr{L}_0u_2 &= \mathscr{L}_1u_1 -\frac{\partial u_0}{\partial t}
        \end{align*}
        Since $\mathscr{L}^*_0\pi=0$, we solve
        \begin{align*}
          \int \pi(y) \frac{\partial u_0}{\partial t} \; dy
          &= \int \pi(y) \mathscr{L}_1u_1 \; dy \\
          \frac{\partial u_0}{\partial t} \int \pi(y) \; dy
          &= \int \pi(y)
            \mathscr{L}_1
            \left[
            \frac{y}{\gamma} \frac{\partial u_0}{\partial x} + \Psi_1(x,t)
            \right] \; dy \\
          \frac{\partial u_0}{\partial t}
          &= \int \pi(y)
            \mathscr{L}_1
            \left[\frac{y}{\gamma} \frac{\partial u_0}{\partial x}\right] \; dy
            + \int \pi(y) \mathscr{L}_1\Psi_1(x,t) \; dy
        \end{align*}
        Taking the first integral on the righthand side
        \begin{align*}
            \int \pi(y)
            \mathscr{L}_1
            \left[\frac{y}{\gamma} \frac{\partial u_0}{\partial x}\right] \; dy
            &=
            \int \pi(y)
            \left(
            -\nabla U(x) \frac{\partial }{\partial y}
            + y \frac{\partial }{\partial x}
            \right)
            \left[\frac{y}{\gamma} \frac{\partial u_0}{\partial x}\right] \; dy\\
            &=
              -\int \pi(y)
                \nabla U(x) \frac{\partial }{\partial y}
                \left[\frac{y}{\gamma} \frac{\partial u_0}{\partial x}\right] \; dy
              +
              \int \pi(y)
              y \frac{\partial }{\partial x}
              \left[\frac{y}{\gamma} \frac{\partial u_0}{\partial x}\right] \; dy\\
            &=
              - \frac{\nabla U(x)}{\gamma} \frac{\partial u_0}{\partial x}
              \int \pi(y) \; dy
              +
              \frac{1}{\gamma} \frac{\partial^2 u_0}{\partial x^2}
              \int y^2 \pi(y) \; dy\\
            &=
              - \frac{\nabla U(x)}{\gamma} \frac{\partial u_0}{\partial x}
              +
              \frac{1}{\gamma} \frac{\partial^2 u_0}{\partial x^2}
              \mathbb{E}^\pi y^2
        \end{align*}
        Next, the second integral on the righthand side
        \begin{align*}
          \int \pi(y) \mathscr{L}_1\Psi_1(x,t) \; dy
          &=
          \int \pi(y)
            \left(
            -\nabla U(x) \frac{\partial }{\partial y}
            + y \frac{\partial }{\partial x}
            \right) \Psi_1(x,t)
            \; dy \\
          &=
          -\int \pi(y)
            \nabla U(x) \frac{\partial \Psi_1}{\partial y}
            \; dy
          +\int \pi(y)
            y \frac{\partial \Psi_1}{\partial x}
            \; dy \\
          &=
          0 +\frac{\partial \Psi_1}{\partial x}\int \pi(y) y \; dy \\
          &=
          \frac{\partial \Psi_1}{\partial x}\mathbb{E}^\pi y
        \end{align*}
        Putting everything together and using the distribution
        (\ref{q2e.1}) to substitute in for the expectations, we have
        \begin{align}
          \frac{\partial u_0}{\partial t}
          &=
          - \frac{\nabla U(x)}{\gamma} \frac{\partial u_0}{\partial x}
          + \frac{1}{\gamma} \frac{\partial^2 u_0}{\partial x^2}
          \mathbb{E}^\pi y^2
          +
          \frac{\partial \Psi_1}{\partial x}\mathbb{E}^\pi y \notag\\
          &=
          - \frac{\nabla U(x)}{\gamma} \frac{\partial u_0}{\partial x}
          + \frac{1}{\gamma} \frac{\partial^2 u_0}{\partial x^2}
          \cdot \frac{\sigma^2}{\gamma}
          + \frac{\partial \Psi_1}{\partial x} \cdot 0 \notag\\
          \frac{\partial u_0}{\partial t}
          &=
          - \frac{\nabla U(x)}{\gamma} \frac{\partial u_0}{\partial x}
          + \frac{\sigma^2}{\gamma^2} \frac{\partial^2 u_0}{\partial x^2}
          \label{q2e.2}
        \end{align}

      \item % Question 2f
        Equation~\ref{q2e.2} above is the backward equation for the following SDE
        \begin{align*}
          dX_t
          &=
          - \frac{\nabla U(X_t)}{\gamma} dt
          + \frac{\sqrt{2}\sigma}{\gamma} dW_t \\
          &=
          - \frac{\nabla U(X_t)}{\gamma} dt
          + \frac{\sqrt{2}\sqrt{\gamma/\beta}}{\gamma} dW_t \\
          \Leftrightarrow \qquad
          dX_t &=
          - \frac{\nabla U(X_t)}{\gamma} dt
          + D dW_t
          \qquad D = \sqrt{2\beta^{-1}\gamma^{-1}}
        \end{align*}

    \end{enumerate}




\end{enumerate}

\end{document}


%%%%%%%%%%%%%%%%%%%%%%%%%%%%%%%%%%%%%%%%%%%%%%%%%%%%%%%%%%%%%%%%%%%%%%%%
%%%%%%%%%%%%%%%%%%%%%%%%%%%%%%%%%%%%%%%%%%%%%%%%%%%%%%%%%%%%%%%%%%%%%%%%
%%%%%%%%%%%%%%%%%%%%%%%%%%%%%%%%%%%%%%%%%%%%%%%%%%%%%%%%%%%%%%%%%%%%%%%%

%%%% SAMPLE CODE %%%%%%%%%%%%%%%%%%%%%%%%%%%%%%%%%%%%%%

    %% BIBLIOGRAPHIES %%

        \cite{LabelInSourcesFile}  %Use in text; cites
        \citep{LabelInSourcesFile} %Use in text; cites in parens

        \nocite{LabelInSourceFile} % Includes in refs w/o specific citation
        \bibliographystyle{apalike}  % Or some other style

        % To ditch the ``References'' header
        \begingroup
        \renewcommand{\section}[2]{}
        \endgroup

        \bibliography{sources} % where sources.bib has all the citation info

    %% SPACING %%

        \vspace{1in}
        \hspace{1in}


    %% INCLUDING PDF PAGE %%

        \includepdf{file.pdf}


    %% INCLUDING CODE %%

        \verbatiminput{file.ext}
            %---Includes verbatim text from the file
        \texttt{text}
            %---Renders text in courier, or code-like, font

        \matlabcode{file.m}
            %---Includes Matlab code with colors and line numbers


    %% INCLUDING FIGURES %%

        % Basic Figure with size scaling
            \begin{figure}[h!]
               \centering
               \includegraphics[scale=1]{file.pdf}
            \end{figure}

        % Basic Figure with specific height
            \begin{figure}[h!]
               \centering
               \includegraphics[height=5in, width=5in]{file.pdf}
            \end{figure}

        % Figure with cropping, where the order for trimming is  L, B, R, T
            \begin{figure}
               \centering
               \includegraphics[trim={1cm, 1cm, 1cm, 1cm}, clip]{file.pdf}
            \end{figure}


        % Side by Side figures
            \begin{figure}[h!]
                \centering
                \mbox{\subfigure{
                    \includegraphics[scale=1]{file1.pdf}
                }\quad\subfigure{
                    \includegraphics[scale=1]{file2.pdf}
                }
                }
            \end{figure}


