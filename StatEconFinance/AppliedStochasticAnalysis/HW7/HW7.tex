\documentclass[12pt]{article}
\author{Matthew D. Cocci}
\title{Homework 7}
\date{\today}
%% Formatting & Spacing %%%%%%%%%%%%%%%%%%%%%%%%%%%%%%%%%%%%

%\usepackage[top=1in, bottom=1in, left=1in, right=1in]{geometry} % most detailed page formatting control
\usepackage{fullpage} % Simpler than using the geometry package; std effect
\usepackage{setspace}
%\onehalfspacing
\usepackage{microtype}


%% Header %%%%%%%%%%%%%%%%%%%%%%%%%%%%%%%%%%%%%%%%%%%%%%%%%

%\usepackage{fancyhdr}
%\pagestyle{fancy}
%\lhead{}
%\rhead{}
%\chead{}
%\setlength{\headheight}{15.2pt}
    %---Make the header bigger to avoid overlap

%\renewcommand{\headrulewidth}{0.3pt}
    %---Width of the line

%\setlength{\headsep}{0.2in}
    %---Distance from line to text


%% Mathematics Related %%%%%%%%%%%%%%%%%%%%%%%%%%%%%%%%%%%

\usepackage{amsmath}
\usepackage{amsfonts}
\usepackage{mathrsfs}
\usepackage{amsthm} %allows for labeling of theorems
\theoremstyle{plain}
\newtheorem{thm}{Theorem}[section]
\newtheorem{lem}[thm]{Lemma}
\newtheorem{prop}[thm]{Proposition}
\newtheorem{cor}[thm]{Corollary}

\theoremstyle{definition}
\newtheorem{defn}[thm]{Definition}
\newtheorem{ex}[thm]{Example}

\theoremstyle{remark}
\newtheorem*{rem}{Remark}
\newtheorem*{note}{Note}

% Below supports left-right alignment in matrices so the negative
% signs don't look bad
\makeatletter
\renewcommand*\env@matrix[1][c]{\hskip -\arraycolsep
  \let\@ifnextchar\new@ifnextchar
  \array{*\c@MaxMatrixCols #1}}
\makeatother


%% Font Choices %%%%%%%%%%%%%%%%%%%%%%%%%%%%%%%%%%%%%%%%%

\usepackage[T1]{fontenc}
\usepackage{lmodern}
\usepackage[utf8]{inputenc}
%\usepackage{blindtext}


%% Figures %%%%%%%%%%%%%%%%%%%%%%%%%%%%%%%%%%%%%%%%%%%%%%

\usepackage{graphicx}
\usepackage{subfigure}
    %---For plotting multiple figures at once
%\graphicspath{ {Directory/} }
    %---Set a directory for where to look for figures


%% Hyperlinks %%%%%%%%%%%%%%%%%%%%%%%%%%%%%%%%%%%%%%%%%%%%
\usepackage{hyperref}
\hypersetup{
    colorlinks,
        %---This colors the links themselves, not boxes
    citecolor=black,
        %---Everything here and below changes link colors
    filecolor=black,
    linkcolor=black,
    urlcolor=black
}

%% Including Code %%%%%%%%%%%%%%%%%%%%%%%%%%%%%%%%%%%%%%%

\usepackage{verbatim}
    %---For including verbatim code from files, no colors

\usepackage{listings}
\usepackage{color}
\definecolor{mygreen}{RGB}{28,172,0}
\definecolor{mylilas}{RGB}{170,55,241}
\newcommand{\matlabcode}[1]{%
    \lstset{language=Matlab,%
        basicstyle=\footnotesize,%
        breaklines=true,%
        morekeywords={matlab2tikz},%
        keywordstyle=\color{blue},%
        morekeywords=[2]{1}, keywordstyle=[2]{\color{black}},%
        identifierstyle=\color{black},%
        stringstyle=\color{mylilas},%
        commentstyle=\color{mygreen},%
        showstringspaces=false,%
            %---Without this there will be a symbol in
            %---the places where there is a space
        numbers=left,%
        numberstyle={\tiny \color{black}},%
            %---Size of the numbers
        numbersep=9pt,%
            %---Defines how far the numbers are from the text
        emph=[1]{for,end,break,switch,case},emphstyle=[1]\color{red},%
            %---Some words to emphasise
    }%
    \lstinputlisting{#1}
}
    %---For including Matlab code from .m file with colors,
    %---line numbering, etc.

%% Bibliographies %%%%%%%%%%%%%%%%%%%%%%%%%%%%%%%%%%%%

%\usepackage{natbib}
    %---For bibliographies
%\setlength{\bibsep}{3pt} % Set how far apart bibentries are

%% Misc %%%%%%%%%%%%%%%%%%%%%%%%%%%%%%%%%%%%%%%%%%%%%%

\usepackage{enumitem}
    %---Has to do with enumeration
\usepackage{appendix}
\usepackage{pdfpages}
    %---For including whole pdf pages as a page in doc


%% User Defined %%%%%%%%%%%%%%%%%%%%%%%%%%%%%%%%%%%%%%%%%%

%\newcommand{\nameofcmd}{Text to display}



%%%%%%%%%%%%%%%%%%%%%%%%%%%%%%%%%%%%%%%%%%%%%%%%%%%%%%%%%%%%%%%%%%%%%%%%
%% BODY %%%%%%%%%%%%%%%%%%%%%%%%%%%%%%%%%%%%%%%%%%%%%%%%%%%%%%%%%%%%%%%%
%%%%%%%%%%%%%%%%%%%%%%%%%%%%%%%%%%%%%%%%%%%%%%%%%%%%%%%%%%%%%%%%%%%%%%%%


\begin{document}
\maketitle

%\tableofcontents


\begin{enumerate}
\item % Question 1
  To simplify notation below, given a partition $P= \{t_j\}_0^n$, let
  $||\Delta t_j|| = \sup_j \Delta t_j$.
  \begin{enumerate}
    \item % Question 1a
      For some partition $\{t_j\}_0^{n-1}$, we want to evaluate the mean
      square limit of the following sum:
      \begin{align*}
        \sum_j \frac{1}{2} (W_j+W_{j+1}) \Delta W_j
        &= \frac{1}{2} \sum_j (W_j+W_{j+1})(W_{j+1}-W_{j}) \\
        &= \frac{1}{2} \sum_j W^2_{j+1}-W^2_{j} + W_j W_{j+1} - W_j W_{j+1}\\
        &= \frac{1}{2} \sum_j W^2_{j+1}-W^2_{j}
      \end{align*}
      which is a telescoping sum that simplifies to
      \begin{align*}
        \sum_j \frac{1}{2} (W_j+W_{j+1}) \Delta W_j
        &= \frac{1}{2} \left( W^2_n-W^2_0\right)
        = \frac{1}{2} W^2_t
      \end{align*}
      Since this is independent of the partition, we don't need to worry
      about taking the limit.

      Since this question evaluated a sum that approximates the
      Stratonovich Integral $\int^t_0 W_s \circ dW_s$, it's not
      surprising that we get the answer we would expect under the chain
      rule.

    \item % Question 1b
      Let $P_1$ be a partition $\{t_{1,j}\}_0^{n-1}$ of some interval. We
      want to compute
      \begin{align*}
        \sum_j W_{j+\frac{1}{2}} \Delta W_j
      \end{align*}
      However, note that we can define another partition of the interval
      \begin{align*}
        P_2 = \{t_{2,j}\}_0^{2n-1}
        = t_{1,0}, \left( t_{1,0} + \frac{1}{2}\right), t_{1,1},
        \left(t_{1,1}+ \frac{1}{2}\right), t_{1,2},
        \left(t_{1,2}+ \frac{1}{2}\right), \ldots, t_{1,n}
      \end{align*}
      This partition is equally valid and ``fills in'' the points of
      distance $1/2$ from lefthand boundary of the intervals defining
      $P_1$. As a result, $P_2$ is a refinement of $P_1$, i.e.
      $P_1\subset P_2$.

      With this new partition in mind, we can rewrite

    \item % Question 1c
      Finally, we want compute the following:
      \begin{align*}
        \lim_{||\Delta t_j||\rightarrow 0}
        E\left\lvert \sum_j W_j^2 \Delta W_j\right\rvert^2
      \end{align*}
      First, let's expand out the summation
      \begin{align*}
        \sum_j W_j^2 \Delta W_j
        &= \sum_j (W_{j+1}^2 -W_{j+1}^2 + W_j^2) ( W_{j+1} - W_j)\\
        &= \sum_j W_{j+1}^3 - W_{j}W_{j+1}^2 - W_{j+1}^3
          + W_{j}W_{j+1}^2 + W_{j}^2W_{j+1} - W_{j}^3\\
        &= \sum_j W_{j+1}^3 - W_{j}^3 - W_{j+1}^3 + W_{j}^2W_{j+1}\\
        &= \sum_j \Delta(W^3_{j}) - \sum_j W_{j+1}( W_{j+1}^2 - W_{j}^2)\\
        &= \sum_j \Delta(W^3_{j}) - \sum_j W_{j+1}(\Delta W_{j}^2)
      \end{align*}
      Now the first part is a telescoping sum, so that the above will
      simplify to the following, regardless of the partition:
      \begin{align}
        \sum_j W_j^2 \Delta W_j
        &= W^3_t - W^3_0 - \sum_j W_{j+1}(\Delta W_{j}^2)\notag\\
        &= W^3_t - \sum_j W_{j+1}(\Delta W_{j}^2)
        \label{q1c.1}
      \end{align}
      Next, we'll use the following identity from the lecture notes:
      \begin{align*}
        2 W_j (\Delta W_j) &= \Delta W_j^2 - (\Delta W_j)^2 \\
        \Rightarrow \qquad
        \Delta W_j^2 &= (\Delta W_j)^2  +  2 W_j (\Delta W_j)
      \end{align*}
      Substituting this in for $\Delta W_j^2$ in Expression~\ref{q1c.1},
      and making some additional simplifications, we get
      \begin{align*}
        \sum_j W_j^2 \Delta W_j
        &= W^3_t - \sum_j W_{j+1}
          \left[(\Delta W_j)^2  +  2 W_j (\Delta W_j)\right]\\
        &= W^3_t - \sum_j (W_{j+1}-W_j + W_j)
          \left[(\Delta W_j)^2  +  2 W_j (\Delta W_j)\right]\\
        &= W^3_t - \sum_j (\Delta W_{j}+ W_j)
          \left[(\Delta W_j)^2  +  2 W_j (\Delta W_j)\right]\\
        \text{Distributing} \qquad
        &= W^3_t
          - \sum_j
            (\Delta W_{j})^3
            + 2 W_j (\Delta W_j)^2
            + W_j(\Delta W_j)^2
            + 2 W_j^2 (\Delta W_j)\\
        \text{Breaking up} \qquad
        &= W^3_t
          - \sum_j 2 W_j^2 (\Delta W_j)
          - \sum_j
            (\Delta W_{j})^3
            + 3 W_j (\Delta W_j)^2
      \end{align*}
      Altogether, this gives the following equality
      \begin{align*}
        \sum_j W_j^2 \Delta W_j
        &= W^3_t
          - \sum_j 2 W_j^2 (\Delta W_j)
          - \sum_j
            (\Delta W_{j})^3
            + 3 W_j (\Delta W_j)^2 \\
        \Rightarrow\quad
        3\sum_j W_j^2 \Delta W_j
        &= W^3_t
          - \sum_j
            (\Delta W_{j})^3
            + 3 W_j (\Delta W_j)^2 \\
        \Rightarrow\quad
        \sum_j W_j^2 \Delta W_j
        &= \frac{1}{3}W^3_t
          - \frac{1}{3}\sum_j
            (\Delta W_{j})^3
            + 3 W_j (\Delta W_j)^2
      \end{align*}
      So far, everything was done without taking limits, and the work
      above is independent of the partition.

      Lastly, the claim is that as $||\Delta t_j||\rightarrow 0$, we get
      mean square convergence as follows:
      \begin{align*}
        \frac{1}{3}\sum_j (\Delta W_j)^3 \xrightarrow{m.s.} 0
        \qquad\text{and}\qquad
        \sum_j W_j (\Delta W_j)^2 \xrightarrow{m.s.} \int^t_0 W_s ds
      \end{align*}
      To see the first, note that $\Delta W_j$ is a normally distributed
      random variable; therefore, in computing the mean square
      difference from zero,
      \begin{align*}
        E\left\lvert \frac{1}{3} \sum_j (\Delta W_j)^3 - 0\right\rvert^2
        &=\frac{1}{9} E\left( \sum_j (\Delta W_j)^3\right)^2\notag\\
        &=\frac{1}{9} E\left( \sum_j (\Delta W_j)^6
          + 2\sum_{i<j}(\Delta W_j)^3 (\Delta W_i)^3\right)\\
        \text{By linearity of $E[\;\cdot\;]$} \qquad
        &=\frac{1}{9} \sum_j E(\Delta W_j)^6
        + \frac{2}{9}\sum_{i<j} E\left[(\Delta W_j)^3 (\Delta W_i)^3\right]\\
        \text{By Normality} \qquad
        &= \frac{1}{9} \sum_j 15( \Delta t_j)^3
          + \frac{2}{9}\sum_{i<j} E\left[(\Delta W_j)^3 (\Delta W_i)^3\right]\\
        &\leq \frac{15}{9} ||\Delta t_j||^2 \sum_j (\Delta t_j)
          + \frac{2}{9}\sum_{i<j} E\left[(\Delta W_j)^3 (\Delta W_i)^3\right]\\
        &\leq \frac{15}{9} ||\Delta t_j||^2 \cdot t
          + \frac{2}{9}\sum_{i<j} E\left[(\Delta W_j)^3 (\Delta W_i)^3\right]\\
        \text{By independent increments} \qquad
        &\leq \frac{15}{9} ||\Delta t_j||^2 \cdot t
        + \frac{2}{9}\sum_{i<j} E\left[(\Delta W_j)^3] E[(\Delta W_i)^3\right]\\
        \text{By normality} \qquad
        &\leq \frac{15}{9} ||\Delta t_j||^2 \cdot t
        + \frac{2}{9}\sum_{i<j} 0 \cdot 0
      \end{align*}
      Hence as $||\Delta_j||\rightarrow 0$, the above goes to 0,
      implying that $\frac{1}{3}\sum_j (\Delta W_j)^3)
      \xrightarrow{m.s.} 0$ as we wanted.

      Lastly, we want to show $\sum_j W_j (\Delta W_j)^2
      \xrightarrow{m.s.} \int^t_0 W_s ds$.
      \begin{align*}
        E\left( \sum_j W_j (\Delta W_j)^2 - \int^t_0 W_s ds\right)^2
      \end{align*}

  \end{enumerate}

  \item % Question 2
    \begin{enumerate}
      \item % Question 2a
        To evaluate $\int^t_0 W_t^2 dW_t$ with Ito's formula, we start
        by defining $X_t = W_t$ so that $dX_t = dW_t$. We also start by
        defining the function
        \begin{align*}
          f(t,x) = \frac{1}{3} x^3 \quad \Rightarrow \quad
          \frac{\partial f}{\partial t} = 0 \quad
          \frac{\partial f}{\partial x} = x^2 \quad
          \frac{\partial^2 f}{\partial x^2} = 2x
        \end{align*}
        This was chosen because it is the function we would expect if
        the usual chain rule held. Now, substituting the above into
        Ito's formula (where $Y_t=f(X_t) = \frac{1}{3}X^3_t$) yields
        \begin{align*}
          d\left(\frac{1}{3} X_t^3\right)
          &= 0 \cdot dt + X^2_t dX_t
          + \frac{1}{2} \left(2X_t\right) \left(dX_t\right)^2
        \end{align*}
        Substituting in $X_t=W_t$ and $dX_t=dW_t$ gives
        \begin{align*}
          d\left(\frac{1}{3} W_t^3\right)
          &= W^2_t dW_t + \frac{1}{2} \left(2W_t\right)
          \left(dW_t\right)^2\\
          \Rightarrow \qquad
          d\left(\frac{1}{3} W_t^3\right)
          &= W^2_t dW_t + W_t dt
        \end{align*}
        where $dW_t^2 = dt$ was used above.

        Now, integrating from $0$ to $t$ and taking this out of
        differential form, we get
        \begin{align*}
          \frac{1}{3} W_t^3 -
          \frac{1}{3} W_0^3
          &= \int^t_0 W^2_s dW_s + \int^t_0 W_s ds
        \end{align*}
        Rearranging, and using $W_0=0$, we get
        \begin{align*}
          \int^t_0 W^2_s dW_s =
          \frac{1}{3} W_t^3 - \int^t_0 W_s ds
        \end{align*}

      \item % Question 2b
        To calculate the following expectation, we will use Ito's
        Isometry formula:
        \begin{align*}
          E\left(\int^t_0 W_s^2 dW_s\right)^2
          &= E\int^t_0 \left(W_s^2\right)^2 ds\\
          &= E\int^t_0 W_s^4 ds\\
          &= \int^t_0 EW_s^4 ds
        \end{align*}
        We can exchange the integral and the expectation because $W^4_t$
        is a non-negative process, allowing us to apply Fubini's
        Theorem.

        Next, we use the fact that $W_s$ is a normally distributed
        random variable with zero mean and variance $s$ for all
        $s\in[0,t]$. As a result, we can substitute in for the
        expectation $E[W_s^4]$ based on the moments of normally
        distributed RVs:
        \begin{align*}
          E\left(\int^t_0 W_s^2 dW_s\right)^2
          &= \int^t_0 EW_s^4 ds\\
          &= \int^t_0 3s^2 ds = s^3|^t_0\\
          &= t^3
        \end{align*}

    \end{enumerate}

  \item % Question 3
    Throughout this question, I will use the fact that if $Z_t = W_t$,
    then $Z_t$ satisfies the SDE
    \begin{equation}
      dZ_t = dW_t
      \label{q3.0}
    \end{equation}
    \begin{enumerate}
      \item % Question 3a
        Suppose that $f(t,z) = z/(1+t)$. Then we want to know the SDE of
        $X_t=Z_t/(1+t)$.  To do so, we can use Ito's formula.

        But before applying Ito's formula, let's first find the
        necessary derivatives of $f(t,y)$:
        \begin{align*}
          \frac{\partial f}{\partial t} &= -\frac{z}{(1+t)^2} \\
          \frac{\partial f}{\partial z} &= \frac{1}{(1+t)} \\
          \frac{\partial^2 f}{\partial z^2} &= 0
        \end{align*}
        Substituting into Ito's Formula:
        \begin{align*}
          dX_t = - \frac{Z_t}{(1+t)^2} dt
          + \frac{1}{(1+t)} dZ_t + \frac{1}{2}\cdot 0
        \end{align*}
        Substituting in for $dZ_t$ using Expression~\ref{q3.0} and the
        fact that $X_t = Z_t/(1+t)$, the above simplifies to
        \begin{align*}
          dX_t &= - \frac{X_t}{(1+t)} dt + \frac{1}{(1+t)} dW_t \\
          \Leftrightarrow \qquad
          dX_t&= \frac{1}{(1+t)} (-X_t dt + dW_t)
        \end{align*}

      \item % Question 3b
        Next, we want to find the SDE satisfied by $X_t = b \sin W_t = b \sin
        Z_t$. (In this problem $b=1$, but I leave it general for
        Question 3c.) So in this case, $f(t,z) = b\sin(z)$, giving the
        following derivatives:
        \begin{align*}
          \frac{\partial f}{\partial t} &= 0 \\
          \frac{\partial f}{\partial z} &= b\cos(z) \\
          \frac{\partial^2 f}{\partial z^2} &= -b\sin(z)
        \end{align*}
        Using Ito's Formula
        \begin{align*}
          dX_t = 0 \cdot dt + b \cos(Z_t) dZ_t
          + \frac{1}{2} \left(-b \sin(Z_t)\right) (dZ_t)^2
        \end{align*}
        Substiting in $Z_t = W_t$


      \item % Question 3c
        For $X_t = a \cos W_t$ and $Y_t = b \sin W_t$, we can determine
        the SDE they satisfy by applying Ito's lemma for the following
        functions of $Z_t$
        \begin{alignat*}{3}
          f_x(z,t) &= a\cos z \qquad & f_y(z,t) &= b \sin z \\
          \frac{\partial f_x}{\partial t}
            &= 0
            \qquad &
          \frac{\partial f_y}{\partial t}
            &= 0 \\
          \frac{\partial f_x}{\partial z}
            &= -a\sin z
            \qquad &
          \frac{\partial f_y}{\partial z}
            &= b\cos z \\
          \frac{\partial^2 f_x}{\partial z^2}
            &= -a\cos z
            \qquad &
          \frac{\partial^2 f_y}{\partial z^2}
            &= -b\sin z
        \end{alignat*}
        Applying Ito's formula, we get
        \begin{align*}
          dX_t = 0 - a \sin Z_t dZ_t - \frac{1}{2} a\cos Z_t (dZ_t)^2\\
          dY_t = 0 + b \cos Z_t dZ_t - \frac{1}{2} b\sin Z_t (dZ_t)^2
        \end{align*}
        Subsitituting in $Z_t = W_t$ and $dZ_t = dW_t$, we get
        \begin{align*}
          dX_t = - a \sin W_t dW_t - \frac{1}{2} a\cos W_t (dW_t)^2\\
          dY_t =   b \cos W_t dW_t - \frac{1}{2} b\sin W_t (dW_t)^2
        \end{align*}
        Using $dW_t^2 = dt$,
        \begin{align*}
          dX_t = - \frac{1}{2} a\cos W_t dt - a \sin W_t dW_t  \\
          dY_t = - \frac{1}{2} b\sin W_t dt + b \cos W_t dW_t
        \end{align*}
        Next, identify $X_t$ and $Y_t$ as the coefficients on $dW_t$ and
        $dt$, then substitute in:
        \begin{align*}
          dX_t = - \frac{1}{2} X_t dt - \frac{a}{b} Y_t dW_t  \\
          dY_t = - \frac{1}{2} Y_t dt + \frac{b}{a} X_t dW_t
        \end{align*}

    \end{enumerate}

\end{enumerate}



\end{document}


%%%%%%%%%%%%%%%%%%%%%%%%%%%%%%%%%%%%%%%%%%%%%%%%%%%%%%%%%%%%%%%%%%%%%%%%
%%%%%%%%%%%%%%%%%%%%%%%%%%%%%%%%%%%%%%%%%%%%%%%%%%%%%%%%%%%%%%%%%%%%%%%%
%%%%%%%%%%%%%%%%%%%%%%%%%%%%%%%%%%%%%%%%%%%%%%%%%%%%%%%%%%%%%%%%%%%%%%%%

%%%% SAMPLE CODE %%%%%%%%%%%%%%%%%%%%%%%%%%%%%%%%%%%%%%

    %% BIBLIOGRAPHIES %%

        \cite{LabelInSourcesFile}  %Use in text; cites
        \citep{LabelInSourcesFile} %Use in text; cites in parens

        \nocite{LabelInSourceFile} % Includes in refs w/o specific citation
        \bibliographystyle{apalike}  % Or some other style

        % To ditch the ``References'' header
        \begingroup
        \renewcommand{\section}[2]{}
        \endgroup

        \bibliography{sources} % where sources.bib has all the citation info

    %% SPACING %%

        \vspace{1in}
        \hspace{1in}


    %% INCLUDING PDF PAGE %%

        \includepdf{file.pdf}


    %% INCLUDING CODE %%

        \verbatiminput{file.ext}
            %---Includes verbatim text from the file
        \texttt{text}
            %---Renders text in courier, or code-like, font

        \matlabcode{file.m}
            %---Includes Matlab code with colors and line numbers


    %% INCLUDING FIGURES %%

        % Basic Figure with size scaling
            \begin{figure}[h!]
               \centering
               \includegraphics[scale=1]{file.pdf}
            \end{figure}

        % Basic Figure with specific height
            \begin{figure}[h!]
               \centering
               \includegraphics[height=5in, width=5in]{file.pdf}
            \end{figure}

        % Figure with cropping, where the order for trimming is  L, B, R, T
            \begin{figure}
               \centering
               \includegraphics[trim={1cm, 1cm, 1cm, 1cm}, clip]{file.pdf}
            \end{figure}


        % Side by Side figures
            \begin{figure}[h!]
                \centering
                \mbox{\subfigure{
                    \includegraphics[scale=1]{file1.pdf}
                }\quad\subfigure{
                    \includegraphics[scale=1]{file2.pdf}
                }
                }
            \end{figure}


