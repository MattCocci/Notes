\documentclass[12pt]{article}
\author{Matthew D. Cocci}
\title{Homework 10}
\date{\today}
%% Formatting & Spacing %%%%%%%%%%%%%%%%%%%%%%%%%%%%%%%%%%%%

%\usepackage[top=1in, bottom=1in, left=1in, right=1in]{geometry} % most detailed page formatting control
\usepackage{fullpage} % Simpler than using the geometry package; std effect
\usepackage{setspace}
%\onehalfspacing
\usepackage{microtype}


%% Header %%%%%%%%%%%%%%%%%%%%%%%%%%%%%%%%%%%%%%%%%%%%%%%%%

%\usepackage{fancyhdr}
%\pagestyle{fancy}
%\lhead{}
%\rhead{}
%\chead{}
%\setlength{\headheight}{15.2pt}
    %---Make the header bigger to avoid overlap

%\renewcommand{\headrulewidth}{0.3pt}
    %---Width of the line

%\setlength{\headsep}{0.2in}
    %---Distance from line to text


%% Mathematics Related %%%%%%%%%%%%%%%%%%%%%%%%%%%%%%%%%%%

\usepackage{amsmath}
\usepackage{amsfonts}
\usepackage{mathrsfs}
\usepackage{amsthm} %allows for labeling of theorems
\usepackage{accents}
\theoremstyle{plain}
\newtheorem{thm}{Theorem}[section]
\newtheorem{lem}[thm]{Lemma}
\newtheorem{prop}[thm]{Proposition}
\newtheorem{cor}[thm]{Corollary}

\theoremstyle{definition}
\newtheorem{defn}[thm]{Definition}
\newtheorem{ex}[thm]{Example}

\theoremstyle{remark}
\newtheorem*{rem}{Remark}
\newtheorem*{note}{Note}

% Below supports left-right alignment in matrices so the negative
% signs don't look bad
\makeatletter
\renewcommand*\env@matrix[1][c]{\hskip -\arraycolsep
  \let\@ifnextchar\new@ifnextchar
  \array{*\c@MaxMatrixCols #1}}
\makeatother


%% Font Choices %%%%%%%%%%%%%%%%%%%%%%%%%%%%%%%%%%%%%%%%%

\usepackage[T1]{fontenc}
\usepackage{lmodern}
\usepackage[utf8]{inputenc}
%\usepackage{blindtext}


%% Figures %%%%%%%%%%%%%%%%%%%%%%%%%%%%%%%%%%%%%%%%%%%%%%

\usepackage{graphicx}
\usepackage{subfigure}
    %---For plotting multiple figures at once
%\graphicspath{ {Directory/} }
    %---Set a directory for where to look for figures


%% Hyperlinks %%%%%%%%%%%%%%%%%%%%%%%%%%%%%%%%%%%%%%%%%%%%
\usepackage{hyperref}
\hypersetup{
    colorlinks,
        %---This colors the links themselves, not boxes
    citecolor=black,
        %---Everything here and below changes link colors
    filecolor=black,
    linkcolor=black,
    urlcolor=black
}

%% Including Code %%%%%%%%%%%%%%%%%%%%%%%%%%%%%%%%%%%%%%%

\usepackage{verbatim}
    %---For including verbatim code from files, no colors

\usepackage{listings}
\usepackage{color}
\definecolor{mygreen}{RGB}{28,172,0}
\definecolor{mylilas}{RGB}{170,55,241}
\newcommand{\matlabcode}[1]{%
    \lstset{language=Matlab,%
        basicstyle=\footnotesize,%
        breaklines=true,%
        morekeywords={matlab2tikz},%
        keywordstyle=\color{blue},%
        morekeywords=[2]{1}, keywordstyle=[2]{\color{black}},%
        identifierstyle=\color{black},%
        stringstyle=\color{mylilas},%
        commentstyle=\color{mygreen},%
        showstringspaces=false,%
            %---Without this there will be a symbol in
            %---the places where there is a space
        numbers=left,%
        numberstyle={\tiny \color{black}},%
            %---Size of the numbers
        numbersep=9pt,%
            %---Defines how far the numbers are from the text
        emph=[1]{for,end,break,switch,case},emphstyle=[1]\color{red},%
            %---Some words to emphasise
    }%
    \lstinputlisting{#1}
}
    %---For including Matlab code from .m file with colors,
    %---line numbering, etc.

%% Bibliographies %%%%%%%%%%%%%%%%%%%%%%%%%%%%%%%%%%%%

%\usepackage{natbib}
    %---For bibliographies
%\setlength{\bibsep}{3pt} % Set how far apart bibentries are

%% Misc %%%%%%%%%%%%%%%%%%%%%%%%%%%%%%%%%%%%%%%%%%%%%%

\usepackage{enumitem}
    %---Has to do with enumeration
\usepackage{appendix}
\usepackage{pdfpages}
    %---For including whole pdf pages as a page in doc


%% User Defined %%%%%%%%%%%%%%%%%%%%%%%%%%%%%%%%%%%%%%%%%%

%\newcommand{\nameofcmd}{Text to display}



%%%%%%%%%%%%%%%%%%%%%%%%%%%%%%%%%%%%%%%%%%%%%%%%%%%%%%%%%%%%%%%%%%%%%%%%
%% BODY %%%%%%%%%%%%%%%%%%%%%%%%%%%%%%%%%%%%%%%%%%%%%%%%%%%%%%%%%%%%%%%%
%%%%%%%%%%%%%%%%%%%%%%%%%%%%%%%%%%%%%%%%%%%%%%%%%%%%%%%%%%%%%%%%%%%%%%%%


\begin{document}
\maketitle

%\tableofcontents


\begin{enumerate}
  \item % Question 1
    In this question, we will reference GBM
    \begin{align*}
      dX_t = \lambda X_t dt + \sigma X_t dW_t
    \end{align*}
    \begin{enumerate}
      \item % Question 1a
        Letting $\rho(x,t)$ denote the density at time $t$ with
        $\rho_0(x)$ representing the initial probability density, the
        corresponding Fokker-Planck equation for GBM is
        \begin{align*}
          \frac{\partial \rho}{\partial t}
          &= \mathscr{L}^*\rho\\
          &= -\frac{\partial}{\partial x}
            \left(
            \lambda x \rho - \frac{1}{2}
            \frac{\partial}{\partial x}\left((\sigma x)^2\rho\right)
            \right)\\
          &= -\lambda \frac{\partial}{\partial x}
            \left( x \rho\right)
            + \sigma^2 \frac{1}{2}
            \frac{\partial^2}{\partial x^2}(x^2\rho)
          \qquad \text{where } \rho(x,0) = \rho_0(x)
        \end{align*}

      \item % Question 1b
        For a stationary distribution $\rho_s$ to exist, we must have
        \begin{align*}
          0 &= \mathscr{L}^*\rho_s\\
          &= -\frac{\partial}{\partial x}
            \left(
            \lambda x \rho_s - \frac{1}{2}
            \frac{\partial}{\partial x}\left((\sigma x)^2\rho_s\right)
            \right)
        \end{align*}
        Let's look for a solution with the object in parens equal to
        zero:
        \begin{align*}
          0
          &=
          - \lambda x \rho_s + \frac{1}{2}
            \frac{\partial}{\partial x}\left((\sigma x)^2\rho_s\right)\\
          &=
          - \lambda x \rho_s + \frac{1}{2}
            \sigma^2 \left[
              2x\rho_s + x^2\frac{\partial \rho_s}{\partial x}
            \right]\\
          &=
          - 2\lambda x \rho_s +
            \sigma^2 \left[
              2x\rho_s + x^2\frac{\partial \rho_s}{\partial x}
            \right]\\
          0 &=
            x^2\frac{\partial \rho_s}{\partial x}
            + 2x\rho_s (\sigma^2 -\lambda)
        \end{align*}
        Now since Geometric Brownian Motion is always positive (unless
        it starts at initial condition $X_0=0$), we can divide through
        by $x$ and simplify further:
        \begin{align*}
          0 &=
            x\frac{\partial \rho_s}{\partial x}
            + 2\rho_s (\sigma^2 -\lambda) \\
          \Leftrightarrow\qquad
            x\frac{\partial \rho_s}{\partial x}
            &=
            - 2\rho_s (\sigma^2 -\lambda) \\
            \frac{d \rho_s}{\rho_s}
            &=
            - 2\rho_s (\sigma^2 -\lambda) \frac{1}{x} dx\\
          \Rightarrow\qquad
            \ln\rho_s
            &=
            - 2\rho_s (\sigma^2 -\lambda) \int^\infty_0 \frac{1}{x} dx
        \end{align*}
        Therefore, the stationary distribution $\rho_s$ cannot be
        defined as the the integral on the righthand side
        \begin{align*}
          \int^\infty_0 \frac{1}{x} dx = (\ln x)|^\infty_0
        \end{align*}
        is not defined.

      \item % Question 1c
        To describe how the moments evolve, we want to use version 1 of
        the backward equation. So let $u^{(n)}(x,t):=E[X^n|X_0=x]$. Then
        \begin{align*}
          \frac{\partial u^{(n)}}{\partial t} = \mathscr{L} u^{(n)}
          \qquad
          \text{where } u^{(n)}(x,0) = x^n
        \end{align*}
        Substituting in, we get
        \begin{align*}
          \frac{\partial u^{(n)}}{\partial t} =
          (\lambda x) \frac{du^{(n)}}{dx}
          + \frac{1}{2}(\sigma x)^2 \frac{\partial u^{(n)}}{\partial x^2}
        \end{align*}


      \item % Question 1d
        Letting $f(X_t) = X_t^n$, we calculate the following derivatives
        necessary to apply Ito's formula:
        \begin{align*}
          \frac{\partial f}{\partial t} = 0
          \qquad
          \frac{\partial f}{\partial x} = nx^{n-1}
          \qquad
          \frac{\partial^2 f}{\partial x^2} = n(n-1)x^{n-2}
        \end{align*}
        we can apply Ito's formula to write
        \begin{align*}
          df(X_t)
          &= \left(
            nX_t^{n-1} (\lambda X_t)
            + \frac{1}{2}n(n-1)X_t^{n-2} (\sigma X_t)^2
            \right) dt
            + n X_t^{n-1} (\sigma X_t)dW_t \\
          dX^n_t
          &= nX_t^{n}\left(
            \lambda + \frac{1}{2}(n-1) \sigma^2
            \right) dt
            + n X_t^{n} \sigma dW_t \\
          \Rightarrow\qquad
          X^n_t - X^n_0
          &= \int^t_0 nX_s^{n}\left(
            \lambda + \frac{1}{2}(n-1) \sigma^2
            \right) ds
            + \int^t_0 n X_s^{n} \sigma dW_s
        \end{align*}
        Now apply the conditional expectation $E[\;\cdot\;|\; X_0=x_0]$
        to the statement
        \begin{align*}
          E[X^n_t\;|\;X_0=x_0] - E[X^n_0 \;|\; X_0=x_0]
          &= E\left[ \int^t_0 nX_s^{n}\left(
            \lambda + \frac{1}{2}(n-1) \sigma^2
            \right) ds\;\big| \;X_0=x_0\right]\\
          &\qquad
            + E\left[ \int^t_0 n X_s^{n} \sigma dW_s
                \;\big|\; X_0=x_0\right]
        \end{align*}
        Since the Ito Integral is a martingale, the second term in the
        sum drops out. We also substitute in for $M_n$ to get
        \begin{align*}
          M_n(x_0,t) - M_n(x_0,0)
          &= E\left[ \int^t_0 nX_s^{n}\left(
            \lambda + \frac{1}{2}(n-1) \sigma^2
            \right) ds\;\big| \;X_0=x_0\right]\\
          &= \int^t_0 n
            E\left[X_s^{n}| \;X_0=x_0\right]
            \left(
            \lambda + \frac{1}{2}(n-1) \sigma^2
            \right) ds\\
          \Leftrightarrow \qquad
          M_n(x_0,t) - x_0^n
          &= \int^t_0 n
            M_n(x_0,s)
            \left(
            \lambda + \frac{1}{2}(n-1) \sigma^2
            \right) ds
        \end{align*}
        Differentiating with respect to $t$:
        \begin{align*}
          \frac{d}{dt}\left[
          M_n(x_0,t) - x_0^n
          \right]
          &=
          \frac{d}{dt}\left[
          \int^t_0 n
            M_n(x_0,s)
            \left(
            \lambda + \frac{1}{2}(n-1) \sigma^2
            \right) ds
          \right]\\
          \frac{dM_n}{dt}
          &=
            nM_n
            \left( \lambda + \frac{1}{2}(n-1) \sigma^2 \right)
        \end{align*}
        with $M_n(0) = x_0^n$.

      \item % Question 1e
        We can solve for $M_n(x_0,t)$ using separation of variables:
        \begin{align*}
          \frac{dM_n}{M_n}
          &= n \left( \lambda + \frac{1}{2}(n-1) \sigma^2 \right) dt\\
          \ln M_n
          &=
          \int_0^t
          n \left( \lambda + \frac{1}{2}(n-1) \sigma^2 \right) ds\\
          &=
          n \left( \lambda + \frac{1}{2}(n-1) \sigma^2 \right) |_0^t\\
          M_n
          &=
          x_0\exp\left\{
            n \left( \lambda + \frac{1}{2}(n-1) \sigma^2 \right)t
          \right\}
        \end{align*}

      \item % Question 1f
        To answer this, we can look at the time evolution of $M_t$ again:
        \begin{align*}
          \frac{dM_t}{dt}
          &=
          \left(
          \lambda n + \frac{\sigma^2}{2}n(n-1)
          \right)M_n
          %\\
          %&=
          %\left(
          %n\left(\lambda-\frac{\sigma^2}{2}\right)  + \frac{\sigma^2}{2}n^2
          %\right)M_n
        \end{align*}
        Since geometric Brownian motion is always positive, it's moments
        $M_n$ must also be positive. Therefore, to get the moments to
        decay to zero, we need
        \begin{align*}
          \lambda n + \frac{\sigma^2}{2}n(n-1)
          &< 0 \\
          \Leftrightarrow\qquad
          \lambda + \frac{\sigma^2}{2}(n-1)
          &< 0
        \end{align*}
        To get at least the first moment to decay to zero, it must be
        the case that $\lambda<0$. Now, taking $\lambda<0$ and
        $\sigma$ as given, we can find the maximal $N$ such that all
        lower moments decay to zero by solving for $n$ in the above
        expression:
        \begin{align*}
          n &< -\lambda \frac{2}{\sigma^2}+ 1
        \end{align*}
        Note that since $\lambda$ is negative, $-\lambda$ is actually a
        positive number.  Intuitively, this says that the smaller the
        diffusion term relative to the negative drift, the more moments
        will decay to zero.
    \end{enumerate}

  \item % Question 2
    \begin{enumerate}
      \item % Question 2a

      \item % Question 2b
        To compute the evolution of the total probability, we use the
        fact that
        \begin{align*}
          \frac{dP_{tot}}{dt}
          =
          \frac{d}{dt}
          \int_D \rho
          &= \int_{\partial D} -\underaccent{\bar}{j}\cdot \hat{n}
        \end{align*}
        Since the boundary is just a single point, $x=0$, and since
        $\underaccent{\bar}{j}$ is given as
        \begin{align*}
          \underaccent{\bar}{j}\cdot \hat{n} = \kappa \rho
        \end{align*}
        we have that
        \begin{align*}
          \frac{dP_{tot}}{dt}
          = \int_{x=0}
          - \kappa \rho(x,t)
          =
          - \kappa \rho(0,t)
        \end{align*}
        Since the Ornstein-Uhlenbeck process is known to have the
        following distribution based on its initial condition.

        Total probability is not conserved. It is decreasing at a rate
        equal to some fraction of the density at the boundary.
    \end{enumerate}

  \item % Question 3
    \begin{enumerate}
      \item % Question 3a
        $X_t$ evolves on $[0,L]$ as follows:
        \begin{align*}
          dX_t =
          \begin{cases}
            -F \; dt + \sigma dW_t & X_t \in [0,d]\\
            \sigma dW_t & X_t \in (d,L]\\
          \end{cases}
        \end{align*}

      \item % Question 3b
        First, the forward equation:
        \begin{align*}
          \text{For $x\in[0,d]$}\qquad
          \frac{\partial \rho}{\partial t}
          &=
          F\frac{\partial \rho}{\partial x}
          + \frac{\sigma^2}{2} \frac{\partial^2 \rho}{\partial x^2}
          \qquad
          \rho(x,0) = \rho_0(x)
          \\
          \text{For $x\in(d,L]$}\qquad
          \frac{\partial \rho}{\partial t}
          &=
          \frac{\sigma^2}{2} \frac{\partial^2 \rho}{\partial x^2}
          \qquad\qquad\quad
          \rho(x,0) = \rho_0(x)
        \end{align*}
        Next, the backward equation:
        \begin{align*}
          \text{For $x\in[0,d]$}\qquad
          \frac{\partial u}{\partial t}
          &= -F \frac{\partial u}{\partial x}
          + \frac{\sigma^2}{2}\frac{\partial^2 u}{\partial x^2}
          \qquad
          u(x,0) = f(x)\\
          \text{For $x\in(d,L]$}\qquad
          \frac{\partial u}{\partial t}
          &=
          \frac{\sigma^2}{2}\frac{\partial^2 u}{\partial x^2}
          \qquad\quad\qquad
          u(x,0) = f(x)
        \end{align*}
        where $u(x,t) = E[f(X_t)|X_0=x]$.

      \item % Question 3c
        %To find the stationary distribution, first note that at the
        %boundary $x=d$, we must match flows out of $[0,d]$ with flows
        %into $(d,L]$:
        %\begin{align*}
          %F \rho + \frac{\sigma^2}{2} \frac{d\rho}{dx}
          %&= - \frac{\sigma^2}{2} \frac{d\rho}{dx}\\
          %\Rightarrow\qquad
          %\frac{d\rho}{\rho}
          %&=\frac{-F}{\sigma^2}{dx}\\
          %\rho &= C_0e^{\frac{-F}{\sigma^2}x}|_d = C_0e^{\frac{-F}{\sigma^2}d}
        %\end{align*}
        %This relates the two solutions we will find on the
        %sub-intervals.

        Now let's look at the stationary distribution on the interval
        $[0,d]$:
        \begin{align*}
          0 &= -F \rho - \frac{\sigma^2}{2} \frac{\partial \rho}{\partial x}\\
          \frac{\partial \rho}{\rho } &= -\frac{2F }{\sigma^2} dx \\
          \rho &= C_1e^{-\frac{2F }{\sigma^2} x}
        \end{align*}
        Next, let's look at the stationary distribution on $(d,L]$:
        \begin{align*}
          0 &= - \frac{\sigma^2}{2} \frac{\partial \rho}{\partial x}\\
          \Rightarrow
          \qquad
          \rho &= C_2
        \end{align*}
        To get everything to match, we need at $d$
        \begin{align*}
          %C_0e^{\frac{-F}{\sigma^2}d} =
          C_1e^{-\frac{2F }{\sigma^2} d} = C_2
        \end{align*}
        %We choose $C_1$ so that
        %\begin{align*}
          %C_1=
          %C_0e^{\frac{F }{\sigma^2} d}
        %\end{align*}
        %We can choose $C_2$ so that
        %\begin{align*}
          %C_2 = C_0e^{\frac{-F}{\sigma^2}d}
        %\end{align*}
        Therefore, this gives
        \begin{align*}
          \rho =
          \begin{cases}
            C_1e^{-\frac{2F }{\sigma^2} x} & x\in[0,d] \\
            C_1e^{-\frac{2F}{\sigma^2}d} & x\in(d,L]
          \end{cases}
        \end{align*}
        Lastly, now that everything is a function of $C_1$, we choose
        that so everything integrates to 1.
        \begin{align*}
          1 &= \int_0^L \rho\; dx\\
          &= \int_0^d C_1e^{-\frac{2F }{\sigma^2} x}\;dx
          + C_1e^{-\frac{2F}{\sigma^2}d}\int^L_d  dx\\
          &= \left(-C_1\frac{\sigma^2}{2F}e^{-\frac{2F}{\sigma^2} x}\right)\big|_0^d
          + C_1e^{-\frac{2F}{\sigma^2}d}(L-d)\\
          1 &= C_1\frac{\sigma^2}{2F}\left(1-e^{-\frac{2F}{\sigma^2} d}\right)
          + C_1e^{-\frac{2F}{\sigma^2}d}(L-d)\\
          \Rightarrow
          C_1
          &= \left[\frac{\sigma^2}{2F}\left(1-e^{-\frac{2F}{\sigma^2} d}\right)
          + e^{-\frac{2F}{\sigma^2}d}(L-d)\right]^{-1}
        \end{align*}
        Implying the stationary distribution is
        \begin{align*}
          \rho =
          \begin{cases}
            C_1e^{-\frac{2F }{\sigma^2} x} & x\in[0,d] \\
            C_1e^{-\frac{2F}{\sigma^2}d} & x\in(d,L]
          \end{cases}
          \qquad
          \text{where }
          C_1
          &= \left[\frac{\sigma^2}{2F}\left(1-e^{-\frac{2F}{\sigma^2} d}\right)
          + e^{-\frac{2F}{\sigma^2}d}(L-d)\right]^{-1}
        \end{align*}
    \end{enumerate}

  \item % Question 4
    The Langevin equation is defined
    \begin{align*}
      m\ddot{X}_t + \gamma \dot{X}_t + \nabla U(X_t)
      = \sqrt{2}\sigma \eta(t)
    \end{align*}
    \begin{enumerate}
      \item % Question 4a
        We first define
        \begin{align*}
          V_t := \dot{X}_t = \frac{dX_t}{dt}
          \qquad\Rightarrow\qquad
          dX_t = V_t dt
        \end{align*}
        We can also use the definition of $V_t$ to get
        \begin{align*}
        \ddot{X}_t = \frac{d}{dt} \dot{X}_t = \frac{d}{dt} V_t
        \end{align*}
        Now substituting these identities into the Langevin equation, we
        get
        \begin{align*}
          m \frac{dV_t}{dt} + \gamma V_t + \nabla U(X_t)
          = \sqrt{2}\sigma \eta(t)
        \end{align*}
        Multiplying through by $dt$ and rearranging, we can rewrite in
        SDE form as a system of equations
        \begin{align*}
          m \; dV_t &+ \gamma V_t dt + \nabla U(X_t) dt
          = \sqrt{2}\sigma dW_t \\\\
          \Rightarrow\qquad
          dV_t &=
          -\frac{1}{m}
          (\gamma V_t + \nabla U(X_t)) dt
          + \frac{\sqrt{2}\sigma}{m} dW_t \\
          dX_t &= V_t dt
        \end{align*}
        Writing in more convenient vector notation, this is equivalent
        to
        \begin{align}
          d
          \begin{pmatrix}
            V_t \\ X_t
          \end{pmatrix}
          =
          \begin{pmatrix}
            -\frac{1}{m} (\gamma V_t + \nabla U(X_t)) \\
            V_t
          \end{pmatrix}
          dt
          +
          \begin{pmatrix}
            \frac{\sqrt{2}\sigma}{m} \\ 0
          \end{pmatrix}
          dW_t
          \label{q4a.1}
        \end{align}

      \item % Question 4b
        Using (\ref{q4a.1}), we can write down the Fokker-Planck
        Equation:
        \begin{align}
          \frac{\partial \rho}{\partial t}
          &=
          -\nabla \cdot (b \rho)
          + \frac{1}{2} \nabla^2 : (a\rho)
          \label{q4b.1}
        \end{align}
        where $b$ is the drift vector in (\ref{q4a.1}) and $a$
        represesnts
        \begin{align*}
          a = \sigma \sigma^T
          =
          \begin{pmatrix}
            \frac{2\sigma^2}{m^2} & 0 \\
            0 & 0
          \end{pmatrix}
        \end{align*}
        We can start simplifying terms:
        \begin{align*}
          -\nabla \cdot (b \rho)
          &=
          \frac{\partial}{\partial v}
          \left(\frac{1}{m} (\gamma v + \nabla U(x)) \rho\right)
          - \frac{\partial}{\partial x} (v\rho) \\
          &=
          \frac{\partial}{\partial v}
          \left(
          \frac{1}{m} \gamma v \rho
          \right)
          +
          \frac{\partial}{\partial v}
          \left(
          \frac{1}{m} \nabla U(x) \rho
          \right)
          - v\frac{\partial\rho}{\partial x} \\
          &=
          \frac{\gamma}{m}\frac{\partial}{\partial v}
          \left( v \rho \right)
          +
          \frac{\nabla U(x)}{m} \frac{\partial \rho}{\partial v}
          - v\frac{\partial\rho}{\partial x}
        \end{align*}
        Next, the component with the diffusion terms, which is easy
        since the zeros force all but one term to drop:
        \begin{align*}
          \frac{1}{2} \nabla^2 : (a\rho)
          &= \frac{1}{2}\frac{\partial^2}{\partial v^2}
          \left(\frac{2\sigma^2}{m^2}\rho\right)\\
          &= \frac{\sigma^2}{m^2}\frac{\partial^2\rho}{\partial v^2}
        \end{align*}
        Putting this together, we have the following Fokker-Planck
        Equation
        \begin{align*}
          \frac{\partial \rho}{\partial t}
          &=
          - v\frac{\partial\rho}{\partial x}
          + \frac{1}{m}
          \left(
          \gamma \frac{\partial}{\partial v} \left( v \rho \right)
          +
          \nabla U(x)\frac{\partial \rho}{\partial v}
          \right)
          + \frac{\sigma^2}{m^2}\frac{\partial^2\rho}{\partial v^2}
        \end{align*}

      \item % Question 4c
        By definition, the invariant distribution $\rho_s(x,v) =
        Z^{-1} e^{-\beta H(x,v)}$ with $H(x,v)=\frac{mv^2}{2} + U(x)$
        must be such that
        \begin{align*}
          \frac{\partial \rho_s}{\partial t}
          &=
          \mathscr{L}^* \rho_s = 0
        \end{align*}
        Applying $\mathscr{L}^*$ to $\rho_s$, we get:
        \begin{align*}
          \mathscr{L}^*\rho_s
          &=
          -\nabla \cdot
          \begin{pmatrix}
            -\frac{1}{m} (\gamma v + \nabla U(x)) \rho_s\\
            v\rho_s
          \end{pmatrix}
          + \frac{1}{2}
          \nabla^2 :
          \begin{pmatrix}
            \frac{2\sigma^2}{m^2}\rho_s & 0 \\
            0 & 0
          \end{pmatrix}
        \end{align*}
        Computing the relevant derivatives for this expression, we get
        \begin{align*}
          -\frac{d}{dv}
          \left( -\frac{1}{m} (\gamma v + \nabla U(x)) \rho_s \right)
          &=
          \frac{1}{m}
          \left(
          \gamma \rho_s
          +(\gamma v + \nabla U(x))
          \frac{d}{dv} \rho_s
          \right)\\
          &=
          \frac{1}{m}
          \left( \gamma \rho_s
          - (\gamma v + \nabla U(x))\beta mv\rho_s \right)\\
          &=
          \frac{1}{m}
          \left(
          \gamma \rho_s
          -
          (\gamma v + \nabla U(x))\frac{\gamma}{\sigma^2} mv\rho_s
          \right)\\
          &=
          \frac{\gamma}{m}
          \rho_s
          -(\gamma v + \nabla U(x))
          \frac{\gamma}{\sigma^2} v\rho_s
        \end{align*}
        Next, we compute
        \begin{align*}
          -\frac{d}{dx} (v\rho_s)
          &= v\beta \nabla U(x)\rho_s\\
          &= v\frac{\gamma}{\sigma^2} \nabla U(x)\rho_s
        \end{align*}
        Moving to the diffusion component, all that's left is
        \begin{align*}
          \frac{1}{2}\frac{\partial^2}{\partial v^2}
          \left(\frac{2\sigma^2}{m^2}\rho_s\right)
          &=
          \frac{\sigma^2}{m^2}
          \frac{\partial}{\partial v}
          \left( -\beta mv\rho_s \right)\\
          &=
          \frac{\sigma^2}{m^2}
          \left( -\beta m\rho_s + (\beta mv)^2 \rho_s\right)\\
          &=
          \frac{\sigma^2}{m^2}
          \left( -\frac{\gamma}{\sigma^2} m\rho_s
            + \left(\frac{\gamma}{\sigma^2} mv\right)^2 \rho_s\right)\\
          &=
          \left( -\frac{\gamma}{m} \rho_s
            + \frac{\gamma^2}{\sigma^2} v^2 \rho_s\right)
        \end{align*}
        Putting this all together:
        \begin{align*}
          \mathscr{L}^*\rho_s
          &=
          \left[
            \left(
            \frac{\gamma}{m}
            \rho_s
            -(\gamma v + \nabla U(x))
            \frac{\gamma}{\sigma^2} v\rho_s
            \right)
            + v\frac{\gamma}{\sigma^2} \nabla U(x)\rho_s
          \right]
          +
          \left( -\frac{\gamma}{m} \rho_s
            + \frac{\gamma^2}{\sigma^2} v^2 \rho_s\right)\\
          &=
          \rho_s\left(
            \frac{\gamma}{m}
            -\frac{\gamma^2}{\sigma^2} v^2
            - v\frac{\gamma}{\sigma^2} \nabla U(x)
            + v\frac{\gamma}{\sigma^2} \nabla U(x)
          -\frac{\gamma}{m}
            + \frac{\gamma^2}{\sigma^2} v^2 \right)\\
            &= 0
        \end{align*}
        Therefore $\rho_s(x,v)$ is the invariant distribution.

      \item % Question 4d
        We now calculate the steady-state flux, $\underaccent{\bar}{j}$.
        \begin{align*}
          \underaccent{\bar}{j}
          =
          b\rho_s - \frac{1}{2}\nabla \cdot(a\rho)
          &=
          \begin{pmatrix}
            -\frac{1}{m} (\gamma v + \nabla U(x)) \rho_s\\
            v\rho_s
          \end{pmatrix}
          - \frac{1}{2}
          \nabla \cdot
          \begin{pmatrix}
            \frac{2\sigma^2}{m^2}\rho_s & 0 \\
            0 & 0
          \end{pmatrix}\\
          &=
          \begin{pmatrix}
            -\frac{1}{m} (\gamma v + \nabla U(x)) \rho_s\\
            v\rho_s
          \end{pmatrix}
          -
          \begin{pmatrix}
            -\beta mv \frac{\sigma^2}{m^2}\rho_s \\
            0
          \end{pmatrix}\\
          &=
          \begin{pmatrix}
            -\frac{1}{m} (\gamma v + \nabla U(x)) \rho_s\\
            v\rho_s
          \end{pmatrix}
          -
          \begin{pmatrix}
            mv \frac{\gamma}{m}\rho_s \\
            0
          \end{pmatrix}
        \end{align*}
        This is non-zero.

      \item % Question 4
        %This will now all be multidimensional, i.e. $X_t, V_t \in
        %\mathbb{R}^n$.

        %First, the system of equations:
        %\begin{align}
          %d
          %\begin{pmatrix}
            %V_t \\ X_t
          %\end{pmatrix}
          %=
          %\begin{pmatrix}
            %-\frac{1}{m} (\gamma(X_t) V_t + \nabla U(X_t)) \\
            %V_t
          %\end{pmatrix}
          %dt
          %+
          %\begin{pmatrix}
            %\frac{\sqrt{2}\sigma(X_t)}{m} \\ 0
          %\end{pmatrix}
          %dW_t
          %\label{q4a.1}
        %\end{align}
        %where the vector $(V_t X_t)'$ stacks the $n$-dimensional vectors
        %$V_t$ and $X_t$.

        %Now the Fokker-Planck Equation:
        %\begin{align}
          %\frac{\partial \rho}{\partial t}
          %&=
          %-\nabla \cdot (b \rho)
          %+ \frac{1}{2} \nabla^2 : (a\rho)
          %\label{q4b.1}
        %\end{align}
        %where $b$ is the drift vector in (\ref{q4a.1}) and $a$
        %represesnts
        %\begin{align*}
          %a = \sigma \sigma^T
          %=
          %\begin{pmatrix}
            %\frac{2\sigma^2}{m^2} & 0 \\
            %0 & 0
          %\end{pmatrix}
        %\end{align*}
        %We can start simplifying terms:
        %\begin{align*}
          %-\nabla \cdot (b \rho)
          %&=
          %\frac{\partial}{\partial v}
          %\left(\frac{1}{m} (\gamma v + \nabla U(x)) \rho\right)
          %- \frac{\partial}{\partial x} (v\rho) \\
          %&=
          %\frac{\partial}{\partial v}
          %\left(
          %\frac{1}{m} \gamma v \rho
          %\right)
          %+
          %\frac{\partial}{\partial v}
          %\left(
          %\frac{1}{m} \nabla U(x) \rho
          %\right)
          %- v\frac{\partial\rho}{\partial x} \\
          %&=
          %\frac{\gamma}{m}\frac{\partial}{\partial v}
          %\left( v \rho \right)
          %+
          %\frac{\nabla U(x)}{m} \frac{\partial \rho}{\partial v}
          %- v\frac{\partial\rho}{\partial x}
        %\end{align*}
        %Next, the component with the diffusion terms, which is easy
        %since the zeros force all but one term to drop:
        %\begin{align*}
          %\frac{1}{2} \nabla^2 : (a\rho)
          %&= \frac{1}{2}\frac{\partial^2}{\partial v^2}
          %\left(\frac{2\sigma^2}{m^2}\rho\right)\\
          %&= \frac{\sigma^2}{m^2}\frac{\partial^2\rho}{\partial v^2}
        %\end{align*}
        %Putting this together, we have the following Fokker-Planck
        %Equation
        %\begin{align*}
          %\frac{\partial \rho}{\partial t}
          %&=
          %- v\frac{\partial\rho}{\partial x}
          %+ \frac{1}{m}
          %\left(
          %\gamma \frac{\partial}{\partial v} \left( v \rho \right)
          %+
          %\nabla U(x)\frac{\partial \rho}{\partial v}
          %\right)
          %+ \frac{\sigma^2}{m^2}\frac{\partial^2\rho}{\partial v^2}
        %\end{align*}


    \end{enumerate}
\end{enumerate}

\end{document}


%%%%%%%%%%%%%%%%%%%%%%%%%%%%%%%%%%%%%%%%%%%%%%%%%%%%%%%%%%%%%%%%%%%%%%%%
%%%%%%%%%%%%%%%%%%%%%%%%%%%%%%%%%%%%%%%%%%%%%%%%%%%%%%%%%%%%%%%%%%%%%%%%
%%%%%%%%%%%%%%%%%%%%%%%%%%%%%%%%%%%%%%%%%%%%%%%%%%%%%%%%%%%%%%%%%%%%%%%%

%%%% SAMPLE CODE %%%%%%%%%%%%%%%%%%%%%%%%%%%%%%%%%%%%%%

    %% BIBLIOGRAPHIES %%

        \cite{LabelInSourcesFile}  %Use in text; cites
        \citep{LabelInSourcesFile} %Use in text; cites in parens

        \nocite{LabelInSourceFile} % Includes in refs w/o specific citation
        \bibliographystyle{apalike}  % Or some other style

        % To ditch the ``References'' header
        \begingroup
        \renewcommand{\section}[2]{}
        \endgroup

        \bibliography{sources} % where sources.bib has all the citation info

    %% SPACING %%

        \vspace{1in}
        \hspace{1in}


    %% INCLUDING PDF PAGE %%

        \includepdf{file.pdf}


    %% INCLUDING CODE %%

        \verbatiminput{file.ext}
            %---Includes verbatim text from the file
        \texttt{text}
            %---Renders text in courier, or code-like, font

        \matlabcode{file.m}
            %---Includes Matlab code with colors and line numbers


    %% INCLUDING FIGURES %%

        % Basic Figure with size scaling
            \begin{figure}[h!]
               \centering
               \includegraphics[scale=1]{file.pdf}
            \end{figure}

        % Basic Figure with specific height
            \begin{figure}[h!]
               \centering
               \includegraphics[height=5in, width=5in]{file.pdf}
            \end{figure}

        % Figure with cropping, where the order for trimming is  L, B, R, T
            \begin{figure}
               \centering
               \includegraphics[trim={1cm, 1cm, 1cm, 1cm}, clip]{file.pdf}
            \end{figure}


        % Side by Side figures
            \begin{figure}[h!]
                \centering
                \mbox{\subfigure{
                    \includegraphics[scale=1]{file1.pdf}
                }\quad\subfigure{
                    \includegraphics[scale=1]{file2.pdf}
                }
                }
            \end{figure}


