\documentclass[12pt]{article}

\author{Matthew D. Cocci}
\title{Homework 4 \\ {\normalsize Question 7}}
\date{\today}

%% Formatting & Spacing %%%%%%%%%%%%%%%%%%%%%%%%%%%%%%%%%%%%

%\usepackage[top=1in, bottom=1in, left=1in, right=1in]{geometry} % most detailed page formatting control
\usepackage{fullpage} % Simpler than using the geometry package; std effect
\usepackage{setspace}
%\onehalfspacing
\usepackage{microtype}


%% Header %%%%%%%%%%%%%%%%%%%%%%%%%%%%%%%%%%%%%%%%%%%%%%%%%

%\usepackage{fancyhdr}
%\pagestyle{fancy}
%\lhead{}
%\rhead{}
%\chead{}
%\setlength{\headheight}{15.2pt}
    %---Make the header bigger to avoid overlap

%\renewcommand{\headrulewidth}{0.3pt}
    %---Width of the line

%\setlength{\headsep}{0.2in}
    %---Distance from line to text


%% Mathematics Related %%%%%%%%%%%%%%%%%%%%%%%%%%%%%%%%%%%
\usepackage{amsmath}
\usepackage{amsfonts}
\usepackage{mathrsfs}
\usepackage{amsthm} %allows for labeling of theorems
\theoremstyle{plain}
\newtheorem{thm}{Theorem}[section]
\newtheorem{lem}[thm]{Lemma}
\newtheorem{prop}[thm]{Proposition}
\newtheorem{cor}[thm]{Corollary}

\theoremstyle{definition}
\newtheorem{defn}[thm]{Definition}
\newtheorem{ex}[thm]{Example}

\theoremstyle{remark}
\newtheorem*{rem}{Remark}
\newtheorem*{note}{Note}

% Below supports left-right alignment in matrices so the negative % signs don't look bad
\makeatletter
\renewcommand*\env@matrix[1][c]{\hskip -\arraycolsep
  \let\@ifnextchar\new@ifnextchar
  \array{*\c@MaxMatrixCols #1}}
\makeatother


%% Font Choices %%%%%%%%%%%%%%%%%%%%%%%%%%%%%%%%%%%%%%%%%

\usepackage[T1]{fontenc}
\usepackage{lmodern}
\usepackage[utf8]{inputenc}
%\usepackage{blindtext}


%% Figures %%%%%%%%%%%%%%%%%%%%%%%%%%%%%%%%%%%%%%%%%%%%%%

\usepackage{graphicx}
\usepackage{subfigure}
    %---For plotting multiple figures at once
%\graphicspath{ {Directory/} }
    %---Set a directory for where to look for figures


%% Hyperlinks %%%%%%%%%%%%%%%%%%%%%%%%%%%%%%%%%%%%%%%%%%%%
\usepackage{hyperref}
\hypersetup{
    colorlinks,
        %---This colors the links themselves, not boxes
    citecolor=black,
        %---Everything here and below changes link colors
    filecolor=black,
    linkcolor=black,
    urlcolor=black
}

%% Including Code %%%%%%%%%%%%%%%%%%%%%%%%%%%%%%%%%%%%%%%

\usepackage{verbatim}
    %---For including verbatim code from files, no colors

\usepackage{listings}
\usepackage{color}
\definecolor{mygreen}{RGB}{28,172,0}
\definecolor{mylilas}{RGB}{170,55,241}
\newcommand{\matlabcode}[1]{%
    \lstset{language=Matlab,%
        basicstyle=\footnotesize,%
        breaklines=true,%
        morekeywords={matlab2tikz},%
        keywordstyle=\color{blue},%
        morekeywords=[2]{1}, keywordstyle=[2]{\color{black}},%
        identifierstyle=\color{black},%
        stringstyle=\color{mylilas},%
        commentstyle=\color{mygreen},%
        showstringspaces=false,%
            %---Without this there will be a symbol in
            %---the places where there is a space
        numbers=left,%
        numberstyle={\tiny \color{black}},%
            %---Size of the numbers
        numbersep=9pt,%
            %---Defines how far the numbers are from the text
        emph=[1]{for,end,break,switch,case},emphstyle=[1]\color{red},%
            %---Some words to emphasise
    }%
    \lstinputlisting{#1}
}
    %---For including Matlab code from .m file with colors,
    %---line numbering, etc.

%% Bibliographies %%%%%%%%%%%%%%%%%%%%%%%%%%%%%%%%%%%%

%\usepackage{natbib}
    %---For bibliographies
%\setlength{\bibsep}{3pt} % Set how far apart bibentries are

%% Misc %%%%%%%%%%%%%%%%%%%%%%%%%%%%%%%%%%%%%%%%%%%%%%

\usepackage{enumitem}
    %---Has to do with enumeration
\usepackage{appendix}
\usepackage{pdfpages}
    %---For including whole pdf pages as a page in doc


%% User Defined %%%%%%%%%%%%%%%%%%%%%%%%%%%%%%%%%%%%%%%%%%

%\newcommand{\nameofcmd}{Text to display}



%%%%%%%%%%%%%%%%%%%%%%%%%%%%%%%%%%%%%%%%%%%%%%%%%%%%%%%%%%%%%%%%%%%%%%%%
%% BODY %%%%%%%%%%%%%%%%%%%%%%%%%%%%%%%%%%%%%%%%%%%%%%%%%%%%%%%%%%%%%%%%
%%%%%%%%%%%%%%%%%%%%%%%%%%%%%%%%%%%%%%%%%%%%%%%%%%%%%%%%%%%%%%%%%%%%%%%%


\begin{document}
\maketitle

%\tableofcontents

\begin{enumerate}
  \item[7.] % Question 7
    Before computing covariances, let's start with the mean of $Y_t$:
    \begin{align*}
      E[Y_t] = E[A_t + iB_t] = EA_t + iEB_t
    \end{align*}
    Now let's compute the covariance function of $Y_t$:
    \begin{align*}
      C_Y(t) &= E[(Y_{s+t}-EY_{s+t})\overline{(Y_s- EY_s)}]\\
      &=
        E\left[
          \left\{
            \left(A_{t+s} + i B_{t+s}\right)
            - \left(EA_{t+s} + i EB_{t+s}\right)
          \right\}
        \overline{%
          \left\{
            \left(A_{s} + i B_{s}\right)
            - \left(EA_{s} + i EB_{s}\right)
          \right\}
        }
        \right]\\
      &=
        E\left[
          \left\{
            \left(A_{t+s} - EA_{t+s}\right)
            + i\left( B_{t+s}- EB_{t+s}\right)
          \right\}
        \overline{%
          \left\{
            \left(A_{s} - EA_{s}\right)
            + i \left( B_{s}- EB_{s}\right)
          \right\}
        }
        \right]\\
      &=
        E\left[
          \left\{
            \left(A_{t+s} - EA_{t+s}\right)
            + i\left( B_{t+s}- EB_{t+s}\right)
          \right\}
          \left\{
            \left(A_{s} - EA_{s}\right)
            - i \left( B_{s}- EB_{s}\right)
          \right\}
        \right]
    \end{align*}
    Now that everything everything is conjugate correctly, we can
    distribute and simplify a bit:
    \begin{align*}
      C_Y(t)
      &=
        E\left[
            \left(A_{t+s} - EA_{t+s}\right)
            \left(A_{s} - EA_{s}\right)
          \right]
        - i E\left[
            \left(A_{t+s} - EA_{t+s}\right)
            \left( B_{s}- EB_{s}\right)
          \right] \\
      &\qquad
        + i E\left[
            \left( B_{t+s}- EB_{t+s}\right)
            \left(A_{s} - EA_{s}\right)
          \right]
        + E\left[
          \left( B_{t+s}- EB_{t+s}\right)
         \left( B_{s}- EB_{s}\right)
        \right]
    \end{align*}
    Now since $A_t$ and $B_t$ have identical distributions, the middle
    two terms (with the leading $i$) will cancel since they are
    symmetric expressions from the standpoint of the distributions of
    the RV's, leaving
    \begin{align*}
      C_Y(t)
      &=
        E\left[
            \left(A_{t+s} - EA_{t+s}\right)
            \left(A_{s} - EA_{s}\right)
          \right]
        + E\left[
          \left( B_{t+s}- EB_{t+s}\right)
         \left( B_{s}- EB_{s}\right)
        \right]
    \end{align*}
    Distributing and simplifying further, we are eventually left with:
    \begin{align*}
      C_Y(t)
      &=
      E\left[A_{t+s}A_s\right] - (EA_{t+s})(EA_s)
      + E\left[B_{t+s}B_s\right] - (EB_{t+s})(EB_s)
    \end{align*}
    which is actually just the sum of the two covariance functions for
    $A_t$ and $B_t$. Moreover, since $A$ and $B$ have identical
    distributions, we can simplify further to
    \begin{align*}
      C_Y(t)
      &=
      2 \left\{
        E\left[A_{t+s}A_s\right] - (EA_{t+s})(EA_s)\right\}
    \end{align*}
    Since $X_t = A_t$ and will have covariance function $C_X(t) =
    E\left[A_{t+s}A_s\right]- (EA_{t+s})(EA_s)$, we should choose
    $C_Y(t)=2C_X(t)$ as oru covariance function for $Y$.

  \item % Question 7b


\end{enumerate}


%% APPPENDIX %%

% \appendix




\end{document}


%%%%%%%%%%%%%%%%%%%%%%%%%%%%%%%%%%%%%%%%%%%%%%%%%%%%%%%%%%%%%%%%%%%%%%%%
%%%%%%%%%%%%%%%%%%%%%%%%%%%%%%%%%%%%%%%%%%%%%%%%%%%%%%%%%%%%%%%%%%%%%%%%
%%%%%%%%%%%%%%%%%%%%%%%%%%%%%%%%%%%%%%%%%%%%%%%%%%%%%%%%%%%%%%%%%%%%%%%%

%%%% SAMPLE CODE %%%%%%%%%%%%%%%%%%%%%%%%%%%%%%%%%%%%%%

    %% BIBLIOGRAPHIES %%

        \cite{LabelInSourcesFile}  %Use in text; cites
        \citep{LabelInSourcesFile} %Use in text; cites in parens

        \nocite{LabelInSourceFile} % Includes in refs w/o specific citation
        \bibliographystyle{apalike}  % Or some other style

        % To ditch the ``References'' header
        \begingroup
        \renewcommand{\section}[2]{}
        \endgroup

        \bibliography{sources} % where sources.bib has all the citation info

    %% SPACING %%

        \vspace{1in}
        \hspace{1in}


    %% INCLUDING PDF PAGE %%

        \includepdf{file.pdf}


    %% INCLUDING CODE %%

        \verbatiminput{file.ext}
            %---Includes verbatim text from the file
        \texttt{text}
            %---Renders text in courier, or code-like, font

        \matlabcode{file.m}
            %---Includes Matlab code with colors and line numbers


    %% INCLUDING FIGURES %%

        % Basic Figure with size scaling
            \begin{figure}[h!]
               \centering
               \includegraphics[scale=1]{file.pdf}
            \end{figure}

        % Basic Figure with specific height
            \begin{figure}[h!]
               \centering
               \includegraphics[height=5in, width=5in]{file.pdf}
            \end{figure}

        % Figure with cropping, where the order for trimming is  L, B, R, T
            \begin{figure}
               \centering
               \includegraphics[trim={1cm, 1cm, 1cm, 1cm}, clip]{file.pdf}
            \end{figure}


        % Side by Side figures
            \begin{figure}[h!]
                \centering
                \mbox{\subfigure{
                    \includegraphics[scale=1]{file1.pdf}
                }\quad\subfigure{
                    \includegraphics[scale=1]{file2.pdf}
                }
                }
            \end{figure}


