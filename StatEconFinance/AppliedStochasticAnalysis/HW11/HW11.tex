\documentclass[12pt]{article}
\author{Matthew D. Cocci}
\title{Homework 11}
\date{\today}
%% Formatting & Spacing %%%%%%%%%%%%%%%%%%%%%%%%%%%%%%%%%%%%

%\usepackage[top=1in, bottom=1in, left=1in, right=1in]{geometry} % most detailed page formatting control
\usepackage{fullpage} % Simpler than using the geometry package; std effect
\usepackage{setspace}
%\onehalfspacing
\usepackage{microtype}


%% Header %%%%%%%%%%%%%%%%%%%%%%%%%%%%%%%%%%%%%%%%%%%%%%%%%

%\usepackage{fancyhdr}
%\pagestyle{fancy}
%\lhead{}
%\rhead{}
%\chead{}
%\setlength{\headheight}{15.2pt}
    %---Make the header bigger to avoid overlap

%\renewcommand{\headrulewidth}{0.3pt}
    %---Width of the line

%\setlength{\headsep}{0.2in}
    %---Distance from line to text


%% Mathematics Related %%%%%%%%%%%%%%%%%%%%%%%%%%%%%%%%%%%

\usepackage{amsmath}
\usepackage{amsfonts}
\usepackage{mathrsfs}
\usepackage{amsthm} %allows for labeling of theorems
\usepackage{accents}
\theoremstyle{plain}
\newtheorem{thm}{Theorem}[section]
\newtheorem{lem}[thm]{Lemma}
\newtheorem{prop}[thm]{Proposition}
\newtheorem{cor}[thm]{Corollary}

\theoremstyle{definition}
\newtheorem{defn}[thm]{Definition}
\newtheorem{ex}[thm]{Example}

\theoremstyle{remark}
\newtheorem*{rem}{Remark}
\newtheorem*{note}{Note}

% Below supports left-right alignment in matrices so the negative
% signs don't look bad
\makeatletter
\renewcommand*\env@matrix[1][c]{\hskip -\arraycolsep
  \let\@ifnextchar\new@ifnextchar
  \array{*\c@MaxMatrixCols #1}}
\makeatother


%% Font Choices %%%%%%%%%%%%%%%%%%%%%%%%%%%%%%%%%%%%%%%%%

\usepackage[T1]{fontenc}
\usepackage{lmodern}
\usepackage[utf8]{inputenc}
%\usepackage{blindtext}


%% Figures %%%%%%%%%%%%%%%%%%%%%%%%%%%%%%%%%%%%%%%%%%%%%%

\usepackage{graphicx}
\usepackage{subfigure}
    %---For plotting multiple figures at once
%\graphicspath{ {Directory/} }
    %---Set a directory for where to look for figures


%% Hyperlinks %%%%%%%%%%%%%%%%%%%%%%%%%%%%%%%%%%%%%%%%%%%%
\usepackage{hyperref}
\hypersetup{
    colorlinks,
        %---This colors the links themselves, not boxes
    citecolor=black,
        %---Everything here and below changes link colors
    filecolor=black,
    linkcolor=black,
    urlcolor=black
}

%% Including Code %%%%%%%%%%%%%%%%%%%%%%%%%%%%%%%%%%%%%%%

\usepackage{verbatim}
    %---For including verbatim code from files, no colors

\usepackage{listings}
\usepackage{color}
\definecolor{mygreen}{RGB}{28,172,0}
\definecolor{mylilas}{RGB}{170,55,241}
\newcommand{\matlabcode}[1]{%
    \lstset{language=Matlab,%
        basicstyle=\footnotesize,%
        breaklines=true,%
        morekeywords={matlab2tikz},%
        keywordstyle=\color{blue},%
        morekeywords=[2]{1}, keywordstyle=[2]{\color{black}},%
        identifierstyle=\color{black},%
        stringstyle=\color{mylilas},%
        commentstyle=\color{mygreen},%
        showstringspaces=false,%
            %---Without this there will be a symbol in
            %---the places where there is a space
        numbers=left,%
        numberstyle={\tiny \color{black}},%
            %---Size of the numbers
        numbersep=9pt,%
            %---Defines how far the numbers are from the text
        emph=[1]{for,end,break,switch,case},emphstyle=[1]\color{red},%
            %---Some words to emphasise
    }%
    \lstinputlisting{#1}
}
    %---For including Matlab code from .m file with colors,
    %---line numbering, etc.

%% Bibliographies %%%%%%%%%%%%%%%%%%%%%%%%%%%%%%%%%%%%

%\usepackage{natbib}
    %---For bibliographies
%\setlength{\bibsep}{3pt} % Set how far apart bibentries are

%% Misc %%%%%%%%%%%%%%%%%%%%%%%%%%%%%%%%%%%%%%%%%%%%%%

\usepackage{enumitem}
    %---Has to do with enumeration
\usepackage{appendix}
\usepackage{pdfpages}
    %---For including whole pdf pages as a page in doc


%% User Defined %%%%%%%%%%%%%%%%%%%%%%%%%%%%%%%%%%%%%%%%%%

%\newcommand{\nameofcmd}{Text to display}



%%%%%%%%%%%%%%%%%%%%%%%%%%%%%%%%%%%%%%%%%%%%%%%%%%%%%%%%%%%%%%%%%%%%%%%%
%% BODY %%%%%%%%%%%%%%%%%%%%%%%%%%%%%%%%%%%%%%%%%%%%%%%%%%%%%%%%%%%%%%%%
%%%%%%%%%%%%%%%%%%%%%%%%%%%%%%%%%%%%%%%%%%%%%%%%%%%%%%%%%%%%%%%%%%%%%%%%


\begin{document}
\maketitle

%\tableofcontents


\begin{enumerate}
  \item % Question 1
    For a process $X_t=W_t$ (Brownian motion) and a function $q(x)$
    satisfying
    \begin{align*}
      \Delta q = 0
      \qquad
      \begin{cases}
        q = 1 & \text{on } \Gamma_1 \subset \partial D \\
        q = 0 & \text{on } \Gamma_2 \subset \partial D \\
        \nabla q \cdot n = 0 &
          \text{on } \partial D \setminus (\Gamma_1 \cup \Gamma_2) \\
      \end{cases}
    \end{align*}
    This is equivalent to the standard boundary value problem
    \begin{align*}
      \mathscr{L}q = -g(x) \qquad u(x) = f(x) \quad \text{on $\partial D$}
    \end{align*}
    where $g(x)=0$ and the generator is
    \begin{align*}
      \mathscr{L}q = b\cdot \nabla q +
      \frac{1}{2} a : \nabla^2 q = \frac{1}{2}\nabla^2 q
    \end{align*}
    because $b=0$ and $a=1$ for standard Brownian motion.

    Now define $\Gamma := \Gamma_1 \cup \Gamma_2$. Let $\tau_\Gamma$
    represent the stopping time when Brownian motion first hits
    $\Gamma_1$ or $\Gamma_2$. So the process is stopped at $\Gamma$ and
    reflected otherwise. Then
    \begin{align*}
      q(x)
      &= \mathbb{E}^x\left(
        f(X_{\tau_\Gamma}) + \int^{\tau_\Gamma}_0 g(X_t) dt
        \right) \\
      &= \mathbb{E}^x\left(
        1|_{X_{\tau_\Gamma}\in \Gamma_1} + \int^{\tau_\Gamma}_0 0 dt
        \right) \\
      &= \mathbb{E}^x\left( 1|_{X_{\tau_\Gamma}\in \Gamma_1}\right)\\
      \Rightarrow\qquad
      q(x)
      &= \mathbb{P}\left( {X_{\tau_\Gamma}\in \Gamma_1} | X_0 =x\right)
    \end{align*}
    In words, this is the probability that when process $X_t$ is in
    $\Gamma_1$ when it first hits $\Gamma = \Gamma_1 \cup \Gamma_2$.

  \item % Question 2
    Throughout this question, $W_t$ is Brownian motion starting at
    $W_0=0$. In addition
    \begin{align*}
      \tau_n = \inf \{t \; | \; W_t \not\in [-n,1]\}
    \end{align*}
    \begin{enumerate}
      \item % Question 2a
        We want to calculate $E\tau_n$. We saw in class that if we let
        $T_1(x) = E[\tau_n | W_0=x]$, this must satisfy
        \begin{align*}
          \mathscr{L}T_1 = -1
          \qquad \text{s.t.}
          \begin{cases}
            T_1(-n) = 0 \\
            T_1(1)=0
          \end{cases}
        \end{align*}
        where $\mathscr{L}$ is the generator of the process. But since
        the process is simple Brownian motion, this reduces to
        \begin{align*}
          -1 = \mathscr{L}T_1
          &= 0 \cdot \frac{\partial T_1}{\partial x}
          + \frac{1}{2} 1\cdot \frac{\partial^2 T_1}{\partial x^2}\\
          \Rightarrow\qquad
          -1 &= \frac{1}{2} \frac{\partial^2 T_1}{\partial x^2}\\
          \Leftrightarrow\qquad
          -2 &= \frac{\partial^2 T_1}{\partial x^2}
        \end{align*}
        Solving this equation gives
        \begin{align*}
          T_1(x) &= -x^2 + Cx + D
        \end{align*}
        Using the boundary conditions gives
        \begin{align*}
          0 &= -(-n)^2 + C(-n) + D
          \;\;\quad \Rightarrow
          C = 1-n\\
          0 &= -(1)^2 + C(1) + D \quad\qquad\qquad
          D = n\\\\
          \Rightarrow \quad
          T_1(x) &= -x^2 + (1-n)x + n
        \end{align*}
        Starting at initial value of 0,
        \begin{align*}
          \mathbb{E}\tau_n = T_1(0) = n
        \end{align*}

      \item % Question 2b
        We will solve for $q(x) = P(\text{Brownian Motion Hits $-n$
        before 1})$ by solving
        \begin{align*}
          \frac{\partial^2 q}{\partial x^2}
          &= 0
          \qquad
          \text{s.t.}
          \begin{cases}
            q(-n) = 1 \\
            q(1)  = 0 \\
          \end{cases}
        \end{align*}
        Solving for $q(x)$ gives
        \begin{align*}
          q(x) &= Cx + D
        \end{align*}
        Substituting in boundary conditions gives
        \begin{align*}
          1 &= C(-n) + D
          \quad \Rightarrow \quad
          C = -\frac{1}{(n+1)}\\
          0 &= C + D
          \quad\qquad\qquad\quad
          D = \frac{1}{(n+1)}
        \end{align*}
        Therefore,
        \begin{align*}
          q(x) &= -\frac{1}{(n+1)}\cdot x + \frac{1}{(n+1)}\\
               &= \frac{1-x}{(n+1)}
        \end{align*}
        Given a starting point of zero, the result becomes:
        \begin{align*}
          q(0) &= \frac{1}{n+1}
        \end{align*}
        Intuitively, this sense. For $n=1$, the interval is symmetric,
        and exiting at $-1$ is equally as likely as exiting at 1.
        Moreover, the probability of exiting at $-n$ decreases with $n$.
        These are both exactly as we would expect.

      \item % Question 2c
        To compute the expectation $\mathbb{E}W_{\tau_n}$ directly,
        notice that $W_{\tau_n}$ can only take on one of two values:
        $-n$ or $1$. Moreover, the previous question pinned down the
        probabilities of these objects, giving
        \begin{align*}
          \mathbb{E}W_{\tau_n}
          &=
          \left(\frac{1}{n+1}\right)(-n)
          + \left(1-\frac{1}{n+1}\right)\cdot 1\\
          &=
          -\frac{n}{n+1}
          + \frac{n+1 - 1}{n+1}\\
          &= 0
        \end{align*}

      \item % Question 2d
        To see why $\mathbb{E}\tau_\infty =\infty$, recall from part
        that
        \begin{align*}
          \mathbb{E}^x\tau_n = -x^2 + (1-n)x+n
        \end{align*}
        Taking the limit as $n\rightarrow\infty$ shows that
        $\mathbb{E}^x\tau_n=\infty$. Intuitively, Brownian motion can
        drift into and remain in negative territory for an arbitrary
        amount of time. This prevents the expected value from being
        finite.

      \item % Question 2e
        $\tau_\infty$ is almost surely finite because Brownian motion
        $W_t$ has variance $t$, so the average size of its drift from 0
        grows without bound.

        So start by suppose that $x$ is any finite number, and we want
        to know the probability that $W_t$ stays within $[-x,x]$ as
        $t\rightarrow\infty$. Then,
        \begin{align*}
          \lim_{t\rightarrow\infty}
          P(W_t\in[-x,x])
          &=
          \lim_{t\rightarrow\infty}
          \int^x_{-x} \phi\left(y/\sqrt{t}\right) dy\\
          &=
          \lim_{t\rightarrow\infty}
          \Phi\left(x/\sqrt{t}\right) -\Phi\left(-x/\sqrt{t}\right)\\
          &=0
        \end{align*}
        where $\phi$ is the standard normal pdf and $\Phi$ is the standard normal cdf.

        Therefore $\tau_\infty$ is almost surely finite, since, with
        probability 1, it will exit any finite boundary.
        %the probability that $\tau_\infty$ is finite given starting
        %point $W_0=0$ is
        %\begin{align*}
          %P(\tau_\infty<\infty)
          %&= \lim_{t\rightarrow\infty} P(W_t > 1)\\
          %&= \lim_{t\rightarrow\infty} \int^\infty_1 \phi\left(x/\sqrt{t}\right) dx\\
          %&= \lim_{t\rightarrow\infty} \Phi(\infty) - \Phi\left(1/\sqrt{t}\right)\\
          %&= \Phi(\infty) - \lim_{t\rightarrow\infty} \Phi\left(1/\sqrt{t}\right)\\
          %&= 1 - 0 = 1
        %\end{align*}

      \item % Question 2f
        Here, I am not sure how to show this, given that $\tau_\infty$
        was defined as the first time $W_t$ hits 1; therefore,
        $\mathbb{E}W_{\tau_\infty}$ will surely equal 1.
        %To compute the expectation, we must compute
        %\begin{align*}
          %EW_{\tau_\infty}
          %&=
          %\lim_{n\rightarrow\infty}
          %-n\left(\frac{1}{n+1}\right)
          %+
          %\lim_{n\rightarrow\infty}
          %1\left(1-\frac{1}{n+1}\right)\\
          %&=
          %\infty
          %+
          %1\\
          %&=
          %\lim_{n\rightarrow\infty}
          %-n\left(\frac{1}{n+1}\right)
          %+
          %\lim_{n\rightarrow\infty}
          %1\left(1-\frac{1}{n+1}\right)\\
          %&=
          %\infty
          %+
          %1\\
        %\end{align*}
    \end{enumerate}

  \item % Question 3
    Where $T_n(x):=E[T^n|X_0=x]$, we want to show that $T_n$ satisfies
    \begin{align*}
      \mathscr{L}T_n = -n T_{n-1} \qquad T_n(\partial D)=0
    \end{align*}
    Very similar to the notes, where $n=1$, start with the fact that
    $-\partial_t G$ is the pdf of $T$; therefore, by definition
    \begin{align*}
      T_n(x)=\mathbb{E}[T^n|X_0=x]
      &= -\int^\infty_0 t^n \frac{\partial G}{\partial t} dt\\
      \text{Int.\ by Parts and L'Hospital's Rule} \qquad
      &= \int^\infty_0 n t^{n-1} G \; dt
    \end{align*}
    From there, apply the operator, use the fact that it commutes, and
    use the definition of $T_{n-1}(x)$ and the backward equation to get
    \begin{align*}
      \mathscr{L}T_n(x)=
      &= \mathscr{L}\int^\infty_0 n t^{n-1} G \; dt
      = n\int^\infty_0  t^{n-1} \mathscr{L}G \; dt\\
      &= n\int^\infty_0  t^{n-1} \partial_tG \; dt\\
      &= n(-\mathbb{E}[T^{n-1}|X_0=x])\\
      \Rightarrow\qquad
      \mathscr{L}T_n=
      &= -nT_{n-1}
    \end{align*}

  \item % Question 4
    We want to find a parabolic equation for which the following
    function is a solution:
    \begin{align}
      v(x,t) = \mathbb{E}e^{-\frac{1}{2} \int^t_0 (W_s+x)^2 ds}
      \label{q4.1}
    \end{align}
    This suggests the Feynman-Kac formula for a time-homogeneous
    process, which pairs the PDE and solution
    \begin{align}
      \frac{\partial v}{\partial t}
        &= \mathscr{L}v - cv \qquad v(x,0) = f(x)\label{q4.2}\\
      v(x,t) &=
        \mathbb{E}^x\left[
          e^{-\int^t_0 c(X_s) ds}f(X_t)
        \right]
      \label{q4.3}
    \end{align}
    To get the PDE, define the stochastic process
    \begin{align}
      dX_t &= dW_t \qquad X_0 = x \notag\\
      \Rightarrow \qquad
      X_t &= W_t + x \label{q4.4}
    \end{align}
    Next set the functions
    \begin{align}
      c(x) = \frac{1}{2}x^2\qquad
      f(x) = 1
      \label{q4.5}
    \end{align}
    Therefore, the above choice of $c$, $f$, and $X_t$ in (\ref{q4.5})
    and (\ref{q4.4}) will force the solution of the PDE to match
    (\ref{q4.3}). Making this PDE explicit:
    \begin{align*}
      \text{From~\ref{q4.4}}
      \qquad
      \mathscr{L}v &= 0 \cdot \frac{\partial v}{\partial x}
        + \frac{1}{2} \cdot 1 \cdot \frac{\partial^2 v}{\partial x^2}
        = \frac{1}{2} \frac{\partial^2 v}{\partial x^2}
        \\\\
      \Rightarrow\qquad
      \frac{\partial v}{\partial t}
      &= \frac{1}{2}\left(\frac{\partial^2 v}{\partial x^2}
      - x^2 v\right)
      \qquad v(x,0) = 1
    \end{align*}

  \item % Question 5
    We want to solve for $v(x,y,t)$ given the PDE
    \begin{align*}
      v_t
      &=
      x v_x + y v_y + 2 v_{xx} + 2 v_{xy} + v_{yy}
      -v\sqrt{x^2+y^2}
      \quad\qquad v(x,y,0) = e^{-\frac{1}{2}(x^2+y^2)}
    \end{align*}
    This is possible with Feynman-Kac, thinking of $v(\mathbf{x},t)$,
    where $\mathbf{x}$ is a vector
    \begin{align*}
      \mathbf{x} = \begin{pmatrix} x \\ y \end{pmatrix}
    \end{align*}
    rather than a scalar.

    Therefore, we want to get the PDE above into the form
    \begin{align}
      \label{q5.0}
      v_t = \mathscr{L}v - cv
      \qquad v(\mathbf{x},0) = f(\mathbf{x})
    \end{align}
    To do so, start by letting
    \begin{align*}
      f(\mathbf{x},0) &= e^{-\frac{1}{2}|\mathbf{x}|} \\
      c(\mathbf{x}) &= |\mathbf{x}|^{\frac{1}{2}}
    \end{align*}
    Next, we need to find a suitable stochastic process with a generator
    that would fit (\ref{q5.0}). That means matching the coefficients on
    the derivative terms to
    \begin{align*}
      \mathscr{L}v
      &=
      b(\mathbf{x}) \nabla v
      + \frac{1}{2} a(\mathbf{x}) : \nabla^2 v
    \end{align*}
    For the drift, choose
    \begin{align*}
      b(\mathbf{x}) = \mathbf{x}
    \end{align*}
    For the diffusion term, we need
    \begin{align*}
      2v_{xx} + 2v_{xy} + v_{yy}
      &=
        \frac{1}{2}
        \begin{pmatrix}
          a_{11} & a_{12} \\
          a_{12} & a_{22} \\
        \end{pmatrix}
        :
        \begin{pmatrix}
          v_{xx} & v_{xy} \\
          v_{xy} & v_{yy} \\
        \end{pmatrix}\\
      \Leftrightarrow\qquad
      2v_{xx} + 2v_{xy} + v_{yy}
      &= \frac{1}{2}
      (a_{11} v_{xx}
      + a_{12}v_{xy})
      +\frac{1}{2}
      (a_{12} v_{xy}
      + a_{22}v_{yy})\\
      \Leftrightarrow\qquad
      4v_{xx} + 4v_{xy} + 2v_{yy}
      &=
      a_{11} v_{xx}
      + 2a_{12}v_{xy}
      + a_{22}v_{yy}
    \end{align*}
    Matching coefficients, we get
    \begin{align*}
      a = \begin{pmatrix}
        a_{11} & a_{12} \\
        a_{12} & a_{22} \\
      \end{pmatrix}
      =
      \begin{pmatrix}
        4 & 2 \\
        2 & 2 \\
      \end{pmatrix}
    \end{align*}
    Taking the Cholesky decomposition, we get
    \begin{align*}
      a = \sigma \sigma^T &=
      \begin{pmatrix}
        2 & 0 \\
        1 & 1 \\
      \end{pmatrix}
      \begin{pmatrix}
        2 & 0 \\
        1 & 1 \\
      \end{pmatrix}
      \\
      \Rightarrow \qquad
      \sigma &=
      \begin{pmatrix}
        2 & 0 \\
        1 & 1 \\
      \end{pmatrix}
    \end{align*}
    Then the solution $v(x,y,t) = v(\mathbf{x},t)$ is given by
    \begin{align*}
      v(\mathbf{x},t)
      &= \mathbb{E}^\mathbf{x}
      \left[
        e^{-\int^t_0 |\mathbf{X}_t|^{\frac{1}{2}} ds}
        \cdot e^{-\frac{1}{2}|\mathbf{X}_t|}
      \right]\\
      &= \mathbb{E}^\mathbf{x}
      \left[
        e^{-\left(\frac{1}{2}|\mathbf{X}_t|+\int^t_0 |\mathbf{X}_t|^{\frac{1}{2}} ds\right)}
      \right]
    \end{align*}
    where $X_t$ is a 2-dimensional diffusion solving
    \begin{align*}
      d\mathbf{X}_t
      &=
      \mathbf{X}_t dt
      + \begin{pmatrix}
          2 & 0 \\
          1 & 1 \\
        \end{pmatrix} d\mathbf{W}_t
    \end{align*}



\end{enumerate}

\end{document}


%%%%%%%%%%%%%%%%%%%%%%%%%%%%%%%%%%%%%%%%%%%%%%%%%%%%%%%%%%%%%%%%%%%%%%%%
%%%%%%%%%%%%%%%%%%%%%%%%%%%%%%%%%%%%%%%%%%%%%%%%%%%%%%%%%%%%%%%%%%%%%%%%
%%%%%%%%%%%%%%%%%%%%%%%%%%%%%%%%%%%%%%%%%%%%%%%%%%%%%%%%%%%%%%%%%%%%%%%%

%%%% SAMPLE CODE %%%%%%%%%%%%%%%%%%%%%%%%%%%%%%%%%%%%%%

    %% BIBLIOGRAPHIES %%

        \cite{LabelInSourcesFile}  %Use in text; cites
        \citep{LabelInSourcesFile} %Use in text; cites in parens

        \nocite{LabelInSourceFile} % Includes in refs w/o specific citation
        \bibliographystyle{apalike}  % Or some other style

        % To ditch the ``References'' header
        \begingroup
        \renewcommand{\section}[2]{}
        \endgroup

        \bibliography{sources} % where sources.bib has all the citation info

    %% SPACING %%

        \vspace{1in}
        \hspace{1in}


    %% INCLUDING PDF PAGE %%

        \includepdf{file.pdf}


    %% INCLUDING CODE %%

        \verbatiminput{file.ext}
            %---Includes verbatim text from the file
        \texttt{text}
            %---Renders text in courier, or code-like, font

        \matlabcode{file.m}
            %---Includes Matlab code with colors and line numbers


    %% INCLUDING FIGURES %%

        % Basic Figure with size scaling
            \begin{figure}[h!]
               \centering
               \includegraphics[scale=1]{file.pdf}
            \end{figure}

        % Basic Figure with specific height
            \begin{figure}[h!]
               \centering
               \includegraphics[height=5in, width=5in]{file.pdf}
            \end{figure}

        % Figure with cropping, where the order for trimming is  L, B, R, T
            \begin{figure}
               \centering
               \includegraphics[trim={1cm, 1cm, 1cm, 1cm}, clip]{file.pdf}
            \end{figure}


        % Side by Side figures
            \begin{figure}[h!]
                \centering
                \mbox{\subfigure{
                    \includegraphics[scale=1]{file1.pdf}
                }\quad\subfigure{
                    \includegraphics[scale=1]{file2.pdf}
                }
                }
            \end{figure}


