\documentclass[a4paper,12pt]{scrartcl}

\author{Matthew Cocci}
\title{Linear and Discrete Optimization}
\date{}
\usepackage{enumitem} %Has to do with enumeration	
\usepackage{amsfonts}
\usepackage{amsmath}
\usepackage{amsthm} %allows for labeling of theorems
\usepackage[T1]{fontenc}
\usepackage[utf8]{inputenc}
\usepackage{blindtext}
\usepackage{graphicx}
\usepackage{hyperref} % For internal/external linking. 
				 % [hidelinks] removes boxes
\hypersetup{
    colorlinks=true,
    linkcolor=black,
    citecolor=black,
    filecolor=black,
    urlcolor=black,
}
% \usepackage{url} % allows for url display, non-clickable
%\numberwithin{equation}{section} 
   % This labels the equations in relation to the sections 
      % rather than other equations
%\numberwithin{equation}{subsection} %This labels relative to subsections
\newtheorem{thm}{Theorem}[section]
\newtheorem{lem}[thm]{Lemma}
\newtheorem{prop}[thm]{Proposition}
\newtheorem{cor}[thm]{Corollary}
\setkomafont{disposition}{\normalfont\bfseries}
\usepackage{appendix}
\usepackage{subfigure} % For plotting multiple figures at once
\usepackage{verbatim}

\begin{document}
\maketitle

%\tableofcontents %adds it here

\section{Introduction to Linear Optimization}

\subsection{Definition of a Linear Program}

Linear optimization, also known as \emph{linear programming}, 
consists of a linear \emph{objective function} and \emph{linear 
inequalities}:
\begin{align*}
   \max \; \;& c_1 x_1 + \cdots + c_n x_n \\
   \text{s.t.} \; \;&a_{11} x_1 + \cdots + a_{1n} x_n \leq b_n \\
   & \qquad\qquad \vdots \\
   &a_{m1} x_1 + \cdots + a_{mn} x_n \leq b_m 
\end{align*}
So we want to find the values of the \emph{variables}
$x_1, \ldots, x_n \in \mathbb{R}$ such that
the objective function is maximized while still satisfying the 
linear inequalities. Using more convenient matrix notation,
our linear program above can be rewritten more compactly as
\begin{equation}
   \label{lp}
   \max \{ c^T x \; : \; x \in \mathbb{R}^n , \; Ax \leq b \} 
\end{equation}
If we want to find the minimum instead of the max, we can easily
use the fact that the minimum of some set $S$ ($\min S$) is equivalent
to $-\max S$.

\subsection{Feasability, Optimality, Boundedness}

We call a solution $x \in \mathbb{R}^n$ \emph{feasible} if $x$
satisfies all the linear inequalities.  We call a linear program
feasible if it has a feasible solution.
\\
\\
A feasible solution $s \in \mathbb{R}^n$ is an \emph{optimal} solution
of a linear program if $c^T x \geq c^T y $ for all $y\in \mathbb{R}^n$.
\\
\\
A linear program is \emph{bounded} if there exists a constant $M\in
\mathbb{R}$ such that $c^T x \leq \mathbb{R}$ for all feasible 
$x \in \mathbb{R}^n$.

\subsection{Linear Algebra Defintions}

Recall that for $A \in \mathbb{R}^{m\times n}$, the \emph{kernel}
of $A$, the \emph{image} of $A$, and the \emph{rowspace}
of $A$ are defined
\begin{align*}
   \text{ker}(A) &= \{ x \in \mathbb{R}^n \; | \; Ax =0 \} \subset
      \mathbb{R}^n\\
   \text{im}(A) &= \{ A x \; | \; x\in \mathbb{R}^n \} \subset
      \mathbb{R}^m\\
   \text{rowspace}(A) &= \{ y \in \mathbb{R}^n \; | \; \exists \lambda
   \in \mathbb{R}^m \text{ s.t. } A^T \lambda = y \}
\end{align*}
Note, the rowspace is just the set of all linear combinations of the
rows of $A$.
\\
\\
{\sl Result:}
With these definitions in hand, we can show that a solution to the 
linear program in Equation \ref{lp} is feasible and unbounded if
$b \in \text{im}(A)$ and if $c \in \text{ker} A \setminus \{ 0 \}$. 
\begin{proof}
   First, $b \in \text{im}(A)$ proves feasibility. It also implies that
   $\exists\; x^* \in \mathbb{R}^n$ such that $A x^* = b$. This
   and the fact that $c \in \text{ker}(A)\setminus \{0\}$
   allows us to write
   \begin{align*}
      A(x^* + \lambda c) &= Ax^* + \lambda A c \\
      &= b + 0 = b
   \end{align*}
   which proves $x^* + \lambda c$ is a feasible solution, satisfying
   all of the linear inequalities.
   \\\\
   So to finish up the unboundedness proof, we suppose the opposite:
   that the linear program is bounded:
      \[ c^T y \leq M \qquad \forall y \in \mathbb{R}^n \]
   Since $x^* + \lambda c$ is feasible, then it should also be bounded:
   \begin{align*}
      \Rightarrow \quad c^T (x^* + \lambda c ) &\leq M\\
      c^T x^* + \lambda c^T c \leq M 
   \end{align*}
   Now since $c \neq 0$, we know that $c^T c$ will be greater than
   0. Now let's rearrange and choose 
      \[ \lambda \geq \frac{M - c^T x^*}{c^T c} \]
   in which case $c^T ( x^* + \lambda c) > M$.
\end{proof}



\newpage
\section{The Geometry of Linear Programming}

\subsection{General Definitions}

{\sl Polyhedron:}
A set $P$ of vectors in $\mathbb{R}^n$ is a polyhedron
if $P=\{ x\in \mathbb{R}^n \; | \; Ax \leq b \}$ for some matrix $A$
and some vector $b$.  By our definition of a linear program in 
Equation \ref{lp}, we see that the set of feasible solutions is 
a polyhedron.\footnote{Just one quick note for generality: The empty
set, $\emptyset$, is also a polyhedron as $\emptyset = \{ x \in 
\mathbb{R}^n \; |\; \mathbf{0}^T x \leq -1 \}$, 
satisfying the definition of a polyhedron.} 
\\
\\
{\sl Half space:} This is defined as $\{ x \in \mathbb{R}^n
   \; | \; a^T x \leq \beta \}$, where $a \in \mathbb{R}^n \setminus 
   \mathbf{0}$.
\\\\
{\sl Hyperplane:} This is defined as $\{ x \in \mathbb{R}^n
   \; | \; a^T x = \beta \}$, where $a \in \mathbb{R}^n \setminus 
   \mathbf{0}$.
\\\\
{\sl Valid:} An inequality $a^T x \leq \beta$ is valid for a 
   polyhedron, $P$, if each $x^* \in P$ satisfies $a^T x^* \leq 
   \beta$.
\\\\
{\sl Active:} An inequality $a^T x \leq \beta$ is active 
   at $x^* \in \mathbb{R}^n$ if $a^T x^* = \beta$. 


\subsection{Vertices}

We'll offer three equivalent definitions, each of which offers some
different intuition or an additional operational advantage:
\begin{enumerate}
   \item A point $x^* \in P$ is a \emph{vertex} of $P$ if 
      there exists an inequality $a^T x \leq \beta$ such that
      \begin{enumerate}
	 \item $a^T x \leq \beta$ is valid for $P$.
	 \item $a^T x \leq \beta$ is active at $x^*$ and not active
	    at any other point in $P$.
      \end{enumerate}
   \item There's also an equivalent definition which says $x^* \in P$
      is a vertex if and only if there exists a vector $c \in 
      \mathbb{R}^n$ such that $x^*$ is the unique optimal solution of
      the linear program $\max \{ c^T x \; | \; x \in P \}$.
   \item Suppose we have $x^* \in P = \{ x \in \mathbb{R}^n \; | \;
      Ax \leq b \}$ as a vertex of $P$. 
      Then this vertex satisfies a smaller number of active contraints
      (relative to all of the constraints that define the linear 
      program and $P$). Therefore, we express those active constraints 
      that we want to enforce with $\bar{A}x \leq \bar{b}$. 
      
      Now since $x^*$ is a vertex satisfying those
      certain active constraints, that means $x^*$ is the unique 
      solution of $\bar{A} x = \bar{b}$. That's equivalent to the 
      statement that $\text{rank}(\bar{A}) = n$, which is equivalent 
      to the columns of $\bar{A}$ being linearly independent.

      So if we suspect that $x^*$ is a vertex of our linear program, 
      then we examine the subsystem, $\bar{A} x \leq \bar{b}$, 
      of our LP that $x^*$ satisfies with equality, and we look 
      at the rank of $\bar{A}$. If $x^*$ really is a vertex of $P$,
      the the rank of $\bar{A}$ equals $n$ and the columns will 
      be linearly independent.
      
\end{enumerate}


   


\end{document}



%%%% INCLUDING FIGURES %%%%%%%%%%%%%%%%%%%%%%%%%%%%

   % H indicates here 
   %\begin{figure}[h!]
   %   \centering
   %   \includegraphics[scale=1]{file.pdf}
   %\end{figure}

%   \begin{figure}[h!]
%      \centering
%      \mbox{
%	 \subfigure{
%	    \includegraphics[scale=1]{file1.pdf}
%	 }\quad
%	 \subfigure{
%	    \includegraphics[scale=1]{file2.pdf} 
%	 }
%      }
%   \end{figure}
 

%%%%% Including Code %%%%%%%%%%%%%%%%%%%%%5
% \verbatiminput{file.ext}    % Includes verbatim text from the file
% \texttt{text}	  % includes text in courier, or code-like, font
