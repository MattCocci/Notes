\documentclass[12pt]{article}

\author{Matthew D. Cocci}
\title{ECO-513: Notes}
\date{\today}

%% Formatting & Spacing %%%%%%%%%%%%%%%%%%%%%%%%%%%%%%%%%%%%

%\usepackage[top=1in, bottom=1in, left=1in, right=1in]{geometry} % most detailed page formatting control
\usepackage{fullpage} % Simpler than using the geometry package; std effect
\usepackage{setspace}
%\onehalfspacing
\usepackage{microtype}

%% Formatting %%%%%%%%%%%%%%%%%%%%%%%%%%%%%%%%%%%%%%%%%%%%%

%\usepackage[margin=1in]{geometry}
    %   Adjust the margins with geometry package
%\usepackage{pdflscape}
    %   Allows landscape pages
%\usepackage{layout}
    %   Allows plotting of picture of formatting



%% Header %%%%%%%%%%%%%%%%%%%%%%%%%%%%%%%%%%%%%%%%%%%%%%%%%

%\usepackage{fancyhdr}
%\pagestyle{fancy}
%\lhead{}
%\rhead{}
%\chead{}
%\setlength{\headheight}{15.2pt}
    %   Make the header bigger to avoid overlap

%\fancyhf{}
    %   Erase header settings

%\renewcommand{\headrulewidth}{0.3pt}
    %   Width of the line

%\setlength{\headsep}{0.2in}
    %   Distance from line to text


%% Mathematics Related %%%%%%%%%%%%%%%%%%%%%%%%%%%%%%%%%%%

\usepackage{amsmath}
\usepackage{amssymb}
\usepackage{amsfonts}
\usepackage{mathrsfs}
\usepackage{mathtools}
\usepackage{amsthm} %allows for labeling of theorems
%\numberwithin{equation}{section} % Number equations by section
\usepackage{bbm} % For bold numbers

\theoremstyle{plain}
\newtheorem{thm}{Theorem}[section]
\newtheorem{lem}[thm]{Lemma}
\newtheorem{prop}[thm]{Proposition}
\newtheorem{cor}[thm]{Corollary}

\theoremstyle{definition}
\newtheorem{defn}[thm]{Definition}
\newtheorem{assump}[thm]{Assumption}
\newtheorem{ex}[thm]{Example}

\theoremstyle{remark}
\newtheorem*{rmk}{Remark}
\newtheorem*{note}{Note}

% Below supports left-right alignment in matrices so the negative
% signs don't look bad
\makeatletter
\renewcommand*\env@matrix[1][c]{\hskip -\arraycolsep
  \let\@ifnextchar\new@ifnextchar
  \array{*\c@MaxMatrixCols #1}}
\makeatother


%% Font Choices %%%%%%%%%%%%%%%%%%%%%%%%%%%%%%%%%%%%%%%%%

\usepackage[T1]{fontenc}
\usepackage{lmodern}
\usepackage[utf8]{inputenc}
%\usepackage{blindtext}
\usepackage{courier}


%% Figures %%%%%%%%%%%%%%%%%%%%%%%%%%%%%%%%%%%%%%%%%%%%%%

\usepackage{tikz}
\usetikzlibrary{decorations.pathreplacing}
%\usetikzlibrary{arrows.meta}
\usepackage{graphicx}
\usepackage{subfigure}
    %   For plotting multiple figures at once
%\graphicspath{ {Directory/} }
    %   Set a directory for where to look for figures


%% Hyperlinks %%%%%%%%%%%%%%%%%%%%%%%%%%%%%%%%%%%%%%%%%%%%
\usepackage{hyperref}
\hypersetup{%
    colorlinks,
        %   This colors the links themselves, not boxes
    citecolor=black,
        %   Everything here and below changes link colors
    filecolor=black,
    linkcolor=black,
    urlcolor=black
}

%% Colors %%%%%%%%%%%%%%%%%%%%%%%%%%%%%%%%%%%%%%%%%%%%%%%

\usepackage{color}
\definecolor{codegreen}{RGB}{28,172,0}
\definecolor{codelilas}{RGB}{170,55,241}

% David4 color scheme
\definecolor{d4blue}{RGB}{100,191,255}
\definecolor{d4gray}{RGB}{175,175,175}
\definecolor{d4black}{RGB}{85,85,85}
\definecolor{d4orange}{RGB}{255,150,100}

%% Including Code %%%%%%%%%%%%%%%%%%%%%%%%%%%%%%%%%%%%%%%

\usepackage{verbatim}
    %   For including verbatim code from files, no colors
\usepackage{listings}
    %   For including code snippets written directly in this doc

\lstdefinestyle{bash}{%
  language=bash,%
  basicstyle=\footnotesize\ttfamily,%
  showstringspaces=false,%
  commentstyle=\color{gray},%
  keywordstyle=\color{blue},%
  xleftmargin=0.25in,%
  xrightmargin=0.25in
}
\lstdefinestyle{log}{%
  basicstyle=\scriptsize\ttfamily,%
  showstringspaces=false,%
  xleftmargin=0.25in,%
  xrightmargin=0.25in
}


\lstdefinestyle{matlab}{%
  language=Matlab,%
  basicstyle=\footnotesize\ttfamily,%
  breaklines=true,%
  morekeywords={matlab2tikz},%
  keywordstyle=\color{blue},%
  morekeywords=[2]{1}, keywordstyle=[2]{\color{black}},%
  identifierstyle=\color{black},%
  stringstyle=\color{codelilas},%
  commentstyle=\color{codegreen},%
  showstringspaces=false,%
    %   Without this there will be a symbol in
    %   the places where there is a space
  %numbers=left,%
  %numberstyle={\tiny \color{black}},%
    %   Size of the numbers
  numbersep=9pt,%
    %   Defines how far the numbers are from the text
  emph=[1]{for,end,break,switch,case},emphstyle=[1]\color{blue},%
    %   Some words to emphasise
}

\newcommand{\matlabcode}[1]{%
    \lstset{style=matlab}%
    \lstinputlisting{#1}
}
    %   For including Matlab code from .m file with colors,
    %   line numbering, etc.

\lstdefinelanguage{Julia}%
  {morekeywords={abstract,break,case,catch,const,continue,do,else,elseif,%
      end,export,false,for,function,immutable,import,importall,if,in,%
      macro,module,otherwise,quote,return,switch,true,try,type,typealias,%
      using,while},%
   sensitive=true,%
   %alsoother={$},%
   morecomment=[l]\#,%
   morecomment=[n]{\#=}{=\#},%
   morestring=[s]{"}{"},%
   morestring=[m]{'}{'},%
}[keywords,comments,strings]

\lstdefinestyle{julia}{%
    language         = Julia,
    basicstyle       = \scriptsize\ttfamily,
    keywordstyle     = \bfseries\color{blue},
    stringstyle      = \color{codegreen},
    commentstyle     = \color{codegreen},
    showstringspaces = false,
    literate         = %
      {ρ}{{$\rho$}}1
      {ℓ}{{$\ell$}}1
      {∑}{{$\Sigma$}}1
      {Σ}{{$\Sigma$}}1
      {√}{{$\sqrt{}$}}1
      {θ}{{$\theta$}}1
      {ω}{{$\omega$}}1
      {ɛ}{{$\varepsilon$}}1
      {φ}{{$\varphi$}}1
      {σ²}{{$\sigma^2$}}1
      {Φ}{{$\Phi$}}1
      {ϕ}{{$\phi$}}1
      {Dₑ}{{$D_e$}}1
      {Σ}{{$\Sigma$}}1
      {γ}{{$\gamma$}}1
      {δ}{{$\delta$}}1
      {τ}{{$\tau$}}1
      {μ}{{$\mu$}}1
      {β}{{$\beta$}}1
      {Λ}{{$\Lambda$}}1
      {λ}{{$\lambda$}}1
      {r̃}{{$\tilde{\text{r}}$}}1
      {α}{{$\alpha$}}1
      {σ}{{$\sigma$}}1
      {π}{{$\pi$}}1
      {∈}{{$\in$}}1
      {∞}{{$\infty$}}1
}


%% Bibliographies %%%%%%%%%%%%%%%%%%%%%%%%%%%%%%%%%%%%

%\usepackage{natbib}
    %---For bibliographies
%\setlength{\bibsep}{3pt} % Set how far apart bibentries are

%% Misc %%%%%%%%%%%%%%%%%%%%%%%%%%%%%%%%%%%%%%%%%%%%%%

\usepackage{enumitem}
    %   Has to do with enumeration
\usepackage{appendix}
%\usepackage{natbib}
    %   For bibliographies
\usepackage{pdfpages}
    %   For including whole pdf pages as a page in doc
\usepackage{pgffor}
    %   For easier looping


%% User Defined %%%%%%%%%%%%%%%%%%%%%%%%%%%%%%%%%%%%%%%%%%

%\newcommand{\nameofcmd}{Text to display}
\newcommand*{\Chi}{\mbox{\large$\chi$}} %big chi
    %   Bigger Chi

% In math mode, Use this instead of \munderbar, since that changes the
% font from math to regular
\makeatletter
\def\munderbar#1{\underline{\sbox\tw@{$#1$}\dp\tw@\z@\box\tw@}}
\makeatother

% Misc Math
\newcommand{\ra}{\rightarrow}
\newcommand{\diag}{\text{diag}}
\newcommand{\ch}{\text{ch}}
\newcommand{\dom}{\text{dom}}
\newcommand{\one}[1]{\mathbf{1}_{#1}}


% Command to generate new math commands:
% - Suppose you want to refer to \boldsymbol{x} as just \bsx, where 'x'
%   is any letter. This commands lets you generate \bsa, \bsb, etc.
%   without copy pasting \newcommand{\bsa}{\boldsymbol{a}} for each
%   letter individually. Instead, just include
%
%     \generate{bs}{\boldsymbol}{a,...,z}
%
% - Uses pgffor package to loop
% - Example with optional argument. Will generate \bshatx to represent
%   \boldsymbol{\hat{x}} for all letters x
%
%     \generate[\hat]{bshat}{\boldsymbol}{a,...,z}

\newcommand{\generate}[4][]{%
  % Takes 3 arguments (maybe four):
  % - 1   wrapcmd (optional, defaults to nothing)
  % - 2   newname
  % - 3   mathmacro
  % - 4   Names to loop over
  %
  % Will produce
  %
  %   \newcommand{\newnameX}{mathmacro{wrapcmd{X}}}
  %
  % for each X in argument 4

  \foreach \x in {#4}{%
    \expandafter\xdef\csname%
      #2\x%
    \endcsname%
    {\noexpand\ensuremath{\noexpand#3{\noexpand#1{\x}}}}
  }
}


% MATHSCR: Gen \sX to stand for \mathscr{X} for all upper case letters
\generate{s}{\mathscr}{A,...,Z}


% BOLDSYMBOL: Generate \bsX to stand for \boldsymbol{X}, all upper and
% lower case.
%
% Letters and greek letters
\generate{bs}{\boldsymbol}{a,...,z}
\generate{bs}{\boldsymbol}{A,...,Z}
\newcommand{\bstheta}{\boldsymbol{\theta}}
\newcommand{\bsmu}{\boldsymbol{\mu}}
\newcommand{\bsSigma}{\boldsymbol{\Sigma}}
\newcommand{\bsvarepsilon}{\boldsymbol{\varepsilon}}
\newcommand{\bsalpha}{\boldsymbol{\alpha}}
\newcommand{\bsbeta}{\boldsymbol{\beta}}
\newcommand{\bsOmega}{\boldsymbol{\Omega}}
\newcommand{\bshatOmega}{\boldsymbol{\hat{\Omega}}}
\newcommand{\bshatG}{\boldsymbol{\hat{G}}}
\newcommand{\bsgamma}{\boldsymbol{\gamma}}
\newcommand{\bslambda}{\boldsymbol{\lambda}}

% Special cases like \bshatb for \boldsymbol{\hat{b}}
\generate[\hat]{bshat}{\boldsymbol}{b,y,x,X,V,S,W}
\newcommand{\bshatbeta}{\boldsymbol{\hat{\beta}}}
\newcommand{\bshatmu}{\boldsymbol{\hat{\mu}}}
\newcommand{\bshattheta}{\boldsymbol{\hat{\theta}}}
\newcommand{\bshatSigma}{\boldsymbol{\hat{\Sigma}}}
\newcommand{\bstildebeta}{\boldsymbol{\tilde{\beta}}}
\newcommand{\bstildetheta}{\boldsymbol{\tilde{\theta}}}
\newcommand{\bsbarbeta}{\boldsymbol{\overline{\beta}}}
\newcommand{\bsbarg}{\boldsymbol{\overline{g}}}

% Redefine \bso to be the zero vector
\renewcommand{\bso}{\boldsymbol{0}}

% Transposes of all the boldsymbol shit
\newcommand{\bsbp}{\boldsymbol{b'}}
\newcommand{\bshatbp}{\boldsymbol{\hat{b'}}}
\newcommand{\bsdp}{\boldsymbol{d'}}
\newcommand{\bsgp}{\boldsymbol{g'}}
\newcommand{\bsGp}{\boldsymbol{G'}}
\newcommand{\bshp}{\boldsymbol{h'}}
\newcommand{\bsSp}{\boldsymbol{S'}}
\newcommand{\bsup}{\boldsymbol{u'}}
\newcommand{\bsxp}{\boldsymbol{x'}}
\newcommand{\bsyp}{\boldsymbol{y'}}
\newcommand{\bsthetap}{\boldsymbol{\theta'}}
\newcommand{\bsmup}{\boldsymbol{\mu'}}
\newcommand{\bsSigmap}{\boldsymbol{\Sigma'}}
\newcommand{\bshatmup}{\boldsymbol{\hat{\mu'}}}
\newcommand{\bshatSigmap}{\boldsymbol{\hat{\Sigma'}}}

% MATHCAL: Gen \calX to stand for \mathcal{X}, all upper case
\generate{cal}{\mathcal}{A,...,Z}

% MATHBB: Gen \X to stand for \mathbb{X} for some upper case
\generate{}{\mathbb}{R,Q,C,Z,N,Z,E}
\newcommand{\Rn}{\mathbb{R}^n}
\newcommand{\RN}{\mathbb{R}^N}
\newcommand{\Rk}{\mathbb{R}^k}
\newcommand{\RK}{\mathbb{R}^K}
\newcommand{\RL}{\mathbb{R}^L}
\newcommand{\Rl}{\mathbb{R}^\ell}
\newcommand{\Rm}{\mathbb{R}^m}
\newcommand{\Rnn}{\mathbb{R}^{n\times n}}
\newcommand{\Rmn}{\mathbb{R}^{m\times n}}
\newcommand{\Rnm}{\mathbb{R}^{n\times m}}
\newcommand{\Rkn}{\mathbb{R}^{k\times n}}
\newcommand{\Cn}{\mathbb{C}^n}
\newcommand{\Cnn}{\mathbb{C}^{n\times n}}

% Dot over
\newcommand{\dx}{\dot{x}}
\newcommand{\ddx}{\ddot{x}}
\newcommand{\dy}{\dot{y}}
\newcommand{\ddy}{\ddot{y}}

% First derivatives
\newcommand{\dydx}{\frac{dy}{dx}}
\newcommand{\dfdx}{\frac{df}{dx}}
\newcommand{\dfdy}{\frac{df}{dy}}
\newcommand{\dfdz}{\frac{df}{dz}}

% Second derivatives
\newcommand{\ddyddx}{\frac{d^2y}{dx^2}}
\newcommand{\ddydxdy}{\frac{d^2y}{dx dy}}
\newcommand{\ddydydx}{\frac{d^2y}{dy dx}}
\newcommand{\ddfddx}{\frac{d^2f}{dx^2}}
\newcommand{\ddfddy}{\frac{d^2f}{dy^2}}
\newcommand{\ddfddz}{\frac{d^2f}{dz^2}}
\newcommand{\ddfdxdy}{\frac{d^2f}{dx dy}}
\newcommand{\ddfdydx}{\frac{d^2f}{dy dx}}


% First Partial Derivatives
\newcommand{\pypx}{\frac{\partial y}{\partial x}}
\newcommand{\pfpx}{\frac{\partial f}{\partial x}}
\newcommand{\pfpy}{\frac{\partial f}{\partial y}}
\newcommand{\pfpz}{\frac{\partial f}{\partial z}}


% argmin and argmax
\DeclareMathOperator*{\argmin}{arg\;min}
\DeclareMathOperator*{\argmax}{arg\;max}


% Various probability and statistics commands
\newcommand{\iid}{\overset{iid}{\sim}}
\newcommand{\vc}{\operatorname{vec}}
\newcommand{\Cov}{\operatorname{Cov}}
\newcommand{\rank}{\operatorname{rank}}
\newcommand{\trace}{\operatorname{tr}}
\newcommand{\Corr}{\operatorname{Corr}}
\newcommand{\Var}{\operatorname{Var}}
\newcommand{\asto}{\xrightarrow{a.s.}}
\newcommand{\pto}{\xrightarrow{P}}
\newcommand{\uto}{\xrightarrow{u}}
\newcommand{\msto}{\xrightarrow{m.s.}}
\newcommand{\dto}{\xrightarrow{d}}
\newcommand{\Lpto}{\xrightarrow{L_p}}
\newcommand{\Lqto}[1]{\xrightarrow{L_{#1}}}
\newcommand{\plim}{\text{plim}_{n\rightarrow\infty}}


% Redefine real and imaginary from fraktur to plain text
\renewcommand{\Re}{\operatorname{Re}}
\renewcommand{\Im}{\operatorname{Im}}

% Shorter sums: ``Sum from X to Y''
% - sumXY  is equivalent to \sum^Y_{X=1}
% - sumXYz is equivalent to \sum^Y_{X=0}
\newcommand{\sumnN}{\sum^N_{n=1}}
\newcommand{\sumin}{\sum^n_{i=1}}
\newcommand{\sumjn}{\sum^n_{j=1}}
\newcommand{\sumim}{\sum^m_{i=1}}
\newcommand{\sumik}{\sum^k_{i=1}}
\newcommand{\sumiN}{\sum^N_{i=1}}
\newcommand{\sumkn}{\sum^n_{k=1}}
\newcommand{\sumtT}{\sum^T_{t=1}}
\newcommand{\sumninf}{\sum^\infty_{n=1}}
\newcommand{\sumtinf}{\sum^\infty_{t=1}}
\newcommand{\sumnNz}{\sum^N_{n=0}}
\newcommand{\suminz}{\sum^n_{i=0}}
\newcommand{\sumknz}{\sum^n_{k=0}}
\newcommand{\sumtTz}{\sum^T_{t=0}}
\newcommand{\sumninfz}{\sum^\infty_{n=0}}
\newcommand{\sumtinfz}{\sum^\infty_{t=0}}

\newcommand{\prodnN}{\prod^N_{n=1}}
\newcommand{\prodin}{\prod^n_{i=1}}
\newcommand{\prodiN}{\prod^N_{i=1}}
\newcommand{\prodkn}{\prod^n_{k=1}}
\newcommand{\prodtT}{\prod^T_{t=1}}
\newcommand{\prodnNz}{\prod^N_{n=0}}
\newcommand{\prodinz}{\prod^n_{i=0}}
\newcommand{\prodknz}{\prod^n_{k=0}}
\newcommand{\prodtTz}{\prod^T_{t=0}}

% Bounds
\newcommand{\atob}{_a^b}
\newcommand{\ztoinf}{_0^\infty}
\newcommand{\kinf}{_{k=1}^\infty}
\newcommand{\ninf}{_{n=1}^\infty}
\newcommand{\minf}{_{m=1}^\infty}
\newcommand{\tinf}{_{t=1}^\infty}
\newcommand{\nN}{_{n=1}^N}
\newcommand{\tT}{_{t=1}^T}
\newcommand{\kinfz}{_{k=0}^\infty}
\newcommand{\ninfz}{_{n=0}^\infty}
\newcommand{\minfz}{_{m=0}^\infty}
\newcommand{\tinfz}{_{t=0}^\infty}
\newcommand{\nNz}{_{n=0}^N}

% Limits
\newcommand{\limN}{\lim_{N\rightarrow\infty}}
\newcommand{\limn}{\lim_{n\rightarrow\infty}}
\newcommand{\limk}{\lim_{k\rightarrow\infty}}
\newcommand{\limt}{\lim_{t\rightarrow\infty}}
\newcommand{\limT}{\lim_{T\rightarrow\infty}}
\newcommand{\limhz}{\lim_{h\rightarrow 0}}

% Shorter integrals: ``Integral from X to Y''
% - intXY is equivalent to \int^Y_X
\newcommand{\intab}{\int_a^b}
\newcommand{\intzN}{\int_0^N}

% Check and x symbols
\usepackage{pifont}
\newcommand{\cmark}{\text{\ding{51}}}
\newcommand{\xmark}{\text{\ding{55}}}

%%%%%%%%%%%%%%%%%%%%%%%%%%%%%%%%%%%%%%%%%%%%%%%%%%%%%%%%%%%%%%%%%%%%%%%%
%% BODY %%%%%%%%%%%%%%%%%%%%%%%%%%%%%%%%%%%%%%%%%%%%%%%%%%%%%%%%%%%%%%%%
%%%%%%%%%%%%%%%%%%%%%%%%%%%%%%%%%%%%%%%%%%%%%%%%%%%%%%%%%%%%%%%%%%%%%%%%


\begin{document}
\maketitle

\tableofcontents


\clearpage
\section{Causality and Exogeneity}

Model
\begin{align*}
  y_n
  = \bsx_n'\bsbeta + \varepsilon_t
\end{align*}
Note that $\bsx_n$ could potentially contained lagged $y_n$. But that
limits whether or not a particular assumption is reasonable for time
series.
\begin{table}[htbp!]
\centering
\begin{tabular}{c|lcccc}
    &
    & Implies
    & Justifies
    & Consistency \&
    & Reasonable for
  \\
  $\bsx_t$ is\dots
    & Defining Feature
    & $\E[\varepsilon_t\varepsilon_{s}]=0$
    & GLS
    & Asymptotic $\calN$
    & Time Series
  \\\hline\hline
  Strictly Exogenous
    & $0=\E[\varepsilon_t|\{\bsx_s\}_{s=-\infty}^\infty]$
    & \xmark
    & \cmark
    & \cmark
    & \xmark
  \\
  Predetermined
    %& $0=\E[\varepsilon_t|\{\bsx_s,y_{s-1}\}_{s=-\infty}^t]$
    & $0=\E[\varepsilon_t|\{\bsx_s\}_{s=-\infty}^t]$
    & \cmark
    & \xmark
    & \cmark
    & \cmark
  \\
  Weaker Predetermined
    & $0=\E[\varepsilon_t\bsx_t]$
    & \xmark
    & \xmark
    & \cmark
    & \cmark
\end{tabular}
\end{table}

Weaker predetermined is used in Hayashi.

Strict exogeneity Generally impossible when $\bsx_n$ has lagged $y_n$ in
it.  Hence, generally not imposed in models with AR structure.
Mostly imposed when $y_n$ and $\bsx_n$ completely distinct

Predetermined and exogeneity are different. One does not imply the
other.
But if already assumping absence of serial correlation, then strict
exogeneity is stronger and does imply predetermined.
But no need to assume that for strict exogeneity.





\clearpage
\section{Asymptotics}

Asymptotic analysis is useful as a general \emph{framework} or
\emph{device} with two applications:
\begin{itemize}
  \item Finding/deriving approximate distributions, tests,
    test statistics, confidence regions
  \item \emph{Given} tests, test statistics, confidence regions,
    we can evalute their asymptotic optimality, efficiency, or quality,
    in an asymptotic variance or asymptotic power sense
\end{itemize}
This is carried by starting with likelihood/model $p(y_{1:N}|\theta)$ fo
the sample $y_{1:N}$, taking $N\ra\infty$, and invoking the LLN and CLT
to get to get the asymptotic distribution of some function $S(y_{1:N})$
of the data, whether that function is an estimator, a test, a confidence
region, etc.

The magic lies in the fact that, while the finite sample distribution of
$y_{1:N}$ is often complicated or intractable, the LLN and CLT
drastically simplify the asymptotic distribution of $S(y_{1:N})$ into
something normal. Asymptotic normality of estimators, test statistics,
and even models (in the limits of experiments literature) are a hallmark
of asymptotic statistics.

Note that all of htis assumes the asymptotic distribution is a good
approximation of the finite sample distribution.
There's a priori no reason this should be the case, and asymptotics does
not explicitly imply bounds on the error or mistake. But can do
simulation studies to test whether it's a good approximation.


\clearpage
\section{Contiguity and Limits of Experiments}

\subsection{Experiments and Sequence of Experiments}

\begin{defn}(Model/Experiment)
A \emph{model} or \emph{experiment} is the triple
$(\Omega,\sF,\{P_\theta\}_{\theta\in  \Theta})$ consisting of a measurable space
$(\Omega,\sF)$ together with a collection of probability measures for
that measurable space $\{P_\theta\}_{\theta\in  \Theta}$.
Often though, we don't even specify the measurable space $(\Omega,\sF)$
since it's either obvious or unimportant, and we just refer to
$\{P_\theta\}_{\theta\in  \Theta}$ as the model/experiment. That's where
all the action is anyway.

In particular, $\{P_\theta\}_{\theta\in  \Theta}$ represents the collection of all
possible distributions for some RV $X$.
Fixing $\theta$, we can make ex ante probability statements about the
distribution of $X$. Or, given an observation $X$, we can do inference
about which distribution $P_\theta$ most likely produced that particular
realization.
Most generally, $ \Theta$ represents an arbitrary index set.
However in practice, we often consider distributions $P_\theta$ in some
\emph{parameteric} family that are absolutely continuous with respect to
the Lebesgue or counting measure, hence they permit a density or mass
function of the form $p(y|\theta)$ (aka a likelihood) with $\theta$ representing a
parameter vector (rather than just an arbitrary index).
\end{defn}
\begin{rmk}
From now on, I will use the customary phrase ``experiment'' rather than
``model.''
\end{rmk}

\begin{defn}(Sequence of Experiments)
A \emph{sequence of experiments} is the sequence
\begin{align*}
  \big\{\Omega_n,\sF_n,\{P_{n,\theta}\}_{\theta\in  \Theta}
  \big\}_{n\in \N}
\end{align*}
Each element is a measurable space $(\Omega_n,\sF_n)$ and
collection of candidate measures $\{P_{n,\theta}\}_{\theta\in  \Theta}$.

Most often, such a sequence arises in thinking about an iid sample
$\{X_i\}_{i=1}^n$, where each observation $X_i$ has distribution $P_\theta$
on $(\Omega,\sF)$.
Then the $n$th element in the sequence of experiments is the
triplet $(\Omega_n,\sF_n,\{P_{n,\theta}\}_{\theta\in  \Theta})$ where
$(\Omega_n,\sF_n)$ is the product space and $P_{n,\theta}$ the product
measure for the sample $\{X_i\}_{i=1}^n$, built from $n$ copies of
underlying $P_\theta$.
\end{defn}


\clearpage
\subsection{Contiguity}

\begin{defn}(Contiguity)
Suppose we have sequence of experiments
\begin{align*}
  \big\{\Omega_n,\sF_n,\{P_n,Q_n\}\big\}_{n\in\N}
\end{align*}
Sequence of measures $Q_n$ is \emph{contiguous} to $P_n$, denoted $Q_n
\vartriangleleft P_n$, if and only if
\begin{align*}
  P_n[A_n]\ra 0
  \quad\implies\quad
  Q_n[A_n]\ra 0
\end{align*}
This generalizes absolute continuity of measures to \emph{sequences} of
measures.
Contiguity is useful because it allows us to transfer convergence in
probability or distribution under the baseline experiment with $P_n$ to
the contiguous experiment with $Q_n$. Often the baseline experiment with
$P_n$ is some well-behaved experiment we know a lot about, while $Q_n$
is a trickier experiment that's we'd \emph{like} to know more about.

Correct image: Sequence of measures ``on top of each other'' in the
limit, can't distinguish.
%The set of contiguous models are often idexed by the rate of approach,
%i.e. $\theta_0+gT^{-1/2}$ contiguous for \emph{all} $g\in\R$.
\end{defn}

\begin{thm}\emph{(LeCam's First Lemma: Establishing Contiguity)}
\label{thm:lecams1st}
The following are equivalent
\begin{enumerate}[label=\emph{(\roman*)}]
  \item $Q_n \vartriangleleft P_n$, i.e. $Q_n$ contiguous to $P_n$
  %\item $\frac{dP_n}{dQ_n}\dto_{Q_n} U$ with $P[U>0]=1$
  \item $\frac{dQ_n}{dP_n}\underset{P_n}{\dto} L$ with
    $\E[L]=1$.\footnote{%
      We are implicitly assuming $Q_n<<P_n$ for all $n$ so
      $\frac{dQ_n}{dP_n}$ is well-defined
    }
  \item For any sequence of RVs $X_n:\Omega_n\ra\Rk$,
    $X_n = o_{P_n}(1)$
    implies
    $X_n = o_{Q_n}(1)$
\end{enumerate}
\end{thm}
\begin{rmk}
Often, we use (ii) to establish (i).
Intuition: If the limiting distribution of the sequence of likelihood
ratios behaves like a Radon-Nikodym/likelihood-ratio, no mass has
escaped to infinity and the experiments are close even in the limit.
\end{rmk}

\begin{thm}
\label{thm:lecams3rd}
\emph{(Generalized LeCam's Third Lemma)}
%: Deriving Asymptotic
%Distribution of the Contiguous Model from Baseline Model)}
Suppose we have a sequence of experiments with measures $P_n,Q_n$ and
RV $X_n$ (generally some test, statistic, etc.) such that $Q_n<<P_n$ for
all $n$ and
\begin{align*}
  \left(\frac{dQ_n}{dP_n},X_n\right)
  \quad\underset{P_n}{\dto}\quad
  (L,X)
  \qquad \text{where}\quad
  \E[L]=1
\end{align*}
Then by the previous theorem, $Q_n\vartriangleleft P_n$ and,
additionally
\begin{align}
  X_n\;\underset{Q_n}{\dto}\;\tilde{X}
  \qquad \text{where}\quad
  P_{\tilde{X}}\big[\tilde{X}\in B\big]
  = \E_{\tilde{X}}\big[\mathbf{1}\{\tilde{X}\in B\}\big]
  =\E_X\big[L\cdot \mathbf{1}\{X\in B\}\big]
  \label{lecams3rd}
\end{align}
\end{thm}
\begin{rmk}
Given $Q_n\vartriangleleft P_n$, Part (iii) of Lecam's First Lemma
allows us to transfer convergence in \emph{probability} under $P_n$ to
$Q_n$.
Lecam's Third Lemma similarly lets us transfer convergence in
\emph{distribution}, and also fully specifies the distribution under
$Q_n$ as in Expression~\ref{lecams3rd}.
\end{rmk}

%\begin{cor}\emph{(Specialized LeCam's Third Lemma)}
%Suppose that we have
%\begin{align*}
  %\left(X_n,\ln \frac{dQ_n}{dP_n}\right)
  %\dto_{P_n}
  %\calN\left(
  %\begin{pmatrix}
    %\mu \\ -\frac{1}{2}\sigma^2
  %\end{pmatrix},\;
  %\begin{pmatrix}
    %\Sigma & \tau  \\
    %\tau' & \sigma^2
  %\end{pmatrix}
  %\right)
%\end{align*}
%The all the conditions of teh above theorem are satisfied, hence
%$Q_n\vartriangleleft P_n$ and
%\begin{align*}
  %X_n
  %\dto_{Q_n}
  %\calN(\mu+\tau, \Sigma)
%\end{align*}
%Note, asymptotic covariance of $X_n$ is the same under $Q_n$ but mean
%shifted.
%\end{cor}

%\begin{ex}
%Suppose we have
%\begin{align}
  %\frac{dQ_n}{dP_n}\dto_{P_n}
  %e^{\calN(\mu,\sigma^2)}
  %\label{lanintro}
%\end{align}
%Then already, $P_n\vartriangleleft Q_n$ since (ii) is equivalent to (i).
%Moreover, $Q_n\vartriangleleft P_n$ iff $\mu=-\frac{1}{2}\sigma^2$ since
%(iii) is equivalent to (i).
%This is important because Assumption~\ref{lanintro} means that the model
%is locally asymptotically normal, which is a special concept we will
%define and study below.
%\end{ex}

%\begin{proof}
%(Theorem~\ref{thm:lecams1st})
%\end{proof}



\clearpage
\subsection{Limits of Experiments}


\begin{defn}(Convergence to Limit Experiment)
\label{defn:limitexp}
We say that the sequence of experiments
$\{\Omega_n,\sF_n,\{P_{n,g}\}_{g\in G}\}_{n\in\N}$
\emph{converges to limit experiment}
$(\Omega,\sF,\{P_{g}\}_{g\in G})$ if,
for every possible baseline parameter value $g_0\in G$ and
finite subset $S\subseteq  G$,
\begin{align}
  \left(
  \frac{dP_{n,g}}{dP_{n,g_0}}(X_n)
  \right)_{g\in S}
  \quad\underset{g_0}{\dto}\quad
  \left(
  \frac{dP_{g}}{dP_{g_0}}(X)
  \right)_{g\in S}
  \label{limexp}
\end{align}
In words, this is ``Joint convergence in distribution
(taking the \emph{baseline} law as that with parameter $g_0$)
of the \emph{random vector} where each element is a likelihood ratio
with a different $g\in S$.''
The likelihood ratios have arguments $X_n$ and $X$ to remind that the
LRs are themselves RVs whose distributions depend on the realization of
the observation $X_n$ (in the $n$th experiment in the sequence) and $X$
(in the limit experiment).
If Expression~\ref{limexp} holds, then observing $X_n$ in the $n$th
experiment in the sequence is just like (i.e. has the same asymptotic
properties as) observing $X$ in the limit experiment.

We most often use such convergence to \emph{approximate} the
\emph{sequence} of experiments with the \emph{limiting} experiment.
Intuitively, since an experiment with large $n$ has a LR whose
distribution behaves like the LR in the limit experiment, we might
reasonably think the limit experiment provides a good approximation for
experiments in the converging sequence.
This is especially common when $n$ represents the sample size, and each
experiment in the sequence is the joint distribution for an increasingly
large sample.
This also directly generalizes to \emph{entire models/experiments} the
common practice of approximating a sequence of estimators or tests with
their asymptotic limit.
\end{defn}

\begin{prop}\emph{(Limit Experiment Implies Contiguity)}
\label{prop:impcontiguity}
Suppose the sequence of experiments
$\{\Omega_n,\sF_n,\{P_{n,g}\}_{g\in G}\}_{n\in\N}$
converges to limit experiment $(\Omega,\sF,\{P_{g}\}_{g\in G})$
as in Definition~\ref{defn:limitexp}.
Then for any $g,g_0\in G$,
sequence $P_{n,g}$ is contiguous to $P_{n,g_0}$.
\end{prop}
\begin{proof}
By assumption, Expression~\ref{limexp} holds for arbitrary $g,g_0$.
Since $L:=\frac{dP_g}{dP_{g_0}}$ is a RN derivative between $P_{g}$ and
$P_{g_0}$ in the limit experiment, we know it has expectation $\E[L]=1$.
So then, since Part (ii) of Theorem~\ref{thm:lecams1st} is equivalent to
Part (i), $P_{n,g}\vartriangleleft P_{n,g_0}$.
\end{proof}

\begin{thm}
\label{thm:limitexp}
Suppose sequence of experiments
$\{\Omega_n,\sF_n,\{P_{n,g}\}_{g\in G}\}_{n\in\N}$
converges to limit experiment
$(\Omega,\sF,\{P_{g}\}_{g\in G})$.
Suppose also that sequence of statistics $T_n=T_n(X_n)$ converges in
distribution under any $g$ to \emph{some} (generally $g$-specific) RV on
some probability space, i.e.
\begin{align*}
  T_n \quad\underset{g}{\dto}\quad T_g
  \qquad\forall g
\end{align*}
Then we can represent $T_g$ as a (possibly randomized) function of the
observation $X$ in the limit experiment, i.e.
%Then there exists a (possibly randomized) statistic $T$
%\emph{in the limit experiment}
\begin{align*}
  T_n \underset{g}{\dto} T(X,U)
  \qquad
  \forall g
\end{align*}
\end{thm}
%\begin{rmk}
%The theorem statement might seem a little redundant regarding
%convergence in distribution. But note the subtlety: We only assume $T_n$
%converges in distribution to \emph{some} RV in \emph{some} probability
%space that we don't necessarily care about. Then this theorem guarantees
%that $T_n$ \emph{also} converges to an RV that is a \emph{function} of
%$X$ in the \emph{particular} probability space
%$(\Omega,\sF,\{P_g\}_{g\in G})$ that is the limit
%experiment.
%\end{rmk}

\begin{proof}
(\emph{Establish Joint Convergence under $g_0$})
By assumption, we have \emph{individually}
\begin{align}
  \frac{dP_{n,g}}{dP_{n,g_0}}
  \quad\underset{g_0}{\dto}\quad
  \frac{dP_{g}}{dP_{g_0}}
  \qquad\text{and}\qquad
  T_n
  \quad\underset{g_0}{\dto}\quad
  T_{g_0}
  \label{limitmarginals}
\end{align}
We want to establish \emph{joint} convergence of these objects under
$g_0$, in addition to this (weaker) convergence of the marginals.

To start, by Part (i) of Prohorov's Theorem, since
$\frac{dP_{n,g}}{dP_{n,g_0}}$ and $T_n$ converge in distribution, both
are uniformly tight under $g_0$ individually.
But by the definition of uniform tightness, it's clear that if a finite
number of RVs are uniformly tight individually, the \emph{joint} vector
$(\frac{dP_{n,g}}{dP_{n,g_0}},T_n)$ is \emph{also} uniformly tight under
$g_0$.
And then, by Part (ii) Prohorov's Theorem, uniform tighness of the
vector sequence implies that there exists a subsequence that converges
in distribution under $g_0$, i.e. there exists a set of indices
$\{n_i\}_{i=1}^\infty$ such that
\begin{align*}
  \left(\frac{dP_{n_i,g}}{dP_{n_i,g_0}},T_{n_i}\right)
  \quad\underset{g_0}{\dto}\quad
  \left(Y,Z\right)
\end{align*}
for some RVs $Y,Z$. But we can pin down those RVs. In particular, since
the marginals of limiting RV $(Y,Z)$ must match the marginals we already
know/assumed in Expressions~\ref{limitmarginals}, we can deduce that, in
fact, the subsequence converges to
\begin{align}
  \left(\frac{dP_{n_i,g}}{dP_{n_i,g_0}},T_{n_i}\right)
  \quad\underset{g_0}{\dto}\quad
  \left(\frac{dP_g}{dP_{g_0}},T_{g_0}\right)
  \label{lecams3rdstep}
\end{align}
Again, this is just along the subsequence, but we will move to the
general limit in a bit.

(\emph{Derive Distribution Under $g$})
Now that we have joint convergence under $g_0$, we want the distribution
of the statistic under $g$.
By Proposition~\ref{prop:impcontiguity},
$P_{n_i,g}\vartriangleleft P_{n_i,g_0}$, i.e. the subsequence of
measures $\{P_{n_i,g}\}_{i=1}^\infty$ is contiguous to
$\{P_{n_i,g_0}\}_{i=1}^\infty$.
Therefore, we can use LeCam's Third Lemma in Theorem~\ref{thm:lecams3rd}
to get the limiting distribution under $g$.
In particular, we have
\begin{align}
  T_{n_i}\;\underset{g}{\dto}\;\tilde{T}_g
  \qquad \text{where}\quad
  P_{g}[\tilde{T}_g\in B]
  = \E_{g}[\mathbf{1}\{\tilde{T}_g\in B\}]
  =\E_{g_0}\left[\frac{dP_g}{dP_{g_0}}\cdot \mathbf{1}\{T_{g_0}\in B\}\right]
  \label{tildeT}
\end{align}
This is the limiting distribution of the subsequence
$\{T_{n_i}\}_{i=1}^\infty$. But because the original sequence $T_n$
converges in distribution to $T_g$ (for any $g$) by assumption, this
limit of the subsequence must equal the limit of the sequence, i.e.
$\tilde{T}_g=T_g$. Hence Expression~\ref{tildeT} also characterizes the
distribution of the limit $T_g$ of the original sequence $T_n$.

(\emph{Construct Function $T(X,U)$})
Finally, we can always choose a function $T$ so that
\begin{align*}
  \begin{pmatrix}
    T(X,U) \\ X
  \end{pmatrix}
  \sim
  \begin{pmatrix}
    T_{g_0} \\ X
  \end{pmatrix}
\end{align*}
Check that this has the correct distribution under $g$ by importance
sampling.


\end{proof}



%\subsubsection{Local Asymptotic Normality}

%\begin{defn}(Local Asymptotic Normality)
%Model is locally asymptotical normal if
%\begin{align*}
  %\ln
  %\frac{p(y_{1:T}|\theta_0+gT^{-1/2})}{p(y_{1:T}|\theta_0)}
  %\dto
  %\calN\left(-\frac{1}{2}g'\calI g, g'\calI g\right)
%\end{align*}
%Just like small sample behavior of log like from model
%\begin{align*}
  %Y \sim \calN(\calI^{1/2}g,I_k)
%\end{align*}
%\end{defn}



\clearpage
\section{Limits of Experiments}

\begin{rmk}
As $N\ra\infty$, an unknown parameter $\theta$ will eventually be known
with arbitrary precision, and any sensible test of the parameter value
will reject an incorrect null with probability one.
So how do we compare tests that, asymptotically, all have the same power
(i.e. reject for sure)?
Well, we look at the likelihood ratio (which forms the basis for
sensible tests) and consider how it behaves when the alternative
approaches the null at an appropriate rate so that, asymptotically, the
null and alternative remain tough to distinguish.

We do this be reparameterizing the model/experiment.
If we do it correctly, then the LR will converge to the LR ratio of a
single observation Gaussian model/experiment.
Consider a known $\theta_0$, and define local parameter
$\sqrt{N}(\theta-\theta_0)$, which implies that we're considering
$\theta=\theta_0+\frac{g}{\sqrt{N}}$.
Note that $h$ is the wedge between the (properly scaled) true parameter
$\theta_0$ and $\theta$.
For large $N$, the following models/experiments have similar statistical
properties
\begin{align*}
  \{P_{\theta_0+g/\sqrt{N}}\}_{h\in H}
  \qquad
  \{\calN(h,\calI_{\theta_0}^{-1})\}_{h\in H}
\end{align*}
The log-likelihood ratio of this model has the same distribution as the
limit distribution of the above LAN model.
\end{rmk}


\begin{defn}(Log-Likelihood Approximation)
Suppose $p_\theta$ is a density for $P_\theta$ with respect to some
measure $\mu$. Suppose the density is differentiable,
and we take a first order expansion of the log likelihood ratio of the
model with $\theta+\frac{g}{\sqrt{N}}$ against the model with $\theta$
(about $\theta$):
\begin{align*}
  \ln\frac{p(y_{1:N}|\theta+\frac{g}{\sqrt{N}})}{p(y_{1:N}|\theta)}
  =
  \frac{g}{\sqrt{N}}
  \sumnN s_n(\theta)
  +
  \frac{1}{2}\frac{g^2}{N}
  \sumnN
  h_n(\theta)
\end{align*}
Need to suppose the log likelihood permits a forecast error
decomposition.
Under certain conditions
\begin{align*}
  \frac{g}{\sqrt{N}}
  \sumnN s_n(\theta)
  +
  \frac{1}{2}\frac{g^2}{N}
  \sumnN
  h_n(\theta)
  \quad\dto\quad
  \calN(0,\calI)
  -\frac{g^2}{2}
  \calI
\end{align*}
\end{defn}


\clearpage
\section{Nonparametric Estimation}

%Estimate CDF as a step function, and density from that step function

\subsection{Kernel Density Estimation}

\begin{defn}(Univariate Second Order Kernel)
A univariate \emph{second order kernel} $K(u)$ is a real-valued
weighting function $K:\R\ra\R_+$ satisfying:
\begin{enumerate}[label=(\roman*)]
  \item (\emph{Density}): $K(u)\geq 0$ and
    $1=\int_\R K(u)\;du$
  \item (\emph{Symmetric}): $K(u)=K(-u)$ for all $u\in\R$
  \item (\emph{Finite Variance}): To calculate second moments from
    the kernel density estimate, need
    \begin{align*}
      \int u^2 K(u)\;du =: k_2 \in (0,\infty)
    \end{align*}
\end{enumerate}
\end{defn}

\begin{defn}(Kernel Density Estimator)
Suppose we have an iid sample $\{X_n\}\nN$ where each observation has
density $p(x)$.
The \emph{kernel density estimator} $\hat{p}(x)$ of density $p$ at point
$x$ in the support (using kernel $K$ and bandwith $h$) is given by
\begin{align}
  \hat{p}(x)
  =
  \frac{1}{N}
  \sumnN
  \frac{1}{h}
  K\left(
  \frac{x-X_n}{h}
  \right)
  \label{kernel}
\end{align}
Note this estimator is an \emph{average} of iid
$\frac{1}{h} K\left( \frac{x-X_n}{h} \right)$ terms, which will be
useful for deriving the mean and variance of this estimator since
the mean of $\hat{p}(x)$ will be the mean of any individual term, while
the variance will be $1/N$ times the variance of any individual term.
\end{defn}

\begin{prop}\emph{(Mean and Variance of Kernel Density Estimator)}
Suppose we have an iid sample $X_n$ with density function $p(x)$, which
permits a second order Taylor Expansion so that we can write
\begin{align}
  p(x+hu)
  =
  p(x)
  + p'(x) hu
  + \frac{1}{2}p''(x)h^2u^2
  + O(h^3)
  \label{kerneltaylor1}
\end{align}
Then the kernel density estimator $\hat{p}(x)$ of density $p$ at point
$x$ has mean and variance
\begin{alignat}{3}
  \E[\hat{p}(x)]
  &=
  p(x)
  +
  \frac{1}{2}
  h^2
  p''(x)
  \int
  u^2
  K(u)
  \;du
  + O(h^3)
  &&=
  p(x) + O(h^2)
  \label{kernelmean}
  \\
  \Var(\hat{p}(x))
  &=
  \frac{1}{Nh}
  p(x)
  \int
  K(u)^2
  \,
  du
  + O(N^{-1})
  &&=
  O((Nh)^{-1})
  \label{kernelvar}
\end{alignat}
\end{prop}
\begin{rmk}
First notice that $\hat{p}(x)$ is a \emph{biased} estimator of $p(x)$
with the bias of size $h^2$.
Second notice that the variance of the estimator goes to zero at rate
$Nh$ (which we can think of as the ``effective sample size'') rather
than the usual rate $N$.
Third notice the bias-variance tradeoff.
Taking $h\ra 0$ decreases bias in Expression~\ref{kernelmean} but
blows up the variance in Expression~\ref{kernelvar}.
To make $\hat{p}(x)$ a consistent estimator, we need to think about
sending $h\ra 0$ at some appropriate rate as $N$ grows so that both the
bias and the variance shrink to zero as $N\ra\infty$.
\end{rmk}

\clearpage
\begin{proof}
(\emph{Mean})
Because the estimator $\hat{p}(x)$ in Definition~\ref{kernel} is an
average of iid terms, its mean is the mean of any individual term, as
remarked:
\begin{align}
  \E[\hat{p}(x)]
  =
  \E\left[
  \frac{1}{h}
  K\left(
  \frac{x-X_n}{h}
  \right)
  \right]
  =
  \frac{1}{h}
  \int
  K\left(
  \frac{x-v}{h}
  \right)
  \;
  p(v)
  \;dv
  \label{kernelmeanstart}
\end{align}
Next, change variables to integrate over $u=(v-x)/h$ rather
than over $v$,\footnote{%
  This is a standard trick when working with Kernel estimators.
}
\begin{align*}
  \E[\hat{p}(x)]
  =
  \int
  K(-u)
  \;
  p(x+hu)
  \;du
\end{align*}
Use the fact that $K$ is symmetric $K(u)=K(-u)$ along with the second
order Taylor expansion of $p$ about $x$ in
Expression~\ref{kerneltaylor1} to get:
\begin{align*}
  \E[\hat{p}(x)]
  =
  \int
  K(u)
  \bigg[
  p(x)
  + p'(x) hu
  + \frac{1}{2}p''(x)h^2u^2
  + O(h^3)
  \bigg]
  \;du
\end{align*}
Terms with $x$ only are a constant with respect to the variable of
integration $u$. Moreover, $K(u)$ integrates to one; $uK(u)$ integrates
to zero ($K$ is symmetric).
All this implies Eq.~\ref{kernelmean}.

(\emph{Variance})
Because the estimator $\hat{p}(x)$ in Definition~\ref{kernel} is an
average of iid terms, its variance is $1/N$ times the variance of any
individual term:
\begin{align*}
  \Var(\hat{p}(x))
  &=
  \frac{1}{N}
  \Var\left(
  \frac{1}{h}
  K\left(
  \frac{x-X_n}{h}
  \right)
  \right)
\end{align*}
Then by the usual variance formula $\Var(X)=\E[X^2]-\E[X]^2$:
\begin{align}
  \Var(\hat{p}(x))
  &=
  \frac{1}{N}
  \bigg\{
  \underbrace{%
    \E\left[
    \frac{1}{h^2}
    K\left(
    \frac{x-X_n}{h}
    \right)^2
    \right]
  }_{(i)}
  -
  \underbrace{%
    \E\left[
    \frac{1}{h}
    K\left(
    \frac{x-X_n}{h}
    \right)
    \right]^2
  }_{(ii)}
  \bigg\}
  \label{kernelvarstart}
\end{align}
First note (ii) equals $\E[\hat{p}(x)]^2$ (by
Expression~\ref{kernelmeanstart}).
Then, from Expression~\ref{kernelmean}, we can also write
\begin{align}
  (ii)
  =
  \E[\hat{p}(x)]^2
  &=
  p(x)^2 + O(h^2)
  \label{kernelvarapprox}
\end{align}
That's good enough for (ii).
Next, we want to compute (i). So start by writing out
\begin{align*}
  (i)
  =
  \frac{1}{h^2}
  \E\left[
  K\left(
  \frac{x-X_n}{h}
  \right)^2
  \right]
  =
  \frac{1}{h^2}
  \int
  K\left(
  \frac{x-v}{h}
  \right)^2
  p(v)
  \;
  dv
\end{align*}
Change variables again with $u=(v-x)/h$ and simplify with $K(u)=K(-u)$:
\begin{align*}
  (i)
  =
  \frac{1}{h}
  \int
  K\left( u \right)^2
  p(x+hu)
  \;
  du
\end{align*}
Further approximating Expression~\ref{kerneltaylor1} as
$p(x+hu) = p(x) + O(h)$ and substituting in:
\begin{align*}
  (i)
  =
  \frac{1}{h}
  \int
  K\left( u \right)^2
  \left[
  p(x)
  + O(h)
  \right]
  \;
  du
  &=
  p(x)
  \frac{1}{h}
  \int
  K\left( u \right)^2
  \;
  du
  + O(1)
\end{align*}
Plugging this for (i) and Expression~\ref{kernelvarapprox} for (ii) into
Expression~\ref{kernelvarstart}, we can then simplify all the way to
Expression~\ref{kernelvar}.
\end{proof}


\clearpage
\begin{prop}\emph{(MSE of $\hat{p}(x)$)}
Mean square error of estimator $\hat{p}(x)$ is
\begin{align*}
  MSE(\hat{p}(x))
  &=
  \text{Bias}(\hat{p}(x))^2
  + \Var(\hat{p}(x))
  \\
  &\approx
  (c_1 h^2)^2
  +
  c_2 (Nh)^{-1}
\end{align*}
where $c_1$ and $c_2$ are constants implied by the $O(h^2)$
and $O((Nh)^{-1})$ terms in Expressions~\ref{kernelmean} and
\ref{kernelvar}. From the full expressions, we see that they equal
\begin{align*}
  c_1
  &= \frac{1}{2}p''(x)\int u^2K(u)\,du
  \quad\qquad
  c_2
  = p(x)\int K(u)^2\,du
\end{align*}
\end{prop}

\begin{cor}\emph{(MSE-Optimal $h$)}
Hence, the MSE minimizing bandwith satisfies is
\begin{align*}
  h
  =
  \left(
  \frac{c_2}{4c_1^2}
  \right)^{1/5}
  N^{-1/5}
  \propto
  N^{-1/5}
\end{align*}
The MSE at this optimal bandwidth is therefore
\begin{align*}
  MSE(\hat{p}(x))
  &\propto
  N^{-4/5}
\end{align*}
This is slower than the $N^{-1}$ rate that we have in parameteric
models.
\end{cor}
\begin{proof}
Simply look at the first order conditions (with respect to $h$) of the
MSE expression:
\begin{align*}
  0
  &=
  4c_1^2 h^3
  -
  c_2
  \frac{N}{N^2h^2}
  \quad\implies\quad
  h
  =
  \left(
  \frac{c_2}{4c_1^2}
  \frac{1}{N}
  \right)^{1/5}
\end{align*}
\end{proof}

\begin{defn}(MISE of $\hat{p}(x)$)
\end{defn}
MISE definition
MISE result
MISE optimal


\clearpage
\subsection{Local Regression}

This section covers nonparametric estimation of the conditional
expectation function (CEF)
\begin{align}
  m(x):=\E[y_n|X_n=x]
  =
  \int
  y
  \frac{f(x,y)}{f(x)}
  \;dy
  \label{CEF}
\end{align}
where $X_n$ is some vector of covariates. This is an important object
because the CEF is the best (i.e. MSE-minimizing) forecaster of $y_n$
given $X_n$.  Unlike linear regression, nonparametric regression will
not look for a \emph{linear} approximation to the CEF $m(x)$, but rather
will try to estimate the nonlinear function $m(x)$ \emph{directly}.

\begin{defn}{(Nadayara-Watson Kernel Regression Estimator)}
The Nadayara-Watson kernel regression estimator of $m(x)=\E[y_n|X_n=x]$
replaces the densities $f(x,y),f(x)$ in Expression~\ref{CEF} with kernel
density estimates $\hat{f}(x,y)$ and $\hat{f}(x)$:
\begin{align*}
  \hat{m}(x)
  &=
  \int
  y\frac{\hat{f}(x,y)}{\hat{f}(x)}
  \;dy
  =
  \frac{%
    \sumnN
    K\left(
    \frac{X_n-x}{h}
    \right)
    y_n
  }{%
    \sumnN
    K\left( \frac{X_n-x}{h}
    \right)
  }
\end{align*}
where the final equality came from simplifying, which we now show.
\end{defn}
\begin{proof}
First, start with our kernel density estimates
\begin{align*}
  \hat{f}(x)
  &=
  \frac{1}{N}
  \sumnN
  \frac{1}{h}
  K\left(
  \frac{X_n-x}{h}
  \right)
  \qquad
  \hat{f}(x,y)
  =
  \frac{1}{N}
  \sumnN
  \frac{1}{h^2}
  K\left(
  \frac{X_n-x}{h}
  \right)
  \tilde{K}\left(
  \frac{y_n-y}{h}
  \right)
\end{align*}
where $\tilde{K}$ is the kernel for $y$, which potentially differs from
that for $x$ (it won't really matter).
Plug these into the integral expression, rewrite the resulting integral
of the sum as a sum of integrals, and pull out anything with just an $x$
in it from the integral expression since constant with respect to the
variable of integration $y$:
\begin{align*}
  \hat{m}(x)
  =
  \frac{%
    \sumnN
    \left[
    K\left(
    \frac{X_n-x}{h}
    \right)
    \int
    y
    \frac{1}{h}
    \tilde{K}\left(
    \frac{y_n-y}{h}
    \right)
    \;dy
    \right]
  }{%
    \sumnN
    K\left(
    \frac{X_n-x}{h}
    \right)
  }
\end{align*}
Change variables with $u=(y-y_n)/h$, using the fact that
$\tilde{K}(u)=\tilde{K}(-u)$ to get
\begin{align*}
  \hat{m}(x)
  &=
  \frac{%
    \sumnN
    \left[
    K\left(
    \frac{X_n-x}{h}
    \right)
    \int
    (y_n+uh)
    \tilde{K}(u)
    \;du
    \right]
  }{%
    \sumnN
    K\left(
    \frac{X_n-x}{h}
    \right)
  }
  \\
  &=
  \frac{%
    \sumnN
    \left[
    K\left(
    \frac{X_n-x}{h}
    \right)
    \left[
    y_n
    \int
    \tilde{K}(u)
    \;du
    +
    h
    \int
    u
    \tilde{K}(u)
    \;du
    \right]
    \right]
  }{%
    \sumnN
    K\left(
    \frac{X_n-x}{h}
    \right)
  }
\end{align*}
Since $\tilde{K}$ is a kernel, the first integral is one, and the second
integral is zero, and the Nadayara-Watson estimator is obtained.
\end{proof}

\clearpage

We can also derive estimators of $m(x)$ through local weighted least
squares, weighting observations $(y_n,X_n)$ with $X_n$ nearby $x$ most
highly, and downweighting those observations farther away.
The regression is also ``local'' in the sense that there will be a
separate regression for each $x$.
%Whereas parametric linear regression
%regression imposes a \emph{globally} linear structure on $m(x)$, with
%common

\begin{defn}(Local Constant Regression Estimator, Nadayara Watson)
The \emph{local constant regression estimator} of $m(x)=\E[y_n|X_n=x]$
is
\begin{align*}
  \hat{m}(x)
  =
  \argmin_{b}
  \sumnN
  K\left(
  \frac{X_n-x}{h}
  \right)
  (y_n-b)^2
\end{align*}
Note that this is simply weighted least squares regression of $y_n$ on a
constant $b$, with weights given by the distance of $X_n$ from the point
$x$ that we're estimating $m(x)$ at.
Each different point $x$ will have it's own different $\hat{m}(x)$
corresponding to the $x$-specific optimal $b$ under the $x$-specific
weights.
From the FOCs, we get that
\begin{align*}
  \hat{m}(x)
  &=
  \frac{%
    \sumnN
    K\left(
    \frac{X_n-x}{h}
    \right)
    y_n
  }{%
    \sumnN
    K\left( \frac{X_n-x}{h}
    \right)
  }
\end{align*}
which just so happens to be the same estimator as Nadayara-Watson.
\end{defn}

\begin{defn}(Local Linear, Local Polynomial Regression)
The \emph{local polynomial regression estimator} of $m(x)=\E[y_n|X_n=x]$
is given by \begin{align*}
  \hat{m}(x)
  &=
  \hat{\beta}_0
  \\
  \text{where}\quad
  (\hat{\beta}_0,\ldots,\hat{\beta}_p)
  &=
  \argmin_{b_0,\ldots,b_p}
  \sumnN
  K\left(
  \frac{X_n-x}{h}
  \right)
  \big(y_n-[b_0+b_1(X_n-x)+\cdots +b_p(X_n-x)^p]\big)^2
\end{align*}
The case wehre $p=1$ is called \emph{local linear regression}.
This is estimator is much better at the boundaries, where the local
constant regression estimator does poorly.
\end{defn}



\clearpage
\section{Extremum Estimation}


\begin{defn}(Criterion Functions)
Throughout this section, $Q_n:\Theta\ra\R$ denotes a
\emph{sample criterion function} that is a random function
which depends on sample size $n$.\footnote{%
  Strictly speaking, $n$ need not index sample size. Many of the proofs
  to follow generically work for any sequence of functions
  $\{Q_n(\theta)\}\ninf$. It's just that $n$ usually denotes sample size
  in practice.
}
Deterministic function $Q:\Theta\ra\R$ denotes the
\emph{population criterion function}, which is generally the probability
limit of $Q_n$.
Without loss of generality, we will try to minimize these functions.

These criterion functions can and should be interpreted very generally.
Even though the argument $\theta$ of $Q_n,Q$ is typically a finite
parameter vector in $\Theta=\Rk$, the parameter space $\Theta$ could
just as well be an infinite dimensional \emph{function space} so that
$Q_n(\theta),Q(\theta)$ are functionals taking some function $\theta$ as
their argument.
\end{defn}

\begin{defn}(Extremum Estimator)
Estimator $\hat{\theta}$ is said to be an \emph{extremum estimator}
associated with sample criterion function $Q_n$ if it satisfies
\begin{align*}
  Q_n(\hat{\theta})
  &=
  \inf_{\theta\in \Theta}
  Q_n(\theta)
  +
  o_P(1)
\end{align*}
In words, $\hat{\theta}$ minimizes $Q_n(\theta)$, up to some ``error''
that shrinks to zero in probability.
\end{defn}

\begin{defn}(Pointwise Convergence of Random Functions)
Sequence of random functions $Q_n(\theta)$ is said to
\emph{converge pointwise} to deterministic function $Q(\theta)$
if
\begin{align*}
  \forall \theta
  \qquad
  |Q_n(\theta)-Q(\theta)|
  =
  o_P(1)
\end{align*}
\end{defn}

\begin{defn}(Uniform Convergence of Random Functions)
Sequence of random functions $Q_n(\theta)$ is said to
\emph{converge uniformly} to deterministic function $Q(\theta)$
if
\begin{align*}
  \sup_{\theta\in\Theta}
  |Q_n(\theta)-Q(\theta)|
  =
  o_P(1)
\end{align*}
This implies pointwise convergence but is strictly stronger, much like
how uniform continuity is stronger than pointwise continuity.
\end{defn}


\begin{defn}(Identifiable Uniqueness)
\emph{Identifiable uniqueness} holds for population criterion function
$Q(\theta)$ if $\exists! \;\theta_0$ such that
\begin{align*}
  \forall \varepsilon>0
  \qquad
  \inf_{\theta\in\Theta \setminus B_\varepsilon(\theta_0)}
  Q(\theta)
  > Q(\theta_0)
\end{align*}
In words, $\theta_0$ is the uniquely identified mininum of $Q(\theta)$
if it's the unique point such that---for \emph{any} open ball of
\emph{any} size around $\theta_0$---the points outside that ball all
lead to a larger value of the population criterion function.
\end{defn}

\begin{rmk}
First, why this somewhat pedantic definition?
It rules out particularly weird and intractable criterion functions.

Second, it's important to note that this a very clinical definition of
$\theta_0$ and does \emph{not} imbue that $\theta_0$ with any
economic/substantive meaning, imply that $\theta_0$ is the parameter
of interest, or imply that the model is correctly specified.
Those more weighty conclusions follow only if we've made enough
economic/substantive assumptions about the model and the data generating
process such that $Q(\theta)$ is the right criterion function and
$\theta_0$ the true parameter of interest (rather than just a lowly
minimum, as we have here).
\end{rmk}

\begin{lem}\emph{(Sufficient Conditions for Identifiable Uniqueness)}
Suppose that
\begin{enumerate}[label=\emph{(\roman*)}]
  \item $\Theta$ compact
  \item $Q(\theta)$ uniquely minimized at $\theta_0$
  \item $Q(\theta)$ continuous
\end{enumerate}
Then identifiable uniqueness holds for $Q(\theta)$.
\end{lem}
\begin{proof}
Let $\varepsilon>0$ be given.
Since $\Theta$ compact and $B_\varepsilon(\theta_0)$ open, choice
set $\Theta\setminus B_\varepsilon(\theta_0)$ is compact.
Since $Q(\theta)$ continuous, a unique minimum $\theta_1$ is attained
on this compact choice set.
Therefore, $\inf_{\theta\in\Theta\setminus B_\varepsilon(\theta_0)}
Q(\theta)=Q(\theta_1)<Q(\theta_0)$,
where the last inequality (which is precisely identifiable uniqueness)
followed because $\theta_0$ is the unique minimizer of $Q$.
\end{proof}

\begin{thm}\emph{(Sufficient Conditions for Consistency)}
Suppose
\begin{enumerate}[label=\emph{(\roman*)}]
  \item $\hat{\theta}$ is an extremum estimator, i.e.
    $Q_n(\hat{\theta}) = \inf_{\theta\in \Theta} Q_n(\theta) + o_P(1)$
  \item Sample criterion function $Q_n(\theta)$ converges uniformly to
    population $Q(\theta)$
  \item Identifiable uniqueness holds for $Q(\theta)$,
    implying a unique minimizer $\theta_0$
\end{enumerate}
Then $\hat{\theta}$ is a consistent estimator for $\theta_0$.
\end{thm}

\clearpage
\begin{proof}
To show consistency, we must show that for given $\varepsilon>0$,
\begin{align*}
  P\big[
    \lVert\hat{\theta}-\theta_0\rVert
    > \varepsilon
  \big]
  \;\ra\;
  0
\end{align*}
First note that we can rewrite the event in the probability as
\begin{align*}
  \{
    \lVert\hat{\theta}-\theta_0\rVert
    > \varepsilon
  \}
  =
  \{
    \hat{\theta}\not\in B_\varepsilon(\theta_0)
  \}
  =
  \{
    \hat{\theta}\in \Theta\setminus B_\varepsilon(\theta_0)
  \}
  =
  \{
    Q(\hat{\theta})
    >
    Q(\theta_0)
    +
    \delta
  \}
\end{align*}
where the third equality followed from identifiable uniqueness, since
$\hat{\theta}\in\Theta\setminus B_\varepsilon(\theta_0)$ implies
$Q(\hat{\theta})>Q(\theta_0)$, which implies
$Q(\hat{\theta})>Q(\theta_0)+\delta$ for some $\delta>0$.
Next, rewrite the event again:
\begin{align*}
  \{
    \lVert\hat{\theta}-\theta_0\rVert
    > \varepsilon
  \}
  =
  \{
    Q(\hat{\theta})
    -
    Q(\theta_0)
    >
    \delta
  \}
\end{align*}
Thus to show $P[\lVert\hat{\theta}-\theta_0\rVert > \varepsilon]$ goes
to zero, we can show $P[Q(\hat{\theta})-Q(\theta_0)>\delta]$ goes to
zero. And to do that, break things up:
\begin{align*}
  Q(\hat{\theta})
  -
  Q(\theta_0)
  =
  \underbrace{%
    Q(\hat{\theta})
    - Q_n(\hat{\theta})
  }_{(i)}
  +
  \underbrace{%
    Q_n(\hat{\theta})
    - Q_n(\theta_0)
  }_{(ii)}
  +
  \underbrace{%
    Q_n(\theta_0)
    - Q(\theta_0)
  }_{(iii)}
\end{align*}
From there, note that
\begin{align*}
  Q(\hat{\theta}) - Q(\theta_0) > \delta
  \;\implies\;
  \text{at least one of the following is true:}
  \;
  \begin{cases}
    A= \{Q(\hat{\theta}) - Q_n(\hat{\theta}) \;\;\;> \delta/3\}\\
    B= \{Q_n(\hat{\theta}) - Q_n(\theta_0) > \delta/3 \}\\
    C = \{Q_n(\theta_0) - Q(\theta_0) > \delta/3\} \\
  \end{cases}
\end{align*}
This is easiest to see by noting that if events $A,B,C$ were \emph{all}
false, there's no way we could have $Q(\hat{\theta}-Q(\theta_0)>\delta$.
Then from there, we have that
\begin{align*}
  P[\lVert\hat{\theta}-\theta_0\rVert>\varepsilon]
  =
  P[Q(\hat{\theta}) - Q(\theta_0) > \delta]
  &\leq P[\text{$A$ or $B$ or $C$}] \\
  &\leq P[A]+P[B]+P[C]
\end{align*}
And so to bound our target
$P[ \lVert\hat{\theta}-\theta_0\rVert > \varepsilon ]$,
we can bound the probabilities of $A,B,C$ individually given
$\varepsilon$ and the corresponding $\delta$.
So first, uniform convergence of $Q_n$ to $Q$ implies
\begin{align*}
  P[A]
  =
  P\big[Q(\hat{\theta})-Q_n(\hat{\theta})>\delta/3\big]
  \leq
  P\left[\sup_{\theta\in\Theta} |Q(\theta)-Q_n(\theta)|>\delta/3\right]
  =
  o(1)
\end{align*}
Next, definition of an EE
$Q_n(\hat{\theta})=\inf_{\theta\in\Theta}Q_n(\theta)+o_P(1)$
implies $Q_n(\hat{\theta})\leq Q_n(\theta_0)+o_P(1)$, thus
\begin{align*}
  P[B]
  =
  P[Q_n(\hat{\theta})-Q_n(\theta_0)>\delta/3]
  \leq
  P[o_P(1)>\delta/3]
  =
  o(1)
\end{align*}
Finally, uniform convergence of $Q_n$ to $Q$ again buys us
\begin{align*}
  P[C]
  =
  P\big[Q_n(\theta_0)-Q(\theta_0)>\delta/3\big]
  \leq
  P\left[\sup_{\theta\in\Theta} |Q(\theta)-Q_n(\theta)|>\delta/3\right]
  =
  o(1)
\end{align*}
Thus, since $P[A],P[B],P[C]$ are all $o(1)$ (i.e. go to zero), their sum
also goes to zero hence
\begin{align*}
  P[\lVert\hat{\theta}-\theta_0\rVert>\varepsilon]
  &\leq P[A]+P[B]+P[C]
  = o(1)
\end{align*}
\end{proof}


\begin{defn}(Stochastic Equicontinuity)
Sequence of random functions $H_n(\theta)$ are
\emph{stochastically equicontinuous} if, for all $\varepsilon>0$, there
exists a $\delta>0$ such that
\begin{align*}
  \liminf_{n\ra\infty}
  P
  \left[
    \sup_{\theta\in\Theta}
    \sup_{\theta'\in B_\delta(\theta)}
    |H_n(\theta)-H_n(\theta')|
    \leq
    \varepsilon
  \right]
  \geq 1-\varepsilon
\end{align*}
The key point is that $\delta$ works \emph{uniformly} over all
$\theta\in\Theta$, i.e. does not depend upon $\theta$ and so can be used
even while looking at $\sup_{\theta\in\Theta}$.
This is a generalization to a sequence of \emph{random} functions the
notion of equicontinuity or uniform continuity for a single
\emph{deterministic} function.
There, equicontinuity means that for all $\varepsilon>0$, there exists a
$\delta$ that works \emph{uniformly} over all $\theta$ (i.e. depends
only on $\varepsilon$, not $\theta$) such that the statement inside the
square brackets is true.

Note, it may look deep/important that $\varepsilon$ shows up inside the
square brackets and outside it. But it's really not.
We just need a small number inside the square brackets, a big number
outside. We could generalize to make them different, but it wouldn't
change anything.
\end{defn}

\clearpage
\begin{thm}\emph{(Relating Pointwise \& Uniform Convergence)}
Two directions, near-converses
\begin{enumerate}[label=\emph{(\alph*)}]
  \item \emph{(Compact $\Theta$ + SE + PW $\Rightarrow$ UC)}:
    Suppose that
    \begin{enumerate}[label=\emph{(\roman*)}]
      \item $\Theta$ is compact
      \item $H_n(\theta)$ is a stochastically equicontinuous sequence of
        random functions
      \item $H_n(\theta)\pto 0$ for all $\theta\in\Theta$ (i.e.
        pointwise convergence).
    \end{enumerate}
    Then $H_n(\theta)\uto 0$
  \item \emph{(UC $\Rightarrow$ SE)}
    Suppose that $H_n(\theta)\uto 0$.
    Then $H_n(\theta)$ is a stochastically equicontinuous sequence of
    random functions
\end{enumerate}
For extremum estimation, we should think of
$H_n(\theta)=|Q_n(\theta)-Q(\theta)|$.
\end{thm}
\begin{proof}
(a)
To show uniform convergence of $H_n$ to $0$, must show for any
$\varepsilon>0$ that
\begin{align}
  (*)
  =
  \limn
  P\left[
  \sup_{\theta\in\Theta}
  |H_n(\theta)|\geq\varepsilon
  \right]
  \leq \varepsilon
  \label{Pse}
\end{align}
In other words, we must show $H_n$ isn't far from $0$ for any given
$\varepsilon>0$.
So first, assumption (ii) stochastic equicontinuity lets us control the
smoothness of the function $H_n$ in small neighborhoods of any given
point. In particular, given that $\varepsilon$, there exists a $\delta$
such that
\begin{align}
  \limsup_{n\ra\infty}\;\;
  P\left[
    \sup_{\theta\in\Theta}
    \sup_{\theta'\in B_\delta(\theta)}
    |H_n(\theta)-H_n(\theta')|
    \geq
    \varepsilon
  \right]
  \leq \varepsilon
\end{align}
Given that $\varepsilon$ and corresponding $\delta$, by compactness of
$\Theta$, there exists a finite cover of $\Theta$ by balls
$\{B_\delta(\theta_j)\}_{j=1}^J$ centered at finite number of points
$\{\theta_j\}_{j=1}^J$.\footnote{%
  This was the main point of compactness: it lets us focus our attention
  on a finite number of points $\{\theta_j\}_{j=1}^J$, while SE tells us
  that the values of the function $H_n$ are close in small neighborhood
  of those points.
}
And since the set of balls covers $\Theta$, we can equivalently rewrite
$(*)$ as
\begin{align*}
  (*)
  =
  P\left[
  \max_j
  \sup_{\theta'\in B_\delta(\theta_j)}
  |H_n(\theta')|
  >\varepsilon
  \right]
  &\leq
  P\left[
  \max_j
  \sup_{\theta'\in B_\delta(\theta_j)}
  \bigg(
  \big|H_n(\theta')-H_n(\theta_j)\big|
  +
  \big|H_n(\theta_j)\big|
  \bigg)
  \geq\varepsilon
  \right]
  \\
  &\leq
  P\bigg[
  \underbrace{%
    \max_j
    \sup_{\theta'\in B_\delta(\theta_j)}
    \big|H_n(\theta')-H_n(\theta_j)\big|
  }_{A}
  +
  \underbrace{%
    \max_j
    \big|H_n(\theta_j)\big|
    \vphantom{\sup_{\theta'\in B_\delta(\theta_j)}}
  }_{B}
  \geq\varepsilon
  \bigg]
\end{align*}
Note the second term dropped the $\sup$ because the argument is at fixed
at $\theta_j$.
Next, $A+B>\varepsilon$ implies at least one of $A,B$ is
individually greater than $\varepsilon/2$.\footnote{%
  They couldn't \emph{both} be less than $\varepsilon/2$ with their sum
  greater than $\varepsilon$.
}
Then the above implies
\begin{align*}
  (*)
  \quad
  &\leq
  \quad
  P\big[
    \{A> \varepsilon/2\}
    \cup \{B>\varepsilon/2\}
    \geq\varepsilon
  \big]
  \quad
  \leq
  \quad
  P[A>\varepsilon/2]
  +
  P[B>\varepsilon/2]
\end{align*}
Writing out $A,B,C$, the above says
\begin{align*}
  (*)
  &\leq
  P\bigg[
    \max_j
    \sup_{\theta'\in B_\delta(\theta_j)}
    |H_n(\theta')-H_n(\theta_j)|
    >\varepsilon/2
  \bigg]
  +
  P\bigg[
    \max_j
    |H_n(\theta_j)|
    >\varepsilon/2
  \bigg]
  \\
  &\leq
  \underbrace{%
    P\bigg[
      \sup_{\theta\in \Theta}
      \sup_{\theta'\in B_\delta(\theta_j)}
      |H_n(\theta')-H_n(\theta)|
      >\varepsilon/2
    \bigg]
  }_{\ra\varepsilon}
  +
  \underbrace{%
    \sum_j
    P\bigg[
      |H_n(\theta_j)|
      >\varepsilon/2
    \bigg]
  }_{\ra0}
\end{align*}
Convergence of the first term as $n\ra\infty$ because of assumed SE of
sequence $H_n$. Convergence of the sum to zero because $H_n(\theta_j)$
is $o_P(1)$ for each $\theta_j$ by Assumption (iii) and the sum of $J$
terms that are $o_P(1)$ is also $o_P(1)$.
Hence, the limit of $(*)$ is less than $\varepsilon$.

(b)
Given any $\varepsilon>0$, we need to show that there exists a
$\delta>0$ such that
\begin{align*}
  \limsup_{n\ra\infty}\;\;
  P\left[
    \sup_{\theta\in\Theta}
    \sup_{\theta'\in B_\delta(\theta)}
    |H_n(\theta)-H_n(\theta')|
    \geq
    \varepsilon
  \right]
  \leq \varepsilon
\end{align*}
But note that
\begin{align*}
  P\left[
    \sup_{\theta\in\Theta}
    \sup_{\theta'\in B_\delta(\theta)}
    |H_n(\theta)-H_n(\theta')|
    \geq
    \varepsilon
  \right]
  \leq
  P\left[
    \sup_{\theta\in\Theta}
    2|H_n(\theta)|
    \geq
    \varepsilon
  \right]
\end{align*}
By assumption, $H_n(\theta)\uto 0$, hence the above statement
probability goes to zero as $n\ra\infty$ by definition of uniform
convergence.
\end{proof}

\begin{thm}\emph{(Stochastic Equicontinuity of $M$-Estimators)}
Suppose that
\begin{enumerate}[label=\emph{(\roman*)}]
  \item $\Theta$ is compact
  \item $W_i$ identically distributed.\footnote{%
      Note, \emph{not} necessary to be \emph{independent}.
      This accomodates stationary and processes with identically
      distributed marginals for each observation, but conditionally
      dependent observations.
    }
  \item $\E\left[\sup_{\theta\in\Theta} m(W_i,\theta)\right]<\infty$
  \item For each $\theta$, the function $m(W_i,\theta)$ is continuous
    almost everywhere in $W_i$.\footnote{%
      Note that the points of discontinuity in the $W_i$ argument are
      allowed to vary across $\theta$
    }
\end{enumerate}
Then
\begin{enumerate}[label=\emph(\alph*)]
  \item $\E[m(W_i,\theta)]$ is a continuous function of $\theta$
  \item
    $H_n(\theta)=\frac{1}{n}\sumin\big(m(W_i,\theta)-\E[m(W_i,\theta)]\big)$
    is a SE sequence of functions
\end{enumerate}
\end{thm}
%\begin{proof}
%For both parts, it will be useful to define the modulus of continuity
%for observation $i$:
%\begin{align*}
  %\omega_{i,\delta}
  %:=
  %\sup_{\theta\in\Theta}
  %\sup_{\theta'\in B_\delta(\theta)}
  %\lVert m(W_i,\theta)-m(W_i,\theta')\rVert
%\end{align*}
%Note that by Assumption (ii)---continuity of $m(W_i,\theta)$.
%(a)
%Suppose $\varepsilon>0$ and $\theta$ given.
%Note that
%To show continuity, we must show that there exists a $\delta>0$ such
%that
%\begin{align*}
  %\lVert \E[m(W_i,\theta)]-\E[m(W_i,\theta')]\rVert
  %\leq
  %\E\big[\lVert m(W_i,\theta)-m(W_i,\theta')\rVert\big]
  %\leq
  %\E[\omega_{i,\delta}]
%\end{align*}
%\end{proof}


\clearpage
\begin{thm}\emph{(Pointwise LLN)}
Suppose $\forall\theta\in\Theta$,
$\E[\lVert m(W_i,\theta)\rVert]<\infty$ and the $W_i$ iid.
Then
\begin{align*}
  \frac{1}{n}
  \sumin m(W_i,\theta)
  \quad\pto\quad
  \E[m(W_i,\theta)]
  \qquad
  \forall \theta\in\Theta
\end{align*}
\end{thm}

\begin{cor}\emph{(Uniform LLN)}
Suppose conditions (i)-(iv) of the previous theorem hold and
\begin{align*}
  \forall \theta\in\Theta
  \qquad
  \frac{1}{n}
  \sumin
  m(W_i,\theta)
  \quad\pto\quad
  \E[m(W_i,\theta)]
\end{align*}
Then
\begin{align}
  \sup_{\theta\in\Theta}
  \left\lVert
  \frac{1}{n}
  \sumin
  m(W_i,\theta)
  -
  \E[m(W_i,\theta)]
  \right\rVert =
  o_P(1)
\end{align}
\end{cor}
\begin{proof}
%By the previous Theorem, $H_n$ in Expression~\ref{Hn} is SE.
Since $\Theta$ compact, $H_n$ SE, and pointwise convergence assumed, we
get that $H_n$ converges uniformly to $Q(\theta)$ by Part (a) of that
one big theorem.
By previous theorem, $Q(\theta)=\E[m(W_i,\theta)]$ is continuous.
If $Q(\theta)$ is uniquely minimized at $\theta_0$, Lemma gives us
identifable uniqueness.
Then if $\hat{\theta}$ is an extremum esitmator, since $Q_n$ converges
uniformly and ID holds, $\hat{\theta}$ is consistent.
\end{proof}

\begin{lem}
Suppose some data-dependent function $L_n(\theta)$ converges uniformly
to continuous function $L(\theta)$ in some small neighborhood
$B_\varepsilon(\theta_0)$ about the true parameter value $\theta_0$:
\begin{align*}
  \sup_{\theta\in B_\varepsilon(\theta_0)}
  \lVert L_n(\theta) - L(\theta)\rVert
  = o_P(1)
\end{align*}
Suppose also that we have a consistent estimator
$\hat{\theta}\pto\theta_0$. Then
\begin{align*}
  L_n(\hat{\theta})
  \pto L(\theta_0)
\end{align*}
\end{lem}

\begin{thm}
Suppose $\hat{\theta}$ is an extremum estimator defined from twice
differentiable criterion function $Q_n(\theta)$.
If also
\begin{align*}
  \nabla_\theta
  Q_n(\hat{\theta})
  =
  o_P(n^{-1/2})
\end{align*}
Then
\begin{align*}
  \sqrt{n}(\hat{\theta}-\theta_0)
  =
  -(\nabla_{\theta}^2 Q_n(\tilde{\theta}))^{-1}
  \big[
  \sqrt{n}\cdot \nabla_{\theta} Q_n(\theta_0)
  \big]
  +
  o_P(1)
\end{align*}
which is asymptotically normal by the CLT.
\end{thm}


\clearpage
\section{VARs}

Three types of concepts, all distinct
\begin{itemize}
  \item Shock: Economically meaningful primitive exogeneous forces.
    They have the following characteristics
    \begin{itemize}
      \item Exogenous with respect to other current and lagged endogenous
        variables in the model. This is putting a finer point on
        ``primitive'' and ``exogenous''
      \item Uncorrelated with each other so we can identify unique
        causual effects
      \item Represent unanticipated movements in exogenous variables or
        news about future movements in exogeneous variables
    \end{itemize}
  \item Innovation: The residuals from a reduced-form VAR
  \item Instrument
\end{itemize}
These three concepts are all generally distinct.
It is only through the process of \emph{identification} that we
link/equate innovations with the shocks

Model
\begin{align*}
  Y_t
  &= B(L)Y_t + \Omega \varepsilon_t
  \qquad\text{where}\quad
  \varepsilon_t \sim (0,D)
  \\
  B(L)
  &= B_0 + \sum_{k=1}^p B_kL^k
\end{align*}
Corresponding reudced form
\begin{align*}
  A(L)Y_t
  &= \eta_t
  \qquad\text{where}\quad
  \eta_t
  \sim (0,\Sigma_\eta)
  \\
  A(L)
  &= I-\sum_{k=1}^p A_kL^k
\end{align*}
Can then link reduced form innovations $\eta_t$ to the structural shocks
$\varepsilon_t$ by rewriting
\begin{align*}
  (I-B_0)Y_t
  &=
  \sum_{k=1}^p B_k Y_{t-k}
  +
  \Omega \varepsilon_t
  \\
  Y_t
  &=
  \sum_{k=1}^p A_k Y_{t-k}
  +
  \eta_t
\end{align*}
From this, deduce that
\begin{align*}
  \eta_t
  &=
  (I-B_0)^{-1}\Omega\varepsilon_t
  \\
  \iff\quad
  \eta_t
  &= B_0\eta_t + \Omega \varepsilon_t
\end{align*}

Identification approaches
\begin{itemize}
  \item Triangularization/Cholesky: Impose zero restrictions
    so that an endogenous variable does not respond to other endogenous
    variables contemporaneously.
    The variable ordered ``first'' responds to no other endogenous
    variables.
    The variable ordered ``last'' respond to everything else in the
    system.

  \item Nonzero restrictions, structural VAR: set a coefficient to some
    number which is known for good theoretical reasons

  \item Heteroscedasticity

  \item High Frequency identification: Use daily data to isolate effect
    of Fed announcements on rates. If you control for the Fed's
    information set, then this is plausible identification of shock/news

  \item
    External Instruments/Proxy SVARs:
    External series $Z_t$ (outside the endogenous variable vector $Y_t$)
    is a valid instrument for identifying shock $\varepsilon_{ti}$ if
    \begin{align*}
      \E[Z_t\varepsilon_{ti}] &\neq 0 \\
      \E[Z_t\varepsilon_{tj}] &= 0
      \qquad
      \forall j\neq i
    \end{align*}
    The first condition ensures relevance/usefullnes. The second
    condition ensures exogeneity.
    How to implement
    \begin{enumerate}
      \item Estimate reduced form VAR to get estimated residuals
        $\{\hat{\eta}_t\}$
      \item Regression $\eta_{tj}$ on $\eta_{ti}$ for $j\neq i$ using
        $Z_t$ as the instrument.
        These provide unbiased estimates of the ceofficients
    \end{enumerate}

  \item
    Long Run Restrictions:
    Moving average rep
    \begin{align*}
      Y_t = C(L)\eta_t
      \qquad\text{where}\quad
      C(L) = A(L)^{-1}
    \end{align*}
    Can write $Y_t$ as
    \begin{align*}
      Y_t = D(L)\varepsilon_t = C(L)H\varepsilon_t
    \end{align*}
    Then want to say that some shock has no effect on some variable in
    the long run. This amounts to restricting $D^{ij}(1)=0$.
    This can accommodate whether you want zero long run effect on level
    or growth rate, depending on whether the variables included are
    levels or growth rates.

  \item Sign Restrictions

  \item FAVARs: For including more information

  \item DSGE
\end{itemize}
Jorda local projection method: For overcoming the negative effects of
misspecification on estimated IRFs. In the usual way they are computed,
there's a possibility that the errors would just be compounded.
Jorda's solution is like iterated vs. direct forecasting.



\clearpage
\section{Cochrane on Unit Roots}

\subsection{Facts about Unit Roots and Shit}

Cochrane's Time Series book has a nice little history of this
literature. He also mentiones Campbell and Perron

Random walk
\begin{align}
  y_t = y_{t-1} + \varepsilon_t
  \label{rw}
\end{align}
Spectral density of an AR(1) with $\phi\ra 1$ approaching random walk
\begin{align*}
  \lim_{\phi\ra 1}f(\omega)
  =
  \lim_{\phi\ra 1}
  \frac{\sigma^2}{1+\phi^2-2\phi \cos(\omega)}
  =
  \frac{\sigma^2}{2(1-\cos(\omega))}
\end{align*}
As $\omega\ra 0$, $f(\omega)\ra \infty$, so the spectral density is
super peaked close to $\omega=0$, i.e. at low frequency.

Random walk a special case of a unit root process. A unit root process
or difference stationary process or I(1) process generalizes the RW and
is any process
\begin{align}
  y_t &= \mu + y_{t-1} + a(L)\varepsilon_t
  \label{I1}
\end{align}
where $a(L)\varepsilon_t$ is stationary.
This is unit root because the implicit lag polynomial on $y_t$,
$(1-z)$, has a root/zero at $z=1$.

Trend stationary is special case of Model~\ref{I1}
\begin{align*}
  y_t &= \mu t + b(L)\varepsilon_t
\end{align*}
This arises when $a(L)$ in Model~\ref{I1} has a unit root that we can
factor out so that
\begin{align*}
  y_t
  &= \mu + y_{t-1} + a(L)\varepsilon_t
  \\
  (1-L)y_t
  &= \mu + (1-L)b(L)\varepsilon_t
  \\
  \iff\quad
  y_t
  &=
  \mu t
  +
  b(L)\varepsilon_t
\end{align*}
Hence, if the model is trend-stationary, it is \emph{also} I(1).
But compare what you get after first differencing the unit root and
trend stationary processes
\begin{align*}
  \Delta y_t
  &= \mu + a(L)\varepsilon_t\\
  \Delta y_t
  &=
  \mu + (1-L)b(L)\varepsilon_t
\end{align*}
In the case of Model~\ref{I1}, the $a(L)$ that remains is generally
invertible and you can get an AR rep for $\Delta y_t$.
But in the trend stationary case, the $(1-L)b(L)$ that remains is
\emph{not} invertible, so no AR rep available.

Rather than think of I(1) processes mechanically as a generalization of
the random walk, the game of studying these processes is figuring out
the implications for the level or long run forecast of a process.

Spectral density at frequency zero of the I(1) process is $a(1)$.
This determines the low frequency movements.
If a pure random walk, then $a(1)=1$.
If trend stationary, then $a(1)=0$.
Of course, intermediate possibilities.

By beveridge nelson, every I(1) process can be written as a sum of a RW
and a stationary process.
There are many ways to do such a decomposition but beveridge nelson is
nice.
In beveridge nelson, the random walk component today is the value if you
forecast the model forever into the future, then run back a linear
trend. The point today in that run-backward linear trend is the random
walk value.
In other words, it's the

OLS estimate of $\phi$ in the following model is downward biased
with too-tight standard errors when $\phi$ close to one:
\begin{align*}
  y_t = \phi y_{t-1} + \varepsilon_t
\end{align*}
So our $<1$ OLS estimates could actually have been generated by random
walks.

Suppose you have a random walk model and estimate it like it's something
trend stationary. This is also bad, and will lead to finding misleading
nonexistent linear trends that are actualy just random walk.

Suppose you regress a random walk on another random walk. The
coefficient estimate will be misleadingly high.



\clearpage
\subsection{How big is the Random Walk in GNP}

Punchline: GNP does actually revert back to trend, but only over several
years. So in the short run, GNP behaves like a unit root.
So if you fit a model to the short run behavior, you will incorrectly
infer long run persistence too that isn't actually there.

Idea: ``Measure size of random walk component in GNP from the variance
of its long differences.''
Two competing models
\begin{align*}
  \text{Pure Random Walk:}&\quad
  y_t
  = bt + \sum_{j=0}^\infty a_j \varepsilon_{t-j}
  \\
  \text{Stationary about Trend:}&\quad
  y_t
  = \mu + y_{t-1} + \varepsilon_t
\end{align*}
Notice
\begin{align*}
  \text{Pure Random Walk}
  \quad&\implies\quad
  \Var(y_t-y_{t-k})
  = k\sigma^2_\varepsilon
  \\
  \text{Stationary about Trend}
  \quad&\implies\quad
  \Var(y_t-y_{t-k})
  = 2\sigma^2_y
\end{align*}
Can also form variance ratio
\begin{align*}
  R
  &= \frac{\Var(y_t-y_{t-k})/k}{\Var(y_t-y_{t-1})}
\end{align*}


\clearpage
\section{Great Mortgaging}

Introduce ``long-run annual-frequency dataset on disaggregated
(mortgage vs. business) bank credit for 17 advanced economies since
1870.''
Growth in credit has mostly been rapid growth in mortgage lending, and
that is robust across countries.

Questions:
- We see greater bank lending to households relative to businesses.
  Does this happen in low-growth economies?
- Maybe mortgage lending is high private return to the banks but low
  social return because it's not investing in technology or productive
  assets. Need a mechanism for this to happen in GE
- Aiyagari with shocks to the borrowing limit that is below zero
- Why was mortgage loan growth so strong \emph{then}? Has to be
  securitization and demand for higher yielding assets.
  But the run up is still slightly after Solomon brothers did their
  thing.
- Why is mortgage lending so different from nonmortgage business
  lending? Is it the size of the balance sheet or the composition (i.e.
  the simple fact that banks mostly do that now)?
  We see a run up in bank loans to GDP. But really it's the composition
  that has changed.
- Look at banks/credit unions and lenders that differ in the amount that
  they lend to households vs. to businesses. Are the ones that lend to
  households more fragile?


\clearpage
\section{Jump Processes}


\begin{defn}(Cadlag Function)
A function $f:[0,T]\ra\R^d$ is \emph{cadlag} if the $f$ is
right-continuous, and the lefthand limit $\lim_{\delta \ra 0}
f(t-\delta)$ exists for any point in the domain, though this lefthand
limit need \emph{not} equal the righthand limit.
This allows for jumps.
\end{defn}
\begin{defn}(Random, Stopping, Hitting Times)
A \emph{random time} $\tau$ is a nonnegative random variable.

A \emph{stopping time} with respect to filtration $\{\sF_t\}$ is a
random time that is also $\sF_t$-measurable for all $t$, i.e.
$\{\tau\leq t\}\in\sF_t$. In effect, this means that any instant, we can
tell whether or not the event (i.e. the process stopping) has happened.

Given process $X_t$, a \emph{hitting time} $\tau_A$ is a random variable
that encodes the first time $X_t$ hits open set $A$:
\begin{align*}
  \tau_A(\omega):=\argmin_{t} X_t(\omega)\in A
\end{align*}
\end{defn}


\begin{defn}(Counting Process)
Suppose we have an increasing sequence\footnote{%
  I assume ``increasing sequence'' means $T_n\leq T_{n+m}$ for all
  $m\geq 0$, realization by realization
}
of random times $T_n:\Omega\ra\R_+$ with
$P[\limn T_n=\infty]=1$.
Define the \emph{counting process} as $\N$-valued random variable $X_t$
that counts the number of random times (i.e. number of $T_n$'s)
occurring within $[0,t]$.
\begin{align*}
  X_t
  =
  \#\{
    n
    \;|\;
    T_n\leq t,
    \qquad n\geq 1
  \}
  =
  \sumninf
  \mathbf{1}_{\{T_n\leq t\}}
\end{align*}
$X_t$ is Cadlag by construction, and also piecewise constants with jumps
of one unit.
\end{defn}


\begin{defn}(Poisson Process)
The \emph{Poisson Process} $N_t$ is a particular counting process where
each random time $T_n$ is a partial sum of iid exponential RVs:
\begin{align*}
  T_n = \sumin \tau_i
  \qquad
  \tau_i\iid \text{Exp}(\lambda)
\end{align*}
Denote this process by $N_t\sim\text{Poisson}(\lambda t)$ since
\begin{align*}
  P[N_t=n]
  =
  e^{-\lambda t}
  \frac{(\lambda t)^n}{n!}
\end{align*}
$N_t$ is the only counting process with independent and stationary
increments.
\end{defn}

\begin{defn}(Random Measure)
Given  an increasing sequence of random times $T_n:\Omega\ra\R_+$,
define \emph{random measure} $\mu:\Omega\times(\R_+,\sB(\R_+))\ra \N$ on
the natural numbers:
%The sequence $T_n:\Omega\ra\R_+$ that we used to define a counting
%process $X_t$ to count the number of points in $[0,t]$ also
\begin{align*}
  \mu(\omega,A)
  =
  \#\{
    n
    \;|\;
    T_n(\omega)\in A\in\sB(\R_+),\;\; n\geq 1
  \}
\end{align*}
This measure is \emph{random} because $\mu$ is indexed by $\omega$ in
its first argument.
And for each $\omega\in\Omega$, $\mu(\omega,A)$ gives the number of
jumps (or jump times $T_n$) in the set $A$.
Of course, since jumps are random, each different $\omega\in\Omega$
implies a different time/realization for each $T_n(\omega)$, hence a
different answer for how many jumps occurred in $A$.
%We can of course take the expectation over $\Omega$ to get
%\begin{align*}
%\end{align*}

Since the sequence of random times $\{T_n\}$ can be used to define a
corresponding counting process $X_t$, we can offer an alternative
definition of $X_t$ using the random measure $\mu$:
\begin{align}
  X_t(\omega)
  =
  \mu(\omega,[0,t])
  =
  \int_0^t
  \mu(\omega,ds)
  \label{jumprelate}
\end{align}
Thus the number of jumps of $X_t$ in interval $[s,t]$ is given by
$\mu(\omega,[s,t])$.

For every counting process, there is a corresponding jump measure, and
vice versa.
\end{defn}

\begin{defn}
Let $\mu(\omega,A)$ be the associated jump measure for Poisson process
$N_t\sim\text{Poisson}(\lambda t)$.
Since jump measure $\mu(\omega,A)$ is random, we can take it's
expectation and use Expression~\ref{jumprelate} along with
$N_t\sim\text{Poisson}(\lambda t)$
to simplify
\begin{align*}
  \E[\mu(\omega,[0,t])]
  =
  \E[N_t(\omega)]
  =
  \lambda t
\end{align*}
We can think of $\mu$ as the derivative of $N_t$ in the following sense
\begin{align*}
  \frac{dN_t(\omega)}{dt}
  &=
  \mu(\omega,[t,t+dt])
  \qquad\text{where}\quad
  \mu
  = \sum_{n=1}^\infty \delta_{T_n(\omega)}
\end{align*}
i.e. $\mu$ is the sum of Dirac measures at all jump times.
\end{defn}

\begin{defn}
Let $(\Omega,\sF,P)$ be a probability space.
Let $(\R^d,\sG)$ be a a measurable space, and $\lambda$ a positive Radon
measure on that space.
Define a \emph{Poisson random measure}
\begin{align*}
  \mu:\Omega\times(\R^d,\sG)\ra \N
\end{align*}
such that for almost all $\omega\in\Omega$,
$\mu(\omega,\cdot)$ is a $\N$-valued Radon measure on $(\R^d,\sG)$
satisfiying, for any $A\in\sG$,
$\mu(\omega,A)<\infty$ and
\begin{align*}
  P[\mu(\omega,A)=n]
  =
  e^{-\lambda[A]}
  \frac{\big(\lambda[A]\big)^n}{n!}
\end{align*}
and for any disjoint measurable sets $A_1,\ldots,A_m\in\sG$, the RVs
$\mu(\omega,A_1),\ldots,\mu(\omega,A_m)$ are independent.

Can also associated a Poisson random measure with a sequence of points
$\{X_n(\omega)\}\ninf$ in $\R^d$ such that
\begin{align*}
  \mu(\omega,A)
  &=
  \sumninf \mathbf{1}_{\{X_n(\omega)\in A\}}
  \\
  \mu
  &=
  \sumninf \delta_{X_n(\omega)}
\end{align*}
\end{defn}

\clearpage
\subsection{Levy Processes}

\begin{defn}(Levy Process)
Cadlag process $X$ on $\R^d$ with $X_0=0$, satisfying the following three
properties
\begin{enumerate}[label=(\roman*)]
  \item Independent Increments:
    For any $s<t<u$, the RVs
    $X_u-X_t$ and $X_t-X_s$ are independent.
  \item Stationary Increments:
    The distribution of $X_{t+s}-X_t$ does not depend upon $t$
  \item Stochastic Continuity:
    For all $t,\varepsilon>0$,
    $\lim_{s\ra 0} P[|X_{t+s}-X_t|>\varepsilon]=0$.

    Note that this does not mean the process $X_t$ is strictly
    continuous, or that any finite interval $[0,T]$ only has a countable
    or finite number of jumps.
    Rather it only means that jumps do not happen at deterministic
    times, and that at any given time $t$, the probability of a jump is
    zero.
\end{enumerate}
\end{defn}



%% APPPENDIX %%

% \appendix




\end{document}


%%%%%%%%%%%%%%%%%%%%%%%%%%%%%%%%%%%%%%%%%%%%%%%%%%%%%%%%%%%%%%%%%%%%%%%%
%%%%%%%%%%%%%%%%%%%%%%%%%%%%%%%%%%%%%%%%%%%%%%%%%%%%%%%%%%%%%%%%%%%%%%%%
%%%%%%%%%%%%%%%%%%%%%%%%%%%%%%%%%%%%%%%%%%%%%%%%%%%%%%%%%%%%%%%%%%%%%%%%

%%%% SAMPLE CODE %%%%%%%%%%%%%%%%%%%%%%%%%%%%%%%%%%%%%%

    %% VIEW LAYOUT %%

        \layout

    %% LANDSCAPE PAGE %%

        \begin{landscape}
        \end{landscape}

    %% BIBLIOGRAPHIES %%

        \cite{LabelInSourcesFile}  %Use in text; cites
        \citep{LabelInSourcesFile} %Use in text; cites in parens

        \nocite{LabelInSourceFile} % Includes in refs w/o specific citation
        \bibliographystyle{apalike}  % Or some other style

        % To ditch the ``References'' header
        \begingroup
        \renewcommand{\section}[2]{}
        \endgroup

        \bibliography{sources} % where sources.bib has all the citation info

    %% SPACING %%

        \vspace{1in}
        \hspace{1in}

    %% URLS, EMAIL, AND LOCAL FILES %%

      \url{url}
      \href{url}{name}
      \href{mailto:mcocci@raidenlovessusie.com}{name}
      \href{run:/path/to/file.pdf}{name}


    %% INCLUDING PDF PAGE %%

        \includepdf{file.pdf}


    %% INCLUDING CODE %%

        %\verbatiminput{file.ext}
            %   Includes verbatim text from the file

        \texttt{text}
            %   Renders text in courier, or code-like, font

        \matlabcode{file.m}
            %   Includes Matlab code with colors and line numbers

        \lstset{style=bash}
        \begin{lstlisting}
        \end{lstlisting}
            % Inline code rendering


    %% INCLUDING FIGURES %%

        % Basic Figure with size scaling
            \begin{figure}[h!]
               \centering
               \includegraphics[scale=1]{file.pdf}
            \end{figure}

        % Basic Figure with specific height
            \begin{figure}[h!]
               \centering
               \includegraphics[height=5in, width=5in]{file.pdf}
            \end{figure}

        % Figure with cropping, where the order for trimming is  L, B, R, T
            \begin{figure}
               \centering
               \includegraphics[trim={1cm, 1cm, 1cm, 1cm}, clip]{file.pdf}
            \end{figure}

        % Side by Side figures: Use the tabular environment


