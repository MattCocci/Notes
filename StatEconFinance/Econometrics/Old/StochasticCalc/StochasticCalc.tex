\documentclass[12pt]{article}

\author{Matthew D. Cocci}
\title{Stochastic Calculus}
\date{\today}

%% Formatting & Spacing %%%%%%%%%%%%%%%%%%%%%%%%%%%%%%%%%%%%

%\usepackage[top=1in, bottom=1in, left=1in, right=1in]{geometry} % most detailed page formatting control
\usepackage{fullpage} % Simpler than using the geometry package; std effect
\usepackage{setspace}
%\onehalfspacing
\usepackage{microtype}

%% Formatting %%%%%%%%%%%%%%%%%%%%%%%%%%%%%%%%%%%%%%%%%%%%%

%\usepackage[margin=1in]{geometry}
    %   Adjust the margins with geometry package
%\usepackage{pdflscape}
    %   Allows landscape pages
%\usepackage{layout}
    %   Allows plotting of picture of formatting



%% Header %%%%%%%%%%%%%%%%%%%%%%%%%%%%%%%%%%%%%%%%%%%%%%%%%

%\usepackage{fancyhdr}
%\pagestyle{fancy}
%\lhead{}
%\rhead{}
%\chead{}
%\setlength{\headheight}{15.2pt}
    %   Make the header bigger to avoid overlap

%\fancyhf{}
    %   Erase header settings

%\renewcommand{\headrulewidth}{0.3pt}
    %   Width of the line

%\setlength{\headsep}{0.2in}
    %   Distance from line to text


%% Mathematics Related %%%%%%%%%%%%%%%%%%%%%%%%%%%%%%%%%%%

\usepackage{amsmath}
\usepackage{amssymb}
\usepackage{amsfonts}
\usepackage{mathrsfs}
\usepackage{mathtools}
\usepackage{amsthm} %allows for labeling of theorems
%\numberwithin{equation}{section} % Number equations by section
\usepackage{bbm} % For bold numbers

\theoremstyle{plain}
\newtheorem{thm}{Theorem}[section]
\newtheorem{lem}[thm]{Lemma}
\newtheorem{prop}[thm]{Proposition}
\newtheorem{cor}[thm]{Corollary}

\theoremstyle{definition}
\newtheorem{defn}[thm]{Definition}
\newtheorem{ex}[thm]{Example}

\theoremstyle{remark}
\newtheorem*{rmk}{Remark}
\newtheorem*{note}{Note}

% Below supports left-right alignment in matrices so the negative
% signs don't look bad
\makeatletter
\renewcommand*\env@matrix[1][c]{\hskip -\arraycolsep
  \let\@ifnextchar\new@ifnextchar
  \array{*\c@MaxMatrixCols #1}}
\makeatother


%% Font Choices %%%%%%%%%%%%%%%%%%%%%%%%%%%%%%%%%%%%%%%%%

\usepackage[T1]{fontenc}
\usepackage{lmodern}
\usepackage[utf8]{inputenc}
%\usepackage{blindtext}
\usepackage{courier}


%% Figures %%%%%%%%%%%%%%%%%%%%%%%%%%%%%%%%%%%%%%%%%%%%%%

\usepackage{tikz}
\usetikzlibrary{decorations.pathreplacing}
\usetikzlibrary{arrows.meta}
\usepackage{graphicx}
\usepackage{subfigure}
    %   For plotting multiple figures at once
%\graphicspath{ {Directory/} }
    %   Set a directory for where to look for figures


%% Hyperlinks %%%%%%%%%%%%%%%%%%%%%%%%%%%%%%%%%%%%%%%%%%%%
\usepackage{hyperref}
\hypersetup{%
    colorlinks,
        %   This colors the links themselves, not boxes
    citecolor=black,
        %   Everything here and below changes link colors
    filecolor=black,
    linkcolor=black,
    urlcolor=black
}

%% Colors %%%%%%%%%%%%%%%%%%%%%%%%%%%%%%%%%%%%%%%%%%%%%%%

\usepackage{color}
\definecolor{codegreen}{RGB}{28,172,0}
\definecolor{codelilas}{RGB}{170,55,241}

% David4 color scheme
\definecolor{d4blue}{RGB}{100,191,255}
\definecolor{d4gray}{RGB}{175,175,175}
\definecolor{d4black}{RGB}{85,85,85}
\definecolor{d4orange}{RGB}{255,150,100}

%% Including Code %%%%%%%%%%%%%%%%%%%%%%%%%%%%%%%%%%%%%%%

\usepackage{verbatim}
    %   For including verbatim code from files, no colors
\usepackage{listings}
    %   For including code snippets written directly in this doc

\lstdefinestyle{bash}{%
  language=bash,%
  basicstyle=\footnotesize\ttfamily,%
  showstringspaces=false,%
  commentstyle=\color{gray},%
  keywordstyle=\color{blue},%
  xleftmargin=0.25in,%
  xrightmargin=0.25in
}
\lstdefinestyle{log}{%
  basicstyle=\scriptsize\ttfamily,%
  showstringspaces=false,%
  xleftmargin=0.25in,%
  xrightmargin=0.25in
}


\lstdefinestyle{matlab}{%
  language=Matlab,%
  basicstyle=\footnotesize\ttfamily,%
  breaklines=true,%
  morekeywords={matlab2tikz},%
  keywordstyle=\color{blue},%
  morekeywords=[2]{1}, keywordstyle=[2]{\color{black}},%
  identifierstyle=\color{black},%
  stringstyle=\color{codelilas},%
  commentstyle=\color{codegreen},%
  showstringspaces=false,%
    %   Without this there will be a symbol in
    %   the places where there is a space
  %numbers=left,%
  %numberstyle={\tiny \color{black}},%
    %   Size of the numbers
  numbersep=9pt,%
    %   Defines how far the numbers are from the text
  emph=[1]{for,end,break,switch,case},emphstyle=[1]\color{blue},%
    %   Some words to emphasise
}

\newcommand{\matlabcode}[1]{%
    \lstset{style=matlab}%
    \lstinputlisting{#1}
}
    %   For including Matlab code from .m file with colors,
    %   line numbering, etc.

\lstdefinelanguage{Julia}%
  {morekeywords={abstract,break,case,catch,const,continue,do,else,elseif,%
      end,export,false,for,function,immutable,import,importall,if,in,%
      macro,module,otherwise,quote,return,switch,true,try,type,typealias,%
      using,while},%
   sensitive=true,%
   %alsoother={$},%
   morecomment=[l]\#,%
   morecomment=[n]{\#=}{=\#},%
   morestring=[s]{"}{"},%
   morestring=[m]{'}{'},%
}[keywords,comments,strings]

\lstdefinestyle{julia}{%
    language         = Julia,
    basicstyle       = \scriptsize\ttfamily,
    keywordstyle     = \bfseries\color{blue},
    stringstyle      = \color{codegreen},
    commentstyle     = \color{codegreen},
    showstringspaces = false,
    literate         = %
      {ρ}{{$\rho$}}1
      {ℓ}{{$\ell$}}1
      {∑}{{$\Sigma$}}1
      {Σ}{{$\Sigma$}}1
      {√}{{$\sqrt{}$}}1
      {θ}{{$\theta$}}1
      {ω}{{$\omega$}}1
      {ν}{{$\nu$}}1
      {ω²}{{$\omega^2$}}1
      {ɛ}{{$\varepsilon$}}1
      {φ}{{$\varphi$}}1
      {σ²}{{$\sigma^2$}}1
      {Φ}{{$\Phi$}}1
      {ϕ}{{$\phi$}}1
      {Dₑ}{{$D_e$}}1
      {Σ}{{$\Sigma$}}1
      {γ}{{$\gamma$}}1
      {δ}{{$\delta$}}1
      {τ}{{$\tau$}}1
      {μ}{{$\mu$}}1
      {û}{{$\hat{u}$}}1
      {μ̂}{{$\hat{\mu}$}}1
      {x̂}{{$\hat{x}$}}1
      {ω̂}{{$\hat{\omega}$}}1
      {ω̂²}{{$\hat{\omega}^2$}}1
      {β}{{$\beta$}}1
      {Λ}{{$\Lambda$}}1
      {λ}{{$\lambda$}}1
      {r̃}{{$\tilde{\text{r}}$}}1
      {α}{{$\alpha$}}1
      {σ}{{$\sigma$}}1
      {σ²}{{$\sigma^2$}}1
      {π}{{$\pi$}}1
      {∈}{{$\in$}}1
      {∞}{{$\infty$}}1
}


%% Bibliographies %%%%%%%%%%%%%%%%%%%%%%%%%%%%%%%%%%%%

%\usepackage{natbib}
    %---For bibliographies
%\setlength{\bibsep}{3pt} % Set how far apart bibentries are

%% Misc %%%%%%%%%%%%%%%%%%%%%%%%%%%%%%%%%%%%%%%%%%%%%%

\usepackage{enumitem}
    %   Has to do with enumeration
\usepackage{appendix}
%\usepackage{natbib}
    %   For bibliographies
\usepackage{pdfpages}
    %   For including whole pdf pages as a page in doc
\usepackage{pgffor}
    %   For easier looping


%% User Defined %%%%%%%%%%%%%%%%%%%%%%%%%%%%%%%%%%%%%%%%%%

%\newcommand{\nameofcmd}{Text to display}
\newcommand*{\Chi}{\mbox{\large$\chi$}} %big chi
    %   Bigger Chi

% In math mode, Use this instead of \munderbar, since that changes the
% font from math to regular
\makeatletter
\def\munderbar#1{\underline{\sbox\tw@{$#1$}\dp\tw@\z@\box\tw@}}
\makeatother

% Misc Math
\newcommand{\ra}{\rightarrow}
\newcommand{\diag}{\text{diag}}
\newcommand{\ch}{\text{ch}}
\newcommand{\dom}{\text{dom}}
\newcommand{\one}[1]{\mathbf{1}_{#1}}


% Command to generate new math commands:
% - Suppose you want to refer to \boldsymbol{x} as just \bsx, where 'x'
%   is any letter. This commands lets you generate \bsa, \bsb, etc.
%   without copy pasting \newcommand{\bsa}{\boldsymbol{a}} for each
%   letter individually. Instead, just include
%
%     \generate{bs}{\boldsymbol}{a,...,z}
%
% - Uses pgffor package to loop
% - Example with optional argument. Will generate \bshatx to represent
%   \boldsymbol{\hat{x}} for all letters x
%
%     \generate[\hat]{bshat}{\boldsymbol}{a,...,z}

\newcommand{\generate}[4][]{%
  % Takes 3 arguments (maybe four):
  % - 1   wrapcmd (optional, defaults to nothing)
  % - 2   newname
  % - 3   mathmacro
  % - 4   Names to loop over
  %
  % Will produce
  %
  %   \newcommand{\newnameX}{mathmacro{wrapcmd{X}}}
  %
  % for each X in argument 4

  \foreach \x in {#4}{%
    \expandafter\xdef\csname%
      #2\x%
    \endcsname%
    {\noexpand\ensuremath{\noexpand#3{\noexpand#1{\x}}}}
  }
}


% MATHSCR: Gen \sX to stand for \mathscr{X} for all upper case letters
\generate{s}{\mathscr}{A,...,Z}


% BOLDSYMBOL: Generate \bsX to stand for \boldsymbol{X}, all upper and
% lower case.
%
% Letters and greek letters
\generate{bs}{\boldsymbol}{a,...,z}
\generate{bs}{\boldsymbol}{A,...,Z}
\newcommand{\bstheta}{\boldsymbol{\theta}}
\newcommand{\bsmu}{\boldsymbol{\mu}}
\newcommand{\bsSigma}{\boldsymbol{\Sigma}}
\newcommand{\bsvarepsilon}{\boldsymbol{\varepsilon}}
\newcommand{\bstildevarepsilon}{\boldsymbol{\tilde{\varepsilon}}}
\newcommand{\bsalpha}{\boldsymbol{\alpha}}
\newcommand{\bsbeta}{\boldsymbol{\beta}}
\newcommand{\bsOmega}{\boldsymbol{\Omega}}
\newcommand{\bshatOmega}{\boldsymbol{\hat{\Omega}}}
\newcommand{\bshatG}{\boldsymbol{\hat{G}}}
\newcommand{\bsgamma}{\boldsymbol{\gamma}}
\newcommand{\bslambda}{\boldsymbol{\lambda}}

% Special cases like \bshatb for \boldsymbol{\hat{b}}
\generate[\hat]{bshat}{\boldsymbol}{b,y,x,X,V,S,W}
\newcommand{\bshatbeta}{\boldsymbol{\hat{\beta}}}
\newcommand{\bshatmu}{\boldsymbol{\hat{\mu}}}
\newcommand{\bshattheta}{\boldsymbol{\hat{\theta}}}
\newcommand{\bshatSigma}{\boldsymbol{\hat{\Sigma}}}
\newcommand{\bstildebeta}{\boldsymbol{\tilde{\beta}}}
\newcommand{\bstildey}{\boldsymbol{\tilde{y}}}
\newcommand{\bstildeX}{\boldsymbol{\tilde{X}}}
\newcommand{\bstildetheta}{\boldsymbol{\tilde{\theta}}}
\newcommand{\bsbarbeta}{\boldsymbol{\overline{\beta}}}
\newcommand{\bsbarg}{\boldsymbol{\overline{g}}}

% Redefine \bso to be the zero vector
\renewcommand{\bso}{\boldsymbol{0}}

% Transposes of all the boldsymbol shit
\newcommand{\bsbp}{\boldsymbol{b'}}
\newcommand{\bshatbp}{\boldsymbol{\hat{b'}}}
\newcommand{\bsdp}{\boldsymbol{d'}}
\newcommand{\bsgp}{\boldsymbol{g'}}
\newcommand{\bsGp}{\boldsymbol{G'}}
\newcommand{\bshp}{\boldsymbol{h'}}
\newcommand{\bsSp}{\boldsymbol{S'}}
\newcommand{\bsup}{\boldsymbol{u'}}
\newcommand{\bsxp}{\boldsymbol{x'}}
\newcommand{\bsyp}{\boldsymbol{y'}}
\newcommand{\bsthetap}{\boldsymbol{\theta'}}
\newcommand{\bsmup}{\boldsymbol{\mu'}}
\newcommand{\bsSigmap}{\boldsymbol{\Sigma'}}
\newcommand{\bshatmup}{\boldsymbol{\hat{\mu'}}}
\newcommand{\bshatSigmap}{\boldsymbol{\hat{\Sigma'}}}

% MATHCAL: Gen \calX to stand for \mathcal{X}, all upper case
\generate{cal}{\mathcal}{A,...,Z}

% MATHBB: Gen \X to stand for \mathbb{X} for some upper case
\generate{}{\mathbb}{R,Q,C,Z,N,Z,E}
\newcommand{\Rn}{\mathbb{R}^n}
\newcommand{\RN}{\mathbb{R}^N}
\newcommand{\Rk}{\mathbb{R}^k}
\newcommand{\RK}{\mathbb{R}^K}
\newcommand{\RL}{\mathbb{R}^L}
\newcommand{\Rl}{\mathbb{R}^\ell}
\newcommand{\Rm}{\mathbb{R}^m}
\newcommand{\Rnn}{\mathbb{R}^{n\times n}}
\newcommand{\Rmn}{\mathbb{R}^{m\times n}}
\newcommand{\Rnm}{\mathbb{R}^{n\times m}}
\newcommand{\Rkn}{\mathbb{R}^{k\times n}}
\newcommand{\Cn}{\mathbb{C}^n}
\newcommand{\Cnn}{\mathbb{C}^{n\times n}}

% Dot over
\newcommand{\dx}{\dot{x}}
\newcommand{\ddx}{\ddot{x}}
\newcommand{\dy}{\dot{y}}
\newcommand{\ddy}{\ddot{y}}

% First derivatives
\newcommand{\dydx}{\frac{dy}{dx}}
\newcommand{\dfdx}{\frac{df}{dx}}
\newcommand{\dfdy}{\frac{df}{dy}}
\newcommand{\dfdz}{\frac{df}{dz}}

% Second derivatives
\newcommand{\ddyddx}{\frac{d^2y}{dx^2}}
\newcommand{\ddydxdy}{\frac{d^2y}{dx dy}}
\newcommand{\ddydydx}{\frac{d^2y}{dy dx}}
\newcommand{\ddfddx}{\frac{d^2f}{dx^2}}
\newcommand{\ddfddy}{\frac{d^2f}{dy^2}}
\newcommand{\ddfddz}{\frac{d^2f}{dz^2}}
\newcommand{\ddfdxdy}{\frac{d^2f}{dx dy}}
\newcommand{\ddfdydx}{\frac{d^2f}{dy dx}}


% First Partial Derivatives
\newcommand{\pypx}{\frac{\partial y}{\partial x}}
\newcommand{\pfpx}{\frac{\partial f}{\partial x}}
\newcommand{\pfpy}{\frac{\partial f}{\partial y}}
\newcommand{\pfpz}{\frac{\partial f}{\partial z}}


% argmin and argmax
\DeclareMathOperator*{\argmin}{arg\;min}
\DeclareMathOperator*{\argmax}{arg\;max}


% Various probability and statistics commands
\newcommand{\iid}{\overset{iid}{\sim}}
\newcommand{\vc}{\operatorname{vec}}
\newcommand{\Cov}{\operatorname{Cov}}
\newcommand{\rank}{\operatorname{rank}}
\newcommand{\trace}{\operatorname{trace}}
\newcommand{\Corr}{\operatorname{Corr}}
\newcommand{\Var}{\operatorname{Var}}
\newcommand{\asto}{\xrightarrow{a.s.}}
\newcommand{\pto}{\xrightarrow{p}}
\newcommand{\msto}{\xrightarrow{m.s.}}
\newcommand{\dto}{\xrightarrow{d}}
\newcommand{\Lpto}{\xrightarrow{L_p}}
\newcommand{\Lqto}[1]{\xrightarrow{L_{#1}}}
\newcommand{\plim}{\text{plim}_{n\rightarrow\infty}}


% Redefine real and imaginary from fraktur to plain text
\renewcommand{\Re}{\operatorname{Re}}
\renewcommand{\Im}{\operatorname{Im}}

% Shorter sums: ``Sum from X to Y''
% - sumXY  is equivalent to \sum^Y_{X=1}
% - sumXYz is equivalent to \sum^Y_{X=0}
\newcommand{\sumnN}{\sum^N_{n=1}}
\newcommand{\summM}{\sum^M_{m=1}}
\newcommand{\sumin}{\sum^n_{i=1}}
\newcommand{\sumjn}{\sum^n_{j=1}}
\newcommand{\sumim}{\sum^m_{i=1}}
\newcommand{\sumik}{\sum^k_{i=1}}
\newcommand{\sumiN}{\sum^N_{i=1}}
\newcommand{\sumkn}{\sum^n_{k=1}}
\newcommand{\sumtT}{\sum^T_{t=1}}
\newcommand{\sumninf}{\sum^\infty_{n=1}}
\newcommand{\sumtinf}{\sum^\infty_{t=1}}
\newcommand{\sumnNz}{\sum^N_{n=0}}
\newcommand{\suminz}{\sum^n_{i=0}}
\newcommand{\sumknz}{\sum^n_{k=0}}
\newcommand{\sumtTz}{\sum^T_{t=0}}
\newcommand{\sumninfz}{\sum^\infty_{n=0}}
\newcommand{\sumtinfz}{\sum^\infty_{t=0}}

\newcommand{\prodnN}{\prod^N_{n=1}}
\newcommand{\prodin}{\prod^n_{i=1}}
\newcommand{\prodiN}{\prod^N_{i=1}}
\newcommand{\prodkn}{\prod^n_{k=1}}
\newcommand{\prodtT}{\prod^T_{t=1}}
\newcommand{\prodnNz}{\prod^N_{n=0}}
\newcommand{\prodinz}{\prod^n_{i=0}}
\newcommand{\prodknz}{\prod^n_{k=0}}
\newcommand{\prodtTz}{\prod^T_{t=0}}

% Bounds
\newcommand{\atob}{_a^b}
\newcommand{\ztoinf}{_0^\infty}
\newcommand{\kinf}{_{k=1}^\infty}
\newcommand{\ninf}{_{n=1}^\infty}
\newcommand{\minf}{_{m=1}^\infty}
\newcommand{\tinf}{_{t=1}^\infty}
\newcommand{\nN}{_{n=1}^N}
\newcommand{\tT}{_{t=1}^T}
\newcommand{\kinfz}{_{k=0}^\infty}
\newcommand{\ninfz}{_{n=0}^\infty}
\newcommand{\minfz}{_{m=0}^\infty}
\newcommand{\tinfz}{_{t=0}^\infty}
\newcommand{\nNz}{_{n=0}^N}

% Limits
\newcommand{\limN}{\lim_{N\rightarrow\infty}}
\newcommand{\limn}{\lim_{n\rightarrow\infty}}
\newcommand{\limk}{\lim_{k\rightarrow\infty}}
\newcommand{\limt}{\lim_{t\rightarrow\infty}}
\newcommand{\limT}{\lim_{T\rightarrow\infty}}
\newcommand{\limhz}{\lim_{h\rightarrow 0}}

% Shorter integrals: ``Integral from X to Y''
% - intXY is equivalent to \int^Y_X
\newcommand{\intab}{\int_a^b}
\newcommand{\intzN}{\int_0^N}


%%%%%%%%%%%%%%%%%%%%%%%%%%%%%%%%%%%%%%%%%%%%%%%%%%%%%%%%%%%%%%%%%%%%%%%%
%% BODY %%%%%%%%%%%%%%%%%%%%%%%%%%%%%%%%%%%%%%%%%%%%%%%%%%%%%%%%%%%%%%%%
%%%%%%%%%%%%%%%%%%%%%%%%%%%%%%%%%%%%%%%%%%%%%%%%%%%%%%%%%%%%%%%%%%%%%%%%


\begin{document}
\maketitle

\section{Ito's Lemma}

\begin{lem}\emph{(Ito's Lemma)}
Let $X_t$ be an Ito process with associated SDE
\begin{align}
   dx_t = \mu(x_t,t)\; dt + \sigma(x_t,t)\; dw_t
   \label{itoslemmadiffusion}
\end{align}
Suppose we change variables $y_t=f(x_t,t)$.
Then $y_t=f(x_t,t)$ also an Ito process with SDE
\begin{align}
  dY_t
  =
  \partial_t f \,dt
  +
  \partial_x f \,dx_t
  +
  \frac{1}{2}
  +
  \partial_{xx} f \,(dx_t)^2
  \\
   dY_t
   = df(X_t,t)
   =
   \frac{\partial f}{\partial x} \; dX_t
   + \frac{\partial f}{\partial t} \; dt
   + \frac{1}{2}
    \frac{\partial^2 f}{\partial x^2} \; \sigma^2(X_t,t) \; dX_t^2
   \label{itoslemma0}
\end{align}
Note that this can be rewritten into the standard representation of
Ito's Lemma by using $dX_t^2=dt$, substituting in
Expression~\ref{itoslemmadiffusion} for $dX_t$, and then collecting
terms to get
\begin{align}
  \label{itoslemma}
  dY_t
  = df(X_t,t)
  &=
  \left[
  \mu(X_t,t)
  \; \partial_x f(X_t,t)
  +
  \partial_t f(X_t,t)
  + \frac{1}{2}
    \sigma^2(X_t,t)
    \;\partial_{xx}
    f(X_t,t)
  \right]
  \; dt
  \\
  &\qquad
  +
  \sigma(X_t,t)
  \;
  \partial_x f(X_t,t)\; dW_t
\end{align}
\end{lem}
\begin{rmk}
Notice that Expression~\ref{itoslemma0} looks like the standard total
derivative of $f$, just with an extra second order term, which normally
disappears in a nonstochastic world but must here be retained because
argument $X_t$ is in fact stochastic.
\end{rmk}


%\begin{lem}\emph{(Multi-Dimensional Ito's Lemma)}
%Let $\bsX_t$ be a $N$-dimensional Ito process with associated SDE
%\begin{align*}
  %d\bsX_t
  %= \underbrace{\bsmu(\bsX_t,t)}_{(N\times 1)}\;dt
  %+ \underbrace{\bssigma(\bsX_t,t)}_{(N\times N)}\;d\bsW_t
%\end{align*}
%Suppose we change variables $Y_t=(\bsX_t,t)$ where
%$\bsf: \R^N\times [0,\infty) \rightarrow \R$ is some function twice
%differentiable in $\bsx$, once differentiable in $t$.
%Then $Y_t=f(\bsX_t,t)$ is a also an Ito diffusion satisfying the
%following SDE\footnote{%
  %Given in three different, equivalent representations since the result
  %is often presented in any of the following ways.
%}
%\begin{align}
   %dY_t
   %= df(\bsX_t,t)
   %&=
   %\frac{\partial f}{\partial \bsx'} \; d\bsX_t
   %+ \frac{\partial f}{\partial t} \; dt
   %+ \frac{1}{2}
    %\frac{\partial^2 f}{\partial\bsx\partial\bsx'} \; \sigma^2(\bsX_t,t)
    %\;dt
%\end{align}
%\begin{align}
  %dY_t
  %= df(\bsX_t,t)
  %&=
  %\left[
  %\bsmu(\bsX_t,t)'\,
  %\frac{\partial f}{\partial \bsx}
  %+
  %\frac{\partial f}{\partial t}
  %+
  %\frac{1}{2}
  %\bssigma(\bsX_t,t)\bssigma(\bsX_t,t)'
  %\frac{\partial^2 f}{\partial \bsx\partial\bsx'}
  %\right]
  %\;dt
  %+
  %\bssigma(\bsX_t,t)'
  %\frac{\partial f}{\partial \bsx}
  %\;d\bsW_t
  %\\
  %dY_t
  %= df(\bsX_t,t)
  %&=
  %\left[
  %\bsmu(\bsX_t,t)
  %\cdot
  %\nabla_x
  %+
  %\nabla_t
  %+
  %\frac{1}{2}
  %\bssigma(\bsX_t,t)\bssigma(\bsX_t,t)'
  %\nabla_x^2
  %\right]
  %f(\bsX_t,t)
  %\;dt
  %+
  %\bssigma(\bsX_t,t)\cdot
  %\nabla_x
  %f(\bsX_t,t)
  %\;d\bsW_t
%\end{align}
%where all derivatives are evaluated at $(X_t,t)$.

%For the sake of completeness, let's generalize Ito's Lemma to consider
%the case of a finite number of Ito processes, $X_1, X_2, \ldots, X_n$,
   %\[ dX_i(t) = \mu_i(t)\; dt + \sigma_i(t) \; dW_i(t).\]
%Next, define $Y(t) = f(X_1(t), \ldots, X_n(t), t)$ for some
%differentiable function $f$.  Then multi-dimensional Ito's Lemma says
%\begin{align*}
    %dY(t) &= \sum^n_{i=1} f_{X_i}(X_1(t), \ldots, X_n(t), t) \;
      %dX_i(t) \\
    %&+ f_{t}(X_1(t), \ldots, X_n(t), t) \; dt \\
    %&+ \frac{1}{2} \sum^n_{i=1}\sum^n_{j=1}
      %f_{X_i,X_j}(X_1(t), \ldots, X_n(t), t) \; \rho_{ij}\sigma_i(t)
      %\sigma_j(t) \; dt
%\end{align*}
%where $\rho_{ij}$ is the correlation coefficient between $dW_i$ and
%$dW_j$. Note that you'll have to plug back in for the $dX_i$ in the
%first sum.
%\\
%\\
%We'll mostly consider with the two-dimensional case for the
%two specific instances below:
%\begin{itemize}
   %\item[i.]{$Y(t) = X_1(t)X_2(t)$, which gives us
      %\[ dY(t) = X_2(t) \; dX_1(t) + X_1(t) \; dX_2(t) +
   %\rho_{12}\sigma_1(t)\sigma_2(t) \; dt \]
      %Note that you'll have to plug back in for $dX_1$ and $dX_2$.
      %This is a type of integration-by-parts formula because (after
      %rearranging terms) it relates $X_1\; dX_2$ to $X_2\; dX_1$.
   %}

   %\item[ii.]{$Y(t) = X_1(t)/X_2(t)$, which gives us
      %\begin{align*}
   %dY(t) = \frac{1}{X_2(t)} dX_1(t) - \frac{X_1(t)}{X_2(t)^2}
   %dX_2(t) +  \frac{X_1(t)}{X_2(t)^3} \sigma_2(t)^2 \;dt -
   %\frac{1}{X_2(t)^2} \; \rho_{12}\sigma_1(t)\sigma_2(t) \; dt.
      %\end{align*}
      %Note that you'll have to plug back in for $dX_1$ and $dX_2$.
      %This is a type of integration-by-parts formula because (after
      %rearranging terms) it relates $X_1\; dX_2$ to $X_2\; dX_1$.
      %}
%\end{itemize}

%\end{lem}


\clearpage
\subsection{Forward and Backward Equations}

In the sections above, we considered time-homogeneous SDEs of the form
\begin{align}
  dX_t = b(X_t) dt + \sigma(X_t) dW_t
  \label{kolmogorovsde}
\end{align}
which we tried to solve for the process $X_t$, writing $X_t$ as a
function of primitives $(t,W_t)$.
This section pursues another approach, instead using the SDE to
characterize the evolution
\begin{enumerate}
  \item The probability distribution of $X_t$ via the
    \emph{Forward Equation}
  \item Functions or moments of $X_t$ via the \emph{Backard Equation}
\end{enumerate}
And instead of solving SDE~\ref{kolmogorovsde}, we solve PDEs.
%Since $X_t$ is Markov (by the fact that it solves
%the above SDE with particular smoothness and growth conditions on $b$
%and $\sigma$), we will often deal with the transition density:
%\begin{align*}
  %p(x,t|y,s) := P(X_t\in [x,x+dx) \; | \; X_s=y)
%\end{align*}

\begin{defn}(Generator of a Process)
Given a continuous function $f$ and a time-homogeneous process
(multi-dimensional) stochastic process $X_t$, define the
\emph{generator} of $X_t$
\begin{align}
  (\mathscr{A}f)(x)
  :=
  \lim_{t\rightarrow0} \frac{E[f(X_t)|X_0=x]-f(x)}{t}
  \label{genA}
\end{align}
In words, the generator of a time-homogeneous process is the
expected infinitesimal change of some function of that process.
\end{defn}

\begin{thm}\emph{(Functional Form of Generator)}
Given a time homogeneous process $X_t\in\R^N$ satisfying
SDE~\ref{kolmogorovsde}, the generator $\mathscr{A}$ of $X_t$ satisfies
$\mathscr{L}$
\begin{align}
  \mathscr{A}f
  &= b \cdot \nabla f
  + \frac{1}{2}
  \trace\left(\sigma \sigma^T\nabla^2 f\right)
  \label{genL} \\
  \iff \qquad
  &= \sumnN b_n \frac{\partial f}{\partial x_n}
  + \frac{1}{2} \sumnN \summM
    \left(\sigma \sigma^T\right)
    \frac{\partial^2 f}{\partial x_n \partial x_m}\notag
\end{align}
where $\nabla^2 f$ is the Hessian matrix of $f$.
\end{thm}
\begin{rmk}
Though $\mathscr{A}f = \mathscr{L}f$, we use separate notation to
distinguish that $\mathscr{A}$ and $\mathscr{L}$ come from different
%places: a limit as in (\ref{genA}) versus a simple lifting of
coefficients from an SDE as in (\ref{genL}), respectively.
\end{rmk}
\begin{proof}
The result follows directly from Ito's Lemma. Therefore apply
multi-dimensional Ito's Lemma to $f(X_t)$ for homogeneous process $X_t$:
\begin{align*}
  df(X_t) &=
  \left(b\cdot \nabla f + \frac{1}{2}\sigma\sigma^T : \nabla^2 f\right)
  dt
  + \nabla f\cdot \sigma \;dW_t\\
  f(X_t) - f(X_s) &=
  \int^t_s
  \left(b\cdot \nabla f + \frac{1}{2}\sigma\sigma^T : \nabla^2 f\right)
  d\tau
  +
  \int^t_s
  \nabla f\cdot \sigma \;dW_\tau
\end{align*}
Now take conditional expectation $E[\;\cdot\;|X_s=x]$ and use the fact
that the Ito Integral has expectation of zero:
\begin{align}
  E[f(X_t)|f(X_s)=x] - f(x) &=
  E\left[
  \int^t_s
  \left(b\cdot \nabla f + \frac{1}{2}\sigma\sigma^T : \nabla^2 f\right)
  d\tau
  \; \big|\; f(X_s)=x\right]\notag\\
  &\qquad
  +
  E\left[
  \int^t_s
  \nabla f\cdot \sigma \;dW_\tau
  \; \big|\;f(X_s)=x\right]
  \notag\\
  &=
  E\left[ \int^t_s (\mathscr{L} f)(X_\tau) d\tau
    \;\big|\:f(X_s)=x\right] + 0
  \label{genL.proof}
\end{align}
From there, form the limit that defines the generator:
\begin{align*}
  (\mathscr{A}f)(x)
  &=\lim_{t\rightarrow0} \frac{E[f(X_t)\;|\;f(X_s)=x] - f(x)}{t}\\
  \text{Sub in (\ref{genL.proof})} \Rightarrow\qquad
  &=\lim_{t\rightarrow0} \frac{1}{t}
  E\left[ \int^t_s (\mathscr{L} f)(X_\tau) d\tau \; \big|\:f(X_s)=x\right]\\
  &=(\mathscr{L}f)(x)
\end{align*}
where the last step follows from the Dominated Convergence Theorem since
$f$ and all of its derivatives are bounded (K\&S p 323).
\end{proof}

\begin{thm}{(Forward Kolmogorov Equation and Fokker-Planck Equation)}
There are two formulations, both of which make use of $\mathscr{L}^*$,
defined as
\begin{align*}
  \mathscr{L}^*f_x = -\nabla_x \cdot (b f)
  +\frac{1}{2}\nabla^2_x : (a f)
\end{align*}
The two formulations are as follows:
\begin{enumerate}
  \item The transition probability density $p(x,t|y,s)$ solves
    \begin{align*}
      \frac{\partial p}{\partial t} = \mathscr{L}^*_x p
      \qquad p(x,s|y,s) = \delta(x-y)
    \end{align*}

  \item If $\rho_0(x)$ is the initial probability desnity and
    $\rho(x,t)$ is the density at time $t$, then $\rho(x,t)$ solves
    \begin{align}
      \frac{\partial \rho}{\partial t} = \mathscr{L}^* \rho
      \qquad \text{where } \; \rho(x,0) = \rho_0(x)
      \label{fpeqn}
    \end{align}
\end{enumerate}
\begin{rmk}
$\mathscr{L}^*$ is the adjoint of $\mathscr{L}$ meaning that
on the Hilbert Space with inner product
$\langle\cdot,\cdot\rangle_{L^2}$,\footnote{%
  The inner product $\langle\cdot,\cdot\rangle_{L^2}$ is defined as
  \begin{align*}
    \langle f,g\rangle_{L^2}
    &= \int_X fg\;d\mu
  \end{align*}
}
we have that $\langle\mathscr{L}f,g\rangle_{L^2} =\langle
f,\mathscr{L}^*g\rangle_{L^2}$
\end{rmk}
\end{thm}
\begin{proof}
We take each case separately:
\begin{enumerate}
  \item By Ito's Lemma and results we saw above, we can write:
    \begin{align*}
      E[f(X_t) \;|\; X_s=y] -f(y)
      &= E\left[\int^t_s (\mathscr{L}f(X_\tau) d\tau \;\big|\;
          f(X_s)=y\right]
    \end{align*}
    Writing the expectation out explicitly,
    \begin{align*}
      \int_{\mathbb{R}^d} f(x)p(x,t|y,s)\;dx  -f(y)
      &= \int_{\mathbb{R}^d} \int^t_s
        (\mathscr{L}f)(x) \cdot p(x,\tau|y,s)\; d\tau dx
    \end{align*}
    Differentiating both sides with respect to $t$,
    \begin{align*}
      \int_{\mathbb{R}^d} f(x) \frac{\partial p}{\partial t}\;dx
      &= \int_{\mathbb{R}^d}
        (\mathscr{L}f)(x) \cdot p\; dx
    \end{align*}
    And this holds for all test functions $f$, so $p$ is a weak
    solution.
\end{enumerate}
\end{proof}

\begin{thm}{\emph{(Backward Komogorov Equation)}}
There are 3 formulations. For each, let $p(x,t|y,s)$ denote the
transition probability of $X_t$, and for a function $f\in
C^2(\mathbb{R}^d)$, Then it is the case that
\begin{enumerate}
  \item Define $u_t^{(x)} := \mathbb{E}[f(X_t)|X_0=x]$ for a
    time-homogeneous process $X_t$. Then given boundary condition
    $u^{(x)}_0=f(x)$,
    \begin{align*}
      \frac{\partial u^{(x)}_t}{\partial t} = \mathscr{L} u^{(x)}_t
      \qquad t>0
    \end{align*}
  \item For a possibly time-inhomogeneous SDE with solution $X_t$,
    define $v^{(y)}_t:=E[f(X_T)|X_t~=~y]$. Then given boundary
    condition $v^{(y)}_T=f(y)$, we have
    \begin{align*}
      \frac{\partial v^{(y)}_t}{\partial t}
      + \mathscr{L} v^{(y)}_t = 0
      \qquad t < T
    \end{align*}
  \item For a Markov process solving a time-inhomogeneous SDE
    and for boundary condition $p(y,t|x,s) = \delta(x-y)$,
    \begin{align*}
      \frac{\partial p}{\partial s} + \mathscr{L}_y p = 0
      \qquad s<t
    \end{align*}
    where $\mathscr{L}_y$ denotes that the derivatives in $\mathscr{L}$
    are with respect to variable $y$.
\end{enumerate}
\end{thm}
\begin{rmk}
The second and third formulations should be solved \emph{backward} in
time, which gives the theorem its name.
\end{rmk}
\begin{proof}
We prove each in turn:
  \begin{enumerate}
    \item Recall from Ito's Lemma that for some stochastic process $X_t$
      and some twice-differentiable function $g: \mathbb{R}^d \times
      [0,\infty) \rightarrow \mathbb{R}^d$ (a function of both $X_t$ and
      $t$), we have
      \begin{align*}
        d\left(g(X_t,t)\right)
        &=
        \left(
        \frac{\partial g}{\partial t}
        + b \cdot \nabla_x g
        + \frac{1}{2} \sigma \sigma^T : \nabla_x^2 g
        \right)dt
        +
        \frac{\partial g}{\partial x} \sigma dW_t
      \end{align*}
    \item Relabel time as $t' = t-s$ and consider the function
      $u(x,t')$.
  \end{enumerate}
\end{proof}





\end{document}


%%%%%%%%%%%%%%%%%%%%%%%%%%%%%%%%%%%%%%%%%%%%%%%%%%%%%%%%%%%%%%%%%%%%%%%%
%%%%%%%%%%%%%%%%%%%%%%%%%%%%%%%%%%%%%%%%%%%%%%%%%%%%%%%%%%%%%%%%%%%%%%%%
%%%%%%%%%%%%%%%%%%%%%%%%%%%%%%%%%%%%%%%%%%%%%%%%%%%%%%%%%%%%%%%%%%%%%%%%

%%%% SAMPLE CODE %%%%%%%%%%%%%%%%%%%%%%%%%%%%%%%%%%%%%%

    %% VIEW LAYOUT %%

        \layout

    %% LANDSCAPE PAGE %%

        \begin{landscape}
        \end{landscape}

    %% BIBLIOGRAPHIES %%

        \cite{LabelInSourcesFile}  %Use in text; cites
        \citep{LabelInSourcesFile} %Use in text; cites in parens

        \nocite{LabelInSourceFile} % Includes in refs w/o specific citation
        \bibliographystyle{apalike}  % Or some other style

        % To ditch the ``References'' header
        \begingroup
        \renewcommand{\section}[2]{}
        \endgroup

        \bibliography{sources} % where sources.bib has all the citation info

    %% SPACING %%

        \vspace{1in}
        \hspace{1in}

    %% URLS, EMAIL, AND LOCAL FILES %%

      \url{url}
      \href{url}{name}
      \href{mailto:mcocci@raidenlovessusie.com}{name}
      \href{run:/path/to/file.pdf}{name}


    %% INCLUDING PDF PAGE %%

        \includepdf{file.pdf}


    %% INCLUDING CODE %%

        %\verbatiminput{file.ext}
            %   Includes verbatim text from the file

        \texttt{text}
            %   Renders text in courier, or code-like, font

        \matlabcode{file.m}
            %   Includes Matlab code with colors and line numbers

        \lstset{style=bash}
        \begin{lstlisting}
        \end{lstlisting}
            % Inline code rendering


    %% INCLUDING FIGURES %%

        % Basic Figure with size scaling
            \begin{figure}[h!]
               \centering
               \includegraphics[scale=1]{file.pdf}
            \end{figure}

        % Basic Figure with specific height
            \begin{figure}[h!]
               \centering
               \includegraphics[height=5in, width=5in]{file.pdf}
            \end{figure}

        % Figure with cropping, where the order for trimming is  L, B, R, T
            \begin{figure}
               \centering
               \includegraphics[trim={1cm, 1cm, 1cm, 1cm}, clip]{file.pdf}
            \end{figure}

        % Side by Side figures: Use the tabular environment


