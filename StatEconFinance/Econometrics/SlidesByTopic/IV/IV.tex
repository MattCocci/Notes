%%%%%%%%%%%%%%%%%%%%%%%%%%%%%%%%%%%%%%%%%%%%%%%%%%%%%%%%%%%%%%%%%%%%%
%% Metropolis %%%%%%%%%%%%%%%%%%%%%%%%%%%%%%%%%%%%%%%%%%%%%%%%%%%%%%%
%%%%%%%%%%%%%%%%%%%%%%%%%%%%%%%%%%%%%%%%%%%%%%%%%%%%%%%%%%%%%%%%%%%%%

%\documentclass{beamer}
%\usepackage{cmbright}
%\usetheme{metropolis}

%%%%%%%%%%%%%%%%%%%%%%%%%%%%%%%%%%%%%%%%%%%%%%%%%%%%%%%%%%%%%%%%%%%%%
%% Metropolis with Sidebar %%%%%%%%%%%%%%%%%%%%%%%%%%%%%%%%%%%%%%%%%%
%%%%%%%%%%%%%%%%%%%%%%%%%%%%%%%%%%%%%%%%%%%%%%%%%%%%%%%%%%%%%%%%%%%%%

\documentclass[aspectratio=169, handout]{beamer}
%\documentclass{beamer}
\usepackage{cmbright}
\usepackage{comment}

\useoutertheme[right, width=0.20\paperwidth]{sidebar}
\usecolortheme{metropolis}
\useinnertheme[sectionpage=none]{metropolis}
\usefonttheme{metropolis}

\makeatletter
\beamer@headheight=1.75\baselineskip     %controls the height of the headline, default is 2.5
\makeatother

\usepackage{etoolbox}
\patchcmd\insertverticalnavigation{\dohead}{\vskip-35pt\dohead}{}{}

%\setbeamertemplate{sidebar canvas right}{}
%\setbeamertemplate{sidebar right}{BOB}


%%%%%%%%%%%%%%%%%%%%%%%%%%%%%%%%%%%%%%%%%%%%%%%%%%%%%%%%%%%%%%%%%%%%%
%% Simple Template %%%%%%%%%%%%%%%%%%%%%%%%%%%%%%%%%%%%%%%%%%%%%%%%%%
%%%%%%%%%%%%%%%%%%%%%%%%%%%%%%%%%%%%%%%%%%%%%%%%%%%%%%%%%%%%%%%%%%%%%

%\documentclass{beamer}
%\usetheme{Boadilla}
%\usepackage{lmodern}

%%%%%%%%%%%%%%%%%%%%%%%%%%%%%%%%%%%%%%%%%%%%%%%%%%%%%%%%%%%%%%%%%%%%%
%% Sidebar Template %%%%%%%%%%%%%%%%%%%%%%%%%%%%%%%%%%%%%%%%%%%%%%%%%
%%%%%%%%%%%%%%%%%%%%%%%%%%%%%%%%%%%%%%%%%%%%%%%%%%%%%%%%%%%%%%%%%%%%%

%\documentclass[serif]{beamer}  % For serif latex font
%\usepackage{pxfonts}
%\usepackage{eulervm}
%\usepackage{mathpazo}

%\usetheme{Goettingen}
%\usecolortheme{lily}

%%%%%%%%%%%%%%%%%%%%%%%%%%%%%%%%%%%%%%%%%%%%%%%%%%%%%%%%%%%%%%%%%%%%%
%% Common Settings %%%%%%%%%%%%%%%%%%%%%%%%%%%%%%%%%%%%%%%%%%%%%%%%%%
%%%%%%%%%%%%%%%%%%%%%%%%%%%%%%%%%%%%%%%%%%%%%%%%%%%%%%%%%%%%%%%%%%%%%

\title[]{Instrumental Variables \\ Matt Cocci}
\author[]{}
\date{\today}

\setbeamertemplate{section in toc}[sections numbered]
\setbeamertemplate{subsection in toc}[subsections numbered]
%\setbeamertemplate{section in toc}[square unnumbered]
%\setbeamertemplate{subsection in toc}[square unnumbered]

%% Mathematics Related %%%%%%%%%%%%%%%%%%%%%%%%%%%%%%%%%%%

\usepackage{amsmath}
\usepackage{amssymb}
\usepackage{amsfonts}
\usepackage{mathrsfs}
\usepackage{mathtools}
\usepackage{amsthm} %allows for labeling of theorems
%\numberwithin{equation}{section} % Number equations by section
\usepackage{bbm} % For bold numbers

%% Figures %%%%%%%%%%%%%%%%%%%%%%%%%%%%%%%%%%%%%%%%%%%%%%

\usepackage{tikz}
\usetikzlibrary{decorations.pathreplacing}
\usetikzlibrary{arrows.meta}
\usepackage{pgfplots}
\usepgfplotslibrary{dateplot}
\usepackage{graphicx}
\usepackage{subfigure}
    %   For plotting multiple figures at once
%\graphicspath{ {Directory/} }
    %   Set a directory for where to look for figures

%% Misc %%%%%%%%%%%%%%%%%%%%%%%%%%%%%%%%%%%%%%%%%%%%%%


% For easier looping
\usepackage{pgffor}

\usepackage{pdfpages}

%% User Defined %%%%%%%%%%%%%%%%%%%%%%%%%%%%%%%%%%%%%%%%%%

%\newcommand{\nameofcmd}{Text to display}
\newcommand*{\Chi}{\mbox{\large$\chi$}} %big chi
    %   Bigger Chi

% In math mode, Use this instead of \munderbar, since that changes the
% font from math to regular
\makeatletter
\def\munderbar#1{\underline{\sbox\tw@{$#1$}\dp\tw@\z@\box\tw@}}
\makeatother

% Misc Math
\newcommand{\ra}{\rightarrow}
\newcommand{\diag}{\text{diag}}
\newcommand{\proj}{\operatorname{proj}}
\newcommand{\ch}{\text{ch}}
\newcommand{\dom}{\text{dom}}
\newcommand{\one}[1]{\mathbf{1}_{#1}}


% Command to generate new math commands:
% - Suppose you want to refer to \boldsymbol{x} as just \bsx, where 'x'
%   is any letter. This commands lets you generate \bsa, \bsb, etc.
%   without copy pasting \newcommand{\bsa}{\boldsymbol{a}} for each
%   letter individually. Instead, just include
%
%     \generate{bs}{\boldsymbol}{a,...,z}
%
% - Uses pgffor package to loop
% - Example with optional argument. Will generate \bshatx to represent
%   \boldsymbol{\hat{x}} for all letters x
%
%     \generate[\hat]{bshat}{\boldsymbol}{a,...,z}

\newcommand{\generate}[4][]{%
  % Takes 3 arguments (maybe four):
  % - 1   wrapcmd (optional, defaults to nothing)
  % - 2   newname
  % - 3   mathmacro
  % - 4   Names to loop over
  %
  % Will produce
  %
  %   \newcommand{\newnameX}{mathmacro{wrapcmd{X}}}
  %
  % for each X in argument 4

  \foreach \x in {#4}{%
    \expandafter\xdef\csname%
      #2\x%
    \endcsname%
    {\noexpand\ensuremath{\noexpand#3{\noexpand#1{\x}}}}
  }
}


% MATHSCR: Gen \sX to stand for \mathscr{X} for all upper case letters
\generate{s}{\mathscr}{A,...,Z}


% BOLDSYMBOL: Generate \bsX to stand for \boldsymbol{X}, all upper and
% lower case.
%
% Letters and greek letters
\generate{bs}{\boldsymbol}{a,...,z}
\generate{bs}{\boldsymbol}{A,...,Z}
\newcommand{\bstheta}{\boldsymbol{\theta}}
\newcommand{\bsmu}{\boldsymbol{\mu}}
\newcommand{\bsSigma}{\boldsymbol{\Sigma}}
\newcommand{\bsvarepsilon}{\boldsymbol{\varepsilon}}
\newcommand{\bsalpha}{\boldsymbol{\alpha}}
\newcommand{\bsbeta}{\boldsymbol{\beta}}
\newcommand{\bsOmega}{\boldsymbol{\Omega}}
\newcommand{\bshatOmega}{\boldsymbol{\hat{\Omega}}}
\newcommand{\bshatG}{\boldsymbol{\hat{G}}}
\newcommand{\bsgamma}{\boldsymbol{\gamma}}
\newcommand{\bslambda}{\boldsymbol{\lambda}}

% Special cases like \bshatb for \boldsymbol{\hat{b}}
\generate[\hat]{bshat}{\boldsymbol}{b,y,x,X,V,S,W}
\newcommand{\bshatbeta}{\boldsymbol{\hat{\beta}}}
\newcommand{\bshatmu}{\boldsymbol{\hat{\mu}}}
\newcommand{\bshattheta}{\boldsymbol{\hat{\theta}}}
\newcommand{\bshatSigma}{\boldsymbol{\hat{\Sigma}}}
\newcommand{\bstildebeta}{\boldsymbol{\tilde{\beta}}}
\newcommand{\bstildetheta}{\boldsymbol{\tilde{\theta}}}
\newcommand{\bsbarbeta}{\boldsymbol{\overline{\beta}}}
\newcommand{\bsbarg}{\boldsymbol{\overline{g}}}

% Redefine \bso to be the zero vector
\renewcommand{\bso}{\boldsymbol{0}}

% Transposes of all the boldsymbol shit
\newcommand{\bsbp}{\boldsymbol{b'}}
\newcommand{\bshatbp}{\boldsymbol{\hat{b'}}}
\newcommand{\bsdp}{\boldsymbol{d'}}
\newcommand{\bsgp}{\boldsymbol{g'}}
\newcommand{\bsGp}{\boldsymbol{G'}}
\newcommand{\bshp}{\boldsymbol{h'}}
\newcommand{\bsSp}{\boldsymbol{S'}}
\newcommand{\bsup}{\boldsymbol{u'}}
\newcommand{\bsxp}{\boldsymbol{x'}}
\newcommand{\bsyp}{\boldsymbol{y'}}
\newcommand{\bsthetap}{\boldsymbol{\theta'}}
\newcommand{\bsmup}{\boldsymbol{\mu'}}
\newcommand{\bsSigmap}{\boldsymbol{\Sigma'}}
\newcommand{\bshatmup}{\boldsymbol{\hat{\mu'}}}
\newcommand{\bshatSigmap}{\boldsymbol{\hat{\Sigma'}}}

% MATHCAL: Gen \calX to stand for \mathcal{X}, all upper case
\generate{cal}{\mathcal}{A,...,Z}

% MATHBB: Gen \X to stand for \mathbb{X} for some upper case
\generate{}{\mathbb}{R,Q,C,Z,N,Z,E}
\newcommand{\Rn}{\mathbb{R}^n}
\newcommand{\RN}{\mathbb{R}^N}
\newcommand{\Rk}{\mathbb{R}^k}
\newcommand{\RK}{\mathbb{R}^K}
\newcommand{\RL}{\mathbb{R}^L}
\newcommand{\Rl}{\mathbb{R}^\ell}
\newcommand{\Rm}{\mathbb{R}^m}
\newcommand{\Rnn}{\mathbb{R}^{n\times n}}
\newcommand{\Rmn}{\mathbb{R}^{m\times n}}
\newcommand{\Rnm}{\mathbb{R}^{n\times m}}
\newcommand{\Rkn}{\mathbb{R}^{k\times n}}
\newcommand{\Rnk}{\mathbb{R}^{n\times k}}
\newcommand{\Rkk}{\mathbb{R}^{k\times k}}
\newcommand{\Cn}{\mathbb{C}^n}
\newcommand{\Cnn}{\mathbb{C}^{n\times n}}

% Dot over
\newcommand{\dx}{\dot{x}}
\newcommand{\ddx}{\ddot{x}}
\newcommand{\dy}{\dot{y}}
\newcommand{\ddy}{\ddot{y}}

% First derivatives
\newcommand{\dydx}{\frac{dy}{dx}}
\newcommand{\dfdx}{\frac{df}{dx}}
\newcommand{\dfdy}{\frac{df}{dy}}
\newcommand{\dfdz}{\frac{df}{dz}}

% Second derivatives
\newcommand{\ddyddx}{\frac{d^2y}{dx^2}}
\newcommand{\ddydxdy}{\frac{d^2y}{dx dy}}
\newcommand{\ddydydx}{\frac{d^2y}{dy dx}}
\newcommand{\ddfddx}{\frac{d^2f}{dx^2}}
\newcommand{\ddfddy}{\frac{d^2f}{dy^2}}
\newcommand{\ddfddz}{\frac{d^2f}{dz^2}}
\newcommand{\ddfdxdy}{\frac{d^2f}{dx dy}}
\newcommand{\ddfdydx}{\frac{d^2f}{dy dx}}


% First Partial Derivatives
\newcommand{\pypx}{\frac{\partial y}{\partial x}}
\newcommand{\pfpx}{\frac{\partial f}{\partial x}}
\newcommand{\pfpy}{\frac{\partial f}{\partial y}}
\newcommand{\pfpz}{\frac{\partial f}{\partial z}}


% argmin and argmax
\DeclareMathOperator*{\argmin}{arg\;min}
\DeclareMathOperator*{\argmax}{arg\;max}


% Various probability and statistics commands
\newcommand{\iid}{\overset{iid}{\sim}}
\newcommand{\vc}{\operatorname{vec}}
\newcommand{\Cov}{\operatorname{Cov}}
\newcommand{\rank}{\operatorname{rank}}
\newcommand{\trace}{\operatorname{trace}}
\newcommand{\Corr}{\operatorname{Corr}}
\newcommand{\Var}{\operatorname{Var}}
\newcommand{\asto}{\xrightarrow{a.s.}}
\newcommand{\pto}{\xrightarrow{p}}
\newcommand{\msto}{\xrightarrow{m.s.}}
\newcommand{\dto}{\xrightarrow{d}}
\newcommand{\Lpto}{\xrightarrow{L_p}}
\newcommand{\Lqto}[1]{\xrightarrow{L_{#1}}}
\newcommand{\plim}{\text{plim}_{n\rightarrow\infty}}


% Redefine real and imaginary from fraktur to plain text
\renewcommand{\Re}{\operatorname{Re}}
\renewcommand{\Im}{\operatorname{Im}}

% Shorter sums: ``Sum from X to Y''
% - sumXY  is equivalent to \sum^Y_{X=1}
% - sumXYz is equivalent to \sum^Y_{X=0}
\newcommand{\sumnN}{\sum^N_{n=1}}
\newcommand{\sumin}{\sum^n_{i=1}}
\newcommand{\sumjn}{\sum^n_{j=1}}
\newcommand{\sumim}{\sum^m_{i=1}}
\newcommand{\sumik}{\sum^k_{i=1}}
\newcommand{\sumiN}{\sum^N_{i=1}}
\newcommand{\sumkn}{\sum^n_{k=1}}
\newcommand{\sumtT}{\sum^T_{t=1}}
\newcommand{\sumninf}{\sum^\infty_{n=1}}
\newcommand{\sumtinf}{\sum^\infty_{t=1}}
\newcommand{\sumnNz}{\sum^N_{n=0}}
\newcommand{\suminz}{\sum^n_{i=0}}
\newcommand{\sumknz}{\sum^n_{k=0}}
\newcommand{\sumtTz}{\sum^T_{t=0}}
\newcommand{\sumninfz}{\sum^\infty_{n=0}}
\newcommand{\sumtinfz}{\sum^\infty_{t=0}}

\newcommand{\prodnN}{\prod^N_{n=1}}
\newcommand{\prodin}{\prod^n_{i=1}}
\newcommand{\prodiN}{\prod^N_{i=1}}
\newcommand{\prodkn}{\prod^n_{k=1}}
\newcommand{\prodtT}{\prod^T_{t=1}}
\newcommand{\prodnNz}{\prod^N_{n=0}}
\newcommand{\prodinz}{\prod^n_{i=0}}
\newcommand{\prodknz}{\prod^n_{k=0}}
\newcommand{\prodtTz}{\prod^T_{t=0}}

% Bounds
\newcommand{\atob}{_a^b}
\newcommand{\ztoinf}{_0^\infty}
\newcommand{\kinf}{_{k=1}^\infty}
\newcommand{\ninf}{_{n=1}^\infty}
\newcommand{\minf}{_{m=1}^\infty}
\newcommand{\tinf}{_{t=1}^\infty}
\newcommand{\nN}{_{n=1}^N}
\newcommand{\tT}{_{t=1}^T}
\newcommand{\kinfz}{_{k=0}^\infty}
\newcommand{\ninfz}{_{n=0}^\infty}
\newcommand{\minfz}{_{m=0}^\infty}
\newcommand{\tinfz}{_{t=0}^\infty}
\newcommand{\nNz}{_{n=0}^N}

% Limits
\newcommand{\limN}{\lim_{N\rightarrow\infty}}
\newcommand{\limn}{\lim_{n\rightarrow\infty}}
\newcommand{\limk}{\lim_{k\rightarrow\infty}}
\newcommand{\limt}{\lim_{t\rightarrow\infty}}
\newcommand{\limT}{\lim_{T\rightarrow\infty}}
\newcommand{\limhz}{\lim_{h\rightarrow 0}}

% Shorter integrals: ``Integral from X to Y''
% - intXY is equivalent to \int^Y_X
\newcommand{\intab}{\int_a^b}
\newcommand{\intzN}{\int_0^N}


%%%%%%%%%%%%%%%%%%%%%%%%%%%%%%%%%%%%%%%%%%%%%%%%%%%%%%%%%%%%%%%%%%%%%
%% Presentation %%%%%%%%%%%%%%%%%%%%%%%%%%%%%%%%%%%%%%%%%%%%%%%%%%%%%
%%%%%%%%%%%%%%%%%%%%%%%%%%%%%%%%%%%%%%%%%%%%%%%%%%%%%%%%%%%%%%%%%%%%%

\begin{document}

\begin{frame}[plain]
\titlepage
\end{frame}


\begin{frame}{Outline}
\tableofcontents[hideallsubsections]
%\tableofcontents
\end{frame}


\section{IV Model}

\begin{frame}{General IV Model}
Model
\begin{align*}
  Y_i &= \beta_0 + X_i\beta + W_i \gamma + u_i
\end{align*}
with first stage
\begin{align*}
  X_{ji}
  &=
  \pi_0
  +
  Z_i\pi_z
  +
  W_i\pi_w
  +
  v_i
  \qquad
  j=1,\ldots,k
\end{align*}
Assumptions
\begin{itemize}
  \item Exogeneity
    \begin{align*}
      \E[u_i|W_i,Z_i] = 0
    \end{align*}
    or, weaker, control variables assumption
    \begin{align*}
      \E[u_i|W_i,Z_i]
      =
      \E[u_i|W_i]
    \end{align*}
  \item
    Relevance:
    $(1,W_i,\tilde{X}_i)$ not perfectly multicolinear,
    where $\tilde{X}_i$ are fitted values from first stage.
\end{itemize}
\end{frame}


\section{Weak IV}

{\footnotesize
\begin{frame}{Weak IV: Why the F-Stat?}
Recall the weak IV \alert{rule of thumb}: $F$-stat bigger than 10 in
first-stage regression, not as worried about weak instrument (assuming
homoskedasticity).

\alert{Problem}:
But why look at $F$-stat and not other things?
Say $R^2$, $t$-stat, correlation between $X$ and $Z$?

\alert{Model}:
\begin{align*}
  Y_i &= \gamma_0 + \gamma_1 X_i + u_i
  \\
  X_i &= \pi_0 + \pi_1 Z_i + v_i
\end{align*}
Also assume homoskedasticity.

\alert{IV estimator}
\begin{align*}
  \hat{\gamma}_1
  &=
  \frac{\widehat{\Cov(Y_i,Z_i)}}{\widehat{\Cov(X_i,Z_i)}}
  %=
  %\frac{\frac{1}{n}\sumin \tilde{Y}_i\tilde{Z}_i}{{\frac{1}{n}\sumin \tilde{X}_i\tilde{Z}_i}}
\end{align*}
where $\tilde{X}_i=X_i-\bar{X}$.
\end{frame}
}


{\scriptsize
\begin{frame}{Weak IV and its Two Components}
Given estimate
\begin{align*}
  \hat{\gamma}_1
  &=
  \frac{\widehat{\Cov(Y_i,Z_i)}}{\widehat{\Cov(X_i,Z_i)}}
  %=
  %\frac{\frac{1}{n}\sumin \tilde{Y}_i\tilde{Z}_i}{{\frac{1}{n}\sumin \tilde{X}_i\tilde{Z}_i}}
\end{align*}
\alert{Weak IV}:
Normal approx for sampling dist. of $\hat{\gamma}$ is bad.
\begin{itemize}
  \item \alert{Sampling Dist.}:
    Draw sample $\{Y_i,X_i,Z_i\}_{i=1}^n$, compute
    $\widehat{\Cov(Y_i,Z_i)}$,
    $\widehat{\Cov(X_i,Z_i)}$,
    $\hat{\gamma}_1$.
    Repeat.
  \item If $\Cov(X_i,Z_i)\approx 0$, then $\widehat{\Cov(X_i,Z_i)}$
    near zero---a sensitive area where different sample draws
    and $\widehat{\Cov(X_i,Z_i)}$ estimates lead to large differences in
    $\hat{\gamma}_1$
  \item \alert{But} if $n$ \alert{very large} and
    $\Cov(X_i,Z_i)\approx 0$ but $\neq 0$, estimates of
    $\widehat{\Cov(X_i,Z_i)}$ might not vary much from sample draw to
    sample draw, and so that non-normality-inducing sensitivity is never
    exploited
\end{itemize}
So we see two parts for weak IV to bite:
\begin{itemize}
  \item \alert{Magnitude} of $\Cov(X_i,Z_i)$: How far away from zero?
  \item \alert{Precision} of estimate $\widehat{\Cov(X_i,Z_i)}$:
    Is $n$ so large that it doesn't vary much sample draw to sample
    draw?
\end{itemize}
\end{frame}
}


{\footnotesize
\begin{frame}{Weak IV: Why the F-Stat?}
%\alert{Problem}:
%But why look at $F$-stat and not other things?
%Say $R^2$, $t$-stat, correlation between $X$ and $Z$?
\alert{Model}:
\begin{align*}
  Y_i &= \gamma_0 + \gamma_1 X_i + u_i
  \\
  X_i &= \pi_0 + \pi_1 Z_i + v_i
\end{align*}
Also assume homoskedasticity.
\begin{itemize}
  \item In this setting, reported $F$ stat tests $\alert{\pi_1=0}$.
  \item Collection of definitions/results we will use
    \begin{itemize}
      \item Test of single hypothesis $\pi_1=0$:
        \begin{align*}
          F
          =
          t^2
          :=
          \left(
          \frac{\hat{\pi}_1}{\sigma^2_{1}/\sqrt{n}}
          \right)^2
        \end{align*}
      \item Under homoskedasticity,
        \begin{align*}
          \tilde{F}
          &=
          \frac{n-k}{k-1}
          \frac{R^2}{(1-R^2)}
        \end{align*}
      \item Single regressor:
        First stage $R^2=\Corr(X_i,Z_i)^2$
      %\item IV estimator
        %\begin{align*}
          %\hat{\gamma}_1
          %&=
          %\frac{\widehat{\Cov(Y_i,Z_i)}}{\widehat{\Cov(X_i,Z_i)}}
        %\end{align*}
    \end{itemize}
\end{itemize}
\end{frame}
}


{\footnotesize
\begin{frame}{Weak IV: Why the F-Stat?}
Notice that the $F$ stat checks the two important considerations of
weak IV:
\begin{align*}
  \tilde{F}
  &=
  \underbrace{\frac{n-k}{k-1}}_{%
    \substack{%
      \text{Precision}
    }
  }
  \underbrace{%
    \frac{R^2}{1-R^2}
  }_{%
    \text{Magnitude}
  }
\end{align*}
Other possible measures
\begin{itemize}
  \item \alert{$t$-stat}: Equivalent to $F$-stat.
    Can also see it takes into account magnitude of effect $\hat{\pi}_1$
    and precision $n$
  \item \alert{$R^2$}: Measures of the relationship between $X_i$ and
    $Z_i$ in the sample.
    Doesn't account for precision.
  \item \alert{$\Corr(X_i,Z_i)$}: Same issue as $R^2$
\end{itemize}
\alert{Punchline}:
$F$-stat checks both things we want to check
\begin{itemize}
  \item \alert{Magnitude}: Is the denominator in the estimator close to
    zero?
  \item \alert{Precision}: Is the denominator sufficiently precisely
    estimated that it doesn't induce non-normality.
\end{itemize}
Problem 1(e):
$R^2=0.0048$, and yet $F=1238.17$ because $n=254,654$.
\end{frame}
}




\end{document}






% Sample code
\pause to add breaks
\textcolor{red}{text to show}.
\vspace{20pt}

\usebeamercolor{dull text}
\alert<+>{Words that you want to alert}

