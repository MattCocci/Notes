\documentclass[a4paper,12pt]{scrartcl}

\author{Matthew Cocci}
\title{Econometrics}
\date{\today}
\usepackage{enumitem} %Has to do with enumeration	
\usepackage{amsfonts}
\usepackage{amsmath}
\usepackage{amsthm} %allows for labeling of theorems
\usepackage[T1]{fontenc}
\usepackage[utf8]{inputenc}
\usepackage{blindtext}
\usepackage{graphicx}
\usepackage[hidelinks]{hyperref} % For internal/external linking. 
				 % [hidelinks] removes boxes
% \usepackage{url} % allows for url display, non-clickable
%\numberwithin{equation}{section} 
   % This labels the equations in relation to the sections 
      % rather than other equations
%\numberwithin{equation}{subsection} %This labels relative to subsections
\newtheorem{thm}{Theorem}[section]
\newtheorem{lem}[thm]{Lemma}
\newtheorem{prop}[thm]{Proposition}
\newtheorem{cor}[thm]{Corollary}
\setkomafont{disposition}{\normalfont\bfseries}
\usepackage{appendix}
\usepackage{subfigure} % For plotting multiple figures at once
\usepackage{verbatim} % for including verbatim code from a file
\usepackage{natbib} % for bibliographies

\begin{document}
\maketitle

%\tableofcontents %adds it here

\section{Notation}

The most general type of data we'll consider is  panel data, where the outcome of interest is denoted $Y_{it}$, the outcome for individual $i$ at time $t$.

\emph{Cross-sectional data} is just a special case of panel data where there is only one time period.  Similarly, univariate time series have no ``individual'' or $i$ dimension; we just consider variation over time.

Therefore, in using this notation, we can be explicit about the way we introduce independent predictors, also known as covariates or independent variables. Righthand side variables with only an $i$ subscript only will be assumed constant over time, as in the case of individual fixed effects.  Similarly, anything with a $t$ subscript only will be assumed fixed across individuals, as in time fixed effects.

This particularly expressive notation will be very helpful as we consider different classes of models.




%%%% APPPENDIX %%%%%%%%%%%

\appendix


\section{Adjusted R-Squared}

Adjusted R-squared, $R^2_A$, is computed 
\[ R^2_A = 1 - \frac{e'e }{(n-k) s_y^2} \]
where $e$ is the vector of error/residual terms resulting from a 
model estimate,
$n$ is the number of observations, $k$ equals the number of 
independent variables on the right hand side of the equation, and
$s_y^2$ is the unbiased variance estimate of the dependent variable.



\end{document}



%%%% INCLUDING FIGURES %%%%%%%%%%%%%%%%%%%%%%%%%%%%

   % H indicates here 
   %\begin{figure}[h!]
   %   \centering
   %   \includegraphics[scale=1]{file.pdf}
   %\end{figure}

%   \begin{figure}[h!]
%      \centering
%      \mbox{
%	 \subfigure{
%	    \includegraphics[scale=1]{file1.pdf}
%	 }\quad
%	 \subfigure{
%	    \includegraphics[scale=1]{file2.pdf} 
%	 }
%      }
%   \end{figure}
 

%%%%% Including Code %%%%%%%%%%%%%%%%%%%%%5
% \verbatiminput{file.ext}    % Includes verbatim text from the file
% \texttt{text}	  % includes text in courier, or code-like, font
