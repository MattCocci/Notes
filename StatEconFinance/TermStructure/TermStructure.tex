\documentclass[12pt]{article}

\author{Matthew D. Cocci}
\title{Title}
\date{\today}

%% Formatting & Spacing %%%%%%%%%%%%%%%%%%%%%%%%%%%%%%%%%%%%

\usepackage[top=0.60in, bottom=0.60in, left=0.60in, right=0.60in]{geometry} % most detailed page formatting control
%\usepackage{fullpage} % Simpler than using the geometry package; std effect
\usepackage{setspace}
%\onehalfspacing
\usepackage{microtype}

%% Formatting %%%%%%%%%%%%%%%%%%%%%%%%%%%%%%%%%%%%%%%%%%%%%

%\usepackage[margin=1in]{geometry}
    %   Adjust the margins with geometry package
%\usepackage{pdflscape}
    %   Allows landscape pages
%\usepackage{layout}
    %   Allows plotting of picture of formatting
\usepackage{multicol}
\setlength{\columnsep}{1cm}



%% Header %%%%%%%%%%%%%%%%%%%%%%%%%%%%%%%%%%%%%%%%%%%%%%%%%

%\usepackage{fancyhdr}
%\pagestyle{fancy}
%\lhead{}
%\rhead{}
%\chead{}
%\setlength{\headheight}{15.2pt}
    %   Make the header bigger to avoid overlap

%\fancyhf{}
    %   Erase header settings

%\renewcommand{\headrulewidth}{0.3pt}
    %   Width of the line

%\setlength{\headsep}{0.2in}
    %   Distance from line to text


%% Mathematics Related %%%%%%%%%%%%%%%%%%%%%%%%%%%%%%%%%%%

\usepackage{amsmath}
\usepackage{amssymb}
\usepackage{amsfonts}
\usepackage{mathrsfs}
\usepackage{amsthm} %allows for labeling of theorems
%\numberwithin{equation}{section} % Number equations by section
\usepackage{bbm} % For bold numbers

\theoremstyle{plain}
\newtheorem{thm}{Theorem}[section]
\newtheorem{lem}[thm]{Lemma}
\newtheorem{prop}[thm]{Proposition}
\newtheorem{cor}[thm]{Corollary}

\theoremstyle{definition}
\newtheorem{defn}[thm]{Definition}
\newtheorem{ex}[thm]{Example}

\theoremstyle{remark}
\newtheorem*{rmk}{Remark}
\newtheorem*{note}{Note}

% Below supports left-right alignment in matrices so the negative
% signs don't look bad
\makeatletter
\renewcommand*\env@matrix[1][c]{\hskip -\arraycolsep
  \let\@ifnextchar\new@ifnextchar
  \array{*\c@MaxMatrixCols #1}}
\makeatother


%% Font Choices %%%%%%%%%%%%%%%%%%%%%%%%%%%%%%%%%%%%%%%%%

\usepackage[T1]{fontenc}
\usepackage{lmodern}
\usepackage[utf8]{inputenc}
%\usepackage{blindtext}
\usepackage{courier}


%% Figures %%%%%%%%%%%%%%%%%%%%%%%%%%%%%%%%%%%%%%%%%%%%%%

\usepackage{tikz}
\usetikzlibrary{decorations.pathreplacing}
\usetikzlibrary{arrows.meta}
\usepackage{graphicx}
\usepackage{subfigure}
    %   For plotting multiple figures at once
%\graphicspath{ {Directory/} }
    %   Set a directory for where to look for figures


%% Hyperlinks %%%%%%%%%%%%%%%%%%%%%%%%%%%%%%%%%%%%%%%%%%%%
\usepackage{hyperref}
\hypersetup{%
    colorlinks,
        %   This colors the links themselves, not boxes
    citecolor=black,
        %   Everything here and below changes link colors
    filecolor=black,
    linkcolor=black,
    urlcolor=black
}

%% Colors %%%%%%%%%%%%%%%%%%%%%%%%%%%%%%%%%%%%%%%%%%%%%%%

\usepackage{color}
\definecolor{codegreen}{RGB}{28,172,0}
\definecolor{codelilas}{RGB}{170,55,241}

% David4 color scheme
\definecolor{d4blue}{RGB}{100,191,255}
\definecolor{d4gray}{RGB}{175,175,175}
\definecolor{d4black}{RGB}{85,85,85}
\definecolor{d4orange}{RGB}{255,150,100}

%% Including Code %%%%%%%%%%%%%%%%%%%%%%%%%%%%%%%%%%%%%%%

\usepackage{verbatim}
    %   For including verbatim code from files, no colors
\usepackage{listings}
    %   For including code snippets written directly in this doc

\lstdefinestyle{bash}{%
  language=bash,%
  basicstyle=\footnotesize\ttfamily,%
  showstringspaces=false,%
  commentstyle=\color{gray},%
  keywordstyle=\color{blue},%
  xleftmargin=0.25in,%
  xrightmargin=0.25in
}
\lstdefinestyle{log}{%
  basicstyle=\scriptsize\ttfamily,%
  showstringspaces=false,%
  xleftmargin=0.25in,%
  xrightmargin=0.25in
}


\lstdefinestyle{matlab}{%
  language=Matlab,%
  basicstyle=\footnotesize\ttfamily,%
  breaklines=true,%
  morekeywords={matlab2tikz},%
  keywordstyle=\color{blue},%
  morekeywords=[2]{1}, keywordstyle=[2]{\color{black}},%
  identifierstyle=\color{black},%
  stringstyle=\color{codelilas},%
  commentstyle=\color{codegreen},%
  showstringspaces=false,%
    %   Without this there will be a symbol in
    %   the places where there is a space
  %numbers=left,%
  %numberstyle={\tiny \color{black}},%
    %   Size of the numbers
  numbersep=9pt,%
    %   Defines how far the numbers are from the text
  emph=[1]{for,end,break,switch,case},emphstyle=[1]\color{blue},%
    %   Some words to emphasise
}

\newcommand{\matlabcode}[1]{%
    \lstset{style=matlab}%
    \lstinputlisting{#1}
}
    %   For including Matlab code from .m file with colors,
    %   line numbering, etc.

\lstdefinelanguage{Julia}%
  {morekeywords={abstract,break,case,catch,const,continue,do,else,elseif,%
      end,export,false,for,function,immutable,import,importall,if,in,%
      macro,module,otherwise,quote,return,switch,true,try,type,typealias,%
      using,while},%
   sensitive=true,%
   %alsoother={$},%
   morecomment=[l]\#,%
   morecomment=[n]{\#=}{=\#},%
   morestring=[s]{"}{"},%
   morestring=[m]{'}{'},%
}[keywords,comments,strings]

\lstdefinestyle{julia}{%
    language         = Julia,
    basicstyle       = \scriptsize\ttfamily,
    keywordstyle     = \bfseries\color{blue},
    stringstyle      = \color{codegreen},
    commentstyle     = \color{codegreen},
    showstringspaces = false,
    literate         = %
      {ρ}{{$\rho$}}1
      {ℓ}{{$\ell$}}1
      {∑}{{$\Sigma$}}1
      {Σ}{{$\Sigma$}}1
      {√}{{$\sqrt{}$}}1
      {θ}{{$\theta$}}1
      {ω}{{$\omega$}}1
      {ɛ}{{$\varepsilon$}}1
      {φ}{{$\varphi$}}1
      {σ²}{{$\sigma^2$}}1
      {Φ}{{$\Phi$}}1
      {ϕ}{{$\phi$}}1
      {Dₑ}{{$D_e$}}1
      {Σ}{{$\Sigma$}}1
      {γ}{{$\gamma$}}1
      {δ}{{$\delta$}}1
      {τ}{{$\tau$}}1
      {μ}{{$\mu$}}1
      {β}{{$\beta$}}1
      {Λ}{{$\Lambda$}}1
      {λ}{{$\lambda$}}1
      {r̃}{{$\tilde{\text{r}}$}}1
      {α}{{$\alpha$}}1
      {σ}{{$\sigma$}}1
      {π}{{$\pi$}}1
      {∈}{{$\in$}}1
      {∞}{{$\infty$}}1
}


%% Bibliographies %%%%%%%%%%%%%%%%%%%%%%%%%%%%%%%%%%%%

%\usepackage{natbib}
    %---For bibliographies
%\setlength{\bibsep}{3pt} % Set how far apart bibentries are

%% Misc %%%%%%%%%%%%%%%%%%%%%%%%%%%%%%%%%%%%%%%%%%%%%%

\usepackage{enumitem}
    %   Has to do with enumeration
\usepackage{appendix}
%\usepackage{natbib}
    %   For bibliographies
\usepackage{pdfpages}
    %   For including whole pdf pages as a page in doc
\usepackage{pgffor}
    %   For easier looping


%% User Defined %%%%%%%%%%%%%%%%%%%%%%%%%%%%%%%%%%%%%%%%%%

%\newcommand{\nameofcmd}{Text to display}
\newcommand*{\Chi}{\mbox{\large$\chi$}} %big chi
    %   Bigger Chi

% In math mode, Use this instead of \munderbar, since that changes the
% font from math to regular
\makeatletter
\def\munderbar#1{\underline{\sbox\tw@{$#1$}\dp\tw@\z@\box\tw@}}
\makeatother

% Misc Math
\newcommand{\ra}{\rightarrow}
\newcommand{\diag}{\text{diag}}
\newcommand{\ch}{\text{ch}}
\newcommand{\dom}{\text{dom}}
\newcommand{\one}[1]{\mathbf{1}_{#1}}


% Command to generate new math commands:
% - Suppose you want to refer to \boldsymbol{x} as just \bsx, where 'x'
%   is any letter. This commands lets you generate \bsa, \bsb, etc.
%   without copy pasting \newcommand{\bsa}{\boldsymbol{a}} for each
%   letter individually. Instead, just include
%
%     \generate{bs}{\boldsymbol}{a,...,z}
%
% - Uses pgffor package to loop
% - Example with optional argument. Will generate \bshatx to represent
%   \boldsymbol{\hat{x}} for all letters x
%
%     \generate[\hat]{bshat}{\boldsymbol}{a,...,z}

\newcommand{\generate}[4][]{%
  % Takes 3 arguments (maybe four):
  % - 1   wrapcmd (optional, defaults to nothing)
  % - 2   newname
  % - 3   mathmacro
  % - 4   Names to loop over
  %
  % Will produce
  %
  %   \newcommand{\newnameX}{mathmacro{wrapcmd{X}}}
  %
  % for each X in argument 4

  \foreach \x in {#4}{%
    \expandafter\xdef\csname%
      #2\x%
    \endcsname%
    {\noexpand\ensuremath{\noexpand#3{\noexpand#1{\x}}}}
  }
}


% MATHSCR: Gen \sX to stand for \mathscr{X} for all upper case letters
\generate{s}{\mathscr}{A,...,Z}


% BOLDSYMBOL: Generate \bsX to stand for \boldsymbol{X}, all upper and
% lower case.
%
% Letters and greek letters
\generate{bs}{\boldsymbol}{a,...,z}
\generate{bs}{\boldsymbol}{A,...,Z}
\newcommand{\bstheta}{\boldsymbol{\theta}}
\newcommand{\bsmu}{\boldsymbol{\mu}}
\newcommand{\bsSigma}{\boldsymbol{\Sigma}}
\newcommand{\bsvarepsilon}{\boldsymbol{\varepsilon}}
\newcommand{\bsalpha}{\boldsymbol{\alpha}}
\newcommand{\bsbeta}{\boldsymbol{\beta}}
\newcommand{\bsgamma}{\boldsymbol{\gamma}}
\newcommand{\bslambda}{\boldsymbol{\lambda}}

% Special cases like \bshatb for \boldsymbol{\hat{b}}
\generate[\hat]{bshat}{\boldsymbol}{b,y,x,X,V,S,W}
\newcommand{\bshatbeta}{\boldsymbol{\hat{\beta}}}
\newcommand{\bshatmu}{\boldsymbol{\hat{\mu}}}
\newcommand{\bshattheta}{\boldsymbol{\hat{\theta}}}
\newcommand{\bshatSigma}{\boldsymbol{\hat{\Sigma}}}
\newcommand{\bstildebeta}{\boldsymbol{\tilde{\beta}}}
\newcommand{\bsbarbeta}{\boldsymbol{\overline{\beta}}}

% Redefine \bso to be the zero vector
\renewcommand{\bso}{\boldsymbol{0}}

% Transposes of all the boldsymbol shit
\newcommand{\bsbp}{\boldsymbol{b'}}
\newcommand{\bshatbp}{\boldsymbol{\hat{b'}}}
\newcommand{\bsdp}{\boldsymbol{d'}}
\newcommand{\bsgp}{\boldsymbol{g'}}
\newcommand{\bsGp}{\boldsymbol{G'}}
\newcommand{\bshp}{\boldsymbol{h'}}
\newcommand{\bsSp}{\boldsymbol{S'}}
\newcommand{\bsup}{\boldsymbol{u'}}
\newcommand{\bsxp}{\boldsymbol{x'}}
\newcommand{\bsyp}{\boldsymbol{y'}}
\newcommand{\bsthetap}{\boldsymbol{\theta'}}
\newcommand{\bsmup}{\boldsymbol{\mu'}}
\newcommand{\bsSigmap}{\boldsymbol{\Sigma'}}
\newcommand{\bshatmup}{\boldsymbol{\hat{\mu'}}}
\newcommand{\bshatSigmap}{\boldsymbol{\hat{\Sigma'}}}

% MATHCAL: Gen \calX to stand for \mathcal{X}, all upper case
\generate{cal}{\mathcal}{A,...,Z}

% MATHBB: Gen \X to stand for \mathbb{X} for some upper case
\generate{}{\mathbb}{R,Q,C,Z,N,Z,E}
\newcommand{\Rn}{\mathbb{R}^n}
\newcommand{\RN}{\mathbb{R}^N}
\newcommand{\Rk}{\mathbb{R}^k}
\newcommand{\RK}{\mathbb{R}^K}
\newcommand{\RL}{\mathbb{R}^L}
\newcommand{\Rl}{\mathbb{R}^\ell}
\newcommand{\Rm}{\mathbb{R}^m}
\newcommand{\Rnn}{\mathbb{R}^{n\times n}}
\newcommand{\Rmn}{\mathbb{R}^{m\times n}}
\newcommand{\Rnm}{\mathbb{R}^{n\times m}}
\newcommand{\Rkn}{\mathbb{R}^{k\times n}}
\newcommand{\Cn}{\mathbb{C}^n}
\newcommand{\Cnn}{\mathbb{C}^{n\times n}}

% Dot over
\newcommand{\dx}{\dot{x}}
\newcommand{\ddx}{\ddot{x}}
\newcommand{\dy}{\dot{y}}
\newcommand{\ddy}{\ddot{y}}

% First derivatives
\newcommand{\dydx}{\frac{dy}{dx}}
\newcommand{\dfdx}{\frac{df}{dx}}
\newcommand{\dfdy}{\frac{df}{dy}}
\newcommand{\dfdz}{\frac{df}{dz}}

% Second derivatives
\newcommand{\ddyddx}{\frac{d^2y}{dx^2}}
\newcommand{\ddydxdy}{\frac{d^2y}{dx dy}}
\newcommand{\ddydydx}{\frac{d^2y}{dy dx}}
\newcommand{\ddfddx}{\frac{d^2f}{dx^2}}
\newcommand{\ddfddy}{\frac{d^2f}{dy^2}}
\newcommand{\ddfddz}{\frac{d^2f}{dz^2}}
\newcommand{\ddfdxdy}{\frac{d^2f}{dx dy}}
\newcommand{\ddfdydx}{\frac{d^2f}{dy dx}}


% First Partial Derivatives
\newcommand{\pypx}{\frac{\partial y}{\partial x}}
\newcommand{\pfpx}{\frac{\partial f}{\partial x}}
\newcommand{\pfpy}{\frac{\partial f}{\partial y}}
\newcommand{\pfpz}{\frac{\partial f}{\partial z}}


% argmin and argmax
\DeclareMathOperator*{\argmin}{arg\;min}
\DeclareMathOperator*{\argmax}{arg\;max}
\newenvironment{rcases}
  {\left.\begin{aligned}}
  {\end{aligned}\right\rbrace}


% Various probability and statistics commands
\newcommand{\iid}{\overset{iid}{\sim}}
\newcommand{\vc}{\operatorname{vec}}
\newcommand{\Cov}{\operatorname{Cov}}
\newcommand{\rank}{\operatorname{rank}}
\newcommand{\trace}{\operatorname{trace}}
\newcommand{\Corr}{\operatorname{Corr}}
\newcommand{\Var}{\operatorname{Var}}
\newcommand{\asto}{\xrightarrow{a.s.}}
\newcommand{\pto}{\xrightarrow{p}}
\newcommand{\msto}{\xrightarrow{m.s.}}
\newcommand{\dto}{\xrightarrow{d}}
\newcommand{\Lpto}{\xrightarrow{L_p}}
\newcommand{\Lqto}[1]{\xrightarrow{L_{#1}}}
\newcommand{\plim}{\text{plim}_{n\rightarrow\infty}}


% Redefine real and imaginary from fraktur to plain text
\renewcommand{\Re}{\operatorname{Re}}
\renewcommand{\Im}{\operatorname{Im}}

% Shorter sums: ``Sum from X to Y''
% - sumXY  is equivalent to \sum^Y_{X=1}
% - sumXYz is equivalent to \sum^Y_{X=0}
\newcommand{\sumnN}{\sum^N_{n=1}}
\newcommand{\sumin}{\sum^n_{i=1}}
\newcommand{\sumjn}{\sum^n_{j=1}}
\newcommand{\sumim}{\sum^m_{i=1}}
\newcommand{\sumik}{\sum^k_{i=1}}
\newcommand{\sumiN}{\sum^N_{i=1}}
\newcommand{\sumkn}{\sum^n_{k=1}}
\newcommand{\sumtT}{\sum^T_{t=1}}
\newcommand{\sumninf}{\sum^\infty_{n=1}}
\newcommand{\sumtinf}{\sum^\infty_{t=1}}
\newcommand{\sumnNz}{\sum^N_{n=0}}
\newcommand{\suminz}{\sum^n_{i=0}}
\newcommand{\sumknz}{\sum^n_{k=0}}
\newcommand{\sumtTz}{\sum^T_{t=0}}
\newcommand{\sumninfz}{\sum^\infty_{n=0}}
\newcommand{\sumtinfz}{\sum^\infty_{t=0}}

\newcommand{\prodnN}{\prod^N_{n=1}}
\newcommand{\prodin}{\prod^n_{i=1}}
\newcommand{\prodiN}{\prod^N_{i=1}}
\newcommand{\prodkn}{\prod^n_{k=1}}
\newcommand{\prodtT}{\prod^T_{t=1}}
\newcommand{\prodnNz}{\prod^N_{n=0}}
\newcommand{\prodinz}{\prod^n_{i=0}}
\newcommand{\prodknz}{\prod^n_{k=0}}
\newcommand{\prodtTz}{\prod^T_{t=0}}

% Bounds
\newcommand{\atob}{_a^b}
\newcommand{\ztoinf}{_0^\infty}
\newcommand{\kinf}{_{k=1}^\infty}
\newcommand{\ninf}{_{n=1}^\infty}
\newcommand{\minf}{_{m=1}^\infty}
\newcommand{\tinf}{_{t=1}^\infty}
\newcommand{\nN}{_{n=1}^N}
\newcommand{\kinfz}{_{k=0}^\infty}
\newcommand{\ninfz}{_{n=0}^\infty}
\newcommand{\minfz}{_{m=0}^\infty}
\newcommand{\tinfz}{_{t=0}^\infty}
\newcommand{\nNz}{_{n=0}^N}

% Limits
\newcommand{\limN}{\lim_{N\rightarrow\infty}}
\newcommand{\limn}{\lim_{n\rightarrow\infty}}
\newcommand{\limk}{\lim_{k\rightarrow\infty}}
\newcommand{\limt}{\lim_{t\rightarrow\infty}}
\newcommand{\limT}{\lim_{T\rightarrow\infty}}
\newcommand{\limhz}{\lim_{h\rightarrow 0}}

% Shorter integrals: ``Integral from X to Y''
% - intXY is equivalent to \int^Y_X
\newcommand{\intab}{\int_a^b}
\newcommand{\intzN}{\int_0^N}


%%%%%%%%%%%%%%%%%%%%%%%%%%%%%%%%%%%%%%%%%%%%%%%%%%%%%%%%%%%%%%%%%%%%%%%%
%% BODY %%%%%%%%%%%%%%%%%%%%%%%%%%%%%%%%%%%%%%%%%%%%%%%%%%%%%%%%%%%%%%%%
%%%%%%%%%%%%%%%%%%%%%%%%%%%%%%%%%%%%%%%%%%%%%%%%%%%%%%%%%%%%%%%%%%%%%%%%


\begin{document}
%\maketitle

\begin{multicols*}{2}
\paragraph{Accounting}
Let $P_t(n)$ denote the price at time $t$ of a default-risk-free zero
coupon bond that pays \$1 $n$ years hence. The corresponding
continuously compounded yield $y_t(n)$ satisfies
\begin{align}
  P_t(n) &= \exp\big(-n y_t(n) \big)
  \label{P}
  \\
  \implies\quad
  y_t(n) &= -\frac{1}{n}\ln\big( P_t(n)\big)
  \label{y}
\end{align}
We define the \emph{short rate} as the limit as $n\ra 0$:
\begin{align*}
  r_t := \lim_{n\ra 0} y_t(n)
  = \lim_{n\ra 0}
  -\frac{1}{n}\ln\big(P_t(n)\big)
\end{align*}
Using zero coupon bound prices, we can back out the implied $m$-year
duration, beginning $n$-years-ahead \emph{forward rate} at time $t$ as
\begin{align}
  f_t(n,m) &=
  -\frac{1}{m}
  \ln
  \left(
  \frac{P_t(n+m)}{P_t(n)}
  \right)
  \label{fwdP}
  \\
  &=
  \frac{1}{m}
  \big(
  (n+m)y_t(n+m)
  -ny_t(n)
  \big)
  \label{fwdy}
\end{align}
We now take the limit as $m\ra 0$, applying L'Hospital to
Expression~\ref{fwdy} to get the instantaneous forward rate
$n$-years-ahead
\begin{align*}
  f_t(n,0)
  %&=
  %\lim_{m\ra 0}
  %\frac{1}{m}
  %\big(
  %(n+m)y_t(n+m)
  %-ny_t(n)
  %\big)
  %\\
  &=
  \lim_{m\ra 0}
  y_t(n+m)
  +
  (n+m)\frac{\partial y_t(n+m)}{\partial n}
  \\
  &=
  \left(
  y_t(n)
  +
  n\frac{\partial y_t(n)}{\partial n}
  \right)
  \\
  &=
  \frac{\partial }{\partial n}
  \big[
  ny_t(n)
  \big]
  =
  -\frac{\partial }{\partial n}
  \big[
  \ln P_t(n)
  \big]
\end{align*}
We can integrate the last line to express any arbitrary yield $y_t(n)$
as a stringing-together of forward rate agreements:
\begin{align*}
  \int_0^n f_t(s,0) \; ds
  &=
  \int_0^n
  \frac{\partial }{\partial s}
  \big[
  sy_t(s)
  \big]
  \; ds
  \\
  \implies\quad
  y_t(n)
  &=
  \frac{1}{n}
  \int_0^n f_t(s,0) \; ds
\end{align*}
or, alternatively, take $m=1$ in Expression~\ref{fwdy}, rearrange, and
recursively expand to get
\begin{align*}
  y_t(n+1)
  &=
  \frac{1}{n+1}
  \big(
  ny_t(n)
  + f_t(n,1)
  \big)
  \\
  \implies\quad
  y_t(n)
  &=
  \frac{1}{n}
  \sum_{s=1}^{n}
  f_t(s-1,1)
\end{align*}
\columnbreak

Next, we define \emph{holding period return} as the return to buying an
$n$-year zero-coupon bond at $t$ and selling it $m$ years later at time
$t+m$:
\begin{align*}
  hpr_t(n,m)
  &=
  \frac{1}{m}
  \ln
  \left(
  \frac{P_{t+m}(n-m)}{P_t(n)}
  \right)
\end{align*}
This should be compared to simply buying the $m$-year bond to begin
with, which had yield $y_t(m)$. Therefore, we define the corresponding
\emph{excess holding period return} as
\begin{align*}
  exr_t(n,m)
  = hpr_t(n,m) - y_t(m)
\end{align*}

\paragraph{Asset Pricing}
Next, the stochastic discount factor $M_{t+1}$ (a strictly positive
random variable) satisfies the following relationship for any asset that
has price $P_t$ today at $t$ and stochastic payoff $X_{t+1}$ at time
$t+1$
\begin{align*}
  P_t = \E_t\big[M_{t+1}X_{t+1}\big]
\end{align*}
where the expectations is taken according to the historical aka
data-generating measure $Q$.
Therefore, a zero coupon bonds that pay $X_{t+1}=1$ in every state of
the world next year satisfies
\begin{align*}
  P_t(1)
  = \exp(-y_t(1))
  &= \E_t[M_{t+1}]
\end{align*}
For longer maturities, we string together a series of one-period
investments (first at price $P_t(1)$, then $P_{t+1}(1)$, etc.) and then
iterate expectations to get
\begin{align*}
  P_t(n)
  &= \E_t\left[\prod_{i=1}^n M_{t+i}\right]
\end{align*}
Alternatively, we can rewrite the pricing equation by taking the
expectation under the risk neutral probability measure $Q^*$:
\begin{align*}
  P_t = \E_t^*[X_{t+1}]
\end{align*}
In this framework, we can express the price of risk-free zero coupon
bond as the expected cost of continuously rolling over an investment in
the short rate
\begin{align*}
  P_t(n)
  &=
  \E^*_t\left[
    \exp\left(
    -\int_t^{t+\tau} r_s \; ds
    \right)
  \right]
\end{align*}
\end{multicols*}



\clearpage
\begin{multicols*}{2}
\paragraph{Affine Term Structure Models}
This class of models makes two assumptions:
\begin{enumerate}[label=(\roman*)]
  \item There is a $K\times 1$ vector of factors $X_t$ that determine
    the short rate:
    \begin{align*}
      y_t(1) &= \delta_0 + \delta_1' X_t
    \end{align*}
  \item Those factors evolve according to
    \begin{align*}
      X_{t+1} &= \mu + \Phi X_t + \Sigma \varepsilon_{t+1}
      \quad
      %\text{where}\;
      \varepsilon_{t+1}
      \sim N(0,I_K)
    \end{align*}
    Hence, $X_{t+1}$ is conditionally normal.

  \item
    The pricing kernel is conditionally lognormal satisfying
    \begin{align*}
      \ln(M_{t+1})
      &=
      -y_t(1) - \frac{1}{2}\lambda_t'\lambda_t
      - \lambda_t'\varepsilon_{t+1}
    \end{align*}
    where, for some $\lambda_0\in \R^K$ and
    $\lambda_1\in\R^{K\times K}$,
    \begin{align*}
      \lambda_t = \lambda_0 + \lambda_1 X_t
    \end{align*}
    The random variable $\lambda_t$, called the ``price of risk'',
    specifies comovement between the discount factor $M_{t+1}$ and the
    underlying driving shocks $\varepsilon_{t+1}$ to the factors
    $X_{t+1}$. The other terms are there to satisfy theory.

    In particular, given time $t$ information,
    $\varepsilon_{t+1}\sim N(0,I_K)$ is the only source of randomness,
    so that $\ln(M_{t+1})$ satisfies
    \begin{align*}
      \ln(M_{t+1})
      \sim
      N\left(
      -y_t(1)-\frac{1}{2}\lambda_t'\lambda_t
      ,\; \lambda_t'\lambda_t
      \right)
    \end{align*}
    Therefore, the mean formula for a log-normal distribution tells us
    that
    \begin{align*}
      \E_t[M_{t+1}]
      &= \exp\left(-y_t(1)
        -\frac{1}{2}\lambda_t'\lambda_t
        +\frac{1}{2}\lambda_t'\lambda_t
      \right)
      \\
      &= \exp(-y_t(1))
      = P_t(1)
    \end{align*}
    which is the basic asset pricing equation.
\end{enumerate}
\columnbreak

\end{multicols*}

\clearpage
First, start with
\begin{align*}
  \lambda_{t+1}
  &= \lambda_0 + \lambda_1 X_{t+1} \\
  &= \lambda_0 + \lambda_1\big(\mu+\Phi X_t+\Sigma\varepsilon_{t+1}\big)  \\
  &= \lambda_0 + \lambda_1 X_t - \lambda_1 X_t + \lambda_1\big(\mu+\Phi X_t+\Sigma\varepsilon_{t+1}\big)  \\
  \implies\quad
  \lambda_{t+1}
  &= \lambda_t
  + \lambda_1\big(\mu+(\Phi - I_K) X_t+\Sigma\varepsilon_{t+1}\big)
\end{align*}
Hence,
\begin{align*}
  \lambda_{t+k}
  &= \lambda_{t+k-1}
  + \lambda_1\big(\mu+(\Phi - I_K) X_{t+k-1}+\Sigma\varepsilon_{t+k}\big)
  \\
  &=
  \lambda_{t+k-2}
  + \lambda_1\big(\mu+(\Phi - I_K) X_{t+k-2}+\Sigma\varepsilon_{t+k-1}\big)
  + \lambda_1\big(\mu+(\Phi - I_K) X_{t+k-1}+\Sigma\varepsilon_{t+k}\big)
  \\
  &=
  \lambda_{t}
  + k \lambda_1 \mu
  + \lambda_1(\Phi-I_K)
  \sum_{i=0}^{k-1} X_{t+i}
  + \lambda_1\Sigma \sum_{i=1}^k \varepsilon_{t+i}
  \\
  &=
  \lambda_{0} + \lambda_1 X_t
  + k \lambda_1 \mu
  + \lambda_1(\Phi-I_K)X_t
  + \lambda_1(\Phi-I_K)
  \sum_{i=1}^{k-1} X_{t+i}
  + \lambda_1\Sigma \sum_{i=1}^k \varepsilon_{t+i}
  \\
  \implies\quad
  \lambda_{t+k}
  &=
  \lambda_{0}
  + \lambda_1
  \left[
  k \mu
  + \Phi X_t
  + (\Phi-I_K)
  \sum_{i=1}^{k-1} X_{t+i}
  + \sum_{i=1}^k \Sigma \varepsilon_{t+i}
  \right]
\end{align*}
Next, we write $X_{t+i}$ in terms of $X_t$ and subsequent shocks
\begin{align*}
  X_{t+i}
  &=
  \mu + \Phi X_{t+i-1}
  + \Sigma \varepsilon_{t+i}
  \\
  &=
  \mu + \Phi
  \big(
  \mu + \Phi X_{t+i-2}
  + \Sigma \varepsilon_{t+i-1}
  \big)
  + \Sigma \varepsilon_{t+i}
  \\
  &=
  \left(
  \sum_{j=1}^i \Phi^{i-j}
  \right)
  \mu
  +
  \Phi^i X_t
  +
  \sum_{j=1}^i \Phi^{i-j}\Sigma\varepsilon_{t+j}
  \\
  &=
  (\Phi - I_K)^{-1}(\Phi^i - I_K)
  \mu
  +
  \Phi^i X_t
  +
  \sum_{j=1}^i \Phi^{i-j}\Sigma\varepsilon_{t+j}
\end{align*}
Where $S = \sum_{j=1}^i \Phi^{i-j}$ so that
\begin{align*}
  S
  &= \Phi^{i-1} + \Phi^{i-2} + \cdots + \Phi + I_K
  \\
  \implies\quad
  \Phi S &=
  \Phi^{i} + \Phi^{i-1} + \cdots + \Phi^2 + \Phi
  \\
  \implies\quad
   S &=
  (\Phi - I_K)^{-1}(\Phi^i - I_K)
\end{align*}
Therefore,
\begin{align*}
  (\Phi-I_K)
  \sum_{i=1}^{k-1} X_{t+i}
  &=
  (\Phi-I_K)
  \sum_{i=1}^{k-1}
  \left[
  (\Phi - I_K)^{-1}(\Phi^i - I_K)
  \mu
  +
  \Phi^i X_t
  +
  \sum_{j=1}^i \Phi^{i-j}\Sigma\varepsilon_{t+j}
  \right]
  \\
  &=
  \sum_{i=1}^{k-1}
  (\Phi^i - I_K)
  \mu
  +
  (\Phi-I_K)
  \left[
  \sum_{i=1}^{k-1}
  \Phi^i X_t
  +
  \sum_{i=1}^{k-1}
  \sum_{j=1}^i \Phi^{i-j}\Sigma\varepsilon_{t+j}
  \right]
  \\
  &=
  \left(
  \sum_{i=1}^{k-1}
  \Phi^i
  \right)
  \mu
  -
  (k-1)
  \mu
  +
  (\Phi-I_K)
  \left[
  \left(
  \sum_{i=1}^{k-1}
  \Phi^i
  \right)
  X_t
  +
  \sum_{i=1}^{k-1}
  \sum_{j=1}^i \Phi^{i-j}\Sigma\varepsilon_{t+j}
  \right]
\end{align*}
We can simplify again:
\begin{align*}
  S
  &=
  \sum_{i=1}^{k-1}
  \Phi^{i}
  = \Phi + \Phi^2 + \cdots + \Phi^{k-1}
  \\
  \Phi S
  &= \Phi^2 + \Phi^3 + \cdots + \Phi^{k}
  \\
  \implies\quad
  S
  &=
  (\Phi-I_K)^{-1}
  (\Phi^{k} - \Phi)
\end{align*}
Plugging back in
\begin{align*}
  (\Phi-I_K)
  \sum_{i=1}^{k-1} X_{t+i}
  &=
  (\Phi-I_K)^{-1}
  (\Phi^{k} - \Phi)
  \mu
  -
  (k-1)
  \mu
  +
  (\Phi-I_K)
  \left[
  (\Phi-I_K)^{-1}
  (\Phi^{k} - \Phi)
  X_t
  +
  \sum_{i=1}^{k-1}
  \sum_{j=1}^i \Phi^{i-j}\Sigma\varepsilon_{t+j}
  \right]
  \\
  &=
  (\Phi-I_K)^{-1}
  (\Phi^{k} - \Phi)
  \mu
  -
  (k-1)
  \mu
  +
  (\Phi^{k} - \Phi)
  X_t
  +
  (\Phi-I_K)
  \sum_{i=1}^{k-1}
  \sum_{j=1}^i \Phi^{i-j}\Sigma\varepsilon_{t+j}
\end{align*}
Next, we consider
\begin{align*}
  \sum_{i=1}^{k-1}
  \sum_{j=1}^i \Phi^{i-j}\Sigma\varepsilon_{t+j}
  &=
  \sum_{j=1}^1 \Phi^{1-j}\Sigma\varepsilon_{t+j}
  +
  \sum_{j=1}^2 \Phi^{2-j}\Sigma\varepsilon_{t+j}
  +
  \cdots
  +
  \sum_{j=1}^{k-1} \Phi^{k-j-1}\Sigma\varepsilon_{t+j}
  \\
  &=
  \Sigma\varepsilon_{t+1}
  +
  \big(
  \Phi\Sigma\varepsilon_{t+1}
  +
  \Sigma\varepsilon_{t+2}
  \big)
  +
  \cdots
  +
  \big(
  \Phi^{k-2}\Sigma\varepsilon_{t+1}
  +\cdots+
  \Sigma\varepsilon_{t+k-1}
  \big)
  \\
  &=
  \sum_{i=1}^{k-1}
  \sum_{j=0}^{k-1-i}
  \Phi^{j}\Sigma \varepsilon_{t+i}
\end{align*}
Next, we compute
\begin{align*}
  S_i
  &=
  \sum_{j=0}^{k-1-i}
  \Phi^j
  = I_K + \Phi + \cdots + \Phi^{k-1-i}
  \\
  \Phi S_i
  &= \Phi + \Phi^2 + \cdots + \Phi^{k-i}
  \\
  \implies\quad
  S_i
  &=
  (\Phi-I_K)^{-1}
  (\Phi^{k-i}-I_K)
\end{align*}
Therefore
\begin{align*}
  \sum_{i=1}^{k-1}
  \sum_{j=1}^i \Phi^{i-j}\Sigma\varepsilon_{t+j}
  &=
  \sum_{i=1}^{k-1}
  \sum_{j=0}^{k-1-i}
  \Phi^{j}\Sigma \varepsilon_{t+i}
  =
  (\Phi-I_K)^{-1}
  \sum_{i=1}^{k-1}
  (\Phi^{k-i}-I_K)
  \Sigma \varepsilon_{t+i}
\end{align*}
So we have
\begin{align*}
  (\Phi-I_K)
  \sum_{i=1}^{k-1} X_{t+i}
  &=
  (\Phi-I_K)^{-1}
  (\Phi^{k} - \Phi)
  \mu
  -
  (k-1)
  \mu
  +
  (\Phi^{k} - \Phi)
  X_t
  +
  \sum_{i=1}^{k-1}
  (\Phi^{k-i}-I_K)
  \Sigma \varepsilon_{t+i}
\end{align*}
Hence
\begin{align*}
  \lambda_{t+k}
  &=
  \lambda_{0}
  + \lambda_1
  \left[
  k \mu
  + \Phi X_t
  +
  \left(
  (\Phi-I_K)
  \sum_{i=1}^{k-1} X_{t+i}
  \right)
  + \sum_{i=1}^k \Sigma \varepsilon_{t+i}
  \right]
  \\
  &=
  \lambda_{0}
  + \lambda_1
  \left[
  k \mu
  + \Phi X_t
  +
  \left(
  (\Phi-I_K)^{-1}
  (\Phi^{k} - \Phi)
  \mu
  -
  (k-1)
  \mu
  +
  (\Phi^{k} - \Phi)
  X_t
  +
  \sum_{i=1}^{k-1}
  (\Phi^{k-i}-I_K)
  \Sigma \varepsilon_{t+i}
  \right)
  + \sum_{i=1}^k \Sigma \varepsilon_{t+i}
  \right]
  \\
  &=
  \lambda_{0}
  + \lambda_1
  \left[
  (\Phi-I_K)^{-1}
  (\Phi^{k} - \Phi)
  \mu
  +
  \mu
  +
  \Phi^{k}
  X_t
  +
  \sum_{i=1}^{k-1}
  (\Phi^{k-i}-I_K)
  \Sigma \varepsilon_{t+i}
  + \sum_{i=1}^{k-1} \Sigma \varepsilon_{t+i}
  + \Sigma\varepsilon_{t+k}
  \right]
  \\
  &=
  \lambda_{0}
  + \lambda_1
  \left[
  (\Phi-I_K)^{-1}
  (\Phi^{k} - \Phi)
  \mu
  +
  \mu
  +
  \Phi^{k}
  X_t
  +
  \sum_{i=1}^{k-1}
  \Phi^{k-i}
  \Sigma \varepsilon_{t+i}
  + \Sigma\varepsilon_{t+k}
  \right]
\end{align*}
Next, we price more general $P_t(n)$ recursively.
\begin{align*}
  P_t(n)
  &= \E_t
  \left[
    \prod_{i=1}^n M_{t+i}
  \right]
  \\
  &= \E_t
  \left[
    \prod_{i=0}^{n-1}
    \exp\left(
    -y_{t+i}(1)
    - \frac{1}{2}\lambda_{t+i}'\lambda_{t+i}
    - \lambda_{t+i}'\varepsilon_{t+i+1}
    \right)
  \right]
  \\
  &= \E_t
  \left[
    \exp\left(
    -\sum_{i=0}^{n-1}
    y_{t+i}(1)
    + \frac{1}{2}\lambda_{t+i}'\lambda_{t+i}
    + \lambda_{t+i}'\varepsilon_{t+i+1}
    \right)
  \right]
  \\
  &= \E_t
  \left[
    \exp\left(
    -\sum_{i=0}^{n-1}
    \big[
    \delta_0 + \delta_1'X_{t+i}
    \big]
    + \lambda_{t+i}'
    \left[
    \frac{1}{2}\lambda_{t+i}
    +\varepsilon_{t+i+1}
    \right]
    \right)
  \right]
  \\
  &= \E_t
  \left[
    \exp\left(
    -\sum_{i=0}^{n-1}
    \big[
    \delta_0 + \delta_1'X_{t+i}
    \big]
    +
    \big(
    \lambda_0 + \lambda_1X_{t+i}
    \big)'
    \left[
    \frac{1}{2}
    \big(
    \lambda_0 + \lambda_1 X_{t+i}
    \big)
    +\varepsilon_{t+i+1}
    \right]
    \right)
  \right]
  \\
  &= \E_t
  \left[
    \exp\left(
    -\sum_{i=0}^{n-1}
    \delta_0 +
    \delta_1'X_{t+i}
    +
    \frac{1}{2}
    \big(
    \lambda_0 + \lambda_1X_{t+i}
    \big)'
    \big(
    \lambda_0 + \lambda_1X_{t+i}
    \big)
    +
    \big(
    \lambda_0 + \lambda_1X_{t+i}
    \big)'
    \varepsilon_{t+i+1}
    \right)
  \right]
  \\
  &= \E_t
  \left[
    \exp\left(
    -\sum_{i=0}^{n-1}
    \delta_0 +
    \delta_1'X_{t+i}
    +
    \frac{1}{2}
    \big(
    \lambda_0' + X_{t+i}'\lambda_1'
    \big)
    \big(
    \lambda_0 + \lambda_1X_{t+i}
    \big)
    +
    \big(
    \lambda_0' + X_{t+i}'\lambda_1'
    \big)
    \varepsilon_{t+i+1}
    \right)
  \right]
  \\
  &= \E_t
  \left[
    \exp\left(
    -\sum_{i=0}^{n-1}
    \delta_0 +
    \delta_1'X_{t+i}
    +
    \frac{1}{2}
    \big(
    \lambda_0'\lambda_0
    +\lambda_0'\lambda_1X_{t+i}
    + X_{t+i}'\lambda_1'\lambda_0
    + X_{t+i}'\lambda_1'\lambda_1X_{t+i}
    \big)
    +
    \lambda_0'
    \varepsilon_{t+i+1}
    + X_{t+i}'\lambda_1'
    \varepsilon_{t+i+1}
    \right)
  \right]
  \\
  &= \E_t
  \left[
    \exp\left(
    -\sum_{i=0}^{n-1}
    \delta_0 +
    \big(
    \delta_1'
    + \frac{1}{2}\lambda_0'\lambda_1
    \big)
    X_{t+i}
    +
    \frac{1}{2}
    \big(
    \lambda_0'\lambda_0
    + X_{t+i}'\lambda_1'\lambda_0
    + X_{t+i}'\lambda_1'\lambda_1X_{t+i}
    \big)
    +
    \lambda_0'
    \varepsilon_{t+i+1}
    + X_{t+i}'\lambda_1'
    \varepsilon_{t+i+1}
    \right)
  \right]
  \\
\end{align*}


%% APPPENDIX %%

% \appendix




\end{document}


%%%%%%%%%%%%%%%%%%%%%%%%%%%%%%%%%%%%%%%%%%%%%%%%%%%%%%%%%%%%%%%%%%%%%%%%
%%%%%%%%%%%%%%%%%%%%%%%%%%%%%%%%%%%%%%%%%%%%%%%%%%%%%%%%%%%%%%%%%%%%%%%%
%%%%%%%%%%%%%%%%%%%%%%%%%%%%%%%%%%%%%%%%%%%%%%%%%%%%%%%%%%%%%%%%%%%%%%%%

%%%% SAMPLE CODE %%%%%%%%%%%%%%%%%%%%%%%%%%%%%%%%%%%%%%

    %% VIEW LAYOUT %%

        \layout

    %% LANDSCAPE PAGE %%

        \begin{landscape}
        \end{landscape}

    %% BIBLIOGRAPHIES %%

        \cite{LabelInSourcesFile}  %Use in text; cites
        \citep{LabelInSourcesFile} %Use in text; cites in parens

        \nocite{LabelInSourceFile} % Includes in refs w/o specific citation
        \bibliographystyle{apalike}  % Or some other style

        % To ditch the ``References'' header
        \begingroup
        \renewcommand{\section}[2]{}
        \endgroup

        \bibliography{sources} % where sources.bib has all the citation info

    %% SPACING %%

        \vspace{1in}
        \hspace{1in}

    %% URLS, EMAIL, AND LOCAL FILES %%

      \url{url}
      \href{url}{name}
      \href{mailto:mcocci@raidenlovessusie.com}{name}
      \href{run:/path/to/file.pdf}{name}


    %% INCLUDING PDF PAGE %%

        \includepdf{file.pdf}


    %% INCLUDING CODE %%

        %\verbatiminput{file.ext}
            %   Includes verbatim text from the file

        \texttt{text}
            %   Renders text in courier, or code-like, font

        \matlabcode{file.m}
            %   Includes Matlab code with colors and line numbers

        \lstset{style=bash}
        \begin{lstlisting}
        \end{lstlisting}
            % Inline code rendering


    %% INCLUDING FIGURES %%

        % Basic Figure with size scaling
            \begin{figure}[h!]
               \centering
               \includegraphics[scale=1]{file.pdf}
            \end{figure}

        % Basic Figure with specific height
            \begin{figure}[h!]
               \centering
               \includegraphics[height=5in, width=5in]{file.pdf}
            \end{figure}

        % Figure with cropping, where the order for trimming is  L, B, R, T
            \begin{figure}
               \centering
               \includegraphics[trim={1cm, 1cm, 1cm, 1cm}, clip]{file.pdf}
            \end{figure}

        % Side by Side figures: Use the tabular environment


