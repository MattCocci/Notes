\documentclass[12pt]{article}

\author{Matthew D. Cocci}
\title{Macroeconomics}
\date{\today}

%% Formatting & Spacing %%%%%%%%%%%%%%%%%%%%%%%%%%%%%%%%%%%%

%\usepackage[top=1in, bottom=1in, left=1in, right=1in]{geometry} % most detailed page formatting control
\usepackage{fullpage} % Simpler than using the geometry package; std effect
\usepackage{setspace}
%\onehalfspacing
\usepackage{microtype}

%% Formatting %%%%%%%%%%%%%%%%%%%%%%%%%%%%%%%%%%%%%%%%%%%%%

%\usepackage[margin=1in]{geometry}
    %   Adjust the margins with geometry package
%\usepackage{pdflscape}
    %   Allows landscape pages
%\usepackage{layout}
    %   Allows plotting of picture of formatting



%% Header %%%%%%%%%%%%%%%%%%%%%%%%%%%%%%%%%%%%%%%%%%%%%%%%%

%\usepackage{fancyhdr}
%\pagestyle{fancy}
%\lhead{}
%\rhead{}
%\chead{}
%\setlength{\headheight}{15.2pt}
    %   Make the header bigger to avoid overlap

%\fancyhf{}
    %   Erase header settings

%\renewcommand{\headrulewidth}{0.3pt}
    %   Width of the line

%\setlength{\headsep}{0.2in}
    %   Distance from line to text


%% Mathematics Related %%%%%%%%%%%%%%%%%%%%%%%%%%%%%%%%%%%

\usepackage{amsmath}
\usepackage{amssymb}
\usepackage{amsfonts}
\usepackage{mathrsfs}
\usepackage{amsthm} %allows for labeling of theorems
%\numberwithin{equation}{section} % Number equations by section
\theoremstyle{plain}
\newtheorem{thm}{Theorem}[section]
\newtheorem{lem}[thm]{Lemma}
\newtheorem{prop}[thm]{Proposition}
\newtheorem{cor}[thm]{Corollary}

\theoremstyle{definition}
\newtheorem{defn}[thm]{Definition}
\newtheorem{ex}[thm]{Example}

\theoremstyle{remark}
\newtheorem*{rmk}{Remark}
\newtheorem*{note}{Note}

% Below supports left-right alignment in matrices so the negative
% signs don't look bad
\makeatletter
\renewcommand*\env@matrix[1][c]{\hskip -\arraycolsep
  \let\@ifnextchar\new@ifnextchar
  \array{*\c@MaxMatrixCols #1}}
\makeatother


%% Font Choices %%%%%%%%%%%%%%%%%%%%%%%%%%%%%%%%%%%%%%%%%

\usepackage[T1]{fontenc}
\usepackage{lmodern}
\usepackage[utf8]{inputenc}
%\usepackage{blindtext}
\usepackage{courier}


%% Figures %%%%%%%%%%%%%%%%%%%%%%%%%%%%%%%%%%%%%%%%%%%%%%

\usepackage{tikz}
\usetikzlibrary{decorations.pathreplacing}
\usepackage{graphicx}
\usepackage{subfigure}
    %   For plotting multiple figures at once
%\graphicspath{ {Directory/} }
    %   Set a directory for where to look for figures


%% Hyperlinks %%%%%%%%%%%%%%%%%%%%%%%%%%%%%%%%%%%%%%%%%%%%
\usepackage{hyperref}
\hypersetup{%
    colorlinks,
        %   This colors the links themselves, not boxes
    citecolor=black,
        %   Everything here and below changes link colors
    filecolor=black,
    linkcolor=black,
    urlcolor=black
}

%% Colors %%%%%%%%%%%%%%%%%%%%%%%%%%%%%%%%%%%%%%%%%%%%%%%

\usepackage{color}
\definecolor{codegreen}{RGB}{28,172,0}
\definecolor{codelilas}{RGB}{170,55,241}

% David4 color scheme
\definecolor{d4blue}{RGB}{100,191,255}
\definecolor{d4gray}{RGB}{175,175,175}
\definecolor{d4black}{RGB}{85,85,85}
\definecolor{d4orange}{RGB}{255,150,100}

%% Including Code %%%%%%%%%%%%%%%%%%%%%%%%%%%%%%%%%%%%%%%

\usepackage{verbatim}
    %   For including verbatim code from files, no colors
\usepackage{listings}
    %   For including code snippets written directly in this doc

\lstdefinestyle{bash}{%
  language=bash,%
  basicstyle=\footnotesize\ttfamily,%
  showstringspaces=false,%
  commentstyle=\color{gray},%
  keywordstyle=\color{blue},%
  xleftmargin=0.25in,%
  xrightmargin=0.25in
}

\lstdefinestyle{matlab}{%
  language=Matlab,%
  basicstyle=\footnotesize\ttfamily,%
  breaklines=true,%
  morekeywords={matlab2tikz},%
  keywordstyle=\color{blue},%
  morekeywords=[2]{1}, keywordstyle=[2]{\color{black}},%
  identifierstyle=\color{black},%
  stringstyle=\color{codelilas},%
  commentstyle=\color{codegreen},%
  showstringspaces=false,%
    %   Without this there will be a symbol in
    %   the places where there is a space
  numbers=left,%
  numberstyle={\tiny \color{black}},%
    %   Size of the numbers
  numbersep=9pt,%
    %   Defines how far the numbers are from the text
  emph=[1]{for,end,break,switch,case},emphstyle=[1]\color{red},%
    %   Some words to emphasise
}

\newcommand{\matlabcode}[1]{%
    \lstset{style=matlab}%
    \lstinputlisting{#1}
}
    %   For including Matlab code from .m file with colors,
    %   line numbering, etc.

%% Bibliographies %%%%%%%%%%%%%%%%%%%%%%%%%%%%%%%%%%%%

%\usepackage{natbib}
    %---For bibliographies
%\setlength{\bibsep}{3pt} % Set how far apart bibentries are

%% Misc %%%%%%%%%%%%%%%%%%%%%%%%%%%%%%%%%%%%%%%%%%%%%%

\usepackage{enumitem}
    %   Has to do with enumeration
\usepackage{appendix}
%\usepackage{natbib}
    %   For bibliographies
\usepackage{pdfpages}
    %   For including whole pdf pages as a page in doc


%% User Defined %%%%%%%%%%%%%%%%%%%%%%%%%%%%%%%%%%%%%%%%%%

%\newcommand{\nameofcmd}{Text to display}
\newcommand*{\Chi}{\mbox{\large$\chi$}} %big chi
    %   Bigger Chi

% In math mode, Use this instead of \munderbar, since that changes the
% font from math to regular
\makeatletter
\def\munderbar#1{\underline{\sbox\tw@{$#1$}\dp\tw@\z@\box\tw@}}
\makeatother

% Limits
\newcommand{\limN}{\lim_{N\rightarrow\infty}}
\newcommand{\limn}{\lim_{n\rightarrow\infty}}
\newcommand{\limt}{\lim_{t\rightarrow\infty}}
\newcommand{\limT}{\lim_{T\rightarrow\infty}}
\newcommand{\limhz}{\lim_{h\rightarrow 0}}

% Misc Math
\newcommand{\Prb}{\mathrm{P}}
\newcommand{\ra}{\rightarrow}
\newcommand{\diag}{\text{diag}}
\newcommand{\ch}{\text{ch}}
\newcommand{\dom}{\text{dom}}

% Script
\newcommand{\sF}{\mathscr{F}}
\newcommand{\sB}{\mathscr{B}}
\newcommand{\sL}{\mathscr{L}}
\newcommand{\sM}{\mathscr{M}}
\newcommand{\sT}{\mathscr{T}}
\newcommand{\sA}{\mathscr{A}}

% Mathcal
\newcommand{\calB}{\mathcal{B}}
\newcommand{\calD}{\mathcal{D}}
\newcommand{\calF}{\mathcal{F}}
\newcommand{\calG}{\mathcal{G}}
\newcommand{\calH}{\mathcal{H}}

% Dot over
\newcommand{\dx}{\dot{x}}
\newcommand{\ddx}{\ddot{x}}
\newcommand{\dy}{\dot{y}}
\newcommand{\ddy}{\ddot{y}}

% Derivatives
\newcommand{\dydx}{\frac{dy}{dx}}
\newcommand{\pypx}{\frac{\partial y}{\partial x}}

% Blackboard
\newcommand{\R}{\mathbb{R}}
\newcommand{\Rn}{\mathbb{R}^n}
\newcommand{\Rk}{\mathbb{R}^n}
\newcommand{\Rnn}{\mathbb{R}^{n\times n}}
\newcommand{\C}{\mathbb{C}}
\newcommand{\Cn}{\mathbb{C}^n}
\newcommand{\Cnn}{\mathbb{C}^{n\times n}}
\newcommand{\E}{\mathbb{E}}
\newcommand{\N}{\mathbb{N}}

\DeclareMathOperator*{\argmin}{arg\;min}
\DeclareMathOperator*{\argmax}{arg\;max}
\newenvironment{rcases}
  {\left.\begin{aligned}}
  {\end{aligned}\right\rbrace}

% Various probability and statistics commands
\newcommand{\Cov}{\operatorname{Cov}}
\newcommand{\Corr}{\operatorname{Corr}}
\newcommand{\Var}{\operatorname{Var}}
\newcommand{\asto}{\xrightarrow{a.s.}}
\newcommand{\pto}{\xrightarrow{p}}
\newcommand{\msto}{\xrightarrow{m.s.}}
\newcommand{\dto}{\xrightarrow{d}}
\newcommand{\Lpto}{\xrightarrow{L_p}}
\newcommand{\plim}{\text{plim}_{n\rightarrow\infty}}

% Redefine real and imaginary from fraktur to plain text
\renewcommand{\Re}{\operatorname{Re}}
\renewcommand{\Im}{\operatorname{Im}}

% Shorter sums: ``Sum from X to Y''
% - sumXY  is equivalent to \sum^Y_{X=1}
% - sumXYz is equivalent to \sum^Y_{X=0}
\newcommand{\sumnN}{\sum^N_{n=1}}
\newcommand{\sumin}{\sum^n_{i=1}}
\newcommand{\sumkn}{\sum^n_{k=1}}
\newcommand{\sumtT}{\sum^T_{t=1}}
\newcommand{\sumninf}{\sum^\infty_{n=1}}
\newcommand{\sumtinf}{\sum^\infty_{t=1}}
\newcommand{\sumnNz}{\sum^N_{n=0}}
\newcommand{\suminz}{\sum^n_{i=0}}
\newcommand{\sumknz}{\sum^n_{k=0}}
\newcommand{\sumtTz}{\sum^T_{t=0}}
\newcommand{\sumninfz}{\sum^\infty_{n=0}}
\newcommand{\sumtinfz}{\sum^\infty_{t=0}}

% Shorter integrals: ``Integral from X to Y''
% - intXY is equivalent to \int^Y_X
\newcommand{\intab}{\int_a^b}
\newcommand{\intzN}{\int_0^N}


%%%%%%%%%%%%%%%%%%%%%%%%%%%%%%%%%%%%%%%%%%%%%%%%%%%%%%%%%%%%%%%%%%%%%%%%
%% BODY %%%%%%%%%%%%%%%%%%%%%%%%%%%%%%%%%%%%%%%%%%%%%%%%%%%%%%%%%%%%%%%%
%%%%%%%%%%%%%%%%%%%%%%%%%%%%%%%%%%%%%%%%%%%%%%%%%%%%%%%%%%%%%%%%%%%%%%%%


\begin{document}
\maketitle

\tableofcontents

\clearpage

\section{Building Blocks}

\subsection{Utility Functions}

\begin{defn}(Absolute Risk Aversion)
Given twice-differentiable utility function $u$,
\emph{absolute risk aversion} is defined
\begin{align*}
  A(c) = -\frac{u''(c)}{u'(c)}
\end{align*}
\end{defn}

\begin{defn}(Relative Risk Aversion)
Given twice-differentiable utility function $u$,
\emph{relative risk aversion} is defined
\begin{align*}
  R(c) = cA(c) = -\frac{c u''(c)}{u'(c)}
\end{align*}
\end{defn}

\begin{defn}(Intertemporal Elasticity of Substitution)
Given differentiable utility function $u$,
the \emph{intertemporal elasticity of substitution} is defined
\begin{align*}
  IES
  = -\frac{\partial \ln(c_{t+1}/c_t)}{\partial r}
  = -\frac{\partial \ln(c_{t+1}/c_t)}{%
    \partial \ln(u'(c_{t+1})/u'(c_t))}
\end{align*}
High IES means that the the consumer will change her consumption
dramatically given small changes in the interest rate---high elasticity,
comfortable substituting across time.

Low IES corresponds to a strong smoothing motive. The consumer is
\emph{not} very responsive to changes in the interest rate.
\end{defn}

\begin{defn}(Power Utility or CRRA)
We define the period power utility function as
\begin{align*}
  u(c) &= \frac{c^{1-\sigma}-1}{1-\sigma} \\
  u'(c) &= c^{-\sigma} \\
  u''(c) &= -\sigma c^{-\sigma-1}
\end{align*}
It has the feature that
\begin{align*}
  \lim_{\sigma\ra 1}
  \frac{c^{1-\sigma}-1}{1-\sigma}
  = \ln(c)
\end{align*}
which is why we have $-1$ in the numerator (so we can apply L'Hospital),
though it's just a constant that doesn't really affect $u$.
\\
\\
It exhibits constant relative risk aversion
\begin{align*}
  R(c) &=
  -\frac{c u''(c)}{u'(c)}
  = -\frac{c (-\sigma c^{-\sigma-1})}{c^{-\sigma}}
  = \sigma
\end{align*}
The IES is also tied to the coefficient of relative risk aversion:
\begin{align*}
  IES
  = \frac{1}{\sigma}
\end{align*}
\end{defn}

\begin{prop}
The Power Utility Function characterizes the class of all utility
functions with Constant Relative Risk Aversion.
\end{prop}
\begin{proof}
To have constant relative risk aversion, we need
\begin{align*}
  -\frac{cu''(c)}{u'(c)} = k
  \quad\iff\quad
  cu''(c) + ku'(c) = 0
\end{align*}
for some constant $k$. We want to solve for $u$ that satisfies this. To
do so, define $v(c)=u'(c)$, substitute, and solve:
\begin{align*}
  v'(c) + \frac{k}{c}v(c) &= 0
\end{align*}
We will use an integrating factor
\begin{align*}
  e^{\int \frac{k}{c} \; dc}=e^{k\ln c} = (e^{\ln c})k = c^k
\end{align*}
Therefore, multiply through and solve
\begin{align*}
  c^k v'(c) + kc^{k-1}v(c) &= 0 \\
  \left(c^k v(c)\right)' &= 0 \\
  c^k v(c) &= 1 \\
  \implies \quad v(c) = u'(c) &= c^{-k} \\
  \implies \quad u(c) &= c^{1-k}
\end{align*}
where I left out constants in lots of spots because scaling by a
constant or adding constants doesn't matter for utility. But otherwise,
you see that this is exactly the form of power utility where $k=\sigma$.
\end{proof}

\begin{prop}
Given a balanced growth path for consumption such that
$c_{t+1} = (1+g) c_{t}$,
Power Utility is the only functional form where the ratio of marginal
utilty between periods satisfies
\begin{align*}
  \frac{u'(c_t)}{u'((1+g)c_t)} = k
\end{align*}
for some constant $k$ and all $t$.
\end{prop}
\begin{rmk}
This is often super useful, because some balanced growth path results
require that the ratio of marginal utility between periods is constant.
This tells us that we have to use this form of utility.
\end{rmk}
\begin{proof}
We want to prove that Power Utility is the \emph{only} functional form
of utility displaying this property. Start with what we need to hold:
\begin{align}
  \frac{u'(c)}{u'((1+g)c)} = k
  \quad\iff\quad
  u'(c) = ku'((1+g)c)
  \label{kexp}
\end{align}
Differentiate both sides:
\begin{align*}
  u''(c) &= ku''((1+g)c)(1+g)
\end{align*}
Sub in the expression for $k$ from Equation~\ref{kexp}:
\begin{align*}
  u''(c) &=
  \frac{u'(c)}{u'((1+g)c)} u''((1+g)c)(1+g) \\
  \iff \qquad
  \frac{u''(c)}{u'(c)} &=
  \frac{u''((1+g)c)(1+g)}{u'((1+g)c)}
\end{align*}
Mulitply through by $-c$
\begin{align*}
  -\frac{u''(c)c}{u'(c)} &=
  -\frac{u''((1+g)c)(1+g)c}{u'((1+g)c)}
\end{align*}
But each ratio is relative risk aversion at either $c_t$ or $c_{t+1}$.
So this says that relative risk aversion for all $t=1,2,\ldots$ even
though consumption $c_t$ is growing at rate of $g$. In other words,
relative risk aversion is constant no matter the level of $c_t$. But
power utility is the only class of utility functions satisfying this
property.
\end{proof}



\section{Static Resource Allocation Problems}

\begin{defn}{(Economy)}
We define an \emph{economy} $E$ for $I$ individuals and $K$ goods as a
list
\begin{align*}
  E = \{(X_i,u_i,\omega_i)\}^I_{i=1}
\end{align*}
\end{defn}

\clearpage

\section{Neoclassical Growth Model}

\subsection{Baseline Model Specification}
We set up the model
\begin{enumerate}
  \item \emph{Preferences}: $\sumtinfz \beta^t u(c_t)$

  \item \emph{Technology}: Output $y_t$ is produced by a CRS production
    function that takes capital $k_t$ and labor $h_t$ as input:
    \begin{align*}
      y_t &= F(k_t,h_t)
    \end{align*}
    Output can be used for consumption $c_t$ or investment $i_t$:
    \begin{align*}
      y_t &= c_i + i_t
    \end{align*}
    Investment is how the household accumulate capital, though
    depreciation eats away at the stock of capital at rate
    $\delta\in[0,1]$:
    \begin{align*}
      k_{t+1} &= (1-\delta)k_t + i_t
    \end{align*}

  \item \emph{Endowments}: Each household has one unit of time for labor
    each period, and initial capital level $k_0=\bar{k}_0$ is given.
\end{enumerate}
We place the following assumptions on the problem:
\begin{itemize}
  \item $u \in C^2$ (i.e.\ is continuously twice differentiable) with
    $u' > 0$ and $u$ strictly concave. Moreover, $u'(0)=+\infty$, so the
    consumer really hates zero consumption (which will guarantee an
    interior solution)
  \item $\beta\in(0,1)$
  \item $F\in C^2$, strictly increasing, weakly concave in $(k,h)$
    jointly, strictly concave in $k,h$ individually, plus CRS
  \item \emph{Inada Conditions}
    \begin{align*}
      F(0,h) &= 0 \\
      \lim_{k\ra 0} F_1(k,h) &= +\infty
      \quad\iff \quad\text{As $k\ra 0$, it becomes super productive to get some} \\
      \lim_{k\ra \infty} F_1(k,h) &= 0
      \;\qquad\iff\quad\text{As $k\ra \infty$, diminishing marginal returns}
    \end{align*}
\end{itemize}
The model has the following tradeoffs
\begin{enumerate}
  \item Consumption today vs. tomorrow
  \item Consumption vs leisure today
\end{enumerate}
This leads to social planner problem.
\begin{align*}
  \max_{\{c_t\},\{k_t\},\{h_t\}}
    &\sumtinfz \beta^t u(c_t) \\
    \text{s.t.} &\quad
    c_t + k_{t+1}
    = F(k_t,h_t) + (1-\delta) k_t\\
    &\quad c_t, k_t \geq 0 \\
    &\quad k_0 = \bar{k}_0
\end{align*}
We can eliminate choice of $\{c_t\}$ and turn this into equivalent
problem
\begin{align*}
  \max_{\{k_t\}}
    &\sumtinfz \beta^t
      u\left(F(k_t,h_t) + (1-\delta) k_t - k_{t+1}\right) \\
    \text{s.t.} &\quad
    F(k_t,h_t) + (1-\delta) k_t - k_{t+1} \geq 0 \\
    & \quad k_t \geq 0 \\
    &\quad k_0 = \bar{k}_0
\end{align*}
Then the pareto efficient allocation is completely characterized by
\begin{align}
  \frac{u'(c_{t-1})}{\beta u'(c_{t})}
  &= F_1(k_t,1) + (1-\delta) \notag \\
  c_t &= F(k_t,1) + (1-\delta)k_t - k_{t+1}\notag\\
  \limt \beta^{t-1}u'(c_{t-1}) k_t &= 0 \notag\\
  k_0 &= \bar{k}_0\notag
\end{align}
The first equation says MRS = MRT. On the LHS, it's clear that the
quantity represents the marginal rate of substitution. On the RHS, it's
less clear, but follows from the fact that decreasing consumption by one
unit to but capital increases tomorrow's consumption by $F_1(k_t,1)$ and
allows the individual to decrease investment by $(1-\delta)$ next period
since we carried extra capital forward.

If you start off steady state, you can examine evolution of $k_t$ and
$c_t$ by using
\begin{align*}
  \frac{u'(c_t)}{u'(c_{t+1})} &= f'(k_{t+1}) + (1-\delta) \\
  c_t &= f(k_t) + (1-\delta) k_t - k_{t+1}
\end{align*}
Given $c_t$ and $k_t$, you can solve the second equation for $k_{t+1}$
and then use that to solve the first equation for $c_{t+1}$.

\subsection{Solow Model}

Fixed output-share of investment each period, $s$. Capital evolves
according to
\begin{align*}
  k_{t+1} &= sf(k_t) + (1-\delta)k_t \\
  k_0 &= \hat{k}_0
\end{align*}
By investment equation, steady state capital must satisfy
\begin{align*}
  k^* &= sf(k^*) + (1-\delta)k^*\\
  \Leftrightarrow\quad
  \frac{\delta}{s} k^* &= f(k^*)
\end{align*}


\clearpage
\subsection{Leisure}

We set up the model
\begin{enumerate}
  \item \emph{Preferences}: Both consumption and liesure enter the
    utility function $u(c,1-h)$ which satisfy the following conditions
    that will help us get an interior solution
    \begin{align*}
      \lim_{h\ra 0} \; u_2(c,1-h) &= 0 \\
      \lim_{h\ra 1} \; u_2(c,1-h) &= +\infty
    \end{align*}
    The first says that the consumer gets less and less utility from a
    life of increasing liesure, while the second says that a bit of
    liesure is infinitely valuable if the household is working all of
    the time.

  \item \emph{Technology}: Output $y_t$ is produced by a CRS production
    function that takes capital $k_t$ and labor $h_t$ as input:
    \begin{align*}
      y_t &= F(k_t,h_t)
    \end{align*}
    Output can be used for consumption $c_t$ or investment $i_t$:
    \begin{align*}
      y_t &= c_i + i_t
    \end{align*}
    Investment is how the household accumulate capital, though
    depreciation eats away at the stock of capital at rate
    $\delta\in[0,1]$:
    \begin{align*}
      k_{t+1} &= (1-\delta)k_t + i_t
    \end{align*}

  \item \emph{Endowments}: Each household has one unit of time to be
    split among labor and leisure each period, and initial capital level
    $k_0=\bar{k}_0$ is given.
\end{enumerate}
This leads to social planner problem.
\begin{align*}
  \max_{\{c_t\},\{k_t\},\{h_t\}}
    &\sumtinfz \beta^t u(c_t,1-h_t) \\
    \text{s.t.} &\quad
    c_t + k_{t+1}
    = F(k_t,h_t) + (1-\delta) k_t\\
    &\quad c_t, k_t \geq 0 \\
    &\quad h_t \in[0,1] \\
    &\quad k_0 = \bar{k}_0
\end{align*}
We can eliminate choice of $\{c_t\}$ and turn this into equivalent
problem
\begin{align*}
  \max_{\{k_t\},\{h_t\}}
    &\sumtinfz \beta^t
      u\left(F(k_t,h_t) + (1-\delta) k_t - k_{t+1}, \; 1-h_t\right) \\
    \text{s.t.} &\quad
    F(k_t,h_t) + (1-\delta) k_t - k_{t+1} \geq 0 \\
    & \quad k_t \geq 0 \\
    &\quad h_t \in[0,1] \\
    &\quad k_0 = \bar{k}_0
\end{align*}
Then the pareto efficient allocation is now completely characterized by
\begin{align}
  \frac{u_1(c_{t-1},1-h_{t-1})}{\beta u_1(c_{t},1-h_t)}
    &= F_1(k_t,h_t) + (1-\delta) \notag \\
  \frac{u_2(c_t,1-h_t)}{u_1(c_t,1-h_t)}
    &= F_2(k_t,h_t) \label{hchoice}\\
  c_t &= F(k_t,h_t) + (1-\delta)k_t - k_{t+1}\notag\\
  \limt \beta^{t-1}u'(c_{t-1},1-h_{t-1}) k_t &= 0 \notag\\
  k_0 &= \bar{k}_0\notag
\end{align}
This looks like before, except we now have Equation~\ref{hchoice}, which
says that the marginal rate of substitution between consumption and
leisure should equal the marginal product of supplying an extra unit of
labor.

Now, we look for a steady state, which gives us
\begin{align}
  F_1(k^*,h^*) + (1-\delta)
    &= \frac{1}{\beta} \notag \\
  \frac{u_2(F(k^*,h^*) - \delta k^*,\; 1-h^*)}{%
    u_1(F(k^*,h^*) - \delta k^*,\;1-h^*)}
    &= F_2(k^*,h^*) \notag
\end{align}
which is two equations in two unknowns. We can use the CRS property of
$F$, which implies that $F_1$ is homogeneous of degree zero, to simplify
the consumption Euler Equation
\begin{align*}
  \frac{1}{\beta}
    &= F_1\left(\frac{k^*}{h^*},1\right) + (1-\delta)
\end{align*}
This pins down the capital labor ratio $\frac{k^*}{h^*}$ at a unique
value (by the Inada conditions). Next, notice that by the CRS property
of $F$, we can simplify terms in the utility equations
\begin{align*}
  F(k^*,h^*) - \delta k^*
  &= h^*\left[F\left(\frac{k^*}{h^*},1\right) - \delta \frac{k^*}{h^*}\right] \\
  &= B_2h^*
  \qquad \text{where} \quad
  B_2:= F\left(\frac{k^*}{h^*},1\right) - \delta \frac{k^*}{h^*}
\end{align*}
Again, since $F_2$ is homogenous of degree zero, so define constant
\begin{align*}
  B_1
  :=
  F_2(k^*,h^*)
  =
  F_2\left(\frac{k^*}{h^*},1\right)
\end{align*}
So we can rewrite the second FOC as
\begin{align*}
  \frac{u_2(B_2h^*,1-h^*)}{%
    u_1(B_2h^*,1-h^*)}
    &= B_1 \notag
\end{align*}
There exists at least one solution, but in general, there is no unique
solution unless we assume both $c_t$ and $1-h_t$ are normal goods.


\clearpage
\section{Endogenous Growth}

We modify the time-homogeneous production function. There's three ways
you can do it
\begin{enumerate}
  \item $y_t = F(A_t k_t,h_t)$: Capital augmenting
  \item $y_t = F(k_t,A_t h_t)$: Labor augmenting
  \item $y_t = A_t F(k_t,h_t)$: Neutral or factor neutral technological
    change because it affects the production function generally---not
    just one of the factors. It's as if the production function displays
    constant returns to scale, so you augment both factors and the
    constant $A_t$ gets pulled out in front.
\end{enumerate}
Note that for something like Cobb-Douglas, there's no difference.

\subsection{Baseline Growth Model}

We set up the model
\begin{enumerate}
  \item \emph{Preferences}: $\sumtinfz \beta^t u(c_t)$
  \item \emph{Technology}: Assume $\{A_t\}$ given, so deterministic
    technology growth at a constant rate (for now).\footnote{%
      Add randomness and you get RBC models.}
    We go for labor-augmenting technological change.
    \begin{align*}
      y_t &= F(k_t,A_th_t) \\
      y_t &= c_i + i_t \\
      k_{t+1} &= (1-\delta)k_t + i_t
    \end{align*}
  \item \emph{Endowments}: One unit of time for labor each period, and
    $k_0=\bar{k}_0$.
\end{enumerate}
This leads to social planner problem.
\begin{align*}
  \max_{\{c_t\},\{k_t\}}
    &\sumtinfz \beta^t u(c_t) \\
    \text{s.t.} &\quad
    c_t + k_{t+1}
    = F(k_t,A_t) + (1-\delta) k_t\\
    &\quad c_t \geq 0 \\
    &\quad k_0 = \bar{k}_0
\end{align*}
Then the pareto efficient allocation is completely characterized by
\begin{align}
  \frac{u'(c_{t-1}}{\beta u'(c_{t})}
  &= F_1(k_t,A_t) + (1-\delta) \label{eulerA}\\
  c_t &= F(k_t,A_t) + (1-\delta)k_t - k_{t+1}\notag\\
  \limt \beta^{t-1}u'(c_{t-1}) k_t &= 0 \notag\\
  k_0 &= \bar{k}_0\notag
\end{align}
We will now attempt to find a \emph{balanced growth path}, which is the
natural generalization of steady state that considers asymptotic
behavior. A balanced growth path is a solution in which all of the
variables grow at \emph{constant} rates, though \emph{not necessarily}
the same rate.\footnote{%
  For our simple model, it will be the case that things grow at the same
  rate. But the concept of a ``balanced growth path'' is more general.
}

So we know that $A_t$ grows at some constant rate $g$. We want $k_t$ and
$c_t$ to grow at a constant rates $g_k$ and $g_c$ as well, though they
could differ from $g$. So we want
\begin{align*}
  c_t &= F(k_t,A_t) + (1-\delta) k_t - k_{t+1} \\
  (1+g_c) c_0 &= F((1+g_k)k_t,(1+g)^tA_0) + (1-\delta) k_t - k_{t+1} \\
\end{align*}
For $c_t$ to grow at a constant rate, we need $g_k=g$. Then we can pull
out $(1+g)$ from inside the production function and everything on the
RHS grows at $g$. Then $c_t$ grows at rate $g$ too.

Next, notice that since $k_t$ and $A_t$ are growing at $g$, the ratio
between the goods is not changing. Hence $F_1(k_t,A_t)$ is constant
since the derivative is homogeneous of degree zero (which follows from
$F$ being HoD1).

\subsection{Baseline with Change of Variables}

Since everything is growing at rate $g$, define
$\tilde{k}_t = \frac{k_t}{(1+g)^t}$ or equivalently
$k_t=(1+g)^t\cdot \tilde{k}_t$. Moreover, normalize $A_0=1$ and
substitute $A_t=(1+g)^t$. Substitute. Then the Social Planner's problem
becomes
\begin{align*}
  \max_{\{k_t\}}
    &\sumtinfz \beta^t u\left[F\left((1+g)^t\tilde{k}_t,(1+g)^{t}\right)
      + (1-\delta) (1+g)^t\tilde{k}_t - (1+g)^t\tilde{k}_{t+1}\right] \\
    \text{s.t.} &\quad
    F((1+g)^t\tilde{k}_t,(1+g)^t) + (1-\delta) (1+g)^t\tilde{k}_t
      - (1+g)^t\tilde{k}_{t+1}\geq 0\\
    &\quad k_0 = \bar{k}_0
\end{align*}
We can pull out $(1+g)^t$ in both the objective function and the
constraint to simplify to
\begin{align*}
  \max_{\{k_t\}}
    &\sumtinfz \beta^t u\left[F\left(\tilde{k}_t,\right)
      + (1-\delta) (1+g)^t\tilde{k}_t - (1+g)^t\tilde{k}_{t+1}\right] \\
    \text{s.t.} &\quad
    F((1+g)^t\tilde{k}_t,(1+g)^t) + (1-\delta) (1+g)^t\tilde{k}_t
      - (1+g)^t\tilde{k}_{t+1}\geq 0\\
    &\quad k_0 = \bar{k}_0
\end{align*}

Then the pareto efficient allocation is completely characterized by
\begin{align}
  \frac{u'(c_{t-1}}{\beta u'(c_{t})}
  &= F_1(k_t,A_t) + (1-\delta) \\
  c_t &= F(k_t,A_t) + (1-\delta)k_t - k_{t+1}\notag\\
  \limt \beta^{t-1}u'(c_{t-1}) k_t &= 0 \notag\\
  k_0 &= \bar{k}_0\notag
\end{align}
We will now attempt to find a \emph{balanced growth path}, which is the
natural generalization of steady state that considers asymptotic
behavior. A balanced growth path is a solution in which all of the
variables grow at \emph{constant} rates, though \emph{not necessarily}
the same rate.\footnote{%
  For our simple model, it will be the case that things grow at the same
  rate. But the concept of a ``balanced growth path'' is more general.
}

So we know that $A_t$ grows at some constant rate $g$. We want $k_t$ and
$c_t$ to grow at a constant rates $g_k$ and $g_c$ as well, though they
could differ from $g$. So we want
\begin{align*}
  c_t &= F(k_t,A_t) + (1-\delta) k_t - k_{t+1} \\
  (1+g_c) c_0 &= F((1+g_k)k_t,(1+g)^tA_0) + (1-\delta) k_t - k_{t+1} \\
\end{align*}
For $c_t$ to grow at a constant rate, we need $g_k=g$. Then we can pull
out $(1+g)$ from inside the production function and everything on the
RHS grows at $g$. Then $c_t$ grows at rate $g$ too.

Next, notice that since $k_t$ and $A_t$ are growing at $g$, the ratio
between the goods is not changing. Hence $F_1(k_t,A_t)$ is constant
since the derivative is homogeneous of degree zero (which follows from
$F$ being HoD1).


\clearpage
\section{Romer Model}

\subsection{Preferences, Technology, Endowments}

Specialization drives productivity. Primitives
\begin{enumerate}
  \item \emph{Preferences}: $\sumtinfz \beta^t u(c_t)$ with
    $u(c)=\frac{c^{1-\sigma}-1}{1-\sigma}$ for $\sigma>0$.
  \item \emph{Technology}:
    Output $y_t$ is produced by spreading out aggregate capital $k_t$
    among a mass of $N_t$ production processes that all use capital
    differently:
    \begin{align*}
      y_t &=
      \left(
      \int_{0}^{N_t} k_t(i)^\alpha \;di \right)
      h_t^{1-\alpha} \\
      k_t &= \int_0^{N_t} k_t(i) \; di
    \end{align*}
    where $N_t$ is an index of specialization that will grow over time
    as additional (specialized) production processes come into
    existence through R\&D. But given $N_t$ fixed, note that the
    production function is CRS.

    Output can be used for consumption, investment in aggregate
    capital, or investment in finding more production processes
    (i.e.\ R\&D).
    \begin{align*}
      y_t &= c_t + i_t^k + i_t^N \\
      k_{t+1} &= (1-\delta)k_t + i_t^k \\
      N_{t+1} &= N_t + i_t^N
    \end{align*}

  \item \emph{Endowments}: Each household has one unit of time, while
    capital and the number of production processes start at
    $k_0=\bar{k}_0$ and $N_0=\bar{N}_0$.
\end{enumerate}
Note that there are no irreversibilities, so the stock of knowledge can
be turned back into consumption $c$, but that won't be a problem since
we'll be finding a BGP.


\subsection{Social Planner's Problem}

First, we know that aggregate capital will be spread out equally among
the production processes, so that $k_t(i) = \frac{k_t}{N_t}$. Therefore
\begin{align*}
  \int_0^{N_t} k_t(i)^\alpha \; di
  &=
  \int_0^{N_t} \left(\frac{k_t}{N_t}\right)^\alpha \; di
  =
  \left(\frac{k_t}{N_t}\right)^\alpha N_t
  = k_t^\alpha N_t^{1-\alpha}
\end{align*}
Coupled with the fact that $h_t=1$, this allows us to write the
production function more compactly as
\begin{align*}
  F(k_t,h_t)
  &= \left( \int_0^{N_t} k_t(i)^\alpha \; di\right)
    h_t^{1-\alpha}
  = k_t^{\alpha} N_t^{1-\alpha}
\end{align*}
This allows us to write the social planner's problem
\begin{align*}
  \max_{\{k_t\},\{N_t\}} \; & \sumtinfz \beta^t u(c_t) \\
  \text{s.t.} \; &
    c_t =
    k_t^\alpha N_t^{1-\alpha}
    - [k_{t+1}-(1-\delta)k_t] + [N_{t+1}-N_t] \\
    & k_0 = \bar{k}_0 \\
    & N_0 = \bar{N}_0 \\
    & c_t, k_t,N_t\geq 0
\end{align*}
Hence, first order conditions
\begin{align*}
  \frac{u'(c_{t-1})}{\beta u'(c_t)}
  &=
  \alpha \left(\frac{k_t}{N_t}\right)^{\alpha-1} + (1-\delta) \\
  \frac{u'(c_{t-1})}{\beta u'(c_t)}
  &=
  (1-\alpha) \left(  \frac{k_t}{N_t} \right)^{\alpha} + 1
\end{align*}
Equating
\begin{align*}
  \alpha \left(\frac{k_t}{N_t}\right)^{\alpha-1} + (1-\delta)
  &=
  (1-\alpha) \left(  \frac{k_t}{N_t} \right)^{\alpha} + 1
\end{align*}
This needs to be true at each $t$, hence $\frac{k_t}{N_t}$ is a
constant i.e.
\begin{align*}
  \frac{1}{\tilde{A}} = \frac{k_t}{N_t}
  \quad\implies\quad
  N_t = \tilde{A}k_t
\end{align*}
Moreover, there's a unique value for the ratio since the RHS is
increasing in $\frac{k_t}{N_t}$, while the LHS is decreasing.
\\
\\
And if we plug this back into the Euler Equation
\begin{align*}
  \frac{u'(c_{t-1})}{\beta u'(c_t)}
  &=
  \alpha \left(\frac{1}{\tilde{A}}\right)^{\alpha-1} + (1-\delta)
\end{align*}
which is exactly identical to the $Ak$ model where
$A=\alpha \left(\frac{1}{\tilde{A}}\right)^{\alpha-1}$.


\clearpage
\section{Competitive Equilibrium}

\subsection{Arrow-Debreu Competetive Equilibrium (ADCE): Baseline
Formulation}

In this subsection, we formulate an
\emph{Arrow-Debreu Competetive Equilibrium} (ADCE) for a simple
production-based economy involving households that consume output,
supply labor, and own capital along with firms that produce output by
buying labor and renting capital from the households (paying a rental
rate).

Note that there will be no real link between time periods for capital.
You can change capital a lot from $k_t$ to $k_{t+1}$. You could link up
time periods by adding adjustment costs, a feature we ignore for now.

In this context, and ADCE is a list of sequences of
$\{c_t^*\}$  Consumption,
$\{k_t^*\}$  Capital,
$\{h_t^*\}$  Hours,
$\{p_t^*\}$  Prices,
$\{w_t^*\}$  Wages, and
$\{r_t^*\}$  rental rate of capital,
that satisfy the following conditions:
\begin{enumerate}
  \item \emph{Household Optimization}: Taking prices,
    wages, and the rate of return on capital
    ($\{p^*_t\}$, $\{w^*_t\}$, $\{r^*_t\}$) as given, the
    sequences for consumption, capital, and hours of labor
    ($\{c_t^*\}$, $\{k_t^*\}$, and $\{h_t^*\}$)
    solve the utility maximization problem
    \begin{align}
      (\{c_t^*\}, \{k_t^*\}, \{h_t^*\})
      =
      \argmax_{\{c_t\}, \{k_t\}, \{h_t\}}
        \; &\sumtinfz \beta^t u(c_t)
        \label{defn:adce-hh-objfcn}\\
      \text{s.t.} \; &
        \sumtinfz p^*_t(c_t + k_{t+1}-(1-\delta)k_t) \leq
        \sumtinfz (r^*_t k_t + w^*_t h_t)
        \label{defn:adce-hh-budget}\\
      c_t &\geq 0 \notag\\
      h_t &\in [0,1] \notag\\
      k_0 &= \bar{k}_0 \; \text{given} \notag
    \end{align}
    In other words the ``star'' sequences correspond exactly to what the
    household optimizing household would choose to do, taking things
    outside of their control as given.

    A few remarks:
    \begin{enumerate}
      \item Inequality~\ref{defn:adce-hh-budget} acts as a budget
        constraint that stipulates total lifetime spending on
        consumption and capital must be less than or equal to total
        lifetime capital-rental income and labor income. Note that it is
        a single budget constraint for time zero, not a sequence of
        budget constraints each period.

      \item Outside of the budget constraint, the remaning constraints
        enforce nonnegative consumption, supply of hours limited by
        hours in the day (normalized to one), and capital given.

      \item There is no non-negativity constraint on capital since the
        firm's production function will be defined for positive capital
        values only. Hence they will never demand negative capital, so
        we can leave it out of the constraints here.

      \item Labor $h_t$ is not in the utility function, so it is pretty
        obvious that the household would choose to supply $h_t=1$ units
        of labor for all $t$. That earns housholds the most consumption
        (which they value) at the cost of leisure (which we assume they
        don't care about).

        So it might seem like we could have left the $\{h_t\}$ out of
        the definition of equilibrium altogether. However, that is not
        the case. Even though households will choose to supply one unit
        of labor, firms don't know that. We must keep the labor decision
        in the model because that will allow us to pin down the wage
        rate.
    \end{enumerate}

  \item \emph{Firm Optimization}: Taking
    prices, wages, and the rate of return on capital
    ($\{p^*_t\}$, $\{w^*_t\}$, $\{r^*_t\}$) as given,
    the sequences for capital to rent and labor hours to buy
    ($\{k_t^*\}$ and $\{h_t^*\}$)
    solve the profit maximization problem
    \begin{align}
      (\{k_t^*\}, \{h_t^*\})
      =
      \argmax_{\{k_t\}, \{h_t\}}
        \; &\sumtinfz \beta^t (p^*_t F(k_t,h_t) - w^*_t h_t - r^*_t k_t)
        \label{defn:adce-firm-objfcn}\\
      \text{s.t.} \;
      &k_t \geq 0 \notag\\
      &h_t \geq 0 \notag
    \end{align}
    A few remarks:
    \begin{enumerate}
      \item Firms don't worry about the fact that we must have
        $h_t\leq 1$. To them, they can buy as many labor hours as they
        want at rate $w^*_t$. In reality, the household optimization
        side of the equilibrium will enforce $h_t\leq 1$.

      \item Though there does not appear to be any discounting in the
        firm's objective function, it is implicit in the prices. As
        we'll see below, $p_t\ra 0$, which will take care of the
        discounting.
    \end{enumerate}

  \item \emph{Market Clearing}: Production is used up as either
    consumption or investment
    \begin{align*}
      F(k^*_t,h_t^*) = c_t^* + k_{t+1}^* - (1-\delta) k_t^*
    \end{align*}
\end{enumerate}
We are now in a position to characterize equilibrium. Rather than appeal
to the welfare theorem to say that the outcome is Pareto optimal (hence
it looks exactly like the outcome of the planning problem), we will look
at the first order conditions to the different optimization problems we
have. It turns out that they \emph{will} match the planning problem,
which will offer a kind of proof of the first welfare theorem.
\begin{enumerate}
  \item \emph{Household FOCs}: Again there is a single budget
    constraint at time zero, not one for each period. So
    Inequality~\ref{defn:adce-hh-budget} has only one associated
    multiplier, $\mu$. Therefore, differentiating with respect to
    consumption and capital, we get FOCs for time $t$ as:
    \begin{align}
      \beta^t u'(c_t) &= \mu p_t^* \label{hhfoc1}\\
      \mu[r_t^* + p_t^* (1-\delta)]
      &= \mu p_{t-1}^* \label{hhfoc2}
    \end{align}
    while clearly $h_t=1$.

    These are necessary conditions for optimal $c_t$ that follow from
    the fact that $u'(0)=\infty$, so we have an interior solution,
    implying the KKT conditions are necessary.

  \item \emph{Firm FOCs}: We have
    \begin{align}
      p^*_t F_1(k_t,h_t) &= r_t^* \label{firmfoc1}\\
      p^*_t F_2(k_t,h_t) &= w_t^* \notag
    \end{align}
    Notice that capital appears nowhere in those equations.
    We instead have Equation~\ref{firmfoc1} relating \emph{prices},
    rather than quantities of capital. This is called a
    \emph{no arbitrage condition}. More on that later.

  \item \emph{Transversality Condition}:
    Think about a finite-time version of this problem. The household has
    to choose consumption and investment for the last period, denoted
    time $T$. At $T$, the household can either consume output in the
    form of $c_T$ or invest to keep a positive $k_{T+1}$. But since $T$
    is the last period, investing output to keep $k_{T+1}>0$ looks
    pretty dumb when you could instead use that output for higher
    consumption $c_T$.  So it must be that \emph{either} $k_{T+1}=0$
    (i.e.\ you consume all output in the last period and invest nothing)
    or the value of consumption---the marginal utility
    $u'(c_T)$---equals zero so that more consumption $c_T$ isn't
    actually worth shit. In math, you'd need
    \begin{align*}
      \beta^{T} u'(c_T) k_{T+1} = 0
    \end{align*}
    The transversality condition is the infinite horizon analog to this
    problem, written
    \begin{align*}
      \limt \beta^t u'(c_t) k_{t+1}=0
    \end{align*}

  \item \emph{Pareto Optimality}: We want to show the competetive
    equilibrium matches the solution to the planning problem. Since
    we're considering the equilibrium sequences, everything will have
    stars ${}^*$.

    Start by dividing Equation~\ref{hhfoc1} at time $t-1$ by its time
    $t$ analog to get
    \begin{align}
      \frac{u'(c_{t-1}^*)}{\beta u'(c_{t}^*)} = \frac{p_{t-1}^*}{p_t^*}
      \label{pratio1}
    \end{align}
    Take household FOC Equation~\ref{hhfoc2}, cancel $\mu$'s, and divide
    by $p_t^*$:
    \begin{align*}
      \frac{r_t^*}{p_t^*} + (1-\delta)
      &= \frac{p_{t-1}^*}{p_t^*}
    \end{align*}
    Into the above equation, substitute Equation~\ref{firmfoc1} in for
    $r_t^*$:
    \begin{align}
      F_1(k_t,h_t) + (1-\delta)
      &= \frac{p_{t-1}^*}{p_t^*}
      \label{pratio2}
    \end{align}
    How equate Equations~\ref{pratio1} and \ref{pratio2}:
    \begin{align*}
      \frac{u'(c_{t-1}^*)}{\beta u'(c_{t}^*)}
      = F_1(k_t,h_t) + (1-\delta)
    \end{align*}
    So we have this, together with market clearing, $h_t=1$, the
    transversality condition, and an initial capital level:
    \begin{align*}
      \frac{u'(c_{t-1}^*)}{\beta u'(c_{t}^*)}
      &= F_1(k_t,h_t) + (1-\delta) \\
      F(k_t,h_t) &= c_t + k_{t+1} - (1-\delta) k_t \\
      h_t &= 1\\
      \limt \beta^t u'(c_t) k_{t+1}&=0\\
      k_0 &= \bar{k}_0 \; \text{given}
    \end{align*}
    These are exactly the conditions that characterized the Pareto
    efficient allocations from the planning problem.

  \item \emph{Prices}: Normalize $p_0^*=1$, which will also remove
    indeterminacy problems. We get the wage and capital rental rates
    from the firm's first order conditions:
    \begin{align*}
      r_t^* &= p_t^* F_1(k^*_t,h^*_t) \\
      w_t^* &= p_t^* F_2(k^*_t,h^*_t)
    \end{align*}
    We get tomorrow's price by Solving Equation~\ref{pratio1} for
    $p_{t+1}$:
    \begin{align}
       p_t^*
       = p_{t-1}^*\beta \frac{u'(c_{t}^*)}{u'(c_{t-1}^*)}
    \end{align}
\end{enumerate}
Lastly, a \emph{steady state ADCE} is a value $k^*$ and an ADCE for the
economy with $\bar{k}_0=k^*$ and $\{k_t^*\}$ satisfying $k^*_t=k^*$ for
all $t$. Consumption must be a constant $c^*$ because solving the market
clearing equation for consumption expresses consumption as a function of
constants:
\begin{align*}
  c^* := c_t^* = F(k^*,1) - \delta k^*
  \qquad \forall t
\end{align*}
Therefore, by Equation~\ref{pratio1}
\begin{align*}
  \frac{u'(c^*)}{\beta u'(c^*)} &= \frac{p_{t-1}^*}{p_t^*} \\
  \implies \quad p_t &= \beta p_{t-1}^* = \beta^t p_0^*
\end{align*}
Hence, prices are \emph{not} constant---they are falling. The same logic
can be used to show that wages and rental rates are falling
monotonically in steady state:
\begin{align*}
  r_t^* &= p_t^* F_1(k^*,1) = \beta^t p_0^* F_1(k^*,1) = \beta^t r_0^*\\
  w_t^* &= p_t^* F_2(k^*,1) = \beta^t p_0^* F_2(k^*,1) = \beta^t w_0^*
\end{align*}
Hence we have justified the earlier claim that ``prices embed
discounting'' within the firm's optimization problem.

\subsection{Arrow-Debreu Competetive Equilibrium (ADCE): Taxation}

In this subsection, we suppose that there is a permanent proportional
tax on capital income at rate $\tau_k$, where the revenues fund a
lump-sum transfer to households, subject to a period-by-period budget
constraint.

In this context, and ADCE is a list of sequences of
$\{c_t^*\}$  Consumption,
$\{k_t^*\}$  Capital,
$\{h_t^*\}$  Hours,
$\{p_t^*\}$  Prices,
$\{w_t^*\}$  Wages,
$\{r_t^*\}$  Rate of return on capital, and
$\{T_t^*\}$ Transfers
that satisfy the following conditions:

\begin{enumerate}
  \item \emph{Household Optimization}: The household---taking prices,
    wages, the rate of return on capital, and transfers
    ($\{p^*_t\}$, $\{w^*_t\}$, $\{r^*_t\}$, $\{T_t^*\}$) as
    given---chooses sequences for consumption, capital, and hours of
    labor
    ($\{c_t\}$, $\{k_t\}$, and $\{h_t\}$)
    to maximize the present discounted value of utility, which happen to
    exactly match equilibrium sequences
    ($\{c_t^*\}$, $\{k_t^*\}$, and $\{h_t^*\}$)
    \begin{align}
      (\{c_t^*\}, \{k_t^*\}, \{h_t^*\})
      =
      \argmax_{\{c_t\}, \{k_t\}, \{h_t\}}
        \; &\sumtinfz \beta^t u(c_t)
        \label{defn:adce-hh-objfcn-taxes}\\\notag\\
      \text{s.t.} \;
        \sumtinfz p^*_t(c_t + k_{t+1}-(1-\delta)k_t) &\leq
        \sumtinfz ((1-\tau_k)r^*_t k_t + w^*_t h_t + T^*_t)
        \label{defn:adce-hh-budget-taxes}\\
      c_t &\geq 0 \notag\\
      h_t &\in [0,1] \notag\\
      k_0 &= \bar{k}_0 \; \text{given} \notag
    \end{align}

  \item \emph{Firm Optimization}: The firm---taking
    prices, wages, and the rate of return on capital
    ($\{p^*_t\}$, $\{w^*_t\}$, $\{r^*_t\}$) as given---chooses
    the amount of capital to rent and labor hours to buy
    ($\{k_t\}$, $\{h_t\}$)
    that maximizes the present discount value of profits, which happens
    to exactly match equilibrium sequences
    ($\{k_t^*\}$ and $\{h_t^*\}$)
    \begin{align}
      (\{k_t^*\}, \{h_t^*\})
      =
      \argmax_{\{k_t\}, \{h_t\}}
        \; &\sumtinfz \beta^t (p^*_t F(k_t,h_t) - w^*_t h_t - r^*_t k_t)
        \label{defn:adce-firm-objfcn-taxes}\\
      \text{s.t.} \;
      &k_t \geq 0 \notag\\
      &h_t \geq 0 \notag
    \end{align}

  \item \emph{Government}: For each $t$:
    \begin{align}
      \tau_k r^*_t k^*_t = T^*_t
      \label{taxes-govt}
    \end{align}

  \item \emph{Market Clearing}: For each $t$:
    \begin{align}
      F(k^*_t,h_t^*) = c_t^* + k^*_{t+1} - (1-\delta)k_t^*
      \label{taxes-clear}
    \end{align}
\end{enumerate}
As before, we can characterize the necessary conditions of the ADCE:

\begin{enumerate}
  \item \emph{Household FOCs}: Again there is a single budget
    constraint at time zero, not one for each period. So
    Inequality~\ref{defn:adce-hh-budget-taxes} has only one associated
    multiplier, $\mu$. Therefore, differentiating with respect to
    consumption and capital, we get FOCs for time $t$ as:
    \begin{align}
      \beta^t u'(c_t) &= \mu p_t^* \label{taxes-hhfoc1}\\
      \mu[(1-\tau_k)r_t^* + p_t^* (1-\delta)]
      &= \mu p_{t-1}^* \label{taxes-hhfoc2}
    \end{align}
    while clearly $h_t=1$.

  \item \emph{Firm FOCs}: These will be the same as in the no taxation
    case:
    \begin{align}
      p^*_t F_1(k_t,h_t) &= r_t^* \label{taxes-firmfoc1}\\
      p^*_t F_2(k_t,h_t) &= w_t^* \notag
    \end{align}
    Notice that capital appears nowhere in those equations.
    We instead have Equation~\ref{taxes-firmfoc1} relating
    \emph{prices}, rather than quantities of capital. This is called a
    \emph{no arbitrage condition}.

  \item \emph{Transversality Condition}:
    \begin{align*}
      \limt \beta^t u'(c_t) k_{t+1}=0
    \end{align*}

  \item \emph{Euler Equation}: As before, we can derive an Euler
    equation
    \begin{align}
      \frac{u'(c_{t-1}^*)}{\beta u'(c_{t}^*)}
      = (1-\tau_k) r_t^* + (1-\delta)
    \end{align}

  \item \emph{Steady State ADCE}: The steady state level of capital
    satisfies
    \begin{align*}
      (1-\tau_k) f'(k^*) &= \frac{1}{\beta} - (1-\delta)
    \end{align*}
    which is decreasing in $\tau_k$.

  \item \emph{Government Waste}: Suppose that the government did not
    lump-sum capital income taxes back to households. Suppose instead
    that the government just buries the output. Then there will be
    changes to equilibrium.

    Specifically, if the government wastes taxes revenue, replace
    sequence $\{T^*_t\}$ with $\{G^*_t\}$, and remove $T^*_t$ terms from
    the household budget constraint, since they never see the tax
    revenue.  Equation~\ref{defn:adce-hh-budget-taxes} becomes
    \begin{align*}
        \sumtinfz p^*_t(c_t + k_{t+1}-(1-\delta)k_t) &\leq
        \sumtinfz ((1-\tau_k)r^*_t k_t + w^*_t h_t)
    \end{align*}
    We also modify the Government Budget Equation~\ref{taxes-govt} and
    Market Clearing Equation~\ref{taxes-clear}
    \begin{align*}
      G^*_t &=  \tau_k r^*_t k^*_t \\
      F(k^*_t,h_t^*) &= c_t^* + k^*_{t+1} - (1-\delta)k_t^* + G^*_t
    \end{align*}
    Of the key equations summarizing the ADCE allocation, only the
    market clearing constraint (which pins down the level of
    consumption) is affected. We see that the government waste reduces
    consumption.
    Notably, however, the consumption Euler equation does not change. In
    other words, households don't change how they allocate cash money
    across time---they just need to reduce their consumption by the
    amount of waste in each period.
\end{enumerate}

\clearpage
\subsection{Rates of Return in an ADCE}

There are two ways for the consumer to move money through time in an
ACDE, with two corresponding rates of return:
\begin{enumerate}
  \item \emph{Real Rate of Return}, $R_t^*$: Give up one unit of
    consumption today, saving $p_t^*$, and buy $\frac{p_t^*}{p_{t+1^*}}$
    units of consumption tomorrow:
    \begin{align}
      1+R_t^* = \frac{p_t^*}{p_{t+1}^*} = \frac{1}{\beta}
      \label{noarb1}
    \end{align}
    where the last equality followed from a steady relationship.

  \item \emph{Real Rate of Return on Capital}: Give up one unit of
    consumption today, invest in capital, and use all proceeds from the
    capital to consume at time $t+1$
    \begin{align}
      \frac{r_{t+1}^* + (1-\delta)p_{t+1}^*}{p_{t+1}^*}
      \label{noarb2}
    \end{align}
    In words, you give up one unit of consumption today and instead buy
    one unit of capital for use tomorrow.
    First, that extra unit of capital gets you extra rental income
    $r_{t+1}^*$ tomorrow.
    Second, you also get the liquidation value of that unit of capital
    (net of depreciation) at tomorrow's price $(1-\delta)p_{t+1}^*$
    The sum of these two things $r_{t+1}^* + (1-\delta)p_{t+1}^*$ is the
    numerator.

    Lastly, you gave up one unit of consumption today at price $p_t^*$,
    but consumption will be priced differently tomorrow.  So we should
    divide this extra consumption tomorrow by $p_{t+1}^*$ to make the
    periods comparable. That's the denominator.
\end{enumerate}
So we've established two ways to move money through time: via simple
reallocation of consumption and via investment. By any reasonable notion
of equilibrium, we would expect no arbitrage, i.e.\ for
Equation~\ref{noarb1} (in whichever of the three representations we
choose) to equal Equation~\ref{noarb2}:
\begin{align*}
  \frac{p_t^*}{p_{t+1}^*}
  &=
  \frac{r_{t+1}^* + (1-\delta)p_{t+1}^*}{p_{t+1}^*} \\
  \Leftrightarrow\quad
  \frac{p_t^*}{p_{t+1}^*}
  &=
  \frac{r_{t+1}^*}{p_{t+1}^*} + (1-\delta)
\end{align*}
But recall that this equation popped out of the ADCE. It was also the
reason why the FOC for capital didn't have the \emph{quantity} of
capital involved, but rather this no arbitrage condition (implicitly).


\clearpage
\section{Static Model with Variety}

\subsection{Preferences, Technology, Endowments}

We define the usual trifecta of primitives for a macro problem:
\begin{enumerate}
  \item \emph{Preferences}:
    Given mass of goods $N$ and the preference parameter $\rho\in(0,1)$,
    define utility over consumption of a continuum of goods, captured by
    function $c:[0,N]\ra \R$, as
    \begin{align*}
      U(c) &=
      u\left(
      \left[
      \int_0^N c(i)^\rho \; di
      \right]^{1/\rho}
      \right)
    \end{align*}
    where $c(i)$ is consumption of the $i$th good and $u$ is some
    standard utility function and $u$ is some standard utility function
    defined for one good. In this way, big $U$ aggregates the utility of
    consuming $c(i)$ units of each $i$th good.  The above has limiting
    cases
    \begin{itemize}
      \item $\rho\ra 0$: Cobb-Douglas
      \item $\rho\ra 1$: Perfect substitutes
    \end{itemize}

  \item \emph{Technology}:
    There is a variety of goods to consume, each of which is produced by
    a different firm, so we need to define the production function for
    each firm:
    \begin{align*}
      y(i) = A h(i)
    \end{align*}
    where $y(i)$ is output of the $i$th good from the $i$th firm and
    $h(i)$ is labor used to produce it. All of the firms have identical
    technology in the sense that the same amount of labor produces the
    same output of output for any good. We ignore capital for now.

  \item \emph{Endowments}: Each household has one unit of labor to
    supply to the firms.
\end{enumerate}

\clearpage
\subsection{SP Problem, Fixed $N$}

We have Social Planner's problem
\begin{align*}
  \max_{c(i), h(i)}
    & \; u\left(
      \left[ \int_0^N c(i)^\rho \; di \right]^{1/\rho}
    \right)\\
  \text{s.t} & \;
  \intzN h(i) \; di = 1 \\
  &\; c(i) = Ah(i) \\
  &\; c(i),h(i) \geq 0
\end{align*}
Eliminate $c(i)$ from the problem by subbing in:
\begin{align*}
  \max_{h(i)}
    & \; u\left(
    \left[ \int_0^N (Ah(i))^\rho \; di \right]^{1/\rho}
    \right)\\
  \text{s.t} & \;
  \intzN h(i) \; di = 1 \\
  &\; h(i) \geq 0
\end{align*}
Differentiate with respect to $h(i)$ and this gives FOC:
\begin{align*}
  u'\left( \left[ \int_0^N (Ah(i))^\rho \; di \right]^{1/\rho} \right)
  \cdot
  \frac{1}{\rho}\left[ \int_0^N (Ah(i))^\rho \; di \right]^{1/\rho-1}
  \cdot
  \rho A^\rho h(i)^{\rho-1}
  = \lambda
\end{align*}
Dividing the FOC for good $i$ by the FOC for good $j$, we get
\begin{align*}
  \left(
  \frac{h(i)}{h(j)}
  \right)^{\rho-1}
  = 1
  \quad&\implies\quad
  h(i) = h(j)
  \qquad \forall i,j \in[0,N] \\
  &\implies\quad
  h(i) = \frac{1}{N}
\end{align*}
Hence, in equilibrium, labor is split up equally among the firms as
$h^*(i)=\frac{1}{N}$. As a result, consumption will be split evenly as
well and will equal
\begin{align*}
  c(i)^* = \frac{A}{N}
\end{align*}
This will leady to totaly utility at the optimum of
\begin{align*}
  u\left( \left[ \int_0^N c(i)^\rho \; di \right]^{1/\rho} \right)
  =
  u\left( \left[ \int_0^N
      \left[ \frac{A}{N} \right]^\rho \; di
    \right]^{1/\rho} \right)
  =
  u\left( A N^{\frac{1}{\rho}-1}\right)
\end{align*}
Hence, since $\rho\in(0,1)$, utility is \emph{increasing} in $N$.
There's a love of variety in the model.
%That's because marginal utility at $c(i)=0$ is infinite.

\clearpage
\subsection{SP Problem, Endogenous $N$}

As we saw in the last subsection, households love variety, so $N$ is
substantive. But so far, we've just taken it as given. Where does it
come from? Since it is important, we probably want to endogenize it.
But notice that since consumers love variety, the social planner would
just send $N\ra \infty$ if $N$ were a totally free variable. Therefore,
we want to introduce costs to more variety to make sure this problem is
well-defined.

In this problem, we will assume that labor can be used to produce either
new varieties or output of existing varieties. So the household
effectively chooses $N$, but at a cost. Specifically, it costs $\bar{h}$
units of labor to produce one mass unit of firms/variety. In other
words, to have a total mass of $N$ firms, it costs $N\bar{h}$ units of
labor.  The rest of the household's one unit of
labor---i.e.\ the remaining $1-N\bar{h}$ units---goes towards producing
the varieties. This leads to social planner's problem
\begin{align*}
  \max_{c(i),h(i),N}
    & \; u\left(
      \left[ \int_0^N c(i)^\rho \; di \right]^{1/\rho}
    \right)
  \\
  \text{s.t.} & \;
  \intzN h(i) \; di = 1 - N\bar{h}\\
  &\; c(i) = Ah(i) \\
  &\; c(i),h(i),N \geq 0
\end{align*}
Eliminate $c(i)$:
\begin{align*}
  \max_{h(i),N}
    & \; u\left(
      \left[ \int_0^N \left(Ah(i)\right)^\rho \; di \right]^{1/\rho}
    \right)
  \\
  \text{s.t} & \;
  \intzN h(i) \; di = 1 - N\bar{h}\\
  &\; h(i),N \geq 0
\end{align*}
Notice that the labor supply problem hasn't really changed. If we get
the FOCs with respect to $h(i)$, it will still be the case that the
household should allocate labor evenly among the firms to consume evenly
across varieties. Only now, there are only $1-N\bar{h}$ units of labor
to allocate towards producing the varieties (by the constraint) so that
$h(i)=\frac{1-N\bar{h}}{N}$. Hence, we can sub in to simplify
\begin{align*}
  \max_{N\geq 0}
    & \; u\left(
      \left[ \int_0^N \left(A\frac{1-N\bar{h}}{N}\right)^\rho \; di \right]^{1/\rho}
    \right) \\
  \Leftrightarrow\quad
  \max_{N\geq 0}
    & \; u\left(
      A(1-N\bar{h})
      N^{\frac{1}{\rho}-1}
    \right)
\end{align*}
But since $u$ is an increasing function, we can just maximize
\begin{align*}
  \max_{N\geq 0}
    & \;
      A(1-N\bar{h})
      N^{\frac{1}{\rho}-1}
\end{align*}
which has FOCs
\begin{align*}
  A\bar{h} N^{\frac{1}{\rho}-1}
  &=
  \left(\frac{1}{\rho}-1\right)
  A(1-N\bar{h}) N^{\frac{1}{\rho}-2} \\
  \Rightarrow\quad
  N &= \frac{1-\rho}{\bar{h}}
\end{align*}
So as $\rho\ra 1$ and the consumer views the goods more as perfect
substitutes, the social planner opts for fewer firms (and vice versa).
Moreover, as the cost $\bar{h}$ of creating new firms decreases, the
social planner opts for more firms.


\clearpage
\subsection{ADCE, Fixed $N$}

We now characterize an ADCE in an economy with variety , which a list
$c^*(i)$ consumption of each good, $h^*$ total labor supply, $h^*(i)$
labor supply to each firm, and $p^*(i)$ price of each good (wages left
out because normalized to one), that satisfy the following conditions:
\begin{enumerate}
  \item \emph{Household Optimization}: Taking prices $p^*(i)$ as given,
    consumption $c^*(i)$, and total labor supply $h^*$ solve
    \begin{align*}
      (c^*(i), h^*)
      = \max_{c(i), h}
        & \; u\left(
          \left[ \int_0^N c(i)^\rho \; di \right]^{1/\rho}
        \right)\\
      \text{s.t} & \;
        \intzN p^*(i) c(i) \; di= h \\
      &\; h\in[0,1] \\
      &\; c(i) \geq 0
    \end{align*}
    Note that the household is not optimizing over $h(i)$. They don't
    care which firm they work for. They just care amount the total level
    of labor $h$ they supply since that's how they earn money to buy
    consumption.

  \item \emph{Firm Optimization}: With each firm taking prices $p^*(i)$
    as given, $h^*(i)$ solves the profit maximization problem:
    \begin{align*}
      h^*(i)
      &=
      \max_{h(i)}
      p^*(i) Ah(i) - h(i) \\
      \text{s.t.} \;
      & \; h(i) \geq 0
    \end{align*}

  \item \emph{Market Clearing}: All of the labor provided to the
    different firms must aggregate to the amount of total labor
    provided, and everything produced must be consumed:
    \begin{align*}
       h^* &= \intzN h^*(i) \; di \\
       c^*(i) &= Ah^*(i)
    \end{align*}
\end{enumerate}
We do the usual things
\begin{enumerate}
  \item \emph{Household FOCs}: First with respect to consumption of the
    $i$th good $c(i)$:
    \begin{align*}
      u'\left(
        \left[ \int_0^N c(i)^\rho \; di \right]^{1/\rho}
      \right)
      \frac{1}{\rho}
      \left[ \int_0^N c(i)^\rho \; di \right]^{1/\rho-1}
      \rho c(i)^{\rho-1}
      = \lambda p^*(i)
    \end{align*}
    Dividing the FOC for $c(i)$ by the FOC for $c(j)$:
    \begin{align}
      \left(
      \frac{c(i)}{c(j)}
      \right)^{\rho-1}
      =
      \frac{p^*(i)}{p^*(j)}
      \label{hhfoc-variety}
    \end{align}

  \item \emph{Firm}: It must be the case that profits are zero,
    otherwise the firm would just make infinite profits. Therefore, we
    need
    \begin{align*}
      p^*(i) = \frac{1}{A}
      \qquad \forall i
    \end{align*}

  \item \emph{Market Clearing}: Since prices are all the same at
    $\frac{1}{A}$, Equation~\ref{hhfoc-variety} tells us that
    consumption of all goods must be equal since the prices are equal.
    Therefore, by market clearing, labor must be distributed equally
    among the firms, i.e. $h(i) = \frac{1}{N}$.
\end{enumerate}
Note that the firms are price takers in this market. They're not really
exploiting consumer's desire for variety to make more cash money.  And
since they're just price takers (so that there are no distortions in
this market), the outcome is exactly the same as the Social Planner's
problem---consumption and labor is divided equally among all firms. But
that's not very interesting. We want to see what happens when firms are
price-\emph{setters}, not price-takers.


\subsection{Monopolistic Price Setting and Demand}

First, firms will solve for the consumer's demand function. Since
consumer's take prices as given, we return to their utility maximization
problem given income $I$.\footnote{%
  More generally, we'd model income as wage rate times hours supplied,
  by it won't make much difference here, so we ignore it for now
}
We want to isolate how demand for good $i$ changes as price $p(i)$
changes---holding all other prices and the consumer's wealth constant,
which the firm takes as given. So firms consider the consumer's utility
maximization problem:
\begin{align*}
  \max_{c(i) \geq 0}
    & \; u\left(
      \left[ \int_0^N c(i)^\rho \; di \right]^{1/\rho}
    \right)\\
  \text{s.t} & \;
    \intzN p(i) c(i) \; di= I
\end{align*}
The consumer forms first order condition on $c(i)$ of
\begin{align*}
  u'\left(
    \left[ \int_0^N c(i)^\rho \; di \right]^{1/\rho}
  \right)
  \frac{1}{\rho}
  \left[ \int_0^N c(i)^\rho \; di \right]^{1/\rho-1}
  \rho c(i)^{\rho-1}
  = \lambda p(i)
\end{align*}
Dividing the FOC for $c(i)$ by the FOC for $c(j)$:
\begin{align}
  \left(
  \frac{c(i)}{c(j)}
  \right)^{\rho-1}
  =
  \frac{p(i)}{p(j)}
  &\quad \iff \quad
  c(i)
  =
  c(j)
  \left(
  \frac{p(i)}{p(j)}
  \right)^{\frac{1}{\rho-1}}
  \label{cidemand}
\end{align}
We can think of Equation~\ref{cidemand} as summarizing demand for the
$i$th good (for any $i\in[0,N]$) as a function of demand for some fixed
reference good $j$ and the price ratio between $i$ and $j$.
So we can substitute this expression for $c(i)$ into the budget equation
and simplify:
\begin{align*}
  I &= \intzN p(i) c(i) \; di
  = \intzN p(i)
    c(j)
    \left(
    \frac{p(i)}{p(j)}
    \right)^{\frac{1}{\rho-1}} \notag
    \; di \\
  %&=
    %\frac{c(j)}{p(j)^{\frac{1}{\rho-1}}}
    %\intzN p(i)
    %p(i)^{\frac{1}{\rho-1}} \notag
    %\; di \\
  \implies\quad
  I
  &=
    \frac{c(j)}{p(j)^{\frac{1}{\rho-1}}}
    \intzN
    p(i)^{\frac{\rho}{\rho-1}} \notag
    \; di
\end{align*}
Therefore, we can summarize demand for good $j$ as
\begin{align*}
  c(j)
  =
  p(j)^{\frac{1}{\rho-1}}
  \cdot
  \frac{I}{\intzN p(i)^{\frac{\rho}{\rho-1}} \notag \; di}
\end{align*}
And since our choice of reference good was arbitrary, this holds for any
$j$. Notice that this is extremely tractable since the fraction is a
constant from the perspective of firm $j$. So we have demand for good
$j$ written quite simply as
\begin{align*}
  c(j) = B p(j)^{\frac{1}{\rho-1}}
  \qquad\text{where} \quad B=
  \frac{I}{\intzN p(i)^{\frac{\rho}{\rho-1}} \notag \; di}
\end{align*}
Now return to the firm's problem. Taking wages (normalized to one) as
given, it's profit maximizing decision involves setting prices, choosing
output, and choosing labor to hire
\begin{align*}
  \max_{p(i), \,h(i), \,y(i)} \; &p(i) y(i) - h(i)
\end{align*}
But by market clearing and our assumptions on technology,
$c(i)=y(i)=Ah(i)$, so sub in
\begin{align*}
  \max_{p(i)} \; &p(i) c(i) - \frac{c(i)}{A} \\
  \max_{p(i)}
  \; &p(i) \left[B p(i)^{\frac{1}{\rho-1}}\right]
    - \frac{1}{A} \left[B p(i)^{\frac{1}{\rho-1}}\right] \\
  \max_{p(i)}
  \; & B p(i)^{\frac{\rho}{\rho-1}}
    - \frac{B}{A} \left[p(i)^{\frac{1}{\rho-1}}\right]
\end{align*}
Differentiating with respect to $p(i)$, this leads to FOC
\begin{align*}
  0&=
  B {\frac{\rho}{\rho-1}} p(i)^{\frac{1}{\rho-1}}
    - \frac{B}{A} \frac{1}{\rho-1}p(i)^{\frac{1}{\rho-1}-1} \\
  \Rightarrow\quad
  p(i) &= \frac{1}{\rho A}
\end{align*}
Note that this has a very nice form, as $\frac{1}{A}$ is the marginal
cost of production, while $\frac{1}{\rho}$ is refers to how willing
households are to substitute across varieties.


\subsection{MCE, fixed $N$}

We now characterize an MCE (monopolistic competitive equilibrium) in an
economy with variety but no entry, which a list $c^*(i)$ consumption of
each good, $h^*$ total labor supply, $h^*(i)$ labor supply to each firm,
$d^*_i(p)$ demand for good $i$ given price $p$ (holding all other prices
and profits constant), $p^*(i)$ prices for each good (wages left out
because normalized to one), and $\pi^*(i)$ profits that satisfy the
following conditions:
\begin{enumerate}
  \item \emph{Household Optimization}: Taking prices $p^*(i)$ and
    profits $\pi^*(i)$ as given, consumption $c^*(i)$, and total labor
    supply $h^*$ solve
    \begin{align*}
      (c^*(i), h^*)
      = \max_{c(i), h}
        & \; u\left(
          \left[ \int_0^N c(i)^\rho \; di \right]^{1/\rho}
        \right)\\
      \text{s.t} & \;
        \intzN p^*(i) c(i) \; di \leq h
          + \intzN \pi^*(i) \; di \\
      &\; h\in[0,1] \\
      &\; c(i) \geq 0
    \end{align*}
    while also $d^*_i(p)$ is the optimal choice for good $i$ as a
    function of price, holding all other prices and profits constant.

  \item \emph{Firm Optimization}: With each firm taking \emph{demand}
    (not prices this time) $d_i^*(p)$ as given, $p^*(i)$ and $h^*(i)$
    solve the profit maximization problem:
    \begin{align*}
      (p^*(i),h^*(i))
      &=
      \max_{p(i),h(i)}
      p(i) Ah(i) - h(i) \\
      \text{s.t.} \;
      & \; Ah(i) = d_i^*[p(i)] \\
      & \; h(i) \geq 0\\
      & \; \pi^*(i) = p^*(i) Ah^*(i) - h^*(i)
    \end{align*}

  \item \emph{Market Clearing}: All of the labor provided to the
    different firms must aggregate to the amount of total labor
    provided, and everything produced must be consumed:
    \begin{align*}
       h^* &= \intzN h^*(i) \; di \\
       c^*(i) &= Ah^*(i)
    \end{align*}
\end{enumerate}

\clearpage
\subsection{Symmetric MCE with Entry}

The goal here is to take everything the same, but let $N$ be endogenous.
We know what the equilibrium looks like for fixed $N$. It will just be a
matter of letting $N$ be a free parameter that adjusts until profits
from entry are zero.

So we now characterize a symmetric MCE in an economy with entry, which a
list
$c^*(i)$ consumption of each good,
$h^*$ total labor supply,
$h^{f*}$ labor for each producing firm,
%$h^*(i)$ labor supply to each firm,
$d^*(p)$ demand for each good $i$ given price $p$ (holding all other
prices and profits constant),
$p^*$ prices for each good (wages left out because normalized to one),
and $N^*>0$ mass of firms that satisfy the following conditions:
\begin{enumerate}
  \item \emph{Household Optimization}: Taking prices $p^*$ as given,
    consumption $c^*(i)$ and total labor supply $h^*$ solve
    \begin{align*}
      (c^*(i), h^*)
      = \max_{c(i), h}
        & \; u\left(
          \left[ \int_0^N c(i)^\rho \; di \right]^{1/\rho}
        \right)\\
      \text{s.t} & \;
        \intzN p^*(i) c(i) \; di \leq h \\
      &\; h\in[0,1] \\
      &\; c(i) \geq 0
    \end{align*}
    while also $d^*(p)$ is the optimal choice each good as a function of
    price, holding all other prices and profits constant.

  \item \emph{Firm Optimization}: With each firm taking \emph{demand}
    (not prices this time) $d^*(p)$ as given, $p^*$ and $h^*(i)$
    solve the profit maximization problem:
    \begin{align*}
      (p^*(i),h^*(i))
      &=
      \max_{p(i),h(i)}
      p(i) Ah(i) - h(i) \\
      \text{s.t.} \;
      & \; Ah(i) = d_i^*[p(i)] \\
      & \; h(i) \geq 0\\
      & \; \pi^*(i) = p^*(i) Ah^*(i) - h^*(i)
    \end{align*}

  \item \emph{Market Clearing}: All of the labor provided to the
    different firms must aggregate to the amount of total labor
    provided, and everything produced must be consumed:
    \begin{align*}
       h^* &= \intzN h^*(i) \; di \\
       c^*(i) &= Ah^*(i)
    \end{align*}
\end{enumerate}

\clearpage
\subsection{MCE with R\&D plus Entry}

Output $y_t$ is the ``final good'' produced by mixing many intermediate
goods $z_t(i)$ according to production function
\begin{align*}
  y_t = \left( \intzN z_t(i)^\alpha \; di \right) h_t^{1-\alpha}
\end{align*}
The intermediate goods are produced directly by capital, rather than the
final good as we've had before.
\begin{enumerate}
  \item \emph{Firm Optimization}:
    There are now three types of firms, each with their own optimization
    problems:
    \begin{enumerate}
      \item \emph{Final Goods Producers}:
        These take a variety of intermediate goods and produce a homo
        \begin{align*}
          \max_{z_t} \; & p^*_{z,t} z_t
        \end{align*}

      \item \emph{R\&D Sector}: These firms have linear technology in
        producing new intermediate products. They are monopolists on
        intermediate products---no other firm can create new
        intermediate products. Profits must be zero,
        \begin{align*}
          p_t^* (p_{N,t}^*-1) &= 0 \\
          N_0^* &= \bar{N}_0
        \end{align*}
        I multiplied by $p_t^*$ to normalize all prices. Within
        parentheses, we see that it costs one unit to create a new where
        $p_{N,t}^*$ is the price that it can be sold at.

      \item \emph{Intermediate Goods Sector}:
        \begin{align*}
          \max_{k_{z,t}, \; p_{z,t}}
          p_{z,t} \, k_{z,t}
          - r_t^* k_{z,t}
        \end{align*}

    \end{enumerate}
    With each firm taking \emph{demand}
    (not prices this time) $d^*(p)$ as given, $p^*$ and $h^*(i)$
    solve the profit maximization problem:
    \begin{align*}
      (p^*(i),h^*(i))
      &=
      \max_{p(i),h(i)}
      p(i) Ah(i) - h(i) \\
      \text{s.t.} \;
      & \; Ah(i) = d_i^*[p(i)] \\
      & \; h(i) \geq 0\\
      & \; \pi^*(i) = p^*(i) Ah^*(i) - h^*(i)
    \end{align*}

  \item \emph{Household Optimization}: Taking prices $p^*$ as given,
    consumption $c^*(i)$ and total labor supply $h^*$ solve
    \begin{align*}
      (c^*(i), h^*)
      = \max_{c(i), h}
        & \; u\left(
          \left[ \int_0^N c(i)^\rho \; di \right]^{1/\rho}
        \right)\\
      \text{s.t} & \;
        \intzN p^*(i) c(i) \; di \leq h \\
      &\; h\in[0,1] \\
      &\; c(i) \geq 0
    \end{align*}
    while also $d^*(p)$ is the optimal choice each good as a function of
    price, holding all other prices and profits constant.

  \item \emph{Market Clearing}: All of the labor provided to the
    different firms must aggregate to the amount of total labor
    provided, and everything produced must be consumed:
    \begin{align*}
       h^* &= \intzN h^*(i) \; di \\
       c^*(i) &= Ah^*(i)
    \end{align*}
\end{enumerate}




\clearpage
\section{Static Model with Variety in Variety}


We define the usual trifecta of primitives for a macro problem:
\begin{enumerate}
  \item \emph{Preferences}:
    Given mass of goods $N$ and the preference parameter $\rho\in(0,1)$,
    define utility over consumption of a continuum of goods, captured by
    function $c:[0,N]\ra \R$, as
    \begin{align*}
      U(c) &=
      u\left(
      \left[
      \int_0^N [a(i)c(i)]^\rho \; di
      \right]^{1/\rho}
      \right)
    \end{align*}

  \item \emph{Technology}:
    There is a variety of goods to consume, each of which is produced by
    a different firm, so we need to define the production function for
    each firm:
    \begin{align*}
      y(i) = A(i) h(i)
    \end{align*}
    where $y(i)$ is output of the $i$th good from the $i$th firm, $A(i)$
    is productivity in producing the $i$th good, and $h(i)$ is labor
    used to produce the $i$th good.

  \item \emph{Endowments}: Each household has one unit of labor to
    supply to the firms.
\end{enumerate}
Social planner's problem
\begin{align*}
  \max_{c(i),h(i)}
  &\;
  u\left( \left[ \int_0^N [a(i)c(i)]^\rho \; di \right]^{1/\rho} \right)
  \\
  \text{s.t.}
  &\; \intzN h(i) \; di = 1 \\
  &\; c(i) = A(i) h(i) \\
  &\; c(i),h(i)\geq 0
\end{align*}
Eliminate $h(i)$ and use the fact that $u$ is increasing to rewrite the
problem as just the maximization of CES-weighted consumption:
\begin{align*}
  C=
  \max_{c(i)\geq 0}
  &\;
  \left[ \int_0^N [a(i)c(i)]^\rho \; di \right]^{1/\rho}
  \\
  \text{s.t.}
  &\; \intzN \frac{c(i)}{A(i)} \; di = 1
\end{align*}
where $C$ is CES-weighted consumption at the maximum.
FOCs with respect to $c(i)$:
\begin{align*}
  \frac{\lambda}{A(i)}
  &=
  \frac{1}{\rho}\left[
    \int_0^N [a(i)c(i)]^\rho \; di
  \right]^{\frac{1}{\rho}-1}
  \rho [a(i) c(i)]^{\rho-1} a(i)
  \\
  \frac{\lambda}{A(i)}
  &=
  C^{1-\rho}
  a(i)^\rho c(i)^{\rho-1}
  \\
  \frac{1}{A(i)}
  &=
  \lambda^{-1}
  C^{1-\rho}
  a(i)^\rho c(i)^{\rho-1}
\end{align*}
Subbing this into the constraint
\begin{align*}
  1&=
  \intzN \frac{c(i)}{A(i)} \; di
  =
  \intzN
    c(i)
    \lambda^{-1} C^{1-\rho} a(i)^\rho c(i)^{\rho-1}
    \; di \\
  \lambda &=
  C^{1-\rho}
  \intzN
    [a(i)c(i)]^{\rho}
    \; di \\
  \lambda &= C^{1-\rho} C^\rho = C
\end{align*}
Therefore, we have identified $\lambda = C$. Using this and solving the
FOC for $c(i)$, you get
\begin{align*}
  c(i)
  &=
  \left(
  \frac{C^{\rho}}{A(i)a(i)^\rho}
  \right)^{\frac{1}{\rho-1}}
  %=
  %\left(\frac{C}{a(i)}\right)^{\frac{\rho}{\rho-1}}
  %A(i)^{\frac{1}{1-\rho}}
  =
  A(i)
  \left(\frac{A(i)a(i)}{C}\right)^{\frac{\rho}{1-\rho}}
\end{align*}
Subbing this into the definition of $C$
\begin{align*}
  C
  = \left[ \int_0^N [a(i)c(i)]^\rho \; di \right]^{1/\rho}
  &=
  \left[ \int_0^N
    \left[a(i)
    \left(
    \frac{C^{\rho}}{A(i)a(i)^\rho}
    \right)^{\frac{1}{\rho-1}}
    \right]^\rho \; di \right]^{1/\rho} \\
  C
  &=
  \left[ \int_0^N
    \left[
    A(i)a(i)
    \right]^{\frac{\rho}{1-\rho}}
    \; di \right]^{\frac{1-\rho}{\rho}}
\end{align*}
Hence, consumption can be written
\begin{align*}
  c(i)
  &=
  A(i)
  \left(\frac{A(i)a(i)}{C}\right)^{\frac{\rho}{1-\rho}} \\
  &=
  A(i)
  \frac{[A(i)a(i)]^{\frac{\rho}{1-\rho}}}{%
    \int_0^N \left[ A(i)a(i) \right]^{\frac{\rho}{1-\rho}} \; di
  }
\end{align*}
Hence, labor allocation is
\begin{align*}
  h(i) =
  \frac{[A(i)a(i)]^{\frac{\rho}{1-\rho}}}{%
    \int_0^N \left[ A(i)a(i) \right]^{\frac{\rho}{1-\rho}} \; di
  }
\end{align*}




%% APPPENDIX %%

% \appendix




\end{document}


%%%%%%%%%%%%%%%%%%%%%%%%%%%%%%%%%%%%%%%%%%%%%%%%%%%%%%%%%%%%%%%%%%%%%%%%
%%%%%%%%%%%%%%%%%%%%%%%%%%%%%%%%%%%%%%%%%%%%%%%%%%%%%%%%%%%%%%%%%%%%%%%%
%%%%%%%%%%%%%%%%%%%%%%%%%%%%%%%%%%%%%%%%%%%%%%%%%%%%%%%%%%%%%%%%%%%%%%%%

%%%% SAMPLE CODE %%%%%%%%%%%%%%%%%%%%%%%%%%%%%%%%%%%%%%

    %% VIEW LAYOUT %%

        \layout

    %% LANDSCAPE PAGE %%

        \begin{landscape}
        \end{landscape}

    %% BIBLIOGRAPHIES %%

        \cite{LabelInSourcesFile}  %Use in text; cites
        \citep{LabelInSourcesFile} %Use in text; cites in parens

        \nocite{LabelInSourceFile} % Includes in refs w/o specific citation
        \bibliographystyle{apalike}  % Or some other style

        % To ditch the ``References'' header
        \begingroup
        \renewcommand{\section}[2]{}
        \endgroup

        \bibliography{sources} % where sources.bib has all the citation info

    %% SPACING %%

        \vspace{1in}
        \hspace{1in}

    %% URLS, EMAIL, AND LOCAL FILES %%

      \url{url}
      \href{url}{name}
      \href{mailto:mcocci@raidenlovessusie.com}{name}
      \href{run:/path/to/file.pdf}{name}


    %% INCLUDING PDF PAGE %%

        \includepdf{file.pdf}


    %% INCLUDING CODE %%

        %\verbatiminput{file.ext}
            %   Includes verbatim text from the file

        \texttt{text}
            %   Renders text in courier, or code-like, font

        \matlabcode{file.m}
            %   Includes Matlab code with colors and line numbers

        \lstset{style=bash}
        \begin{lstlisting}
        \end{lstlisting}
            % Inline code rendering


    %% INCLUDING FIGURES %%

        % Basic Figure with size scaling
            \begin{figure}[h!]
               \centering
               \includegraphics[scale=1]{file.pdf}
            \end{figure}

        % Basic Figure with specific height
            \begin{figure}[h!]
               \centering
               \includegraphics[height=5in, width=5in]{file.pdf}
            \end{figure}

        % Figure with cropping, where the order for trimming is  L, B, R, T
            \begin{figure}
               \centering
               \includegraphics[trim={1cm, 1cm, 1cm, 1cm}, clip]{file.pdf}
            \end{figure}

        % Side by Side figures: Use the tabular environment


