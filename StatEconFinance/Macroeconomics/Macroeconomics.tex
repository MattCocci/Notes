\documentclass[12pt]{article}

\author{Matthew D. Cocci}
\title{Macroeconomics}
\date{\today}

%% Formatting & Spacing %%%%%%%%%%%%%%%%%%%%%%%%%%%%%%%%%%%%

%\usepackage[top=1in, bottom=1in, left=1in, right=1in]{geometry} % most detailed page formatting control
\usepackage{fullpage} % Simpler than using the geometry package; std effect
\usepackage{setspace}
%\onehalfspacing
\usepackage{microtype}

%% Formatting %%%%%%%%%%%%%%%%%%%%%%%%%%%%%%%%%%%%%%%%%%%%%

%\usepackage[margin=1in]{geometry}
    %   Adjust the margins with geometry package
%\usepackage{pdflscape}
    %   Allows landscape pages
%\usepackage{layout}
    %   Allows plotting of picture of formatting



%% Header %%%%%%%%%%%%%%%%%%%%%%%%%%%%%%%%%%%%%%%%%%%%%%%%%

%\usepackage{fancyhdr}
%\pagestyle{fancy}
%\lhead{}
%\rhead{}
%\chead{}
%\setlength{\headheight}{15.2pt}
    %   Make the header bigger to avoid overlap

%\fancyhf{}
    %   Erase header settings

%\renewcommand{\headrulewidth}{0.3pt}
    %   Width of the line

%\setlength{\headsep}{0.2in}
    %   Distance from line to text


%% Mathematics Related %%%%%%%%%%%%%%%%%%%%%%%%%%%%%%%%%%%

\usepackage{amsmath}
\usepackage{amssymb}
\usepackage{amsfonts}
\usepackage{mathrsfs}
\usepackage{amsthm} %allows for labeling of theorems
%\numberwithin{equation}{section} % Number equations by section
\theoremstyle{plain}
\newtheorem{thm}{Theorem}[section]
\newtheorem{lem}[thm]{Lemma}
\newtheorem{prop}[thm]{Proposition}
\newtheorem{cor}[thm]{Corollary}

\theoremstyle{definition}
\newtheorem{defn}[thm]{Definition}
\newtheorem{ex}[thm]{Example}

\theoremstyle{remark}
\newtheorem*{rmk}{Remark}
\newtheorem*{note}{Note}

% Below supports left-right alignment in matrices so the negative
% signs don't look bad
\makeatletter
\renewcommand*\env@matrix[1][c]{\hskip -\arraycolsep
  \let\@ifnextchar\new@ifnextchar
  \array{*\c@MaxMatrixCols #1}}
\makeatother


%% Font Choices %%%%%%%%%%%%%%%%%%%%%%%%%%%%%%%%%%%%%%%%%

\usepackage[T1]{fontenc}
\usepackage{lmodern}
\usepackage[utf8]{inputenc}
%\usepackage{blindtext}
\usepackage{courier}


%% Figures %%%%%%%%%%%%%%%%%%%%%%%%%%%%%%%%%%%%%%%%%%%%%%

\usepackage{tikz}
\usetikzlibrary{decorations.pathreplacing}
\usepackage{graphicx}
\usepackage{subfigure}
    %   For plotting multiple figures at once
%\graphicspath{ {Directory/} }
    %   Set a directory for where to look for figures


%% Hyperlinks %%%%%%%%%%%%%%%%%%%%%%%%%%%%%%%%%%%%%%%%%%%%
\usepackage{hyperref}
\hypersetup{%
    colorlinks,
        %   This colors the links themselves, not boxes
    citecolor=black,
        %   Everything here and below changes link colors
    filecolor=black,
    linkcolor=black,
    urlcolor=black
}

%% Colors %%%%%%%%%%%%%%%%%%%%%%%%%%%%%%%%%%%%%%%%%%%%%%%

\usepackage{color}
\definecolor{codegreen}{RGB}{28,172,0}
\definecolor{codelilas}{RGB}{170,55,241}

% David4 color scheme
\definecolor{d4blue}{RGB}{100,191,255}
\definecolor{d4gray}{RGB}{175,175,175}
\definecolor{d4black}{RGB}{85,85,85}
\definecolor{d4orange}{RGB}{255,150,100}

%% Including Code %%%%%%%%%%%%%%%%%%%%%%%%%%%%%%%%%%%%%%%

\usepackage{verbatim}
    %   For including verbatim code from files, no colors
\usepackage{listings}
    %   For including code snippets written directly in this doc

\lstdefinestyle{bash}{%
  language=bash,%
  basicstyle=\footnotesize\ttfamily,%
  showstringspaces=false,%
  commentstyle=\color{gray},%
  keywordstyle=\color{blue},%
  xleftmargin=0.25in,%
  xrightmargin=0.25in
}

\lstdefinestyle{matlab}{%
  language=Matlab,%
  basicstyle=\footnotesize\ttfamily,%
  breaklines=true,%
  morekeywords={matlab2tikz},%
  keywordstyle=\color{blue},%
  morekeywords=[2]{1}, keywordstyle=[2]{\color{black}},%
  identifierstyle=\color{black},%
  stringstyle=\color{codelilas},%
  commentstyle=\color{codegreen},%
  showstringspaces=false,%
    %   Without this there will be a symbol in
    %   the places where there is a space
  numbers=left,%
  numberstyle={\tiny \color{black}},%
    %   Size of the numbers
  numbersep=9pt,%
    %   Defines how far the numbers are from the text
  emph=[1]{for,end,break,switch,case},emphstyle=[1]\color{red},%
    %   Some words to emphasise
}

\newcommand{\matlabcode}[1]{%
    \lstset{style=matlab}%
    \lstinputlisting{#1}
}
    %   For including Matlab code from .m file with colors,
    %   line numbering, etc.

%% Bibliographies %%%%%%%%%%%%%%%%%%%%%%%%%%%%%%%%%%%%

%\usepackage{natbib}
    %---For bibliographies
%\setlength{\bibsep}{3pt} % Set how far apart bibentries are

%% Misc %%%%%%%%%%%%%%%%%%%%%%%%%%%%%%%%%%%%%%%%%%%%%%

\usepackage{enumitem}
    %   Has to do with enumeration
\usepackage{appendix}
%\usepackage{natbib}
    %   For bibliographies
\usepackage{pdfpages}
    %   For including whole pdf pages as a page in doc


%% User Defined %%%%%%%%%%%%%%%%%%%%%%%%%%%%%%%%%%%%%%%%%%

%\newcommand{\nameofcmd}{Text to display}
\newcommand*{\Chi}{\mbox{\large$\chi$}} %big chi
    %   Bigger Chi

% In math mode, Use this instead of \munderbar, since that changes the
% font from math to regular
\makeatletter
\def\munderbar#1{\underline{\sbox\tw@{$#1$}\dp\tw@\z@\box\tw@}}
\makeatother

% Limits
\newcommand{\limN}{\lim_{N\rightarrow\infty}}
\newcommand{\limn}{\lim_{n\rightarrow\infty}}
\newcommand{\limt}{\lim_{t\rightarrow\infty}}
\newcommand{\limT}{\lim_{T\rightarrow\infty}}
\newcommand{\limhz}{\lim_{h\rightarrow 0}}

% Misc Math
\newcommand{\Prb}{\mathrm{P}}
\newcommand{\ra}{\rightarrow}
\newcommand{\diag}{\text{diag}}
\newcommand{\ch}{\text{ch}}
\newcommand{\dom}{\text{dom}}

% Script
\newcommand{\sF}{\mathscr{F}}
\newcommand{\sB}{\mathscr{B}}
\newcommand{\sL}{\mathscr{L}}
\newcommand{\sM}{\mathscr{M}}
\newcommand{\sT}{\mathscr{T}}
\newcommand{\sA}{\mathscr{A}}

% Mathcal
\newcommand{\calB}{\mathcal{B}}
\newcommand{\calD}{\mathcal{D}}
\newcommand{\calF}{\mathcal{F}}
\newcommand{\calG}{\mathcal{G}}
\newcommand{\calH}{\mathcal{H}}

% Dot over
\newcommand{\dx}{\dot{x}}
\newcommand{\ddx}{\ddot{x}}

% Blackboard
\newcommand{\R}{\mathbb{R}}
\newcommand{\Rn}{\mathbb{R}^n}
\newcommand{\Rk}{\mathbb{R}^n}
\newcommand{\Rnn}{\mathbb{R}^{n\times n}}
\newcommand{\C}{\mathbb{C}}
\newcommand{\Cn}{\mathbb{C}^n}
\newcommand{\Cnn}{\mathbb{C}^{n\times n}}
\newcommand{\E}{\mathbb{E}}
\newcommand{\N}{\mathbb{N}}

\DeclareMathOperator*{\argmin}{arg\;min}
\DeclareMathOperator*{\argmax}{arg\;max}
\newenvironment{rcases}
  {\left.\begin{aligned}}
  {\end{aligned}\right\rbrace}

% Various probability and statistics commands
\newcommand{\Cov}{\operatorname{Cov}}
\newcommand{\Corr}{\operatorname{Corr}}
\newcommand{\Var}{\operatorname{Var}}
\newcommand{\asto}{\xrightarrow{a.s.}}
\newcommand{\pto}{\xrightarrow{p}}
\newcommand{\msto}{\xrightarrow{m.s.}}
\newcommand{\dto}{\xrightarrow{d}}
\newcommand{\Lpto}{\xrightarrow{L_p}}
\newcommand{\plim}{\text{plim}_{n\rightarrow\infty}}

% Redefine real and imaginary from fraktur to plain text
\renewcommand{\Re}{\operatorname{Re}}
\renewcommand{\Im}{\operatorname{Im}}

% Shorter sums: ``Sum from X to Y''
% - sumXY  is equivalent to \sum^Y_{X=1}
% - sumXYz is equivalent to \sum^Y_{X=0}
\newcommand{\sumnN}{\sum^N_{n=1}}
\newcommand{\sumin}{\sum^n_{i=1}}
\newcommand{\sumkn}{\sum^n_{k=1}}
\newcommand{\sumtT}{\sum^T_{t=1}}
\newcommand{\sumninf}{\sum^\infty_{n=1}}
\newcommand{\sumtinf}{\sum^\infty_{t=1}}
\newcommand{\sumnNz}{\sum^N_{n=0}}
\newcommand{\suminz}{\sum^n_{i=0}}
\newcommand{\sumknz}{\sum^n_{k=0}}
\newcommand{\sumtTz}{\sum^T_{t=0}}
\newcommand{\sumninfz}{\sum^\infty_{n=0}}
\newcommand{\sumtinfz}{\sum^\infty_{t=0}}


%%%%%%%%%%%%%%%%%%%%%%%%%%%%%%%%%%%%%%%%%%%%%%%%%%%%%%%%%%%%%%%%%%%%%%%%
%% BODY %%%%%%%%%%%%%%%%%%%%%%%%%%%%%%%%%%%%%%%%%%%%%%%%%%%%%%%%%%%%%%%%
%%%%%%%%%%%%%%%%%%%%%%%%%%%%%%%%%%%%%%%%%%%%%%%%%%%%%%%%%%%%%%%%%%%%%%%%


\begin{document}
\maketitle

%\tableofcontents

\section{Utility Functions}

\begin{defn}(Absolute Risk Aversion)
Given twice-differentiable utility function $u$,
\emph{absolute risk aversion} is defined
\begin{align*}
  A(c) = -\frac{u''(c)}{u'(c)}
\end{align*}
\end{defn}

\begin{defn}(Relative Risk Aversion)
Given twice-differentiable utility function $u$,
\emph{relative risk aversion} is defined
\begin{align*}
  R(c) = cA(c) = -\frac{c u''(c)}{u'(c)}
\end{align*}
\end{defn}

\begin{defn}(Intertemporal Elasticity of Substitution)
Given differentiable utility function $u$,
the \emph{intertemporal elasticity of substitution} is defined
\begin{align*}
  IES
  = -\frac{\partial \ln(c_{t+1}/c_t)}{\partial r}
  = -\frac{\partial \ln(c_{t+1}/c_t)}{%
    \partial \ln(u'(c_{t+1})/u'(c_t))}
\end{align*}
High IES means that the the consumer will change her consumption
dramatically given small changes in the interest rate---high elasticity,
comfortable substituting across time.

Low IES corresponds to a strong smoothing motive. The consumer is
\emph{not} very responsive to changes in the interest rate.
\end{defn}

\begin{defn}(Power Utility or CRRA)
We define the period power utility function as
\begin{align*}
  u(c) &= \frac{c^{1-\sigma}-1}{1-\sigma} \\
  u'(c) &= c^{-\sigma} \\
  u''(c) &= -\sigma c^{-\sigma-1}
\end{align*}
It has the feature that
\begin{align*}
  \lim_{\sigma\ra 1}
  \frac{c^{1-\sigma}-1}{1-\sigma}
  = \ln(c)
\end{align*}
It exhibits constant relative risk aversion
\begin{align*}
  R(c) &=
  -\frac{c u''(c)}{u'(c)}
  = -\frac{c (-\sigma c^{-\sigma-1})}{c^{-\sigma}}
  = \sigma
\end{align*}
The IES is also tied to the coefficient of relative risk aversion:
\begin{align*}
  IES
  = \frac{1}{\sigma}
\end{align*}
\end{defn}


\section{Static Resource Allocation Problems}

\begin{defn}{(Economy)}
We define an \emph{economy} $E$ for $I$ individuals and $K$ goods as a
list
\begin{align*}
  E = \{(X_i,u_i,\omega_i)\}^I_{i=1}
\end{align*}
\end{defn}


\section{Solow Model}

Fixed output-share of investment each period, $s$. Capital evolves
according to
\begin{align*}
  k_{t+1} &= sf(k_t) + (1-\delta)k_t \\
  k_0 &= \hat{k}_0
\end{align*}
Unique positive steady state, $k^*$

\clearpage
\section{Competitive Equilibrium}

\subsection{Arrow-Debreu Competetive Equilibrium (ADCE): Household
Ownership of Capital}

In this subsection, we formulate an
\emph{Arrow-Debreu Competetive Equilibrium} (ADCE) for a simple
production-based economy involving households that consume output,
supply labor, and own capital along with firms that produce output by
buying labor and renting capital from the households (paying a rate of
return).

Note that there will be no real link between time periods for capital.
You can change capital a lot from $k_t$ to $k_{t+1}$. You could link up
time periods by adding adjustment costs, a feature we ignore for now.

In this context, and ADCE is a list of sequences of
$\{c_t^*\}$  Consumption,
$\{k_t^*\}$  Capital,
$\{h_t^*\}$  Hours,
$\{p_t^*\}$  Prices,
$\{w_t^*\}$  Wages, and
$\{r_t^*\}$  Rate of return on capital,
that satisfy the following conditions:
\begin{enumerate}
  \item \emph{Household Optimization}: The household---taking prices,
    wages, and the rate of return on capital
    ($\{p^*_t\}$, $\{w^*_t\}$, $\{r^*_t\}$) as given---chooses
    sequences for consumption, capital, and hours of labor
    ($\{c_t\}$, $\{k_t\}$, and $\{h_t\}$)
    to maximize the present discounted value of utility, which happen to
    exactly match equilibrium sequences
    ($\{c_t^*\}$, $\{k_t^*\}$, and $\{h_t^*\}$)
    \begin{align}
      (\{c_t^*\}, \{k_t^*\}, \{h_t^*\})
      =
      \argmax_{\{c_t\}, \{k_t\}, \{h_t\}}
        \; &\sumtinfz \beta^t u(c_t)
        \label{defn:adce-hh-objfcn}\\
      \text{s.t.} \; &
        \sumtinfz p^*_t(c_t + k_{t+1}-(1-\delta)k_t) \leq
        \sumtinfz (r^*_t k_t + w^*_t h_t)
        \label{defn:adce-hh-budget}\\
      c_t &\geq 0 \notag\\
      h_t &\in [0,1] \notag\\
      k_0 &= \bar{k}_0 \; \text{given} \notag
    \end{align}
    A few remarks:
    \begin{enumerate}
      \item Inequality~\ref{defn:adce-hh-budget} acts as a budget
        constraint that stipulates total lifetime spending on
        consumption and capital must be less than or equal to total
        lifetime capital-rental income and labor income. Note that it is
        a single budget constraint for time zero, not a sequence of
        budget constraints each period.

      \item Outside of the budget constraint, the remaning constraints
        enforce nonnegative consumption, supply of hours limited by
        hours in the day (normalized to one), and capital given.

      \item There is no non-negativity constraint on capital since the
        firm's production function will be defined for positive capital
        values only. Hence they will never demand negative capital, so
        we can leave it out of the constraints here.

      \item Labor $h_t$ is not in the utility function, so it is pretty
        obvious that the household would choose to supply $h_t=1$ units
        of labor for all $t$. That earns housholds the most consumption
        (which they value) at the cost of leisure (which we assume they
        don't care about).

        So it might seem like we could have left the $\{h_t\}$ out of
        the definition of equilibrium altogether. However, that is not
        the case. Even though households will choose to supply one unit
        of labor, firms don't know that. We must keep the labor decision
        in the model because that will allow us to pin down the wage
        rate.
    \end{enumerate}

  \item \emph{Firm Optimization}: The firm---taking
    prices, wages, and the rate of return on capital
    ($\{p^*_t\}$, $\{w^*_t\}$, $\{r^*_t\}$) as given---chooses
    the amount of capital to rent and labor hours to buy
    ($\{k_t\}$, $\{h_t\}$)
    that maximizes the present discount value of profits, which happens
    to exactly match equilibrium sequences
    ($\{k_t^*\}$ and $\{h_t^*\}$)
    \begin{align}
      (\{k_t^*\}, \{h_t^*\})
      =
      \argmax_{\{k_t\}, \{h_t\}}
        \; &\sumtinfz \beta^t (p^*_t F(k_t,h_t) - w^*_t h_t - r^*_t k_t)
        \label{defn:adce-firm-objfcn}\\
      \text{s.t.} \;
      &k_t \geq 0 \notag\\
      &h_t \geq 0 \notag
    \end{align}
    A few remarks:
    \begin{enumerate}
      \item Firms don't worry about the fact that we must have
        $h_t\leq 1$. To them, they can buy as many labor hours as they
        want at rate $w^*_t$. In reality, the household optimization
        side of the equilibrium will enforce $h_t\leq 1$.

      \item Though there does not appear to be any discounting in the
        firm's objective function, it is implicit in the prices. As
        we'll see below, $p_t\ra 0$, which will take care of the
        discounting.
    \end{enumerate}

  \item \emph{Market Clearing}: Production is used up as either
    consumption or investment
    \begin{align*}
      F(k^*_t,h_t^*) = c_t^* + k_{t+1}^* - (1-\delta) k_t^*
    \end{align*}
\end{enumerate}

We are now in a position to characterize equilibrium. Rather than appeal
to the welfare theorem to say that the outcome is Pareto optimal (hence
it looks exactly like the outcome of the planning problem), we will look
at the first order conditions to the different optimization problems we
have. It turns out that they \emph{will} match the planning problem,
which will offer a kind of proof of the first welfare theorem.
\begin{enumerate}
  \item \emph{Household FOCs}: Again there is a single budget
    constraint at time zero, not one for each period. So
    Inequality~\ref{defn:adce-hh-budget} has only one associated
    multiplier, $\mu$. Therefore, differentiating with respect to
    consumption and capital, we get FOCs for time $t$ as:
    \begin{align}
      \beta^t u'(c_t) &= \mu p_t^* \label{hhfoc1}\\
      \mu[r_t^* + p_t^* (1-\delta)]
      &= \mu p_{t-1}^* \label{hhfoc2}
    \end{align}
    while clearly $h_t=1$.

    These are necessary conditions for optimal $c_t$ that follow from
    the fact that $u'(0)=\infty$, so we have an interior solution,
    implying the KKT conditions are necessary.

  \item \emph{Firm FOCs}: We have
    \begin{align}
      p^*_t F_1(k_t,h_t) &= r_t^* \label{firmfoc1}\\
      p^*_t F_2(k_t,h_t) &= w_t^* \notag
    \end{align}
    Notice that capital appears nowhere in those equations.
    We instead have Equation~\ref{firmfoc1} relating \emph{prices},
    rather than quantities of capital. This is called a
    \emph{no arbitrage condition}.

  \item \emph{Transversality Condition}:
    Think about a finite-time version of this problem. The household has
    to choose consumption and investment for the last period, denoted
    time $T$. At $T$, the household can either consume output in the
    form of $c_T$ or invest to keep a positive $k_{T+1}$. But since $T$
    is the last period, investing output to keep $k_{T+1}>0$ looks
    pretty dumb when you could instead use that output for higher
    consumption $c_T$.  So it must be that \emph{either} $k_{T+1}=0$
    (i.e.\ you consume all output in the last period and invest nothing)
    or the value of consumption---the marginal utility
    $u'(c_T)$---equals zero so that more consumption $c_T$ isn't
    actually worth shit. In math, you'd need
    \begin{align*}
      \beta^{T} u'(c_T) k_{T+1} = 0
    \end{align*}
    The transversality condition is the infinite horizon analog to this
    problem, written
    \begin{align*}
      \limt \beta^t u'(c_t) k_{t+1}=0
    \end{align*}

  \item \emph{Pareto Optimality}: We want to show the competetive
    equilibrium matches the solution to the planning problem. Since
    we're considering the equilibrium sequences, everything will have
    ${}^*$s.

    Start by dividing Equation~\ref{hhfoc1} at time $t-1$ by its time
    $t$ analog to get
    \begin{align}
      \frac{u'(c_{t-1}^*)}{\beta u'(c_{t}^*)} = \frac{p_{t-1}^*}{p_t^*}
      \label{pratio1}
    \end{align}
    Take household FOC Equation~\ref{hhfoc2}, cancel $\mu$'s, and divide
    by $p_t^*$:
    \begin{align*}
      \frac{r_t^*}{p_t^*} + (1-\delta)
      &= \frac{p_{t-1}^*}{p_t^*}
    \end{align*}
    Into the above equation, substitute Equation~\ref{firmfoc1} in for
    $r_t^*$:
    \begin{align}
      F_1(k_t,h_t) + (1-\delta)
      &= \frac{p_{t-1}^*}{p_t^*}
      \label{pratio2}
    \end{align}
    How equate Equations~\ref{pratio1} and \ref{pratio2}:
    \begin{align*}
      \frac{u'(c_{t-1}^*)}{\beta u'(c_{t}^*)}
      = F_1(k_t,h_t) + (1-\delta)
    \end{align*}
    So we have this, together with market clearing, $h_t=1$, the
    transversality condition, and an initial capital level:
    \begin{align*}
      \frac{u'(c_{t-1}^*)}{\beta u'(c_{t}^*)}
      &= F_1(k_t,h_t) + (1-\delta) \\
      F(k_t,h_t) &= c_t + k_{t+1} - (1-\delta) k_t \\
      h_t &= 1\\
      \limt \beta^t u'(c_t) k_{t+1}&=0\\
      k_0 &= \bar{k}_0 \; \text{given}
    \end{align*}
    These are exactly the conditions that characterized the Pareto
    efficient allocations from the planning problem.

  \item \emph{Prices}: Normalize $p_0^*=1$, which will also remove
    indeterminacy problems. We get the wage and capital rental rates
    from the firm's first order conditions:
    \begin{align*}
      r_t^* &= p_t^* F_1(k^*_t,h^*_t) \\
      w_t^* &= p_t^* F_2(k^*_t,h^*_t)
    \end{align*}
    We get tomorrow's price by Solving Equation~\ref{pratio1} for
    $p_{t+1}$:
    \begin{align}
       p_t^*
       = p_{t-1}^*\beta \frac{u'(c_{t}^*)}{u'(c_{t-1}^*)}
    \end{align}
\end{enumerate}
Lastly, a \emph{steady state ADCE} is a value $k^*$ and an ADCE for the
economy with $\bar{k}_0=k^*$ and $\{k_t^*\}$ satisfying $k^*_t=k^*$ for
all $t$. Consumption must be a constant $c^*$ because solving the market
clearing equation for consumption expresses consumption as a function of
constants:
\begin{align*}
  c^* := c_t^* = F(k^*,1) - \delta k^*
  \qquad \forall t
\end{align*}
Therefore, by Equation~\ref{pratio1}
\begin{align*}
  \frac{u'(c^*)}{\beta u'(c^*)} &= \frac{p_{t-1}^*}{p_t^*} \\
  \implies \quad p_t &= \beta p_{t-1}^* = \beta^t p_0^*
\end{align*}
Hence, prices are \emph{not} constant---they are falling. The same logic
can be used to show that wages and rental rates are falling
monotonically in steady state:
\begin{align*}
  r_t^* &= p_t^* F_1(k^*,1) = \beta^t p_0^* F_1(k^*,1) = \beta^t r_0^*\\
  w_t^* &= p_t^* F_2(k^*,1) = \beta^t p_0^* F_2(k^*,1) = \beta^t w_0^*
\end{align*}
Hence we have justified the earlier claim that ``prices embed
discounting'' within the firm's optimization problem.

\subsection{Arrow-Debreu Competetive Equilibrium (ADCE): Taxation}

In this subsection, we suppose that there is a permanent proportional
tax on capital income at rate $\tau_k$, where the revenues fund a
lump-sum transfer to households, subject to a period-by-period budget
constraint.

In this context, and ADCE is a list of sequences of
$\{c_t^*\}$  Consumption,
$\{k_t^*\}$  Capital,
$\{h_t^*\}$  Hours,
$\{p_t^*\}$  Prices,
$\{w_t^*\}$  Wages,
$\{r_t^*\}$  Rate of return on capital, and
$\{T_t^*\}$ Transfers
that satisfy the following conditions:

\begin{enumerate}
  \item \emph{Household Optimization}: The household---taking prices,
    wages, the rate of return on capital, and transfers
    ($\{p^*_t\}$, $\{w^*_t\}$, $\{r^*_t\}$, $\{T_t^*\}$) as
    given---chooses sequences for consumption, capital, and hours of
    labor
    ($\{c_t\}$, $\{k_t\}$, and $\{h_t\}$)
    to maximize the present discounted value of utility, which happen to
    exactly match equilibrium sequences
    ($\{c_t^*\}$, $\{k_t^*\}$, and $\{h_t^*\}$)
    \begin{align}
      (\{c_t^*\}, \{k_t^*\}, \{h_t^*\})
      =
      \argmax_{\{c_t\}, \{k_t\}, \{h_t\}}
        \; &\sumtinfz \beta^t u(c_t)
        \label{defn:adce-hh-objfcn-taxes}\\\notag\\
      \text{s.t.} \;
        \sumtinfz p^*_t(c_t + k_{t+1}-(1-\delta)k_t) &\leq
        \sumtinfz ((1-\tau_k)r^*_t k_t + w^*_t h_t + T^*_t)
        \label{defn:adce-hh-budget-taxes}\\
      c_t &\geq 0 \notag\\
      h_t &\in [0,1] \notag\\
      k_0 &= \bar{k}_0 \; \text{given} \notag
    \end{align}

  \item \emph{Firm Optimization}: The firm---taking
    prices, wages, and the rate of return on capital
    ($\{p^*_t\}$, $\{w^*_t\}$, $\{r^*_t\}$) as given---chooses
    the amount of capital to rent and labor hours to buy
    ($\{k_t\}$, $\{h_t\}$)
    that maximizes the present discount value of profits, which happens
    to exactly match equilibrium sequences
    ($\{k_t^*\}$ and $\{h_t^*\}$)
    \begin{align}
      (\{k_t^*\}, \{h_t^*\})
      =
      \argmax_{\{k_t\}, \{h_t\}}
        \; &\sumtinfz \beta^t (p^*_t F(k_t,h_t) - w^*_t h_t - r^*_t k_t)
        \label{defn:adce-firm-objfcn-taxes}\\
      \text{s.t.} \;
      &k_t \geq 0 \notag\\
      &h_t \geq 0 \notag
    \end{align}

  \item \emph{Government}: For each $t$:
    \begin{align}
      \tau_k r^*_t k^*_t = T^*_t
      \label{taxes-govt}
    \end{align}

  \item \emph{Market Clearing}: For each $t$:
    \begin{align}
      F(k^*_t,h_t^*) = c_t^* + k^*_{t+1} - (1-\delta)k_t^*
      \label{taxes-clear}
    \end{align}
\end{enumerate}
As before, we can characterize the necessary conditions of the ADCE:

\begin{enumerate}
  \item \emph{Household FOCs}: Again there is a single budget
    constraint at time zero, not one for each period. So
    Inequality~\ref{defn:adce-hh-budget-taxes} has only one associated
    multiplier, $\mu$. Therefore, differentiating with respect to
    consumption and capital, we get FOCs for time $t$ as:
    \begin{align}
      \beta^t u'(c_t) &= \mu p_t^* \label{taxes-hhfoc1}\\
      \mu[(1-\tau_k)r_t^* + p_t^* (1-\delta)]
      &= \mu p_{t-1}^* \label{taxes-hhfoc2}
    \end{align}
    while clearly $h_t=1$.

  \item \emph{Firm FOCs}: These will be the same as in the no taxation
    case:
    \begin{align}
      p^*_t F_1(k_t,h_t) &= r_t^* \label{taxes-firmfoc1}\\
      p^*_t F_2(k_t,h_t) &= w_t^* \notag
    \end{align}
    Notice that capital appears nowhere in those equations.
    We instead have Equation~\ref{taxes-firmfoc1} relating
    \emph{prices}, rather than quantities of capital. This is called a
    \emph{no arbitrage condition}.

  \item \emph{Transversality Condition}:
    \begin{align*}
      \limt \beta^t u'(c_t) k_{t+1}=0
    \end{align*}

  \item \emph{Euler Equation}: As before, we can derive an Euler
    equation
    \begin{align}
      \frac{u'(c_{t-1}^*)}{\beta u'(c_{t}^*)}
      = (1-\tau_k) r_t^* + (1-\delta)
    \end{align}

  \item \emph{Steady State ADCE}: The steady state level of capital
    satisfies
    \begin{align*}
      (1-\tau_k) f'(k^*) &= \frac{1}{\beta} - (1-\delta)
    \end{align*}
    which is decreasing in $\tau_k$.

  \item \emph{Government Waste}: Suppose that the government did not
    lump-sum capital income taxes back to households. Suppose instead
    that the government just buries the output. Then there will be
    changes to equilibrium.

    Specifically, if the government wastes taxes revenue, replace
    sequence $\{T^*_t\}$ with $\{G^*_t\}$, and remove $T^*_t$ terms from
    the household budget constraint, since they never see the tax
    revenue.  Equation~\ref{defn:adce-hh-budget-taxes} becomes
    \begin{align*}
        \sumtinfz p^*_t(c_t + k_{t+1}-(1-\delta)k_t) &\leq
        \sumtinfz ((1-\tau_k)r^*_t k_t + w^*_t h_t)
    \end{align*}
    We also modify the Government Budget Equation~\ref{taxes-govt} and
    Market Clearing Equation~\ref{taxes-clear}
    \begin{align*}
      G^*_t &=  \tau_k r^*_t k^*_t \\
      F(k^*_t,h_t^*) &= c_t^* + k^*_{t+1} - (1-\delta)k_t^* + G^*_t
    \end{align*}
    Of the key equations summarizing the ADCE allocation, only the
    market clearing constraint (which pins down the level of
    consumption) is affected. We see that the government waste reduces
    consumption.
    Notably, however, the consumption Euler equation does not change. In
    other words, households don't change how they allocate cash money
    across time---they just need to reduce their consumption by the
    amount of waste in each period.
\end{enumerate}

\subsection{Sequence of Markets Competetive Equilibrium}
\subsection{Recursive Competetive Equilibrium}

%% APPPENDIX %%

% \appendix




\end{document}


%%%%%%%%%%%%%%%%%%%%%%%%%%%%%%%%%%%%%%%%%%%%%%%%%%%%%%%%%%%%%%%%%%%%%%%%
%%%%%%%%%%%%%%%%%%%%%%%%%%%%%%%%%%%%%%%%%%%%%%%%%%%%%%%%%%%%%%%%%%%%%%%%
%%%%%%%%%%%%%%%%%%%%%%%%%%%%%%%%%%%%%%%%%%%%%%%%%%%%%%%%%%%%%%%%%%%%%%%%

%%%% SAMPLE CODE %%%%%%%%%%%%%%%%%%%%%%%%%%%%%%%%%%%%%%

    %% VIEW LAYOUT %%

        \layout

    %% LANDSCAPE PAGE %%

        \begin{landscape}
        \end{landscape}

    %% BIBLIOGRAPHIES %%

        \cite{LabelInSourcesFile}  %Use in text; cites
        \citep{LabelInSourcesFile} %Use in text; cites in parens

        \nocite{LabelInSourceFile} % Includes in refs w/o specific citation
        \bibliographystyle{apalike}  % Or some other style

        % To ditch the ``References'' header
        \begingroup
        \renewcommand{\section}[2]{}
        \endgroup

        \bibliography{sources} % where sources.bib has all the citation info

    %% SPACING %%

        \vspace{1in}
        \hspace{1in}

    %% URLS, EMAIL, AND LOCAL FILES %%

      \url{url}
      \href{url}{name}
      \href{mailto:mcocci@raidenlovessusie.com}{name}
      \href{run:/path/to/file.pdf}{name}


    %% INCLUDING PDF PAGE %%

        \includepdf{file.pdf}


    %% INCLUDING CODE %%

        %\verbatiminput{file.ext}
            %   Includes verbatim text from the file

        \texttt{text}
            %   Renders text in courier, or code-like, font

        \matlabcode{file.m}
            %   Includes Matlab code with colors and line numbers

        \lstset{style=bash}
        \begin{lstlisting}
        \end{lstlisting}
            % Inline code rendering


    %% INCLUDING FIGURES %%

        % Basic Figure with size scaling
            \begin{figure}[h!]
               \centering
               \includegraphics[scale=1]{file.pdf}
            \end{figure}

        % Basic Figure with specific height
            \begin{figure}[h!]
               \centering
               \includegraphics[height=5in, width=5in]{file.pdf}
            \end{figure}

        % Figure with cropping, where the order for trimming is  L, B, R, T
            \begin{figure}
               \centering
               \includegraphics[trim={1cm, 1cm, 1cm, 1cm}, clip]{file.pdf}
            \end{figure}

        % Side by Side figures: Use the tabular environment


