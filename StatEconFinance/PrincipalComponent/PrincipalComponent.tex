\documentclass[12pt]{article}

\author{Matthew Cocci}
\title{Principal Component Analysis}
\date{\today}

%% Spacing %%%%%%%%%%%%%%%%%%%%%%%%%%%%%%%%%%%%%%%%%%%%%%%%

\usepackage{fullpage}
\usepackage{setspace}
%\onehalfspacing
\usepackage{microtype}


%% Header %%%%%%%%%%%%%%%%%%%%%%%%%%%%%%%%%%%%%%%%%%%%%%%%%

%\pagestyle{fancy} 
%\lhead{}
%\rhead{}
%\chead{}
%\setlength{\headheight}{15.2pt} 
    %---Make the header bigger to avoid overlap

%\renewcommand{\headrulewidth}{0.3pt} 
    %---Width of the line

%\setlength{\headsep}{0.2in}    
    %---Distance from line to text
            

%% Mathematics Related %%%%%%%%%%%%%%%%%%%%%%%%%%%%%%%%%%%

\usepackage{amsmath}
\usepackage{amsfonts}
\usepackage{amsthm} %allows for labeling of theorems
\newtheorem{thm}{Theorem}[section]
\newtheorem{lem}[thm]{Lemma}
\newtheorem{prop}[thm]{Proposition}
\newtheorem{cor}[thm]{Corollary}
%\numberwithin{equation}{section} 
    %---This labels the equations in relation to the sections 
    %---rather than other equations
%\numberwithin{equation}{subsection} 
    %---This labels relative to subsections


%% Font Choices %%%%%%%%%%%%%%%%%%%%%%%%%%%%%%%%%%%%%%%%%

\usepackage[T1]{fontenc}
\usepackage[utf8]{inputenc}
%\usepackage{blindtext}


%% Figures %%%%%%%%%%%%%%%%%%%%%%%%%%%%%%%%%%%%%%%%%%%%%%

\usepackage{graphicx}
\usepackage{subfigure} 
    %---For plotting multiple figures at once
%\graphicspath{ {Directory/} }
    %---Set a directory for where to look for figures


%% Hyperlinks %%%%%%%%%%%%%%%%%%%%%%%%%%%%%%%%%%%%%%%%%%%%
\usepackage{hyperref} 
\hypersetup{	
    colorlinks,		
        %---This colors the links themselves, not boxes
    citecolor=black,	
        %---Everything here and below changes link colors
    filecolor=black,
    linkcolor=black,
    urlcolor=black
}

%% Including Code %%%%%%%%%%%%%%%%%%%%%%%%%%%%%%%%%%%%%%% 

\usepackage{verbatim} 
    %---For including verbatim code from files, no colors

\usepackage{listings}
\usepackage{color}
\definecolor{mygreen}{RGB}{28,172,0}
\definecolor{mylilas}{RGB}{170,55,241}
\newcommand{\matlabcode}[1]{%
    \lstset{language=Matlab,%
        basicstyle=\footnotesize,%
        breaklines=true,%
        morekeywords={matlab2tikz},%
        keywordstyle=\color{blue},%
        morekeywords=[2]{1}, keywordstyle=[2]{\color{black}},%
        identifierstyle=\color{black},%
        stringstyle=\color{mylilas},%
        commentstyle=\color{mygreen},%
        showstringspaces=false,%
            %---Without this there will be a symbol in 
            %---the places where there is a space
        numbers=left,%
        numberstyle={\tiny \color{black}},% 
            %---Size of the numbers
        numbersep=9pt,% 
            %---Defines how far the numbers are from the text
        emph=[1]{for,end,break,switch,case},emphstyle=[1]\color{red},%
            %---Some words to emphasise
    }%
    \lstinputlisting{#1}
}
    %---For including Matlab code from .m file with colors,
    %---line numbering, etc. 


%% Misc %%%%%%%%%%%%%%%%%%%%%%%%%%%%%%%%%%%%%%%%%%%%%% 

\usepackage{enumitem} 
    %---Has to do with enumeration	
\usepackage{appendix}
%\usepackage{natbib} 
    %---For bibliographies
\usepackage{pdfpages}
    %---For including whole pdf pages as a page in doc


%% User Defined %%%%%%%%%%%%%%%%%%%%%%%%%%%%%%%%%%%%%%%%%% 

%\newcommand{\nameofcmd}{Text to display}



%%%%%%%%%%%%%%%%%%%%%%%%%%%%%%%%%%%%%%%%%%%%%%%%%%%%%%%%%%%%%%%%%%%%%%%% 
%% BODY %%%%%%%%%%%%%%%%%%%%%%%%%%%%%%%%%%%%%%%%%%%%%%%%%%%%%%%%%%%%%%%%
%%%%%%%%%%%%%%%%%%%%%%%%%%%%%%%%%%%%%%%%%%%%%%%%%%%%%%%%%%%%%%%%%%%%%%%% 


\begin{document}
\maketitle

%\tableofcontents 

\section{Overview}

Principal Component Analysis is a type of \emph{feature extraction}
that takes a number of ``features'' or independent variables, then 
uses them in combination to build and rank new variables which 
explain the largest share of variability in some response (or
dependent) variable. This differs from \emph{feature selection}
which takes a list of features and selects a subset to use as is
(free of any combination or modification).

Principal Component Analysis can also be used to combat the 
\emph{curse of dimensionality}.  Loosely speaking, this technical
problem arises when you have many features or independent variables
that might explain some dependent variable. As the number of 
features grows, the data becomes ever sparser in a higher 
dimensional space.\footnote{Say that an outcome $y$ is dependent
upon the roll of $N$ dice, $\{x_1, \ldots, x_N\}$. As $N$ increases, 
there are exponentially more outcomes that might 
result--$6^N$ to be precise. So as $N$ increases, the space within 
which the data lives grows exponentially, which necessitates 
exponentially larger samples to match the explanatory power you
would have in dimensions with lower $N$.}
Principal Component Analysis can be used to combine features, rank
them in order or importance, and throw out features that don't
explain much variance in the dependent variable.


\section{Intuition}

Principal Component Analysis really just finds new ways
to look at the data by changing the axes.  Imagine we had
the following data.

PCA highlights the fact that there is nothing sacred
about the axes I have drawn there.  We can express the
data points relative to a new set of axes so that the
variation along different dimensions is more explicit. In
linear algebra terminology, we can use another \emph{basis}.

\section{Implementation}




%% APPPENDIX %%

% \appendix




\end{document}


%%%%%%%%%%%%%%%%%%%%%%%%%%%%%%%%%%%%%%%%%%%%%%%%%%%%%%%%%%%%%%%%%%%%%%%% 
%%%%%%%%%%%%%%%%%%%%%%%%%%%%%%%%%%%%%%%%%%%%%%%%%%%%%%%%%%%%%%%%%%%%%%%%
%%%%%%%%%%%%%%%%%%%%%%%%%%%%%%%%%%%%%%%%%%%%%%%%%%%%%%%%%%%%%%%%%%%%%%%% 

%%%% SAMPLE CODE %%%%%%%%%%%%%%%%%%%%%%%%%%%%%%%%%%%%%%

    %% BIBLIOGRAPHIES %%

        \cite{LabelInSourcesFile} 
        \citep{LabelInSourcesFile} Cites in parens
        \nocite{LabelInSourceFile} includes in refs w/o specific citation
        \bibliographystyle{apalike} 
        \bibliography{sources.bib} where sources.bib is file

    %% SPACING %%

        \vspace{1in}
        \hspace{1in}


    %% INCLUDING PDF PAGE %%

        \includepdf{file.pdf}


    %% INCLUDING CODE %%

        \verbatiminput{file.ext}    
            %---Includes verbatim text from the file
        \texttt{text}	  
            %---Renders text in courier, or code-like, font

        \matlabcode{file.m}	  
            %---Includes Matlab code with colors and line numbers


    %% INCLUDING FIGURES %%

        % Basic Figure with size scaling
            \begin{figure}[h!]
               \centering
               \includegraphics[scale=1]{file.pdf}
            \end{figure}

        % Basic Figure with specific height
            \begin{figure}[h!]
               \centering
               \includegraphics[height=5in, width=5in]{file.pdf}
            \end{figure}

        % Figure with cropping, where the order for trimming is  L, B, R, T
            \begin{figure}
               \centering
               \includegraphics[trim={1cm, 1cm, 1cm, 1cm}, clip]{file.pdf}
            \end{figure}


        % Side by Side figures
            \begin{figure}[h!]
                \centering
                \mbox{\subfigure{
                    \includegraphics[scale=1]{file1.pdf}
                }\quad\subfigure{
                    \includegraphics[scale=1]{file2.pdf} 
                }
                }
            \end{figure}
    

