\documentclass[a4paper,12pt]{scrartcl}

\author{Matt Cocci}
\title{Convergence}
\date{}
\usepackage{enumitem} %Has to do with enumeration	
\usepackage{amsfonts}
\usepackage{amsmath}
\usepackage{amsthm} %allows for labeling of theorems
\usepackage[T1]{fontenc}
\usepackage[utf8]{inputenc}
\usepackage{blindtext}
\usepackage{graphicx}
%\numberwithin{equation}{section} 
%, This labels the equations in relation to the sections rather than other equations
%\numberwithin{equation}{subsection} %This labels relative to subsections
\newtheorem{thm}{Theorem}[section]
\newtheorem{lem}[thm]{Lemma}
\newtheorem{prop}[thm]{Proposition}
\newtheorem{cor}[thm]{Corollary}
\setkomafont{disposition}{\normalfont\bfseries}



\begin{document}
\begin{center}
   \LARGE
   \textbf{Convergence}
\end{center}

\section{Definitions}

Suppose we are working on a probability space $(\Omega, \mathcal{F},P)$
and we consider random variables $\{ X_n \}$, where $n=1,2,\ldots$ 
each with distributions functions $F_n$:

\paragraph{Almost Surely} Used in the \emph{Strong Law of Large Numbers}.
Defined
\[ P\left(\lim_{n\rightarrow\infty} X_n = X\right) =1.\]


\paragraph{In Probability} Note, this is the type of convergence established by the \emph{Weak Law of Large Numbers}.
\[ lim_{n\rightarrow\infty} 
P\left(|X_n - X| > \epsilon\right) = 0, \qquad \forall \epsilon \]

\paragraph{In $p$-norm} All $X_n$ and $X$ have finite $p$th moment
and 
\[ \lim_{n\rightarrow\infty} E\left[|X_n - X|^p\right] = 0, \qquad
   0<p<\infty \]

\paragraph{In Distribution} We say $X_n$ with distributions $F_n$ 
converge \emph{In Distribution} to $X$ with distribution $F$ if
   \[ \lim_{n\rightarrow\infty} F_n(x) = F(x) \]
for all $x \in \mathbb{R}$ at which $F$ is continuous. Also called
\textbf{Weak Convergence}.

\section{Relationships}

The concepts just defined are related in the following way:
\begin{itemize}
   \item[-]{Almost surely $\Rightarrow$ In Probability.}
   \item[-]{In Probability $\Rightarrow$ there's a deterministic 
      subsequence that converges Almost Surely.}
   \item[-]{In $p$-norm $\Rightarrow$ In Probability.}
   \item[-]{Almost Surely and In $p$-norm, undecidable.}
   \item[-]{Almost Surely, In Probability, and In $p$-norm each
      $\Rightarrow$ In Distribution.}
\end{itemize}

\section{Related Concepts}

\paragraph{Consistency} A sequence of estimators $\{ \hat{\theta}_n \}$
where $n=1,2,\ldots$ is \emph{consistent} for parameter $\theta$ if
$\hat{\theta}_n$ converges \emph{In Probability} to $\theta$.

\paragraph{Strongly Consistent} If convergence of $\hat{\theta}_n$
to $\theta$ holds with probability 1.



\end{document}

