\documentclass[a4paper,12pt]{scrartcl}

\author{Matthew Cocci}
\title{Kalman Filter}
\date{\today}
\usepackage{enumitem} %Has to do with enumeration	
\usepackage{amsfonts}
\usepackage{amsmath}
\usepackage{amsthm} %allows for labeling of theorems
\usepackage[T1]{fontenc}
\usepackage[utf8]{inputenc}
\usepackage{blindtext}
\usepackage{graphicx}
%\usepackage[hidelinks]{hyperref} % For internal/external linking. 
				 % [hidelinks] removes boxes
% \usepackage{url} % allows for url display, non-clickable
%\numberwithin{equation}{section} 
   % This labels the equations in relation to the sections 
      % rather than other equations
%\numberwithin{equation}{subsection} %This labels relative to subsections
\newtheorem{thm}{Theorem}[section]
\newtheorem{lem}[thm]{Lemma}
\newtheorem{prop}[thm]{Proposition}
\newtheorem{cor}[thm]{Corollary}
\setkomafont{disposition}{\normalfont\bfseries}
\usepackage{appendix}
\usepackage{subfigure} % For plotting multiple figures at once
\usepackage{verbatim} % for including verbatim code from a file
\usepackage{natbib} % for bibliographies

\begin{document}
\maketitle

% \tableofcontents %adds it here

\section{Basic Idea and Terminology}

Here's the basic procedure associated with the Kalman
Filter:
\begin{enumerate}
    \item Start with a prior for some variable of interest
	in the current period, $p(x)$.
    \item Observe the current measurement $y_t$.
    \item ``Filter'' out the noise and 
	compute the filtering distribution: $p_t(x | y)$.
    \item Compute the predictive distribution $p_{t+1}(x)$
	from the filtering distribution and your model.
    \item Increment $t$ by one, and go back to step 1, taking
	the predictive distribution as your prior.

\end{enumerate}

\section{Example}

Suppose we want to measure some variable $x$. We will 
assume a prior that is multivariate normal such that
    \[ x \sim \text{N}(\hat{x}, \Sigma) \]
Next, we ``measure'' $x$ by matching it to an observable in
a measurement equation:
    \[ y = G x + v \qquad v\sim \text{N}(0, R) \]
where $R$ is positive definite, while $G$ and $R$ are both 
$2 \times 2$. 
\\
\\
We then ``filter'' out the noise, updating our view of $x$ in
light of the data in the filtering step using Bayes' Rule:






%%%% APPPENDIX %%%%%%%%%%%

% \appendix

%\cite{LabelInSourcesFile} 
%\citep{LabelInSourcesFile} Cites in parens
%\nocite{LabelInSourceFile} includes in refs w/o specific citation
%\bibliographystyle{apalike} 
%\bibliography{sources.bib} where sources.bib is file




\end{document}



%%%% INCLUDING FIGURES %%%%%%%%%%%%%%%%%%%%%%%%%%%%

   % H indicates here 
   %\begin{figure}[h!]
   %   \centering
   %   \includegraphics[scale=1]{file.pdf}
   %\end{figure}

%   \begin{figure}[h!]
%      \centering
%      \mbox{
%	 \subfigure{
%	    \includegraphics[scale=1]{file1.pdf}
%	 }\quad
%	 \subfigure{
%	    \includegraphics[scale=1]{file2.pdf} 
%	 }
%      }
%   \end{figure}
 

%%%%% Including Code %%%%%%%%%%%%%%%%%%%%%5
% \verbatiminput{file.ext}    % Includes verbatim text from the file
% \texttt{text}	  % includes text in courier, or code-like, font
