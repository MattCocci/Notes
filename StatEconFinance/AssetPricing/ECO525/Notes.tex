\documentclass[12pt]{article}

\author{Matthew D. Cocci}
\title{ECO-525: Notes to Second Half}
\date{\today}

%% Formatting & Spacing %%%%%%%%%%%%%%%%%%%%%%%%%%%%%%%%%%%%

%\usepackage[top=1in, bottom=1in, left=1in, right=1in]{geometry} % most detailed page formatting control
\usepackage{fullpage} % Simpler than using the geometry package; std effect
\usepackage{setspace}
%\onehalfspacing
\usepackage{microtype}

%% Formatting %%%%%%%%%%%%%%%%%%%%%%%%%%%%%%%%%%%%%%%%%%%%%

%\usepackage[margin=1in]{geometry}
    %   Adjust the margins with geometry package
%\usepackage{pdflscape}
    %   Allows landscape pages
%\usepackage{layout}
    %   Allows plotting of picture of formatting



%% Header %%%%%%%%%%%%%%%%%%%%%%%%%%%%%%%%%%%%%%%%%%%%%%%%%

%\usepackage{fancyhdr}
%\pagestyle{fancy}
%\lhead{}
%\rhead{}
%\chead{}
%\setlength{\headheight}{15.2pt}
    %   Make the header bigger to avoid overlap

%\fancyhf{}
    %   Erase header settings

%\renewcommand{\headrulewidth}{0.3pt}
    %   Width of the line

%\setlength{\headsep}{0.2in}
    %   Distance from line to text


%% Mathematics Related %%%%%%%%%%%%%%%%%%%%%%%%%%%%%%%%%%%

\usepackage{amsmath}
\usepackage{amssymb}
\usepackage{amsfonts}
\usepackage{mathrsfs}
\usepackage{mathtools}
\usepackage{amsthm} %allows for labeling of theorems
%\numberwithin{equation}{section} % Number equations by section
\usepackage{bbm} % For bold numbers

\theoremstyle{plain}
\newtheorem{thm}{Theorem}[section]
\newtheorem{lem}[thm]{Lemma}
\newtheorem{prop}[thm]{Proposition}
\newtheorem{cor}[thm]{Corollary}

\theoremstyle{definition}
\newtheorem{defn}[thm]{Definition}
\newtheorem{ex}[thm]{Example}

\theoremstyle{remark}
\newtheorem*{rmk}{Remark}
\newtheorem*{note}{Note}

% Below supports left-right alignment in matrices so the negative
% signs don't look bad
\makeatletter
\renewcommand*\env@matrix[1][c]{\hskip -\arraycolsep
  \let\@ifnextchar\new@ifnextchar
  \array{*\c@MaxMatrixCols #1}}
\makeatother


%% Font Choices %%%%%%%%%%%%%%%%%%%%%%%%%%%%%%%%%%%%%%%%%

\usepackage[T1]{fontenc}
\usepackage{lmodern}
\usepackage[utf8]{inputenc}
%\usepackage{blindtext}
\usepackage{courier}


%% Figures %%%%%%%%%%%%%%%%%%%%%%%%%%%%%%%%%%%%%%%%%%%%%%

\usepackage{tikz}
\usetikzlibrary{decorations.pathreplacing}
%\usetikzlibrary{arrows.meta}
\usepackage{graphicx}
\usepackage{subfigure}
    %   For plotting multiple figures at once
%\graphicspath{ {Directory/} }
    %   Set a directory for where to look for figures


%% Hyperlinks %%%%%%%%%%%%%%%%%%%%%%%%%%%%%%%%%%%%%%%%%%%%
\usepackage{hyperref}
\hypersetup{%
    colorlinks,
        %   This colors the links themselves, not boxes
    citecolor=black,
        %   Everything here and below changes link colors
    filecolor=black,
    linkcolor=black,
    urlcolor=black
}

%% Colors %%%%%%%%%%%%%%%%%%%%%%%%%%%%%%%%%%%%%%%%%%%%%%%

\usepackage{color}
\definecolor{codegreen}{RGB}{28,172,0}
\definecolor{codelilas}{RGB}{170,55,241}

% David4 color scheme
\definecolor{d4blue}{RGB}{100,191,255}
\definecolor{d4gray}{RGB}{175,175,175}
\definecolor{d4black}{RGB}{85,85,85}
\definecolor{d4orange}{RGB}{255,150,100}

%% Including Code %%%%%%%%%%%%%%%%%%%%%%%%%%%%%%%%%%%%%%%

\usepackage{verbatim}
    %   For including verbatim code from files, no colors
\usepackage{listings}
    %   For including code snippets written directly in this doc

\lstdefinestyle{bash}{%
  language=bash,%
  basicstyle=\footnotesize\ttfamily,%
  showstringspaces=false,%
  commentstyle=\color{gray},%
  keywordstyle=\color{blue},%
  xleftmargin=0.25in,%
  xrightmargin=0.25in
}
\lstdefinestyle{log}{%
  basicstyle=\scriptsize\ttfamily,%
  showstringspaces=false,%
  xleftmargin=0.25in,%
  xrightmargin=0.25in
}


\lstdefinestyle{matlab}{%
  language=Matlab,%
  basicstyle=\footnotesize\ttfamily,%
  breaklines=true,%
  morekeywords={matlab2tikz},%
  keywordstyle=\color{blue},%
  morekeywords=[2]{1}, keywordstyle=[2]{\color{black}},%
  identifierstyle=\color{black},%
  stringstyle=\color{codelilas},%
  commentstyle=\color{codegreen},%
  showstringspaces=false,%
    %   Without this there will be a symbol in
    %   the places where there is a space
  %numbers=left,%
  %numberstyle={\tiny \color{black}},%
    %   Size of the numbers
  numbersep=9pt,%
    %   Defines how far the numbers are from the text
  emph=[1]{for,end,break,switch,case},emphstyle=[1]\color{blue},%
    %   Some words to emphasise
}

\newcommand{\matlabcode}[1]{%
    \lstset{style=matlab}%
    \lstinputlisting{#1}
}
    %   For including Matlab code from .m file with colors,
    %   line numbering, etc.

\lstdefinelanguage{Julia}%
  {morekeywords={abstract,break,case,catch,const,continue,do,else,elseif,%
      end,export,false,for,function,immutable,import,importall,if,in,%
      macro,module,otherwise,quote,return,switch,true,try,type,typealias,%
      using,while},%
   sensitive=true,%
   %alsoother={$},%
   morecomment=[l]\#,%
   morecomment=[n]{\#=}{=\#},%
   morestring=[s]{"}{"},%
   morestring=[m]{'}{'},%
}[keywords,comments,strings]

\lstdefinestyle{julia}{%
    language         = Julia,
    basicstyle       = \scriptsize\ttfamily,
    keywordstyle     = \bfseries\color{blue},
    stringstyle      = \color{codegreen},
    commentstyle     = \color{codegreen},
    showstringspaces = false,
    literate         = %
      {ρ}{{$\rho$}}1
      {ℓ}{{$\ell$}}1
      {∑}{{$\Sigma$}}1
      {Σ}{{$\Sigma$}}1
      {√}{{$\sqrt{}$}}1
      {θ}{{$\theta$}}1
      {ω}{{$\omega$}}1
      {ɛ}{{$\varepsilon$}}1
      {φ}{{$\varphi$}}1
      {σ²}{{$\sigma^2$}}1
      {Φ}{{$\Phi$}}1
      {ϕ}{{$\phi$}}1
      {Dₑ}{{$D_e$}}1
      {Σ}{{$\Sigma$}}1
      {γ}{{$\gamma$}}1
      {δ}{{$\delta$}}1
      {τ}{{$\tau$}}1
      {μ}{{$\mu$}}1
      {β}{{$\beta$}}1
      {Λ}{{$\Lambda$}}1
      {λ}{{$\lambda$}}1
      {r̃}{{$\tilde{\text{r}}$}}1
      {α}{{$\alpha$}}1
      {σ}{{$\sigma$}}1
      {π}{{$\pi$}}1
      {∈}{{$\in$}}1
      {∞}{{$\infty$}}1
}


%% Bibliographies %%%%%%%%%%%%%%%%%%%%%%%%%%%%%%%%%%%%

%\usepackage{natbib}
    %---For bibliographies
%\setlength{\bibsep}{3pt} % Set how far apart bibentries are

%% Misc %%%%%%%%%%%%%%%%%%%%%%%%%%%%%%%%%%%%%%%%%%%%%%

\usepackage{enumitem}
    %   Has to do with enumeration
\usepackage{appendix}
%\usepackage{natbib}
    %   For bibliographies
\usepackage{pdfpages}
    %   For including whole pdf pages as a page in doc
\usepackage{pgffor}
    %   For easier looping


%% User Defined %%%%%%%%%%%%%%%%%%%%%%%%%%%%%%%%%%%%%%%%%%

%\newcommand{\nameofcmd}{Text to display}
\newcommand*{\Chi}{\mbox{\large$\chi$}} %big chi
    %   Bigger Chi

% In math mode, Use this instead of \munderbar, since that changes the
% font from math to regular
\makeatletter
\def\munderbar#1{\underline{\sbox\tw@{$#1$}\dp\tw@\z@\box\tw@}}
\makeatother

% Misc Math
\newcommand{\ra}{\rightarrow}
\newcommand{\diag}{\text{diag}}
\newcommand{\ch}{\text{ch}}
\newcommand{\dom}{\text{dom}}
\newcommand{\one}[1]{\mathbf{1}_{#1}}


% Command to generate new math commands:
% - Suppose you want to refer to \boldsymbol{x} as just \bsx, where 'x'
%   is any letter. This commands lets you generate \bsa, \bsb, etc.
%   without copy pasting \newcommand{\bsa}{\boldsymbol{a}} for each
%   letter individually. Instead, just include
%
%     \generate{bs}{\boldsymbol}{a,...,z}
%
% - Uses pgffor package to loop
% - Example with optional argument. Will generate \bshatx to represent
%   \boldsymbol{\hat{x}} for all letters x
%
%     \generate[\hat]{bshat}{\boldsymbol}{a,...,z}

\newcommand{\generate}[4][]{%
  % Takes 3 arguments (maybe four):
  % - 1   wrapcmd (optional, defaults to nothing)
  % - 2   newname
  % - 3   mathmacro
  % - 4   Names to loop over
  %
  % Will produce
  %
  %   \newcommand{\newnameX}{mathmacro{wrapcmd{X}}}
  %
  % for each X in argument 4

  \foreach \x in {#4}{%
    \expandafter\xdef\csname%
      #2\x%
    \endcsname%
    {\noexpand\ensuremath{\noexpand#3{\noexpand#1{\x}}}}
  }
}


% MATHSCR: Gen \sX to stand for \mathscr{X} for all upper case letters
\generate{s}{\mathscr}{A,...,Z}


% BOLDSYMBOL: Generate \bsX to stand for \boldsymbol{X}, all upper and
% lower case.
%
% Letters and greek letters
\generate{bs}{\boldsymbol}{a,...,z}
\generate{bs}{\boldsymbol}{A,...,Z}
\newcommand{\bstheta}{\boldsymbol{\theta}}
\newcommand{\bsmu}{\boldsymbol{\mu}}
\newcommand{\bsSigma}{\boldsymbol{\Sigma}}
\newcommand{\bsvarepsilon}{\boldsymbol{\varepsilon}}
\newcommand{\bsalpha}{\boldsymbol{\alpha}}
\newcommand{\bsbeta}{\boldsymbol{\beta}}
\newcommand{\bsOmega}{\boldsymbol{\Omega}}
\newcommand{\bshatOmega}{\boldsymbol{\hat{\Omega}}}
\newcommand{\bshatG}{\boldsymbol{\hat{G}}}
\newcommand{\bsgamma}{\boldsymbol{\gamma}}
\newcommand{\bslambda}{\boldsymbol{\lambda}}

% Special cases like \bshatb for \boldsymbol{\hat{b}}
\generate[\hat]{bshat}{\boldsymbol}{b,y,x,X,V,S,W}
\newcommand{\bshatbeta}{\boldsymbol{\hat{\beta}}}
\newcommand{\bshatmu}{\boldsymbol{\hat{\mu}}}
\newcommand{\bshattheta}{\boldsymbol{\hat{\theta}}}
\newcommand{\bshatSigma}{\boldsymbol{\hat{\Sigma}}}
\newcommand{\bstildebeta}{\boldsymbol{\tilde{\beta}}}
\newcommand{\bstildetheta}{\boldsymbol{\tilde{\theta}}}
\newcommand{\bsbarbeta}{\boldsymbol{\overline{\beta}}}
\newcommand{\bsbarg}{\boldsymbol{\overline{g}}}

% Redefine \bso to be the zero vector
\renewcommand{\bso}{\boldsymbol{0}}

% Transposes of all the boldsymbol shit
\newcommand{\bsbp}{\boldsymbol{b'}}
\newcommand{\bshatbp}{\boldsymbol{\hat{b'}}}
\newcommand{\bsdp}{\boldsymbol{d'}}
\newcommand{\bsgp}{\boldsymbol{g'}}
\newcommand{\bsGp}{\boldsymbol{G'}}
\newcommand{\bshp}{\boldsymbol{h'}}
\newcommand{\bsSp}{\boldsymbol{S'}}
\newcommand{\bsup}{\boldsymbol{u'}}
\newcommand{\bsxp}{\boldsymbol{x'}}
\newcommand{\bsyp}{\boldsymbol{y'}}
\newcommand{\bsthetap}{\boldsymbol{\theta'}}
\newcommand{\bsmup}{\boldsymbol{\mu'}}
\newcommand{\bsSigmap}{\boldsymbol{\Sigma'}}
\newcommand{\bshatmup}{\boldsymbol{\hat{\mu'}}}
\newcommand{\bshatSigmap}{\boldsymbol{\hat{\Sigma'}}}

% MATHCAL: Gen \calX to stand for \mathcal{X}, all upper case
\generate{cal}{\mathcal}{A,...,Z}

% MATHBB: Gen \X to stand for \mathbb{X} for some upper case
\generate{}{\mathbb}{R,Q,C,Z,N,Z,E}
\newcommand{\Rn}{\mathbb{R}^n}
\newcommand{\RN}{\mathbb{R}^N}
\newcommand{\Rk}{\mathbb{R}^k}
\newcommand{\RK}{\mathbb{R}^K}
\newcommand{\RL}{\mathbb{R}^L}
\newcommand{\Rl}{\mathbb{R}^\ell}
\newcommand{\Rm}{\mathbb{R}^m}
\newcommand{\Rnn}{\mathbb{R}^{n\times n}}
\newcommand{\Rmn}{\mathbb{R}^{m\times n}}
\newcommand{\Rnm}{\mathbb{R}^{n\times m}}
\newcommand{\Rkn}{\mathbb{R}^{k\times n}}
\newcommand{\Cn}{\mathbb{C}^n}
\newcommand{\Cnn}{\mathbb{C}^{n\times n}}

% Dot over
\newcommand{\dx}{\dot{x}}
\newcommand{\ddx}{\ddot{x}}
\newcommand{\dy}{\dot{y}}
\newcommand{\ddy}{\ddot{y}}

% First derivatives
\newcommand{\dydx}{\frac{dy}{dx}}
\newcommand{\dfdx}{\frac{df}{dx}}
\newcommand{\dfdy}{\frac{df}{dy}}
\newcommand{\dfdz}{\frac{df}{dz}}

% Second derivatives
\newcommand{\ddyddx}{\frac{d^2y}{dx^2}}
\newcommand{\ddydxdy}{\frac{d^2y}{dx dy}}
\newcommand{\ddydydx}{\frac{d^2y}{dy dx}}
\newcommand{\ddfddx}{\frac{d^2f}{dx^2}}
\newcommand{\ddfddy}{\frac{d^2f}{dy^2}}
\newcommand{\ddfddz}{\frac{d^2f}{dz^2}}
\newcommand{\ddfdxdy}{\frac{d^2f}{dx dy}}
\newcommand{\ddfdydx}{\frac{d^2f}{dy dx}}


% First Partial Derivatives
\newcommand{\pypx}{\frac{\partial y}{\partial x}}
\newcommand{\pfpx}{\frac{\partial f}{\partial x}}
\newcommand{\pfpy}{\frac{\partial f}{\partial y}}
\newcommand{\pfpz}{\frac{\partial f}{\partial z}}


% argmin and argmax
\DeclareMathOperator*{\argmin}{arg\;min}
\DeclareMathOperator*{\argmax}{arg\;max}


% Various probability and statistics commands
\newcommand{\iid}{\overset{iid}{\sim}}
\newcommand{\vc}{\operatorname{vec}}
\newcommand{\Cov}{\operatorname{Cov}}
\newcommand{\rank}{\operatorname{rank}}
\newcommand{\trace}{\operatorname{trace}}
\newcommand{\Corr}{\operatorname{Corr}}
\newcommand{\Var}{\operatorname{Var}}
\newcommand{\asto}{\xrightarrow{a.s.}}
\newcommand{\pto}{\xrightarrow{p}}
\newcommand{\msto}{\xrightarrow{m.s.}}
\newcommand{\dto}{\xrightarrow{d}}
\newcommand{\Lpto}{\xrightarrow{L_p}}
\newcommand{\Lqto}[1]{\xrightarrow{L_{#1}}}
\newcommand{\plim}{\text{plim}_{n\rightarrow\infty}}


% Redefine real and imaginary from fraktur to plain text
\renewcommand{\Re}{\operatorname{Re}}
\renewcommand{\Im}{\operatorname{Im}}

% Shorter sums: ``Sum from X to Y''
% - sumXY  is equivalent to \sum^Y_{X=1}
% - sumXYz is equivalent to \sum^Y_{X=0}
\newcommand{\sumnN}{\sum^N_{n=1}}
\newcommand{\sumin}{\sum^n_{i=1}}
\newcommand{\sumjn}{\sum^n_{j=1}}
\newcommand{\sumim}{\sum^m_{i=1}}
\newcommand{\sumik}{\sum^k_{i=1}}
\newcommand{\sumiN}{\sum^N_{i=1}}
\newcommand{\sumkn}{\sum^n_{k=1}}
\newcommand{\sumtT}{\sum^T_{t=1}}
\newcommand{\sumninf}{\sum^\infty_{n=1}}
\newcommand{\sumtinf}{\sum^\infty_{t=1}}
\newcommand{\sumnNz}{\sum^N_{n=0}}
\newcommand{\suminz}{\sum^n_{i=0}}
\newcommand{\sumknz}{\sum^n_{k=0}}
\newcommand{\sumtTz}{\sum^T_{t=0}}
\newcommand{\sumninfz}{\sum^\infty_{n=0}}
\newcommand{\sumtinfz}{\sum^\infty_{t=0}}

\newcommand{\prodnN}{\prod^N_{n=1}}
\newcommand{\prodin}{\prod^n_{i=1}}
\newcommand{\prodiN}{\prod^N_{i=1}}
\newcommand{\prodkn}{\prod^n_{k=1}}
\newcommand{\prodtT}{\prod^T_{t=1}}
\newcommand{\prodnNz}{\prod^N_{n=0}}
\newcommand{\prodinz}{\prod^n_{i=0}}
\newcommand{\prodknz}{\prod^n_{k=0}}
\newcommand{\prodtTz}{\prod^T_{t=0}}

% Bounds
\newcommand{\atob}{_a^b}
\newcommand{\ztoinf}{_0^\infty}
\newcommand{\kinf}{_{k=1}^\infty}
\newcommand{\ninf}{_{n=1}^\infty}
\newcommand{\minf}{_{m=1}^\infty}
\newcommand{\tinf}{_{t=1}^\infty}
\newcommand{\nN}{_{n=1}^N}
\newcommand{\tT}{_{t=1}^T}
\newcommand{\kinfz}{_{k=0}^\infty}
\newcommand{\ninfz}{_{n=0}^\infty}
\newcommand{\minfz}{_{m=0}^\infty}
\newcommand{\tinfz}{_{t=0}^\infty}
\newcommand{\nNz}{_{n=0}^N}

% Limits
\newcommand{\limN}{\lim_{N\rightarrow\infty}}
\newcommand{\limn}{\lim_{n\rightarrow\infty}}
\newcommand{\limk}{\lim_{k\rightarrow\infty}}
\newcommand{\limt}{\lim_{t\rightarrow\infty}}
\newcommand{\limT}{\lim_{T\rightarrow\infty}}
\newcommand{\limhz}{\lim_{h\rightarrow 0}}

% Shorter integrals: ``Integral from X to Y''
% - intXY is equivalent to \int^Y_X
\newcommand{\intab}{\int_a^b}
\newcommand{\intzN}{\int_0^N}


%%%%%%%%%%%%%%%%%%%%%%%%%%%%%%%%%%%%%%%%%%%%%%%%%%%%%%%%%%%%%%%%%%%%%%%%
%% BODY %%%%%%%%%%%%%%%%%%%%%%%%%%%%%%%%%%%%%%%%%%%%%%%%%%%%%%%%%%%%%%%%
%%%%%%%%%%%%%%%%%%%%%%%%%%%%%%%%%%%%%%%%%%%%%%%%%%%%%%%%%%%%%%%%%%%%%%%%


\begin{document}
\maketitle

\section{Challenges to Representative Agent Theories}

Standard asset pricing framework of Lucas (1978) and Breeden (1979)

\subsection{Review of Rep-Agent Framework (Cochrane)}

\begin{defn}(Household Problem)
There is a risky asset that costs $p_t$ units of the consumption good
and pays dividends at rate $D_t$ at time $t$.
Household solves
\begin{align*}
  \max \E_0\left[\int_0^\infty e^{-\rho t}u(c_t)\;dt\right]
\end{align*}
This has first order conditions
\begin{align}
  p_t u'(c_t) =
  \E_t\left[\int^\infty_0 e^{-\rho s}u'(c_{t+s})D_{t+s}\;ds\right]
  \label{pemx}
\end{align}
In equilibrium with this rep agent, prices adjust until $c_t=e_t$, the
endowment, for all $t$.
\end{defn}

\begin{defn}(Stochastic Discount Factor)
The above equation suggests defining stochastic discount factor in
levels as follows
\begin{align}
  \Lambda_t
  &:= e^{-\rho t} u'(e_t)
  = e^{-\rho t} u'(c_t)
  \label{sdf}
\end{align}
We can get an SDE describing the evolution of this variable in one of
two ways:
\begin{itemize}
  \item Just writing down a process for this SDF
  \item Specifying an SDE for $c_t$ or $e_t$, then deriving an SDE for
    $\Lambda_t$ by Ito's Lemma
\end{itemize}
\end{defn}

\begin{defn}(Fundamental Pricing Equation)
In continuous time,
\begin{align*}
  0 &= \Lambda_t D_t\;dt + \E_t[d(\Lambda_t p_t)]
\end{align*}
This arises by breaking up the expected payoff from the dividend stream
in Expression~\ref{pemx} into the sum of two components: the
instantaneous dividend flow plus the resale value at time $dt$.
Approximate and take limits.
Breaking up $d(\Lambda_t p_t)$ by Ito's Lemma, we can also rewrite as
\begin{align*}
  0 &= \frac{D_t}{p_t}\;dt +
  \E_t\left[
    \frac{d\Lambda_t}{\Lambda_t}
    +
    \frac{dp_t}{p_t}
    +
    \frac{d\Lambda_t}{\Lambda_t}
    \frac{dp_t}{p_t}
  \right]
\end{align*}
\end{defn}

\begin{defn}(Risk Free Rate)
We can think of this as an asset with $p_t=1$ always (i.e. $dp_t=0$)
that pays dividend $r^f_t$, or as an asset with price following
$dp_t/p_t=r^f_t\;dt$ that pays no dividend. In either case, the above
pricing formula gives
\begin{align*}
  r_t^f\;dt
  &=
  -
  E_t\left[
    \frac{d\Lambda_t}{\Lambda_t}
  \right]
\end{align*}
\end{defn}

\begin{defn}(Expected Return Formula)
The second fundamental pricing formula and the risk free rate means we
can again rewrite the fundamental pricing equation---this time as
\begin{align}
  \E_t\left[
    \frac{dp_t}{p_t}
  \right]
  +
  \frac{D_t}{p_t}\;dt
  &=
  r_t^f\;dt
  -
  \E_t\left[
    \frac{d\Lambda_t}{\Lambda_t}
    \frac{dp_t}{p_t}
  \right]
  \label{expR}
\end{align}
This has the expected return on the LHS.
It equals the risk free rate plus a risk premium that equals covariance
of the SDF with the price of the asset.

Now suppose we have processes for the price and the SDF.
For example, suppose
\begin{align*}
  \frac{dp_t}{p_t}
  &= \mu^p \;dt + \sigma_p dZ_t \\
  \frac{d\Lambda_t}{\Lambda_t}
  &= \mu^\Lambda \;dt + \sigma_\Lambda dZ_t
\end{align*}
If the SDF specified above \emph{really does} price the asset with price
process $p_t$, then it \emph{must} be the case that
Expression~\ref{expR} holds.
Substituting in the prices processes, this implies the following
restrictions on parameters/processes
\begin{align*}
  \mu^p\;dt
  +
  \frac{D_t}{p_t}\;dt
  &=
  r_t^f\;dt
  -
  \sigma_p\sigma_\lambda\;dt
  \quad\implies\quad
  -\sigma_p\sigma_\lambda
  =
  \mu^p
  +
  \frac{D_t}{p_t}
  - r_t^f
\end{align*}
Note that the assumed processes for $p_t$ and $\Lambda_t$ were very
particular, i.e. geometric Brownian motions.
However, the point of this was much more general.
Stated again: Whatever processes we assume for $p_t$ and $\Lambda_t$,
Expression~\ref{expR} \emph{must} hold for $\Lambda_t$ to \emph{really}
price the asset. By substituting in the assumed SDEs for $p_t$ and
$\Lambda_t$ (which are often more general that simple GBM), we can
derive equlibrium restrictions on the processes and parameters.
\end{defn}

\begin{defn}(Risk Neutral Pricing)
We can almost always modify the measure so that
\end{defn}




%Single representative agent who consumes their endowment $c_t=e_t$ each
%period, where
%\begin{align}
  %\frac{de_t}{e_t}
  %&= \mu_e\; dt + \sigma_e \;dZ_t
  %\label{endow}
%\end{align}
%Given SDE~\ref{endow} for the endowment, can derive an SDE for the SDF
%$\Lambda_t$ by Ito's Lemma
%\begin{align*}
  %d\Lambda_t
  %&=
  %-\rho \Lambda_t
  %dt
  %+
  %e^{-\rho t}
  %u''(e_t)
  %de_t
  %+ \frac{1}{2}
  %e^{-\rho t}
  %u'''(e_t)
  %(de_t)^2
  %\\
  %&=
  %-\rho \Lambda_t
  %dt
  %+
  %e^{-\rho t}
  %u'(e_t)
  %\frac{u''(e_t)}{u'(e_t)}
  %de_t
  %+ \frac{1}{2}
  %e^{-\rho t}
  %u'(e_t)
  %\frac{u'''(e_t)}{u'(e_t)}
  %(de_t)^2
  %\\
  %&=
  %-\rho \Lambda_t
  %dt
  %+
  %\Lambda_t
  %\frac{e_tu''(e_t)}{u'(e_t)}
  %\frac{de_t}{e_t}
  %de_t
  %+ \frac{1}{2}
  %\Lambda_t
  %\frac{e_t^2u'''(e_t)}{u'(e_t)}
  %\left(\frac{de_t}{e_t}\right)^2
%\end{align*}
%Hence
%\begin{align*}
  %\frac{d\Lambda_t}{\Lambda_t}
  %&=
  %-\rho
  %\; dt
  %+
  %\frac{e_tu''(e_t)}{u'(e_t)}
  %\frac{de_t}{e_t}
  %+ \frac{1}{2}
  %\frac{e_t^2u'''(e_t)}{u'(e_t)}
  %\left(\frac{de_t}{e_t}\right)^2
%\end{align*}
%Sub in for $de_t$ and $(de_t)^2$:
%\begin{align*}
  %\frac{d\Lambda_t}{\Lambda_t}
  %&=
  %-\rho
  %\; dt
  %+
  %\frac{e_tu''(e_t)}{u'(e_t)}
  %\big[
    %\mu_e\;dt + \sigma_e\;dZ_t
  %\big]
  %+ \frac{1}{2}
  %\frac{e_t^2u'''(e_t)}{u'(e_t)}
  %\sigma_e^2 \;dt
  %\\
  %&=
  %\left[
  %-\rho
  %+
  %\frac{e_tu''(e_t)}{u'(e_t)}
  %\mu_e
  %+
  %\frac{1}{2}
  %\frac{e_t^2u'''(e_t)}{u'(e_t)}
  %\sigma_e^2
  %\right] dt
  %+
  %\frac{e_tu''(e_t)}{u'(e_t)}
  %\sigma_e\;dZ_t
%\end{align*}
%Defining relative risk aversion $\gamma(e_t)$, we can rewrite this
%\begin{align*}
  %\frac{d\Lambda_t}{\Lambda_t}
  %&=
  %\left[
  %-\rho
  %-
  %\gamma(e_t)
  %\mu_e
  %+
  %\frac{1}{2}
  %\frac{e_t^2u'''(e_t)}{u'(e_t)}
  %\sigma_e^2
  %\right] dt
  %-
  %\gamma(e_t)
  %\sigma_e\;dZ_t
  %\\
  %\text{where}\quad
  %\gamma(e_t)
  %&=
  %-\frac{e_tu''(e_t)}{u'(e_t)}
%\end{align*}
%Often though, we just write down an SDE for $\Lambda_t$ directly, rather
%than derive it from some endowment process.



\clearpage
\subsection{Puzzles and Anomalies}

List of things:
\begin{itemize}
  \item \emph{Equity Premium Puzzle}:
    Stocks are not really correlated with consumption growth.
    Hence stocks should not really command any premium, but they do.
    Need really high risk aversion to explain.

    Efforts to explain
    \begin{itemize}
      \item Exotic Preferences (Epstein-Zin)
      \item Distaster Risk
      \item Long-run risk
      \item Heterogeneity in agents
    \end{itemize}

  \item
    \emph{Risk Free Rate Puzzle}:
    The risk-free rate is very low. Consistent with low risk aversion.
    This is a direct tension with the equity premium puzzle.

  \item
    \emph{Volatility Puzzle}:
    Need time-varying risk aversion (hence time varying discount rates),
    otherwise can't justify the volatility of stock returns given the
    volatility of dividend returns (Shiller ``excess volatility''
    result).

  \item
    \emph{Time Series Predictability Puzzle}:
    Dividend-price ratio (and other things) forecast subsequent
    \emph{expected} returns. In other words, expected excess returns are
    time-varying.
    The variance of stocks are \emph{also} time-varying, so that's the
    obvious candidate, even though there's not a strong empirical
    relationship.

  \item
    \emph{Cross Sectional Predictability Puzzle}:
    Fama-French factors

  \item
    \emph{Long-Term Reversal}, DeBondt and Thaler (1985):
    Stock's \emph{past} 3-year return \emph{negatively} predicts
    \emph{future} returns.

  \item
    \emph{Medium-Term Momentum}, Jegadeesh and Titman (1993):
    Stock's \emph{past} $k$-month (for $k\in\{1,\ldots,12\}$)
    \emph{positively} predicts \emph{future} returns.

  \item
    \emph{Post-Earnings Announcement Drift}, Bernard and Thomas (1989):
    Surprise in recent earnings announcement \emph{postively}
    predictures future returns.
    In other words, \emph{under-reaction} to initial announcement.
    Differs from momentum, because the drift does \emph{not} mean-revert
    over longer run.

  \item
    \emph{Turnover}:
    There are two separate turnover results
    \begin{itemize}
      \item \emph{Long-Term}, Datar, Naik, Radcliffe (1998):
        High volume over past \emph{year} \emph{negatively} predicts
        future returns
      \item \emph{Short-Term}, Gervais, Kaniel, Mingelgrin (2001):
        High volume over past \emph{week} \emph{positively} predicts
        future returns
    \end{itemize}

  \item
    \emph{Volatility}:
    Two results
    \begin{itemize}
      \item Stocks with high \emph{beta} have lower alphas
      \item Stocks with high \emph{idiosyncratic volatility} have lower alphas
    \end{itemize}
\end{itemize}



\subsection{Historical Bubbles}

Definition: Asset price vastly exceeding fundamentals.
Examples include
\begin{itemize}
  \item Dutch tulip bubble
  \item 1920s in the US
  \item Internet Bubble
  \item Palm/3Com (Lamont and Thaler)
  \item Chinese Warrants
\end{itemize}
Explanations
\begin{itemize}
  \item Mis-measured fundamentals
  \item Rational bubbles: infinite maturity and unbounded prices
  \item Optimism
  \item Speciulation between investors
\end{itemize}


\subsection{Investor Behavior}


Individual Investors:
Datasets are
\begin{itemize}
  \item SCF
  \item Discount brokerage data, Odean (1998)
  \item 401(k) data, Huberman and Jiang (2006)
  \item Finnish data, Grinblatt and Keloharju (2001)
  \item Swedish data, Calvet, Campbell, Sodini (2009)
\end{itemize}
Topics
\begin{itemize}
  \item Non-participation
  \item Under-diversification:
    \begin{itemize}
      \item Home bias
      \item Concentration in a few stocks in the market
      \item Owning stocks in your own company
    \end{itemize}
  \item Excessive trading, Barber and Odean (2000): those who trade most
    do worse
  \item Disposition effect: Agents more likely to sell stocks where
    they've realized a gain since purchase, versus hold stocks where
    they've realized a loss since purchase
\end{itemize}


\clearpage
\section{Noise Traders}

\subsection{Rational Learning Benchmark in Discrete Time}

\paragraph{Belief Evolution}
Three periods, no discounting between periods.
There's some unobserved fundamental value $f$, which will be the
terminal payoff announced in final period.
\begin{itemize}
  \item $t=0$: Set the prior $p(f)\sim\calN(\overline{f},\sigma_f^2)$

  \item $t=1$: Observe signal $s$ that's correlated with $f$
    \begin{align*}
      s &= f + \varepsilon
      \quad
      \text{where}\quad
       \varepsilon\sim\calN(0, \sigma_\varepsilon^2)
      \qquad\implies\qquad
      p(s|f) \sim\calN(f, \sigma_\varepsilon^2)
    \end{align*}
    Given signal, form posterior
    \begin{align*}
      p(f|s)
      &\propto
      p(s|f)
      \cdot p(f)
      =
      \calN\bigg(
        %\underbrace_{\hat{f}}
        ,\;\;
        \frac{1}{\tau_f+\tau_\varepsilon}
      \bigg)
    \end{align*}
    where $\tau_x=1/\sigma^2_x$, i.e. the inverse variance which is
    called the ``precision.''

  \item $t=2$: Fundamental value $f$ realized and agents get payoff
    equal to $f$.
\end{itemize}
Henceforth, let $\calI_t$ denote the information set at time
$t$, i.e. $\calI_0=\emptyset$, $\calI_1=\{s\}$, $\calI_2=\{s,f\}$.
Using this notation, note that a rational agent's beliefs about $f$
follow a martingale:
\begin{align*}
  \E\big[ \E[f|\calI_{t+1}] \;\big|\;\calI_t\big]
  = \E[f|\calI_{t}]
\end{align*}
which follows from $\calI_t\subset \calI_{t+1}$ and the law of iterated
expectations.

\paragraph{Prices}
For a risk a risk neutral agent, $p_t = \E[f|\calI_t]$ so that price
in each period is
\begin{align*}
  p_0
  &=
  \E[f|\calI_0]
  =
  \overline{f}\\
  p_1
  &=
  \E[f|\calI_1]
  =
  \overline{f}
  + \frac{\tau_\varepsilon}{\tau_f+\tau_\varepsilon}
  (s-\overline{f})
  \\
  p_2
  &=
  \E[f|\calI_2]
  =
  f
\end{align*}

\paragraph{Price Dynamics}
Under risk neutrality, prices follow a martingale:
\begin{align*}
  \E[p_{t+1}|\calI_t]
  = \E\big[\E[f|\calI_{t+1}]|\calI_t\big]
  = \E[f|\calI_t]
  = p_t
\end{align*}
The first equality used risk neutrality, which implies implies
$p_{t+1}=\E[f|\calI_{t+1}]$.  The second equality applied the law of
iterated expectations, which was valid since $\calI_t\subset
\calI_{t+1}$.  Finally, the last equality again used risk neutrality,
which implies $p_{t}=\E[f|\calI_{t}]$.
And $\E[p_{t+1}|\calI_t]=p_t$ is precisely the definition of a
martingale.

The fact that prices follow a martingale implies that price changes have
unconditional \emph{expectation} of zero:
\begin{align*}
  \forall t
  \qquad
  \E[p_{t+1}-p_t]
  =
  \E[p_{t+1}-p_t|\calI_0]
  =
  p_0 - p_0
  = 0
\end{align*}
What about the unconditional \emph{variance} of price changes?
Use the price expressions we derived above to compute
\begin{align*}
  \Var(p_1-p_0)
  &=
  \Var\left(
  \left[
  \overline{f}
  + \frac{\tau_\varepsilon}{\tau_f+\tau_\varepsilon}
  (s-\overline{f})
  \right]
  -\overline{f}
  \right)
  \\
  &=
  \left(
  \frac{\tau_\varepsilon}{\tau_f+\tau_\varepsilon}
  \right)^2
  \Var(s)
  =
  \left(
  \frac{\tau_\varepsilon}{\tau_f+\tau_\varepsilon}
  \right)^2
  \Var(f+\varepsilon)
  \\
  \text{Since $\varepsilon\perp f$}\qquad
  &=
  \left(
  \frac{\tau_\varepsilon}{\tau_f+\tau_\varepsilon}
  \right)^2
  \big[
    \sigma^2_f
    +
    \sigma^2_\varepsilon
  \big]
  =
  \left(
  \frac{\tau_\varepsilon}{\tau_f+\tau_\varepsilon}
  \right)^2
  \left[
    \frac{1}{\tau_f}
    +
    \frac{1}{\tau_\varepsilon}
  \right]
  \\
  \implies\quad
  \Var(p_1-p_0)
  &=
  \frac{\tau_\varepsilon}{\tau_f(\tau_f+\tau_\varepsilon)}
\end{align*}
Next, compute
\begin{align*}
  \Var(p_2-p_1)
  &=
  \Var\left(
    f
    - \left[
    \overline{f}
    + \frac{\tau_\varepsilon}{\tau_f+\tau_\varepsilon}
    (s-\overline{f})
    \right]
  \right)
  \\
  &=
  \Var\left(
    f
    - \frac{\tau_\varepsilon}{\tau_f+\tau_\varepsilon} s
  \right)
  =
  \Var\left(
    f
    - \frac{\tau_\varepsilon}{\tau_f+\tau_\varepsilon} (f+\varepsilon)
  \right)
  \\
  &=
  \Var\left(
   \frac{\tau_f}{\tau_f+\tau_\varepsilon} f
    - \frac{\tau_\varepsilon}{\tau_f+\tau_\varepsilon} \varepsilon
  \right)
  \\
  \text{Since $\varepsilon\perp f$}\qquad
  &=
  \left(\frac{\tau_f}{\tau_f+\tau_\varepsilon}\right)^2
  \Var\left(
    f
  \right)
  +
  \left(\frac{\tau_\varepsilon}{\tau_f+\tau_\varepsilon}\right)^2
  \Var\left(
  \varepsilon
  \right)
  %\\
  %&=
  %\left(\frac{\tau_f}{\tau_f+\tau_\varepsilon}\right)^2
  %\sigma^2_f
  %+
  %\left(\frac{\tau_\varepsilon}{\tau_f+\tau_\varepsilon}\right)^2
  %\sigma^2_\varepsilon
  \\
  &=
  \frac{\tau_f}{\left(\tau_f+\tau_\varepsilon\right)^2}
  + \frac{\tau_\varepsilon}{\left(\tau_f+\tau_\varepsilon\right)^2}
  \\
  \implies\quad
  \Var(p_2-p_1)
  &= \frac{1}{\tau_f+\tau_\varepsilon}
\end{align*}
Note also that price changes are uncorrelated:
\begin{align*}
  \Cov(p_1-p_0,\,p_2-p_1)
  &=
  \Cov\left(
    \frac{\tau_\varepsilon}{\tau_f+\tau_\varepsilon}
    s
    ,\;
    f - \frac{\tau_\varepsilon}{\tau_f+\tau_\varepsilon} s
  \right)
  \\
  &=
  \Cov\left(
    \frac{\tau_\varepsilon}{\tau_f+\tau_\varepsilon} (f+\varepsilon)
    ,\;
    f
    - \frac{\tau_\varepsilon}{\tau_f+\tau_\varepsilon} (f+\varepsilon)
  \right)
  %\\
  %&=
  %\Cov\left(
    %\frac{\tau_\varepsilon}{\tau_f+\tau_\varepsilon}f
    %+ \frac{\tau_\varepsilon}{\tau_f+\tau_\varepsilon}\varepsilon
    %,\;
    %\frac{\tau_f}{\tau_f+\tau_\varepsilon}f
    %- \frac{\tau_\varepsilon}{\tau_f+\tau_\varepsilon} \varepsilon
  %\right)
  \\
  &=
  \frac{\tau_\varepsilon\tau_f}{(\tau_f+\tau_\varepsilon)^2}
  \Var(f)
  - \left(\frac{\tau_\varepsilon}{\tau_f+\tau_\varepsilon}\right)^2
  \Var(\varepsilon)
  \\
  &=
  \frac{\tau_\varepsilon\tau_f}{(\tau_f+\tau_\varepsilon)^2}
  \sigma^2_f
  - \left(\frac{\tau_\varepsilon}{\tau_f+\tau_\varepsilon}\right)^2
  \sigma^2_\varepsilon
  %\\
  %&=
  %\frac{\tau_\varepsilon\tau_f}{(\tau_f+\tau_\varepsilon)^2}
  %\frac{1}{\tau_f}
  %- \left(\frac{\tau_\varepsilon}{\tau_f+\tau_\varepsilon}\right)^2
  %\frac{1}{\tau_\varepsilon}
  %\\
  %&=
  %\frac{\tau_\varepsilon}{(\tau_f+\tau_\varepsilon)^2}
  %- \frac{\tau_\varepsilon}{(\tau_f+\tau_\varepsilon)^2}
  \\
  \implies\quad
  \Cov(p_1-p_0,\;p_2-p_1)
  &= 0
\end{align*}
Summarizing
\begin{align*}
  \E[p_{t+1}-p_t]
  &= 0
  \\
  \Var(p_1-p_0)
  &=
  \frac{\tau_\varepsilon}{\tau_f(\tau_f+\tau_\varepsilon)}
  \\
  \Var(p_2-p_1)
  &= \frac{1}{\tau_f+\tau_\varepsilon}
  \\
  \Cov(p_1-p_0,\;p_2-p_1)
  &= 0
  \\
  \Var(p_1-p_0)
  +\Var(p_2-p_1)
  &=
  \frac{1}{\tau_f}
  =\sigma^2_f
\end{align*}
From this last line, we see that the signal affects the intertemporal
resolution of uncertainty. We still have variance $\sigma_f^2$ overall.
But it's spread between time periods, and at time $t=1$, uncertainty is
lower than the unconditional uncertainty $\sigma^2_f$.
The signal, however, does not lower \emph{overall}/\emph{average}
volatility.
Also, key result: Rational learning means there's no serial correlation
in prices and returns.


\subsection{Rational Learning Benchmark, Binomial}

Periods $t=0,1,2$.
Probability $\pi$ of moving up between periods 0 and 1, and between 1
and 2.


\clearpage
\subsection{Rational Learning, Continuous Time, Continuous State}

\paragraph{Theorem 12.7 (Lipster and Shiryaev)}
We have partially observable vector processes $(\theta_t,\xi_t)$ with
$\theta_t$ unobservable and $\xi_t$ observable following
\begin{align*}
  d\theta_t &=
  \big[a_0+a_1\theta_t\;\big]\,dt
  + \sum_{i=1}^2 b_idW_i(t)
  \\
  d\xi_t
  %&=
  %\big[A_0 + A_1\theta_t\big]dt
  %+ B\;dW_2(t)
  &=
  \big[A_0 + A_1\theta_t\big]dt
  + \sum_{i=1}^2 B_idW_i(t)
\end{align*}
where coefficients $a,A,b,B$ are, in general, vector- and matrix-valued
functions that can \emph{all} depend upon $(t,\xi)$, time and the
observable (though omitted in the above expressions for clarity).
The Brownian motion terms are also vectors, with $W_1$ the same size as
$\theta$ and $W_2$ the same size as $\xi$.
The two Brownian motion terms are assumed independent.\footnote{%
  This is WLOG because both equations can depend on \emph{either}
  Brownian motion term.
}

The task is to construct a distribution characterizing beliefs about
unobservable $\theta_t$ at time $t$, having only observed
$\{\xi_s\}_{s\leq t}$.
With a Gaussian prior distribution
$\theta_0|\xi_0\sim\calN(\hat{\theta}_0,v_0)$, the estimated time-$t$
distribution for $\theta_t|\{\xi_s\}_{s\leq t}$ will
also be Gaussian for all $t$,
\begin{align*}
  \theta_t|\{\xi_s\}_{s\leq t}\sim\calN(\hat{\theta}_t,v_t)
\end{align*}
Hence we need only characterize the evolution of mean and variance
$(\hat{\theta}_t,v_t)$ %(which are, in general, nonlinear functions of $\xi_t$)
(rather than a full distribution), and conditional mean $\hat{\theta}_t$
is an optimal mean square estimate of $\theta_t$.
And by Liptser and Shiryaev, we can describe the evolution of
$(\hat{\theta}_t,v_t)$ by the optimal filtering SDEs:
\begin{align*}
  d\hat{\theta}_t &=
  \big[
  a_0 + a_1\hat{\theta}_t
  \big]\;dt
  %\\
  %&\quad
  +
  \big[
    (b\circ B)
    + v_t A_1'
  \big]
  \big(B\circ B\big)^{-1}
  \big[
    d\xi_t - \big(A_0+A_1\hat{\theta}_t\big)\;dt
  \big]
  \\
  \frac{dv_t}{dt}
  &=
  a_1v_t
  + v_t a_1'
  + (b\circ b)
  %\\
  %&\quad
  -
  \big[
    (b\circ B)
    + v_tA_1'
  \big]
  \big(B\circ B\big)^{-1}
  \big[
    (b\circ B)
    + v_tA_1'
  \big]
\end{align*}
where, again, \emph{all} coefficients are allowed to depend upon
$(t,\xi)$ (though omitted in above expressions for clarity).
Note also
\begin{align*}
  B \circ B
  &= B_1B_1' + B_2B_2'
  \qquad
  b\circ B
  = b_1B_1' + b_2B_2'
  \qquad
  b\circ b
  = b_1b_1' + b_2b_2'
\end{align*}
Of interest will also be the ``steady state'' value $\overline{v}$ where the
effect of the prior has totally washed out, and we've observed a long
enough history of $\xi_t$ that new observations don't reduce uncertainty
about $\theta_t$ any more. In that case, $dv_t/dt=0$ and steady
state $\overline{v}$ solves
\begin{align*}
  0
  &=
  a_1\overline{v}
  + \overline{v} a_1'
  + (b\circ b)
  %\\
  %&\quad
  -
  \big[
    (b\circ B)
    + \overline{v}A_1'
  \big]
  \big(B\circ B\big)^{-1}
  \big[
    (b\circ B)
    + \overline{v}A_1'
  \big]
\end{align*}

\clearpage
\paragraph{Corollary}
Suppose both $\theta_t$ and $\xi_t$ are one-dimensional and each
equation only depends on its own Brownian motion, i.e. $b_2=0$ and
$B_1=0$ so that
\begin{align*}
  d\theta_t &= \big[a_0+a_1\theta_t\;\big]\,dt + b_1\,dW_1(t)
  \\
  d\xi_t &= \big[A_0 + A_1\theta_t\big]dt + B_2\,dW_2(t)
\end{align*}
This is a special case of the above theorem where
$B \circ B = B_2^2$ , $b\circ B = 0$, and $b\circ b = b_1^2$.
Therefore, the optimal filtering formulas simplify
\begin{align*}
  d\hat{\theta}_t &=
  \big[
  a_0 + a_1\hat{\theta}_t
  \big]\;dt
  +
  \frac{A_1v_t}{B_2^2}
  \big[
    d\xi_t - \big(A_0+A_1\hat{\theta}_t\big)\;dt
  \big]
  \\
  \frac{dv_t}{dt}
  &=
  2a_1v_t
  + b_1^2
  %\\
  %&\quad
  -
  \frac{A_1^2v_t^2}{B_2^2}
\end{align*}
%Again, \emph{all} coefficients are allowed to depend upon
%$(t,\xi)$ (though omitted for clarity).

\paragraph{Example}
Suppose we have an observable dividend process $D_t$ whose evolution
depends on some unobserved fundamental process $f_t$ as follows:
\begin{align*}
  df_t
  &= -\lambda(f_t-\overline{f})\;dt + \sigma_f\;dZ_f(t)
  = \big[\lambda \overline{f} -\lambda f_t\big]\;dt + \sigma_f\;dZ_f(t)
  \\
  dD_t
  &= f_t\;dt + \sigma_D \;dZ_D(t)
\end{align*}
(\emph{Beliefs})
By the formulas
\begin{align}
  d\hat{f}_t &=
  -\lambda(
  \hat{f}_t-\overline{f}
  )\;dt
  +
  \frac{\Sigma_t}{\sigma_D}
  \underbrace{%
    \frac{1}{\sigma_D}
    \big[
      dD_t - \hat{f}_t\;dt
    \big]
  }_{\hat{Z}_D(t)}
  \label{fhat}
  \\
  d\Sigma_t
  &=
  \left(
  \sigma^2_f
  -2\lambda \Sigma_t
  - \frac{\Sigma_t^2}{\sigma_D^2}
  \right)dt
  \notag
\end{align}
where $\hat{Z}_D(t)=\frac{1}{\sigma_D}[dD_t-\hat{f}_t\;dt]$ is a
standard brownian motion in the agent's information set.
We can interpret the evolution of the posterior mean and variance as
\begin{align*}
  d\hat{f}_t &=
  -\lambda(
  \hat{f}_t-\overline{f}
  )\;dt
  +
  \frac{\Cov_t(dD_t,df_{t+dt})}{\Var_t(dD_t)}(dD_t-\hat{f}_t\,dt)
  \\
  d\Sigma_t
  %=
  %\left(
  %\sigma^2_f
  %-2\lambda \Sigma_t
  %- \frac{\Sigma_t^2}{\sigma_D^2}
  %\right)dt
  &=
  %\left(
  %\sigma^2_f
  %-2\lambda \Sigma_t
  %\right)dt
  %- \frac{\Sigma_t^2}{\sigma_D^2}dt
  %=
  d\big(\Var_t(f_t)\big)
  - \frac{\Cov_t(dD_t,df_{t+dt})^2}{\Var_t(dD_t)}
\end{align*}
Also, the evolution of $\Sigma_t$ follows a deterministic differential
equation. And we can factor the polynomial in $\Sigma^2_t$ on the RHS of
the differential equation as follows
\begin{align*}
  \frac{d\Sigma_t}{dt}
  &=
  -\frac{1}{\sigma^2_D}
  \left(
  \Sigma_t^2
  +2\lambda \sigma_D^2\Sigma_t
  -\sigma^2_f\sigma_D^2
  \right)
  =
  -\frac{1}{\sigma^2_D}
  (\Sigma_t-a_+)
  (\Sigma_t-a_-)
\end{align*}
where $a_+$ is the postive root of the polynomial and $a_-$ the
negative.
This is a separable differential equation we can solve to get
%\begin{align*}
  %\int
  %\frac{d\Sigma_t}{%
    %(\Sigma_t-a_+)
    %(\Sigma_t-a_-)
  %}
  %&=
  %\int
  %-\frac{1}{\sigma^2_D}
  %dt
  %\\
  %\ln\big(
    %(\Sigma_t-a_+)
    %(\Sigma_t-a_-)
  %\big)
  %f(\Sigma_t)
  %&=
  %-\frac{1}{\sigma^2_D}t
%\end{align*}
%Need to choose $f(\Sigma_t)$ so that
%\begin{align*}
  %\frac{1}{%
    %(\Sigma_t-a_+)
    %(\Sigma_t-a_-)
  %}
  %&=
  %\frac{d}{d\Sigma_t}
  %\left[
  %\ln\big(
    %(\Sigma_t-a_+)
    %(\Sigma_t-a_-)
  %\big)
  %f(\Sigma_t)
  %\right]
  %\\
  %&=
  %\frac{1}{(\Sigma_t-a_+) (\Sigma_t-a_-)}
  %f(\Sigma_t)
  %+
  %\ln\big(
    %(\Sigma_t-a_+)
    %(\Sigma_t-a_-)
  %\big)
  %f'(\Sigma_t)
  %\\
  %&=
  %\frac{1}{(\Sigma_t-a_+) (\Sigma_t-a_-)}
  %f(\Sigma_t)
  %+
  %\ln\big(
    %(\Sigma_t-a_+)
    %(\Sigma_t-a_-)
  %\big)
  %f(\Sigma_t)
  %\frac{f'(\Sigma_t)}{f(\Sigma_t)}
  %\\
  %&=
  %\frac{1}{(\Sigma_t-a_+) (\Sigma_t-a_-)}
  %f(\Sigma_t)
  %-
  %\frac{1}{\sigma_D^2}t
  %\frac{f'(\Sigma_t)}{f(\Sigma_t)}
  %\\
  %1
  %&=
  %f(\Sigma_t)
  %-
  %[(\Sigma_t-a_+) (\Sigma_t-a_-)]
  %\frac{f'(\Sigma_t)}{f(\Sigma_t)}
  %\frac{1}{\sigma_D^2}t
%\end{align*}
\begin{align*}
  \frac{\Sigma_t-a_+}{\Sigma_t-a_-}
  &=
  \frac{\Sigma_0-a_+}{\Sigma_0-a_-}
  e^{-\frac{(a_+-a_-)}{\sigma^2_D}t}
  \qquad
  \text{and}
  \qquad
  \Sigma_t\ra a_+
\end{align*}

\clearpage
(\emph{Pricing})
We might now also consider how a risk-neutral investor would value the
asset that pays the dividend stream, given $\{D_s\}_{s=0}^t$, which
implies $\hat{f}_t$.
Given risk-free interest rate is $r$, the price for a risk-neutral
investor is naturally
\begin{align*}
  p(\hat{f}_t)
  =
  \E\left[
    \int_{0}^\infty e^{-rs}D_{t+s}\;ds
    \;\big|\;\hat{f}_t
  \right]
  =\E\left[\int_t^\infty e^{-r(s-t)}D_s\;ds\;\big|\;\hat{f}_t\right]
\end{align*}
To compute the integral, we need an expression for $D_t$.
To start, recall $\hat{Z}_D(t)$, (the Brownian motion in the investor's
information set), whose definition implies
\begin{align*}
  \hat{Z}_D(t)
  :=
  \frac{1}{\sigma_D}
  \big[
    dD_t-\hat{f}_t\;dt
  \big]
  \quad\implies\quad
  dD_t
  &=
  \hat{f}_t\;dt+\sigma_D\;d\hat{Z}_D(t)
  %\\
  %\quad\implies\quad\;\,
  %D_{t+s}
  %&=
  %D_t
  %+
  %\int_t^{t+s} \hat{f}_{w}\;dw
  %+\sigma_D\int_{t}^{t+s} \hat{Z}_D(w)\;dw
\end{align*}
Next, suppose we use Ito's Lemma to obtain the SDE for $e^{-rs}D_s$:
\begin{align*}
  d\big(e^{-rs} D_s\big)
  &=
  -re^{-rs}D_s\;ds
  + e^{-rs} dD_s
  \\
  &=
  -re^{-rs}D_s\;ds
  + e^{-rs}
  \left[
  \hat{f}_s\;ds+\sigma_D\;d\hat{Z}_D(s)
  \right]
  %\\
  %&=
  %-re^{-rs}D_s\;ds
  %+ e^{-rs} \hat{f}_s\;ds
  %+ e^{-rs}+\sigma_D\hat{Z}_D(s)
\end{align*}
Integrate on both sides
\begin{align*}
  \lim_{T\ra\infty} e^{-rT} D_T
  -
  e^{-rt} D_t
  &=
  -r\int_t^\infty e^{-rs}D_s\;ds
  +
  \int_t^\infty
  e^{-rs} \hat{f}_s\;ds
  +
  \int_t^\infty
  e^{-rs}
  \sigma_D\;d\hat{Z}_D(s)
\end{align*}
Note that since $D_t$ stationary, the limit on the LHS equals zero.
Now take expectation conditional on $\hat{f}_t$.
Note that the last Ito integral will drop, since it's a martingale:
\begin{align*}
  \E\big[
  - e^{-rt} D_t
  \;\big|\;\hat{f}_t\big]
  &=
  -r\E\left[\int_t^\infty e^{-rs}D_s\;ds\;\big|\;\hat{f}_t\right]
  +
  \E\left[
  \int_t^\infty
  e^{-rs}
  \hat{f}_s\;ds\;\big|\;\hat{f}_t\right]
\end{align*}
Multply by $e^{rt}$ and rearrange
\begin{align*}
  r\E\left[\int_t^\infty e^{-r(s-t)}D_s\;ds\;\big|\;\hat{f}_t\right]
  &=
  D_t
  +
  \E\left[ \int_t^\infty e^{-r(s-t)}
  \hat{f}_s\;ds\;\big|\;\hat{f}_t\right]
\end{align*}
Assuming
\begin{align*}
  \hat{f}_{t+w}
  &=
  \hat{f}_{t}
  e^{-\lambda w}
  +
  \overline{f}
  (1-e^{-\lambda w})
  +
  \lambda \overline{f}
  \int_t^{t+w}
  e^{-\lambda (t+w-x)}
  \frac{\Sigma_x}{\sigma_D}
  \;d\hat{Z}_D(x)
\end{align*}
Then
\begin{align*}
  \E\left[ \int_0^\infty e^{-rs}
  \hat{f}_{t+s}\;ds\;\big|\;\hat{f}_t\right]
  &=
  \E\left[ \int_0^\infty
  e^{-rs}
  \left[
  \hat{f}_{t}
  e^{-\lambda s}
  +
  \overline{f}
  (1-e^{-\lambda s})
  \right]
  \;ds
  \;\big|\;\hat{f}_t\right]
  \\
  &=
  \hat{f}_{t}
  \int_0^\infty
  e^{-(\lambda+r) s}
  \;ds
  +
  \overline{f}
  \int_0^\infty
  (
  e^{-rs}
  -e^{-(\lambda+r) s}
  )
  \;ds
  \\
  &=
  \frac{\hat{f}_{t}}{(\lambda+r)}
  +
  \overline{f}
  \left[
  -\frac{1}{r}
  e^{-rs}
  +\frac{1}{\lambda+r}
  e^{-(\lambda+r) s}
  \right]_0^\infty
  \\
  &=
  \frac{\hat{f}_{t}}{(\lambda+r)}
  +
  \overline{f}
  \frac{\lambda}{r(\lambda+r)}
\end{align*}






\clearpage
Assume the asset is ex-dividend, so we drop that $D_t$ on the RHS
because the agent doesn't get that initial payment. Then the price is
\begin{align*}
\end{align*}
To compute the integral on the RHS, want to solve following SDE
to get an expression for $\hat{f}_t$:
\begin{align*}
  d\hat{f}_t
  &=
  -\lambda(
  \hat{f}_t-\overline{f}
  )\;dt
  +
  \frac{\Sigma_t}{\sigma_D}
  \hat{Z}_D(t)
\end{align*}
To solve, use Ito's Lemma on function
$g_t = g(\hat{f}_t,t) = \hat{f}_t e^{\lambda t}$.
The final result is solution
%\begin{align*}
  %dg_t
  %&=
  %\frac{dg}{dt}\;dt
  %+ \frac{dg}{d\hat{f}}\;d\hat{f}
  %+ \frac{1}{2}\frac{d^2g}{d\hat{f}^2}\;(d\hat{f})^2
  %\\
  %&=
  %\left[
  %\frac{dg}{dt}
  %+ \frac{1}{2}\frac{d^2g}{d\hat{f}^2}
    %\frac{\Sigma_t^2}{\sigma_D^2}
  %-\frac{dg}{d\hat{f}}
  %\lambda(
  %\hat{f}_t-\overline{f}
  %)
  %\right]dt
  %+
  %\frac{dg}{d\hat{f}}
  %\frac{\Sigma_t}{\sigma_D}
  %\hat{Z}_D(t)
  %\\
  %&=
  %\left[
  %\lambda g_t
  %-e^{\lambda t}
  %\lambda(
  %\hat{f}_t-\overline{f}
  %)
  %\right]dt
  %+
  %e^{\lambda t}
  %\frac{\Sigma_t}{\sigma_D}
  %\hat{Z}_D(t)
  %\\
  %&=
  %\left[
  %\lambda g_t
  %- \lambda g_t
  %+ e^{\lambda t} \lambda \overline{f}
  %\right]dt
  %+
  %e^{\lambda t}
  %\frac{\Sigma_t}{\sigma_D}
  %\hat{Z}_D(t)
  %\\
  %&=
  %\lambda e^{\lambda t} \overline{f}
  %dt
  %+
  %e^{\lambda t}
  %\frac{\Sigma_t}{\sigma_D}
  %\hat{Z}_D(t)
%\end{align*}
%Then
\begin{align*}
  %g_{t+s}
  %&=
  %g_t
  %+
  %\int_t^{t+s}
  %\lambda \overline{f} e^{\lambda w}\;dw
  %+
  %\int_t^{t+s}
  %\lambda \overline{f} e^{\lambda w}
  %\frac{\Sigma_w}{\sigma_D}
  %\;d\hat{Z}_D(w)
  %\\
  %&=
  %g_t
  %+
  %\big[
  %\overline{f}
  %e^{\lambda w}
  %\big]_t^{t+s}
  %+
  %\lambda \overline{f}
  %\int_t^{t+s}
  %e^{\lambda w}
  %\frac{\Sigma_w}{\sigma_D}
  %\;d\hat{Z}_D(w)
  %\\
  %&=
  %g_t
  %+
  %\overline{f}
  %e^{\lambda t}(e^{\lambda s}-1)
  %+
  %\lambda \overline{f}
  %\int_t^{t+s}
  %e^{\lambda w}
  %\frac{\Sigma_w}{\sigma_D}
  %\;d\hat{Z}_D(w)
  %\\
  %\implies\quad
  %\hat{f}_{t+w}
  %e^{\lambda(t+w)}
  %&=
  %\hat{f}_{t}
  %e^{\lambda t}
  %+
  %\overline{f}
  %e^{\lambda t}(e^{\lambda w}-1)
  %+
  %\lambda \overline{f}
  %\int_t^{t+w}
  %e^{\lambda x}
  %\frac{\Sigma_x}{\sigma_D}
  %\;d\hat{Z}_D(x)
  %\\
  \hat{f}_{t+w}
  &=
  \hat{f}_{t}
  e^{-\lambda w}
  +
  \overline{f}
  (1-e^{-\lambda w})
  +
  \lambda \overline{f}
  \int_t^{t+w}
  e^{-\lambda (t+w-x)}
  \frac{\Sigma_x}{\sigma_D}
  \;d\hat{Z}_D(x)
\end{align*}
Plugging into the expression for $D_{t+s}$, we get
\begin{align*}
  D_{t+s}
  &=
  D_t
  +
  \int_0^{s}
  \left[
  \hat{f}_{t}
  e^{-\lambda w}
  +
  \overline{f}
  (1-e^{-\lambda w})
  +
  \lambda \overline{f}
  \int_t^{t+w}
  e^{-\lambda (t+w-x)}
  \frac{\Sigma_x}{\sigma_D}
  \;d\hat{Z}_D(x)
  \right]
  dw
  \\
  &\qquad
  +\sigma_D[\hat{Z}_D(t+s)-\hat{Z}_D(t)]
  \\
  &=
  D_t
  +
  \hat{f}_{t}
  \int_0^{s}
  e^{-\lambda w}
  \;dw
  +
  \overline{f}
  \int_0^{s}
  (1-e^{-\lambda w})
  \;dw
  +
  \lambda \overline{f}
  \int_0^{s}
  \int_t^{t+w}
  e^{-\lambda (t+w-x)}
  \frac{\Sigma_x}{\sigma_D}
  \;d\hat{Z}_D(x)
  \;dw
  \\
  &\qquad
  +\sigma_D[\hat{Z}_D(t+s)-\hat{Z}_D(t)]
  \\
  &=
  D_t
  -
  \frac{\hat{f}_{t}}{\lambda}
  \big[e^{-\lambda w}\big]_0^{s}
  +
  \overline{f}
  \left[w+\frac{1}{\lambda}e^{-\lambda w}\right]_0^{s}
  +
  \lambda \overline{f}
  \int_0^{s}
  \int_t^{t+w}
  e^{-\lambda (t+w-x)}
  \frac{\Sigma_x}{\sigma_D}
  \;d\hat{Z}_D(x)
  \;dw
  \\
  &\quad
  +\sigma_D[\hat{Z}_D(t+s)-\hat{Z}_D(t)]
  \\
  &=
  D_t
  +
  \frac{\hat{f}_{t}}{\lambda}
  (1-e^{-\lambda s})
  +
  \overline{f}
  \left[s-\frac{1}{\lambda}(1-e^{-\lambda s})\right]
  +
  \lambda \overline{f}
  \int_0^{s}
  \int_t^{t+w}
  e^{-\lambda (t+w-x)}
  \frac{\Sigma_x}{\sigma_D}
  \;d\hat{Z}_D(x)
  \;dw
  \\
  &\quad
  +\sigma_D[\hat{Z}_D(t+s)-\hat{Z}_D(t)]
  \\
  &=
  D_t
  +
  \frac{1}{\lambda}(1-e^{-\lambda s})
  (\hat{f}_t-\overline{f})
  + \overline{f}s
  +
  \lambda \overline{f}
  \int_0^{s}
  \int_t^{t+w}
  e^{-\lambda (t+w-x)}
  \frac{\Sigma_x}{\sigma_D}
  \;d\hat{Z}_D(x)
  \;dw
  +\sigma_D[\hat{Z}_D(t+s)-\hat{Z}_D(t)]
\end{align*}
Now, sub in
\begin{align*}
  p(\hat{f}_t)
  =
  \E\left[
    \int_{0}^\infty e^{-rs}D_{t+s}\;ds
    \;\big|\;\hat{f}_t
  \right]
\end{align*}

\clearpage
Suppose
\begin{align*}
  d\hat{f}_t
  &=
  -\lambda(
  \hat{f}_t-\overline{f}
  )\;dt
  +
  \frac{\Sigma_t}{\sigma_D}
  \;d\hat{Z}_D(t)
  \\
  dD_t
  &=
  \hat{f}_t\;dt+\sigma_D\;d\hat{Z}_D(t)
  \\
  p(\hat{f}_t)
  &=
  \frac{1}{\lambda+r}\hat{f}_t
  +
  \frac{\lambda}{r(\lambda+r)}
  \overline{f}
  \\
  \implies\quad
  dp_t
  &=
  \frac{1}{\lambda+r}
  d\hat{f}_t
  =
  -\frac{\lambda}{\lambda+r}
  (
  \hat{f}_t-\overline{f}
  )\;dt
  +
  \frac{1}{\lambda+r}
  \frac{\Sigma_t}{\sigma_D}
  \;d\hat{Z}_D(t)
\end{align*}
The excess return process is then
\begin{align*}
  dq_t
  &=
  dD_t
  +
  dp_t
  - rp_t\;dt
  \\
  &=
  \left[
  \hat{f}_t\;dt+\sigma_D\;d\hat{Z}_D(t)
  \right]
  +
  \left[
  -\frac{\lambda}{\lambda+r}
  (
  \hat{f}_t-\overline{f}
  )\;dt
  +
  \frac{1}{\lambda+r}
  \frac{\Sigma_t}{\sigma_D}
  \hat{Z}_D(t)
  \right]
  \\
  &\qquad
  -
  r
  \left[
  \left(
  \frac{1}{\lambda+r}\hat{f}_t
  +
  \frac{\lambda}{r(\lambda+r)}
  \overline{f}
  \right)
  \;dt
  \right]
  \\
  &=
  \left[
  \frac{\lambda+r}{\lambda+r}
  \sigma_D +
  \frac{1}{\lambda+r}
  \frac{\Sigma_t}{\sigma_D}
  \right]
  \hat{Z}_D(t)
\end{align*}



\clearpage
\subsection{Rational Learning, Continuous Time, Discrete State}

\paragraph{Theorem 9.1, Liptser and Shiryaev}
Random processes $(\theta_t,\xi_t)$ where $\theta_t$ is an unobservable
component taking with a finite number of states, and observable $\xi_t$
follows
\begin{align*}
  d\xi_t
  &= A_t(\theta_t,\xi_t)\;dt + B_t(\xi_t)\;dW_t
\end{align*}
Given $\{\xi_s\}_{0}^t$, we want to construct estimates of the posterior
probability that $\theta_t=i$ for some state $i$. So define the
posterior probabilities as
\begin{align*}
  \pi^i_t
  &:= P\big[\theta_t=i\;|\;\{\xi_s\}_{s=0}^t\big]
  %\\
  %\lambda_{ij}(t)
  %&:=
  %P[\theta_{t+dt}=j|\theta_t=i]
  %\qquad\forall t
\end{align*}
Moreover, let $\lambda_{ij}(t)$ denote the ifinitesimal transition
probablity from $i$ to $j$ at time $t$ (i.e. the flow rate from the
infinitesimal generator so transition probability of $\lambda\,dt$
implies flow rate/infinitesimal-transition-probability of $\lambda$,
while $1-\lambda\,dt$ implies $-\lambda$).

Then the posterior probablitites follow the following SDE\footnote{%
  Note the coefficient on $dt$ is incorrect in the book's theorem
  statement. See proof for correct coefficient.
}
\begin{align*}
  d\pi_t^j
  &=
  \left[
  \sum_i
  \lambda_{ij}(t)
  \pi^i_t
  \right]
  dt
  +
  \pi^j_t
  \left[
  \frac{A_t(j,\xi)-\sum_i A_t(i,\xi)\pi^i_t}{B_t(\xi)}
  \right]
  d\overline{W}_t
  \\
  \text{where}\quad
  d\overline{W}_t
  &=
  \frac{d\xi_t-\left[\sum_i A_t(i,\xi)\pi^i_t\right]\;dt}{B_t(\xi)}
\end{align*}
%Substituting in $d\overline{W}_t$, this is equivalent to
%\begin{align*}
  %d\pi_t^j
  %&=
  %\left[
  %\sum_i
  %\lambda_{ij}(t)
  %\pi^i_t
  %\right]
  %dt
  %+
  %\frac{\pi^j_t}{B_t(\xi_t)^2}
  %\left[
  %A_t(j,\xi)-\sum_i A_t(i,\xi)\pi^i_t
  %\right]
  %\left[
  %d\xi_t-\left(\sum_i A_t(i,\xi)\pi^i_t\right)dt
  %\right]
  %\\
  %&=
  %\left[
  %\sum_i
  %\lambda_{ij}(t)
  %\pi^j_t
  %-
  %\frac{%
    %\pi^j_t
    %\big(
    %A_t(j,\xi)-\sum_i A_t(i,\xi)\pi^i_t
    %\big)
    %\big(
    %\sum_i A_t(i,\xi)\pi^i_t
    %\big)
  %}{B_t(\xi_t)^2}
  %\right]
  %dt
  %+
  %\frac{%
    %\pi^j_t
    %\big(
    %A_t(j,\xi)-\sum_i A_t(i,\xi)\pi^i_t
    %\big)
  %}{B_t(\xi_t)^2}
  %d\xi_t
%\end{align*}

\paragraph{Corollary}
Consider the special case of two states
$\{\overline{\theta}, \underline{\theta}\}$ with
\begin{align*}
  d\xi_t
  &= \theta_t \, dt + \sigma \,dW_t
\end{align*}
and with transition probabilities in the infinitesimal generator given
by
\begin{align*}
  \lambda_{11}(t)
  = \lambda_{11}
  &= -\lambda
  \qquad
  \lambda_{22}(t)
  = \lambda_{22}
  = -\mu
\end{align*}
With only two states, we only need an SDE for
$\pi_t^1=P[\theta_t=\overline{\theta}|\{\xi_s\}_{s=0}^t]$.
By the formula above,
\begin{align*}
  d\pi_t^1
  &=
  (
  -\lambda\pi_t^1 + \mu\pi_t^2
  )
  \pi_t^1
  dt
  +
  \frac{\pi_t^1}{\sigma^2}
  \big(
  \overline{\theta}
  - \big[\overline{\theta}\pi_t^1 + \underline{\theta}\pi_t^2\big]
  \big)
  \left(
  d\xi_t
  -
  \E_t[d\xi_t]
  \right)
\end{align*}
where $\E_t$ is the expectation taken at time time where the posterior
probability is given by $\pi_t^1$.
Substituting in $\pi_t^2=1-\pi_t^1$ and dropping the superscript, we
then have
\begin{align*}
  d\pi_t
  &=
  (
  -\lambda\pi_t + \mu(1-\pi_t)
  )
  dt
  +
  \frac{\pi_t}{\sigma^2}
  \big(
  \overline{\theta}
  - \big[\overline{\theta}\pi_t + \underline{\theta}(1-\pi_t)\big]
  \big)
  \left(
  d\xi_t
  -
  \E_t[d\xi_t]
  \right)
  \\
  &=
  (
  \mu-(\mu+\lambda)\pi_t
  )
  \,dt
  +
  \left(
  \frac{\overline{\theta} - \underline{\theta}}{\sigma}
  \right)
  \pi_t(1-\pi_t)
  \frac{\left( d\xi_t - \E_t[d\xi_t] \right)}{\sigma}
\end{align*}
Letting $\pi^s = \frac{\mu}{\lambda+\mu}$ denote the steady state value
of $\pi$, rewrite the coefficient on $dt$ to get
\begin{align*}
  d\pi_t
  &=
  (\mu+\lambda)(\pi^s-\pi_t)
  dt
  +
  \left(
  \frac{\overline{\theta} - \underline{\theta}}{\sigma}
  \right)
  \pi_t(1-\pi_t)
  \frac{\left( d\xi_t - \E_t[d\xi_t] \right)}{\sigma}
\end{align*}


\clearpage
\subsection{Overconfidence}

Four dates, no disocounting between periods. There's some unobserved
fundamental value $\theta$, which will be the terminal payoff announced
in the final period.
\begin{itemize}
  \item $t=0$: Set the prior
    $p(\theta)\sim\calN(0,\sigma_\theta^2)$

  \item $t=1$: Informed investor observes private signal $s$ that's
    correlated with $\theta$. In reality,
    \begin{align*}
      s_1 &= \theta + \varepsilon
      \quad
      \text{where}\quad
       \varepsilon\sim\calN(0, \sigma_\theta^2)
      \qquad\implies\qquad
      s_1|\theta \sim\calN(\theta, \sigma_\varepsilon^2)
    \end{align*}
    \emph{However}, informed investors are overconfident and
    \emph{believe} the error has lower variance:
    \begin{align*}
      s_1|\theta \sim\calN(\theta, \sigma_C^2)
      \qquad\text{where}\quad
      \sigma^2_{C}<\sigma^2_\varepsilon
    \end{align*}
    Therefore, they form the posterior for $\theta$ using that incorrect
    variance:
    \begin{align*}
      p(\theta|s_1)
      &\propto
      p(s_1|\theta)
      \cdot p(\theta)
      =
      \calN\bigg(
        \underbrace{
          \frac{\tau_C}{\tau_\theta+\tau_C}
          (\theta+\varepsilon)
        }_{\overline{\theta}_1}
        ,\;\;
        \underbrace{
        \frac{1}{\tau_\theta+\tau_C}
        }_{\sigma^2_1}
      \bigg)
    \end{align*}
    where $\tau_x=1/\sigma^2_x$, i.e. the inverse variance which is
    called the ``precision.''

  \item $t=2$:
    All investors observe a public signal
    \begin{align*}
      s_2
      = \theta + \eta
      \qquad\text{where}\quad
      \begin{cases}
        \eta \sim \calN(0,\sigma^2_P) \\
        \eta \perp \varepsilon
      \end{cases}
    \end{align*}
    Its variance $\sigma^2_P$ is \emph{correctly} observed by investors.
    Therefore, they form the posterior for $\theta$ using their
    posterior from the last time as the prior:
    \begin{align*}
      p(\theta|s_1,s_2)
      &\propto
      p(s_2|\theta)
      \cdot
      p(\theta|s_1)
      =
      \calN\bigg(
        \overline{\theta}_1
        +
        \frac{\tau_P}{\tau_P+\tau_1}
        (s_2-\overline{\theta}_1)
        ,\;\;
        \frac{1}{\tau_P+\tau_1}
      \bigg)
    \end{align*}
    Substituting in for $\overline{\theta}_1$ and $\sigma^2_1$, the mean
    is then
    \begin{align*}
      \overline{\theta}_1
      +
      \frac{\tau_P}{\tau_P+\tau_1}
      (s_2-\overline{\theta}_1)
      &=
      \frac{\tau_C}{\tau_\theta+\tau_C}
      (\theta+\varepsilon)
      +
      \frac{\tau_P}{\tau_P+\tau_1}
      \left(
        (\theta+\eta)
        -\frac{\tau_C}{\tau_\theta+\tau_C}
        (\theta+\varepsilon)
      \right)
      \\
      &=
      \left[
      \frac{\tau_C}{\tau_\theta+\tau_C}
      +
      \frac{\tau_P}{\tau_P+\tau_1}
      \frac{\tau_\theta}{\tau_\theta+\tau_C}
      \right]
      \theta
      +
      \left[
      \frac{\tau_C}{\tau_\theta+\tau_C}
      -
      \frac{\tau_P}{\tau_P+\tau_1}
      \frac{\tau_C}{\tau_\theta+\tau_C}
      \right]
      \varepsilon
      +
      \frac{\tau_P}{\tau_P+\tau_1}
      \eta
      \\
      &=
      \frac{\tau_C(\tau_P+\tau_1)+\tau_P\tau_\theta}{(\tau_\theta+\tau_C)(\tau_P+\tau_1)}
      \theta
      +
      \left[
      \frac{\tau_C(\tau_P+\tau_1)-\tau_P\tau_C}{(\tau_\theta+\tau_C)(\tau_P+\tau_1)}
      \right]
      \varepsilon
      +
      \frac{\tau_P}{\tau_P+\tau_1}
      \eta
    \end{align*}
    Replace
    \begin{align*}
      (\text{Mean})
      &=
      \frac{\tau_C(\tau_P+\tau_1)+\tau_P\tau_\theta}{(\tau_\theta+\tau_C)(\tau_P+\tau_1)}
      \theta
      +
      \left[
      \frac{\tau_C(\tau_P+\tau_1)-\tau_P\tau_C}{(\tau_\theta+\tau_C)(\tau_P+\tau_1)}
      \right]
      \varepsilon
      +
      \frac{\sigma_1^2}{\sigma_P^2+\sigma_1^2}
      \eta
      \\
      &=
      \frac{\tau_C(\tau_P+\tau_1)+\tau_P\tau_\theta}{(\tau_\theta+\tau_C)(\tau_P+\tau_1)}
      \theta
      +
      \left[
      \frac{\tau_C(\tau_P+\tau_1)-\tau_P\tau_C}{(\tau_\theta+\tau_C)(\tau_P+\tau_1)}
      \right]
      \varepsilon
      +
      \frac{\frac{1}{\tau_\theta+\tau_C}}{\sigma_P^2+\frac{1}{\tau_\theta+\tau_C}}
      \eta
      \\
      &=
      \frac{\tau_C(\tau_P+\tau_1)+\tau_P\tau_\theta}{(\tau_\theta+\tau_C)(\tau_P+\tau_1)}
      \theta
      +
      \left[
      \frac{\tau_C(\tau_P+\tau_1)-\tau_P\tau_C}{(\tau_\theta+\tau_C)(\tau_P+\tau_1)}
      \right]
      \varepsilon
      +
      \frac{%
        \frac{1}{\frac{1}{\sigma_\theta^2}+\frac{1}{\sigma_C^2}}
      }{%
        \sigma_P^2
        + \frac{1}{\frac{1}{\sigma_\theta^2}+\frac{1}{\sigma_C^2}}
      }
      \eta
      \\
      &=
      \frac{\tau_C(\tau_P+\tau_1)+\tau_P\tau_\theta}{(\tau_\theta+\tau_C)(\tau_P+\tau_1)}
      \theta
      +
      \left[
      \frac{\tau_C(\tau_P+\tau_1)-\tau_P\tau_C}{(\tau_\theta+\tau_C)(\tau_P+\tau_1)}
      \right]
      \varepsilon
      +
      \frac{%
        \sigma^2_\theta\sigma_C^2
      }{%
        \sigma_P^2\sigma_\theta^2
        +\sigma_P^2\sigma_C^2
        + \sigma^2_\theta\sigma_C^2
      }
      \eta
    \end{align*}


\end{itemize}
\paragraph{Prices}
For a risk-neutral agent, $p_t=\E[\theta|\calI_t]$
(where the expectation is taken under the agent's misspecified measure)
so that price in each period
\begin{align*}
  p_0
  &=
  \E[\theta|\calI_0]
  =
  \overline{\theta} \\
  p_1
  &=
  \E[\theta|\calI_1]
  =
  \overline{\theta}
  + \frac{\tau_C}{\tau_\theta+\tau_C}
  (s_1-\overline{\theta})
  =
  -\frac{\tau_\theta}{\tau_\theta+\tau_C}
  \overline{\theta}
  +
  \frac{\tau_C}{\tau_\theta+\tau_C}
  (\theta+\varepsilon)
  \\
\end{align*}




\clearpage
\section{Limits of Arbitrage}

\subsection{Noise Trader Risk}

Assets
\begin{itemize}
  \item Safe Asset $s$: Pays fixed dividend of $r$, perfectly elastic
    supply. Numeraire, so price is normalized to one.
  \item Risky Asset $u$: Pays fixed dividend $r$, but supply fixed at
    one unit. Price is $p_t$
\end{itemize}
Each period $t$, a unit mass of agents consisting of two types of
agents.
\begin{itemize}
  \item Measure $1-\mu$ of rational agents (denoted by $i$) with
    rational expectations of market prices.
    Denote their expected price by $\E^i[p_{t+1}]=\E[p_{t+1}]$.

  \item Measure $\mu$ of noise traders (denoted by $n$) who misperceive
    the expected price so that, for \emph{all} noise-trading agents,
    there is a common source $\rho_t$ of misperception at time $t$ so
    \begin{align*}
      \E^n_t[p_{t+1}]
      =
      \E_t[p_{t+1}]
      +
      \rho_t
      \qquad\text{where}\quad
      \rho_t
      \sim\calN(\rho^*,\sigma_\rho^2)
    \end{align*}
\end{itemize}
Each type of agent is born with zero wealth.



\clearpage
\section{Heterogeneous Beliefs}

\subsection{Aggree to Disagree}

History of thought:
\begin{itemize}
  \item Harsanyi (1968, Management Science):
    People should have \emph{common prior beliefs} so that any
    difference in subjective beliefs are due entirely to differences in
    subsequent information
  \item Aumann (1976, Annals of Statistics) showed that people with
    common prior cannot agree to disagree.
  \item Milgrom and Stokey (1982, JET) and Tirole (1982, Ecta):
    When all agents are rational and share common beliefs, asymmetric
    information cannot generate trading
\end{itemize}
Hence, asymmetric information \emph{cannot} be the \emph{sole} driver of
trading. So left with the following other explanations for trading
\begin{itemize}
  \item Heterogeneious Priors:
    Savage view, Morris (1995, Economics and Philosophy)
  \item Noise traders: Agents with distorted beliefs, or rationality not
    common knowledge
  \item Asymmetric information + Noise Traders:
    The addition of noise traders is enough to make asymmetric
    information cause trading. Without those noise traders, no trade
    theorem above.
    Examples
    \begin{itemize}
      \item Grossman and Stiglitz (1980, AER)
      \item Kyle (1985, Ecta)
    \end{itemize}
\end{itemize}


\clearpage
\subsection{Short Sale Constraints and Asset Price Bubbles}

\subsubsection{Static Models}

Miller (1997, JF): Short sale constraints + heterogeneous beliefs can
make stocks overpriced, so that price only reflects the view of the
optimists.

Two dates $t=0,1$ with a single risky asset that has terminal payoff
\begin{align*}
  f \sim\calN(\mu,\sigma^2)
\end{align*}
However, investors have diverse beliefs.
Specifically, there is a continuum of agents with beliefs about the mean
satisfying $\mu_i\sim U[\mu-\kappa,\mu+\kappa]$.
Agents maximize expected utility over terminal wealth subject to these
beliefs
\begin{align*}
  \max_{x_i}
  \;&
  \E^i\big[
    -\exp\big\{
      -\gamma(W_0 + (f-p)x_i)
    \big\}
  \big]
  \qquad
  \text{s.t.}\quad
  f \sim\calN(\mu_i,\sigma^2)
\end{align*}
By normality and CARA, this is equivalent to solving
\begin{align*}
  %&\max_{x_i}
  %-\exp\left\{
  %\E^i\big[
      %-\gamma(W_0 + (f-p)x_i)
  %\big]
  %+
  %\frac{1}{2}
  %\Var^i\big(
      %-\gamma(W_0 + (f-p)x_i)
  %\big)
  %\right\}
  %\\
  \iff\qquad
  &\max_{x_i}
  \;\;
  \E^i\big[
      \gamma(W_0 + (f-p)x_i)
  \big]
  -
  \frac{1}{2}
  \Var^i\big(
      -\gamma(W_0 + (f-p)x_i)
  \big)
  \\
  \iff\qquad
  &\max_{x_i}
  \;\;
  \gamma(\mu_i-p)x_i
  -
  \frac{\gamma^2}{2}\sigma^2x^2_i
\end{align*}
Hence demand satisfies
\begin{align*}
  0&=
  \gamma(\mu_i-p)
  -
  \gamma^2\sigma^2
  x_i
  \quad\implies\quad
  x_i
  =
  \frac{\mu_i-p}{\gamma\sigma^2}
\end{align*}
Suppose no short sales are allowed.
With quantity $Q$, that means
\begin{align*}
  Q
  = \int \max\{x_i,0\}\;di
  &=
  \int_{\mu-\kappa}^{\mu+\kappa}
  \max\left\{\frac{\mu_i-p}{\gamma\sigma^2},0\right\}
  f(\mu_i)
  \;d\mu_i
  \\
  &=
  \int_{\max\{p,\mu-\kappa\}}^{\mu+\kappa}
  \frac{\mu_i-p}{\gamma\sigma^2}
  \frac{1}{2\kappa}
  \;d\mu_i
  \\
  &=
  \frac{1}{2\kappa}
  \frac{1}{\gamma\sigma^2}
  \int_{\max\{p,\mu-\kappa\}}^{\mu+\kappa}
  (\mu_i-p)
  \;d\mu_i
  \\
  \implies\quad
  Q
  &=
  \frac{1}{2\kappa}
  \frac{1}{\gamma\sigma^2}
  \frac{(\mu_i-p)^2}{2}|_{\max\{p,\mu-\kappa\}}^{\mu+\kappa}
\end{align*}
Two cases. First suppose $p\leq \mu-\kappa$ so the short sale
constraints are non-binding. Then
\begin{align*}
  Q
  &=
  \frac{1}{4\kappa}
  \frac{1}{\gamma\sigma^2}
  \left[
  (\mu+\kappa-p)^2
  -
  (\mu-\kappa-p)^2
  \right]
  \quad\implies\quad
  \boxed{%
    p = \mu -\gamma\sigma^2 Q
  }
\end{align*}
Recall that's the price assuming
\begin{align*}
  p \leq \mu-\kappa
  \quad\iff\quad
  \mu -\gamma\sigma^2 Q
  \leq \mu-\kappa
  \quad\iff\quad
  \gamma\sigma^2 Q
  \geq \kappa
\end{align*}
That's a restriction on parameters that may or may not hold. If it does,
the above price is the equilibrium price.
If not, then look for another equilibrium where $p>\mu-\kappa$, implying
\begin{align*}
  Q
  &=
  \frac{1}{4\kappa\gamma\sigma^2}
  (\mu+\kappa-p)^2
  \quad\implies\quad
  \boxed{%
    p = \mu+\kappa- 2\sigma\sqrt{\kappa\gamma Q}
  }
\end{align*}







\clearpage
\subsubsection{Dynamic Model: Harrison and Kreps (1978, QJE)}

The right to resell makes investors more eager to hold an asset then if
forced to hold forever.
The expected payoff from buy and hold is lower than the actual initial
valuation with the ability to resell if beliefs are right.

%Time $t=0,1,2$. Agents differ in their beliefs about the probabilities
%of the payoff attaining different terminal values.
%No short sales, so whoever values more buys the asset; whoever values
%less sells it.


%\subsubsection{Dynamic Model: Scheinkman and Xiong (2003, JPE)}


\subsubsection{Complete Markets Eequilibrium with Heterogeneous Beliefs}

Ingredients
\begin{itemize}
  \item
    Endowment economy with a single risky asset whose dividend follows
    \begin{align*}
      \frac{dD_t}{D_t}
      &= f_t\;dt +\sigma dZ(t)\\
      df_t
      &= -\lambda_f(f_t-\mu)\;dt + \sigma_fdZ_f(t)
    \end{align*}
    with $\mu$ unobserved.

  \item
    $I$ agents in the economy with different beliefs about $\mu$.
    Each agent maximizes expected lifetime utility based on that belief
    \begin{align*}
      \E^i\left[
        \int_0^\infty
        e^{-\beta t}
        \frac{1}{1-\gamma}
        c_{t,i}^{1-\gamma}
        \;dt
      \right]
    \end{align*}
  \item
    Market clearing $D_t=\sum_i c_{t,i}$

  \item
    Complete markets means there's a representative agent.


\end{itemize}





\subsection{Endogenous Risk in GE}

Single risky asset that follows
\begin{align*}
  \frac{dD_t}{D_t}
  &= f_t\;dt + \sigma dZ_t
  \\
  df_t
  &= -\lambda_f(f_t-\mu)\;dt + \sigma_fdZ_f(t)
\end{align*}
However, $I$ agents misperceive the mean.
In particular, agent $i$ believes the mean of $f_t$ is $\mu+I_i$, hence
has beliefs
\begin{align*}
  df_t^i
  &= -\lambda_f\big(f_t^i-(\mu+I_i)\big)\;dt + \sigma_fdZ_f^i(t)
\end{align*}
where $f_t^i,Z_f^i(t)$ are processes under the measure for agent $i$.
Everyone maxes utility subject to their beliefs
\begin{align*}
  \max \E^i\int_0^\infty e^{-\beta t}
  \frac{c_{t,i}^{1-\gamma}}{1-\gamma}
  \;dt
\end{align*}
Market clearing
\begin{align*}
  D_t = \sum_{i=1}^I c_{t,i}
\end{align*}
To make things easier, transform agent $i$'s expectation in the
expectation of some representative agent $R$ who has teh correct
beliefs. In particular,
\begin{align*}
  df_t^i
  &=
  -\lambda_f\big(f_t^i-(\mu+I_i)\big)\;dt
  + \sigma_f\big(dZ_f(t) + \alpha \;dt\big)
  \\
  &=
  -\lambda_f\left(
    f_t^i-\mu-I_i- \frac{\sigma_f}{\lambda_f} \alpha\right)\;dt
  + \sigma_fdZ_f(t)
\end{align*}
From this, we see that if we choose
\begin{align*}
  \alpha
  =
  - \frac{\lambda_f}{\sigma_f} I_i
\end{align*}
then the beliefs of agent $i$ will coincide with the truth.
That implies the following RN derivative to change the mesaure
\begin{align*}
  \frac{d\xi_{t,i}}{\xi_{t,i}}
  &=
  \frac{\lambda_f}{\sigma_f} I_i
  dZ_f(t)
\end{align*}
Because complete markets, social planner solves
\begin{align*}
  \max_{c_{t,i}}
  \sum_{i=1}^I
  \xi_{t,i}
  \frac{c_{t,i}^{1-\gamma}}{1-\gamma}
  -\lambda_t
  \left(
  \sum_{i=1}^I c_{t,i}-D_t
  \right)
\end{align*}
First order condition for $c_{t,i}$:
\begin{align*}
  \xi_{t,i}
  c_{t,i}^{-\gamma}
  &=
  \lambda_t
  \quad\implies\quad
  c_{t,i}
  =
  \lambda_t^{-1/\gamma}
  \xi_{t,i}^{1/\gamma}
\end{align*}
Plug back into market clearing
\begin{align*}
  D_t
  &=
  \sum_{i=1}^I c_{t,i}
  =
  \sum_{i=1}^I
  \lambda_t^{-1/\gamma}
  \xi_{t,i}^{1/\gamma}
  \quad\implies\quad
  \lambda_t
  =
  D_t^{-\gamma}
  \left(
  \sum_{i=1}^I
  \xi_{t,i}^{1/\gamma}
  \right)^{\gamma}
\end{align*}
Note that $\lambda_t$ is the SDF, i.e. marginal utility.
It's evolution in agent $R$'s probability measure can be derived by
Ito's Lemma. But first recall all the processes:
\begin{align*}
  \frac{dD_t}{D_t}
  &=
  f_t\;dt + \sigma dZ_t
  \\
  df_t
  &=
  -\lambda_f
  (f_t-\mu)dt
  + \sigma_f dZ_f(t)
  \\
  \frac{d\xi_{t,i}}{\xi_{t,i}}
  &=
  \frac{\lambda_f}{\sigma_f} I_i
  dZ_f(t)
  \qquad i = 1,\ldots,I
\end{align*}
Then by Ito
\begin{align*}
  d\lambda_t
  &=
  \frac{d\lambda}{dD_t}
  dD_t
  +
  \frac{1}{2}
  \frac{d^2\lambda}{dD_t^2}
  (dD_t)^2
  +
  \sum_{i=1}^I
  \frac{d\lambda}{d\xi_{t,i}}
  d\xi_{t,i}
  +
  \frac{1}{2}
  \sum_{i=1}^I
  \frac{d^2\lambda}{d\xi_{t,i}^2}
  (d\xi_{t,i})^2
  \\
  &=
  -\gamma
  \lambda_t
  \frac{dD_t}{D_t}
  +
  \gamma(\gamma+1)
  \frac{1}{2}
  \lambda_t
  \left(
  \frac{dD_t}{D_t}
  \right)^2
  +
  \lambda_t
  \sum_{i=1}^I
  \left(
  \frac{\xi_{t,i}^{1/\gamma}}{\sum_{i=1}^I \xi_{t,i}}
  \right)
  \frac{d\xi_{t,i}}{\xi_{i,t}}
  \\
  &\qquad
  +
  \frac{\lambda_t}{2}
  \frac{\gamma-1}{\gamma}
  \sum_{i=1}^I
  \left[
    \left(
    \frac{\xi_{t,i}^{1/\gamma}}{\sum_{i=1}^I \xi_{t,i}}
    \right)^2
    -
    \left(
    \frac{\xi_{t,i}^{1/\gamma}}{\sum_{i=1}^I \xi_{t,i}}
    \right)
  \right]
  \left(
  \frac{d\xi_{t,i}}{\xi_{i,t}}
  \right)^2
\end{align*}
Define consumption shares
\begin{align*}
  \eta_{t,i}
  =
  \frac{c_{t,i}}{D_t}
  =
  \frac{\xi_{t,i}^{1/\gamma}}{\sum_{i=1}^I \xi_{t,i}}
\end{align*}
Then rewrite Ito thing
\begin{align*}
  d\lambda_t
  &=
  -\gamma
  \lambda_t
  \frac{dD_t}{D_t}
  +
  \gamma(\gamma+1)
  \frac{\lambda_t }{2}
  \left(
  \frac{dD_t}{D_t}
  \right)^2
  +
  \lambda_t
  \sum_{i=1}^I
  \eta_{t,i}
  \frac{d\xi_{t,i}}{\xi_{i,t}}
  +
  \frac{\lambda_t}{2}
  \frac{\gamma-1}{\gamma}
  \sum_{i=1}^I
  \left[
    \eta_{t,i}^2
    -
    \eta_{t,i}
  \right]
  \left(
  \frac{d\xi_{t,i}}{\xi_{i,t}}
  \right)^2
\end{align*}
Substitute in the proceses for $D_t$ and $\xi_{t,i}$ defined/derived
above
\begin{align*}
  d\lambda_t
  &=
  -\gamma
  \lambda_t
  \left[
    f_t\;dt + \sigma \;dZ_t
  \right]
  +
  \frac{\gamma(\gamma+1)\lambda_t\sigma^2}{2}
  \;dt
  +
  \lambda_t
  \frac{\lambda_f}{\sigma_f}
  \left(
  \sum_{i=1}^I
  \eta_{t,i}
  I_i
  \right)
  dZ_f(t)
  \\
  &\qquad
  +
  \left[
  \frac{\lambda_t}{2}
  \frac{\gamma-1}{\gamma}
  \frac{\lambda_f^2}{\sigma^2_f}
  \sum_{i=1}^I
  \left(
    \eta_{t,i}^2
    -
    \eta_{t,i}
  \right)
  I_i^2
  \right]
  \;dt
\end{align*}
Divide through by $\lambda_t$ and group terms
\begin{align*}
  \frac{d\lambda_t}{\lambda_t}
  &=
  \left[
    -\gamma
    f_t
    + \frac{\gamma(\gamma+1)\sigma^2}{2}
    +
    \frac{1}{2}
    \frac{\gamma-1}{\gamma}
    \frac{\lambda_f^2}{\sigma^2_f}
    \sum_{i=1}^I
    \left(
      \eta_{t,i}^2
      -
      \eta_{t,i}
    \right)
    I_i^2
  \right]
  \;dt
  - \gamma \sigma \;dZ_t
  +
  \frac{\lambda_f}{\sigma_f}
  \left(
  \sum_{i=1}^I
  \eta_{t,i}
  I_i
  \right)
  dZ_f(t)
\end{align*}
Thus, the risk free rate is
\begin{align*}
  r_t^f\;dt
  =
  \E
  \left[
  -\frac{d\lambda_t}{\lambda_t}
  \right]
  =
  \left[
    \gamma
    f_t
    - \frac{\gamma(\gamma+1)\sigma^2}{2}
    -
    \frac{1}{2}
    \frac{\gamma-1}{\gamma}
    \frac{\lambda_f^2}{\sigma^2_f}
    \sum_{i=1}^I
    \left(
      \eta_{t,i}^2
      -
      \eta_{t,i}
    \right)
    I_i^2
  \right]
  \;dt
\end{align*}
Hence, we can rewrite
\begin{align*}
  \frac{d\lambda_t}{\lambda_t}
  &=
  -r_t^f\;dt
  jk- \gamma \sigma \;dZ_t
  +
  \frac{\lambda_f}{\sigma_f}
  \left(
  \sum_{i=1}^I
  \eta_{t,i}
  I_i
  \right)
  dZ_f(t)
\end{align*}


















%\subsection{Welfare Analysis}




\clearpage
\section{Equilibrium with Asymmetric Information}

\subsection{CARA Gaussian Problem}

Suppose an agent chooses a portfolio to maximize CARA utility over
terminal wealth $W$. If all returns are Gaussian, than terminal wealth
$W$ will also be Gaussian and the utility maximization problem
simplifies to
\begin{align*}
  \max\; \E[-\exp\{\rho W\}]
  =
  \max \; \E[W] - \frac{\rho}{2} \Var(W)
\end{align*}
where $\rho$ is the risk aversion parameter which is also inverse risk
tolerance, $\eta=1/\rho$.
Note that the expectation and variance can be conditional on some
information set.

Suppose there are two assets available to trade: risk-free and risky,
where
\begin{table}[htbp!]
\centering
\begin{tabular}{c|cccc}
  & Endowment & Holdings after Trading & price per Unit & payoff per unit \\\hline\hline
  Risk-free & $b_0$ & $b$ & 1 & $R_f$ \\
  Risky & $x_0$ & $x$ & $p$ & $R\sim\calN$ \\
\end{tabular}
\end{table}
With this setup, the pre-trading budget constraint is
Budget constraint
\begin{align*}
  W_0 = b_0 + px_0 = b + px
  \quad\implies\quad
  b = b_0 + p(x_0 - x)
\end{align*}
End of period wealth is then given by
\begin{align*}
  W
  &= R_f b + R x
  = R_f [b_0 + p(x_0 - x)] + R x
  \quad \implies\quad
  \begin{cases}
    \E[W]
    \;\;\;= R_f [b_0 + p(x_0 - x)] + \E[R] x
    \\
    \Var[W]
    = x^2\Var(R)
  \end{cases}
\end{align*}
Hence, agents are maximizing
\begin{align*}
  \E[W]
  -\frac{\rho}{2}\Var(W)
  &=
  R_f [b_0 + p(x_0 - x)] + \E[R] x
  - \frac{\rho}{2} x^2\Var(R)
  \\
  \text{FOC w.r.t. $x$:}\qquad
  0 &= -R_fp + \E[R] - \rho x\Var[R]
  \\
  \quad\implies\quad
  x &= \frac{\E[R] - R_fp}{\rho\Var[R]}
\end{align*}


\subsection{Symmetric Info Benchmark}

Suppose $I$ investors like the type above, each with their own risk
aversion coefficient $\rho^i$, but common beliefs so everyone agrees
that $R\sim\calN(\mu,\sigma^2)$.
Letting $X$ denote the exogenously given aggregate supply of the asset,
we can solve for price via market clearing
\begin{align*}
  X =
  \sum_{i=1}^I
  x^i
  =
  \sum_{i=1}^I
  \frac{\E[R] - R_fp}{\rho^i\Var[R]}
  \quad\implies\quad
  \boxed{%
    p
    =
    \frac{1}{R_f}
    \left[
    \E[R]
    -
    X
    \frac{\Var[R]}{ \sum_{i=1}^I \frac{1}{\rho^i} }
    \right]
  }
\end{align*}
Then, in equilibrium, demand satisfies
\begin{align*}
  x^i &=
  \frac{\E[R] - R_fp}{\rho^i\Var[R]}
  =
  \frac{X}{ \rho^i\sum_{i=1}^I \frac{1}{\rho^i} }
\end{align*}


\subsection{Grossman (1976)}

Suppose $I$ investors like the type above, each with their own risk
aversion coefficient $\rho^i$, but \emph{heterogeneous} beliefs due to
private signals correlated with the random fundamental return:
\begin{align*}
  R &\sim \calN(\mu,\sigma^2_R) \\
  s^i &\sim \calN(R, \sigma^2_s)
\end{align*}
We now compute a Rational Expectations Equilibrium:
\begin{itemize}
  \item Conjecture an equilibrium pricing function given all the signals
    \begin{align*}
      p = \alpha_0 + \alpha_1
      S
      \qquad\text{where}\quad
      S
      :=
      \frac{1}{I}\sum_{i=1}^I s^i
      \sim
      \calN\big(R, \sigma^2_{S}\big)
      \bigg)
      \qquad
      \sigma^2_{S}
      =
      \frac{\sigma^2_s}{I}
    \end{align*}

  \item
    Compute Posteriors:
    This is necessary to compute individual demand.
    Agents see their own signal $s^i$ and the price $p$, so their
    information set is at least $(s^i,p)$.
    But in a REE, they \emph{also} know the pricing function and the
    coefficients on the pricing function, so their information set is
    actually $(s^i,p,S)$ since they can infer $S$.
    And that's actually a better signal of returns $R$ than anything.
    So the posterior mean and variance (which are all we need because of
    Gaussianity) satisfy
    \begin{align*}
      \E[R|s_i,p]
      = \E[R|s_i,p,S]
      = \E[R|S]
      &=
      \frac{\tau_R}{\tau_R+\tau_{S}}
      \;\mu
      + \frac{\tau_{S}}{\tau_R+\tau_{S}}
      \;
      S
      \\
      \Var(R|s_i,p)
      =
      \Var(R|S)
      &=
      \frac{1}{\tau_R+\tau_{S}}
    \end{align*}

  \item
    Derive Individual Demand:
    Given the posterior mean and variance for each agent (which happen
    to be the same), we can derive individual demand, which follows the
    same form as always
    \begin{align*}
      x_i
      =
      \frac{\E[R|S]-R_fp}{\rho^i \Var(R|S)}
      =
      \frac{1 }{\rho^i}
      \left[
        \frac{\tau_R}{\tau_R+\tau_{S}}
        \;\mu
        + \frac{\tau_{S}}{\tau_R+\tau_{S}}
        \;
        S
        -R_fp
      \right]
      (\tau_R+\tau_{S})
    \end{align*}


  \item
    Impose market clearing:
    Letting $X$ denote the exogenously given aggregate supply of the
    asset, impose market clearing
    \begin{align*}
      X
      =
      \sum_{i=1}^I x_i
      &=
      \sum_{i=1}^I
      \frac{\E[R|S]-R_fp}{\rho^i \Var(R|S)}
      %\\
      =
      \sum_{i=1}^I
      \frac{1 }{\rho^i}
      \left[
        \frac{\tau_R}{\tau_R+\tau_{S}}
        \;\mu
        + \frac{\tau_{S}}{\tau_R+\tau_{S}}
        \;
        S
        -R_fp
      \right]
      (\tau_R+\tau_{S})
    \end{align*}
    Solve for price
    \begin{align*}
      p
      &=
      \frac{1}{R^f(\tau_R+\tau_S)}
      \left[
      \tau_R \mu
      - X
      \left(
      \sum_{i=1}^I
      \frac{1 }{\rho^i}
      \right)^{-1}
      + \tau_{S} S
      \right]
    \end{align*}
  \item
    Impose Rationality:
    Sub in conjectured pricing function on the LHS, identify
    coefficients
    \begin{align*}
      \alpha_0 + \alpha_1 S
      &=
      \frac{1}{R^f}
      \left[
      \frac{\tau_R}{\tau_R+\tau_S}
       \mu
      -
      \frac{X}{\tau_R+\tau_S}
      \left(
      \sum_{i=1}^I
      \frac{1 }{\rho^i}
      \right)^{-1}
      \right]
      +
      \frac{1}{R^f}
      \frac{\tau_{S}}{\tau_R+\tau_S}
      S
    \end{align*}




\end{itemize}




%\clearpage
%\subsection{Grossman and Stiglitz (1980, AER)}



\clearpage
\subsection{Hellwig (1980, AER)}

Suppose $I$ traders, $\rho$ for investor $i$, and information
dispersed among traders
\begin{align*}
  s^i|R \sim \calN\big(R,(\sigma_\varepsilon^i)^2\big)
\end{align*}
Let $\Delta s^i = s^i-\E[s^i]$, $\Delta X = X-\E[X]$, etc.
\begin{itemize}
  \item
    Conjecture an equilibrium pricing function
    $p = \alpha_0 + \sum_i\alpha_s^i \Delta s^i + \alpha_x \Delta X$

  \item
    Compute Posteriors:
    Everything Gaussian, so just need to pin down mean and variance.
    Moreover, the conditional expectation will be a linear function of
    the random variables in the conditioning information set
    \begin{align*}
      \E[R|s^i,p]
      &=
      \E[R]
      + \beta_s^i \Delta s^i
      + \beta_p^i \Delta p
      \quad\qquad
      \Var[R|s^i,p]
      =
      \frac{1}{\tau^i_{v|s^i,p}}
    \end{align*}
    where $\beta_s^i$ and $\beta_p^i$ are the projection coefficients
    derived from the conditional Gaussian formulas.
    Since they will depend on estimates of $\alpha$, the actual values
    must be determined later, once we determine the coefficients, but we
    know that they will be well-defined by the end of this.

  \item
    Derive individual demand, letting $\eta=1/\rho$:
    \begin{align*}
      x^i
      =
      \eta^i \tau^i_{v|s^i,p}
      \big(
        \E[R|s^i,p]
        - R_fp
      \big)
      =
      \sum_i
      \eta^i \tau^i_{v|s^i,p}
      \big(
        \E[R]
        + \beta_s^i \Delta s^i
        + \beta_p^i \Delta p
        - R_fp
      \big)
    \end{align*}

  \item
    Impose Market Clearing:
    \begin{align*}
      X
      = \sum_i x^i
      &=
      \sum_i
      \eta^i \tau^i_{v|s^i,p}
      \big(
        \E[R]
        + \beta_s^i \Delta s^i
        + \beta_p^i \Delta p
        - R_fp
      \big)
    \end{align*}
    Use the fact that $\E[p]=\alpha_0$ so $\Delta p = (p-\alpha_0)$ to
    get
    \begin{align*}
      X
      %&=
      %\sum_i
      %\eta^i \tau^i_{v|s^i,p}
      %\big(
        %\E[R]
        %+ \beta_s^i \Delta s^i
        %+ \beta_p^i (p-\alpha_0)
        %- R_fp
      %\big)
      %\\
      &=
      \sum_i
      \eta^i \tau^i_{v|s^i,p}
      \big(
        \E[R]
        + \beta_s^i \Delta s^i
        - \beta_p^i \alpha_0
        + (\beta_p^i-R_f) p
      \big)
    \end{align*}
    Solve for $p$
    \begin{align*}
      p
      &=
      \frac{%
        -X
        +
        \sum_i
        \eta^i \tau^i_{v|s^i,p}
        \big(
          \E[R]
          + \beta_s^i \Delta s^i
          - \beta_p^i \alpha_0
        \big)
      }{%
        \sum_i
        \eta^i \tau^i_{v|s^i,p}
        (R_f-\beta_p^i)
      }
    \end{align*}

  \item
    Impose Rationality:
    Substitute in $X=\E[X]+\Delta X$, rearrange/breakup price expression
    \begin{align*}
      p
      &=
      \underbrace{%
        \frac{%
          -\E[X]
          +
          \sum_i
          \eta^i \tau^i_{v|s^i,p}
          \big(
            \E[R]
            - \beta_p^i \alpha_0
          \big)
        }{%
          \sum_i
          \eta^i \tau^i_{v|s^i,p}
          (R_f-\beta_p^i)
        }
      }_{\alpha_0}
      +
      \sum_i
      \underbrace{%
        \left(
        \frac{%
          \eta^i \tau^i_{v|s^i,p}
          \beta_s^i
        }{%
          \sum_j
          \eta^j \tau^j_{v|s^j,p}
          (R_f-\beta_p^j)
        }
        \right)
      }_{\alpha_s^i}
      \Delta s^i
      -
      \underbrace{%
        \frac{%
          1
        }{%
          \sum_i
          \eta^i \tau^i_{v|s^i,p}
          (R_f-\beta_p^i)
        }
      }_{\alpha_x}
      \Delta X
    \end{align*}
    %Identifying coefficients, that means the $\alpha$ coefficients
    %satisfy
    %\begin{align*}
      %\alpha_0
      %&=
      %\frac{%
        %-\E[X]
        %+
        %\sum_i
        %\eta^i \tau^i_{v|s^i,p}
        %\big(
          %\E[R]
          %- \beta_p^i \alpha_0
        %\big)
      %}{%
        %\sum_i
        %\eta^i \tau^i_{v|s^i,p}
        %(R_f-\beta_p^i)
      %}
      %\\
      %\alpha_s^i
      %&=
      %\frac{%
        %\eta^i \tau^i_{v|s^i,p}
      %}{%
        %\sum_j
        %\eta^j \tau^j_{v|s^j,p}
        %(R_f-\beta_p^j)
      %}
      %\beta_s^i
      %\\
      %\alpha_x
      %&=
      %-
      %\frac{%
        %1
      %}{%
        %\sum_i
        %\eta^i \tau^i_{v|s^i,p}
        %(R_f-\beta_p^i)
      %}
    %\end{align*}
\end{itemize}
Special case, $\rho^i=\rho$ for all $i$ so $\eta_i=\eta$ for all $i$.
Also assume $\sigma_\varepsilon^i=\sigma_\varepsilon$ for all $i$.
\begin{itemize}
  \item
    Equilibrium pricing function conjecture becomes
    \begin{align*}
      p =
      \alpha_0
      + \alpha_s \frac{1}{I}\sum_i \Delta s^i
      + \alpha_x \Delta X
    \end{align*}
  \item
    Compute Posteriors:
    With common signal variance, everyone puts the same weights on
    objects, hence the mean and variance simplify to
    \begin{align*}
      \E[R|s^i,p]
      &=
      \E[R]
      + \beta_s \Delta s^i
      + \beta_p \Delta p
      \\
      \Var[R|s^i,p]
      &=
      \frac{1}{\tau}
    \end{align*}
    %where $\beta_s^i$ and $\beta_p^i$ are the projection coefficients
    %derived from the conditional Gaussian formulas.
    %Since they will depend on estimates of $\alpha$, the actual values
    %must be determined later, once we determine the coefficients, but we
    %know that they will be well-defined by the end of this.

  \item
    Individual demand simplifies to
    $x^i = \eta \tau \big( \E[R|s^i,p] - R_fp \big)$

  \item
    %Market clearing
    %\begin{align*}
      %X
      %=
      %\sum_i x^i
      %&=
      %\eta \tau
      %\sum_i
      %\big( \E[R|s^i,p] - R_fp \big)
      %\\
      %&=
      %\eta \tau
      %\sum_i
      %\big(
        %\E[R]
        %+ \beta_s \Delta s^i
        %- \alpha_0\beta_p
        %+ p(\beta_p - R_f)
      %\big)
      %%\\
      %%\implies\quad
      %%p
      %%\sum_i
      %%\eta \tau
      %%(R_f-\beta_p
      %%&=
      %%-X
      %%+
      %%\eta \tau
      %%\sum_i
      %%\big(
        %%\E[R]
        %%+ \beta_s \Delta s^i
        %%- \alpha_0\beta_p
      %%\big)
    %\end{align*}
    Market clearing simplifies the expression for $p$
    \begin{align*}
      p
      &=
      \frac{%
        -X
        +
        \eta \tau
        \sum_i
        \big(
          \E[R]
          + \beta_s \Delta s^i
          - \beta_p \alpha_0
        \big)
      }{%
        \eta \tau
        \sum_i
        (R_f-\beta_p)
      }
    \end{align*}

  \item
    Impose Rationality:
    Substitute in $X=\E[X]+\Delta X$, rearrange/breakup price expression
    \begin{align*}
      p
      &=
      \underbrace{%
        \frac{%
          -\E[X]
          +
          \eta \tau
          \sum_i
          \big(
            \E[R]
            - \beta_p \alpha_0
          \big)
        }{%
          \eta \tau
          \sum_i
          (R_f-\beta_p)
        }
      }_{\alpha_0}
      +
      \underbrace{%
        \frac{%
          I\beta_s
        }{%
          \sum_j
          (R_f-\beta_p)
        }
      }_{\alpha_s}
      \frac{1}{I}
      \sum_i
      \Delta s^i
      +
      \underbrace{%
        \frac{%
          - 1
        }{%
          \eta \tau
          \sum_i
          (R_f-\beta_p)
        }
      }_{\alpha_x}
      \Delta X
    \end{align*}
\end{itemize}
We now derive coefficients through a sequence of steps
\begin{itemize}
  \item
    Recall the assumptions for random variables
    \begin{align*}
      R\sim\calN(\mu,\sigma^2_R)
      \qquad
      s^i|R \sim\calN(R,\sigma^2_\varepsilon)
      \qquad
      X\sim\calN(\overline{X},\sigma^2_x)
    \end{align*}
    Together, these imply that price has variance
    \begin{alignat*}{3}
      \Var(s_i)
      &=\Var(R+\varepsilon)
      = \Var(R)+\Var(\varepsilon)
      &&=\sigma^2_R+\sigma^2_\varepsilon
      \\
      \Cov(s^i,R)
      &=
      \Cov(R+\varepsilon,R)
      &&= \sigma^2_R
      \\
      \Var(p)
      &=
      \Var\left(
        \alpha_s\frac{1}{I}\sum_i\Delta s^i
        +\alpha_x\Delta X
      \right)
      \\
      &=
      \alpha_s^2
      \Var\left(
        R
        +\frac{1}{I}
        \sum_i
        \varepsilon^i
      \right)
      +\alpha_x^2\Var\left(
        \Delta X
      \right)
      &&=
      \alpha_s^2 \left(
        \sigma^2_R+\frac{\sigma^2_\varepsilon}{I}
      \right)
      + \alpha_x^2 \sigma^2_x
      \\
      \Cov( s^i, p)
      &=
      \Cov\left(s^i,\alpha_s \frac{1}{I}\sum_j\Delta s^j\right)
      =
      \alpha_s
      \Cov\left(
        R+\varepsilon^i,\;
        R
        +
        \frac{1}{I}
        \sum_j \varepsilon^j
      \right)
      \\
      &
      =
      \alpha_s
      \Var(R)
      + \frac{\alpha_s}{I}\Var(\varepsilon^i)
      &&=
      \alpha_s\left(\sigma^2_R+\frac{\sigma^2_\varepsilon}{I}\right)
      \\
      \Cov(R,\Delta p)
      &=
      \Cov\left(R,\alpha_s\frac{1}{I}\sum_i \Delta s_i\right)
      = \alpha_s\frac{1}{I}\sum_i \Cov\left(R,\Delta s_i\right)
      &&= \alpha_s\sigma^2_R
    \end{alignat*}

  \item
    First, derive $\beta$ projection coefficients in terms of $\alpha$
    coefficients. To do that, note by the results in the last bullet
    point and the joint normality of everything
    \begin{align*}
      \begin{pmatrix}
        R \\  \Delta s^i \\ \Delta p
      \end{pmatrix}
      \sim
      \calN
      \left(
      \begin{pmatrix}
        \mu \\  0 \\ 0
      \end{pmatrix},\;
      \begin{pmatrix}
        \sigma^2_R
          & \sigma^2_R
          & \alpha_s\sigma^2_R\\
          & \sigma^2_R+\sigma^2_\varepsilon
          & \alpha_s
          \left(\sigma^2_R+\frac{\sigma^2_\varepsilon}{I}\right)\\
          &
          &
          \alpha_s^2
          \left(\sigma^2_R+\frac{\sigma^2_\varepsilon}{I}\right) + \alpha_x^2 \sigma^2_x
      \end{pmatrix}
      \right)
    \end{align*}
    By the usual normal formulas, we get $\beta_s$, $\beta_p$, $\tau$.

\end{itemize}




\clearpage
\subsection{Kyle 1985}

Notes based on
\href{http://www.alexchinco.com/comparing-kyle-and-grossman-stiglitz/}{this}.









%% APPPENDIX %%

% \appendix




\end{document}


%%%%%%%%%%%%%%%%%%%%%%%%%%%%%%%%%%%%%%%%%%%%%%%%%%%%%%%%%%%%%%%%%%%%%%%%
%%%%%%%%%%%%%%%%%%%%%%%%%%%%%%%%%%%%%%%%%%%%%%%%%%%%%%%%%%%%%%%%%%%%%%%%
%%%%%%%%%%%%%%%%%%%%%%%%%%%%%%%%%%%%%%%%%%%%%%%%%%%%%%%%%%%%%%%%%%%%%%%%

%%%% SAMPLE CODE %%%%%%%%%%%%%%%%%%%%%%%%%%%%%%%%%%%%%%

    %% VIEW LAYOUT %%

        \layout

    %% LANDSCAPE PAGE %%

        \begin{landscape}
        \end{landscape}

    %% BIBLIOGRAPHIES %%

        \cite{LabelInSourcesFile}  %Use in text; cites
        \citep{LabelInSourcesFile} %Use in text; cites in parens

        \nocite{LabelInSourceFile} % Includes in refs w/o specific citation
        \bibliographystyle{apalike}  % Or some other style

        % To ditch the ``References'' header
        \begingroup
        \renewcommand{\section}[2]{}
        \endgroup

        \bibliography{sources} % where sources.bib has all the citation info

    %% SPACING %%

        \vspace{1in}
        \hspace{1in}

    %% URLS, EMAIL, AND LOCAL FILES %%

      \url{url}
      \href{url}{name}
      \href{mailto:mcocci@raidenlovessusie.com}{name}
      \href{run:/path/to/file.pdf}{name}


    %% INCLUDING PDF PAGE %%

        \includepdf{file.pdf}


    %% INCLUDING CODE %%

        %\verbatiminput{file.ext}
            %   Includes verbatim text from the file

        \texttt{text}
            %   Renders text in courier, or code-like, font

        \matlabcode{file.m}
            %   Includes Matlab code with colors and line numbers

        \lstset{style=bash}
        \begin{lstlisting}
        \end{lstlisting}
            % Inline code rendering


    %% INCLUDING FIGURES %%

        % Basic Figure with size scaling
            \begin{figure}[h!]
               \centering
               \includegraphics[scale=1]{file.pdf}
            \end{figure}

        % Basic Figure with specific height
            \begin{figure}[h!]
               \centering
               \includegraphics[height=5in, width=5in]{file.pdf}
            \end{figure}

        % Figure with cropping, where the order for trimming is  L, B, R, T
            \begin{figure}
               \centering
               \includegraphics[trim={1cm, 1cm, 1cm, 1cm}, clip]{file.pdf}
            \end{figure}

        % Side by Side figures: Use the tabular environment


