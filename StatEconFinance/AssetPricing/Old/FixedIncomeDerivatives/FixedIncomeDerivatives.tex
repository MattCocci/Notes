\documentclass[a4paper,12pt]{scrartcl}

\author{Matthew Cocci}
\title{Notes to Financial Engineering: \\Fixed Income Derivatives}
\date{}
\usepackage{enumitem} %Has to do with enumeration	
\usepackage{amsfonts}
\usepackage{amsmath}
\usepackage{amsthm} %allows for labeling of theorems
\usepackage[T1]{fontenc}
\usepackage[utf8]{inputenc}
\usepackage{blindtext}
\usepackage{graphicx}
\usepackage[hidelinks]{hyperref} 
%\numberwithin{equation}{section} 
%, This labels the equations in relation to the sections rather than other equations
%\numberwithin{equation}{subsection} %This labels relative to subsections
\newtheorem{thm}{Theorem}[section]
\newtheorem{lem}[thm]{Lemma}
\newtheorem{prop}[thm]{Proposition}
\newtheorem{cor}[thm]{Corollary}
\setkomafont{disposition}{\normalfont\bfseries}
\usepackage{appendix}
\usepackage{subfigure} % For plotting multiple figures at once




\begin{document}
\maketitle

\tableofcontents

\newpage
\section{The Fundamental Theorem of Asset Pricing}

\subsection{Introduction}

Derivatives require special pricing techniques aside from the traditional
discounted cash flow (DCF) approach, as DCF requires an estimate of the
appropriate risk-adjusted rate of return.  However, the risk of a 
derivative varies over time, which makes it difficult to estimate the
derivative's risk-adjusted return.

As a result, derivatives pricing turns to the no-arbitrage approach (NA),
which eliminates the need to build risk into the model.  

\subsection{Trading Strategy and Derivative Pricing Definitions}

A \textbf{trading strategy} is a dynamically-rebalanced portfolio.
\\
\\
A trading strategy is \textbf{self-financing} if it generates no 
intermediate cash inflows and requires no intermediate outflows between
the time the portfolio is initiated and the time it is liquidated.  This
implies that
   \begin{itemize}
      \item[i.]{All dividends are reinvestd.}
      \item[ii.]{Value of the assets sold at a rebalance time must 
	 equal the value of the assets bought.}
   \end{itemize}
A trading strategy is \textbf{strictly positive} if the value of the 
traded portfolio can never become zero or negative.
\\
\\ 
Let $N$ be the value of a \emph{strictly positive}, \emph{self-financing}
trading strategy.  Then $N$ is a \textbf{numeraire process} or, simply,
a \textbf{numeraire}. Here are a few examples:
\begin{itemize}
   \item[-]{Price of Dividend paying asset: 
	 NO, as there are intermediate 
	 cash outflows, violating self-financing condition.}
   \item[-]{The price of a forward contract: NO, as it can go negative,
      violating the strictly positive condition.}
   \item[-]{Price of a Foreign Currency: 
      NO, as it is equivalent to a dividend
      paying asset because you think of it as an investment in an
      interest-bearing account.}
   \item[-]{Price of a non-defaultable zero-coupon bond: YES.}
   \item[-]{Value of a money market account earning the risk free rate,
      where there are no interim deposits or withdrawals: YES.}
\end{itemize}

\subsection{Martingales and Change of Measure}

A \textbf{martingale} is a stochastic process
$X$ with the property 
   \[ E_t[X(T) - X(t)] = 0 \; \Leftrightarrow \; E_t[X(T)] = X(t), \qquad
      T > t. \]
A \textbf{probability measure} is a specification
of the probabilities of all the possible states of the words, mapping
states to real numbers.
\\
\\
Suppose that $\xi$ is a nonnegative random variable on $(\Omega,
\mathcal{F}, P)$ with $E_P[\xi] = 1$. (The subscript $P$ highlights that
the last expectation is with respect to measure $P$.) 
Then define a new measure
   \[ Q: \mathcal{F} \rightarrow [0,1] \]
\begin{equation}
   \label{rnt}
   Q(A) = E\left[1_A \xi\right]=\int_A\xi(\omega) dP(\omega), \qquad
      A\in \mathcal{F} 
\end{equation}
Clearly, $Q$ is a probability measure on $(\Omega, \mathcal{F})$ and
it is absolutely continuous with respect to $P$---i.e. we have
   \[Q(A) > 0 \Rightarrow P(A) > 0.\] 
Note that it is common to write the random variable $\xi$ as
   \[ \xi = \frac{dQ}{dP},\]
and we often refer to $\xi$ as the \emph{Radon-Nikodym derivative}
or the \emph{likelihood ratio} of $Q$ with respect to $P$.

\paragraph{Radon-Nikodym Theorem} If $P$ and $Q$ are two probability
measures on $(\Omega, \mathcal{F})$, then there \emph{will exist}
such a random variable $\xi$ so that Expression \ref{rnt} holds.


\newpage
\subsection{Fundamental Theorem of Asset Pricing (No Dividends)}

\subsubsection{Statement of Theorem}

Suppose we have $n$ non-dividend-paying assets with
price processes $S_1, S_2, \ldots, S_n$.  Let $N$ be some numeraire
process.  Then, barring market imperfection, there are no arbitrage
opportunities among these assets if and only if there exists a strictly
positive probability measure $Q_N$ (so it's dependent upon the numeraire,
$N$) under which each of the processes $S_i / N$ is a martingale.
\begin{itemize}
   \item[-]{Note that $S_i/N$ is the price of asset $i$ in units of the
      numeraire $N$.  Therefore, we call $S_i/N$ the \textbf{normalized
      price process}.}
   \item[-]{The probability measure $Q_N$ will, in general depend on
	 the numeraire.  Therefore, we call $Q_N$ the 
	 \textbf{martingale measure} or \textbf{pricing measure} 
	 associated with the numeraire $N$.} 
   \item[-]{We can paraphrase FTAP by saying that, if the is no arbitrage
	 or market imperfections, then given \emph{any} numeraire 
	 process $N$, there must exist a corresponding martingale
	 measure $Q_N$ under which the normalized price of any 
	 non-dividend paying asset is a martingale:
	    \[ \frac{S_i(t)}{N(t)} = E_t^{Q_N}\left[ \frac{S_i(T)}{N(T)}
	       \right], \qquad T > t. \]
	 From this, we see that changing $N$ will generally change 
	 $Q_N$ as well.
      }
\end{itemize}

\subsubsection{Consequences for Derivative Pricing}

Let $V$ denote the price of a derivative with payoff $V(T)$ at time $T$.
Then we can apply FTAP to get
   \[ V(t) = N(t) E_t^{Q_N} \left[ \frac{S_i(T)}{N(T)}
	       \right], \qquad T > t. \]
Note, the price we get for a deriative is \emph{invariant} to the choice
of the numeraire.

\subsubsection{Special Numeraires and Martingale Measures}

\paragraph{T-forward measure} Let $P(t,T)$ be the price of a
non-defaultable zero-coupon bond with unit face value.  Then
   \[ N(t) = P(t,T) \]
is our numeraire.  The associated martingale measure, denoted 
$Q_T$, is called the \emph{T-forward martingale measure}. This yields
a derivative price of 
   \[ V(t) = E_t^{Q_B} \left[ e^{-\int_t^T r(s) ds} V(T)\right] \]

\paragraph{Risk Neutral Measure} Let's consider the value of a money
market account with unit initial value as our numeraire.  Then
   \[ N(t) = B(t) = e^{\int_0^t r(s) ds} \]
where $r$ is the instantaneous risk-free rate.  The associated 
martingale measure, denoted by $Q_B$, is called the 
\emph{risk-neutral martingale measure}. This yields a derivative price
of 
   \[ V(t) = P(t,T) \; E_t^{Q_T}\left[V(T)\right] \; = \;
      e^{-r(t,T) (T-t)} E_t^{Q_T}\left[V(T)\right] \]
   \[ r(t,T) = -\ln{P(t,T)}/(T-t) \]
If interest rates are stochastic (and they probably are), then this
measure isn't as convenient a tthe $T$-forward measure.

\subsection{FTAP for Dividend-Paying Assets}

Consider an asset with price process $S$ and let $D(t)$ denote the
cumulative dividend paid by the asset from time 0 up to time $t$.
We can consider the undiscounted cash flows from holding an asset
from $t$ to $T$:
   \[ S(T) - S(t) + D(T) - D(t) = GP(T) - GP(t) \]
where $GP(t) = S(t) + D(t)$ is the asset's gain process.
\\
\\
Given a numeraire $N$, the asset's \emph{normalized gain process},
denoted NGP, measures the gains from holding the assets in units of $N$:
\[ NGP(t) = \frac{S(t)}{N(t)} + \int^t_0 \frac{dD(s)}{N(s)}\]
where $dD(s)$ is the dividend paid by the asset at time $s$.

\paragraph{Theorem} Now, let's restate the fundamental theorem of asset
pricing allowing for dividend paying assets.  So again, consider
assets with price processes $S_1, \ldots, S_n$ and cumulative dividend
processes $D_1,\ldots,D_n$, letting $N$ be any numeraire process.
Then there are no arbitrage opportunities across these assets if and
only if there exists a strictly positive probability measure $Q_N$ 
under which each
   \[ \frac{S_i(t)}{N(t)} + \int^t_0 \frac{dD_i(s)}{N(s)} \]
is a martingale. This implies
\[ \frac{S_i(t)}{N(t)}  = E_t^{Q_N}\left[ \frac{S_i(T)}{N(T)} +
   \int^T_t \frac{dD_i(s)}{N(s)}\right] \]
And so the normalized price of any asset is equalt to the conditional
expectation under the martingale measure of the assets normalized payoffs
(including future dividends).

\newpage


\section{Continuous Time Stochastic Processes}

Here, we develop the necessary machinery in continuous time stochastic
processes to model asset price evolution properly and with sufficient
richness and generality. In particular, we discuss a heirarchy of model
classes that includes Brownian Motion $\subset$ Generalized Brownian
Motion $\subset$ Diffusions $\subset$ Ito Processes.

\subsection{Introduction}

A \emph{stochastic process} $X$ is a collection of random variables
indexed by time: $X = \{ X_t: t \in \mathcal{T} \}$. 
\begin{itemize}
   \item[-]{\emph{Discrete Time}: $\mathcal{T}$ countable, and process
      changes only at discrete time intervals.}
   \item[-]{\emph{Continuous Time}: $\mathcal{T}$ uncountable.}
\end{itemize}
\paragraph{Definition} A process $X$ has stationary increments if 
$X_T - X_t$ has the same distribution as $X_{T'} - X_{t'}$ provided
that $T-t= T'-t'$.

\subsection{Brownian Motion}

\paragraph{Definition} The most basic continuous-time process is 
\emph{Brownian Motion} (or the \emph{Wiener Process}).  It has three
defining properties:
\begin{enumerate}
   \item[i.]{$W(0) = O$.}
   \item[ii.]{$W(t)$ is continuous, so no jumps.}
   \item[iii.]{Given any two times, $T>t$, the increment $W(T) - W(t)$ is
      independent of all previous history and normally distributed
      with mean $0$ and variance $T-t$.}
\end{enumerate}
A few consequences of the definition of $W(t)$:
\begin{itemize}
   \item[-]{Brownian motion has independent stationary increments.}
   \item[-]{$W(t)$ is normally distributed with $\mu =0$, $\sigma^2 = t$.
      }
\end{itemize}

\subsection{Generalized Brownian Motion}

\paragraph{Definition} A \emph{generalized Brownian motion} is a 
continuous-time process $X$ with the following property:
   \[ X(t) = X(0)+\mu t + \sigma W(t) \]
where $\mu$ is the \emph{drift}, $\sigma$ is the \emph{volatility},
and $W$ is simple Brownian motion. The differential equation 
equivalent is written:
   \[ dX(t) = \mu \; dt + \sigma \; dW(t).\]
It follows immediately from the definition that
\begin{itemize}
   \item[-]{$X(t)$ is continuous, so no jumps.}
   \item[-]{$X(t)$ is normally distributed with mean $X(0)+\mu t$, 
      variance $\sigma^2 t$.
      }
   \item[-]{Given any two times, $T>t$, the increment $X(T) - X(t)$ is
      independent of all previous history and normally distributed
      with mean $\mu(T-t)$ and variance $\sigma^2(T-t)$.}
   \item[-]{$X$ is a martingale if and only if $\mu = 0$.}
\end{itemize}
\paragraph{Theorem} It also happens that Generalized Brownian motions
are the only continuous time processes with continuous sample
paths and stationary increments.

\subsection{Ito Processes}

Even more general than Brownian Motion (which is retained as a special
case), an \emph{Ito Process} is 
a stochastic process $X$ defined by one of two equivalent formulations:
   \[ dX(t) = \mu(t) \; dt + \sigma(t) \; dW(t) \]
   \[ X_t = X_0 + \int^t_0 \mu(s) \; ds + \int^t_0 \sigma(s) \; dW(s)
      \]
for any arbitrary stochastic processes $\mu$ (the drift) and $\sigma$
(the volatility) along with some Brownian Motion $W(t)$.  Here are 
some properties
\begin{itemize}
   \item[-]{Has continuous sample paths and is a martingale if and only
      if $\mu(t) = 0$. }
   \item[-]{Increments are not necessarily stationary, as $\mu$ and
	 $\sigma$ can change \emph{randomly} with time.}
\end{itemize}

\paragraph{Definition} If the drift and volatility of an Ito process 
depend only upon the current value of the process and time, then
$X$ is a \emph{diffusion}.  Mathematically, $X$ is a \emph{diffusion}
if
   \[ dX(t) = \mu(X(t),t)\; dt+ \sigma(X(t),t)\; dW(t) \]
for some functions $\mu$ and $\sigma$.

\subsection{Ito's Lemma}

Suppose that $X$ is an Ito process defined by 
\begin{equation}
   \label{ito}
   dX(t) = \mu(t) \; dt + \sigma(t) \; dW(t)
\end{equation}
and we define a new process $Y(t) = f(X(t),t)$ where $f$ is some
function that's twice differentiable in $X$ and once in $t$.  Then
we have that
\begin{equation}
   \label{lemma}
   dY(t) = f_X(X(t),t) \; dX(t) + f_t(X(t),t) \; dt + \frac{1}{2} 
   f_{XX}(X(t),t)\sigma(t)^2 \; dt.
\end{equation}
where subscripts on $f$ denote the partial derivatives.\footnote{Note 
that Equation \ref{lemma} almost looks like the chain rule from 
traditional calculus, except for that extra term with $f_{XX}$ partial 
derivative. That arises from the additional variability due to the 
inclusion of stochastic factors like $W(t)$ in the original Ito 
Process.} Subbing Equation \ref{ito} into Equation \ref{lemma}, 
we get that
   \[ dY(t) =\left(f_X(X(t),t)\mu(t) + f_t(X(t),t)  + \frac{1}{2} 
      f_{XX}(X(t),t)\sigma(t)^2 \right) \; dt + 
      f_X(X(t),t)\sigma(t) \; dW(t)
   \]
Thus, it is clear that $Y$ is also an Ito Process by the statement
above with the drift and volatility given by the coefficients on
$dt$ and $dW(t)$ as always.

\paragraph{Using Ito's Lemma} In practice, we use Ito's Lemma
whenever we have a (typically complicated) Ito Process that we want
to solve.  Given the process $X$ and its corresponding Ito Process, 
we posit a function $f$ that could could help. Then we write a
new Ito Process using Ito's lemma with $dY(t) = df(X(t),t)$ on the LHS.
From there, hopefully we can integrate $dY(t)$ easily on the left and
solve out for $X(t)$.

\subsection{Multi-dimensional Ito's Lemma}

For the sake of completeness, let's generalize Ito's Lemma to consider
the case of a finite number of Ito processes, $X_1, X_2, \ldots, X_n$,
   \[ dX_i(t) = \mu_i(t)\; dt + \sigma_i(t) \; dW_i(t).\]
Next, define $Y(t) = f(X_1(t), \ldots, X_n(t), t)$ for some 
differentiable function $f$.  Then multi-dimensional Ito's Lemma says
\begin{align*}
    dY(t) &= \sum^n_{i=1} f_{X_i}(X_1(t), \ldots, X_n(t), t) \; 
      dX_i(t) \\
    &+ f_{t}(X_1(t), \ldots, X_n(t), t) \; dt \\
    &+ \frac{1}{2} \sum^n_{i=1}\sum^n_{j=1} 
      f_{X_i,X_j}(X_1(t), \ldots, X_n(t), t) \; \rho_{ij}\sigma_i(t)
      \sigma_j(t) \; dt 
\end{align*}
where $\rho_{ij}$ is the correlation coefficient between $dW_i$ and
$dW_j$. Note that you'll have to plug back in for the $dX_i$ in the
first sum.
\\
\\
We'll mostly consider with the two-dimensional case for the 
two specific instances below:
\begin{itemize}
   \item[i.]{$Y(t) = X_1(t)X_2(t)$, which gives us
      \[ dY(t) = X_2(t) \; dX_1(t) + X_1(t) \; dX_2(t) +
	 \rho_{12}\sigma_1(t)\sigma_2(t) \; dt \]
      Note that you'll have to plug back in for $dX_1$ and $dX_2$.
      This is a type of integration-by-parts formula because (after
      rearranging terms) it relates $X_1\; dX_2$ to $X_2\; dX_1$.
   }

   \item[ii.]{$Y(t) = X_1(t)/X_2(t)$, which gives us
      \begin{align*}
	 dY(t) = \frac{1}{X_2(t)} dX_1(t) - \frac{X_1(t)}{X_2(t)^2}
	 dX_2(t) +  \frac{X_1(t)}{X_2(t)^3} \sigma_2(t)^2 \;dt -
	 \frac{1}{X_2(t)^2} \; \rho_{12}\sigma_1(t)\sigma_2(t) \; dt. 
      \end{align*}
      Note that you'll have to plug back in for $dX_1$ and $dX_2$.
      This is a type of integration-by-parts formula because (after
      rearranging terms) it relates $X_1\; dX_2$ to $X_2\; dX_1$.
      }
\end{itemize}


\subsection{Geometric Brownian Motion}

Let's consider the process $X$ governed by
   \[ dX(t) = X(t)\mu(t) \; dt + X(t) \sigma(t) dW(t).\]
To solve, let us consider the process $Y(t) = \log X(t)$.  We compute
the partials and apply Ito's Lemma:
   \[ f_X = \frac{1}{X(t)}, \qquad f_{XX} = -\frac{1}{X(t)^2},
      \qquad f_t = 0 \]
   \[ dY(t) = \frac{1}{X(t)} dX(t) - \frac{1}{2}\frac{1}{X(t)^2}
      (\sigma(t) X(t))^2 ) \; dt \]
which simplifies (after subbing in for $dX(t)$) into the expression
   \[ dY(t) = \left( \mu(t) - \frac{1}{2}\sigma(t)^2\right) \; +
      \sigma(t) \; dW(t).\]
Next, integrating both sides and substituting back in with
$Y(t) = \log X(t)$, we get
   \[ Y(t) = Y(0) + \int^t_0 \left( \mu(s)-\frac{1}{2}\sigma(s)^2\right) 
      ds + \int^t_0 \sigma(s) \; dW(s) \]
   \[ X(t)=X(0) e^{\int^t_0 \left( \mu(s)-\frac{1}{2}\sigma(s)^2\right) 
       ds + \int^t_0 \sigma(s) \; dW(s)}\]
where it also follows that $X$ is strictly positive.

\paragraph{Special Case} Suppose that $\mu$ and $\sigma$ are constant,
in which case $X$ follows a \emph{geometric Brownian motion}.  Then
it follows that 
   \[ \log X(t) \sim N\left( \log X(0) + \left(\mu - \frac{1}{2} \sigma^2
      t\right), \sigma^2 t \right) \]
so that we say $X(t)$ is \emph{lognormally distributed}.

\subsection{Girsanov's Theorem}

\paragraph{Theorem} Suppose that $X$ is an Ito process
   \[ dX(t) = X(t) \mu(t) \; dt + X(t) \sigma(t) \; dW(t),\]
where $\mu$ an $\sigma$ are stochastic processes and $W$ is Brownian
Motion under some probability measure $P$---like maybe the real 
world measure.  Then if $Q$ is any other strictly positive probability
measure, then $X$ is also an Ito process under $Q$---i.e., there
exist processes $\hat{\mu}$, $\hat{\sigma}$, and $\hat{W}$ with the
property that
\begin{equation}
\label{Girsanov}
    dX(t) = X(t) \hat{\mu}(t) \; dt + X(t) \hat{\sigma}(t) \; 
      d\hat{W}(t).
\end{equation}
Even better, $\hat{\sigma} = \sigma$.
\\
\\
This is particularly useful because, in general, we will have to work 
with two different probability measures: the true/historical $P$ and
the martingale probability measure $Q_N$. 


\newpage
\section{Interest Rate Derivatives}

Just a quick word about terminology. We set terms at the 
\emph{contracting date} first about 
the \emph{expiration}---the date at which an option-holder must  
make a decision about whether to exercise an option like 
a cap, floor, or swaption. The \emph{maturity}
is when the option would stop paying (assuming you chose to exercise).
Finally, suppose we have a simple one-period 
derivative whose payoff at $T_{i+1}$ depends
on the rate at $T_i$. Then $T_i$ is the so-called \emph{fixing date},
while $T_{i+1}$ is the maturity.



\subsection{Basics and Notation}
Before we begin, let's define a few important terms, assets, 
and concepts.
Throughout, we will use the convention that lowercase rates
denote \emph{continuous compounding} while uppercase rates represents
\emph{simple interest}.  
\begin{itemize}
   \item[-] {\sl Zero Coupon Bond}: Defined by $P(t,T)$, 
      it represents the 
      price at time $t$ of a non-defaultable zero with unit face value.
   \item[-] {\sl Spot Interest Rates}: Available at time $t$ for maturity
      at time $T$,
      \[ r(t,T) = -\frac{\ln P(t,T)}{T-t}, \qquad  
	 R(t,T) = \frac{1-P(t,T)}{(T-t)P(t,T)}\]
   \item[-] 
      {\sl Forward Interest Rates}: These are forward rates available
      at time $t$ for a loan starting at time $T$ with maturity 
      at time $T+\tau$,
      \begin{align*}
	 f(t,T,T+\tau) = \frac{\ln P(t,T) - \ln P(t,T+\tau)}{\tau},
	 \quad F(t,T,T+\tau) = \frac{P(t,T) - P(t,T+\tau)}{\tau
	 P(t,T+\tau)} 
      \end{align*}
\end{itemize}
Now that we have the basics, we'll refer explictly to some schedule or
set of maturities, $\{T_0, T_1, \ldots, T_{n-1}, T_n\}$, just so that
we can simplify notation. We'll assume that the dates are evenly 
spaced so that $T_{i} - T_{i-1}= \tau$.
\begin{align*}
   P_i(t) &= P(t,T_i) \\
   R_i(t) = R(t,T_i) &= \frac{1 - P_i(t)}{(T_i-t)P_i(t)} \\
   F_i(t) = F(t,T_{i-1},T_i)&=\frac{P_{i-1}(t)-P_i(t)}{\tau P_i(t)}\\
\end{align*}
We also have a couple of identities that result from our specification:
\begin{align*}
   F_i(T_{i-1}) &= R_i(T_{i-1})\\
   \frac{P_i(t)}{P_{i-1}(t)} &= \frac{1}{1+F_i(t)\tau}
\end{align*}
Finally, the last thing to mention is that we write the forward
martingale measure (which we will primarily use in our analysis)
as $Q_i = Q_{T_i}$ with $P_i$ as the associated numeraire.

\newpage

\subsection{Forward Rate Agreements}

{\sl Payoff}: A $T_{i-1} \times T_i$ forward rate agreement (FRA)
settled in arrears is a forward contract with payoff at $T_i$ equal
to the difference between interest on a notional $N$ at the floating
rate $R_i(T_{i-1})$ and interest on the same notional at a pre-specified
fixed rate $K$:
   \[ N\cdot (R_i(T_{i-1}) - K) \tau \]
But since we know the payoff to an FRA already at $T_{i-1}$, 
we typically
settle FRA's in advance at the fixing time by taking the present value
of the payoff:
\begin{equation}
   \label{frainit}
    N \cdot P_i(T_{i-1}) \cdot (R_i(T_{i-1}) - K)\tau = 
      N\cdot \frac{(R_i(T_{i-1}) - K)\tau}{1+R_i(T_{i-1})\tau}
\end{equation}
{\sl Terminology}: Everything is from the perspective of the fixed:
\begin{itemize}
   \item[-] The person who pays fixed is \emph{long}, or has a 
      \emph{payer} FRA.
   \item[-] The person who receives fixed is \emph{short}, or has a 
      \emph{receiver} FRA.
\end{itemize}
{\sl Pricing}:
It can be shown that at any time $t \leq T_{i-1}$, an agreement 
for a loan from $T_{i-1}$ to $T_{i}$ is priced as follows:
\begin{align}
   \label{fra1}
    FRA(t,T_i,K) &= P_{i-1}(t) - (1+K\tau)P_i(t) \notag\\
    &= P_i(t)(F_i(t) - K) \tau
\end{align}
where $K$ is the strike. Typically we want to set $K$ so that the initial
value is 0, which implies that at time 0
   \[ K = F_i(0) \]
{\sl Applying FTAP}: By the definition of FTAP, we know that the value
of the FRA should be the discounted future payoff we defined
above in Equation \ref{frainit}, 
so under the T-forward martingale measure:
\begin{align}
   \label{fra2}
   FRA(t,T_i,K) &= P_i(t) E^{Q_i}_t\left[ \frac{P_i(T_{i-1})(R_i(T_{i-1})
      -K)\tau}{P_i(T_{i-1})} \right] \notag\\
      &= P_i(t) \left( E^{Q_i}_t\left[ R_i(T_{i-1})\right] - K\right)
	 \tau
\end{align}
Now if we compare the value today of the FRA, 
expressed two different ways in Equations
\ref{fra1} and \ref{fra2}, we see that we must have
   \[ F_i(t) = E^{Q_i}_t\left[ R_i(T_{i-1})\right] = 
   E^{Q_i}_t\left[ F_i(T_{i-1})\right] \]
implying that the forward rate, $F_i$ is a martingale under 
$Q_i$.\footnote{The third equality follows from an identitity we saw
   in the previous section:
   \[ F_i(T_{i-1}) = R_i(T_{i-1}). \]}

\newpage

\subsection{Interest Rate Swaps}

{\sl Spot Interest-Rate Swap} (IRS): A contract that exchanges
the fixed interest payment $N \cdot K \cdot \tau$---where $K$ is the
fixed interest rate called the \emph{swap rate}---for the variable
interest rate payment $N\cdot R_i(T_{i-1}) \cdot \tau$ at each
settlement date, $\{T_1, \ldots, T_n\}$.\footnote{Just a bit of 
terminology, the length of time $T_n-T_0$ is called the \emph{tenor} of 
the swap.}  Clearly an IRS is simply equivalent to a portfolio of 
FRA's settled in arrears, with the swap rate $K$ usually chosen 
so that the value of the IRS at $T_0$ is zero.\footnote{It's also
a convention that the party who payes fixed is long, or has a 
\emph{payer} IRS.}
\\
\\
{\sl Forward IRS Starting at $T_j$}: A contract that exchanges the
fixed interest payment $N\cdot K \cdot \tau$ for the 
variable interest payment $N\cdot R_i(T_{i-1}) \cdot \tau$ on
each settlement date, $\{T_{j+1}, \ldots, T_n\}$.\footnote{In other
words, the first fixing date, $T_j$, does not coincide with
the contract date, $T_0$, as we have in the standard spot IRS.} 
\\
\\
{\sl Pricing Forward IRS}: First we start by pricing the floating leg
using FTAP then the fixed leg for unit face value:
\begin{align*}
   \text{Floating} \qquad \hfill
   \sum^n_{i=j+1} P_i(t) E^{Q_i}_t \left[ R_i(T_{i-1})\tau \right]
      &=\sum^n_{i=j+1} P_i(t) F_i(t)\tau 
      = \sum^n_{i=j+1} (P_{i-1}(t) - P_i(t)) \\
      &= P_j(t) - P_n(t) \\
   \text{Fixed} \qquad\quad
   \sum^n_{i=j+1} P_i(t)K\tau &=  K \tau \sum^n_{i=j+1} P_i(t)  
      = K \tau A(t,T_j,T_n) \\
   \text{Total value at $t\leq T_j$} \qquad 
   IRS(t,T_j, T_n, K) &= P_j(t) - P_n(t) - K\tau A(t,T_j,T_n)
\end{align*}
Again, we typically want the value of the swap to have zero initial
value, so the previous line implies that what we define as the  
\emph{forward swap rate} at time $t$ is\footnote{If we
get to time $t=T_j$, however, then this simply becomes the swap rate.}
   \[ FSR(t, T_j, T_n) = \frac{P_j(t) - P_n(t)}{\tau A(t,T_j,T_n)} =K \]
Notice that if the swap has a single payment date, the swap collapses
to an FRA and the forward swap rate collapses to the forward rate:
   \[ IRS(t,T_{i-1},T_i,K) = FRA(t,T_i,K), \qquad 
       FSR(t,T_{i-1},T_i) = F_i(t) \]
In fact, a little fiddling shows that the forward swap rate is a 
weighted average of forward rates,
\[ FSR(t,T_j, T_n) = \frac{P_j(t) - P_n(t)}{\tau A(t,T_j,T_n)} = 
   \sum^n_{i=j+1} w_i F_i(t),\qquad\quad w_i = \frac{P_i}{A(t,T_j,T_n)}
   \]
Swaps are quoted in terms of the swap rate, FSR(0,0,$T_n$) with
varying tenors. They are typically against $\tau = 6$ mo. LIBOR 
for Euro swaps but $\tau=3$ mo. LIBOR for USD.



\subsection{Caps and Floors}

{\sl Interest Rate Cap}: Like a payer IRS, but cash flows are only
settled if positive. Thus the net cash flow on a generic settlement
date, $T_i$, is
   \[ N\cdot (R_i(T_{i-1})-K)^+ \tau \]
The buyer of the cap pays the seller an up-front premium, and all 
subsequent cash flows are paid by the seller to the 
buyer.\footnote{The reason
it's called a cap is because it compensates you if floating interest 
rates rise above a certain level, allowing people with floating 
liabilities to hedge. Obviously, caps only make sense for 
those with floating liabilities, where there is some uncertainty
in future cash flows. Contrast this with those who have fixed 
liabilities, who know \emph{exactly} what they will have to pay.}
Note that caps are typically \emph{forward-starting}, so that the
first fixing date is not $T_0=0$ (the contract date), but 
$T_1=\tau$, where $\tau$ is the lag between settlement dates.
\\
\\
{\sl Interest Rate Floor}: Like a receiver IRS, but cash flows are only
settled if positive. Thus the net cash flow on a generic settlement
date, $T_i$, is
   \[ N\cdot (K - R_i(T_{i-1})^+ \tau \]
{\sl Caplets and Floorlets}: Similar to breaking up an IRS into a 
portfolio of FRAs, caps and floors can be thought of as a portfolio
of options on the individual FRAs that compose the swap. We call
these individual options ``caplets'' and ``floorlets.'' In this
way, caplets are call options on the interest rate, $R_i$, while
floorlets are puts on the interest rate.
\\
\\
So  we let Cap$(t,T_n,K)$ denote the value at $t< T_1$ of a cap
with first fixing date $T_1$ and last settlement date $T_n$,
and we also let Cpt$(t,T_i,K)$ denote the value of a caplet with
fixing data $T_{i-1}$ and settlement date $T_i$. Similar notation
holds for floors and floorlets. This allows us to write
   \[ \text{Cap}(t,T_n,K) = \sum^n_{i=2} \text{Cpt}(t,T_i,K), \qquad
      \text{Flr}(t,T_n,K) = \sum^n_{i=2} \text{Flt}(t,T_i,K)\]

\subsection{Cap-Floor Parity}

Next, it's clear that the combination of long a cap and short a floor
will generate the same cash flows as a forward IRS starting at $T_1$ 
so that we arrive at the formula for \emph{cap-floor parity}:
   \[ \text{Cap}(t,T_n,K) - \text{Flr}(t,T_n,K) = IRS(t,T_1,T_n, K) \]
   \[ \text{Cpt}(t,T_i,K) - \text{Flt}(t,T_i,K) = FRA(t,T_i,K) \]
In the special case where $K=FSR(t,T_1,T_n$ in the first expression,
then the value of the IRS is zero and the cap has the same value as
the corresonding floor. Thus the forward swap rate (FSR) 
is the at-the-money strike for caps and floors.

\newpage

\subsection{Black's Formula for Caps and Floors}

\subsubsection{Pricing}
A simple variation of the Black formula allows us to value a caplet
under certain assumptions. FTAP and an identity give us
\begin{align*}
   \text{Cpt}(t,T_i,K) &= P_i(t) E^{Q_i}_t\left[(R_i(T_{i-1})-K)^+ \tau
      \right]\\
      &= P_i(t) E^{Q_i}_t\left[(F_i(T_{i-1})-K)^+\right]\tau
\end{align*}
This expression makes clear that a caplet is simply a call option on 
the forward rate, $F_i$. From there, we recall that $F_i$ is a 
martingale under $Q_i$ so that 
   \[ dF_i(t) = F_i(t) \sigma_i(t) d\hat{W}_i(t) \]
where $\sigma_i$ is the proportional volatility of the forward rate
and $\hat{W}_i$ is a Brownian motion under $Q_i$. By Ito's Lemma,
we get that 
\[ \ln F_i(T_{i-1}) = \ln F_i(t) - \frac{1}{2} \int^{T_{i-1}}_t 
   \sigma_i(s)^2 \; ds + \int^{T_{i-1}}_t  \sigma_i(s) \; d\hat{W}_i(s)
   \]
If we assume that $\sigma(s)$ is deterministic, then 
\[ F_i(T_{i-1}) \sim N\left(-\frac{1}{2}\bar{\sigma}^2_i(T_{i-1} - t),
   \;\;\bar{\sigma}^2_i(T_{i-1}-t) \right), \qquad \bar{\sigma}^2_i = 
   \frac{1}{T_{i-1} - t} \int^{T_{i-1}}_t \sigma_i(s)^2 \; ds \]
\begin{align}
   \label{capletblack}
   \text{Cpt}_B(t,T_i,K) &= P_i(t) \left[ F_i(t) N(y_i) - K N\left(y_i - 
   \bar{\sigma}_i \sqrt{T_{i-1} - t}\right) \right] \tau \notag\\
   \text{where }\quad y_i &= \frac{1}{\bar{\sigma}_i \sqrt{T_{i-1} - t}}
   \left[ \ln \left(\frac{F_i(t)}{K}\right)\right] + \frac{1}{2}
   \bar{\sigma}_i\sqrt{T_{i-1} - t} 
\end{align}
Under the assumptions of the Black formula, the caplet implied 
volatility is equal to the average proportional volatility of the
underlying \emph{forward} rate. The formula for a floorlet is
similar
\begin{align}
   \label{floorletblack}
   \text{Flt}_B(t,T_i,K) &= P_i(t) \left[ K N\left(-y_i + 
   \bar{\sigma}_i \sqrt{T_{i-1}-t}\right)-F_i(t) N(-y_i)\right] \tau 
\end{align}
where $y_i$ is as above. Putting everything together, we get that 
\begin{equation}
   \label{capfloorblack}
   \text{Cap}_B(t,T_n,K) = \sum^n_{i=2} \text{Cpt}_B(t,T_i,), \qquad
   \text{Flr}_B(t,T_n,K) = \sum^n_{i=2} \text{Flt}_B(t,T_i,)
\end{equation}

\subsubsection{Implied Volatilies}
As with equities, market quotes for caps and floors are typically
provided in terms of Black implied volatilities. So if $\text{Cap}_M$ is
the market price, the cap implied volatility, $\sigma_n^{\text{Cap}}$ 
for a cap with maturity $T_n$ will be defined so that
   \[ \text{Cap}_M(t,T_n,K) = \sum^n_{i=2} \text{Cpt}(t,T_i, K; 
      \sigma^{\text{Cap}}_n) \]
In this way, cap implied volatilities are averages of the implied 
volatilities of the different caplets that compose the cap.
\\
\\
If we were to actually look at a plot of the Black Implied Volatilities
for European at the money caps, we'd see a hump shape with noticeable
right skew. This comes about for a few reasons:
\begin{enumerate}
   \item Cap IVs are averages of the IVs of the component caplets.
   \item Caplet IVs are related to the percentage volatilities of 
      forward rates.
   \item Empirically, these percentage volatilities of forward rates
      also display a hump figure similar to what we'd see in the IVs.
\end{enumerate}
Why do we see this empirical hump in the volatilities of forward rates?
Well, it makes a lot of sense.  
\begin{itemize}
   \item[-] Forward rates with a short time to 
      maturity are closely tied to spot rates over the short term. As a
      result, they don't vary much as the Federal Reserve sets the 
      interest rate.
   \item[-] Forward rates over the long term build in effects over 
      several business cycles, so the result is a sort of averaging.
      This means lower observed volatility.
   \item[-] Finally, in the medium term, business cycles can have an
      impact and the time horizon is long enough for Federal Reserve
      policy to change. This generates the hump shape.
\end{itemize}

\newpage
\subsection{Swaptions}

So far, we've seen forward rate agreements, swaps on interest rates,
along with caps and floors, which are a type of option on interest rates.
Now, let's look at \emph{swaptions}, which are options on 
interest rate \emph{swaps}. There are two basic types:
\begin{itemize}
   \item[-] {\sl Payer Swaption}: Gives the right to enter into a payer
      IRS upon exercise.
   \item[-] {\sl Receiver Swaption}: Gives the right to enter into a 
      receiver IRS upon exercise.
\end{itemize}
In either case, the fixed rate in the IRS is equal to the strike of the
swaption, and \emph{no} price is paid upon exercise. Typically, the
swap starts immediately upon exercise so that the 
first reset date of the underlying IRS coincides with the swaptions
exercise date.
\\
\\
Now, if we let PSO$(t,T_j, T_n, K)$ denote the value at time $t$ 
of a a payer swaption with expiration $T_j$ on a swap with unit notional
and tenor $T_n - T_j$ (so that it's a $T_j -t$ years into $T_n-T_j$
years swaption), we can write the value of a payor swaption on the
expiration date (using identities and results derived above) as follows:
\begin{align*}
   \text{PSO}(T_j,T_j,T_n,K) &= \text{IRS}(T_j, T_j, T_n,K)^+ \\
   &= (P_j(T_j) - P_n(T_j) - K \tau A(T_j, T_j, T_n))^+ \\
   &= A(T_j, T_j, T_n) \left( \text{FSR}(T_j,T_j,T_n) - K\right)^+ \tau
\end{align*}
If we consider the payoff from a swaption on a swap with a single
settlement date $T_n$, we get the same payoff as a caplet settled
on the fixing date:
\begin{align*}
   \text{PSO}(T_{n-1}, T_{n-1},T_n,K) &= P_n(T_{n-1})(R_n(T_{n-1}) - 
      K)^+ \tau \\
      \Rightarrow\quad\text{PSO}(t,T_{n-1},T_n,K) &= \text{Cpt}(t,T_n,K) 
\end{align*}
Similarly for receiver swaptons:
\begin{align*}
   \text{RSO}(T_j,T_j,T_n,K) &= (-\text{IRS}(T_j, T_j, T_n,K))^+ \\
   &= (K \tau A(T_j, T_j, T_n) + P_n(T_j) - P_j(T_j))^+ \\
   &= A(T_j, T_j, T_n) \left( K -  \text{FSR}(T_j,T_j,T_n) 
      \right)^+ \tau\\ \\
   \text{RSO}(t, T_{n-1},T_n,K) &= \text{Flt}(t,T_n,K) 
\end{align*}

\subsection{Swaption Parity}

Clearly, long a payer swaption and short the corresponding receiver 
swaption will be equivalent to a long position in a forward IRS:
   \[ \text{PSO}(t,T_j,T_n,K) - \text{RSO}(t,T_j,T_n,K) = 
      \text{IRS}(t,T_j,T_n,K) \]
This relationship is called \emph{swaption parity}. Note that if 
$K=\text{FSR}(t,T_j,T_n)$, the value of the payer swaption is the same as
the value of the receiver swaptions. This means that the forward swap
rate is the at-the-money strike for swaptions---just as for caps and
floors.

\newpage
\subsection{Black's Formula for Swaptions}

\subsubsection{Pricing}

Just like for caps and floors, we can price swaptions using a version
of the Black Formula. To do so, we'll use annuities as our
numeraire. So recall that $A(t,T_j,T_n)$ is the value at time $t$ of an 
annuity paying \$1 at times $\{T_{j+1}, T_{j+2}, \ldots, T_{n}\}$. 
Because there are no intermediate payments before $T_{j+1}$, we can
take this as our numeraire, which we'll denote $Q_A$ and call the
\emph{forward swap measure}. This allows us to write
\begin{align*}
   \text{PSO}(t,T_j,T_n,K) &= A(t,T_j,T_n) E_t^{Q_A} \left[
      \frac{A(T_j,T_j,T_n) (\text{FSR}(T_j,T_j,T_n)-K)^+ \tau}{
      A(T_j,T_j,T_n)} \right] \\
   &= A(t,T_j,T_n) E_t^{Q_A} \left[(\text{FSR}(T_j,T_j,T_n)-K)^+\right]
      \tau 
\end{align*}
Now its clear that 
   \[ \text{FSR}(t,T_j,T_n) = \frac{P_j(t) - P_n(t)}{\tau A(t,T_j,T_n)}
      \]
must be a martingale under $Q_A$ because it represents the discounted
price process of assets available in the market. So assuming the 
FSR is an Ito process, we have
   \[ d\text{FSR}(t,T_j,T_n) = \text{FSR}(t,T_j,T_n) 
      \sigma_{\text{FSR}}(t) \; d\hat{W}(t) \]
where $\sigma_{\text{FSR}}$ is the percentage volatility of the forward
swap rate and $\hat{W}(t)$ is a Brownian motion under $Q_A$. If we
assume that $\sigma_{\text{FSR}}$ is deterministice, it follows
that FSR$(T_j,T_j,T_n)$ is lognormally distributed under $Q_A$ 
and
the value of payer and receiver options are
\begin{align*}
   \text{PSO}_B(t,T_j,T_n,K) &= A(t,T_j,T_n) \left( 
      \text{FSR}(t,T_j,T_n) N(y) - K N(y - \bar{\sigma}_{\text{FSR}}
      \sqrt{T_j - t}) \right) \tau \\
   \text{RSO}_B(t,T_j,T_n,K) &= A(t,T_j,T_n) \left( 
      K N(-y + \bar{\sigma}_{\text{FSR}}\sqrt{T_j - t}) 
      - \text{FSR}(t,T_j,T_n) N(-y)\right) \tau  \\
   y &= \frac{1}{\bar{\sigma}_{\text{FSR}} \sqrt{T_j - t}} \left[
      \ln \left( \frac{\text{FSR}(t,T_j,T_n)}{K} \right) \right] + 
      \frac{1}{2} \bar{\sigma}_{\text{FSR}} \sqrt{T_j - t} \\
   \bar{\sigma}^2_{\text{FSR}} &= \frac{1}{T_j - t} \int^{T_j}_t 
      \sigma_{\text{FSR}}(s)^2 \; ds 
\end{align*}

\subsubsection{Implied Volatilities}

As with caps and floors, swaption quotes are typicall provided in 
terms of Black Implied Volatilities. Quotes are typically arranged
in a volatility matrix in which rows correspond to different
expirations and columns correspond to different swap tenors. 

\newpage
\subsection{Derivatives on Bonds, Rather than Interest Rates}

In previous sections, we derived everything in terms of derivatives
on interest rates (including forward or forward swap rates). As we
go on, however, it will be more useful to think of caps, floors, and
swaptions as options on \emph{bonds} rather than on \emph{interest
rates}. This section examines  
how we can transform payoffs of these derivatives
into payoffs of standard vanilla options on bonds.
\\
\\
{\sl Caplets and Floorlets}: For these particular options, there is
a relation to the prices of Zero Coupon Bonds (ZCBs):
\begin{align*}
   \text{Cpt}(T_{i-1},T_i,K) &= P_i(T_{i-1})(R_i(T_{i-1}) - K)^+ \tau\\
   &= P_i(T_{i-1})\left( \frac{1}{P_i(T_{i-1})} -1 - K\tau\right)^+ \\
   &= (1+K\tau) \left( \frac{1}{1+K\tau} - P_i(T_{i-1}) \right)^+
\end{align*}
This is the payoff of $1+K\tau$ puts with strike $1/(1+K\tau)$ and
expiration $T_{i-1}$ on the Zero Coupon Bond (ZCB) with maturity $T_i$.
\\
\\
Similarly, the value of a floorlet with fixing date $T_{i-1}$,
settlement date $T_i$, and strike $K$ is equal to the value of $1+K\tau$
calls with strike $1/(1+K\tau)$ and expiration $T_{i-1}$ on the
ZCB with maturity $T_i$.
\\
\\
{\sl Swaptions}: Rather than ZCBs, swaptions can be expressed as options
on \emph{coupon} bonds. Recall the payoff of a payer swaption on the
expiration date $T_j$:
\begin{align*}
 \text{PSO}(T_j,T_j,T_n,K) &= \left(P_j(T_j)- 
   P_n(T_j) - K\tau A(T_j, T_j, T_n)\right)^+ \\
   &= \left(1- [P_n(T_j) + K\tau A(T_j, T_j, T_n)]\right)^+ 
\end{align*}
which is the payoff of a put with strike 1 and expiration $T_j$ on
a coupon bond with coupon rate $K$ and maturity $T_n$. This means
the value of a payor swaption with expiration $T_j$ and strike $K$
on a swap with tenor $T_n-T_j$ is equal to the value of a put with 
expiration $T_j$ and unit strike on a coupon bond with coupon rate
$K$ and maturity $T_n$.

\newpage
\section{One-Factor Spot Rate Models}

\subsection{Uses of Interest Rate Models}

Interest rate models serve one or more of three main purposes:
\begin{enumerate}
   \item Forecasting future interest rates.
   \item Pricing bonds.
   \item Pricing interest rates derivatives.
\end{enumerate}
Overall, there's a fair amount of overlap, but also some extremely
important differences. In particular, the first task operates under
the \emph{real-world} probability measure, while the second two
operate under various \emph{martingale} measures.
\\
\\
In addition, while the second two both require operations under 
martingale measure, they differ in their goals. In the case of pricing
bonds, it's not crucial that the model generates prices for zero-coupon
bonds (ZCBs) which perfectly match the current term structure. In fact,
you typically want to capture mispricing, so you'll want some discrepancy
between your model prices and the observeable market prices.
\\
\\
On the other hand, when pricing interest rate derivatives, it doesn't
matter whether bonds are valued \emph{correctly}.\footnote{Considering an
analagous situation, recall that When we valued equity
derivatives, we didn't check whether the stock price equaled the
discounted expected future cash flows---which is the ``correct''
price.} Rather, we just care about the price at which we can buy and sell
the asset so that we can form a replicating portfolio. With this in 
mind, we would therefore like our models to match the initial term
structure---which determines the prices at which we can buy and sell 
bonds---\emph{exactly}.

\newpage 
\subsection{General Form of One-Factor Models}

\emph{One-factor spot-rate models} assume that the instantaneous
spot rate $r(t) = \lim_{T\rightarrow t} r(t,T)$ follow a diffusion:
\begin{equation}
   dr(t) = \mu(r(t),t) \; dt + \sigma(r(t),t) \; d\hat{w}(t)
\end{equation}
where $\hat{w}(t)$ is Brownian motion under the risk neutral measure,
$Q_B$. By FTAP, we know that the price of a riskless ZCB should be
\begin{equation}
   \label{zcb}
   P(t,T) = B(t) E_t^{Q_B}\left[ \frac{1}{B(T)}\right] = 
   E_t^{Q_B} \left[ \exp\left( - \int^T_t r(s) \; ds \right) \right]
\end{equation}
In this way, we can model the \emph{entire} term structure just based
on the short term spot rate.
\\
\\
In general, we will want $P(t,T)$ as expressed in Equation \ref{zcb}
to have an explicit, closed form solution, which becomes clear if we
consider how we would price a call on a ZCB with expiration $t<T$
and maturity $T$. If today is $t_0$, suppose that we have an option
on ZCB that expires at time $t$. Then we want to simulate the price
$P(t,T)$ given what we know today, $P(t_0,T)$. With a closed form
solution, it's easy to take draws for what's inside expectation
operator in Equation \ref{zcb} and average. Otherwise, we'd have
to simulate interest rate paths form $t_0$ to $t$, then from
$t$ to $T$ in order to compute the price $P(t,T)$. This would lead
to exponentialy growth in the number of simulations.

\subsection{General Form of Affine One-Factor Models}

Most one-factor models assume an affine process to satisfy the 
requirement that ZCB prices be available in closed form,\footnote{Recall
the definition of an affine process, which stipulates a process whose
drift and variance rate (which is volatility squared) are 
\emph{affine} (linear) functions of the level of the process. While no
general solution exists to the SDE that defines an affine process, 
the Fourier transform of the probability density of the value of the
process at any time $t$ is known.} Models of the following form
are called \emph{affine one-factor models}:
\begin{equation}
   \label{affine}
   dr(t) = \left( \theta(t) - \kappa(t) r(t)\right) \; dt + 
      \sqrt{ \sigma_0(t)^2 + \sigma_1(t)^2 r(t) } \; d\hat{w}(t) 
\end{equation}
It can also be shown that in any affine one-factor model, ZCB bond
prices are given by
\begin{equation}
   \label{affinep}
   P(t,T) = e^{a(t,T)-b(t,T)r(t)} 
\end{equation}
where the functions $a$ and $b$ solve the two first-order ODEs
\begin{align*}
    b_t(t,T) &= b(t,T) \kappa(t) + \frac{1}{2} b(t,T)^2 \sigma_1(t)^2-1 
      \\
   a_t(t,T) &= b(t,T) \theta(t) - \frac{1}{2} b(t,T)^2 \sigma_0(t)^2
\end{align*}
with boundary conditions $b(T,T)=0$ and $a(T,T)=0$. 

\subsection{Vasicek Model}

The most basic model, the {\sl Vasicek Model}, is a homogenous affine
one-factor spot rate model of the form
\begin{equation}
   dr(t) = \kappa(\theta-r(t)) \; dt + \sigma \; d\hat{w}(t),
   \qquad \kappa \geq0, \quad \theta, \sigma > 0
\end{equation}
It has a few desireable properties:
\begin{itemize}
   \item[-] The spot rate, $r(t)$, is mean reverting if $\kappa >0$.
   \item[-] The functions $a$ and $b$ in Equation \ref{affinep}
      can be computed explicitly.
   \item[-] Explicit formulas can be derived for vanilla bond
      options and more complext derivatives.
\end{itemize}
It does, however have important disadvantages in that the volatility
is not proportional to the level of interest rates, and negative
interest rates are not ruled out. 
\\
\\
{\sl Solution to the Vasicek Model}: Now let's solve the SDE to get
a closed form solution for the interest rate. To do so, we set $z(t) =
r(t) e^{\kappa t}$, get the partials, and apply Ito's Lemma:
\[ \frac{\partial z}{\partial r} = e^{\kappa t}, \qquad  
   \frac{\partial z^2}{\partial r^2} = 0, \qquad
   \frac{\partial z}{\partial t} = \kappa r(t) e^{\kappa t} \]
\begin{align*}
   dz(t) &= \frac{\partial z}{\partial t} dt + 
   \frac{\partial z}{\partial r} dr + \frac{1}{2} 
   \frac{\partial^2 z}{\partial r^2} \sigma^2 dt\\
   &= \kappa r(t) e^{\kappa t} \; dt + e^{\kappa t} \left[
      \kappa \left( \theta - r(t)  \right) \; dt + \sigma\; d\hat{w}(t)
      \right] + 0 \\
   &= \kappa \theta e^{\kappa t} \; dt - e^{\kappa t} \sigma \; 
      d\hat{w}(t) \\
\Rightarrow d\left( r(t) e^{\kappa t}\right) &= 
   \kappa \theta e^{\kappa t} \; dt - e^{\kappa t} \sigma \; 
      d\hat{w}(t) 
\end{align*}
Since the right hand side does not contain $r(t)$, we can take 
Ito-integrate everything:
\begin{align*}
\int^t_0 d\left( r(s) e^{\kappa s}\right) &= \int^t_0
   \kappa \theta e^{\kappa s} \; ds + \int^t_0 e^{\kappa s} \sigma \; 
      d\hat{w}(s) \\
e^{\kappa t} r(t) - e^{\kappa \cdot 0} r(0) &= \theta( e^{\kappa t}-1) 
   +\sigma \int^t_0 e^{\kappa s} d\hat{w}(s)  \\
\Rightarrow \quad  r(t) &= r(0) e^{-\kappa t} + \theta(1-
   e^{-\kappa t}) +\sigma \int^t_0 e^{-\kappa (t-s)} d\hat{w}(s)  
\end{align*}
This implies that given $r(0)$, $r(t)$ is normally distributed 
with\footnote{The variance term uses
a standard Ito Calculus result 
$Var \left( \int^t_0 f(s) \; dW(s) \right) = \int^t_0 f^2(s) \; ds$.} 
\begin{align*}
   Er(t) &= r(0) e^{-\kappa t} + \theta(1- e^{-\kappa t}) \\
   Var[ r(t)]&= Var \left(\sigma \int^t_0 e^{-\kappa (t-s)}d\hat{w}(s)
      \right) = \sigma^2 \int^t_0 e^{-2 \kappa (t-s)} \; ds\\
      &= \frac{\sigma^2}{2\alpha} \left( 1 - e^{-2\kappa t} \right) 
\end{align*}

\newpage 
\subsection{Cox-Ingersoll-Ross (CIR) Model}

An alternative to the Vasicek model, the CIR model is another
homogeneous affine one-factor model:
\begin{equation}
   \label{cir}
   dr(t) = \kappa(\theta - r(t)) \; dt + \sigma \sqrt{r(t)} \; 
      d\hat{w}(t), \qquad \kappa, \theta, \sigma > 0
\end{equation}
It retains the mean reversion feature along with the explicit formula
result while eliminating the possibility of negative interest rates.
\\
\\
The chief disadvantage, however, of both the Vasicek and CIR models
is their inability to match the current term structure---a sympton
of the relatively small amount of free parameters.

\newpage
\subsection{Ho-Lee Model}

The Ho-Lee model was an attempt to develop a model that fits the
initial term structure better. It has characterization
\begin{equation}
   \label{holee}
   dr(t) = \theta(t) \; dt + \sigma \; d\hat{w}(t), \qquad \sigma >0
\end{equation}
In this process, $\theta$ is \emph{time dependent}. This has
a number of intractible features, namely that interest rates are 
not mean-reverting and that there is a high probability of negative 
interest rates.
\\
\\
{\sl Solution to the Ho-Lee Model}: 





\newpage
\subsection{Extended Vasicek Model}

\subsubsection{Governing Process}
The \emph{Extended Vasicek} (or \emph{Hull-White}) model is a 
non-homogeneous affine one-factor model that assumes
\begin{equation}
   dr(t) = \kappa(\theta(t) - r(t))\; dt + \sigma d\hat{w}(t)
\end{equation}
Clearly, this is identical to the traditional Vasicek model with
the crucial exception that $\theta(t)$ is time dependent.\footnote{As
with the Vasicek model, the interest rate is mean-reverting and 
also has the unfortunate possibility of becoming negative. However,
as we see in the solution, the probability of negative interest
rates is small provided that $\theta$ and $\kappa$ are large
while $\sigma$ is small.}
Using Ito's Lemma, with $z(t) =r(t) e^{\kappa t}$ it has solution
\begin{align}
   \label{evsol}
   r(t) &= e^{-\kappa t} \left( r(0) + \kappa \int^t_0 e^{\kappa s}
      \theta(s) \; ds + \sigma \int^t_0 e^{\kappa s} \; d\hat{w}(s) 
      \right) \notag\\
   \Rightarrow \quad r(t) &\sim N\left(e^{-\kappa t} \left[ r(0)
      + \kappa \int^t_0 e^{\kappa s} \theta(s) \; ds\right],
      \; \; \frac{1-e^{-2\kappa t}}{2\kappa} \sigma^2 \right)
\end{align}
That's one characterization, but we could express the EV model another
way by examining the soltuion and letting 
\begin{align*}
   \varphi(t) &= \kappa \int^t_0 e^{-\kappa (t-s)} \theta(s) \; ds\\
   x(t) &= e^{-\kappa t} \left(r(0) + \sigma \int^t_0 e^{\kappa s}
   \; d\hat{w}(s) \right)\\
   \Rightarrow r(t) &= \varphi(t) + x(t)
\end{align*}
It's clear that $\varphi(t)$ is entirely deterministic; only 
$x(t)$ is random.  Moreover, by applying Ito's lemma,\footnote{To do
so, we rewrite the the expression for $x(t)$ as 
\begin{align*}
   e^{\kappa t} x(t) - e^{\kappa 0} x(0) &= \sigma \int^t_0 e^{\kappa s}
      \; d\hat{w}(s), \qquad x(0) = r(0) \\
   \int^t_0 d\left(e^{\kappa s} x(s)\right) &= \sigma 
      \int^t_0 e^{\kappa s} \; d\hat{w}(s), \quad \Leftrightarrow
      \quad d\left(e^{\kappa t} x(t)\right) = \sigma 
      e^{\kappa t} \; d\hat{w}(t)
\end{align*}  
From there, we define $y(t) = e^{\kappa t} x(t)$, implying $x(t) =
e^{-\kappa t} y(t)$. Since we have $dy(t)$, we can use Ito's Lemma to get $dx(t)$.}
we can express
\begin{align*}
    dx(t) &= -\kappa x(t) \; dt + \sigma \; d\hat{w}(t), 
    \qquad x(0) = r(0)
\end{align*}
which is an affine process. This formulation is often more convenient
to work with, and we'll use it to price ZCBs.

\newpage
\subsubsection{Pricing Bonds under the EV Model}

Now that we have a reasonable model of interest rate dynamics, let's
price zero-coupon bonds. Recall how we do that:
\begin{align*}
   P(t,T) &= E\left[e^{ \int^T_t r(s) \; ds } \right] = 
      E\left[e^{ \int^T_t \varphi(s) + x(s) \; ds } \right]  \\
   &= e^{ \int^T_t \varphi(s) \; ds } 
      E\left[e^{ \int^T_t x(s) \; ds } \right]  
\end{align*}
where we could pull the exponential of $\varphi(\cdot)$ in front
because it is entirely deterministic. To compute the expectation,
let's do a couple intermediate steps:
\begin{enumerate}
   \item We can compute the expectation of $x(t)$ simply and in closed
      form by using Ito's lemma similarly to how we used it for
      the complete $r(t)$. If we do so, we get:
      \begin{align*}
	 x(t) &= e^{-\kappa t} \left( x(0) +  \sigma \int^t_0
	    e^{\kappa s} \; d\hat{w}(s)\right) \\
	 \Rightarrow \quad E[x(t)] &= x(0) e^{-\kappa t}
      \end{align*}
   \item From there, we can use that to compute the expectation of the
      integral, which will be useful
      later on:
      \begin{align*} 
	 E\left[\int^t_0 x(s) \; ds\right] &= 
	    E\left[ \int^t_0 
	       e^{-\kappa s} \left( x(0) +  \sigma \int^s_0
	       e^{\kappa u} \; d\hat{w}(u)\right) \; ds\right]\\
	 &= E\left[ x(0) \int^t_0  e^{-\kappa s} \right] +
	    E\left[ \int^t_0 
	       e^{-\kappa s} \left(\sigma \int^s_0
	       e^{\kappa u} \; d\hat{w}(u)\right) \; ds\right]\\
	 &= x(0) E\left[ \int^t_0  e^{-\kappa s} \right] +
	    \sigma E\left[ \int^t_0 
	       \int^s_0
	       e^{\kappa (u-s)} \; d\hat{w}(u) \; ds\right] \\
	 &= \frac{x(0)}{\kappa} \left[ 1- e^{-\kappa t}\right]
      \end{align*}
      The second term drops because it's an expecation of Brownian 
      Increments which all have expectation 0.\footnote{You see this 
      if you switch the order of integration, integrate out $s$, then 
      break it up into two Ito integrals, which will have expecation 
      zero.}
   \item Next, we want the variance of $x(t)$.

\end{enumerate}

FINISH DERIVATION

At the very end, we see that ZCB prices are
\begin{align}
   \label{evmid}
   P(t,T) &= E_t^{Q_B} \left[ e^{-\int^T_t r(s) \; ds} \right]\notag\\
   &= E_t^{Q_B} \left[ e^{-\int^T_t \varphi(s) + x(s) \; ds} \right]
   = e^{-\int^T_t \varphi(s) \; ds}
      E_t^{Q_B} \left[ e^{-\int^T_t x(s) \; ds} \right]\notag\\
   &= e^{-\int_t^T \varphi(s) \; ds - a(T-t) - b(T-t) x(s)} \\
   \text{where }\quad a(\tau) &= \left( \frac{b(\tau) - \tau}{\kappa}
      + \frac{b(\tau)^2}{2}\right)\frac{\sigma^2}{2\kappa}, \qquad\quad
   b(\tau) = \frac{1-e^{-\kappa \tau}}{\kappa}\notag
\end{align}
Equation \ref{evmid} follows from the fact that $x(t)$ is an 
affine process and can be expressed as in Equation \ref{affinep}.


\newpage
\subsubsection{Bond Price Dynamics}

\subsubsection{Calibration}
To calibrate the model, we'll need
to specify $\varphi$ so that the model matches the term structure
on the calibration date, time $t=0$. This means we choose $\varphi$
so that
\[ P(0,T) = e^{-\int^T_0 \varphi(s) \; ds - a(T) - b(T) x(0)} = 
   P_M(0,T) \]
\[ \Leftrightarrow \quad e^{-\int^T_0 \varphi(s) \; ds} = 
   e^{a(T) + b(T) x(0)} P_M(0,T) \]
\emph{for all} $T$, so that our calibration matches the \emph{entire}
term structure. If $\varphi$ is, in fact, chosen so that this is true
for all $T$, then it follows that 
\begin{align*} 
   e^{-\int^T_t \varphi(s) \; ds} &= \frac{
      e^{-\int^T_0 \varphi(s) \; ds}}{e^{-\int^t_0 \varphi(s) \; ds}}\\
   &= \frac{e^{a(T) + b(T) x(0)} P_M(0,T)}{e^{a(t) + b(t) x(0)} 
      P_M(0,t)} 
\end{align*}
This means that we can can plug this into Equation \ref{evmid} and
use the fact that $x(0) = r(0)$ to get
\begin{align*}
   P(t,T) &= A(t,T) e^{-b(T-t) x(t)} \\
   A(t,T) &= \frac{e^{a(T) + b(T) r(0)} P_M(0,T)}{
      e^{a(t) + b(t) r(0)} P_M(0,t)} e^{-a(T-t)}
\end{align*}
We now have the price of ZCBs expressed as a function of the initital
term structure, a stochastic factor $x(t)$, and two constant 
parameters---$\kappa$ and $\sigma$. 
\\
\\
Since $x(t)$ is normally distributed, it follows that $P(t,T)$ are
log-normally distributed, while simply compounded spot rates, which
are expressed
   \[ R(t,T) = \frac{1}{T-t} \left(\frac{1}{P(t,T)}-1\right), \]
will have a shifted log-normal distribution, while continuously
compounded rates are normally distributed.


\subsubsection{Pricing Coupon Bonds and Payer Swaptions}

\newpage
\subsection{Cox-Ingersoll Ross Extension}

\subsubsection{CIR++ versus Extended CIR}

\subsubsection{Pricing ZCB's under the CIR++ Model}


\newpage
\section{Multi-Factor Models}







%%%%%% APPENDIX %%%%%%%%%%%%%%%%%%%%%%

\newpage
\appendix

\section{Principal Component Analysis}

\subsection{Motivation and Setup}
Principal component analysis is relevant for Fixed Income in 
order to determine how many factors a multi-factor spot rate
model might need. Recall that we suppose our zero-coupon bonds can
be priced by
\begin{align}
   \label{pca1}
   P(t,T) &= A(t,T) e^{-\sum^n_{i=1} b_i(T-t) x_i(t)} \notag\\
   \Rightarrow \quad r(t,T) &= -\frac{\ln A(t,T)}{T-t} + 
      \sum^n_{i=1} \frac{b_i(T-t)}{T-t} x_i(t)
\end{align}
\\
\\
We start by assuming  we have a time series of interest rates with
constant times to maturity: $\tau_1, \ldots, \tau_N$. So we might
observe the 1-year rate over time, the 2-year rate over time, etc.
\\
\\
Next, we let $\mathbf{r}(t)$ denote the $N$-dimensional vector of 
observed rates at time $t$. It has $i$th entry $\mathbf{r}_i(t) = 
r(t,\; t + \tau_i)$. Then if the observed rates are determined by
an $n$ factor model, Equation \ref{pca1} tells us that
\begin{align*}
   \mathbf{r}(t) &= \mathbf{a}(t) + \mathbf{B}\mathbf{x}(t)\\
   \text{where } \quad \mathbf{a}_i(t) &= -\ln \left( A(t,t+\tau_i)/
   \tau_i\right), \qquad \mathbf{a}_i(t) \in \mathbf{a}(t) \subset
      \mathbb{R}^N \\
   \mathbf{x}_j(t) &= x_j(t), \qquad \qquad \qquad\qquad
      \mathbf{x}_i(t)\in \mathbf{x}(t) \subset \mathbb{R}^n\\
   \mathbf{B}_{iJ} &= b_j(\tau_i)/\tau_i \qquad \qquad \qquad
   \mathbf{B}_{iJ} \in \mathbf{B} \subset \mathbb{R}^N \times 
      \mathbb{R}^n
\end{align*}
Then we let $\Delta \mathbf{r}(t)$ and $\Delta \mathbf{x}(t)$ denote
the demeaned change $\mathbf{r}(t)$ and $\mathbf{x}(t)$, respectively,
so that
\begin{equation}
   \label{pca2}
   \Delta \mathbf{r}(t) = \mathbf{B} \Delta \mathbf{x}(t)
\end{equation}
If we assume that the factors are not linearly dependent, then the
rank of the $N \times N$ covariance matrix of $\Delta \mathbf{r}$ 
(denoted by $V$) will be $\min\{ n, N\}$. So if we could measure 
the covariance matrix, $V$, without error, we could estimate
the number of factors by simply obtaining data on $\Delta \mathbf{r}$
with $N$ sufficiently large so that $N > n$ and then computing the rank
of $V$.
\\
\\
{\sl Measurement Error:} However, because there will be 
measurement/sampling error, this won't work. Even if rank$(V) = n < N$,
measurement errors will distort linear independence between rows and
columns. So we'll try to find the ``best'' possible approximation
of the observed covariance matrix, $V$, by a covariance matrix of
rank $n$, denoted $V_n$. Then, we'll determine how large $n$ 
has to be in order to obtain an acceptable approximation.

\subsection{Approximating the Covariance Matrix}

We'll want to approximate the covariance matrix with a matrix
of lower rank.  In doing so, we'll need to use eigenvectors. Here's
the main process: If $V$, an $N\times N$ covariance
matrix, then
\begin{enumerate}
   \item $V$ has exactly $N$ distinct eigenvectors and the rank
      of $V$ is equal to the number of non-zero eigenvalues.
   \item Let $\Lambda_N$ be the $N\times N$ diagonal matrix
      whose diagonal elements are the eigenvalues of $V$. Also, 
      let $Q_N$ be the $N\times N$ matrix whose columns are the
      eigenvectors of $V$. Then we have $V = Q_N \Lambda_N Q_N^T$.
   \item The best best rank $n\leq N$ approximation of $V$ is
      \[ V_n = Q_n \Lambda_n Q_n^T \]
      where $\Lambda_n$ is an $n\times n$ diagonal matrix whose
      diagonal elements are the $n$ largest eigenvalues of $V$ and
      $Q_n$ is the $N\times n$ matrix whose columns are the 
      eigenvectors corresonding to the $n$ largest eigenvalues of $V$.
\end{enumerate}
The best rank $n$ approximation $V_n$ to covariance matrix $V$ is the
covariance matrix corresponding to a $n$-factor model, and
the factors in this $n$-factor model are called the first $n$
principal components of $\mathbf{r}$.
\\
\\
Note that it also follows that
   \[ \Delta \mathbf{x}(t) = Q_n^T \Delta\mathbf{r}(t) \]
   \[ \mathbf{B} = Q_n \]
Moreover, the $n$ factors jointly explain a fraction of the total
variance of the interest rates included in $\Delta \mathbf{r}$:
   \[ \frac{\sum^n_{i=1} \lambda_i}{\sum^N_{i=1}\lambda_i } \]
where $\lambda_1, \ldots, \lambda_N$ are the eigenvalues of $V$
sorted in decreasing order.


\newpage
\section{Antithetic Variates: A Variance Reduction Technique}

The technique of \emph{antithetic variates} reduces the sample
size needed to achieve a given precision (in the form of standard error)
in Monte Carlo simulations. Let's go through the steps:
\begin{enumerate}
   \item{Suppose that we want to estimate $EY$ using MC simulation. 
      Instead of simulating a single sample of size $N$, we simulate
      $N/2$ pairs, $(Y_1, Y_2)_i$, where each pair is i.i.d. for
      all pairs $i=1,\ldots,N/2$.}
   \item{We can get an unbiased estimator of $EY$ by first
      computing the average value $\bar{Y}_i = 
      \frac{1}{2}(Y_{i1}+ Y_{i2})$
      then averaging across the $\bar{Y}_i$.} 
   \item{If we simulate the $N/2$ pairs independently, then 
      \begin{align*}
	 Var(\bar{Y}_i) &= Var\left(\frac{1}{2}(Y_{i1}+ Y_{i2})\right) =
	 \frac{1}{4}\left[Var(Y_{i1})+Var(Y_{i2})+2 Cov(Y_{i1}, Y_{i2})
	 \right]
	 \\
	 &=  \frac{(1+\rho)}{2} Var(Y), \quad \text{$\rho$ is the 
	    correlation between $Y_{i1}$, $Y_{i2}$}
      \end{align*}
      So then if we want the total variance of $Y$ over our entire 
      simulation, then we get 
	 \[ Var(\bar{Y}) = \frac{(1+\rho) Var(Y)}{ 2 (N/2)}
	    = \frac{(1+\rho) Var(Y)}{N} \]
      }
   \item{Clearly, by making the correlation negative between the 
      terms $Y_1$ and $Y_2$ in each given pair, we can lower the variance
      of our estimator for $EY$ as a whole.  (Note that the pairs
      themselves are still independent from pair to pair.)}
\end{enumerate}
Note that some of the negative correlation will wash out and we
won't get perfect -1 correlation when we jump from random draws to
prices, as typically the prices are some convoluted \emph{function}
of the random draws.
\\
\\
Also, the gain---while significant---for path \emph{dependent} 
derivatives is not quite as large due to the dependence of the payoffs
upon the entire path.



\end{document}

