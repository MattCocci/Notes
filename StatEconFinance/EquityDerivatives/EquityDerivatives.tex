\documentclass[a4paper,12pt]{scrartcl}

\author{Matthew Cocci}
\title{Notes to Financial Engineering: \\ Equity Derivatives}
\date{}
\usepackage{enumitem} %Has to do with enumeration	
\usepackage{amsfonts}
\usepackage{amsmath}
\usepackage{amsthm} %allows for labeling of theorems
\usepackage[T1]{fontenc}
\usepackage[utf8]{inputenc}
\usepackage{blindtext}
\usepackage{graphicx}
\usepackage[hidelinks]{hyperref} 
%\numberwithin{equation}{section} 
%, This labels the equations in relation to the sections rather than other equations
%\numberwithin{equation}{subsection} %This labels relative to subsections
\newtheorem{thm}{Theorem}[section]
\newtheorem{lem}[thm]{Lemma}
\newtheorem{prop}[thm]{Proposition}
\newtheorem{cor}[thm]{Corollary}
\setkomafont{disposition}{\normalfont\bfseries}
\usepackage{appendix}
\usepackage{subfigure} % For plotting multiple figures at once




\begin{document}
\maketitle

\tableofcontents

\newpage
\section{The Fundamental Theorem of Asset Pricing}

\subsection{Introduction}

Derivatives require special pricing techniques aside from the traditional
discounted cash flow (DCF) approach, as DCF requires an estimate of the
appropriate risk-adjusted rate of return.  However, the risk of a 
derivative varies over time, which makes it difficult to estimate the
derivative's risk-adjusted return.

As a result, derivatives pricing turns to the no-arbitrage approach (NA),
which eliminates the need to build risk into the model.  

\subsection{Trading Strategy and Derivative Pricing Definitions}

A \textbf{trading strategy} is a dynamically-rebalanced portfolio.
\\
\\
A trading strategy is \textbf{self-financing} if it generates no 
intermediate cash inflows and requires no intermediate outflows between
the time the portfolio is initiated and the time it is liquidated.  This
implies that
   \begin{itemize}
      \item[i.]{All dividends are reinvestd.}
      \item[ii.]{Value of the assets sold at a rebalance time must 
	 equal the value of the assets bought.}
   \end{itemize}
A trading strategy is \textbf{strictly positive} if the value of the 
traded portfolio can never become zero or negative.
\\
\\ 
Let $N$ be the value of a \emph{strictly positive}, \emph{self-financing}
trading strategy.  Then $N$ is a \textbf{numeraire process} or, simply,
a \textbf{numeraire}. Here are a few examples:
\begin{itemize}
   \item[-]{Price of Dividend paying asset: 
	 NO, as there are intermediate 
	 cash outflows, violating self-financing condition.}
   \item[-]{The price of a forward contract: NO, as it can go negative,
      violating the strictly positive condition.}
   \item[-]{Price of a Foreign Currency: 
      NO, as it is equivalent to a dividend
      paying asset because you think of it as an investment in an
      interest-bearing account.}
   \item[-]{Price of a non-defaultable zero-coupon bond: YES.}
   \item[-]{Value of a money market account earning the risk free rate,
      where there are no interim deposits or withdrawals: YES.}
\end{itemize}

\subsection{Martingales and Change of Measure}

A \textbf{martingale} is a stochastic process
$X$ with the property 
   \[ E_t[X(T) - X(t)] = 0 \; \Leftrightarrow \; E_t[X(T)] = X(t), \qquad
      T > t. \]
A \textbf{probability measure} is a specification
of the probabilities of all the possible states of the words, mapping
states to real numbers.
\\
\\
Suppose that $\xi$ is a nonnegative random variable on $(\Omega,
\mathcal{F}, P)$ with $E_P[\xi] = 1$. (The subscript $P$ highlights that
the last expectation is with respect to measure $P$.) 
Then define a new measure
   \[ Q: \mathcal{F} \rightarrow [0,1] \]
\begin{equation}
   \label{rnt}
   Q(A) = E\left[1_A \xi\right]=\int_A\xi(\omega) dP(\omega), \qquad
      A\in \mathcal{F} 
\end{equation}
Clearly, $Q$ is a probability measure on $(\Omega, \mathcal{F})$ and
it is absolutely continuous with respect to $P$---i.e. we have
   \[Q(A) > 0 \Rightarrow P(A) > 0.\] 
Note that it is common to write the random variable $\xi$ as
   \[ \xi = \frac{dQ}{dP},\]
and we often refer to $\xi$ as the \emph{Radon-Nikodym derivative}
or the \emph{likelihood ratio} of $Q$ with respect to $P$.

\paragraph{Radon-Nikodym Theorem} If $P$ and $Q$ are two probability
measures on $(\Omega, \mathcal{F})$, then there \emph{will exist}
such a random variable $\xi$ so that Expression \ref{rnt} holds.


\newpage
\subsection{Fundamental Theorem of Asset Pricing (No Dividends)}

\subsubsection{Statement of Theorem}

Suppose we have $n$ non-dividend-paying assets with
price processes $S_1, S_2, \ldots, S_n$.  Let $N$ be some numeraire
process.  Then, barring market imperfection, there are no arbitrage
opportunities among these assets if and only if there exists a strictly
positive probability measure $Q_N$ (so it's dependent upon the numeraire,
$N$) under which each of the processes $S_i / N$ is a martingale.
\begin{itemize}
   \item[-]{Note that $S_i/N$ is the price of asset $i$ in units of the
      numeraire $N$.  Therefore, we call $S_i/N$ the \textbf{normalized
      price process}.}
   \item[-]{The probability measure $Q_N$ will, in general depend on
	 the numeraire.  Therefore, we call $Q_N$ the 
	 \textbf{martingale measure} or \textbf{pricing measure} 
	 associated with the numeraire $N$.} 
   \item[-]{We can paraphrase FTAP by saying that, if the is no arbitrage
	 or market imperfections, then given \emph{any} numeraire 
	 process $N$, there must exist a corresponding martingale
	 measure $Q_N$ under which the normalized price of any 
	 non-dividend paying asset is a martingale:
	    \[ \frac{S_i(t)}{N(t)} = E_t^{Q_N}\left[ \frac{S_i(T)}{N(T)}
	       \right], \qquad T > t. \]
	 From this, we see that changing $N$ will generally change 
	 $Q_N$ as well.
      }
\end{itemize}

\subsubsection{Consequences for Derivative Pricing}

Let $V$ denote the price of a derivative with payoff $V(T)$ at time $T$.
Then we can apply FTAP to get
   \[ V(t) = N(t) E_t^{Q_N} \left[ \frac{S_i(T)}{N(T)}
	       \right], \qquad T > t. \]
Note, the price we get for a deriative is \emph{invariant} to the choice
of the numeraire.

\subsubsection{Special Numeraires and Martingale Measures}

\paragraph{T-forward measure} Let $P(t,T)$ be the price of a
non-defaultable zero-coupon bond with unit face value.  Then
   \[ N(t) = P(t,T) \]
is our numeraire.  The associated martingale measure, denoted 
$Q_T$, is called the \emph{T-forward martingale measure}. This yields
a derivative price of 
   \[ V(t) = E_t^{Q_B} \left[ e^{-\int_t^T r(s) ds} V(T)\right] \]

\paragraph{Risk Neutral Measure} Let's consider the value of a money
market account with unit initial value as our numeraire.  Then
   \[ N(t) = B(t) = e^{\int_0^t r(s) ds} \]
where $r$ is the instantaneous risk-free rate.  The associated 
martingale measure, denoted by $Q_B$, is called the 
\emph{risk-neutral martingale measure}. This yields a derivative price
of 
   \[ V(t) = P(t,T) \; E_t^{Q_T}\left[V(T)\right] \; = \;
      e^{-r(t,T) (T-t)} E_t^{Q_T}\left[V(T)\right] \]
   \[ r(t,T) = -\ln{P(t,T)}/(T-t) \]
If interest rates are stochastic (and they probably are), then this
measure isn't as convenient a tthe $T$-forward measure.

\subsection{FTAP for Dividend-Paying Assets}

Consider an asset with price process $S$ and let $D(t)$ denote the
cumulative dividend paid by the asset from time 0 up to time $t$.
We can consider the undiscounted cash flows from holding an asset
from $t$ to $T$:
   \[ S(T) - S(t) + D(T) - D(t) = GP(T) - GP(t) \]
where $GP(t) = S(t) + D(t)$ is the asset's gain process.
\\
\\
Given a numeraire $N$, the asset's \emph{normalized gain process},
denoted NGP, measures the gains from holding the assets in units of $N$:
\[ NGP(t) = \frac{S(t)}{N(t)} + \int^t_0 \frac{dD(s)}{N(s)}\]
where $dD(s)$ is the dividend paid by the asset at time $s$.

\paragraph{Theorem} Now, let's restate the fundamental theorem of asset
pricing allowing for dividend paying assets.  So again, consider
assets with price processes $S_1, \ldots, S_n$ and cumulative dividend
processes $D_1,\ldots,D_n$, letting $N$ be any numeraire process.
Then there are no arbitrage opportunities across these assets if and
only if there exists a strictly positive probability measure $Q_N$ 
under which each
   \[ \frac{S_i(t)}{N(t)} + \int^t_0 \frac{dD_i(s)}{N(s)} \]
is a martingale. This implies
\[ \frac{S_i(t)}{N(t)}  = E_t^{Q_N}\left[ \frac{S_i(T)}{N(T)} +
   \int^T_t \frac{dD_i(s)}{N(s)}\right] \]
And so the normalized price of any asset is equalt to the conditional
expectation under the martingale measure of the assets normalized payoffs
(including future dividends).

\newpage


\section{Continuous Time Stochastic Processes}

Here, we develop the necessary machinery in continuous time stochastic
processes to model asset price evolution properly and with sufficient
richness and generality. In particular, we discuss a heirarchy of model
classes that includes Brownian Motion $\subset$ Generalized Brownian
Motion $\subset$ Diffusions $\subset$ Ito Processes.

\subsection{Introduction}

A \emph{stochastic process} $X$ is a collection of random variables
indexed by time: $X = \{ X_t: t \in \mathcal{T} \}$. 
\begin{itemize}
   \item[-]{\emph{Discrete Time}: $\mathcal{T}$ countable, and process
      changes only at discrete time intervals.}
   \item[-]{\emph{Continuous Time}: $\mathcal{T}$ uncountable.}
\end{itemize}
\paragraph{Definition} A process $X$ has stationary increments if 
$X_T - X_t$ has the same distribution as $X_{T'} - X_{t'}$ provided
that $T-t= T'-t'$.

\subsection{Brownian Motion}

\paragraph{Definition} The most basic continuous-time process is 
\emph{Brownian Motion} (or the \emph{Wiener Process}).  It has three
defining properties:
\begin{enumerate}
   \item[i.]{$W(0) = O$.}
   \item[ii.]{$W(t)$ is continuous, so no jumps.}
   \item[iii.]{Given any two times, $T>t$, the increment $W(T) - W(t)$ is
      independent of all previous history and normally distributed
      with mean $0$ and variance $T-t$.}
\end{enumerate}
A few consequences of the definition of $W(t)$:
\begin{itemize}
   \item[-]{Brownian motion has independent stationary increments.}
   \item[-]{$W(t)$ is normally distributed with $\mu =0$, $\sigma^2 = t$.
      }
\end{itemize}

\subsection{Generalized Brownian Motion}

\paragraph{Definition} A \emph{generalized Brownian motion} is a 
continuous-time process $X$ with the following property:
   \[ X(t) = X(0)+\mu t + \sigma W(t) \]
where $\mu$ is the \emph{drift}, $\sigma$ is the \emph{volatility},
and $W$ is simple Brownian motion. The differential equation 
equivalent is written:
   \[ dX(t) = \mu \; dt + \sigma \; dW(t).\]
It follows immediately from the definition that
\begin{itemize}
   \item[-]{$X(t)$ is continuous, so no jumps.}
   \item[-]{$X(t)$ is normally distributed with mean $X(0)+\mu t$, 
      variance $\sigma^2 t$.
      }
   \item[-]{Given any two times, $T>t$, the increment $X(T) - X(t)$ is
      independent of all previous history and normally distributed
      with mean $\mu(T-t)$ and variance $\sigma^2(T-t)$.}
   \item[-]{$X$ is a martingale if and only if $\mu = 0$.}
\end{itemize}
\paragraph{Theorem} It also happens that Generalized Brownian motions
are the only continuous time processes with continuous sample
paths and stationary increments.

\subsection{Ito Processes}

Even more general than Brownian Motion (which is retained as a special
case), an \emph{Ito Process} is 
a stochastic process $X$ defined by one of two equivalent formulations:
   \[ dX(t) = \mu(t) \; dt + \sigma(t) \; dW(t) \]
   \[ X_t = X_0 + \int^t_0 \mu(s) \; ds + \int^t_0 \sigma(s) \; dW(s)
      \]
for any arbitrary stochastic processes $\mu$ (the drift) and $\sigma$
(the volatility) along with some Brownian Motion $W(t)$.  Here are 
some properties
\begin{itemize}
   \item[-]{Has continuous sample paths and is a martingale if and only
      if $\mu(t) = 0$. }
   \item[-]{Increments are not necessarily stationary, as $\mu$ and
	 $\sigma$ can change \emph{randomly} with time.}
\end{itemize}

\paragraph{Definition} If the drift and volatility of an Ito process 
depend only upon the current value of the process and time, then
$X$ is a \emph{diffusion}.  Mathematically, $X$ is a \emph{diffusion}
if
   \[ dX(t) = \mu(X(t),t)\; dt+ \sigma(X(t),t)\; dW(t) \]
for some functions $\mu$ and $\sigma$.

\subsection{Ito's Lemma}

Suppose that $X$ is an Ito process defined by 
\begin{equation}
   \label{ito}
   dX(t) = \mu(t) \; dt + \sigma(t) \; dW(t)
\end{equation}
and we define a new process $Y(t) = f(X(t),t)$ where $f$ is some
function that's twice differentiable in $X$ and once in $t$.  Then
we have that
\begin{equation}
   \label{lemma}
   dY(t) = f_X(X(t),t) \; dX(t) + f_t(X(t),t) \; dt + \frac{1}{2} 
   f_{XX}(X(t),t)\sigma(t)^2 \; dt.
\end{equation}
where subscripts on $f$ denote the partial derivatives.\footnote{Note 
that Equation \ref{lemma} almost looks like the chain rule from 
traditional calculus, except for that extra term with $f_{XX}$ partial 
derivative. That arises from the additional variability due to the 
inclusion of stochastic factors like $W(t)$ in the original Ito 
Process.} Subbing Equation \ref{ito} into Equation \ref{lemma}, 
we get that
   \[ dY(t) =\left(f_X(X(t),t)\mu(t) + f_t(X(t),t)  + \frac{1}{2} 
      f_{XX}(X(t),t)\sigma(t)^2 \right) \; dt + 
      f_X(X(t),t)\sigma(t) \; dW(t)
   \]
Thus, it is clear that $Y$ is also an Ito Process by the statement
above with the drift and volatility given by the coefficients on
$dt$ and $dW(t)$ as always.

\paragraph{Using Ito's Lemma} In practice, we use Ito's Lemma
whenever we have a (typically complicated) Ito Process that we want
to solve.  Given the process $X$ and its corresponding Ito Process, 
we posit a function $f$ that could could help. Then we write a
new Ito Process using Ito's lemma with $dY(t) = df(X(t),t)$ on the LHS.
From there, hopefully we can integrate $dY(t)$ easily on the left and
solve out for $X(t)$.

\subsection{Multi-dimensional Ito's Lemma}

For the sake of completeness, let's generalize Ito's Lemma to consider
the case of a finite number of Ito processes, $X_1, X_2, \ldots, X_n$,
   \[ dX_i(t) = \mu_i(t)\; dt + \sigma_i(t) \; dW_i(t).\]
Next, define $Y(t) = f(X_1(t), \ldots, X_n(t), t)$ for some 
differentiable function $f$.  Then multi-dimensional Ito's Lemma says
\begin{align*}
    dY(t) &= \sum^n_{i=1} f_{X_i}(X_1(t), \ldots, X_n(t), t) \; 
      dX_i(t) \\
    &+ f_{t}(X_1(t), \ldots, X_n(t), t) \; dt \\
    &+ \frac{1}{2} \sum^n_{i=1}\sum^n_{j=1} 
      f_{X_i,X_j}(X_1(t), \ldots, X_n(t), t) \; \rho_{ij}\sigma_i(t)
      \sigma_j(t) \; dt 
\end{align*}
where $\rho_{ij}$ is the correlation coefficient between $dW_i$ and
$dW_j$. Note that you'll have to plug back in for the $dX_i$ in the
first sum.
\\
\\
We'll mostly consider with the two-dimensional case for the 
two specific instances below:
\begin{itemize}
   \item[i.]{$Y(t) = X_1(t)X_2(t)$, which gives us
      \[ dY(t) = X_2(t) \; dX_1(t) + X_1(t) \; dX_2(t) +
	 \rho_{12}\sigma_1(t)\sigma_2(t) \; dt \]
      Note that you'll have to plug back in for $dX_1$ and $dX_2$.
      This is a type of integration-by-parts formula because (after
      rearranging terms) it relates $X_1\; dX_2$ to $X_2\; dX_1$.
   }

   \item[ii.]{$Y(t) = X_1(t)/X_2(t)$, which gives us
      \begin{align*}
	 dY(t) = \frac{1}{X_2(t)} dX_1(t) - \frac{X_1(t)}{X_2(t)^2}
	 dX_2(t) +  \frac{X_1(t)}{X_2(t)^3} \sigma_2(t)^2 \;dt -
	 \frac{1}{X_2(t)^2} \; \rho_{12}\sigma_1(t)\sigma_2(t) \; dt. 
      \end{align*}
      Note that you'll have to plug back in for $dX_1$ and $dX_2$.
      This is a type of integration-by-parts formula because (after
      rearranging terms) it relates $X_1\; dX_2$ to $X_2\; dX_1$.
      }
\end{itemize}


\subsection{Geometric Brownian Motion}

Let's consider the process $X$ governed by
   \[ dX(t) = X(t)\mu(t) \; dt + X(t) \sigma(t) dW(t).\]
To solve, let us consider the process $Y(t) = \log X(t)$.  We compute
the partials and apply Ito's Lemma:
   \[ f_X = \frac{1}{X(t)}, \qquad f_{XX} = -\frac{1}{X(t)^2},
      \qquad f_t = 0 \]
   \[ dY(t) = \frac{1}{X(t)} dX(t) - \frac{1}{2}\frac{1}{X(t)^2}
      (\sigma(t) X(t))^2 ) \; dt \]
which simplifies (after subbing in for $dX(t)$) into the expression
   \[ dY(t) = \left( \mu(t) - \frac{1}{2}\sigma(t)^2\right) \; +
      \sigma(t) \; dW(t).\]
Next, integrating both sides and substituting back in with
$Y(t) = \log X(t)$, we get
   \[ Y(t) = Y(0) + \int^t_0 \left( \mu(s)-\frac{1}{2}\sigma(s)^2\right) 
      ds + \int^t_0 \sigma(s) \; dW(s) \]
   \[ X(t)=X(0) e^{\int^t_0 \left( \mu(s)-\frac{1}{2}\sigma(s)^2\right) 
       ds + \int^t_0 \sigma(s) \; dW(s)}\]
where it also follows that $X$ is strictly positive.

\paragraph{Special Case} Suppose that $\mu$ and $\sigma$ are constant,
in which case $X$ follows a \emph{geometric Brownian motion}.  Then
it follows that 
   \[ \log X(t) \sim N\left( \log X(0) + \left(\mu - \frac{1}{2} \sigma^2
      t\right), \sigma^2 t \right) \]
so that we say $X(t)$ is \emph{lognormally distributed}.

\subsection{Girsanov's Theorem}

\paragraph{Theorem} Suppose that $X$ is an Ito process
   \[ dX(t) = X(t) \mu(t) \; dt + X(t) \sigma(t) \; dW(t),\]
where $\mu$ an $\sigma$ are stochastic processes and $W$ is Brownian
Motion under some probability measure $P$---like maybe the real 
world measure.  Then if $Q$ is any other strictly positive probability
measure, then $X$ is also an Ito process under $Q$---i.e., there
exist processes $\hat{\mu}$, $\hat{\sigma}$, and $\hat{W}$ with the
property that
\begin{equation}
\label{Girsanov}
    dX(t) = X(t) \hat{\mu}(t) \; dt + X(t) \hat{\sigma}(t) \; 
      d\hat{W}(t).
\end{equation}
Even better, $\hat{\sigma} = \sigma$.
\\
\\
This is particularly useful because, in general, we will have to work 
with two different probability measures: the true/historical $P$ and
the martingale probability measure $Q_N$. 



\newpage


\section{The Black-Scholes Model}

\subsection{Deriving the Governing Process Under $Q_B$}

Let's put to use everything we've developed so far and derive the 
Black-Scholes Model.
\\
\\
First, we start by specifying the assumptions of the model:
\begin{enumerate}
   \item{The price of the underlying asset, the stock, follows
      an Ito process with volatility proportional to the price
      level:
      \begin{equation}
	 \label{sde}
	 dS(t) = S(t)\mu(t) \; dt + S(t) \sigma \; dW(t).
      \end{equation}
      This is the movement under the real-world probibility measure $P$.
      }
   \item{It pays dividends continuously at a constant rate $\delta$,
      implying that the cumulative dividend at time $t$ is
	 \[ D(t) = \int^t_0 S(s) \delta \; ds .\]
      }
   \item{The instantaneous risk-free rate is \emph{constant} and 
      equal to $r$.}
   \item{There are no arbitrage opportunties or market imperfections.}
\end{enumerate}
Now, we can apply FTAP to say that the value of a Path-Independent
Option (PID) with payoff $V(T) = \varphi(S(T))$ at time $T$ can be 
expressed as
   \[ V(t) = e^{-r(T-t)}E_t^{Q_B}\left[\varphi(S(T))\right].\]
To compute the expression on the RHS, we will need to determine
the distribution of $S(T)$ under the risk-neutral measure $Q_B$, or
we will need to simulate $S(T)$ under $Q_B$.  In that case, 
we will employ Girsanov's Theorem to our SDE from Equation \ref{sde}
to rewrite the evolution of the process now under the risk neutral
measure
\begin{equation}
   \label{girs}
   dS(t) = S(t)\hat{\mu}(t) \; dt + S(t) \sigma \; d\hat{W}(t) 
\end{equation}
where $\hat{W}$ is Brownian motion under $Q_B$. We know everything
written here \emph{except} $\hat{\mu}$, whose computation will
now be our main concern.
\\
\\
So in order to determine $\hat{\mu}$, we will need to compute
the dynamics of the normalized stock gain process:
\begin{equation}
   \label{ngp}
   NGP(t) = \frac{S(t)}{B(t)} + \int^t_0 \frac{dD(s)}{B(s)} =
   \frac{S(t)}{B(t)} + \int^t_0 \frac{S(s)\delta \; ds }{B(s)}.
\end{equation}
Now it follows from Equation \ref{ngp} that
\begin{equation}
   \label{ngp2}
   dNGP(t) = d\left(\frac{S(t)}{B(t)}\right) + \frac{S(t)}{B(t)}
      \delta \; dt. 
\end{equation}
Next, since we assume $r$ to be constant, by the way we defined 
$B(t)$, it follows that 
   \[ B(t) = e^{\int^t_0 r(s) ds} \;\; \Rightarrow 
      \;\; dB(t) = B(t)r \;dt\]
Now if we apply Multi-dimensional Ito's Lemma for the ratio of the
two processes (a special case we discussed above), we get
\begin{equation}
   \label{ratio}
   d\left(\frac{S(t)}{B(t)}\right) = \frac{S(t)}{B(t)} 
      \left(\hat{\mu}(t) - r\right) \; dt + \frac{S(t)}{B(t)}\sigma(t)
      \; d\hat{W}(t).
\end{equation}
Substituting Equation \ref{ratio} into Equation \ref{ngp2}, we finally
get that
\begin{equation}
   \label{ngp3}
   dNGP(t) = \frac{S(t)}{B(t)}\left(\hat{\mu}(t) + \delta - r\right)
   \; dt + \frac{S(t)}{B(t)} \sigma(t) \; d\hat{W}(t).
\end{equation}
Now since FTAP tells us that the normalized gain process must be 
a martingale, we know that the the drift term in Equation \ref{ngp3}
will need to be 0, forcing
   \[ \hat{\mu}(t)= r - \delta.\]
So substituting this result back into Equation \ref{girs}, which
we got through Girsanov's theorem, we obtain
   \[ dS(t) = S(t)(r - \delta) \; dt + S(t)\sigma \; d\hat{W}(t)\]
which, you'll note, forces the stock price to follow a geometric
Brownian Motion under $Q_B$. GBM leads to the following expression
for $S(T)$:
   \[ S(T) = S(t) e^{\left(r - \delta -\frac{1}{2}\sigma^2\right)
      (T-t) + \sigma(\hat{W}(T) - \hat{W}(t))} \]
As a consequence, $\ln S(T)$ is normally distributed with mean
   \[ \ln S(t) + \left(r - \delta - \frac{1}{2}\sigma^2\right) (T-t) \]
and standard deviation $\sigma\sqrt{T-t}$, \emph{under the measure 
$Q_B$}.  From there, we can
easily simulate samples from the distribution of $S(T)$ for 
Monte Carlo simulation, which will be covered in later sections.


\newpage

\subsection{Arriving at a Closed-Form Solution}

With a little bit of statistics, we can work out explicit formulas
for digital and vanilla options under the Black-Scholes model and 
assumptions, so let's build up to that.
\\
\\
Consider $X$, where $\ln X$ is normally distributed with mean $\mu_X$
and standard deviation $\sigma_X$.  Then if $1_{\{X\geq K\}}$ is the
indicator function for $X \geq K$, then we have that
\begin{equation}
   \label{ind}
   E\left[1_{\{X\geq K\}}\right] = P(X\geq K) = \Phi \left(
      \frac{\mu_X - \ln K }{\sigma_X}\right) 
\end{equation}
\begin{equation}
   \label{bitch}
    E\left[X \cdot 1_{\{X\geq K\}}\right] = 
      e^{\mu_X + \frac{1}{2}\sigma_X^2} \; \Phi\left(\frac{\mu_x + 
      \sigma_X^2 - \ln K}{\sigma_X} \right) 
\end{equation}
Coming back to the Black-Scholes model, instead of $X$, we'll take
$S(T)$, where we recall that $\ln S(T)$ is normally distributed under 
$Q_B$ with mean
$\ln S(t) + ( r - \delta - (1/2)\sigma^2) (T-t)$
and standard deviation $\sigma \sqrt{T-t}$. From there, we apply
Result \ref{ind} to say that
   \[ E_t^{Q_B}\left[ 1_{\{S(T)\geq K\}}\right] = 
      \Phi\left( \frac{\ln S(t) + \left(r-\delta - \frac{1}{2} \sigma^2
      \right)(T-t) - \ln K }{\sigma \sqrt{T-t}}\right) \]
We can use this result to price a \emph{cash-or-nothing} (CON) call
which pays $\varphi(S(T)) = 1_{\{S(T)\geq K\}}$ at $T$:
\begin{align}
   \label{con}
   c^{\text{CON}}(t) &= e^{-r(T-t)} 
      E_t^{Q_B}\left[ 1_{\{S(T)\geq K\}}\right]
   \notag\\
   &=  e^{-r(T-t)} 
      \Phi\left( \frac{\ln S(t) + \left(r-\delta - \frac{1}{2} \sigma^2
      \right)(T-t) - \ln K }{\sigma \sqrt{T-t}}\right) 
\end{align}
Now let's adapt Result \ref{bitch} to the Black Scholes model and use
it to price another type of derivative---the so-called 
\emph{all-or-nothing} (AON) call, which has payoff $\varphi(S(T)) = 
S(T) 1_{\{S(T) \geq K\}}$ at time $T$. The price of this option at
$T$ is 
\begin{align}
   \label{aon}
      c^{\text{AON}}(t) &= e^{-r(T-t)} 
      E_t^{Q_B}\left[ 1_{\{S(T)\geq K\}}\right]
      \notag\\
      &= e^{-r(T-t)}  \; S(t) \; e^{(r-\delta)(T-t)} \;
      \Phi\left(\frac{\ln S(t) + \left(r-\delta + \frac{1}{2}\sigma^2
      \right)(T-t) - \ln K}{\sigma \sqrt{T-t}}\right)
      \notag\\
      &= S(t) e^{-\delta (T-t)} \Phi\left(\frac{\ln S(t) + 
	 \left(r-\delta + \frac{1}{2}\sigma^2
	 \right)(T-t) - \ln K}{\sigma \sqrt{T-t}}\right)
\end{align}
From there, we price a \emph{vanilla call} by expressing it as a
portfolio of cash-or-nothing and all-or-nothing calls:
   \[ c(t) = c^{\text{AON}}(t) - K  c^{\text{CON}}(t) \]
We then get the value by using Formulas \ref{con} and \ref{aon}.

\newpage


\subsection{General Logic}

So let's just recap the general steps and intuition that we employed
at arriving at the final solution of a PID, being a bit more general
than the specific Black-Scholes case:
\begin{enumerate}
   \item{First, we posit some governing process for the stock price
      under the real-world probability measure, $P$.}
   \item{Next, use Girsanov's Theorem to assert the existence of a
      governing process under the probability measure $Q$.  As per 
      Girsanov's Theorem, this new process
      will have the same variance as the
      original process, but \emph{not} the same drift (note that
      the drift under $P$ is largely irrelevant in how we price
      derivatives.)}
   \item{Then, figure out how to express the normalized gain process
      for the underlying. According to FTAP, 
      this must be a Martingale under $Q$, implying a drift of 0.}
   \item{Using the fact that the drift of $dNGP$ must be 0, solve
      out for $\hat{\mu}$, and plug that value into the governing SDE
      under $Q$.}
   \item{Finally, solve out the SDE under $Q$, and simulate many 
      sample paths from the solution, computing the value of 
      the derivative at time $T$ under each. Take the average of all of 
      these values to get the approximate expectation under $Q$.}
\end{enumerate}

\subsection{Relaxing Some Assumptions}

Recall that the Black-Scholes model assumes that $r$, $\delta$, and
$\sigma$ are all \emph{constants}. However, we can easily adapt in the
case that each one is a \emph{non-random} function of 
time.\footnote{So no stochastic factors can influence any of those 
parameters}. Specifically, we replace $\theta (T-t)$\footnote{$\theta$
is any one of the three parameters.} with $\bar{\theta}(T-t)$:
   \[ \bar{r} = \frac{1}{T-t} \int^T_t r(s) ds, \qquad
      \bar{\delta} = \frac{1}{T-t} \int^T_t \delta(s) ds, \qquad
      \bar{\sigma^2} = \frac{1}{T-t} \int^T_t \sigma(s)^2 ds \]




\newpage

\section{Monte Carlo Pricing of PIDs}

Let's consider a path-independent derivative (PID) that provides the
payoff $\varphi(S(T))$ at expiration date $T$. FTAP tells us that
\begin{equation}
   \label{vee}
   V(0) = e^{-r T} E^{Q_B}\left[\varphi(S(T))\right],
\end{equation}
where $E^{Q_B}$ is the expectation under the risk-neutral probability.
\\
\\
Next, if we believe the assumptions of the Black-Scholes model, then
\begin{equation}
   \label{st}
   S(T)=S(0)\; e^{\left(r - \delta - \frac{1}{2}\sigma^2\right) T +
      \sigma \hat{W}(T)} 
\end{equation}
If we wish to compute the price $V(0)$ numerically (as opposed to 
analytically), then we use Monte Carlo simulation:
\begin{enumerate}
   \item{Draw a random value from a 
	 $N(0, (\sqrt{T})^2)$ 
      distribution.}
   \item{Plug that value into Expression \ref{st} to get a random
      draw from the distribution of $S(T)$ under $Q_B$.}
   \item{Compute the payoff, $\varphi(S(T))$.}
   \item{Repeat those steps to get many draws of $\varphi(S(T))$ under 
	 $Q_B$, then average them to get an unbiased estimate of 
	 $E^{Q_B}\left[\varphi(S(T))\right]$.}
   \item{Multiply that average by the discount factor $e^{-rT}$ to
      get an estimate of the PID price.}
\end{enumerate}
Now that we've figured out a way to compute a numeric approximation, 
it's naturual to ask how accurate this is.  To do so, we use the
Central Limit Theorem to assert
\[ EX - \bar{X} \sim N\left(0,\; \frac{\sigma_X}{\sqrt{N}}\right) \] 
as the number of draws gets very large, and where $EX$ is the true
mean, and $\bar{X}$ is the sample mean from the Monte Carlo simulation.
Getting even more specific,
   \[ \hat{V}_0 \sim N\left(V_0, \; e^{-r T} 
      \frac{\sigma_\varphi}{\sqrt{N}}\right) \]
where $\sigma_\varphi$ is the standard deviation of the payoff 
$\varphi(S(T))$, $V_0$ is the true price, and $\hat{V}_0$ is the Monte
Carlo estimate.
   
\newpage

\section{Monte Carlo Pricing of PDDs}

Next, consider a generic \emph{discretely-sampled} European 
path-dependent derivative with the payoff dependent upon the price of 
the underlying at a set of dates $\{t_1, \ldots, t_n\}$ where 
$t_i \in [t, T]$, i.e.
   \[ V(T) = \varphi(S(t_1), \ldots, S(t_n)).\]
Recalling FTAP and the path of stock prices under the Black-Scholes 
assumption, 
   \[ V(t) = e^{-r(T-t)} E_t^{Q_B} \left[\varphi(S(t_1), \ldots, S(t_n))
      \right] \]
   \[ S(t_i) = S(t_{i-1}) e^{\left(r-\delta - \frac{1}{2}\sigma^2\right)
      (t_i - t_{i-1}) + \sigma \left(\hat{W}(t_i) - \hat{W}(t_{i-1})
      \right)} \]
It's easy to see that reapplying the $S(t_i)$ formula again and again
allows us to simulate sample paths under $Q_B$.

\paragraph{Note} While path independent derivatives likely allow
analytical solutions, computing solutions to PDDs analytically would
be computationally slow because of the dependence upon the value of the
underlying at several maturity dates.  Therefore, Monte Carlo
simulation is a much better alternative.

\newpage

\section{Forwards and Futures Prices}

Since forward and futures prices are \emph{not} assets prices, 
we cannot immediately apply FTAP and its resulting properties to them.
Instead, let's examine more closely how they are priced in a sufficiently
general setitng with few restrictions.

\subsection{Forward Prices}

We start by examining \emph{forward prices}, which are effectively
fixed delivery prices agreed upon at initiation of a forward contract. 
They differ from the \emph{value} of
a forward contract, which is the resale price or value of an 
already-initiated contract. While the forward price is 
somewhere in the ballpark of the current spot price, $S(t)$, the 
value of a forward contract will begin at 0 and fluctuate with changes
in the value of $S$ over time.

\subsubsection{No Dividend Case}

We will start with forward prices, denoed by $G(t,T)$.  We start with
the fact that the payoff at the delivery date $T$ of a forward contract
with delivery price $K$ will be
\begin{equation}
   \label{veetee}
    V(T) = S(T) - K.
\end{equation}
Next, it follows from FTAP that the \emph{value} of the forward 
contract---note, I'm not yet talking about the forward \emph{price} or
\emph{rate}---at
any time $t<T$ will be, under the $T$-forward measure,
\begin{equation}
   \label{fwd}
   V(t)  = P(t,T) \left( E_t^{Q_T}[S(T)] - K  \right) = P(t,T)
      E_t^{Q_T}[V(T)]
\end{equation}
This follows because the payoff $V(T)$ in Equation \ref{veetee} 
is a payoff at time $T$ that we can price exactly like a PID.
\\
\\
Next, since we want $V(t) = 0$, when we initiate a forward contract, 
this forces
   \[ G(t,T) = E_t^{Q_T}[S(T)] = K \]
where $G(t,T)$ is the delivery price set at time $t$ such for delivery
at time $T$. But, by the fact that $G(T,T) = S(T)$ for forward contracts
we have that
   \[ G(t,T) = E_t^{Q_T}[S(T)] = E_t^{Q_T}[G(T,T)] \]
   \[ \Rightarrow \quad E_t^{Q_T}[G(T,T)-G(t,T)] = 0 \]
In other words, the forward price for delivery at time $T$ is a 
\emph{martingale} under the $T$-forward measure $Q_T$. We can use
this result along with a few others,\footnote{Namely, that the
   discounted process for $S(t)$ must be a martingale and that
   $S(T) = G(T,T)$.} 
   to get the current forward price
   \[ G(t,T) = E_t^{Q_T}[G(T,T)] = E_t^{Q_T}[S(T)] 
      = E_t^{Q_T}\left[\frac{S(T)}{P(T,T)} \right] 
      = \frac{S(t)}{P(t,T)} \]
which agrees with the simple formula for forward pricing.
\\
\\
From Equation \ref{fwd} we also get that the value of a 
\emph{already-initiated} forward contract at time $t$ with 
delivery price $K$ equals
   \[ P(t,T)(G(t,T) - K).\]

\subsubsection{Incorporating Dividends}

Now let's consider an asset paying dividends \emph{continuously}
and at \emph{non-random} rate, $\delta$.
\\
\\
Let $V(t) = S(t) e^{\int^t_0 \delta(s) \; ds}$ be the value of a
portfolio that begins with one share of the underlying and
reinvests all dividends.  Also, let $G_V(t,T)$ denote the forward
price at time $t$ for the delivery of this just-defined portfolio
$V$ at time $T$. Since $V$ itself pays no dividends (they are all
reinvested), we set the forward price on this portfolio using the 
results from the last section:
\[ G_V(t,T) = \frac{V(t)}{P(t,T)} = 
   \frac{S(t) e^{\int^t_0 \delta(s)\; ds} }{P(t,T)} \]
Since $V(T) = S(T) e^{\int^T_0 \delta(s)\; ds}$, it's clear that the 
forward price for delivery of $S$ only at time $T$ must equal
\begin{equation}
   \label{gcocarry}
    G(t,T) = \frac{G_V(t,T)}{ e^{\int^T_0 \delta(s)\; ds}} =
      \frac{S(t)e^{\int^t_0 \delta(s)\; ds}}{P(t,T) 
      e^{\int^T_0 \delta(s)\; ds}}
      = \frac{S(t)e^{-\int^T_t \delta(s)\; ds}}{P(t,T)}
\end{equation}
And under the Black-Scholes assumptions of constant interest rates and
dividends, we get that
   \[ G(t,T) = S(t) e^{(r-\delta)(T-t)} \]
which we know as the cost \emph{cost-of-carry} formula.  Note that
we call Equation \ref{gcocarry} the \emph{generalized cost-of-carry}
formula.

\newpage

\section{Black Model}

We cause the martingale property of forward prices to obtain an 
option-pricing model that is far more general in its assumptions than
the Black-Scholes model. We call this new model the \emph{Black 
Model}. It links forward prices, $G$, to spot prices, $S$, by using
the fact that $G(t,T)$ will converge to $S(T)$ as $t\rightarrow T$.

\subsection{Assumptions}

First, we assume that the \emph{forward price} of the underlying
asset follows an Ito process 
\begin{equation}
   \label{blacksde}
    dG(t,T) = G(t,T) \mu(t) \; dt + G(t,T) \sigma \; dW(t).
\end{equation}
We also assume that there are no arbitrage opportunities or market
imperfections. 

\paragraph{Note} We make no restrictive assumptions about dividends
or interest rates.  That gives us much more general results. We do,
however, retain the constant proportional volatility specification.
BUT, the volatility term, $\sigma$, is the volatility of the 
\emph{forward price}, not the spot price volatility.


\subsection{Pricing PID's Under the Black Model}

Let's price the arbitrary PID's under the Black Model using 
Monte Carlo simulation, where the PID's have payoff
$V(T) = \varphi(S(T))$ at time $T$:\footnote{We use the $T$-forward 
measure, $Q_T$, as it is much easier to compute given stochastic
interest rates. This is in keeping with the more general framework
of the Black Model.}
\begin{equation}
   \label{blackvee}
   V(t) = P(t,T) E_t^{Q_T}[\varphi(S(T))] = 
   P(t,T) E_t^{Q_T}[\varphi(G(T,T))]
\end{equation}
In order to compute this expectation, we'll need to identify the 
stochastic process followed by the forward price under $Q_T$.
\\
\\
So apply Girsanov's Theorem to Equation \ref{blacksde}, we get a new
SDE under $Q_T$:
\begin{equation}
   \label{girsanovblack}
   dG(t,T) = G(t,T) \hat{\mu}(t) \; dt + G(t,T) \sigma \; d\hat{W}(t).
\end{equation}
And since we showed in the section about forwards that $G$ must be
a martingale under $Q_T$, it follows that in Equation 
\ref{girsanovblack} that $\hat{\mu}$ must equal 0. As a result, $G$ 
follows Geometric Brownian Motion without drift. More generally, 
we get
\begin{equation}
   \label{blackgbm}
   S(T) = G(t,T) e^{-\frac{1}{2} \sigma^2 (T-t) 
      + \sigma (\hat{W}(T) - \hat{W}(t)) }
\end{equation}
which we can easily simulate and plug into Equation \ref{blackvee}.

\subsection{Results and Computational Notes}

Given the form of $S(T)$ described in Equation \ref{blackgbm}, we 
have a few immediate results:
\begin{enumerate}
   \item{Conditional on the information at time $t$, 
      \[ \ln S(T) \sim N\left( \ln G(t,T) - \frac{1}{2} \sigma^2 (T-t), 
	 \;\; (\sigma \sqrt{T-t})^2 \right) \]
      under the measure $Q_T$. 
   }
   \item{We can compute the price of a generic PID from Equation 
      \ref{blackvee} after we simulate values of $S(T)$ using 
      Equation \ref{blackgbm}. 
   }
   \item{We can price options analytically under the Black Model
      by using the closed-form solution Black-Scholes model and
      making the following substitutions:
      \[ G(t,T) \rightarrow S(t), \qquad r(t,T) \rightarrow r,
	 \qquad r(t,T) \rightarrow \delta, \qquad \sigma_{spot}
	 \rightarrow \sigma_{fwd}\]
      }
\end{enumerate}
So we can price options with the following explicit formulas, where
   \[ y = \frac{ \ln\left(\frac{G(t,T)}{K}\right) - \frac{1}{2}
      \sigma^2 (T-t)}{\sigma \sqrt{T-t}} \]
and where $\sigma$ is the volatility of the forward price:
\begin{itemize}
   \item[-]{\emph{Cash-or-Nothing}: 
      \[c^{\text{CON}}  = P(t,T) \Phi(y)\]
      }
   \item[-]{\emph{All-or-Nothing}:
      \[c^{\text{AON}}  = P(t,T) G(t,T) \; \Phi(y + \sigma\sqrt{T-t})\]
      }
   \item[-]{\emph{Vanilla Call}:
      \[c(t)  = P(t,T) \left[ G(t,T) \; \Phi(y + \sigma\sqrt{T-t})
	 - K \Phi(y) \right] 
	 \]
      }
   \item[-]{\emph{Vanilla Put}:
      \[c(t)  = P(t,T) \left[ K \Phi(-y) - 
	 G(t,T) \; \Phi(-y - \sigma\sqrt{T-t}) \right] 
	 \]
      }  
\end{itemize}
The above formulas are called the \emph{Black Formulas}. They retain
the Black-Scholes Formulas as a special case under the additional
assumptions that the instantaneous interest rate is constant and that
the stock pays dividends continuously at a constant rate $\delta$.
But the Black Formulas are far more general and allow for stochastic
interest rates and stochastic or discrete dividends.\footnote{We can
   extend this framework to price PDD's as well, which is covered
in Notes 3.}

\newpage
\subsection{Pricing PIDs in the Black Model}

Pricing path-independent options is a bit less straightforward than
pricing PIDs in the Black model. This arises because the 
Black model simulates \emph{forward rates}, $G(t,T)$. Now, we know
that we can simulate forward rates easily by using
   \[ G(t_i, T) = G(t_{i-1}, T) e^{-\frac{1}{2}\sigma^2 (t_i - t_{i-1})
      +\sigma (\hat{w}(t_j) - \hat{w}(t_{j-1}))} \]
and iterating over the sampling times, $\{ t = t_0, t_1, \ldots,
t_n=T\}$.
\\
\\
The complication arises, however, if the derivative's payoff is a 
function of the \emph{stock price}, not the forward rate. Then,
you have two options.

\paragraph{Option 1} Simulate multiple forward prices with delivery
dates corresponding to each sampling date, which 
requires lots of calibrations for volatilities and correlation
coefficients, but requires no assumptions about interest
rates and dividends.

\paragraph{Option 2} Things simplify a bit, however, if you assume
continuous dividends along with deterministic evolution of 
the dividend and instantaneous interest rate. In that case, we can
invert the cost of carry formula to recover the stock price:
   \[ G(t,T) = S(t) e^{\int^T_t (r(s) - \delta(s))\; ds} \qquad
      \Leftrightarrow \qquad
      S(t)= G(t,T) e^{\int^T_t ( \delta(s)- r(s) )\; ds} \]
Note that assuming non-random interest rates means assuming
future interest rates equal current forward rates.



\newpage

\section{Put-Call Parity}

Now let's consider \emph{Put-Call Parity} in it's most general case,
namely where we don't restrict ourselves to constant interest or
dividend rates.
\\
\\
To start, we know that long a call and short a put---with identical
strike, $K$, and maturity, $T$---will put you
in a situation where you own the asset and owe the strike at time
$T$. Mathematically, at expiry time $T$, we will have a payoff of
\begin{equation}
   \label{startpcp}
   c(T)-p(T) = S(T) - K
\end{equation}
But, the RHS of Equation \ref{startpcp} is simply the payoff of a
forward contract, $G$.  So if we consider that same portfolio today 
at time $t$,
we know that the portfolio must be worth the same as the
the corresponding forward on the underlying. So we get
\begin{equation}
   \label{pcp}
   c(t) - p(t) = P(t,T) \left(G(t,T) - K\right).
\end{equation}
This is the most general form of Put-Call Parity, and it only 
requires the assumption of perfect markets.


\newpage

\section{Volatility Surfaces}

In the Black formulas, everything you need to value a simple or vanilla
option is likely to be observable---save the estimated volatility
parameter, $\sigma$.  The current forward price ($G$), the discount
factor ($P$), the strike ($K$), and the time to maturity ($T-t$) are
all there for you to use.  

As a result, there is a one-to-one
mapping between the only disputable parameter, $\sigma$, and the
option price. Consequently, by simply inferring the volatility from the
market prices of options, we can get at an options \emph{implied 
volatility} (IV), which we'll now discuss.
\\
\\
If we let $c^M$ and $p^M$ denote the market value of the call and
put, and if we let $c^B(\sigma)$ and $p^B(\sigma)$ denote the
call and put prices obtained from the Black formula, then the
\emph{implied volatility} of the call
price and the put price, respectively, is that 
$\sigma$ that solves
   \[ c^B(\sigma) = c^M, \qquad p^B(\sigma) = p^M  \]
First, we show that the IV for the put equals the IV for the
call:
   \[ c^B(\sigma) - p^B(\sigma) = P(t,T)(G(t,T) - K) \]
   \[ c^M(\sigma) - p^M(\sigma) = P(t,T)(G(t,T) - K) \]
where the first line follows from put-call parity, 
and the second line follows
if we assume no arbitrage.  So equating and rearranging, we get that
   \[ c^B(\sigma) - c^M = p^B(\sigma) - p^M,\]
which proves that IV of the call must equal that of the put to preclude 
arbitrage. 
\\
\\
Because of the one-to-one correspondence between IV and option 
prices, it's common for OTC dealers to quote the IV rather than the
price.  The IV can then be plugged into the Black Formula to get
the prices.\footnote{Note: That doesn't mean dealers are using the 
Black Model to price the options. In fact, they're probably using 
somethining more sophisticated. But this is a simple way to quote
options, and the convention is used nonetheless.}

\paragraph{Results} We see that while the market quotes Black-implied
volatilities, it does not \emph{price} according to the Black model.
This leads to non-constant volatilities across different strikes
and over time, with the most notable features being
\begin{itemize}
   \item[-] {\sl Volatility Smile}: Short term options give rise to
      a convex pattern across strikes.
   \item[-] {\sl Volatility Skew}: The smile gradually flattens
      out for longer term options, although low strike IVs are
      higher in general.
\end{itemize}

\newpage
\section{Local Volatility Models}

Local volatility models are one class of models which attempts to 
generate prices in line with
the market, which is equivalent to a model
whose prices have Black-implied volatilities that match the observed
volatility smile and skew.  To do so, this model relaxes that 
assumption that the volatility of the spot (or forward) price
is a constant that's proportional to the price level of the asset.
\\
\\
Ideally, we'd also like to incorporate some stylistic facts into
our model about regarding the behavior of volatility. For example,
we know that there is a strong negative correlation between
changes in the VIX and changes in the S\&P.  In practice, this means
generalizing the BS and Black models to allow the volatility
of the underlying asset to vary and be negatively correlated with price.














%%%%%%%%%%%%%%%%%%%%%%%%%%%%%%%%%%%%%%%%%%%%%%%%%%%%%%%%%%%%%%%%%
%%%%%%%%%%%%%%% APPENDICES %%%%%%%%%%%%%%%%%%%
\newpage 
\appendix






%%%%%%%%% APPENDIX %%%%%%%%%%%%%%%%%%%%%%%%%%
\newpage
\appendix

\section{Antithetic Variates: A Variance Reduction Technique}

The technique of \emph{antithetic variates} reduces the sample
size needed to achieve a given precision (in the form of standard error)
in Monte Carlo simulations. Let's go through the steps:
\begin{enumerate}
   \item{Suppose that we want to estimate $EY$ using MC simulation. 
      Instead of simulating a single sample of size $N$, we simulate
      $N/2$ pairs, $(Y_1, Y_2)_i$, where each pair is i.i.d. for
      all pairs $i=1,\ldots,N/2$.}
   \item{We can get an unbiased estimator of $EY$ by first
      computing the average value $\bar{Y}_i = 
      \frac{1}{2}(Y_{i1}+ Y_{i2})$
      then averaging across the $\bar{Y}_i$.} 
   \item{If we simulate the $N/2$ pairs independently, then 
      \begin{align*}
	 Var(\bar{Y}_i) &= Var\left(\frac{1}{2}(Y_{i1}+ Y_{i2})\right) =
	 \frac{1}{4}\left[Var(Y_{i1})+Var(Y_{i2})+2 Cov(Y_{i1}, Y_{i2})
	 \right]
	 \\
	 &=  \frac{(1+\rho)}{2} Var(Y), \quad \text{$\rho$ is the 
	    correlation between $Y_{i1}$, $Y_{i2}$}
      \end{align*}
      So then if we want the total variance of $Y$ over our entire 
      simulation, then we get 
	 \[ Var(\bar{Y}) = \frac{(1+\rho) Var(Y)}{ 2 (N/2)}
	    = \frac{(1+\rho) Var(Y)}{N} \]
      }
   \item{Clearly, by making the correlation negative between the 
      terms $Y_1$ and $Y_2$ in each given pair, we can lower the variance
      of our estimator for $EY$ as a whole.  (Note that the pairs
      themselves are still independent from pair to pair.)}
\end{enumerate}
Note that some of the negative correlation will wash out and we
won't get perfect -1 correlation when we jump from random draws to
prices, as typically the prices are some convoluted \emph{function}
of the random draws.
\\
\\
Also, the gain---while significant---for path \emph{dependent} 
derivatives is not quite as large due to the dependence of the payoffs
upon the entire path.

\newpage
\section{Extras from the Textbook}


This entire section aims to be a little more rigorous and specific
about the approaches to asset pricing.  Especially in the second
section, I hope to be a little more specific about what's going in
the background of the Fundamental Theorem of Asset Pricing.

\subsection{Replication Approach}

One way to price a derivative involves a replicating portfolios.
Suppose you know that one must exist (because, say, the market is 
complete), and you know the value of the derivative, $V(S(t),t)$, is a
relatively nicely function from the standpoint of Ito's Lemma.
\\
\\
First, you use the condition that the portfolio must be
self-financing:
\begin{equation}
   \label{selffi}
   \theta(t)^TS(t) - \theta(0)^TS(0) = \int^t_0 \theta(u)^T \; dS(u),
\end{equation}
which essentially says that a portfolio process, 
$\theta \in \mathbb{R}^d$, should be structured such that the total
incremental gains from trading over the time $(0,T)$ (the right-hand
side) should be equal to the total gains (the left-hand side). Note
that $S(t)$ without the subscript is also a vector for the value of the
$d$ assets.
\\
\\
Next, we suppose that the value of the derivative over time, $V(S(t),t)$,
is relatively well-behaved in the sense that we can apply Ito's Lemma
to it. So let's do that to get
\begin{align}
   \label{itopde}
   V(S(t),t) &= V(S(0),0) + 
   \sum^d_{i=1} \int^t_0 \frac{\partial V(S(u),u)}{\partial S_i} dS_i(u)
   \notag\\
   &+ \int_0^t \left[ \frac{\partial V(S(u),u)}{\partial u} 
   + \frac{1}{2} \sum^d_{i,j =1} S_i(u)S_j(u) \Sigma_{ij}(S(u),u)
   \frac{\partial^2 V(S(u),u)}{\partial S_i \partial S_j} \right] du
\end{align}
where $d$ is the number of assets and $\sigma_{ij}$ is the covariance
between instantaneous returns on assets $i$ and $j$.
\\
\\
So now let's adapt our self-financing consraint to the derivative.
We'll assume that the market is complete so a replicating a portfolio
does exist, allowing us to swap out some elements in Equation 
\ref{selffi}, rearrange, and get
\begin{equation}
   \label{newsf} 
   V(S(t),t) = V(S(0),0) + \sum_{i=1}^d \int_0^t \theta_i(u) \; dS_i(u)
\end{equation}
So now that we have two equations in terms of $V(S(t),t)$, we'll 
eventually want to equate them.  But let's just take note now that
in order for them to be equal, we'll need everything on the second
line of Equation \ref{itopde} to equal 0:
\begin{equation}
   \label{heat}
   \Rightarrow \frac{\partial V(S(u),u)}{\partial u} 
   + \frac{1}{2} \sum^d_{i,j =1} S_i(u)S_j(u) \Sigma_{ij}(S(u),u)
   \frac{\partial^2 V(S(u),u)}{\partial S_i \partial S_j} = 0 
\end{equation}
Next, we'll also need 
\begin{equation}
   \label{theta}
   \theta_i(u) = \frac{\partial V(S(u),u)}{\partial S_i},
      \qquad i = 1,\ldots, d
\end{equation}
But given the fact that $\theta(t)^T S(t)$ is the value of the 
replicating portfolio at time $t$
(and, as a consequence, the derivative as well), we can write
\begin{equation}
   \label{delta}
   V(S,t) = \sum^d_{i=1} \frac{\partial V(S,t)}{\partial S_i} S_i 
\end{equation}
Finally, note that we also have the boundary condition 
\begin{equation}
   \label{bound}
   V(S(T),T) = f(S(T)) 
\end{equation}
where $f$ is the payoff function for the derivative.
\\
\\
\textbf{Note}, 
that the drifts, $\mu_i(t)$ of the assets, $S_i(t)$, do not
appear anywhere in our foundational Equations \ref{delta} and 
\ref{bound}. They are already implicit in the stock prices, $S_i(t)$, but
they don't show up explicitly because our argument had \emph{nothing}
to say about risk preferences. We merely described a self-financing
portfolio that replicates the derivative, and $V(S(0),0)$ is simply
the cost to initiate that strategy---regardless of the investor's risk
preferences. 

\paragraph{Solving the PDE} Let's just clarify how everything relates
when trying to implement this approach:
\begin{enumerate}
   \item{Equation \ref{heat} is the PDE that we have to solve
      out, subject to the boundary condition.}
   \item{Equation \ref{bound} is that boundary condition.}
   \item{Equation \ref{delta} actually gives us the value of the
      derivative security.}
\end{enumerate}

\subsection{Risk-Neutral and Expectation-Type Pricing}

It may be that the price dynamics are very complex or that the 
derivative is path-dependent.  In such cases, a PDE may not be feasible
or even apply.  In such cases, Monte Carlo methods provide a valuable
and manageable alternative.
\\
\\
Suppose we have a risk free asset governed by
   \[ \frac{dN(t)}{N(t)} = r \; dt\qquad \Rightarrow\qquad
      N(t) = N(0)e^{rt} \]
Next, suppose that the market is arbitrage free, so it admits a 
stochastic discount factor, $Z(t)$.  Then since $N(t)$ is 
attainable, the process $N(t)/Z(t)$ will be a martingale. Note
that $N(t)$ clearly corresponds to the numeraire, $N$, that we 
defined earlier in FTAP. Moreover, the stochastic discount factor
$Z(t)$ was sort of swept under the table in the above definition
of FTAP, where I said ``there will be a measure.'' Here $Z(t)$ 
is integral to the converting between measures and translating
expectations under $P$ to expectations under $Q$ or $Q_N$.
\\
\\
Now, if we fix $N(0)$ at 1, and if we normalize the stochastic
discount factor so that $Z(0)=1$, then we see that $N(0)/Z(0)=1$.
Since we already said this is a Martingale, it has expectation
\[ E_P[N(0)/Z(0)]=1, \qquad \text{where $P$ is the real-world
   probability measure.} \]
Then this fact allows us to perform a change of measure to, say, $Q$.
Specifically, this gives us 
   \[ \left(\frac{dQ}{dP}\right)_t = \frac{N(t)}{Z(t)} \]
   \[\Rightarrow \;\; 
      Q(A) = E_P\left[1_A \cdot \left(\frac{dQ}{dP}\right)_t \right]
      = E_0 \left[1_A\frac{N(t)}{Z(t)} \right] \]
So now if we want to take an expectation under the new measure, as 
opposed to the real world one, we compute
   \[ E_Q[Y] = E_P\left[Y\frac{N(t)}{Z(t)}\right] \]
where $Q$ is called the \emph{risk-neutral measure}---a particular 
choice of \emph{equivalent martingale measure}. So now let's go 
back to the fact that with a arbitrage-free marget, there's
a stochastic discount factor---i.e. a positive process $Z(t)$ such
that every attainable price process is a martingale, which states
mathematically and implies
   \[ \frac{V(t)}{Z(t)} = E_P\left[\frac{V(T)}{Z(T)}\right] 
      \Rightarrow \;\; V(t) = E_P\left[V(T)\frac{Z(t)}{Z(T)}\right] \]
   \[ \Rightarrow \;\; V(0) = E_P\left[\frac{V(T)}{Z(T)}\right].\]
If we want to rewrite in terms of the new probability measure, $Q$,
we can say that
   \[V(t) = \beta(t) E_Q\left[\frac{V(T)}{N(T)}\right] \] 
and this should look precisely analogous to the statement of FTAP 
above, just allow $N$ to be a little more general then I treated
it in this section. So now it's clear that we're discounting
the terminal value under under the \emph{risk-free rate}, rather
than under that funky stochastic discount factor (which is likely
to be a bitch to work with and think about). 
\\
\\
Next, since we know that the normalized asset price processes, 
$S_i(t)/N(t)$ will be martingales under $Q$, 
we will have to specify the dynamics that make it so.
This will mean that we will need to find governing processes such that
the drift term is 0.










\end{document}

