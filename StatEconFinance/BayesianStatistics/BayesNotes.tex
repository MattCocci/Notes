\documentclass[12pt]{article}

\author{Matt Cocci}
\title{Notes for Bayesian Statistics}
\date{January 24, 2013}

%% Spacing %%%%%%%%%%%%%%%%%%%%%%%%%%%%%%%%%%%%%%%%%%%%%%%%

\usepackage{fullpage}
\usepackage{setspace}
%\onehalfspacing
\usepackage{microtype}


%% Header %%%%%%%%%%%%%%%%%%%%%%%%%%%%%%%%%%%%%%%%%%%%%%%%%

%\pagestyle{fancy} 
%\lhead{}
%\rhead{}
%\chead{}
%\setlength{\headheight}{15.2pt} 
    %---Make the header bigger to avoid overlap

%\renewcommand{\headrulewidth}{0.3pt} 
    %---Width of the line

%\setlength{\headsep}{0.2in}    
    %---Distance from line to text
            

%% Mathematics Related %%%%%%%%%%%%%%%%%%%%%%%%%%%%%%%%%%%

\usepackage{lmodern}
\usepackage{amsmath}
\usepackage{amsfonts}
\usepackage{amsthm} %allows for labeling of theorems
\newtheorem{thm}{Theorem}[section]
\newtheorem{lem}[thm]{Lemma}
\newtheorem{prop}[thm]{Proposition}
\newtheorem{cor}[thm]{Corollary}
%\numberwithin{equation}{section} 
    %---This labels the equations in relation to the sections 
    %---rather than other equations
%\numberwithin{equation}{subsection} 
    %---This labels relative to subsections


%% Font Choices %%%%%%%%%%%%%%%%%%%%%%%%%%%%%%%%%%%%%%%%%

\usepackage[T1]{fontenc}
\usepackage[utf8]{inputenc}
%\usepackage{blindtext}


%% Figures %%%%%%%%%%%%%%%%%%%%%%%%%%%%%%%%%%%%%%%%%%%%%%

\usepackage{graphicx}
\usepackage{subfigure} 
    %---For plotting multiple figures at once
%\graphicspath{ {Directory/} }
    %---Set a directory for where to look for figures


%% Hyperlinks %%%%%%%%%%%%%%%%%%%%%%%%%%%%%%%%%%%%%%%%%%%%
\usepackage{hyperref} 
\hypersetup{	
    colorlinks,		
        %---This colors the links themselves, not boxes
    citecolor=black,	
        %---Everything here and below changes link colors
    filecolor=black,
    linkcolor=black,
    urlcolor=black
}

%% Including Code %%%%%%%%%%%%%%%%%%%%%%%%%%%%%%%%%%%%%%% 

\usepackage{verbatim} 
    %---For including verbatim code from files, no colors

\usepackage{listings}
\usepackage{color}
\definecolor{mygreen}{RGB}{28,172,0}
\definecolor{mylilas}{RGB}{170,55,241}
\newcommand{\matlabcode}[1]{%
    \lstset{language=Matlab,%
        basicstyle=\footnotesize,%
        breaklines=true,%
        morekeywords={matlab2tikz},%
        keywordstyle=\color{blue},%
        morekeywords=[2]{1}, keywordstyle=[2]{\color{black}},%
        identifierstyle=\color{black},%
        stringstyle=\color{mylilas},%
        commentstyle=\color{mygreen},%
        showstringspaces=false,%
            %---Without this there will be a symbol in 
            %---the places where there is a space
        numbers=left,%
        numberstyle={\tiny \color{black}},% 
            %---Size of the numbers
        numbersep=9pt,% 
            %---Defines how far the numbers are from the text
        emph=[1]{for,end,break,switch,case},emphstyle=[1]\color{red},%
            %---Some words to emphasise
    }%
    \lstinputlisting{#1}
}
    %---For including Matlab code from .m file with colors,
    %---line numbering, etc. 


%% Misc %%%%%%%%%%%%%%%%%%%%%%%%%%%%%%%%%%%%%%%%%%%%%% 

\usepackage{enumitem} 
    %---Has to do with enumeration	
\usepackage{appendix}
%\usepackage{natbib} 
    %---For bibliographies
\usepackage{pdfpages}
    %---For including whole pdf pages as a page in doc


%% User Defined %%%%%%%%%%%%%%%%%%%%%%%%%%%%%%%%%%%%%%%%%% 

%\newcommand{\nameofcmd}{Text to display}



%%%%%%%%%%%%%%%%%%%%%%%%%%%%%%%%%%%%%%%%%%%%%%%%%%%%%%%%%%%%%%%%%%%%%%%% 
%% BODY %%%%%%%%%%%%%%%%%%%%%%%%%%%%%%%%%%%%%%%%%%%%%%%%%%%%%%%%%%%%%%%%
%%%%%%%%%%%%%%%%%%%%%%%%%%%%%%%%%%%%%%%%%%%%%%%%%%%%%%%%%%%%%%%%%%%%%%%% 


\begin{document}
\maketitle

\tableofcontents

\newpage
\section{Statistical Inference: From Classical to Bayesian}

\subsection{Statistical Inference}
In conducting \emph{Statistical Inference}, there are a few
main, recurring goals:
\begin{enumerate}
   \item{\emph{Point Estimation}: What is the ``best'' value
      for an unknown parameter?}
   \item{\emph{Variability Estimation}: Create an interval of
      likely values for some unknown parameter. Under the two
      main paradigms, this is called
      \begin{itemize}
	 \item[-]{Confidence Interval in classical statistics.}
	 \item[-]{Posterior Interval in Bayesian statistics.}
      \end{itemize}
   }
   \item{Figure out probabilities for specific events or
      test hypotheses.}
   \item{Predict out of sample events.}
\end{enumerate}

\subsection{Classical Statistics}

Typically, you start by assuming a \emph{statistical model} that links
observed data to a set of unknown parameters.  For example, you
can assume $Y_i \sim N(\mu,\sigma^2)$, i.i.d.
\\
\\
Next, you want to estimate a vector of parameters, $\mathbb{\theta}$,
which are assumed to be fixed, unknown values. Typically, you do this
by the \textbf{likelihood principle}, where you say that the probability
model is based on some density, $p(Y_i | \mathbf{\theta})$. Then, after
observing data, you compute
   \[ \text{Likelihood} = \prod^n_{i=1} p(y_i | \mathbf{\theta}) \]
Then from there, choose parameter values that make our data as 
likely as possible:
   \[ \hat{\mathbf{\theta}}_{MLE} = \arg \max_{\mathbb{\theta}} 
      p( \mathbf{y} | \mathbf{\theta}) \]
For simple models, this is often possible to compute analytically, but
for larger models, you'll need to resort to optimizing algorithms
like Newton-Raphson.
\\
\\
However, given that there is some uncertainty in terms of which
random sample (of many) that we observed, we would like to quantify
the uncertainty in our measurements.  To do so, we use the concept
of a \textbf{sampling distribution}, which describes the distribution
of values taken on by a parameter or statistic across \emph{all
possible} samples.  This is the basis for hypothesis testing and
confidence intervals.  In very large samples (which classical 
statistics is optimized for), this is a very big deal.
\\
\\
However, there are some disadvantages which we will want to discuss.
\begin{enumerate}
   \item{Assumes parameters are \emph{fixed constants}, 
      disallowing prob. statements about $\theta$.}
   \item{The interpretation of Confidence Intervals is rather unwieldy.
      It's not a probability statement, and only says that in 95\%
      of the many samples we could take, the CIs will contain $\mu$.}
   \item{Most of the variability results are \emph{asymptotic}, 
      implicitly requiring large samples.}
\end{enumerate}
   
\subsection{From Classical to Bayesian Paradigm}

The major feature of Bayesian Statistics is that unknown parameters,
$\theta$, do \emph{not} have fixed values. Instead, they are random 
variables with their own distribution: $p(\theta)$. This allows us
to make direct probability statements about $\theta$, although it
does complicate matters a bit.
\\
\\
As a consequence of how we now view $\theta$, point estimates will
factor much less heavily as we believe that $\theta$ has an entire
distribution of values.

\subsection{Summarizing the Bayesian Approach}

The main goal of the Bayesian approach is to make probability statements
about a parameter $\theta$ or a unobserved data $\tilde{\mathbf{y}}$ 
\emph{conditional} on the observed value of $\mathbf{y}$.\footnote{Note 
that $\mathbf{y}$ is a vector representing the observed
data, $(y_1, \ldots, y_n)$.} Typically, we write these statements
as $p(\theta| \mathbf{y})$ or  $p(\tilde{\mathbf{y}}| \mathbf{y})$.
\\
\\
So to make those probability statements, we must specify a model that
provides a \emph{joint probability distribution} for $\theta$ and
$\mathbf{y}$. However, we typically break this up as follows:
   \[ p(\theta, \textbf{y}) = p(\theta) p(\mathbf{y} | \theta) \]
where $p(\theta)$ is a model for the parameter called the \textbf{prior}
distribution, and we call $p(\mathbf{y} | \theta)$ the \textbf{likelihood
function}, which we'll describe in more detail.\footnote{The 
   likelihood function assumes that name when $\theta$ varies and
   $\textbf{y}$ is \emph{fixed}, so that $p(\mathbf{y} | \theta)$ is
   a particular function of $\theta$.  As a result, we typically use 
   this name \emph{after} we observe data.  However, 
   when $\mathbf{y}$ is not fixed, we can also call it the 
   \emph{sampling distribution}---for obvious reasons---as Gelman does.}
\\
\\
The \emph{likelihood function}, $p(\mathbf{y} | \theta)$ is viewed
as a function of $\theta$ for fixed $\mathbf{y}$---i.e. after you observe
the data.  This is precisely the means through which one \emph{updates}
an initial prior distribution. So having observed data, 
you then \emph{update} the model for $\theta$ using Bayes' Rule:
\begin{equation}
   \label{bayes}
    p(\theta | \mathbf{y}) = \frac{p(\mathbf{y} | \theta) \cdot
      p(\theta)}{p(\mathbf{y})} \propto
      p(\mathbf{y} | \theta) \cdot p(\theta).
\end{equation}
Note that we show the \emph{unnormalized} posterior density after
$\propto$ because $p(\mathbf{y})$ is simply a constant with respect
to $\theta$, so we will often drop it from our analysis
since we can just normalize the numerator by a constant so 
that it integrates to 1.
\\
\\
This value, $p(\mathbf{y})$ in the denominator, is often called 
the \emph{prior predictive distribution} or the \emph{marginal
likelihood}.  It's called the first
as it represents how we would predict $\mathbf{y}$ 
before we observe any data. Specifically, before we see 
$\mathbf{y}$ and update $\theta$ to arrive at a posterior,
we would average the likelihood of observing 
$\mathbf{y}$ \emph{given} $\theta$ over all
possible values of $\theta$ that we allow for in our prior.
We can see this in the formulation of $\mathbf{y}$:
\begin{equation}
    \label{marglik}
    p(\mathbf{y}) = \int p(y, \theta) \; d\theta = 
	\int p(y | \theta) p(\theta) \; d\theta
\end{equation}
It's second moniker,
the marginal likelihood, arises from this same averaging out
of $\theta$ to get a marginal distribution.
\\
\\
Finally, 
we return to the updated result, $p(\theta | \mathbf{y})$,
called the \textbf{posterior}, which we can state  even more
intuitively:
   \[ (\text{Posterior}) = 
      \frac{(\text{Likelihood})(\text{Prior})}{(\text{Marginal 
      Distribution})} \]
Note, if we have a lot of data, they will dominate the prior and
wash out many of the effects of a potentially bad prior. (But in 
that case, why not just use classical statistics in the first place?)

\subsection{Prediction}

Often, we will want to use our probability model to make statements
about out of sample observations, once we have observed some data
and updated our model.  For that reason, we look to the
\textbf{posterior predictive distribution}:
\begin{align}
   p(\tilde{\mathbf{y}} | \mathbf{y}) &= \int p(\tilde{\mathbf{y}}, 
       \theta | y) \; d\theta \label{pospred1}\\
   &= \int p(\tilde{\mathbf{y}} | \theta , y) p(\theta | \mathbf{y})
       \; d\theta \label{pospred2}\\
   &= \int p(\tilde{\mathbf{y}} | 
      \theta) p(\theta | \mathbf{y} ) \; d\theta 
	\label{pospred3}
\end{align}
Intuitively, Equation \ref{pospred1} says that our prediction of 
$\tilde{\mathbf{y}}$, which depends upon $\theta$, 
should \emph{average} over all possible 
values of $\theta$.
\\
\\
Equation \ref{pospred2} tells us how
to do that: consider the likelihood of observing new 
data $\tilde{\mathbf{y}}$ \emph{given} some assumed $\theta$, 
but weight
that likelihood by the probability that $\theta$ actually does
equal that assumed $\theta$.  What should the weight be? Well,
the posterior probability seems sensible, since it represents
the best guess of $\theta$'s distribution given previously
observed data.
\\
\\
Finally, we can jump from Equation \ref{pospred2} to 
\ref{pospred3} because $\mathbf{y}$
and $\tilde{\mathbf{y}}$ are conditionally independent given 
$\theta$.

\subsection{Likelihood and Odds Ratios}

We call the ratio of the posterior density, $p(\theta | \mathbf{y})$ 
evaluated at points $\theta_1$ and $\theta_2$ the \emph{posterior odds}
for $\theta_1$ relative to $\theta_2$. Note that the posterior
odds are equal to the prior odds multiplied by the likelihood 
ratio, $p( \mathbf{y} | \theta_1)/p( \mathbf{y} | \theta_2)$:
   \[ \frac{p(\theta_1 | \mathbf{y})}{p(\theta_2 | \mathbf{y})}
      = \frac{p(\theta_1) p(\mathbf{y} | \theta_1)}{p(\theta_2)
	 p(\mathbf{y} | \theta_2)} \]


\subsection{Advantages and Disadvantages of the Bayesian Approach}

Let's start with the nice results:
\begin{enumerate}
   \item{Variability of $\theta$ summarized by the posterior 
      distribution.}
   \item{We can make direct probability statements about $\theta$ now.}
   \item{Hypothesis testing is less important.  For example, instead
      of testing a hypothesis as to whether the treatment effect of a 
      drug is nonzero, we ask what is the \emph{probability} that
      the effect is greater than zero. Whole different ballgame.}
   \item{We are \emph{not} relying on asymptotic results to come
      to our conclusions.}
\end{enumerate}
Now let's be fair and go over some of the disadvantages:
\begin{enumerate}
   \item{We need to assume or create a prior distribution, leaving
      us to specify two distributions (the likelihood and the 
      prior) as opposed to just one (the likelihood).
   }
   \item{Having to specify a prior can be tough if there is really
      little data or you just have no clue a priori what might
      be a good specification for the prior.}
\end{enumerate}

\newpage

\section{Prior Distributions}

Prior selection is one of the most important topics in Bayesian inference
and much thought and care must be placed into choosing one. This section
hopes the topic in a general framework.


\subsection{Conjugate Priors}

The formal definition of conjugacy is as follows: If $\mathcal{S}$ is
a class of sampling distributions, or likelihoods, 
$p(\mathbf{y} | \theta)$, and $\mathcal{P}$ is a class of prior 
distributions, $p(\theta)$ for $\theta$, then the class
$\mathcal{P}$ is \emph{conjugate} for $\mathcal{S}$ if 
   \[ p(\theta | \mathbf{y}) \in \mathcal{P} 
      \text{ for all } p(\cdot | \theta) \in 
      \mathcal{S} \text{ and } p(\cdot) \in \mathcal{P}.\]
In words, a prior is conjugate if the posterior is of the same form
as the prior. Common examples include
\begin{center}
   \begin{tabular}{ | c | c |}
      \hline
      $\theta \sim$ & $\mathbf{y}|\theta \sim$ \\
      \hline
      Beta(a,b) & Binomial \\ \hline
      Gamma & Poisson \\ \hline
      Normal & Normal, unknown mean  \\\hline
      Gamma & Normal, unknown variance \\\hline
   \end{tabular}
\end{center}
Conjugate prior distributions often have nice interpretations, making
it easy to understand the posterior results, and making it look like
we're adding ``prior counts'' to get from the data to the posterior.

\subsection{Other Classes of Priors}

In addition to \emph{conjugate priors}, there are other principles or
forms of prior distributions we might adopt. Let's discuss some 
important ones along with a few general concepts.

\paragraph{Improper Prior}
In some instances, you can even have an \emph{improper prior} that 
generates a proper posterior nonetheless. Examples include the
Beta(0,0) distribution as a prior, which doesn't integrate to 1, but
effectively adds zero ``prior counts'' to the posterior, which will
integrate to 1 with one observation.

\paragraph{Non-Informative Priors}
Suppose we have virtually no idea about what the prior should look
like. Then we might want to construct a prior that is 
\emph{non-informative}, in the sense that it has as little influence
upon the posterior as possible.

\paragraph{Flat Priors} One way to have a non-informative prior involves
setting $p(\theta) \propto 1$, a constant. This is called a 
\emph{flat prior} with the consequence that the posterior is proportional
to the likelihood:
\begin{align*}
   p(\theta |\mathbf{y}) &\propto  p(\mathbf{y}| \theta ) \cdot 1 
   \\
   &\propto p(\mathbf{y}| \theta )  
\end{align*}
But this approach does lead to a few problems.
\begin{enumerate}
   \item{Namely, if the range
      is not bounded, then a flat prior will be an improper prior, 
      integrating to $\infty$ (ex. Normal distribution). 
      }
   \item{Also, this
      is a highly risky, as this approach may not lead to a proper 
      posterior. }
   \item{Finally, ``flatness'' may not be invariant to transformation.
      Specifically, if $p(\theta) \propto 1$ so that it is flat on the
      natural scale of $\theta$, it does not necessarily follow
      that it will be flat relative to transformations of $\theta$.
      This leads us to \emph{Jeffrey's Invariance Principle} and 
      \emph{Jeffrey's Prior}.\footnote{As an example, let's consider
	 the transformation $\phi = \ln \left(\frac{\theta}{1-\theta}
	 \right)$,
	 the so-called ``log odds.'' If we make our prior flat with
	 respect to $\phi$ flat, so that $p(\phi) \propto 1$, then
	 \[p(\theta) = p(\phi = g^{-1}(\theta)) \left\lvert \frac{
	    d\phi}{d\theta} \right\rvert 
	    = 1 \cdot \left\lvert \frac{
	    d\phi}{d\theta} \right\rvert = \frac{1}{\theta(1-\theta)}
	 \]
      }
      }
\end{enumerate}

\paragraph{Jeffrey's Prior} \emph{Jeffrey's Invariance Principle}
states that any rule for determining a non-informative prior
for $\theta$ should yield the same results if applied to transformations
of theta. Luckily, there does in fact exist such a prior, and it's 
called \emph{Jeffrey's Prior} and is formulated as follows
   \[ p(\theta) \propto \sqrt{J(\theta)}, \qquad
      J(\theta) = - E_\mathbf{y} 
      \left[ \frac{d^2(\ln p(\mathbf{y} | \theta))}{
      d\theta^2} \; \lvert \; \theta \right] \]
where $J(\theta)$ is commonly referred to as ``Expected Fisher 
Information.'' Note that the subscript $\mathbf{y}$ is placed on the 
expectation to indicate explicitly that this expectation is taken 
relative to the $\mathbf{y}$, not $\theta$.  
\\
\\
Once we have $p(\theta)$ from above, it's an added bonus if we can
write Jeffrey's Prior as a \emph{conjugate} prior as well.


\paragraph{Neutral Prior} The goal of a \emph{neutral prior} is to
craft a prior so that
   \[ P(\theta > \hat{\theta}_{\text{MLE}} | \mathbf{y}) \approx 1/2 \]
which tries to ensure that the posterior distribution is cented at the 
MLE.




\newpage

\section{Single Parameter Models}

In this section we will take a look at various probability distributions
and how we can apply the Bayesian approach to such problems.


\subsection{Binomial Data with Conjugate Beta Prior}

\subsubsection{Likelihood}

Consider the random variable $\mathbf{y}$, 
which is the number of successes in
$n$ trials $y_1, \ldots, y_n$, so that it is binomially distributed
with parameters $n,\theta$. This gives the sampling model (or 
likelihood)
   \[p(\mathbf{y} = y | \theta) = \binom{n}{y} \theta^y (1-\theta)^{n-y}
      \]

\subsubsection{Conjugate Prior and Special Cases}

For our prior distribution, let's suppose that $\theta$ has a beta
distribution
   \[ p(\theta) = \frac{\Gamma(a + b)}{\Gamma(a)\Gamma(b)}
      \theta^{a-1}(1-\theta)^{b-1}.\]
If we get specific and set $a=b=1$ so that our prior is the uniform
distribution,
we will obtain what's called the ``Wilson Estimate'' in our
posterior which has
many nice properites. In effect, his prior makes each point in the 
entire range of $\theta$, $[0,1]$, equally likely.
\\
\\
Also, we could set $a=b=0$ to get an improper, non-informative prior
which nonetheless gives a proper posterior.

\subsubsection{Posterior Distribution}

As a result, we get the posterior distribution
\begin{align*}
    p(\theta | \mathbf{y}) &\propto  p(Y | \theta) p(\theta) 
    \\
    &\propto \binom{n}{y} \theta^y (1-\theta)^{n-y}  \times 
      \frac{\Gamma(a + b)}{\Gamma(a)\Gamma(b)}
      \theta^{a-1}(1-\theta)^{b-1}
   \\
   &\propto  \theta^{y+a -1} (1-\theta)^{n-y+b-1}
\end{align*}
As a result, the posterior has a beta distribution, $\theta | \mathbf{y}
\sim \text{Beta}(y+a, \; n-y+b)$. From there, it's easy to compute
expectaitons and variances given the properties of the beta
distribution. For examply,
   \[ E[\theta | \mathbf{y} ] = \frac{y+a}{n+a+b},\]
which lends a nice intuitive interpreation. Namely, $a$ acts as prior 
counts and $a+b$ acts as prior total trials. Here, it's very easy
to see how our prior influences the analysis.


\subsubsection{Posterior Inference}

Say we want to compare the parameter for two different populations, $a$
and $b$. Assuming the two populations are independent, we can compute
\begin{align*}
    P(\theta_a, \theta_b | \mathbf{y}_a, \mathbf{y}_b) 
      &\propto P(\mathbf{y}_a, \mathbf{y}_b | \theta_a, \theta_b)
      P(\theta_a, \theta_b) 
   \\
   &\propto P(\mathbf{y}_a, \mathbf{y}_b | \theta_a, \theta_b)
      P(\theta_a) P(\theta_b)
   \\
   &\propto P(\mathbf{y}_a| \theta_a) P(\theta_a) 
      P(\mathbf{y}_b | \theta_b )
      P(\theta_b)
\end{align*}
To compare the two, we compute
   \[ P(\theta_a > \theta_b) \int \int_{\theta_a > \theta_b}
      P(\mathbf{y}_a, \mathbf{y}_b | \theta_a, \theta_b) \; d\theta_a
      \; d\theta_b \]
which is easy to do with simulation:
\begin{enumerate}
   \item{Draw $N$ values from $\theta_a | \mathbf{y}_a$.}
   \item{Draw $N$ values from $\theta_b | \mathbf{y}_b$.}
   \item{Pair up values just generated and tabulate the proportion
      of $\theta_a > \theta_b$.}
\end{enumerate}
And if we want to do prediction, we will need to compute
   \[ P(\tilde{\mathbf{y}}_a | \mathbf{y}_a) = \int 
      p(\tilde{\mathbf{y}}_a | \theta_a) p(\theta_a | \mathbf{y}_a)
      \; d\theta_a \]
which is to say, we find the probability of new results
$\tilde{\mathbf{y}}_a$ given the observed data, which boils down to
finding the probability of new results given parameter $\theta$
which is conditioned on the observed data. Again, this is something
we can do by simulation.

\newpage
\subsection{Poisson Data}

\subsubsection{Likelihood}

We'll model the likelihood for a vector of $n$ iid RV's as a
Poisson Random variable,
   \[ p(\mathbf{y} | \theta) \propto \theta^{\Sigma y_i} e^{-n\theta} \]
   
\subsubsection{Conjugate Prior and Special Cases}

Naturally, we would like a \emph{conjugate prior}, and the Gamma 
distribution
happens to be the perfect candidate. 
So $\theta \sim \text{Gamma}(\alpha,
\beta)$ and
   \[ p(\theta) \propto \theta^{\alpha - 1} e^{-\beta \theta}.\]
Next, we could consider \emph{Jeffrey's Prior}, 
which---after running through
the straightforward calculations---would give us
   \[ p(\theta) \propto \theta^{-1/2} \]
   \[ \Rightarrow \theta \sim \text{Gamma}(1/2, 0) \]
which is improper, but will give a proper posterior nonetheless.
\\
\\
Next, if we want to try a \emph{neutral prior}, then the neutral prior
is
   \[ \theta \sim \text{Gamma}(1/3, 0) \]


\subsubsection{Posterior Distribution}

This will give us a posterior distribution,
   \[ p(\theta | \mathbf{y}) \propto \theta^{ \Sigma y_i + \alpha
      - 1} e^{-(n+ \beta) \theta} \]
   \[ \Rightarrow \theta | \mathbf{y} \sim \text{Gamma}( \Sigma \; y_i
      + \alpha \; , \; n + \beta ) \]


\subsubsection{Extension: Rate, Exposure Modification} 
It may be convenient to extend
the Poisson model for data points $y_1, \ldots, y_n$ to the form
   \[ y_i \sim \text{Poisson}(x_i\theta) \]
where $\theta$ is the \emph{rate} and $x_i$ \emph{exposure} of the 
$i$th unit. Typically, the values $x_i$ are known constants, while
$\theta$ is some unknown parameter. If we set out our model to 
accomodate this extension, we get
\begin{align*}
   \theta &\sim \text{Gamma}(\alpha, \beta) \\
   p(y|\theta) &\propto \theta^{\Sigma y_i} e^{-(\Sigma x_i)\theta}\\
   \theta | y &\sim \text{Gamma}\left( \alpha + \Sigma y_i, \; 
      \beta + \Sigma x_i\right)
\end{align*}
As an example, suppose $\theta$ is the Poisson rate of plane crashes,
which is scaled by the $x_i$ representing miles flown in a given year.

\newpage

\subsection{Normal Data (Known Variance)}

\subsubsection{Likelihood}
Let's assume for our likelihood that the random variables, $y_i$
follow a normal distribution where
\begin{align*}
    p(y_i) = \frac{1}{\sigma \sqrt{2\pi}} e^{-\frac{1}{2\sigma^2}
    (y_i - \mu)^2} \qquad p(\mathbf{y} | \theta) &= \left(\frac{1}{  
       \sigma \sqrt{2\pi}}\right)^n \; \Pi \; e^{-\frac{1}{2\sigma^2}
       \Sigma (y_i - \mu)^2}
   \\
   &\propto \exp\left\{ -\frac{1}{2\sigma^2}
      \Sigma (y_i - \mu)^2 \right\}
\end{align*}
and $\sigma$ is known and fixed.  


\subsubsection{Conjugate Prior and Resulting Posterior}
Next, in choosing our prior for the parameter $\mu$, we will assume
that $\mu \sim N(\mu_0, \tau^2)$, and, again, $\sigma$ is constant. 
This implies a posterior distribution\footnote{
Clearly, as $n\rightarrow \infty$, we get the same result as the Central
Limit Theorem.}
\begin{align*}
   \mu \sim N(\mu_0, \tau^2) \quad \Rightarrow \quad
   \mu | \mathbf{y} \sim N\left( \frac{\frac{n}{\sigma^2} \bar{y}
   + \frac{1}{\tau^2} \mu_0}{\frac{n}{\sigma^2} + \frac{1}{\tau^2}}
   \; , \; \frac{1}{\frac{n}{\sigma^2} + \frac{1}{\tau^2}} \right)
\end{align*}
One way to view this is as the prior mean adjusted towards
the observed data, $\mathbf{y}$:
\[ \mu_{\text{pos}} = \alpha + (\bar{y} - \alpha) 
   \frac{\tau^2}{\frac{\sigma^2}{n} + \tau^2} \]
Now, if we want a particularly \emph{uninformative prior}, we can 
let $\tau \rightarrow \infty$, which will give us the Central Limit
Theorem. But note, our prior will be improper.

\subsubsection{Jeffrey's Prior}
Next, if we want \emph{Jeffrey's prior}, 
we can do the computeations and get
   \[ p(\mu) \propto 1 \]
or the flat prior over the entire real line, which gets us to the
Central Limit Theorem-type posterior
   \[ \mu | \mathbf{y} \sim N(\bar{y}, \sigma^2/n) \]


\newpage
\subsubsection{Posterior Predictive Distribution}
To get the \emph{predictive distribution}, we can integrate
\begin{equation}
   \label{op1}
    p(\tilde{\mathbf{y}} | \mathbf{y}) = \int^\infty_{-\infty}
      p(\tilde{\mathbf{y}} | \mu) p( \mu | \mathbf{y}) \; d\mu 
\end{equation}
Now that's a bit of a tough integration, so we'll often compute
that by simulation. But that doesn't stop us from summarizing the
posterior predictive distribution. Since we know that 
$p(\tilde{\mathbf{y}} | \mu)$ is normal, and since $p(\mu | \mathbf{y})$
is normal, their product will be too. So we need only find the 
appropriate mean and variance to say all we need to say.
\begin{enumerate}
   \item{Since the normal is conjugate to itself, we know that this 
      posterior predictive distribution \emph{will} be normal.}
   \item{So we can compute the expectation
      \begin{align*}
	 E[\tilde{\mathbf{y}} | \mathbf{y} ] &=  
	    E[ E(\tilde{\mathbf{y}} | \mu) |  \mathbf{y} ]  
	    = E[\mu | \mathbf{y}] 
	 \\
	 &= \frac{\frac{n}{\sigma^2} \bar{y}
	 + \frac{1}{\tau^2} \mu_0}{\frac{n}{\sigma^2} + \frac{1}{\tau^2}}
      \end{align*}
	 }
   \item{Next, we can compute the variance
      \begin{align*}
	 Var[\tilde{\mathbf{y}} | \mathbf{y} ] &=  
	    E[ Var(\tilde{\mathbf{y}} | \mu) |  \mathbf{y} ]  
	    + Var[ E(\tilde{\mathbf{y}} | \mu) | \mathbf{y} ]
	 \\
	 &= E[ \sigma^2 | \mathbf{y} ] + Var( \mu | \mathbf{y}) 
	 \\
	 &= \sigma^2 + \frac{1}{\frac{n}{\sigma^2} + \frac{1}{\tau^2}}
      \end{align*}
	 }
\end{enumerate}
We now have two alternatives. First, we can approximate the integral
in Expression \ref{op1} through simulation. Or, alternatively, we
can use conjugacy along with the mean and variance which we just derived
to sample straight from the normal distribution defined
\[ \tilde{y} | y\sim N\left(\frac{\frac{n}{\sigma^2} \bar{y}
      + \frac{1}{\tau^2} \mu_0}{\frac{n}{\sigma^2} + \frac{1}{\tau^2}},
       \; \sigma^2 + \frac{1}{\frac{n}{\sigma^2} + 
      \frac{1}{\tau^2}}\right) \]




\newpage
\subsection{Normal Data (Known Mean, Unknown Variance)}

Given a vector of length $n$, denoted $y$, of random draws from 
a normal distribution with unknown variance and known mean $\mu$, we
have a likelihood of
   \[ p(y|\sigma^2) \propto \sigma^{-n} \exp\left\{ 
      -\frac{1}{2\sigma^2} \sum^n_{i=1} (y_i - \mu)^2 \right\} \]
The \emph{conjugate prior} is an inverse Gamma distribution,
\[ \sigma^2 \sim \text{InvGamma}(\alpha, \beta) \quad
   \Leftrightarrow \quad 
   \frac{1}{\sigma^2} \sim \text{Gamma}(\alpha, \beta)
   \]
   \[ p(\sigma^2) = \frac{\beta}{\Gamma(\alpha)} \; 
      (\sigma^2)^{-(\alpha+1)} 
      e^{-\beta/\sigma^2} \]
where the distribution function comes froNon-Informative Priorm a simple 
transformation of variables.



\newpage
\section{Multiparameter Models}

\subsection{Univariate Normal Model}

Here we begin with one of the most useful models, full stop. We
will extend this in many directions after the initial specification
considered here.

\subsubsection{Likelihood}
In this model, $\theta$ is a vector of two unknown parameters,
$\theta=(\mu,\sigma)$. The likelihood is
\begin{align*}
   p(\mathbf{y}|\mu,\sigma) &= 
      \left(\frac{1}{\sigma\sqrt{2\pi}}\right)^2
      \exp\left\{ -\frac{1}{2\sigma^2} \sum (y_i-\mu)^2 \right\}\\
   &\propto (\sigma^2)^{-n/2} 
      \exp\left\{ -\frac{1}{2\sigma^2} \sum (y_i-\mu)^2 \right\}
\end{align*}

\subsubsection{Conjugate Prior and Resulting Posterior} 

We break the prior density up into two components:
\begin{align*}
   \mu|\sigma^2 &\sim N\left(\mu_0, \; \frac{\sigma^2}{\kappa_0}
   \right), \qquad \quad
   \sigma^2 \sim \text{InvGamma}\left(\frac{\nu_0}{2},
   \; \frac{\nu_0\sigma^2_0}{2}\right)
\end{align*}
Note that $\mu$ and $\sigma^2$ are \emph{a priori} dependent. A high
variance term, $\sigma^2$, will also induce a high-variance prior
distribution for $\mu$. But you need this dependence for conjugacy.
\\
\\
{\sl Joint Posterior Distribution}:
Now if we use this prior with the likelihood, the resulting non-standard
joint posterior distribution simplifies to
\begin{align*}
   p(\mu, \sigma^2 | \mathbf{y}) \propto (\sigma^2)^{- \left(\frac{
   n+\nu_0+1}{2} + 1\right)} \; \exp \left\{ -\frac{1}{2\sigma^2}
   \left( \sum (y_i-\mu)^2 + \kappa_0 (\mu - \mu_0)^2 + \nu_0\sigma_0^2
   \right) \right\}
\end{align*}
{\sl Marginal Posterior Densities}:
Often, we will be interested in the marginal densities. So let's find
the marginal posterior density of $\sigma^2$. With a lot of algebra, 
a nice identity,\footnote{Specifically, we'll use that $\Sigma 
   (y_i - \mu)^2 = \Sigma  (y_i - \bar{y})^2 + n(\bar{y}-\mu)^2$.}, 
and completing the square, we get:
\begin{align}
   \label{sigpos}
   p(\sigma^2|\mathbf{y})&=\int p(\mu, \sigma^2 | \mathbf{y}) \; d\mu
   \notag\\
   &= (\sigma^2)^{-\left(\frac{n+\nu_0}{2}+1\right)} \;
   \exp\left\{ -\frac{1}{2\sigma^2} \left( \left[\sum (y_i-\bar{y})^2
   \right] +\nu_0 \sigma^2_0 + \frac{nk_0}{n + k_0} (\bar{y}-\mu_0)^2
   \right) \right\}\notag\\
   \sigma^2|\mathbf{y} &\sim \text{InvGamma}\left(\frac{n+\nu_0}{2},
   \; \frac{1}{2} \left(\left[\sum (y_i-\bar{y})^2
   \right] +\nu_0 \sigma^2_0 + \frac{nk_0}{n + k_0} (\bar{y}-\mu_0)^2
   \right)\right) 
\end{align}
This has a nice clear interpretation, where the second parameter is a
combination of the sum of squares, $\nu_0\sigma^2$, and the discrepancy
between the data mean and the prior mean. 
\\
\\
Next, if we consider the
marginal posterior distribution for the mean, we get
\begin{align}
   \label{mupos}
   p(\mu | \sigma^2,\mathbf{y}) &\propto \exp\left\{ -\frac{1}{2\sigma^2}
   \left( \sum (y_i - \mu)^2 + \kappa_0 (\mu-\mu_0)^2\right)\right\}
   \notag\\
   \mu | \sigma^2,\mathbf{y} &\propto N\left( \frac{\frac{n}{\sigma^2}
   \bar{y} + \frac{\kappa_0}{\sigma^2}\mu_0}{\frac{n}{\sigma^2} +
   \frac{k_0}{\sigma^2}}, \; \frac{1}{\frac{n}{\sigma^2} +
   \frac{k_0}{\sigma^2}}\right)
\end{align}
To obtain samples from the joint posterior distribution, first sample
from $\sigma^2$ from the Equation \ref{sigpos} Inverse Gamma
distribution.\footnote{Alternatively, you could use the same parameters
from Equation \ref{sigpos} to draw from the Gamma distribution, and
then take the reciprocal to get a value for $\sigma^2$.}
Then, using this value for $\sigma^2$, sample from a normal distribution
with the parameters from Equation \ref{mupos}. Then repeat this
process many times to get samples of $\mu,\sigma|y$.

\subsubsection{Non-Informative Prior}

{\sl Non-Informative Prior}: To get this distribution, we can let
$\kappa_0, \nu_0\rightarrow 0$, using the notation from the conjugate
prior distribution described above. This gives us
   \[ p(\mu) \propto 1, \qquad p(\sigma^2) \propto (\sigma^2)^{-1} \]
Putting this together, you get a joint prior density of
   \[ p(\mu, \sigma^2) \propto \frac{1}{\sigma^2}, \quad\Leftrightarrow
      \quad p(\mu, \ln \sigma) \propto 1 \]
where the second expression indicates the prior is flat on the 
$\mu, \ln \sigma$ scale. 
\\
\\
{\sl Posterior}: As a result, our posterior distributions
will be\footnote{The distribution
      for $\mu|\sigma^2,\mathbf{y}$ follows from the fact that this
      is equivalent to the one-parameter normal case with known
      variance}
\begin{align*}
   \sigma^2 | \mathbf{y} &\sim \text{InvGamma}\left(\frac{n}{2}, \;
   \frac{1}{2} \sum (y_i - \bar{y})^2 \right)\\
   \mu | \sigma^2, \mathbf{y} &\sim N\left(\bar{y}, \frac{\sigma^2}{n}
      \right) \qquad \qquad  
\end{align*}
So we just factored our posterior density into the product of conditional
and marginal posterior densities: $p(\mu, \sigma^2 |y) = 
p(\mu|\sigma^2,y) p(\sigma^2|y)$.
So, to sample, just pull from the Inverse Gamma for $\sigma^2$, 
then use it to pull from the normal.
\\
\\
We already saw the marginal posterior density for $\sigma^2$ above. 
Now let's say we want the posterior \emph{marginal} density of $\mu$. We 
can integrate analytically to get
\begin{align*}
   p(\mu|\mathbf{y}) &= \int p(\mu,\sigma^2| \mathbf{y}) \; d\sigma^2\\
   &\propto \left[ 1 + \frac{n(\mu-\bar{y})^2}{(n-1)s^2} \right]^{-
      \frac{n}{2} }, \qquad s^2 = \frac{1}{n-1} \sum (y_i - \bar{y})^2
\end{align*}
which we recognize to be the good old t-distribution:
\[ \mu|\mathbf{y} \sim t_{n-1}\left(\bar{y}, \frac{s^2}{n}\right) \]



\newpage
\subsubsection{Semi-Conjugate Prior}

{\sl Prior Distribution}:
This particular prior allows for a priori independence between
$\mu$ and $\sigma^2$. To do so, take
\begin{align*}
   \mu &\sim N(\mu_0, \tau_0^2)\\
   \sigma^2 &\sim \text{InvGamma}\left(\frac{\nu_0}{2}, \; \frac{\nu_0
      \sigma_0^2}{2}\right)
\end{align*}
{\sl Posterior}: This will give us a joint posterior distribution of 
\begin{align*}
   p(\mu, \sigma^2 | y) &= (\sigma^2)^{-n/2} \exp\left\{
      -\frac{1}{2\sigma^2} \sum (y_i - \bar{y})^2 \right\}
      \exp\left\{ -\frac{1}{2\tau_0^2} (\mu-\mu_0)^2\right\} \\
      &\quad \times (\sigma^2)^{-\left(\frac{\nu_0}{2} +1\right)} 
      \exp\left\{ \frac{-\nu_0\sigma_0}{2\sigma^2}\right\}
\end{align*}
which doesn't really simplify any further to something nice or standard.
But, if we consider the conditional posterior density for $\mu$, we
do get something familiar:
   \[ \mu | \sigma^2, y \sim N\left(\frac{\frac{n}{\sigma^2} \bar{y}
     +\frac{1}{\tau_0^2}\mu_0}{\frac{n}{\sigma^2} + \frac{1}{\tau_0^2}},
       \; \frac{1}{\frac{n}{\sigma^2} + 
       \frac{1}{\tau_0^2}}\right) \]
But in order to use this, we need a value of $\sigma^2$ to condition on.
\\
\\
\emph{Marginal Posterior Density for $\sigma^2$}: 
Next, we can integrate out $\mu$ from the joint posterior. This yields
\begin{align*}
   p(\sigma^2 | y)&\propto \left(\frac{n}{\sigma^2} + \frac{1}{\tau_0^2}
   \right)^{1/2}  (\sigma^2)^{-\left(\frac{n+\nu_0}{2}+1\right)}
    \exp\left\{
      -\frac{1}{2\sigma^2} \sum (y_i - \mu_0)^2 \right\}\\
     &\quad \times\exp\left\{ -\frac{1}{2\tau_0^2} (\mu-\mu_0)^2\right\}
      \exp\left\{ \frac{-\nu_0\sigma_0}{2\sigma^2}\right\}
\end{align*}


\newpage
\subsection{Multivariate Normal Model}

Suppose $y_i$ is a vector of length $d$ with multivariate normal
distribution, 
   \[ y_i | \mu, \Sigma \sim N(\mu, \Sigma) \]
where $\mu$ is a column vector of length $d$ and $\Sigma$ is a
$d\times d$ matrix.  Then the likelihood function is 
   \[ p(y_i |\mu,\Sigma) \propto |\Sigma|^{-\frac{1}{2}} 
   \exp \left( -\frac{1}{2}(y_i-\mu)^T \Sigma^{-1} (y_i-\mu)\right)\]
Now suppose that we have $n$ such random variables. Then the joint
likelihood is written
   \[ p(y_1, \ldots, y_n | \mu, \Sigma) \propto 
      |\Sigma|^{-\frac{n}{2}}
      \exp \left( -\frac{1}{2} \sum_{i=1}^n
      (y_i-\mu)^T \Sigma^{-1} (y_i-\mu)\right)\]






\newpage
\section{Bayesian Regression}

In this section, I'll let $y$ be either a single observation or an
$n\times1$ vector, which will be clear from the context. To help 
establish that context, I'll let $x_i$ denote covariates for a 
\emph{single} observation, while $X$ will denote a \emph{matrix}
of covariates with dimension $n\times p$ for $n$ observations and
$p$ covariates.


\subsection{Likelihood for Simple and Multiple Regression}

In the simplest case, we will extend the simple $N(\mu,\sigma^2)$ model
to allow for conditional means, as in simple regression. 
Therefore, our model for the $i$th observation will be
\[ y_i = \beta x_i + \varepsilon_i, \qquad 
   \varepsilon_i \sim N(0,\sigma^2) \]
where $x_i$ is a vector of $p$ known covariates, and $\beta$ is a 
$p$-dimensional coefficient vector. Typically, we'll want to estimate
$\beta$ and $\sigma^2$. In our notation, we can even be a bit broader
and write
\begin{align*}
    y &= X \beta  + \varepsilon, \qquad  \varepsilon \sim 
      \text{N}(0, \sigma^2 I_n) \\  
      y| \beta, X, \sigma^2 &\sim \text{N}(X\beta, \sigma^2 I_n) 
\end{align*}
where $y$ is an $n\times1$ vector of observations and
$X$ is an $n\times p$ matrix for $n$ observations and $p$
covariates. This setup also implies errors that are 
independent and have equal variance. 


\subsection{Flat Prior}

In the case of the flat, non-informative prior, we recall the 
multiparameter normal case and set
   \[ p(\beta, \sigma^2) \propto (\sigma^2)^{-1}.\]
Now, we'll just recycle and adapt everything we did above, 
in the multiparameter normal case. This will give us a 
joint posterior distribution of
\begin{align}
   p(\beta, \sigma^2 | y) &\propto 
   p(y | \beta, \sigma^2 ) p(\beta, \sigma^2) \notag\\
   &\propto (\sigma^2)^{-1} (\sigma^2)^{-\frac{n}{2}}
      \exp\left\{ \frac{1}{2\sigma^2} 
   (y - X\beta)^T (y-X\beta)\right\} \notag\\
   &\propto (\sigma^2)^{-\left(\frac{n+2}{2}\right)} 
      \exp\left\{ y^T y-2 y^T X\beta+\beta^T X^T X \beta\right\} \notag
\end{align}
Now, let's say we want to compute the conditional posterior density
for $\beta$.  In that case, any unnecessary
$\sigma$ or $y$ terms can be considered constants
and wash out. This allows us to rewrite the last line, make 
substitutions, and complete the square
to get\footnote{The substitution $\hat{\beta} =
(X'X)^{-1} X y$ is the standard MLE estimate for $\beta$ in 
regression.}
\begin{align}
   p(\beta | \sigma^2, y) &\propto \exp\left\{ \beta^T X^T X \beta -
      2y^T X\beta\right\} \notag\\
   \begin{tabular}{l} $V_\beta = (X^T X)^{-1}$ \\ $\hat{\beta} =
      (X^T X)^{-1} X^T y$ \end{tabular} \qquad\Rightarrow \qquad  
      &\propto 
      \exp\left\{ -\frac{1}{2\sigma^2} (\beta^T V_\beta \beta  - 
      2\hat{\beta}^T V_\beta \beta) \right\} \notag\\
   &\propto \exp\left\{ -\frac{1}{2\sigma^2} (\beta - \hat{\beta})
   V_\beta^{-1} (\beta - \hat{\beta}) \right\}\notag
\end{align}
which happens to be the density of a multivariate normal distribution.
This means 
\[ \beta | \sigma^2, y, X \sim 
   \text{MVN}_p(\hat{\beta},\;\sigma^2 V_\beta)
   \]
Now we have to handle the conditional posterior density of $\sigma^2$:
\begin{align*}
   p(\sigma^2 | y, X) &= \frac{ p(\beta, \sigma^2 | y)}{p(\beta |
      \sigma^2, y)} \\
   &\propto \frac{ (\sigma^2)^{-\left(\frac{n+2}{2}\right)} 
   \exp\left\{ -\frac{1}{2\sigma^2} (y-X\beta)^T(y - X\beta)\right\}}{
	 (\det|\sigma^2 V_\beta|)^{-\frac{1}{2}} 
	 \exp\left\{ -\frac{1}{2\sigma^2} (\beta - \hat{\beta})^T
	 \; V^{-1}_\beta\; (\beta - \hat{\beta})\right\}
      }
\end{align*}
Since $\beta$ appears nowhere in the LHS, we can plug
in anything, like, say, $\beta = \hat{\beta}$, and is guaranteed to
cancel. This simplifies to
\begin{align*}
    p(\sigma^2 | y, X) &\propto (\sigma^2)^{-\left(\frac{n-p}{2}+ 1
      \right)} \exp\left\{ -\frac{1}{2\sigma^2} \sum (y_i - x_i \beta)^2
      \right\} \\
    \Rightarrow \quad \sigma^2 | y,X &\sim \text{InvGamma}\left(
      \frac{n-p}{2}, \; \frac{1}{2} \sum^n_{i=1} (y_i - x_i\beta)^2 
      \right) 
\end{align*}
Like the multiparameter normal with the non-informative prior, the
marginal posterior density is easy enough to get through integration
\begin{align*}
   p(\beta | y, X) &= \int^\infty_0 p(\beta, \sigma^2 |y, X)\; 
      d\sigma^2 \\
   &\propto \left[ 1+ \frac{1}{\sum (y_i - x_i\beta)} 
      (\beta - \hat{\beta})' \; V_\beta^{-1} \;(\beta - \hat{\beta})
      \right]^{-\frac{n}{2}}
\end{align*}
which is a \emph{scaled multivariate t-distribution}.
\\
\\
Next, we can ask about the \emph{posterior predictive distribution}
which is simple enough to integrate analytically and get:
\begin{align*}
   p(\tilde{y} | \tilde{X}, y, X) &= \int \int p(\tilde{y} | \tilde{X},
      \beta, \sigma^2 ) p(\beta | \sigma^2, y, X) p(\sigma^2|y, X)
      \; d\beta \; d\sigma^2 \\
      \Rightarrow \quad \tilde{y} | \tilde{X}, y, X &\sim 
      \text{MVT}_{n-p}(\tilde{X} \hat{\beta}, \;
      S^2( I + \tilde{X} V_\beta \tilde{X}'))
\end{align*}

\newpage
\subsection{Informative Prior with iid $\beta_j \in \beta$}

Now let's say we want to build some more information into our coefficient
vector, $\beta$. Specifically, we can model each component of the
beta vector, $\beta_j \in \beta$, to be iid and distributed
   \[ \beta_j \sim N(\beta_{0j}, \tau^2). \]
As before, we'll take a flat, non-informative prior for the variance of 
the error terms:
   \[ p(\sigma^2) \propto (\sigma^2)^{-1} \]
where $\beta$ and $\sigma^2$ are independent.
\\
\\
This gives a posterior conditional distribution for $\beta$ of 
\begin{align*}
   p(\beta | y, X, \sigma^2) &\propto p(y|\beta, \sigma^2, X) 
      p(\beta, \sigma^2) \\
   &\propto p(y|\beta, \sigma^2, X)  p(\beta) p(\sigma^2) \\
   &\propto \exp\left\{ -\frac{1}{2} (y^* - X^*\beta) \; \Sigma^{-1}
      (y^* - X^*\beta) \right\} \\
   y^* = \begin{pmatrix} Y \\ \beta_0 \end{pmatrix}, \qquad 
   X^* &= \begin{pmatrix} X \\ I_p \end{pmatrix}, \qquad 
   \Sigma = \begin{pmatrix} \sigma^2 I_n & 0 \\
      0 & \tau^2 I_p \end{pmatrix}  
\end{align*}
This implies for the conditional posterior distribution of $\beta$
that
   \[ \beta|y, X, \sigma^2 \sim \text{MVN}_p(\hat{\beta}, V_\beta) \]
   \[ \hat{\beta} = (X^{*'} \Sigma X^*)^{-1} X^{*'} \Sigma^{-1} y^*,
      \qquad V_\beta = (X^{*'} \Sigma^{-1} X^*)^{-1} \]

\subsection{Extensions}

Above, we took a very simple approach to the error terms in the 
likelihood, which gave us independent errors with equal variance.
But we can generalize this a bit and allow correlation between error
terms, and, thus, observed $y$ by specifying instead:
   \[ y \sim \text{MVN}( X\beta,\; \sigma^2 I_n)\quad \Rightarrow \quad
      y \sim \text{MVN}_n( X\beta, \;\Sigma_y).\]
Similarly, we could extend our prior for $\beta$, where we assumed that
all of the components $\beta_j \in \beta$ were iid. 
Instead, we could assume a distribution for the entire coefficient 
vector
   \[ \beta_j \sim N(\beta_{0j}, \tau^2) \quad \Rightarrow \quad
      \beta \sim \text{MVN}_p(\beta_0, \; \Sigma_\beta) \]
In both of these cases, however, things become substantially more
complicated because we have to pre-specify covariance matrices, 
$\Sigma_y$ and $\Sigma_\beta$. In practice, this is pretty tough
to do a priori, 
unless we have a look at the data before we estimate parameters.

\newpage
\subsection{Ridge Regression}

In classical regression, to estimate $\hat{\beta}_{\text{MLE}}$,
we will solve
\[ \hat{\beta}_{\text{MLE}} = \arg \max_{\beta} p(y | \beta, X), 
   \quad \Leftrightarrow \quad \hat{\beta}_{\text{MLE}} = 
   \arg \min_{\beta} \sum^n_{i=1} (y_i - X_i \beta)^2 
   \]
However, in Bayesian regression, the problem becomes
\[ \hat{\beta}= \arg \max_{\beta} p(y | \beta, X)\;p(\beta),
   \quad \Leftrightarrow \quad \hat{\beta} = 
   \arg \max_{\beta} \ln p(y | \beta, X) + \ln p(\beta)
   \]
\[ \Rightarrow \hat{\beta} = \arg \min_\beta \sum^n_{i=1} 
   (y_i - X_i \beta)^2 - \ln p(\beta) \]
where the log of the prior acts as a penalty term. And there just so happen to be a number of interesting penalty.

\paragraph{Example} {\sl Ridge Regression (L2 Regression)}:
Now let's consider the case where 
   \[ \beta \sim \text{MVN}_p(\beta_0, \; \tau^2 I_p).\]
In models with many parameters ($p$ large), it's common to set a
prior where the coefficient vector is assumed to be the zero vector,
so that our optimization problem becomes
\begin{align*}
   \ln p(\beta) &= K - \frac{1}{2\tau^2} \sum^p_{j=1} \beta_j^2\\
   \hat{\beta} &= \arg \min_\beta \sum^n_{i=1} 
      (y_i - X_i \beta)^2 + \lambda  \sum^p_{j=1} \beta_j^2, \qquad
      \lambda = \frac{1}{2\tau^2}
\end{align*}
Now it's clear to see how the $\ln p(\beta)$ acts as a penalty term.
Large values of $\beta_j$ are penalized more in the minimization,
leading the $\beta_j$ estimates to be shrunk back towards 0. 
\\
\\
But we can make the penalty even more extreme by changing the 
$\lambda$ term, which brings us to the next section.

\subsection{Optimization for Non-Conjugate Priors}

{\sl Laplace Prior and Lasso (or L2) Regression}: We set a prior
as follows leading to a corresponding minimization problem:
\[ p(\beta_j) \propto e^{-\lambda|\beta_j|} \]
\[ \hat{\beta} = \arg \min_\beta \sum^n_{i=1} 
   (y_i - X_i \beta)^2 + \lambda  \sum^p_{j=1} |\beta_j| \]
The interesting thing about the Laplace-based model is that many
$\hat{\beta}_j$ will be set \emph{identically} to 0 because of the
cusp that occurs in the prior density.
\\
\\
{\sl More General Priors and Bridge Regression}: We can be even
more general and set
   \[ p(\beta) \propto \exp\{ -\lambda \mathcal{J}(\beta) \} \]
   \[ \Rightarrow \hat{\beta} = \arg \min_\beta \sum^n_{i=1} 
      (y_i - X_i \beta)^2 + \lambda  \mathcal{J}(\beta) \]
Often, it's very common to restrict $\mathcal{J}(\beta)$ to
\[ \mathcal{J}(\beta) = \sum^p_{j=1} |\beta_j|^q \]
which retains as special cases Lasso regression ($q=1$) and 
Ridge regression ($q=2$).




\newpage
\section{Mixture Models}

Let's first discuss the idea of Mixture Models. If we set up a mixture
model, then we assume that any given observation, $y_i \in y$, can
come from one of $k$ standard distributions. Each observation
also has a certain probability of being pulled from any one of 
the $k$ distributions. So to begin and nail down the intuition, 
let's begin with the simple two-parameter case.

\subsection{Likelihood as a Mixture of Two Models}

We assume that an observation can come from one of two distributions,
$A$ or $B$:
\[ y_i\sim\begin{cases}A(\theta_A)&\text{with probability $\alpha$} \\
      B(\theta_B) & \text{with probability $1-\alpha$} \end{cases} \]
Let's assume that the probability density functions for $A$ and $B$
are denoted
   \[ \phi_A(y_i), \qquad \phi_B(y_i) \]
Then we can write the likelihood of a single observation and then
of the data
\begin{align*}
   p(y_i | \theta_A, \theta_B) &= \alpha \phi_A(y_i) + (1-\alpha)
      \phi_B(y_i) \\
   p(y_i | \theta_A, \theta_B) &= \prod^n_{i=1}
      \left[ \alpha \phi_A(y_i) + (1-\alpha)
	 \phi_B(y_i) \right]
\end{align*}
Now clearly, this is a totally non-standard distribution (most-likely).
And on top of that, it's going to be really tough to estimate 
parameters. So to do that, we'll typically employ the technique
of the EM algorithm to estimate parameters.

\subsection{Expectation Maximization (EM) Algorithm}

This method employs ``missing data'' in the sense of data points
or characteristics of the data $y$ that, \emph{if} we had it, it
would make our parameter estimation much easier. One example, 
let's say we had an indicator variable
\[ I_i = \begin{cases} 1 & \text{if $y_i$ is in group $A$} \\
      0 & \text{if $y_i$ is in group $B$} \end{cases} \]
      {\sl Complete Data Likelihood Step}:
From there, we can write out the \emph{complete data likelihood}
as a function of the observed data, the missing data, and the unknown
parameters:
\begin{align*}
    p(y, I | \theta_A, \theta_B) &= \prod^n_{i=1}
      \left[  \phi_A(y_i)\right]^{I_i} 
      \left[\phi_B(y_i) \right]^{1-I_i} \alpha^{I_i}(1-\alpha)^{1-I_i}
      \\
   \ell(y,I|\theta_A, \theta_B) &= \sum^n_{i=1} (I_i) \ln \phi_A(y_i)
   + \sum^n_{i=1} (1-I_i) \ln \phi_B(y_i)  \\
   &\qquad +
   \sum^n_{i=1} I_i \ln \alpha +\sum^n_{i=1} (1-I_i) \ln (1-\alpha)
\end{align*}
{\sl Expectation Step}: 
Now it would be great if we could optimize the last log-likelihood
on the last line, bu we don't \emph{actually} have the values for
$I$. Since we can't plug in actual values for
each $I_i$, we can at least plug in the expected values \emph{given}
the observed data and the current values of the parameters.
\begin{align*}
   \hat{I}_i = E[I_i | y, \theta] &= P(I_i=1 | y, \theta)\\
   &= \frac{ P(y_i | I_i = 1, \theta) P(I_i=1 | \theta)}{
      p(y_i|I_i = 1, \theta) P(I_i=1 |\theta) + 
      P(y_i|I_i=0, \theta) P(I=0|\theta)}
\end{align*}
{\sl Maximization Step}: Next, we plug $\hat{I}_i$ into the complete
data likelihood and find the optimal estimates, $\hat{\theta}$,
for all of our parameters, $\theta$. We do that by plugging into
the log-likelihood, taking the partial derivative with respect to all
the parameters, setting each equation equal to zero, and solving out.
\\
\\
Having calculated the new parameter estimates, calculate the new
$\hat{I}_i$ and iterate until convergence.


\newpage 
\section{Hierarchical Models}

\subsection{General Form}
{\sl Prior and Posterior}: 
Hierarchical Models introduce difference
levels of analysis.  A common example is to have some likelihood
$p(y|\theta)$ for observing the data, but have the parameter $\theta$
come from its own distribution, $p(\theta |\phi)$, where $\phi$.  
Often, we'll take advantage of 
conditional independence in such models to simplify 
posterior analysis. For example:
\begin{align}
   \label{heirarch}
   p(\theta,\phi|y) &\propto p(y| \theta, \phi)\cdot p(\theta, \phi) 
      \notag\\
   &\propto p(y|\theta) \cdot p(\theta|\phi)\cdot p(\phi)
\end{align}
We see above that the prior distribution, which now must be specified
for both $\theta$ and $\phi$, was broken apart into
$p(\theta, \phi) \propto p(\theta|\phi) \cdot p(\phi)$.
\\
\\
{\sl Conditional and Marginal Distributions}: We saw above
in Equation \ref{heirarch} how to write the joint posterior.
Typically, we will break this up into a marginal and conditional
posterior:
\[ p(\theta, \phi|y) \propto p(\theta | \phi, y)\cdot p(\phi | y) \]
This method instructs us how to simulate from the posterior: 
draw $\phi$ from it's marginal posterior distribution ($p(\phi|y)$), 
then sample $\theta$ from it's marginal conditional distribution
($p(\theta | \phi, y)$).
\\
\\
Yet while it's often easy to get $p(\theta | \phi, y)$, we'll have to 
do a bit more work to get $p(\phi |y)$, whether that's brute force
integration or algebraically:
\[ p(\phi|y) \propto \int p(\theta, \phi | y) \; d\theta, \qquad
   \text{or} \qquad p(\phi | y) = \frac{p(\theta, \phi | y)}{
   p(\theta | \phi, y)}\]
The second method works well for some nicely behaved distributions (but
not always).
\\
\\
\paragraph{Note} When possible, avoid assigning uniform prior
distributions
to the logarithm of the standard deviation (or approximate standard
deviation) of the distribution on
$\phi$. That is, 
\[ \text{Don't set} \quad p(\ln \sigma_\phi) \propto 1 \]
It can lead to an improper posterior density on $\phi$---that is,  
$p(\phi | y) \propto \infty$. Rather, set a uniform prior on the
standard deviation parameter itself.

\newpage
\subsection{Normal Model}

For first model, we assume that there are $J$ groups, each with their
own mean ($\mu_j$) but shared variance, where the $\mu_j$ are drawn
from a common distribution
\begin{align*}
   y_{ij} &\sim N(\mu_j, \sigma^2), \qquad i = 1, \ldots, n_j \\
   \mu_j &\sim N(\mu_0, \tau^2), \qquad j = 1, \ldots, J 
\end{align*}
We can think of $\sigma^2$---which we assume to be known---as the 
\emph{within group} variance, while $\tau^2$ is \emph{between group},
which controls between-group sharing of information
and the amount of shrinkage back to $\mu_0$.
\\
\\
{\sl General Form of the Posterior}: Let's examine how conditional
independent allows us to simplify the posterior, keeping in mind
that $\mu$ is a vector of the individual $\mu_j$:
\begin{align*}
   p(\mu, \mu_0, \tau^2 | y) &\propto p(y | \mu, \mu_0, \tau^2 )\cdot 
      p( \mu, \mu_0, \tau^2 ) \\
   &\propto p(y | \mu ) \cdot p( \mu | \mu_0, \tau^2 ) \cdot
      p( \mu_0, \tau^2 )\\
\end{align*}
{\sl Posterior Given Non-Informative Prior}: Suppose we choose
$p(\mu_0) \propto 1$ and $p(\tau^2) \propto 1$. Then we get for our
posterior 
\begin{align*}
   p(\mu,\mu_0,\tau^2 | y) &\propto \prod^J_{j=1} \prod^{n_j}_{i=1}   
      (\sigma^2)^{-\frac{1}{2}} \exp \left\{ -\frac{1}{2\sigma^2}
      (y_{ij} - \mu_j)^2\right\}  \\ 
      &\qquad \times \prod^J_{j=1} 
      (\tau^2)^{-\frac{1}{2}} \exp\left\{-\frac{1}{2\tau^2}(\mu_j -
      \mu_0)^2\right\} \\
\end{align*}
{\sl Breaking Up the Posterior}: 
We can analyze the conditional distributions of the parameters and
get standard distributions. In fact, simple calculations will show that
\begin{align*}
   \mu_j | \mu_0, \tau^2, y &\sim N\left(\frac{ 
      \frac{n_j}{\sigma^2} \bar{y}_j 
      + \frac{1}{\tau^2}\mu_0}{\frac{n_j}{\sigma^2} + \frac{1}{\tau^2}},
      \; \frac{1}{ \frac{n_j}{\sigma^2} + \frac{1}{\tau^2}} \right), 
      \\
   \mu_0 | \tau^2, y &\sim N\left( \hat{\mu}, 
      \;V_\mu\right),\\
   \tau^2 | y &\sim V^{1/2}_\mu \prod^J_{j=1} \left( 
   \frac{\sigma^2}{n_j} + \tau^2\right)^{1/2} 
   \exp \left\{ -\frac{(\bar{y}_j 
   -\hat{\mu})^2}{2 \left( (\sigma^2/n_j) + \tau^2\right)}
   \right\}
\end{align*}
where we have
\[ \bar{y}_j = \frac{1}{n_j} \sum^{n_j}_{i=1} y_{ij}, 
   \qquad \hat{\mu} = \frac{ \sum^J_{j=1} \frac{1}{
      (\sigma^2/n_j)+\tau^2} \bar{y}_j}{ \sum^J_{j=1} \frac{1}{
	 (\sigma^2/n_j) +
	 \tau^2}}, \qquad V_\mu =  \frac{1}{\sum^J_{j=1} 
	 \frac{1}{(\sigma^2/n_j) + \tau^2}} \]
The only distribution that is nonstandard is $\tau^2 | y$, which
will require grid sampling.  
\\
\\
\\
{\sl Posterior Predictive Sampling}: There are two ways we can take
posterior predictive draws, depending on our goal:
\begin{enumerate}
   \item Samples from currently existing group: In this case, we
      consider what would happen if there were more observations
      in group $j$. Here's how we implement:
      \begin{enumerate}
	 \item Draw $\tau^2$ from it's posterior distribution using grid
	    sampling.
	 \item Draw $\mu_0$ from the conditional posterior 
	    distribution of $\mu_0 |\tau^2, y$.  
	 \item Sample $\mu_j$---for the group $j$ you are 
	    considering---from $p(\mu_j | \mu_0, \tau^2, y)$ using
	    the values of $\mu_0$ and $\tau^2$ just obtained.
	 \item Sample from the distribution for 
	    $\tilde{y} \sim N(\mu_j, \sigma^2)$, where $\mu_j$ is
	    the value just obtained.
	 \item Repeat the process.
      \end{enumerate}
   \item Samples from entirely new group $\tilde{j}$: Here, we 
      consider what would happen if we introduce a new group 
      entirely. This will get a very diffuse posterior predictive
      distribution.
      \begin{enumerate}
	 \item Draw $\tau^2$ from it's posterior distribution using grid
	    sampling.
	 \item Draw $\mu_0$ from the conditional posterior 
	    distribution of $\mu_0 |\tau^2, y$.  
	 \item Sample $\tilde{\mu}$ from N$(\mu_0, \tau^2)$ using the
	    $\mu_0$ just obtained.
	 \item Sample $\tilde{y} \sim $ N$(\tilde{\mu}, \sigma^2)$.
	 \item Repeat the process.
      \end{enumerate}
\end{enumerate}

\newpage
\subsection{Normal Model with the EM Algorithm}

We saw the EM Algorithm with Mixture Models; however, this technique, 
where we suppose that there is ``missing data,''
can also useful in Hierarchical Models.  Let's return to the normal
model with known variance:
\begin{align*}
   y_{ij} &\sim N(\mu_j, \sigma^2) \\
   \mu_j &\sim N(\mu_0, \tau^2) 
\end{align*}
In the last section, we considered $\mu$, $\mu_0$ and $\tau^2$ as our
unknown parameters, but suppose we consider only $(\mu_0, \tau^2)$ as
unknown and let all of the components $\mu_j \in \mu$ be ``missing
data.'' Then we could say that the estimation of $\mu_0$ and $\tau^2$
would be really easy if we knew the $\mu_j$. This leads us to the
EM algorithm:
\begin{enumerate}
   \item Expectation Step: We want to compute the expectation of the
      complete data log-likelihood. This differs a bit from the 
      Expectation Step in the Mixture Model implementation, but
      it's a straightforward generalization.
      \[ Q(y_{\text{obs}} | \theta) = E_{y_\text{mis}}\left[ 
	 \ell(y_{\text{obs}}, y_{\text{mis}} | \theta) \right]\]
      In our normal example, this means
      \begin{align*}
	 p(y, \mu | \mu_0, \tau^2) &\propto \prod^m_{j=1} 
	    (\tau^2)^{-1/2} \exp\left\{-\frac{1}{2\tau^2} 
	    (\mu_j - \mu_0)^2\right\} \\
	    &\qquad \times \prod^m_{j=1} \prod^{n_j}_{i=1}
	    \exp\left\{ -\frac{1}{2\sigma^2}(y_{ij}-\mu_j)^2\right\} \\
	 \Rightarrow \quad Q(y_{\text{obs}}|\theta)
	    &= E[\ell(y,\mu|\mu_0, \tau^2)] = E[\ln 
	    p(y, \mu | \mu_0, \tau^2)]
      \end{align*}
   \item Maximization Step: Next, we calculate the $\hat{\theta}$ 
      that maximizes $Q(y_{\text{obs}}|\theta)$
\end{enumerate}

\newpage
\subsection{Binomial Hierarchical Model}

Let's suppose our vector observations, denoted $y$, is binomially
distributed. Moreover, each observation, $y_i$, has its own parameter
$\theta_i$ drawn from a distribution---this is the hierarchical part.
So together, we have
\begin{align*}
   y_i &\sim \text{Binom}(n_i, \theta_i)\\
   \theta_i &\sim \text{Beta}(\alpha, \beta).
\end{align*}
Let's specify the generic posterior, taking advantage of the conditional
independence built into the hierarchical model, noting that $\theta$
is a vector of $\theta_i$:
\begin{align*}
   p(\theta, \alpha, \beta | y) &\propto p(y | \theta, \alpha, \beta)
      \cdot p(\theta, \alpha, \beta) \\
   &\propto p(y | \theta)
      \cdot p(\theta | \alpha, \beta) \cdot p( \alpha, \beta) 
\end{align*}
We can substitute in for the particular distributions, leaving our 
prior distribution on $\alpha$ and $\beta$ unspecified:
\begin{align*}
 p(\theta, \alpha, \beta | y)
   &\propto \left[ \prod^n_{i=1} \binom{n_i}{y_i} \theta_i^{y_i}   
   (1-\theta_i)^{n_i-y_i}\right] \cdot \left[ \prod^n_{i=1}
   \frac{\Gamma(\alpha+\beta)}{\Gamma(\alpha) \Gamma(\beta)}
   \theta_i^{\alpha-1} (1-\theta_i)^{\beta-1}\right] \cdot
   p(\alpha, \beta) \\
   &\propto p(\alpha, \beta) 
      \left[\frac{\Gamma(\alpha+\beta)}{\Gamma(\alpha) \Gamma(\beta)}
	 \right]^n \prod^n_{i=1} \theta_i^{y_i + \alpha -1}
      (1-\theta_i)^{n_i - y_i + \beta - 1} 
\end{align*}
Let's get the conditional posterior distribution of $\theta |
\alpha, \beta, y$:
\begin{align*}
   p(\theta_i | \alpha, \beta, y) &\propto \theta_i^{y_i + \alpha -1}
      (1-\theta_i)^{n_i - y_i + \beta - 1} \\
   \theta_i | \alpha, \beta, y &\sim \text{Beta}(y_i + \alpha, \;
      n_i - y_i + \beta) 
\end{align*}
Now for the posterior distribution of $\alpha$ and $\beta$ we'll need
to make a choice for our prior. We'd probably like to do something
non-informative, like $p(\alpha, \beta)\propto 1$, but that would
lead to an improper posterior. Gelman suggests that we use 
$p(\alpha, \beta) \propto (\alpha + \beta)^{-5/2}$, which is justified
in Exercise 5.7 in the book. This leads to
\begin{align*}
   p(\alpha, \beta | y) &\propto \int p(\theta, \alpha, \beta |y)\;
      d\theta \\
   &\propto (\alpha + \beta)^{-5/2}
      \left[\frac{\Gamma(\alpha+\beta)}{\Gamma(\alpha) \Gamma(\beta)}
	 \right]^n \int \prod^n_{i=1} \theta_i^{y_i + \alpha -1}
	 (1-\theta_i)^{n_i - y_i + \beta - 1} \; d\theta_i
\end{align*}
This looks like a beast of an integral, but notice that what's inside 
the integral is the beta distribution, unnormalized by a constant. So
it's value must be one divided by that normalizing constant,\footnote{
Alternatively, we could have computed $p(\alpha, \beta |y)$
by the other method: $p(\alpha, \beta | y) = 
p(\theta, \alpha, \beta | y)/p(\theta|\alpha, \beta | y)$, which
would have given us the same result (and is easy
to compute).} which we can write
\[ p(\alpha, \beta | y) \propto (\alpha + \beta)^{-5/2}
   \left[\frac{\Gamma(\alpha+\beta)}{\Gamma(\alpha) \Gamma(\beta)}
      \right]^n  \prod^n_{i=1} \frac{ \Gamma(y_i + \alpha) \Gamma(
      n_i - y_i + \beta)}{\Gamma(n_i + \alpha + \beta)} \]
To implement, we sample $\alpha$ and $\beta$ using grid sampling, 
then sample each $\theta_i | \alpha, \beta$.

\newpage
\section{Markov Chain Monte-Carlo (MCMC) Algorithms}

Markov Chain Monte-Carlo Algorithms allow us to sample from the 
posterior distribution of non-standard distributions. We saw 
grid-sampling earlier as a way to do so, but also noted the limitations
when we have posteriors with great than two parameters. 
\\
\\
We also saw Newton Raphson and the EM algorithm as ways to estimate
parameters, but they only give point estimates as opposed to full
posterior distributions.  MCMC algorithms, which is a \emph{stochastic}
optimization algorithm, attempt to overcome these difficulties
by sampling sequentially from a variety of univariate distributions.

\subsection{Gibbs Sampler}

Our goal is to obtain samples for $\theta | y$, where $\theta$ 
is a vector of parameters, $(\theta_1, \ldots, \theta_k)$. Here's
the procedure:
\begin{enumerate}
   \item Start with a set of arbitrary values for the parameter 
      \[ \theta^{(0)} = (\theta_1^{(0)}, \ldots,  \theta_k^{(0)}) \]
   \item Next, we sample iteratively from conditional posterior 
      distributions of each parameter---given current values of
      other parameters---as follows:
      \begin{align*}
	 \theta_1^{(t+1)} &\sim p(\theta_1 | 
	    \theta_2^{(t)}, \ldots,  \theta_k^{(t)}, y)\\
	 \theta_2^{(t+1)} &\sim p(\theta_2 | 
	    \theta_1^{(t+1)},\theta_3^{(t)},\ldots,\theta_k^{(t)}, y)\\
	 \vdots \quad & \qquad \vdots \\
	 \theta_k^{(t+1)} &\sim p(\theta_k | 
	    \theta_1^{(t+1)},\ldots, \theta_{k-1}^{(t+1)}, y)\\
      \end{align*}
      As noted above, the conditional distributions will often be 
      standard distributions that are relatively simple since we
      can absorb $k-1$ parameters into the $\propto$ each time
      we need to get a draw for $\theta_i$.
   \item We iterate through this, and eventually the distribution will
      converge to a range of values for each $\theta_i$. But
      the question is ``How exactly do we know when convergence
      occurs?'' 
      
      To answer this question, we typically run multiple
      Markov chains from multiple starting values, $\theta^{(0)}$.
      Once the chains multiple chains start moving together, we declare
      convergence and remove the ``burn-in'' before the chains 
      converged.
   \item After we remove the burn-in, we still have the problem of
      autocorrelated samples.  Since the Gibbs Sampler generates
      a Markov chain, autocorrelation is baked into the procedure.
      So to obtain independent samples, we will need to evaluate the
      autocorrelation of our chains, ``thin'' them by only taking
      every $m$th value or so, where $m$ is chosen so that 
      there is negligible autocorrelation in the resulting sample.
\end{enumerate}

\newpage
\subsection{Metropolis Algorithm}

The Metropolis algorithm is an MCMC method that uses rejection 
sampling to draw from a posterior distribution. This is particularly
when the conditional posterior distributions are non-standard, so
that the Gibbs sampler has trouble. Here are the steps involved:
\begin{enumerate}
   \item Initialize at a starting point, $\theta^0$. 
      Then for each $t \in \{1, 2, \ldots\}$, 
      sample a \emph{proposal} value, which I'll denote 
      $\theta^*$, from some proposal or ``jumping'' distribution,
      $g(\theta^* | \theta^{t-1})$.\footnote{The
      dependence on the previous value, $\theta^{t-1}$, makes
      our method an Mark Chain Monte Carlo method.} 
      Typically, $g(\cdot | \cdot)$ will be a standard distribution.
   \item Form the ratio of the densities 
      \begin{equation}
	 \label{metalg}
	 r = \frac{p(\theta^* | y)}{p(\theta^{t-1} | y)} 
      \end{equation}
   \item Then, once we have a value for $r$, we set
      \begin{equation}
	 \label{accept}
	  \theta^t = \begin{cases} \theta^* & \text{with probability
	    $\min\{r,1\}$.} \\ \theta^{t-1} & \text{otherwise}
	 \end{cases}
      \end{equation}
      Intuitively, we accept a draw $\theta^*$ if it increases 
      the posterior density. If not, we randomly accept
      it, but only with probability proportional to how good it is.
      In this way, this method acts like a stochastic 
      stepwise mode-finding algorithm.
\end{enumerate}
Note, for the Metropolis algorithm, we \emph{require} that the 
proposal (or ``jumping'') distribution, $g(\cdot | \cdot)$, is 
symmetric, i.e.
\[ g(\theta^* | \theta^{t-1}) = g( \theta^{t-1} |\theta^* ), 
   \qquad \forall \theta^*, \theta^{t-1}, t  \]
The common and convenient normal distribution obeys this rather nice
property and can be used with this algorithm.

\subsection{Metropolis-Hastings Algorithm}

The \emph{Metropolis-Hastings Algorithm} is a direct generalization
of the Metropolis algorithm. It relaxes the restrictive 
requirement that $g(\cdot|\cdot)$, our proposal distribution, be 
symmetric.
\\
\\
Except for one step, the algorithm functions \emph{exactly} the same
way as the Metropolis Algorithm. We only have to modify the 
``attractiveness'' ratio, $r$, given by Equation \ref{metalg}. Here's
the new ratio for the new Metropolis-Hastings Algorithm:
\begin{equation}
   \label{methast}
   r = \frac{ p(\theta^* | y) }{p(\theta^{t-1} | y)} \cdot 
   \frac{ g(\theta^{t-1} | \theta^*)}{ g(\theta^* | \theta^{t-1})} 
\end{equation}
We accept or reject draws $\theta^*$ from $g(\cdot | \cdot)$
just as before, in Expression \ref{accept}.

\subsection{Gibbs Sampler as a Special Case of the M-H
   Algorithm}
The Gibbs sampler can be viewed as a special case of the more general 
Metropolis-Hastings Algorithm. To see why,
suppose we \emph{could} draw directly from the posterior distribution,
$p(\theta | y)$, that we are targeting with our Metropolis-Hastings
algorithm. (Obviously unnecessary to use the M-H Algorithm in that case,
but play along). 
\\
\\
Then we can take the posterior as our proposal distribution, 
$g(\theta^*| \theta^{t-1}) = p(\theta^*| y)$. In that case, the 
ratio in Equation \ref{methast} will \emph{always} be one, so that
every proposed $\theta^*$ is accepted as in the Gibbs sampler.

\newpage
\section{EM Algorithm}

Suppose we hope to estimate the posterior, $p(\phi|y)$, for
a vector of parameters, $\phi$,
despite the posterior distribution being nonstandard. Then suppose
we introduce some ``missing data,'' denoted $\gamma$, in the hope that
knowing $\gamma$ would make the estimation of the posterior of
$\phi$ easier. 
\\
\\
More explicitly, it may be tough to find $p(\phi|y)$, but by introducing
$\gamma$, we may be able to work with $p(\gamma|\phi,y)$ and 
$p(\phi|\gamma,y)$. In particular, 
\begin{enumerate}
   \item Start with a guess for $\phi$ and $\gamma$.
   \item Replace the components of $\gamma$ with
      their conditional expectation, $E[\gamma | \phi]$.
   \item Maximize the posterior
      likelihood given $\gamma$, which is denoted $p(\phi|\gamma,y)$,
      and replace $\phi$ with the maximized values.
   \item Repeat Steps 1 and 2 until convergence. 
\end{enumerate}



%%%%%%%%%% APPENDIX %%%%%%%%%%%%%%%%%%%%%%%%%%%%%%%%%%%%%%%%%%%%

\newpage


\appendix

\newpage
\section{Finding Parameters for and 
Drawing from Non-Standard Distributions}

Noting that some of the resulting posterior distributions won't 
necessarily be standard, it's important to know how to find parameters
and get probabilities
from these so-called \emph{non-standard distributions}. The simplest
is the grid method, and while easy to do from a computational 
standpoint,
it can be difficult when it comes to selecting a proper grid
initially.

\subsection{Grid Method in One Dimension}

Suppose we want to get samples from a non-standard distribution. Then
we follow this process to generates probabilities and then samples for 
$\theta$. 
\begin{enumerate}
   \item Pick a grid of possible values for $\theta$. A fair amount
      of trial and error will be involved here.\footnote{You choose
      the value of $\theta$ that maximize the likelihood or
      (more often) log-likelihood.}
   \item Calculate $m(\theta) = \frac{p(\theta, y)}{\sum p(\theta, y)}$,
      where the sum is the sum over the grid. This generates rough
      approximations of the true probabilities.
   \item Sample one grid value with probabilities proportional to 
      $m(\theta)$.
   \item Repeat the previous step many times to generate a large sample.
\end{enumerate}

\subsection{Grid Method in Two Dimensions} 

Here, suppose there are two parameters which define a non-standard
distribution. We decompose the grid method into two parts:
\begin{enumerate}
   \item Pick a grid of values for $\alpha$ and $\beta$, our two 
      parameters. Again, this will probably take some trial and 
      error.
   \item Calculate the probabilities $m(\alpha, \beta) = \frac{ 
      p(\alpha, \beta |y)}{\sum p(\alpha, \beta | y)}$.
   \item Calculate marginal and conditional posterior probabilities:
      \[ m(\alpha) = \sum_\beta m(\alpha, \beta), \qquad 
	 m(\beta|\alpha) = \frac{m(\alpha, \beta)}{m(\alpha)}\]
   \item Sample a grid value $\alpha_i$ with probability proportional
      to $m(\alpha)$. 
   \item Based on our sampled value of $\alpha_i$, sample grid value
      $\beta_i$ with probability proportional to $p(\beta | \alpha_i)$.
   \item Repeat the last couple steps many times to get distributions.
\end{enumerate}

\subsection{Newton's Method in One Dimension}

Suppose we have some function $f(x)$, and we want to find a root
of this function. Then we start with a linear approximation
to the Taylor's Series expansion 
of $f(x)$ about some starting point, $x_k$:
   \[ f(x) = f(x_k) + f'(x_k)(x-x_k) + O(x^2).\]
Since we want to find a root, we set the last equation equal to
zero and solve for a new guess
\begin{align*}
   f(x) = 0 &= f(x_k) + f'(x_k)(x-x_k) + O(x^2) \\
   \Rightarrow x_{k+1} &= x_k - \frac{f(x_k)}{f'(x_k)} 
\end{align*}
Iterating this process, we can get better and better approximations
of the root of $f(x)$.\footnote{Note that there might not be a
single root. In fact the Newton-Raphson algorithm is very sensitive
to the starting point.}
\\
\\
Now the question is how do we apply this to finding parameters? Well
that's quite simple. Instead of trying finding a root for $f(x)$,
we typically try to maximize the likelihood---which involves finding
a root for $\ell'(\theta)$, the derivative of the log-likelihood.
So let's go through the steps to find a parameter:
\begin{align*}
   \hat{\theta} &= \arg \min_\theta L(\theta) = 
      \arg \min_\theta \ell(\theta) \\
      0 &= \ell'(\hat{\theta}) \qquad \text{Solve for $\hat{\theta}$}\\
      \text{Newton Method} \qquad 0 &= \ell'({\theta}_k) +
      \ell''({\theta}_k)({\theta}_{k+1} - {\theta}_k) 
	 + O({\theta}^2)\\
	 {\theta}_{k+1} &= {\theta}_k - 
	 \frac{\ell'({\theta}_k)}{\ell''({\theta}_k)}
\end{align*}
Again, iterating over this process, we can get better and better
estimates of $\hat{\theta}$.\footnote{The same cautions about 
the sensitivity to starting point still apply.}



\subsection{Newton's Method in Multiple Dimensions}

Again, let's start with the general case, then move to our special case
of finding parameters. So we have a vector valued function, and 
want to expand it into a Taylor Series approximation.:
   \[ f(x+h) = f(x) + f'(x)h + O(h^2) \]
   \[ x = (x_1,\ldots, x_n)^T, \qquad h = (h_1, \ldots, h_n)^T, \qquad
      f(x) = (f_1(x), \ldots, f_n(x) )^T\]
   \[ f'(x) = J = \begin{pmatrix} \frac{\partial f_1}{\partial x_1}
      & \cdots & \frac{\partial f_1}{\partial x_n} \\
      \vdots & \ddots & \vdots \\ \frac{\partial f_n}{\partial x_1}
      & \cdots & \frac{\partial f_n}{\partial x_n}
      \end{pmatrix} \]
Then, you use
\begin{align*} 
   0 &= f(x) + J h \\
   x_{k+1} &= x_k - J^{-1} f(x_k) 
\end{align*} 
Now let's adopt this to our problem of a two-parameter optimization.
Suppose we are faced with
\begin{align}
   \label{jacobian}
   (\hat{\alpha}, \hat{\beta}) &= \arg \max_{\alpha, \beta} L(
   \alpha, \beta) = \arg \max_{\alpha, \beta} \ell(
   \alpha, \beta)\notag \\
\text{Solve} \qquad 0 &= \frac{\partial\ell(
   \hat{\alpha}, \hat{\beta})}{\partial\alpha}, \qquad 0 = 
   \frac{\partial\ell(\hat{\alpha}, \hat{\beta})}{\partial\beta} 
\end{align}
Awesome, so we have a vector valued function that we wish to find
a root for. So let's do it:
\begin{align*}
   \ell(\alpha, \beta) &= \begin{pmatrix} 
      \frac{\partial\ell({\alpha}, {\beta})}{\partial\alpha} &
      \frac{\partial\ell({\alpha}, {\beta})}{\partial\beta}
   \end{pmatrix}\\
   J = \ell'(\alpha, \beta) &= \begin{pmatrix} 
      \frac{\partial^2\ell({\alpha}, {\beta})}{\partial\alpha^2} &
      \frac{\partial^2\ell({\alpha}, {\beta})}{\partial\alpha\partial
      \beta} \\
      \frac{\partial^2\ell({\alpha}, {\beta})}{\partial\alpha\partial
      \beta}&
      \frac{\partial^2\ell({\alpha}, {\beta})}{\partial\beta^2}  
   \end{pmatrix}
\end{align*}
From there, we get an estimate by iteration over
   \[ (\alpha_{k+1}, \beta_{k+1}) = (\alpha_{k}, \beta_{k}) -
      J^{-1} \ell(\alpha_{k}, \beta_{k}) \]
For more general cases, where the number of parameters is greater
than one or two, simply expand the number partial derivatives and
maximization criteria that we labeled Equations \ref{jacobian}.
The only difference is that the Jacobian and function of the parameters
would have a higher dimension.



\newpage
\subsection{Gradient Descent}

Again, suppose we want to find point estimates of the components in
a vector $\theta = (\theta_1, \theta_2, \ldots, \theta_n)$ by
minimizing some cost function, $\mathcal{J}(\theta)$.\footnote{We
can easily make the necessary changes if we want to maximize. In
particular, we can minimize the negative of some target function. (The
name ``cost function'' isn't really appropriate if we're trying to 
maximize.)} Then we update our parameters all at once as follows:
\begin{align*}
   \theta_1^{(t+1)} &= \theta_1^{(t)} - \alpha \frac{\partial}{
      \partial \theta_1^{(t)}}\left[ \mathcal{J}(\theta)\right]\\
   \vdots \qquad & \qquad \vdots \\
   \theta_n^{(t+1)} &= \theta_n^{(t)} - \alpha \frac{\partial}{
      \partial \theta_n^{(t)}}\left[ \mathcal{J}(\theta)\right]
\end{align*}
where $\alpha$ is the learning rate. This parameter controls the
size of our steps and influences the aggressiveness of our 
descent.
We keep updating and iterating through this process until convergence.  
\\
\\
Moreover, notice how this process self-adjusts in the
magnitude of the update.  In particular, as you get close to
the minimum, the derivatives approach zero.  So even with 
$\alpha$ constant, the shrinking derivative means that the gradient
descent algorithm will take smaller steps.


\paragraph{Note} At each iteration, we update all the components of 
$\theta$ \emph{simultaneously}. That is, the equations above are a 
package deal. We don't update one parameter at a time as we did
with Gibbs Sampling. Instead, we take all the partial derivatives
of the cost function, $\mathcal{J}(\theta)$, then plug into the RHS
of the above expressions.







\newpage
\section{Parametric Bootstrap}

In this method, which is in the realm of classical rather than Bayesian
statistics, you have some data $\mathbf{y}$, which you use to estimate
a parameter $\theta$ using Maximum Likelihood Estimation. Using
this $\theta_{\text{MLE}}$, you then simulate a new random sample of 
data using the likelihood,
\[ p(\mathbf{y} | \theta_{\text{MLE}}) \]

\section{Jacobian Transformation}

This method is useful for determining the probability distribution
of some variable $\phi = h(\theta)$,\footnote{Note that $\phi$ must be a 
function of $\theta$ and $\theta$ only.} especially
when we know the probability distribution 
of $\theta$, $p(\theta)$.  Then we apply
   \[ f(\phi) = f(\theta) \left\lvert 
      \frac{d\theta}{d\phi} \right\rvert \]
where $f$ is the pdf and the term following $f(\theta)$ is known as 
the \emph{Jacobian}.

\begin{proof} 
   Just to sketch the proof, let's start with what we know:
   $f(\theta)$ and $\phi = h(\theta)$. We want to find
      \[ f(\phi) = \frac{d}{d\phi} \left[ F(\phi) \right] \]
   where $F$ is the cdf. We can nicely rewrite and rearrange this as
   \begin{align*}
      f_\Phi(\phi) &= \frac{d}{d\phi} 
      \left[ F_\Phi(\phi) \right] = 
      \frac{d}{d\phi} \left[ P(\Phi < \phi ) \right]
      \\
      &= \frac{d}{d\phi} \left[ P( h(\theta) < \phi ) \right]
	 = \frac{d}{d\phi} \left[ P( \theta < h^{-1}(\phi) )  \right]
      \\ 
      &= \frac{d}{d\phi} \int^{h^{-1}(\phi)}_{-\infty} 
	 f_\Theta(\theta)\; d\theta
      \\
      &= f_\Theta\left( h^{-1}(\phi) \right) \left\lvert 
	 \frac{d}{d\phi} \left[ h^{-1}(\phi) \right] \right\rvert
   \end{align*}
And if we use the fact that $\theta = h^{-1}(\phi)$ and drop some of the
pedantic subscripting, we can write a more compact version:
   \[ f(\phi) = f(\theta) \left\lvert 
      \frac{d\theta}{d\phi} \right\rvert\]

\end{proof}

\end{document}

