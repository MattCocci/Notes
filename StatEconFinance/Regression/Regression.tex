\documentclass[12pt]{article}

\author{Matthew D. Cocci}
\title{Regression}
\date{\today}

%% Formatting & Spacing %%%%%%%%%%%%%%%%%%%%%%%%%%%%%%%%%%%%

%\usepackage[top=1in, bottom=1in, left=1in, right=1in]{geometry} % most detailed page formatting control
\usepackage{fullpage} % Simpler than using the geometry package; std effect
\usepackage{setspace}
%\onehalfspacing
\usepackage{microtype}

%% Formatting %%%%%%%%%%%%%%%%%%%%%%%%%%%%%%%%%%%%%%%%%%%%%

%\usepackage[margin=1in]{geometry}
    %   Adjust the margins with geometry package
%\usepackage{pdflscape}
    %   Allows landscape pages
%\usepackage{layout}
    %   Allows plotting of picture of formatting



%% Header %%%%%%%%%%%%%%%%%%%%%%%%%%%%%%%%%%%%%%%%%%%%%%%%%

%\usepackage{fancyhdr}
%\pagestyle{fancy}
%\lhead{}
%\rhead{}
%\chead{}
%\setlength{\headheight}{15.2pt}
    %   Make the header bigger to avoid overlap

%\fancyhf{}
    %   Erase header settings

%\renewcommand{\headrulewidth}{0.3pt}
    %   Width of the line

%\setlength{\headsep}{0.2in}
    %   Distance from line to text


%% Mathematics Related %%%%%%%%%%%%%%%%%%%%%%%%%%%%%%%%%%%

\usepackage{amsmath}
\usepackage{amsfonts}
\usepackage{mathrsfs}
\usepackage{amsthm} %allows for labeling of theorems
\theoremstyle{plain}
\newtheorem{thm}{Theorem}[section]
\newtheorem{lem}[thm]{Lemma}
\newtheorem{prop}[thm]{Proposition}
\newtheorem{cor}[thm]{Corollary}

\theoremstyle{definition}
\newtheorem{defn}[thm]{Definition}
\newtheorem{ex}[thm]{Example}

\theoremstyle{remark}
\newtheorem*{rmk}{Remark}
\newtheorem*{note}{Note}

% Below supports left-right alignment in matrices so the negative
% signs don't look bad
\makeatletter
\renewcommand*\env@matrix[1][c]{\hskip -\arraycolsep
  \let\@ifnextchar\new@ifnextchar
  \array{*\c@MaxMatrixCols #1}}
\makeatother


%% Font Choices %%%%%%%%%%%%%%%%%%%%%%%%%%%%%%%%%%%%%%%%%

\usepackage[T1]{fontenc}
\usepackage{lmodern}
\usepackage[utf8]{inputenc}
%\usepackage{blindtext}


%% Figures %%%%%%%%%%%%%%%%%%%%%%%%%%%%%%%%%%%%%%%%%%%%%%

\usepackage{graphicx}
\usepackage{subfigure}
    %   For plotting multiple figures at once
%\graphicspath{ {Directory/} }
    %   Set a directory for where to look for figures


%% Hyperlinks %%%%%%%%%%%%%%%%%%%%%%%%%%%%%%%%%%%%%%%%%%%%
\usepackage{hyperref}
\hypersetup{
    colorlinks,
        %   This colors the links themselves, not boxes
    citecolor=black,
        %   Everything here and below changes link colors
    filecolor=black,
    linkcolor=black,
    urlcolor=black
}

%% Including Code %%%%%%%%%%%%%%%%%%%%%%%%%%%%%%%%%%%%%%%

\usepackage{verbatim}
    %   For including verbatim code from files, no colors

\usepackage{listings}
\usepackage{color}
\definecolor{mygreen}{RGB}{28,172,0}
\definecolor{mylilas}{RGB}{170,55,241}
\newcommand{\matlabcode}[1]{%
    \lstset{language=Matlab,%
        basicstyle=\footnotesize,%
        breaklines=true,%
        morekeywords={matlab2tikz},%
        keywordstyle=\color{blue},%
        morekeywords=[2]{1}, keywordstyle=[2]{\color{black}},%
        identifierstyle=\color{black},%
        stringstyle=\color{mylilas},%
        commentstyle=\color{mygreen},%
        showstringspaces=false,%
            %   Without this there will be a symbol in
            %   the places where there is a space
        numbers=left,%
        numberstyle={\tiny \color{black}},%
            %   Size of the numbers
        numbersep=9pt,%
            %   Defines how far the numbers are from the text
        emph=[1]{for,end,break,switch,case},emphstyle=[1]\color{red},%
            %   Some words to emphasise
    }%
    \lstinputlisting{#1}
}
    %   For including Matlab code from .m file with colors,
    %   line numbering, etc.

%% Bibliographies %%%%%%%%%%%%%%%%%%%%%%%%%%%%%%%%%%%%

%\usepackage{natbib}
    %---For bibliographies
%\setlength{\bibsep}{3pt} % Set how far apart bibentries are

%% Misc %%%%%%%%%%%%%%%%%%%%%%%%%%%%%%%%%%%%%%%%%%%%%%

\usepackage{enumitem}
    %   Has to do with enumeration
\usepackage{appendix}
%\usepackage{natbib}
    %   For bibliographies
\usepackage{pdfpages}
    %   For including whole pdf pages as a page in doc


%% User Defined %%%%%%%%%%%%%%%%%%%%%%%%%%%%%%%%%%%%%%%%%%

%\newcommand{\nameofcmd}{Text to display}
\newcommand*{\Chi}{\mbox{\large$\chi$}} %big chi
    %   Bigger Chi



%%%%%%%%%%%%%%%%%%%%%%%%%%%%%%%%%%%%%%%%%%%%%%%%%%%%%%%%%%%%%%%%%%%%%%%%
%% BODY %%%%%%%%%%%%%%%%%%%%%%%%%%%%%%%%%%%%%%%%%%%%%%%%%%%%%%%%%%%%%%%%
%%%%%%%%%%%%%%%%%%%%%%%%%%%%%%%%%%%%%%%%%%%%%%%%%%%%%%%%%%%%%%%%%%%%%%%%


\begin{document}
\maketitle

\section{Basic Setup}

\subsection{Potential Outcomes}

Suppose that we want to consider outcome $Y_i$ for individual $i$.
Let $D_i$ denote the treatment for this individual, where
\begin{align*}
  D_i =
  \begin{cases}
    1 & \text{Individual $i$ receives the treatment} \\
    0 & \text{otherwise} \\
  \end{cases}
\end{align*}
Before you know whether this person received the treatment, you could
imagine two \emph{potential outcomes} for individual $i$:
\begin{align*}
  Y_{1i}
  &= \text{Outcome if she receives the treatment ($D_i=1$)} \\
  Y_{0i}
  &= \text{Outcome if she does \emph{not} receive the treatment ($D_i=0$)}
\end{align*}
Really, we will only observe one of these cases; the other is just a
counterfactual. But it helps us to pin down concepts and write the true,
observed outcome $Y_i$ as a function of the treatment and potential
outcomes:
\begin{align}
  Y_i = D_i Y_{1i} + (1 - D_i) Y_{0i}
  \label{potential}
\end{align}
This will simplify and clarify a whole range of results below.

\subsection{Selection Bias}

To measure the effect of some treatment, we might examine the difference
in outcomes that we observe between people who get the treatment and
those who don't, comparing sample estimates of
\begin{align*}
  \mathbb{E}[Y_i | D_i = 1] - \mathbb{E}[Y_i | D_i = 0]
\end{align*}
But by writing $Y_i$ out in terms of potential outcomes as in
Equation~\ref{potential}, we see the problem:
\begin{align}
  \underbrace{\mathbb{E}[Y_i | D_i = 1]
  - \mathbb{E}[Y_i | D_i = 0]}_{\text{Observed Effect}}
  &=
  \mathbb{E}[Y_{1i} | D_i = 1]
  - \mathbb{E}[Y_{0i} | D_i = 0] \notag\\
  &=
  \mathbb{E}[Y_{1i} | D_i = 1] - \mathbb{E}[Y_{0i} | D_i = 1]
  + \mathbb{E}[Y_{0i} | D_i = 1] - \mathbb{E}[Y_{0i} | D_i = 0] \notag\\
  &=
  \underbrace{\mathbb{E}[Y_{1i} - Y_{0i}| D_i = 1]}_{%
    \text{Treatment Effect}}
  + \underbrace{\mathbb{E}[Y_{0i} | D_i = 1]
  - \mathbb{E}[Y_{0i} | D_i = 0]}_{\text{Selection Bias}}
  \label{bias}
\end{align}
In words, the observed effect confounds two things:
\begin{enumerate}
  \item \emph{Selection Bias}: This pollutes our estimate of the
    treatment effect. Selection bias introduces into the observed effect
    the additional complication that people who receive the treatment
    and who don't might \emph{fundamentally} be different types of
    people. Examples:
    \begin{enumerate}
      \item People in the hospital are (tautologically) sicker than the
        average person---that's why they're in the hospital. So don't be
        surprised if they actually end up having \emph{worse} health
        outcomes (despite their hospital stints) than the healthy people
        visiting them.
      \item Kids who get into elite schools would probably have made
        more money anyway, even without an elite-school salary bump.
    \end{enumerate}
    The list goes on and on.
  \item \emph{Treatment Effect}: This is genuine comparison of
    counterfactuals, and really what we want to measure. It represents
    the difference in potential outcomes under the treatment for
    individual $i$. Returning to the examples above, the treatment
    effect instead captures things like:
    \begin{enumerate}
      \item How much better off is a person for going to the hospital
        and getting stitches, rather than trying to walk it off?
      \item How much do the connections and peer effects of Harvard
        translate into more money for your son or daughter, relative to
        four (plus maybe super-senior five) years at Party State? Is it
        worth sacrificing better keg stand form?
    \end{enumerate}
\end{enumerate}

\begin{defn}
\label{defn:selbias}
Selection Bias is \emph{correlation between treatment and
\textbf{potential} outcomes}. Selection bias occurs when those who
receive some treatment and those who don't fundamentally differ, even
before \emph{before} receiving the treatment.
\end{defn}

\begin{rmk}
Note the subtlety in this definition. I'm not saying selection bias is
correlation between treatment and simple outcomes (sans the
``potential'' quantifier). That would be ridiculous. We usually expect
or even \emph{hope} there exists correlation between treatment and
outcomes. We expect smaller class sizes to influence grades; we expect the hospital treatment to help sick patients' outcomes; and so on.

It's correlation between treatment and \emph{potential} outcomes that we
don't want. That's because we want to isolate the pure effect of the
treatment on outcomes, and if there is something in the background that
\emph{systematially} influences both of them, well then we haven't
really pinned down the true nature of the relationship of treatment and
outcomes.
\end{rmk}

\subsection{Randomization}

So how do we get rid of selection bias? The simplest answer is
randomization: Randomly decide who receives the treatment. Then
potential outcomes don't depend upon treatment so
\begin{align*}
  \mathbb{E}[Y_{0i} | D_i=1]
  =\mathbb{E}[Y_{0i} | D_i=0]
\end{align*}
so that the selection bias term in Identity~(\ref{bias}) drops and the
observed effect equals the treatment effect.

\subsection{Regression Setup}

Start with the most basic regression setup, where we try to estimate the
effect $\rho$ of some treatment $D_i$ on individual $i$:
\begin{align}
  Y_i =
  \underbrace{\alpha}_{\mathbb{E}Y_i}
  + \underbrace{\rho}_{Y_{1i} - Y_{0i}} D_i
  + \underbrace{\varepsilon_i}_{Y_{0i}-\mathbb{E}Y_{0i}}
  \label{basic}
\end{align}

\begin{thm}
In the context of the basic regression setup, selection bias is
correlation between the error term $\varepsilon_i$ and the treatment
regressor $D_i$.
\end{thm}
\begin{rmk}
This is very similar to our earlier Definition~\ref{defn:selbias}, only
recast in the terms of regression. To square with that earlier
definition, we need to show that correlation of the error term and $D_i$
is isomorphic to correlation of potential outcomes and $D_i$. Therefore,
the proof will demonstrate that $\varepsilon_i$ \emph{contains} all the
information about potential outcomes here.
\end{rmk}
\begin{proof}
Evaluating (\ref{basic}) under no treatment and treatment:
\begin{align*}
  \mathbb{E}[Y_i | D_i = 1]
    &= \alpha + \rho + \mathbb{E}[\varepsilon_i | D_i = 1]\\
  \mathbb{E}[Y_i | D_i = 0]
    &= \alpha + 0 + \mathbb{E}[\varepsilon_i | D_i = 0] \\\\
  \Rightarrow\qquad
  \mathbb{E}[Y_i | D_i = 1] - \mathbb{E}[Y_i | D_i = 0]
    &= \underbrace{\rho}_{\text{Treatment Effect}} +
    \underbrace{\mathbb{E}[\varepsilon_i | D_i = 1]
      -\mathbb{E}[\varepsilon_i | D_i = 0]}_{\text{Selection Bias}}
\end{align*}
Therefore, we see that the term labeled ``Selection Bias'' will equal
zero if and only if the error terms and the treatment are uncorrelated.
This is equivalent to our earlier definition of Selection Bias if we use
the definition of the error term in (\ref{basic}) to write out
\begin{align*}
  \mathbb{E}[\varepsilon_i | D_i = 1] -\mathbb{E}[\varepsilon_i | D_i = 0]
  &=
  \left(\mathbb{E}[Y_{0i} | D_i = 1] - \mathbb{E}Y_{i0}\right)
  -\left(\mathbb{E}[Y_{0i} | D_i = 0] - \mathbb{E}Y_{i0} \right)\\
  &=
  \mathbb{E}[Y_{0i} | D_i = 1] -\mathbb{E}[Y_{0i} | D_i = 0]
\end{align*}
which is precisely our previous definition of selection bias in
(\ref{bias}).
\end{proof}

Takeaway: If the error terms and treatment status are correlated,
\emph{you're missing something}. Mechanically, the error term is simply
absorbing some fundamental difference in potential outcomes (we just
showed this), suggesting you have an incomplete model of the
relationship between treatment and outcomes. Often, the cure is to include additional controls and estimate the longer regresion
\begin{align}
  Y_i =
  \underbrace{\alpha}_{\mathbb{E}Y_i}
  + \underbrace{X_i' \beta}_{\text{Controls}}
  + \underbrace{\rho}_{Y_{1i} - Y_{0i}} D_i
  + \underbrace{\varepsilon_i}_{Y_{0i}-\mathbb{E}Y_{0i}}
  \label{longreg}
\end{align}


%\tableofcontents




%% APPPENDIX %%

% \appendix




\end{document}


%%%%%%%%%%%%%%%%%%%%%%%%%%%%%%%%%%%%%%%%%%%%%%%%%%%%%%%%%%%%%%%%%%%%%%%%
%%%%%%%%%%%%%%%%%%%%%%%%%%%%%%%%%%%%%%%%%%%%%%%%%%%%%%%%%%%%%%%%%%%%%%%%
%%%%%%%%%%%%%%%%%%%%%%%%%%%%%%%%%%%%%%%%%%%%%%%%%%%%%%%%%%%%%%%%%%%%%%%%

%%%% SAMPLE CODE %%%%%%%%%%%%%%%%%%%%%%%%%%%%%%%%%%%%%%

    %% VIEW LAYOUT %%

        \layout

    %% LANDSCAPE PAGE %%

        \begin{landscape}
        \end{landscape}

    %% BIBLIOGRAPHIES %%

        \cite{LabelInSourcesFile}  %Use in text; cites
        \citep{LabelInSourcesFile} %Use in text; cites in parens

        \nocite{LabelInSourceFile} % Includes in refs w/o specific citation
        \bibliographystyle{apalike}  % Or some other style

        % To ditch the ``References'' header
        \begingroup
        \renewcommand{\section}[2]{}
        \endgroup

        \bibliography{sources} % where sources.bib has all the citation info

    %% SPACING %%

        \vspace{1in}
        \hspace{1in}


    %% INCLUDING PDF PAGE %%

        \includepdf{file.pdf}


    %% INCLUDING CODE %%

        \verbatiminput{file.ext}
            %   Includes verbatim text from the file
        \texttt{text}
            %   Renders text in courier, or code-like, font

        \matlabcode{file.m}
            %   Includes Matlab code with colors and line numbers


    %% INCLUDING FIGURES %%

        % Basic Figure with size scaling
            \begin{figure}[h!]
               \centering
               \includegraphics[scale=1]{file.pdf}
            \end{figure}

        % Basic Figure with specific height
            \begin{figure}[h!]
               \centering
               \includegraphics[height=5in, width=5in]{file.pdf}
            \end{figure}

        % Figure with cropping, where the order for trimming is  L, B, R, T
            \begin{figure}
               \centering
               \includegraphics[trim={1cm, 1cm, 1cm, 1cm}, clip]{file.pdf}
            \end{figure}


        % Side by Side figures
            \begin{figure}[h!]
                \centering
                \mbox{\subfigure{
                    \includegraphics[scale=1]{file1.pdf}
                }\quad\subfigure{
                    \includegraphics[scale=1]{file2.pdf}
                }
                }
            \end{figure}


