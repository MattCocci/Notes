\documentclass[12pt]{article}

\author{Matthew D. Cocci}
\title{Regression}
\date{\today}

%% Formatting & Spacing %%%%%%%%%%%%%%%%%%%%%%%%%%%%%%%%%%%%

%\usepackage[top=1in, bottom=1in, left=1in, right=1in]{geometry} % most detailed page formatting control
\usepackage{fullpage} % Simpler than using the geometry package; std effect
\usepackage{setspace}
%\onehalfspacing
\usepackage{microtype}

%% Formatting %%%%%%%%%%%%%%%%%%%%%%%%%%%%%%%%%%%%%%%%%%%%%

%\usepackage[margin=1in]{geometry}
    %   Adjust the margins with geometry package
%\usepackage{pdflscape}
    %   Allows landscape pages
%\usepackage{layout}
    %   Allows plotting of picture of formatting



%% Header %%%%%%%%%%%%%%%%%%%%%%%%%%%%%%%%%%%%%%%%%%%%%%%%%

%\usepackage{fancyhdr}
%\pagestyle{fancy}
%\lhead{}
%\rhead{}
%\chead{}
%\setlength{\headheight}{15.2pt}
    %   Make the header bigger to avoid overlap

%\fancyhf{}
    %   Erase header settings

%\renewcommand{\headrulewidth}{0.3pt}
    %   Width of the line

%\setlength{\headsep}{0.2in}
    %   Distance from line to text


%% Mathematics Related %%%%%%%%%%%%%%%%%%%%%%%%%%%%%%%%%%%

\usepackage{amsmath}
\usepackage{amsfonts}
\usepackage{mathrsfs}
\usepackage{amsthm} %allows for labeling of theorems
\theoremstyle{plain}
\newtheorem{thm}{Theorem}[section]
\newtheorem{lem}[thm]{Lemma}
\newtheorem{prop}[thm]{Proposition}
\newtheorem{cor}[thm]{Corollary}

\theoremstyle{definition}
\newtheorem{defn}[thm]{Definition}
\newtheorem{ex}[thm]{Example}

\theoremstyle{remark}
\newtheorem*{rem}{Remark}
\newtheorem*{note}{Note}

% Below supports left-right alignment in matrices so the negative
% signs don't look bad
\makeatletter
\renewcommand*\env@matrix[1][c]{\hskip -\arraycolsep
  \let\@ifnextchar\new@ifnextchar
  \array{*\c@MaxMatrixCols #1}}
\makeatother


%% Font Choices %%%%%%%%%%%%%%%%%%%%%%%%%%%%%%%%%%%%%%%%%

\usepackage[T1]{fontenc}
\usepackage{lmodern}
\usepackage[utf8]{inputenc}
%\usepackage{blindtext}


%% Figures %%%%%%%%%%%%%%%%%%%%%%%%%%%%%%%%%%%%%%%%%%%%%%

\usepackage{graphicx}
\usepackage{subfigure}
    %   For plotting multiple figures at once
%\graphicspath{ {Directory/} }
    %   Set a directory for where to look for figures


%% Hyperlinks %%%%%%%%%%%%%%%%%%%%%%%%%%%%%%%%%%%%%%%%%%%%
\usepackage{hyperref}
\hypersetup{
    colorlinks,
        %   This colors the links themselves, not boxes
    citecolor=black,
        %   Everything here and below changes link colors
    filecolor=black,
    linkcolor=black,
    urlcolor=black
}

%% Including Code %%%%%%%%%%%%%%%%%%%%%%%%%%%%%%%%%%%%%%%

\usepackage{verbatim}
    %   For including verbatim code from files, no colors

\usepackage{listings}
\usepackage{color}
\definecolor{mygreen}{RGB}{28,172,0}
\definecolor{mylilas}{RGB}{170,55,241}
\newcommand{\matlabcode}[1]{%
    \lstset{language=Matlab,%
        basicstyle=\footnotesize,%
        breaklines=true,%
        morekeywords={matlab2tikz},%
        keywordstyle=\color{blue},%
        morekeywords=[2]{1}, keywordstyle=[2]{\color{black}},%
        identifierstyle=\color{black},%
        stringstyle=\color{mylilas},%
        commentstyle=\color{mygreen},%
        showstringspaces=false,%
            %   Without this there will be a symbol in
            %   the places where there is a space
        numbers=left,%
        numberstyle={\tiny \color{black}},%
            %   Size of the numbers
        numbersep=9pt,%
            %   Defines how far the numbers are from the text
        emph=[1]{for,end,break,switch,case},emphstyle=[1]\color{red},%
            %   Some words to emphasise
    }%
    \lstinputlisting{#1}
}
    %   For including Matlab code from .m file with colors,
    %   line numbering, etc.

%% Bibliographies %%%%%%%%%%%%%%%%%%%%%%%%%%%%%%%%%%%%

%\usepackage{natbib}
    %---For bibliographies
%\setlength{\bibsep}{3pt} % Set how far apart bibentries are

%% Misc %%%%%%%%%%%%%%%%%%%%%%%%%%%%%%%%%%%%%%%%%%%%%%

\usepackage{enumitem}
    %   Has to do with enumeration
\usepackage{appendix}
%\usepackage{natbib}
    %   For bibliographies
\usepackage{pdfpages}
    %   For including whole pdf pages as a page in doc


%% User Defined %%%%%%%%%%%%%%%%%%%%%%%%%%%%%%%%%%%%%%%%%%

%\newcommand{\nameofcmd}{Text to display}
\newcommand*{\Chi}{\mbox{\large$\chi$}} %big chi
    %   Bigger Chi



%%%%%%%%%%%%%%%%%%%%%%%%%%%%%%%%%%%%%%%%%%%%%%%%%%%%%%%%%%%%%%%%%%%%%%%%
%% BODY %%%%%%%%%%%%%%%%%%%%%%%%%%%%%%%%%%%%%%%%%%%%%%%%%%%%%%%%%%%%%%%%
%%%%%%%%%%%%%%%%%%%%%%%%%%%%%%%%%%%%%%%%%%%%%%%%%%%%%%%%%%%%%%%%%%%%%%%%


\begin{document}
\maketitle

\section{Basic Setup}

\subsection{Potential Outcomes}

Suppose that we want to consider outcome $Y_i$ for individual $i$.
Let $D_i$ denote the treatment for this individual, where
\begin{align*}
  D_i =
  \begin{cases}
    1 & \text{Individual $i$ receives treatment} \\
    0 & \text{otherwise} \\
  \end{cases}
\end{align*}
Before you know whether this person received the treatment, you could
imagine the two \emph{potential outcomes} for individual $i$:
\begin{align*}
  Y_{1i}
  &= \text{Outcome if she receives the treatment ($D_i=1$)} \\
  Y_{0i}
  &= \text{Outcome if she does \emph{not} receive the treatment ($D_i=0$)}
\end{align*}
In practice, we will only observe one of these cases; the other is just
a hypothetical counterfactual. But it helps us to pin down concepts and
write the true, observed outcome $Y_i$ as a function of the treatment
and potential outcomes:
\begin{align}
  Y_i = D_i Y_{1i} + (1 - D_i) Y_{0i}
  \label{potential}
\end{align}

\subsection{Selection Bias}

To measure the effect of some treatment, we might examine the difference
in outcomes that we observe, comparing sample estimates of
\begin{align*}
  \mathbb{E}[Y_i | D_i = 1] - \mathbb{E}[Y_i | D_i = 0]
\end{align*}
But by writing $Y_i$ out in terms of potential outcomes as in
Equation~\ref{potential}, we see the problem:
\begin{align*}
  \underbrace{\mathbb{E}[Y_i | D_i = 1]
  - \mathbb{E}[Y_i | D_i = 0]}_{\text{Observed Effect}}
  &=
  \mathbb{E}[Y_{1i} | D_i = 1]
  - \mathbb{E}[Y_{0i} | D_i = 0] \\
  &=
  \mathbb{E}[Y_{1i} | D_i = 1] - \mathbb{E}[Y_{0i} | D_i = 1]
  + \mathbb{E}[Y_{0i} | D_i = 1] - \mathbb{E}[Y_{0i} | D_i = 0] \\
  &=
  \underbrace{\mathbb{E}[Y_{1i} - Y_{0i}| D_i = 1]}_{%
    \text{Treatment Effect}}
  + \underbrace{\mathbb{E}[Y_{0i} | D_i = 1]
  - \mathbb{E}[Y_{0i} | D_i = 0]}_{\text{Selection Bias}}
\end{align*}
In words, the observed effect confounds two things:
\begin{enumerate}
  \item \emph{Selection Bias}: This pollutes our estimate. Selection
    bias introduces into the observed effect the additional complication
    that people who receive the treatment and who don't might
    \emph{fundamentally} be different types of people. Examples:
    \begin{enumerate}
      \item People in the hospital are (tautologically) sicker than the
        average person---that's why they're in the hospital. So don't be
        surprised if they actually end up having \emph{worse} health
        outcomes, despite their hospital stints, than the healthy people
        visiting them.
      \item Kids who get into elite schools would probably have made
        more money anyway, even without an elite-school salary bump.
    \end{enumerate}
    The list goes on and on.
  \item \emph{Treatment Effect}: This is genuine comparison of
    counterfactuals, and really what we want to measure. It represents
    the difference in potential outcomes under the treatment for
    individual $i$. Returning to the examples above:
    \begin{enumerate}
      \item How much better off is a person for getting stitches, rather
        than trying to walk it off?
      \item How much do the connections and peer effects of Harvard
        translate into more money for your son or daughter, relative to
        four (plus maybe super-senior five) years at Party State? Is it
        worth sacrificing better keg stand form?
    \end{enumerate}
\end{enumerate}

\begin{defn}
Selection Bias is \emph{correlation between treatment and
  \textbf{potential} outcomes}.
\end{defn}
Selection bias is students with high potential being able to influence
whether they get into an elite university and get ``the Harvard
treatment'' and likely salary bump. Selection bias is smart children who

will invest time in their kids' education ensuring that their children
have receive ``the small class size'' treatment. Selection bias is a
real son of a bitch.

Often we hope or expect there to be
correlation between treatment and outcomes (sans the ``potential''
quantifier). We expect smaller class sizes to influence grades. But if smaller classes (the treatment) is correlated with the 

But if people make choices and 
, such that the selection bias term
\begin{align*}
  \text{Selection Bias} =
  \mathbb{E}[Y_{0i} | D_i = 1] - \mathbb{E}[Y_{0i} | D_i = 0]
\end{align*}
does not cancel out

Start with the most basic regression setup, where we try to estimate the
effect $\rho$ of some treatment $D_i$ on individual $i$:
\begin{align}
  Y_i =
  \underbrace{\alpha}_{\mathbb{E}Y_i}
  + \underbrace{\rho}_{Y_{1i} - Y_{0i}} D_i
  + \underbrace{\varepsilon_i}_{Y_{0i}-\mathbb{E}Y_{0i}}
  \label{basic}
\end{align}

%\tableofcontents




%% APPPENDIX %%

% \appendix




\end{document}


%%%%%%%%%%%%%%%%%%%%%%%%%%%%%%%%%%%%%%%%%%%%%%%%%%%%%%%%%%%%%%%%%%%%%%%%
%%%%%%%%%%%%%%%%%%%%%%%%%%%%%%%%%%%%%%%%%%%%%%%%%%%%%%%%%%%%%%%%%%%%%%%%
%%%%%%%%%%%%%%%%%%%%%%%%%%%%%%%%%%%%%%%%%%%%%%%%%%%%%%%%%%%%%%%%%%%%%%%%

%%%% SAMPLE CODE %%%%%%%%%%%%%%%%%%%%%%%%%%%%%%%%%%%%%%

    %% VIEW LAYOUT %%

        \layout

    %% LANDSCAPE PAGE %%

        \begin{landscape}
        \end{landscape}

    %% BIBLIOGRAPHIES %%

        \cite{LabelInSourcesFile}  %Use in text; cites
        \citep{LabelInSourcesFile} %Use in text; cites in parens

        \nocite{LabelInSourceFile} % Includes in refs w/o specific citation
        \bibliographystyle{apalike}  % Or some other style

        % To ditch the ``References'' header
        \begingroup
        \renewcommand{\section}[2]{}
        \endgroup

        \bibliography{sources} % where sources.bib has all the citation info

    %% SPACING %%

        \vspace{1in}
        \hspace{1in}


    %% INCLUDING PDF PAGE %%

        \includepdf{file.pdf}


    %% INCLUDING CODE %%

        \verbatiminput{file.ext}
            %   Includes verbatim text from the file
        \texttt{text}
            %   Renders text in courier, or code-like, font

        \matlabcode{file.m}
            %   Includes Matlab code with colors and line numbers


    %% INCLUDING FIGURES %%

        % Basic Figure with size scaling
            \begin{figure}[h!]
               \centering
               \includegraphics[scale=1]{file.pdf}
            \end{figure}

        % Basic Figure with specific height
            \begin{figure}[h!]
               \centering
               \includegraphics[height=5in, width=5in]{file.pdf}
            \end{figure}

        % Figure with cropping, where the order for trimming is  L, B, R, T
            \begin{figure}
               \centering
               \includegraphics[trim={1cm, 1cm, 1cm, 1cm}, clip]{file.pdf}
            \end{figure}


        % Side by Side figures
            \begin{figure}[h!]
                \centering
                \mbox{\subfigure{
                    \includegraphics[scale=1]{file1.pdf}
                }\quad\subfigure{
                    \includegraphics[scale=1]{file2.pdf}
                }
                }
            \end{figure}


