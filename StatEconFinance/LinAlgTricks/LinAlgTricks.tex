\documentclass[a4paper,12pt]{scrartcl}

\author{Matthew Cocci}
\title{Useful Linear Algebra Tricks for Statistics}
\date{\today}
\usepackage{enumitem} %Has to do with enumeration	
\usepackage{amsfonts}
\usepackage{amsmath}
\usepackage{amsthm} %allows for labeling of theorems
\usepackage[T1]{fontenc}
\usepackage[utf8]{inputenc}
\usepackage{blindtext}
\usepackage{graphicx}
%\usepackage[hidelinks]{hyperref} % For internal/external linking. 
				 % [hidelinks] removes boxes
% \usepackage{url} % allows for url display, non-clickable
%\numberwithin{equation}{section} 
   % This labels the equations in relation to the sections 
      % rather than other equations
%\numberwithin{equation}{subsection} %This labels relative to subsections
\newtheorem{thm}{Theorem}[section]
\newtheorem{lem}[thm]{Lemma}
\newtheorem{prop}[thm]{Proposition}
\newtheorem{cor}[thm]{Corollary}
\setkomafont{disposition}{\normalfont\bfseries}
\usepackage{appendix}
\usepackage{subfigure} % For plotting multiple figures at once
\usepackage{verbatim} % for including verbatim code from a file
\usepackage{natbib} % for bibliographies

\begin{document}
\maketitle

\section{Definitions}

Suppose we have two matrices, $A$ which is $m \times n$ and $B$
which is $p\times q$. Then the {\sl Kronecker Product} of $A$ and $B$ is 
    \[ A \otimes B = \begin{pmatrix} a_{11} B & \cdots & a_{1n} B \\
			    \vdots & \ddots & \vdots \\
			    a_{m1} B & \cdots & a_{mn}B \end{pmatrix}
    \]
which implies that the new matrix is $(mp) \times (nq)$. 
\\
\\
Next, the {\sl vec operator} takes any matrix $A$ that is $m \times n$
and stacks to columns on top of each other (left to right) to 
form a column vector of length $mn$.  Supposing that $a_i$ are column
vectors to simplify notation:
\begin{align*}
    \text{if } A &= \begin{pmatrix} a_1 & \cdots & a_n \end{pmatrix}
	\qquad a_i \in \mathbb{R}^{n\times 1} \\
    \text{then } \mathbf{vec} A &= 
	\begin{pmatrix} a_1 \\ \vdots \\ a_n \end{pmatrix}
\end{align*}

\newpage
\section{Properties with Proofs, Kronecker Product}

\paragraph{Property 1} Let $A$ be $m\times n$, $B$ be $p \times q$,
$C$ be $n\times r$, and $D$ be $q \times s$. Then
\begin{equation}
    (A \otimes B)(C \otimes D) = AC \otimes BD
\end{equation}
\begin{proof} We start by writing:

\begin{align*}
    (A \otimes B)(C \otimes D) = 
    \begin{pmatrix} a_{11} B & \cdots & a_{1n} B \\
	\vdots & \ddots & \vdots \\
	a_{m1} B & \cdots & a_{mn}B \end{pmatrix}
    \begin{pmatrix} c_{11} D & \cdots & c_{1r} D \\
	\vdots & \ddots & \vdots \\
	c_{n1} D & \cdots & c_{nr}D \end{pmatrix}
\end{align*}
Since the matrix $D$ has the same number of rows as $B$ has columns,
we can carry out the multiplication to get

\end{proof}



\end{document}




%%%% INCLUDING FIGURES %%%%%%%%%%%%%%%%%%%%%%%%%%%%

   % H indicates here 
   %\begin{figure}[h!]
   %   \centering
   %   \includegraphics[scale=1]{file.pdf}
   %\end{figure}

%   \begin{figure}[h!]
%      \centering
%      \mbox{
%	 \subfigure{
%	    \includegraphics[scale=1]{file1.pdf}
%	 }\quad
%	 \subfigure{
%	    \includegraphics[scale=1]{file2.pdf} 
%	 }
%      }
%   \end{figure}
 

%%%%% Including Code %%%%%%%%%%%%%%%%%%%%%5
% \verbatiminput{file.ext}    % Includes verbatim text from the file
% \texttt{text}	  % includes text in courier, or code-like, font
