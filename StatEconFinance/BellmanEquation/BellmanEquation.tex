\documentclass[a4paper,12pt]{scrartcl}

\author{Matthew Cocci}
\title{Bellman Equation}
\date{\today}
\usepackage{fullpage}
\usepackage{enumitem} %Has to do with enumeration	
\usepackage{amsfonts}
\usepackage{amsmath}
\usepackage{amsthm} %allows for labeling of theorems
\usepackage[T1]{fontenc}
\usepackage[utf8]{inputenc}
\usepackage{blindtext}
\usepackage{graphicx}
\usepackage{hyperref} 
\hypersetup{	
    colorlinks,		% This colors the links themselves, not boxes
    citecolor=black,	% Everything below changes the link colors
    filecolor=black,
    linkcolor=black,
    urlcolor=black
}
%\numberwithin{equation}{section} 
   % This labels the equations in relation to the sections 
      % rather than other equations
%\numberwithin{equation}{subsection} %This labels relative to subsections
\newtheorem{thm}{Theorem}[section]
\newtheorem{lem}[thm]{Lemma}
\newtheorem{prop}[thm]{Proposition}
\newtheorem{cor}[thm]{Corollary}
\setkomafont{disposition}{\normalfont\bfseries}
\usepackage{appendix}
\usepackage{subfigure} % For plotting multiple figures at once
\usepackage{verbatim} % for including verbatim code from a file
\usepackage{natbib} % for bibliographies

\begin{document}
\maketitle

%\tableofcontents %adds it here


\section{Introduction}

The shortest path problem has the following features:
\begin{enumerate}

    \item You want to get from point $1$ to $N$.

    \item There are a number of possible nodes in between $1$ and $N$,
	permitting a number of different possible paths you might take.

    \item The paths from node to node have different costs.

\end{enumerate}

Therefore, we want a general solution to the problem that
provides us with the \emph{least cost} path to travel from
$1$ to $N$.


\section{The Bellman Equation}

Suppose that at every single node, $i \in \{1, \ldots, N\}$, we \emph{knew}
the least-cost path to get from node $i$ to node $N$. Of course we don't,
unless we're trivially already at node $N$, but just suspend disbelief for
a second and suppose we do.

Now since we know the least-cost paths, we know the ``best-case'' cost to
get from node $i$ to $N$, for all $i$. Denote that ``best-case'' cost
by $J(i)$. Now it makes sense that this function
$J$, for each $i$, should also satisfy what's called the
\emph{Bellman Equation}:
\begin{equation}
    \label{bellman}
    J(i) = \min_{j \in F_i} \left\{ c(i, j) + J(j) \right\}
\end{equation}
In words, this equation says
\begin{enumerate}
    \item To find the best-case cost to go from $i$
	to $N$, consider the set of nodes that you can reach from node $i$,
	denoted $F_i$.

    \item Next, consder the cost of jumping from node $i$ to node $j$,
	denoted $c(i,j)$.

    \item Finally, since you know the best-case cost,
	$J(j)$ for $j \in F_i$, pick the minimum of $c(i,j)$ plus $J(j)$.
\end{enumerate} 

The best solution to our problem must, therefore, satisfy Equation
\ref{bellman}.

\section{Solving for $J$}

We now detail the standard algorithm to find $J$.
\begin{enumerate}
    \item For some large $M$, set
	\begin{equation}
	    J_0(i) = \begin{cases} M & i \neq N \\ 
				    0 & \text{otherwise}
		    \end{cases}
	\end{equation}

    \item Next, set 
	$J_{n+1}(i) = \min_{j \in F_i} \left\{ c(i,j) + J_n(j)\right\}$
	for all $i$.

    \item If $J_{n+1} \neq J_n$, increment $n$ and return to step 2.

\end{enumerate}
The solution thus propagates back from the destination node, $N$.




%%%% APPPENDIX %%%%%%%%%%%

% \appendix

%\cite{LabelInSourcesFile} 
%\citep{LabelInSourcesFile} Cites in parens
%\nocite{LabelInSourceFile} includes in refs w/o specific citation
%\bibliographystyle{apalike} 
%\bibliography{sources.bib} where sources.bib is file




\end{document}



%%%% INCLUDING FIGURES %%%%%%%%%%%%%%%%%%%%%%%%%%%%

   % H indicates here 
   %\begin{figure}[h!]
   %   \centering
   %   \includegraphics[scale=1]{file.pdf}
   %\end{figure}

%   \begin{figure}[h!]
%      \centering
%      \mbox{
%	 \subfigure{
%	    \includegraphics[scale=1]{file1.pdf}
%	 }\quad
%	 \subfigure{
%	    \includegraphics[scale=1]{file2.pdf} 
%	 }
%      }
%   \end{figure}
 

%%%%% Including Code %%%%%%%%%%%%%%%%%%%%%5
% \verbatiminput{file.ext}    % Includes verbatim text from the file
% \texttt{text}	  % includes text in courier, or code-like, font
