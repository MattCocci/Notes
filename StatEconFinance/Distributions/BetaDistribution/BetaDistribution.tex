\documentclass[a4paper,12pt]{scrartcl}

\author{}
\title{The Beta Distribution}
\date{}
\usepackage{enumitem} %Has to do with enumeration	
\usepackage{amsfonts}
\usepackage{amsmath}
\usepackage{amsthm} %allows for labeling of theorems
\usepackage[T1]{fontenc}
\usepackage[utf8]{inputenc}
\usepackage{blindtext}
\usepackage{graphicx}
\usepackage[hidelinks]{hyperref}
%\usepackage{url}
%\numberwithin{equation}{section} 
%, This labels the equations in relation to the sections rather than other equations
%\numberwithin{equation}{subsection} %This labels relative to subsections
%\newtheorem{thm}{Theorem}[section]
%\newtheorem{lem}[thm]{Lemma}
%\newtheorem{prop}[thm]{Proposition}
%\newtheorem{cor}[thm]{Corollary}
\setkomafont{disposition}{\normalfont\bfseries}

% Steele's
\newtheorem{Corollary}{Corollary}
\newtheorem{Proposition}{Proposition}
\newtheorem{Lemma}{Lemma}
\newtheorem{Definition}{Definition}
\newtheorem{Theorem}{Theorem}
\newtheorem{Example}{Example}



\begin{document}

\begin{center} \bf \LARGE
   The Beta Distribution
\end{center}

\section{Introduction}
One particularly useful random variable is the Beta Distribution, which
model proportions relatively well, as it only takes values between
$0$ and $1$, and which also retains the uniform distribution as a special
case.

\section{Density Function}

A random variable $Y$ has a \emph{beta probability distribution} if 
and only if it has density function
\begin{equation}
   \label{pdf}
   f_Y(y) = \begin{cases} \frac{y^{\alpha -1} (1-y)^{\beta-1}}{B(\alpha,
      \beta)}, & 0\leq y \leq 1 \\
	 0, & \text{otherwise}
   \end{cases} \qquad \alpha, \beta > 0 
\end{equation}
where the function $B$ is defined
   \[ B(\alpha, \beta) = \int^1_0 y^{\alpha-1}(1-y)^{\beta-1} \; dy =
      \frac{\Gamma(\alpha)\Gamma(\beta)}{\Gamma(\alpha + \beta)} \]
Varying $\alpha$ and $\beta$ can lead to a vast array of different 
shapes.

\section{Key Statistics}

By a few easy manipulations, it can be shown that the beta distribution
has mean and variance
\begin{equation}
   \label{beta}
    \mu = \frac{\alpha}{\alpha+\beta}, \qquad \sigma^2 = 
      \frac{\alpha\beta}{(\alpha+\beta)^2 (\alpha+\beta+1)} 
\end{equation}

\section{The Beta Function}

The Beta Function can be defined as 
\[ B(\alpha, \beta) = \frac{\Gamma(\alpha) \Gamma(\beta)}{\Gamma(\alpha
   + \beta)} \]
This happens to have several integral representations, two of which
we list:
\begin{align}
   B(\alpha, \beta) &= \int^{\infty}_0 t^{\alpha-1} (1+t)^{-(\alpha+
   \beta)} \; dt \label{int1} \\
   B(\alpha, \beta) = \int^1_0 t^{x-1} (1-t)^{y-1} dt \label{int2}
\end{align}
The proof\footnote{All proofs and further information can be found
   on the Statlect.com website: 
   \url{http://www.statlect.com/subon2/betfun1.htm}. }
that Equation \ref{int1} does indeed equal that of \ref{int1}
requires a bit of clever manipulation, while the Equation \ref{int2} uses
Equation \ref{int1} and makes the substitution 
   \[ s = \frac{t}{1+t} = 1-\frac{1}{1+t}, \qquad \Rightarrow
      t = \frac{s}{1-s} \]




\end{document}

