\documentclass[a4paper,12pt]{scrartcl}

\author{Matthew Cocci}
\title{Notes to Financial Engineering: \\Credit Derivatives}
\date{}
\usepackage{enumitem} %Has to do with enumeration	
\usepackage{amsfonts}
\usepackage{amsmath}
\usepackage{amsthm} %allows for labeling of theorems
\usepackage[T1]{fontenc}
\usepackage[utf8]{inputenc}
\usepackage{blindtext}
\usepackage{graphicx}
\usepackage[hidelinks]{hyperref} 
%\numberwithin{equation}{section} 
%, This labels the equations in relation to the sections rather than other equations
%\numberwithin{equation}{subsection} %This labels relative to subsections
\newtheorem{thm}{Theorem}[section]
\newtheorem{lem}[thm]{Lemma}
\newtheorem{prop}[thm]{Proposition}
\newtheorem{cor}[thm]{Corollary}
\setkomafont{disposition}{\normalfont\bfseries}
\usepackage{appendix}
\usepackage{subfigure} % For plotting multiple figures at once




\begin{document}
\maketitle

\tableofcontents

\newpage
\section{The Fundamental Theorem of Asset Pricing}

\subsection{Introduction}

Derivatives require special pricing techniques aside from the traditional
discounted cash flow (DCF) approach, as DCF requires an estimate of the
appropriate risk-adjusted rate of return.  However, the risk of a 
derivative varies over time, which makes it difficult to estimate the
derivative's risk-adjusted return.

As a result, derivatives pricing turns to the no-arbitrage approach (NA),
which eliminates the need to build risk into the model.  

\subsection{Trading Strategy and Derivative Pricing Definitions}

A \textbf{trading strategy} is a dynamically-rebalanced portfolio.
\\
\\
A trading strategy is \textbf{self-financing} if it generates no 
intermediate cash inflows and requires no intermediate outflows between
the time the portfolio is initiated and the time it is liquidated.  This
implies that
   \begin{itemize}
      \item[i.]{All dividends are reinvestd.}
      \item[ii.]{Value of the assets sold at a rebalance time must 
	 equal the value of the assets bought.}
   \end{itemize}
A trading strategy is \textbf{strictly positive} if the value of the 
traded portfolio can never become zero or negative.
\\
\\ 
Let $N$ be the value of a \emph{strictly positive}, \emph{self-financing}
trading strategy.  Then $N$ is a \textbf{numeraire process} or, simply,
a \textbf{numeraire}. Here are a few examples:
\begin{itemize}
   \item[-]{Price of Dividend paying asset: 
	 NO, as there are intermediate 
	 cash outflows, violating self-financing condition.}
   \item[-]{The price of a forward contract: NO, as it can go negative,
      violating the strictly positive condition.}
   \item[-]{Price of a Foreign Currency: 
      NO, as it is equivalent to a dividend
      paying asset because you think of it as an investment in an
      interest-bearing account.}
   \item[-]{Price of a non-defaultable zero-coupon bond: YES.}
   \item[-]{Value of a money market account earning the risk free rate,
      where there are no interim deposits or withdrawals: YES.}
\end{itemize}

\subsection{Martingales and Change of Measure}

A \textbf{martingale} is a stochastic process
$X$ with the property 
   \[ E_t[X(T) - X(t)] = 0 \; \Leftrightarrow \; E_t[X(T)] = X(t), \qquad
      T > t. \]
A \textbf{probability measure} is a specification
of the probabilities of all the possible states of the words, mapping
states to real numbers.
\\
\\
Suppose that $\xi$ is a nonnegative random variable on $(\Omega,
\mathcal{F}, P)$ with $E_P[\xi] = 1$. (The subscript $P$ highlights that
the last expectation is with respect to measure $P$.) 
Then define a new measure
   \[ Q: \mathcal{F} \rightarrow [0,1] \]
\begin{equation}
   \label{rnt}
   Q(A) = E\left[1_A \xi\right]=\int_A\xi(\omega) dP(\omega), \qquad
      A\in \mathcal{F} 
\end{equation}
Clearly, $Q$ is a probability measure on $(\Omega, \mathcal{F})$ and
it is absolutely continuous with respect to $P$---i.e. we have
   \[Q(A) > 0 \Rightarrow P(A) > 0.\] 
Note that it is common to write the random variable $\xi$ as
   \[ \xi = \frac{dQ}{dP},\]
and we often refer to $\xi$ as the \emph{Radon-Nikodym derivative}
or the \emph{likelihood ratio} of $Q$ with respect to $P$.

\paragraph{Radon-Nikodym Theorem} If $P$ and $Q$ are two probability
measures on $(\Omega, \mathcal{F})$, then there \emph{will exist}
such a random variable $\xi$ so that Expression \ref{rnt} holds.


\newpage
\subsection{Fundamental Theorem of Asset Pricing (No Dividends)}

\subsubsection{Statement of Theorem}

Suppose we have $n$ non-dividend-paying assets with
price processes $S_1, S_2, \ldots, S_n$.  Let $N$ be some numeraire
process.  Then, barring market imperfection, there are no arbitrage
opportunities among these assets if and only if there exists a strictly
positive probability measure $Q_N$ (so it's dependent upon the numeraire,
$N$) under which each of the processes $S_i / N$ is a martingale.
\begin{itemize}
   \item[-]{Note that $S_i/N$ is the price of asset $i$ in units of the
      numeraire $N$.  Therefore, we call $S_i/N$ the \textbf{normalized
      price process}.}
   \item[-]{The probability measure $Q_N$ will, in general depend on
	 the numeraire.  Therefore, we call $Q_N$ the 
	 \textbf{martingale measure} or \textbf{pricing measure} 
	 associated with the numeraire $N$.} 
   \item[-]{We can paraphrase FTAP by saying that, if the is no arbitrage
	 or market imperfections, then given \emph{any} numeraire 
	 process $N$, there must exist a corresponding martingale
	 measure $Q_N$ under which the normalized price of any 
	 non-dividend paying asset is a martingale:
	    \[ \frac{S_i(t)}{N(t)} = E_t^{Q_N}\left[ \frac{S_i(T)}{N(T)}
	       \right], \qquad T > t. \]
	 From this, we see that changing $N$ will generally change 
	 $Q_N$ as well.
      }
\end{itemize}

\subsubsection{Consequences for Derivative Pricing}

Let $V$ denote the price of a derivative with payoff $V(T)$ at time $T$.
Then we can apply FTAP to get
   \[ V(t) = N(t) E_t^{Q_N} \left[ \frac{S_i(T)}{N(T)}
	       \right], \qquad T > t. \]
Note, the price we get for a deriative is \emph{invariant} to the choice
of the numeraire.

\subsubsection{Special Numeraires and Martingale Measures}

\paragraph{T-forward measure} Let $P(t,T)$ be the price of a
non-defaultable zero-coupon bond with unit face value.  Then
   \[ N(t) = P(t,T) \]
is our numeraire.  The associated martingale measure, denoted 
$Q_T$, is called the \emph{T-forward martingale measure}. This yields
a derivative price of 
   \[ V(t) = E_t^{Q_B} \left[ e^{-\int_t^T r(s) ds} V(T)\right] \]

\paragraph{Risk Neutral Measure} Let's consider the value of a money
market account with unit initial value as our numeraire.  Then
   \[ N(t) = B(t) = e^{\int_0^t r(s) ds} \]
where $r$ is the instantaneous risk-free rate.  The associated 
martingale measure, denoted by $Q_B$, is called the 
\emph{risk-neutral martingale measure}. This yields a derivative price
of 
   \[ V(t) = P(t,T) \; E_t^{Q_T}\left[V(T)\right] \; = \;
      e^{-r(t,T) (T-t)} E_t^{Q_T}\left[V(T)\right] \]
   \[ r(t,T) = -\ln{P(t,T)}/(T-t) \]
If interest rates are stochastic (and they probably are), then this
measure isn't as convenient a tthe $T$-forward measure.

\subsection{FTAP for Dividend-Paying Assets}

Consider an asset with price process $S$ and let $D(t)$ denote the
cumulative dividend paid by the asset from time 0 up to time $t$.
We can consider the undiscounted cash flows from holding an asset
from $t$ to $T$:
   \[ S(T) - S(t) + D(T) - D(t) = GP(T) - GP(t) \]
where $GP(t) = S(t) + D(t)$ is the asset's gain process.
\\
\\
Given a numeraire $N$, the asset's \emph{normalized gain process},
denoted NGP, measures the gains from holding the assets in units of $N$:
\[ NGP(t) = \frac{S(t)}{N(t)} + \int^t_0 \frac{dD(s)}{N(s)}\]
where $dD(s)$ is the dividend paid by the asset at time $s$.

\paragraph{Theorem} Now, let's restate the fundamental theorem of asset
pricing allowing for dividend paying assets.  So again, consider
assets with price processes $S_1, \ldots, S_n$ and cumulative dividend
processes $D_1,\ldots,D_n$, letting $N$ be any numeraire process.
Then there are no arbitrage opportunities across these assets if and
only if there exists a strictly positive probability measure $Q_N$ 
under which each
   \[ \frac{S_i(t)}{N(t)} + \int^t_0 \frac{dD_i(s)}{N(s)} \]
is a martingale. This implies
\[ \frac{S_i(t)}{N(t)}  = E_t^{Q_N}\left[ \frac{S_i(T)}{N(T)} +
   \int^T_t \frac{dD_i(s)}{N(s)}\right] \]
And so the normalized price of any asset is equalt to the conditional
expectation under the martingale measure of the assets normalized payoffs
(including future dividends).

\newpage


\section{Continuous Time Stochastic Processes}

Here, we develop the necessary machinery in continuous time stochastic
processes to model asset price evolution properly and with sufficient
richness and generality. In particular, we discuss a heirarchy of model
classes that includes Brownian Motion $\subset$ Generalized Brownian
Motion $\subset$ Diffusions $\subset$ Ito Processes.

\subsection{Introduction}

A \emph{stochastic process} $X$ is a collection of random variables
indexed by time: $X = \{ X_t: t \in \mathcal{T} \}$. 
\begin{itemize}
   \item[-]{\emph{Discrete Time}: $\mathcal{T}$ countable, and process
      changes only at discrete time intervals.}
   \item[-]{\emph{Continuous Time}: $\mathcal{T}$ uncountable.}
\end{itemize}
\paragraph{Definition} A process $X$ has stationary increments if 
$X_T - X_t$ has the same distribution as $X_{T'} - X_{t'}$ provided
that $T-t= T'-t'$.

\subsection{Brownian Motion}

\paragraph{Definition} The most basic continuous-time process is 
\emph{Brownian Motion} (or the \emph{Wiener Process}).  It has three
defining properties:
\begin{enumerate}
   \item[i.]{$W(0) = O$.}
   \item[ii.]{$W(t)$ is continuous, so no jumps.}
   \item[iii.]{Given any two times, $T>t$, the increment $W(T) - W(t)$ is
      independent of all previous history and normally distributed
      with mean $0$ and variance $T-t$.}
\end{enumerate}
A few consequences of the definition of $W(t)$:
\begin{itemize}
   \item[-]{Brownian motion has independent stationary increments.}
   \item[-]{$W(t)$ is normally distributed with $\mu =0$, $\sigma^2 = t$.
      }
\end{itemize}

\subsection{Generalized Brownian Motion}

\paragraph{Definition} A \emph{generalized Brownian motion} is a 
continuous-time process $X$ with the following property:
   \[ X(t) = X(0)+\mu t + \sigma W(t) \]
where $\mu$ is the \emph{drift}, $\sigma$ is the \emph{volatility},
and $W$ is simple Brownian motion. The differential equation 
equivalent is written:
   \[ dX(t) = \mu \; dt + \sigma \; dW(t).\]
It follows immediately from the definition that
\begin{itemize}
   \item[-]{$X(t)$ is continuous, so no jumps.}
   \item[-]{$X(t)$ is normally distributed with mean $X(0)+\mu t$, 
      variance $\sigma^2 t$.
      }
   \item[-]{Given any two times, $T>t$, the increment $X(T) - X(t)$ is
      independent of all previous history and normally distributed
      with mean $\mu(T-t)$ and variance $\sigma^2(T-t)$.}
   \item[-]{$X$ is a martingale if and only if $\mu = 0$.}
\end{itemize}
\paragraph{Theorem} It also happens that Generalized Brownian motions
are the only continuous time processes with continuous sample
paths and stationary increments.

\subsection{Ito Processes}

Even more general than Brownian Motion (which is retained as a special
case), an \emph{Ito Process} is 
a stochastic process $X$ defined by one of two equivalent formulations:
   \[ dX(t) = \mu(t) \; dt + \sigma(t) \; dW(t) \]
   \[ X_t = X_0 + \int^t_0 \mu(s) \; ds + \int^t_0 \sigma(s) \; dW(s)
      \]
for any arbitrary stochastic processes $\mu$ (the drift) and $\sigma$
(the volatility) along with some Brownian Motion $W(t)$.  Here are 
some properties
\begin{itemize}
   \item[-]{Has continuous sample paths and is a martingale if and only
      if $\mu(t) = 0$. }
   \item[-]{Increments are not necessarily stationary, as $\mu$ and
	 $\sigma$ can change \emph{randomly} with time.}
\end{itemize}

\paragraph{Definition} If the drift and volatility of an Ito process 
depend only upon the current value of the process and time, then
$X$ is a \emph{diffusion}.  Mathematically, $X$ is a \emph{diffusion}
if
   \[ dX(t) = \mu(X(t),t)\; dt+ \sigma(X(t),t)\; dW(t) \]
for some functions $\mu$ and $\sigma$.

\subsection{Ito's Lemma}

Suppose that $X$ is an Ito process defined by 
\begin{equation}
   \label{ito}
   dX(t) = \mu(t) \; dt + \sigma(t) \; dW(t)
\end{equation}
and we define a new process $Y(t) = f(X(t),t)$ where $f$ is some
function that's twice differentiable in $X$ and once in $t$.  Then
we have that
\begin{equation}
   \label{lemma}
   dY(t) = f_X(X(t),t) \; dX(t) + f_t(X(t),t) \; dt + \frac{1}{2} 
   f_{XX}(X(t),t)\sigma(t)^2 \; dt.
\end{equation}
where subscripts on $f$ denote the partial derivatives.\footnote{Note 
that Equation \ref{lemma} almost looks like the chain rule from 
traditional calculus, except for that extra term with $f_{XX}$ partial 
derivative. That arises from the additional variability due to the 
inclusion of stochastic factors like $W(t)$ in the original Ito 
Process.} Subbing Equation \ref{ito} into Equation \ref{lemma}, 
we get that
   \[ dY(t) =\left(f_X(X(t),t)\mu(t) + f_t(X(t),t)  + \frac{1}{2} 
      f_{XX}(X(t),t)\sigma(t)^2 \right) \; dt + 
      f_X(X(t),t)\sigma(t) \; dW(t)
   \]
Thus, it is clear that $Y$ is also an Ito Process by the statement
above with the drift and volatility given by the coefficients on
$dt$ and $dW(t)$ as always.

\paragraph{Using Ito's Lemma} In practice, we use Ito's Lemma
whenever we have a (typically complicated) Ito Process that we want
to solve.  Given the process $X$ and its corresponding Ito Process, 
we posit a function $f$ that could could help. Then we write a
new Ito Process using Ito's lemma with $dY(t) = df(X(t),t)$ on the LHS.
From there, hopefully we can integrate $dY(t)$ easily on the left and
solve out for $X(t)$.

\subsection{Multi-dimensional Ito's Lemma}

For the sake of completeness, let's generalize Ito's Lemma to consider
the case of a finite number of Ito processes, $X_1, X_2, \ldots, X_n$,
   \[ dX_i(t) = \mu_i(t)\; dt + \sigma_i(t) \; dW_i(t).\]
Next, define $Y(t) = f(X_1(t), \ldots, X_n(t), t)$ for some 
differentiable function $f$.  Then multi-dimensional Ito's Lemma says
\begin{align*}
    dY(t) &= \sum^n_{i=1} f_{X_i}(X_1(t), \ldots, X_n(t), t) \; 
      dX_i(t) \\
    &+ f_{t}(X_1(t), \ldots, X_n(t), t) \; dt \\
    &+ \frac{1}{2} \sum^n_{i=1}\sum^n_{j=1} 
      f_{X_i,X_j}(X_1(t), \ldots, X_n(t), t) \; \rho_{ij}\sigma_i(t)
      \sigma_j(t) \; dt 
\end{align*}
where $\rho_{ij}$ is the correlation coefficient between $dW_i$ and
$dW_j$. Note that you'll have to plug back in for the $dX_i$ in the
first sum.
\\
\\
We'll mostly consider with the two-dimensional case for the 
two specific instances below:
\begin{itemize}
   \item[i.]{$Y(t) = X_1(t)X_2(t)$, which gives us
      \[ dY(t) = X_2(t) \; dX_1(t) + X_1(t) \; dX_2(t) +
	 \rho_{12}\sigma_1(t)\sigma_2(t) \; dt \]
      Note that you'll have to plug back in for $dX_1$ and $dX_2$.
      This is a type of integration-by-parts formula because (after
      rearranging terms) it relates $X_1\; dX_2$ to $X_2\; dX_1$.
   }

   \item[ii.]{$Y(t) = X_1(t)/X_2(t)$, which gives us
      \begin{align*}
	 dY(t) = \frac{1}{X_2(t)} dX_1(t) - \frac{X_1(t)}{X_2(t)^2}
	 dX_2(t) +  \frac{X_1(t)}{X_2(t)^3} \sigma_2(t)^2 \;dt -
	 \frac{1}{X_2(t)^2} \; \rho_{12}\sigma_1(t)\sigma_2(t) \; dt. 
      \end{align*}
      Note that you'll have to plug back in for $dX_1$ and $dX_2$.
      This is a type of integration-by-parts formula because (after
      rearranging terms) it relates $X_1\; dX_2$ to $X_2\; dX_1$.
      }
\end{itemize}


\subsection{Geometric Brownian Motion}

Let's consider the process $X$ governed by
   \[ dX(t) = X(t)\mu(t) \; dt + X(t) \sigma(t) dW(t).\]
To solve, let us consider the process $Y(t) = \log X(t)$.  We compute
the partials and apply Ito's Lemma:
   \[ f_X = \frac{1}{X(t)}, \qquad f_{XX} = -\frac{1}{X(t)^2},
      \qquad f_t = 0 \]
   \[ dY(t) = \frac{1}{X(t)} dX(t) - \frac{1}{2}\frac{1}{X(t)^2}
      (\sigma(t) X(t))^2 ) \; dt \]
which simplifies (after subbing in for $dX(t)$) into the expression
   \[ dY(t) = \left( \mu(t) - \frac{1}{2}\sigma(t)^2\right) \; +
      \sigma(t) \; dW(t).\]
Next, integrating both sides and substituting back in with
$Y(t) = \log X(t)$, we get
   \[ Y(t) = Y(0) + \int^t_0 \left( \mu(s)-\frac{1}{2}\sigma(s)^2\right) 
      ds + \int^t_0 \sigma(s) \; dW(s) \]
   \[ X(t)=X(0) e^{\int^t_0 \left( \mu(s)-\frac{1}{2}\sigma(s)^2\right) 
       ds + \int^t_0 \sigma(s) \; dW(s)}\]
where it also follows that $X$ is strictly positive.

\paragraph{Special Case} Suppose that $\mu$ and $\sigma$ are constant,
in which case $X$ follows a \emph{geometric Brownian motion}.  Then
it follows that 
   \[ \log X(t) \sim N\left( \log X(0) + \left(\mu - \frac{1}{2} \sigma^2
      t\right), \sigma^2 t \right) \]
so that we say $X(t)$ is \emph{lognormally distributed}.

\subsection{Girsanov's Theorem}

\paragraph{Theorem} Suppose that $X$ is an Ito process
   \[ dX(t) = X(t) \mu(t) \; dt + X(t) \sigma(t) \; dW(t),\]
where $\mu$ an $\sigma$ are stochastic processes and $W$ is Brownian
Motion under some probability measure $P$---like maybe the real 
world measure.  Then if $Q$ is any other strictly positive probability
measure, then $X$ is also an Ito process under $Q$---i.e., there
exist processes $\hat{\mu}$, $\hat{\sigma}$, and $\hat{W}$ with the
property that
\begin{equation}
\label{Girsanov}
    dX(t) = X(t) \hat{\mu}(t) \; dt + X(t) \hat{\sigma}(t) \; 
      d\hat{W}(t).
\end{equation}
Even better, $\hat{\sigma} = \sigma$.
\\
\\
This is particularly useful because, in general, we will have to work 
with two different probability measures: the true/historical $P$ and
the martingale probability measure $Q_N$. 



\newpage
\section{The Market for Credit Derivatives}

\emph{Credit Derivatives} are financial contracts between a buyer and
a seller with payoffs contingent upon credit events affecting
a third party, the \emph{reference credit}.  

The most popular form if credit derivatives is the \emph{Credit Default
Swap} (CDS), which includes as special cases single-name CDS's,
basket CSD's, CDS indices, and CDS index tranches.  While CDS's were
introduced in the early 1990s, CDS indices were introduced in 2003,
and growth exploded between 2001 and the financial crisis in 
2008.\footnote{However, some of the substantial growth in notional 
amounts has been overstated as offsetting trades couldn't properly
cancel each other out prior to subsequent standardization}

\subsection{Single-Name Credit Default Swaps}

In such arrangements, the \emph{protection seller}
agrees to make a payment to the \emph{protection buyer} if a
qualifying credit event affects the third party, \emph{reference entity}
before the maturity of the contract.\footnote{The 
``qualifying credit events'' usually include bankruptcy, default on 
outstanding bonds, and debt restructuring.} 
\\
\\
{\sl Payment Structure:} Payments are as follows.
\begin{itemize}
   \item[-] Should a qualifying credit event occur, the payment
      from protection seller to buyer is the difference between the
      par and post-default makret value of \emph{deliverable obligations}
      issued by the reference entity having total face value equal to
      the notional of the CDS.
   \item[-] In order to get this protection, the protection buyer agrees
      to make a series of quarterly payments until the credit event
      or maturity, whatever comes first.\footnote{Should the credit
      event occur in between payment dates, the buyer still must
      pay the fraction of the next payment that has accrued since
      the last payment date.}
   \item[-] Prior to April 2009, convention dictated that there was
      no upfront payment from CDS buyer to seller. Instead, the
      the buyer just made periodic payments at the quoted annualized
      rate, the \emph{CDS Par Spread}.\footnote{Note that since buying 
      a defaultable bond and a single-name CDS results in a position 
      roughly equivalent to buying a risk-free bond, the CDS
      spread should be approximately equal to the credit spread on the
      defaultable bond.}
   \item[-] Since April 2009, the standard North-American CDS requires
      an up-front payment in addition to standardized quarterly
      coupon payments at a fixed annual rate of 100 basis points for 
      investment grade credits, and 500 basis points for high-yield
      credits.\footnote{Payments due on the 20th of March, June,
      September, and December.} Depending on the risk of the reference
      credit, this means that the upfront payment can either be 
      positive or negative.
\end{itemize}
\newpage
{\sl Market Quotes}: There are two conventions, depending on
the credit quality of the reference entity.
\begin{enumerate}
   \item[-] Quotes for high-yield credits are in terms of the 
      up-front fee.
   \item[-] Quotes for investment-grade credits are in terms of
      a \emph{conventional spread}, in which case the up-front fee
      is the present value of the difference between the quoted spread
      and the coupon rate.
\end{enumerate}
{\sl Settlement}: There are two ways to settle the payment from 
protection seller to buyer should a qualifying credit event occur:
\begin{enumerate}
   \item Physical Delivery: The protection buyer delivers obligations
      with face value equal to the notional amount and receives
      the notional amount.
   \item Cash Settlement: The protection seller pays an amount equal 
      to the notional times \emph{loss given default} (LGD),
      which is defined as 1 minus the \emph{recovery rate}, which is
      the percentage market value of the deliverable obligation. This
      recovery rate is determined through a \emph{credit auction}, 
      which is determined by polling a number of bond dealers.
\end{enumerate}

\subsection{Basket Credit Default Swaps}

Such swaps are equivalent to a portfolio of single-name CDS's. 
They are strctured so that 
\begin{enumerate}
   \item A notional is specified for each reference credit in the 
      basket.
   \item If one of the reference credits experiences a credit event
      priot to maturity, then the protection seller pays 
      the notional for that particular credit times the LGD.
   \item Following a credit event, the affected reference entity is
      removed from the basket and the CDS continues to its original
      maturity, coverin the remaining reference entities in the
      basket, with periodic payments from protection buyer to seller
      computed on the residual notional amount.
\end{enumerate}
{\sl $N$th to Default (N2D) Basket CDS:} These are a variation of the
straight Basket CDS.  Here, the event that triggers the protection
payment is the $N$th credit event affecting a basket of 
reference entities after which the CDS is terminated.  
\begin{itemize}
   \item[-] Cleary, a CDS on a basket of $M$ credits is equivalent to 
      a portfolio with a 1st to Defaults, 2nd to Default, \dots,
      and an $M$th to default CDS on the basket.
   \item[-] The primary advantage of an $N$th to default CDS is that they
      allow a protection buyer to buy only partial protection on a given
      basket of reference entities.
   \item[-] Unlike pricing of straight basket CDS's, the pricing of an
      N2D CDS depedends crucially on the correlation between 
      the default times of the different credits in the basket. In 
      general, the spread on a 1st to Default CDS decreases as 
      correlation increases. The opposite is true for the Last to 
      Default.
\end{itemize}

\newpage
\subsection{CDS Indices} 

CDS Indicees allow investors to buy or sell protection on 
\emph{standardized} baskets of credits.  The main indices
cover different geographical regions, credit quality, and industry
type, and they incude the credit entities
with the most actively traded single-name CDS's in a given segment.
Overall, an index typically includes 20-125 equally weighted credits.
Here's how they work
\begin{itemize}
   \item Once formed the composition of an index reamins static over
      its lifetime, except in the case of credit events where you drop
      the offending entity from the index.\footnote{Every six months a 
      new index series is launched with updated
      components. Although the older index still trades, focus
      concentrates upon the on-the-run series.}
   \item Each index series specifies a quartly coupon to be paid
      by the buyer to the protection seller.
   \item Indices are quoted on either a spread or price
      basis: 
      \begin{itemize} 
	 \item Price of an index: $100\times[1-(\text{up-front fee})]$. 
	 \item Up-front Fee: Present value of the difference between
	    the quoted spread and coupon.
      \end{itemize}
      Note that the par spread on an index is roughly equal to 
      a weighted average of the par spreads on the individual
      credits in the index.
   \item If one of the reference entities in the index defaults, the
      protection seller must pay the protection seller a value equal to 
      \[ \text{Payment} = (\text{Notional}) \times 
	 (\text{Trade Weight of Credit in Index}) \times LGD \]
      After the default, notional is reduced appropriately, and the
      protection buyer pays a quarterly coupon based on this lower
      notional.
\end{itemize}
{\sl CDS Index Tranches:}  TO FINISH

\newpage
\section{Pricing Credit Derivatives}
\subsection{Survival Probabilities and Hazard Rates}

The price of single-name cDS reflects the probability distribution of 
the time of default under the martingale measure.  So, letting $\tau$
be the random time of default of the reference entity, we have 
\begin{align}
   \text{Cumulative Default Prob.}: \qquad\; G(t,s) &= 
      Q_B(\tau \leq s | I_t)\label{defaulttime}\\
   \text{Cumulative Survival Prob.}:\; \; 1-G(t,s) &= Q_B(\tau > s | I_t)
   \label{survprob}
\end{align}
where Equation \ref{defaulttime} is the risk-neutral CDF for $\tau$
and where Equation \ref{survprob} is also under the risk-neutral measure.
We also note that $G(t,t)=0$. 
\\
\\
{\sl Hazard Rates:} Often, it is convenient to express the probability
distriution of default time time inters of \emph{hazard rates} (or the
\emph{forward default rate}) defined 
\begin{equation}
   h(t,s) = \frac{g(t,s)}{1-G(t,s)}, \qquad g(t,s) = \frac{\partial}{
   \partial s} G(t,s) = \text{Density of $G(t,T)$}
\end{equation}
In other words, $h(t,s)$ is the risk-neutral conditional probability
that, at time $t$, the reference entity will default between
times $s$ and $s + ds$ given that it has survived until time $s$.
We can also express survival probabilities in terms of hazard rates, 
shown by
\begin{align*}
   \frac{\partial}{\partial s} \ln (1-G(t,s)) &= - \frac{g(t,s)}{
   1-G(t,s)} = -h(t,s)\\
   \text{Integrate}\quad \ln (1-G(t,s)) &= -\int^s_t h(t,u) \; du \\
   \Rightarrow 1-G(t,s) &= e^{-\int^s_t h(t,u) \; du} \\
   \text{By Definition:} \quad Q_B(\tau > s | I_t) &= 1-G(t,s) = 
      e^{-\int^s_t h(t,u) \; du}
\end{align*}

\newpage
\subsection{Pricing Defaultable ZCB's}

Now that we defined hazard rates, we can express the price of 
defaultable ZCB's and CDS's in terms of hazard rates. To do so, first
price two more basic credit derivatves. 
\\
\\
{\sl First Security, $\mathcal{P}_0(t,T)$:} This asset pays \$1 at 
time $T$ if $\tau > T$ and nothing otherwise. So this is a zero-recovery
defaultable ZCB. Now if we assume
the time of default is independent of the level of interest rates, then
\begin{align*}
   \mathcal{P}_0(t,T) &= B(t) E_t^{Q_B}\left[ \frac{1_{\{\tau>T\}}}{
      B(T)} \right]\\
   \text{By indepedence} \qquad 
      &= B(t) E_t^{Q_B}\left[ \frac{1}{
      B(T)} \right] E_t^{Q_B}\left[ 1_{\{\tau>T\}}\right]\\
   &= P(t,T)(1-G(t,T))\\
   &= P(t,T) e^{-\int^T_t h(t,u) \; du}
\end{align*}
where $P(t,T)$ is the value of a riskless ZCB. So if we assume 
independence of time of default and the level of rates, the value of 
a zero-recovery defaultable ZCB is the value of a riskless ZCB times
the probability of the issuer not defaulting before maturity. We also
have the nice result, if we expand out $P(t,T)$, that
\begin{align*}
 \mathcal{P}_0(t,T) &= P(t,T) e^{-\int^T_t h(t,u) \; du} =
   E_t^{Q_B} \left[ e^{-\int^T_t r(u) \; du}\right] 
   e^{-\int^T_t h(t,u) \; du} \\
   &= E_t^{Q_B} \left[ e^{-\int^T_t (r(u)+h(t,u)) \; du}\right] 
\end{align*}  
So that the price of zero-recovery ZCB is computing by discounting
the promised payment at the risk free rate \emph{plus} the hazard rate.
\\
\\
{\sl Second Security, $\mathcal{D}(t,T)$:} This asset pays \$1 at time
$\tau$ if $\tau \leq T$, nothing otherwise. Then the value of this
security, again assuming that the level of interest rates and the time
of default are independent, is
\begin{align*}
   \mathcal{D}(t,T) &= B(t) E_t^{Q_B} \left[\frac{1_{\{ t \leq \tau 
   \leq T\}}}{B(\tau)} \right] 
   = B(t) E_t^{Q_B} \left[\int^T_t \frac{1_{\{\tau \in ds\}}}{B(s)} 
      \; ds\right] \\
   &= \int^T_t B(t) E_t^{Q_B} \left[\frac{1}{B(s)} 
      \right] E_t^{Q_B} \left[1_{\{\tau \in ds\}}\right] \\
   &= \int^T_t P(t,s) g(t,s) \; ds \\
   &= \int^T_t P(t,s) h(t,s)e^{-\int^s_t h(t,u)\; du} \; ds \\
\end{align*}
From here, we have have nough to price a defaultable ZCB with random
recovery rate.
\\
\\
{\sl Pricing Defaultable ZCB, Random Recovery:} Assuming that interest
rates, time of default, and recovery value are mutually independent, 
we can express the value at time $t$ of a defaultable ZCB with random
recory, $R$, as 
\begin{align*}
   \mathcal{P}(t,T)&=B(t) E_t^{Q_B} \left[ \frac{1_{ \{\tau>T\}}}{B(T)}
      + R \cdot \frac{ 1_{ \{ t < \tau \leq T\}}}{B(\tau)}\right] \\
   &= \mathcal{P}_0(t,T) + E_t^{Q_B} [R] \mathcal{D}(t,T) \\
\end{align*}
Thus, it's clear that it's possible to recover hazard rates from the
prices of defaultable bonds.

\newpage
\subsection{Pricing Single-Name CDS's}

Suppose we have a CDS with unit notional, tenor $T_n$, coupon dates
$\{ T_1, \ldots, T_n\}$, upfront fee $K_u$, coupon rate $K_c$, and
par spread $K_s$.  Assuming the recovery rate, $R$, is independent of 
the time of default and the level of interes rates, we can price 
\begin{itemize}
   \item[-] {\sl Protection Leg:} This is the time 0 value of a 
      payment equal to the loss given default (LGD) at the time of 
      default:
      \begin{align}
	 \label{protleg}
	 V_{ps}(0)&=B(0) E^{Q_B} \left[ \frac{(1-R)1_{ \{ \tau \leq T_n
	 \} }}{B(\tau)} \right] \notag\\
	 &= (1-E^{Q_B}[R]) B(0) E^{Q_B} \left[\frac{1_{\{\tau \leq T_n
	 \} }}{B(\tau)} \right]\notag\\
	 &= (1-E_t^{Q_B}[R]) \mathcal{D}(0,T_n)
      \end{align}
   \item[-] {\sl Premium Leg:} This is the value at time 0 of the 
      up-front payment plus the value of a series of fixed coupon 
      payments, $K_c(T_i-T_{i-1})$ at each coupon date $T_i$ (provided
      that default has not occured at time $T_i$), plus the fractional
      coupon payment of $K_c(\tau-T_{i-1})$ if default occurs between
      $T_{i-1}$ and $T_i$:
      \begin{align}
	 \label{prem1}
	 V_{pb}(0) &= K_u  + K_c\mathcal{A}(0,T_1,T_n) \\
	 \text{where} \quad \mathcal{A}(0,T_1,T_n) &= 
	    \sum^n_{i=1} (T_i-T_{i-1}) \mathcal{P}_0(0,T_i) +
	    \sum^n_{i=1} \int^{T_i}_{T_{i-1}} (s-T_{i-1}) P(0,s)
	    g(0,s) \; ds \notag
      \end{align}
      If we re-evaluate the premium leg in terms of the spread, rather
      than the up-front fee and coupon, then we get
      \begin{equation}
	 \label{prem2}
	 V_{pb}(0) = K_s \mathcal{A}(0,T_1,T_n) 
      \end{equation}
   \item[-] Since the initial value of a CDS must be 0, we know that
      Equation \ref{protleg}, \ref{prem1}, and \ref{prem2} must
      equal each other. This allows us to deduce 
      \begin{align}
	 K_u &= (K_s - K_c) \mathcal{A}(0,T_1,T_n) \label{ku}\\
	 K_s &= \frac{(1-E_t^{Q_B}[R]) \mathcal{D}(0,T_n)}{ 
	    \mathcal{A}(0,T_1,T_n)}\label{ks}
      \end{align}
      Equation \ref{ku} means that the upfront fee is the PV of the
      difference between the spread and the coupon, while Equation
      \ref{ks} shows that the spread is equal to the  value of the
      protection leg divided by an annuity factor.
\end{itemize}
\newpage
{\sl ISDA CDS Standard Model:} Note that market quotes are typically the
spread, which we must convert to an up-front fee.
By using Equations \ref{protleg}, 
\ref{prem2}, and \ref{ku}, we can compute the upfront for CDS's
that are quoted in terms of spread using the conventions specified by 
the ISDA Model:
\begin{enumerate}
   \item We assume that $E_t^{Q_B}[R]$ is equal to the \emph{conventional
      recovery rate}, $R_c$.\footnote{For single-name CDS's, $R_c=40\%$
      for senior unsecured obligations, $R_c = 20\%$ for subordinated
      obligations, and $R_c=25\%$ for emerging markets (both senior
      and subordinated).} We also also assume that the
      hazard rate curve is flat, so that $h(0,T) = h(0,0)$ for all $T$.
   \item Then, we solve for the value of $h(0,0$ that equalizes 
      Equations \ref{protleg} and \ref{prem2}. That means solving 
      \begin{align*}
	 (1-R_c) &\int^{T_n}_0 P(0,s) h(0,0) e^{-h(0,0)s} \; ds  
	 =  K_s  \sum^n_{i=1} (T_i - T_{i=1}) P(0,T_i)
	    e^{-h(0,0)T_i} \\
	 &\quad + K_s \sum^n_{i=1} \int^{T_i}_{T_{i-1}}
	    (s-T_{i-1}) P(0,s) h(0) e^{-h(0,0)s} \; ds 
      \end{align*}
      Then, we substitute $h(0,s)=h(0,0)$ in Equation \ref{ku} to 
      compute the up-front.
\end{enumerate}
{\sl Alternative Approach:} 
Alternatively, we can take a set of CDS market quotes and an estimated
recovery rate, use Equations \ref{protleg}, \ref{prem1}, and 
\ref{prem2} to bootstrap and infer an implied (piecewise constant)
hazard rate curve, and hence an implied default time 
distribution.\footnote{This is very clearly a departure from the ISDA
CDS Standard Model which assumes a flat hazard rate.}


\newpage
\subsection{Pricing Multi-Name CDS's}

\subsubsection{Background and Motivation}
Straight basket CDS's or CDS indicex spreads can be derived directly
from the spreads for individual single-name CDS's. However, the 
determination of the spreads for N2D CDS's or CDS index tranches is a
bit more complication, for the following reasons:
\begin{enumerate}
   \item The structure of N2D CDS's and CDS index tranches
      require us to specify the joint probability distribution of
      the default times of the reference entities in the basket or index
      under the martingale measure.
   \item In order for multi-name CDS's to be priced consistently with
      single-name CDS's, the assumed joint default distribution must
      have \emph{marginals} that are consistent with the default time
      distributions used for pricing CDS's on the individual 
      reference entities.
\end{enumerate}
Therefore, we will be starting from arbitrary default distributions for
the individual credits, and we'll have to construct a joint default 
distribution consistent with those individual distributions.
\\
\\
The most flexible way to do so uses \textbf{copulas}, whose 
characteristics are detailed in the appendix. But note that pricing
is tremendously sensitive to the choice of copula. It's no trivial
matter and should be taken very seriously.

\subsubsection{Implementation}

Once the individual default probability distributions, $F_i(T) = 
G_i(0,T)$, and a copula, $C$, have been calibrated, the default times
$\tau_i$ on a basket of credits can be simulated as follows:
\begin{enumerate}
   \item Draw random variables $(U_1, \ldots, U_m)$ having joint 
      distribution, $C$.
   \item Set $\tau_i = F_i^{-1}(U_i)$, which sets $\tau_i$ so that 
      \[ U_i = F(\tau_i) = G_i(0,\tau_i) = 1- e^{-\int^{\tau_i}_0
      h_i(0,s)\; ds} \]
      where $h_i$ is the hazard rate for credit $i$.
\end{enumerate}
In the special case that selected copula is Gaussian with base 
correlatoin matrix $\rho$, the random variables $(U_1, \ldots, U_m)$ for
step 1 above can be simulated by 
\begin{enumerate}
   \item Drawing $(X_1, \ldots, X_m)$ from a multivariate standard 
      normal distribution with correlation matrix $\rho$.
   \item Setting $U_i = N(X_i)$.
\end{enumerate}
Once we have simulated default times, we can price multi-name CDS's
via Monte Carlo simulation.

\newpage
{\sl Monte Carlo Pricing of Multi-Name N2D CDS:} Suppose that
the Nth to default CDS has tenor $T_n$ and $m$ credits in the basket.
We simulate the normalized cash flows on the protection leg and on the
annuity, $\mathcal{A}$, as follows:
\begin{enumerate}
   \item Simulate the default times $(\tau_1, \ldots, \tau_m)$ using
      the copula.
   \item Determine the time $\tau$ of the $N$th default.
   \item The normalized cash flow on the protection leg is
      \begin{equation}
	 \label{mcpremleg}
	 \frac{1-R}{B(\tau)} \cdot 1_{ \{\tau \leq T_n \} } 
      \end{equation}
      The value $R$ used here can either be constant or drawn from
      some calibrated distribution.
   \item The normalized cash flow on the annuity is 
      \begin{equation}
	 \label{mcprotleg}
	  \sum^n_{i=1} \frac{T_i - T_{i-1}}{B(T_i)} \cdot
	 1_{ \{\tau \geq T_i \} }  + \sum^n_{i=1} 
	 \frac{\tau-T_{i-1}}{B(\tau)} \cdot 
	 1_{ \{T_{i-1}< \tau <T_i \} }  
      \end{equation}
   \item The mean of the simulated values of the normalized cash flows
      on the premium leg and the annuity are then the Monte Carlo 
      estimates of $V_{ps(0)}$ and $\mathcal{A}(0,T_1,T_n)$,
      respectively.
   \item Given $V_{ps}$ and $\mathcal{A}(0,T_1,T_n)$, the value of 
      the premium leg is
	 \[ V_{ps} = K_s \mathcal{A}(0,T_1,T_n) \]
      while the value of the CDS par spread is 
	 \[ K_s = \frac{V_{ps}(0)}{\mathcal{A}(0,T_1,T_n) } \]
\end{enumerate}

\subsection{More General Credit Derivatives}

We saw that the pricing of a CDS depends on the default time 
distribution under $Q_B$ as perceived at the valuation date, $t$. 
Moreover, this distribution can be expressed easily
in terms of hazard rates. But suppose instead that we wanted to 
price a credit swaption with expiration $T$. Then the value of
this swaption will depend upon the value of the underlying
CDS at $T$, wihch depends upon the hazard rates (and thus the
distribution of default times) at time $T$ under $Q_B$. 

So in order
to price general credit derivatives, we need a model of how the
probability distribution of the time of default (or, equivalently,
hazard rates) evolves over time. Therefore, we need a 
\textbf{Dynamic Credit Risk Model}. 

\newpage
\section{Dynamic Credit Risk Models}

Dynamic Credit Risk Models can come in two flavors:
\begin{enumerate}
   \item {\sl Structural} (or {\sl Firm Value}) Models: A relatively
      older form of model, these model default-times as the time 
      when the value of firm assets, $V$, falls below some
      some \emph{default trigger level}. They make explicit assumptions
      about the stochastic process, $V$, and the conditional default
      probabilities are then determined by the distance of $V(t)$ from
      the default trigger level.
   \item {\sl Intensity} (or {\sl Reduced Form}) Models: This relatively
      newer class of models avoids modeling the mechanism that
      triggers default.  Instead, it directly postulates a stochastic 
      process for the default intensity, which is the conditional 
      probability of default over the next instant. This is 
      calibrated to market data.
\end{enumerate}

\subsection{Structural Models}

\subsubsection{Merton Model}

This model assumes that the value of a firms assets follows a 
geometric Brownian motion, as $r(t)$, $\delta(t)$, and 
$\sigma(t)$ are constant:
\begin{equation}
   \label{merton}
   dV(t) = (r(t) - \delta(t)) V(t) \; dt + V(t) \sigma(t) \; d\hat{w}(t)
\end{equation}
The model also assumes that the firm is financed by equity and
a single issue of ZCBs maturing at $T$ with face value $K$ so that
default occurs if $V(T) < K$.
\\
\\
{\sl Pricing Debt:} Because we assume that default can only (!) happen
at time $T$,\footnote{This is a pretty big assumption. Specifically,
we're assuming that there is no uncertainty regarding \emph{when} 
default will occur, just \emph{if}.}the value of the firm's debt at 
time $T$ is
\[ D(T) = \min \left\{ K, \; V(T)\right\} = K - \max \{ 0, \; K-V(T)\}
      \]
Therefore, the value of the debt at any time $t<T$ equals the value
of a riskless ZCB less the value of a put option on the firms value, 
which (because of the assumed model in Equation \ref{merton}) can
be priced by the Black-Scholes Equation.
\\
\\
{\sl Credit Spreads:} If we plot the curves for the credit spread
as a function of time to maturity, we get a hump shaped pattern
because the value of the firm cannot jump (so low probability of
default in the short term), a slightly higher probability of default
exists in the medium term, but in the longer term, the drift is 
positive, so spreads decay to zero.
\newpage
{\sl Default Probabilities}: If we want to to get the cumulative
probability of default, we have to measure the number of times 
that the value of the assets of the firm end up below the default
trigger level at time $T$. In other words, we want
\begin{align*}
   G(t,T) = Q_B(V(T) < K |V(t)) = N(-y(t)) \\
   \text{where} \quad y(t) = \frac{\ln\left(\frac{V(t)e^{-\delta(T-t)}}{
      K e^{-r(T-t)}}\right)}{\sigma\sqrt{T-t}} - \frac{1}{2}
      \sigma \sqrt{T-t} 
\end{align*}
Note, this allows us to get the distribution $G(t,T)$ and, thus,
the hazard rates.\footnote{It just so happens that default can only
occur at time $T$ as stated above. As a result, the distribution
is discountinous at $T$, the density function ($g(t,s)$) 
and hazard rates ($h(t,s)$) equal zero for all $s < T$ and both are
not defined for $s=T$.}
From there we, can price CDS's.
Also, just a quick side note that $y(t)$ is often called the 
\emph{distance to default}, as the
probability of default is inversely related to $y(t)$.
\\
\\
{\sl Modeling Defaults for Multiple Firms:} We can assume that the
value, $V_i$, of each firm follows its own process
   \[ dV_i(t) = V_i(t) (r-\delta_i)\; dt + V_i(t) \sigma_i d\hat{w}_i(t)
      \]
where the $\hat{w}_i(t)$ are Brownian motions under $Q_B$
with some correlation $\rho$. Therefore, we have the cumulative
probability of default is
\begin{align*}
   Q_B(\tau_1 \leq T_1, \tau_2 \leq T_2 | I_t) &= C_\rho(
      Q_B(\tau_1 \leq T_1|I_t), \; Q_B(\tau_2 \leq T_2 | I_t)) \\
   \text{where} \quad C_\rho(x,y) = N_\rho(N^{-1}(x), N^{-1}(y))
\end{align*}
and $N^{-1}$ denotes the inverse standard normal distribution function.
Thus, the merton model assumes the joint default probabilities are
related to the individual default probabilities by the Gaussian copula.

\paragraph{Limitations} However, the Merton Model has a number of
shortcomings. Among them,
\begin{enumerate}
   \item There are often safety covenants that trigger investors
      the right to reorganize a firm if the value falls below a given
      level prior to maturity of the bonds.
   \item The firm must typically make coupon payments; the debt is
      not usually financed by ZCBs.
   \item Finally, the firm typically has a mix of short- and long-term
      debt. Moreover, the value of the firm must remain above the face 
      value of short-term debt to refinance.
   \item Credit spreads convert to zero for very short maturities.
\end{enumerate}

\newpage
\subsubsection{Black-Cox Model}

\emph{First-Passage Models} generalize Merton's model by assume that
default can happen if
\begin{enumerate}
   \item The value of the assets of the firm drops below a default
      trigger level $H(t)$ at any time $t$, in which case the 
      bondholders receive $V(t) = H(t)$.\footnote{It's, therefore,
      clear that the Merton Model is a special case of the Black-Cox
   Model where $H(t)$ is set to 0 for all $t$.}
   \item The value of the assets of the firm at the long-term debt's
      maturity $T$ is below the debt's face value $K$, in which
      case bondholders receive $V(T)$.
\end{enumerate}
{\sl Specifying $H(t)$:} We start by noting that if $H(t) = 
Ke^{-r(T-t)}$, the debt becomes riskless, as the bondholders will 
always receive the PV of the promised payment in the case of default.
Therefore, we assume that $H(t)$ will be less than that. In particular,
Black and Cox assumed that the default boundary has form
\begin{equation}
   \label{defaultbound}
   H(t) = \alpha K e^{-\eta (T-t)}
\end{equation}

   
\newpage 
\subsection{Intensity Models}

Recall that, unlike structural models, \emph{intensity models} don't
try to model the mechanism that triggers default. Rather, they
directly postulated a probability model for the arrival of default.
Specifically, they assume that in each infinitesimal interval, $dt$,
there is some risk-neutral probability $\lambda(t)$ of default
occuring, where $\lambda$ is a given stochastic process.
\\
\\
So we start by postulating that default occurs at the first time of 
a Poisson process, $N$, with stochastic intensity $\lambda$. In this
case, it follows that the time $\tau$ of the first jump has the
expontential distribution
\begin{align*}
   Q_B(\tau > t) &= Q_B(N(t) =0) = e^{-\lambda t} \\
   \Rightarrow 1- G(t,T) &= Q_B(\tau > T | I_t) 
   = Q_B(\tau > T | \tau > t) \\
   &= Q_B(N(T-t) = 0) = e^{-\lambda (T-t)}
\end{align*}
where we recall that $N$ is the value of the Poisson process. Note
that if the Poisson process has stochastic intensity, the above
expression becomes
\begin{equation}
   \label{stoch}
   1-G(t,T)=E_t^{Q_B}\left[ e^{-\int^T_t \lambda(s) \; ds } \right]
\end{equation} 
{\sl Incorporating Hazard Rates:} We recall that we can rewrite
the survival probability as a function of hazard rates, which
allows us to redefine Expression \ref{stoch} as
\begin{equation}
   \label{haz}
   e^{-\int^T_t h(t,s) \; ds} = E_t^{Q_B}\left[
      e^{-\int^T_t \lambda(s) \; ds } \right]
\end{equation}
So once a process for the default intensity has been specified, 
we can easily recover the hazard rates at any time $t$. Moreover,
if we differential both sides of Equation \ref{haz} with respect to 
$T$ and then send $T\rightarrow t$, it follows that 
   \[ \lambda(t) = h(t,t), \qquad \forall t \text{ including $t=0$}\]
{\sl Specifiying a Process for $\lambda$:} We obviously want
$\lambda$ to be non-negative and we want a closed form solution for
survival probabilities in Equation \ref{stoch}, which looks very 
must like the price of a ZCB in a short-rate model, but
with $\lambda$ instead of $r$. So we'll use the CIR process, which
has the non-negativity property and a closed-form solution:
\begin{equation}
   \label{cir}
   d\lambda(t) = \kappa_\lambda(\theta_\lambda - \lambda(t)) \; dt
      + \sigma_\lambda \sqrt{\lambda(t)} \; d\hat{w}(t)
\end{equation}
We can refer back to the Fixed Income notes for the solution.
\\
\\
{\sl General Process:} This leads us to calibrate an intensity model
as follows,


%%%%%%%%%%%% APPENDIX %%%%%%%%%%%%%%%%%%%%%%%%%%%%%%%%%%%%%%%

\newpage
\appendix

\section{Copulas}

First, we need a simple result from statistics. If $X_i$ is a continuous
RV with cumulative distribution function $F_i$, then the RV $U_i = 
F_i(X_i)$, called the \emph{probability integral transform} of $X_i$,
has a Unif(0,1) distribution.  
\\
\\
Because the value of each $X_i$ is
uniquely determined by the value of it's probability integral transform,
$U_i$, then in order to specify the jdf of $(X_1, \ldots, X_m)$,
it's enough to specify the jdf of $(U_1, \ldots, U_m)$.\footnote{This
result is called Sklar's Theorem---namely, that a unique $m$ dimensional
copula exists provided that $X_1, \ldots, X_m$ are continuous RVs.}
Specifically, if $C$ is the jdf of $X_1, \ldots, X_m$, then
\begin{align*}
   F(x_1, \ldots, x_m) &= P(X_1 \leq x_1, \ldots, X_m \leq x_m)\\
   &= P(U_1 \leq F_1(x_1), \ldots, U_m \leq F_m(x_m) )\\
   &= C(F_1(x_1), \ldots, F_m(x_m))
\end{align*}
Therefore the jdf, $C$, of $(U_1, \ldots,U_m)$ must be a multivariate
distribution on $[0,1]^m$ with uniform marginals. This is called
an \textbf{m-dimensional copula}.
\\
\\
Copulas are unique and awesome because a copula, together with
the marginal distributions, \emph{completely} describes
the statistical dependence between random variables, while the 
correlation matrix and marginal distributions only describe
\emph{linear} dependence. 





\end{document}


