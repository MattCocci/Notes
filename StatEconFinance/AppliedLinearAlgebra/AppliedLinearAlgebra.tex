\documentclass[12pt]{article}

\author{Matthew D. Cocci}
\title{Applied Linear Algebra}
\date{\today}

%% Spacing %%%%%%%%%%%%%%%%%%%%%%%%%%%%%%%%%%%%%%%%%%%%%%%%

\usepackage{fullpage}
\usepackage{setspace}
%\onehalfspacing
\usepackage{microtype}


%% Header %%%%%%%%%%%%%%%%%%%%%%%%%%%%%%%%%%%%%%%%%%%%%%%%%

%\pagestyle{fancy} 
%\lhead{}
%\rhead{}
%\chead{}
%\setlength{\headheight}{15.2pt} 
    %---Make the header bigger to avoid overlap

%\renewcommand{\headrulewidth}{0.3pt} 
    %---Width of the line

%\setlength{\headsep}{0.2in}    
    %---Distance from line to text
            

%% Mathematics Related %%%%%%%%%%%%%%%%%%%%%%%%%%%%%%%%%%%

\usepackage{amsmath}
\usepackage{amsfonts}
\usepackage{mathrsfs}
\usepackage{amsthm} %allows for labeling of theorems
\theoremstyle{plain}
\newtheorem{thm}{Theorem}[section]
\newtheorem{lem}[thm]{Lemma}
\newtheorem{prop}[thm]{Proposition}
\newtheorem{cor}[thm]{Corollary}

\theoremstyle{definition}
\newtheorem{defn}[thm]{Definition}
\newtheorem{ex}[thm]{Example}

\theoremstyle{remark}
\newtheorem*{rem}{Remark}
\newtheorem*{note}{Note}

% Below supports left-right alignment in matrices so the negative
% signs don't look bad
\makeatletter
\renewcommand*\env@matrix[1][c]{\hskip -\arraycolsep
  \let\@ifnextchar\new@ifnextchar
  \array{*\c@MaxMatrixCols #1}}
\makeatother

%% Font Choices %%%%%%%%%%%%%%%%%%%%%%%%%%%%%%%%%%%%%%%%%

\usepackage[T1]{fontenc}
\usepackage{lmodern}
\usepackage[utf8]{inputenc}
%\usepackage{blindtext}


%% Figures %%%%%%%%%%%%%%%%%%%%%%%%%%%%%%%%%%%%%%%%%%%%%%

\usepackage{graphicx}
\usepackage{subfigure} 
    %---For plotting multiple figures at once
%\graphicspath{ {Directory/} }
    %---Set a directory for where to look for figures


%% Hyperlinks %%%%%%%%%%%%%%%%%%%%%%%%%%%%%%%%%%%%%%%%%%%%
\usepackage{hyperref} 
\hypersetup{	
    colorlinks,		
        %---This colors the links themselves, not boxes
    citecolor=black,	
        %---Everything here and below changes link colors
    filecolor=black,
    linkcolor=black,
    urlcolor=black
}

%% Including Code %%%%%%%%%%%%%%%%%%%%%%%%%%%%%%%%%%%%%%% 

\usepackage{verbatim} 
    %---For including verbatim code from files, no colors

\usepackage{listings}
\usepackage{color}
\definecolor{mygreen}{RGB}{28,172,0}
\definecolor{mylilas}{RGB}{170,55,241}
\newcommand{\matlabcode}[1]{%
    \lstset{language=Matlab,%
        basicstyle=\footnotesize,%
        breaklines=true,%
        morekeywords={matlab2tikz},%
        keywordstyle=\color{blue},%
        morekeywords=[2]{1}, keywordstyle=[2]{\color{black}},%
        identifierstyle=\color{black},%
        stringstyle=\color{mylilas},%
        commentstyle=\color{mygreen},%
        showstringspaces=false,%
            %---Without this there will be a symbol in 
            %---the places where there is a space
        numbers=left,%
        numberstyle={\tiny \color{black}},% 
            %---Size of the numbers
        numbersep=9pt,% 
            %---Defines how far the numbers are from the text
        emph=[1]{for,end,break,switch,case},emphstyle=[1]\color{red},%
            %---Some words to emphasise
    }%
    \lstinputlisting{#1}
}
    %---For including Matlab code from .m file with colors,
    %---line numbering, etc. 


%% Misc %%%%%%%%%%%%%%%%%%%%%%%%%%%%%%%%%%%%%%%%%%%%%% 

\usepackage{enumitem} 
    %---Has to do with enumeration	
\usepackage{appendix}
%\usepackage{natbib} 
    %---For bibliographies
\usepackage{pdfpages}
    %---For including whole pdf pages as a page in doc


%% User Defined %%%%%%%%%%%%%%%%%%%%%%%%%%%%%%%%%%%%%%%%%% 

%\newcommand{\nameofcmd}{Text to display}



%%%%%%%%%%%%%%%%%%%%%%%%%%%%%%%%%%%%%%%%%%%%%%%%%%%%%%%%%%%%%%%%%%%%%%%% 
%% BODY %%%%%%%%%%%%%%%%%%%%%%%%%%%%%%%%%%%%%%%%%%%%%%%%%%%%%%%%%%%%%%%%
%%%%%%%%%%%%%%%%%%%%%%%%%%%%%%%%%%%%%%%%%%%%%%%%%%%%%%%%%%%%%%%%%%%%%%%% 


\begin{document}
\maketitle

%\tableofcontents 

\newpage
\section{Properties of Matrices}

Here are some words we use to describe matrices.
\begin{enumerate}
\item \emph{Symmetric}: $A=A^T$
\item \emph{Sparse}: A matrix consisting of mostly zeros.
\item \emph{Tridiagonal}: A matrix with three diagonals, something like
\[
  A = 
  \begin{bmatrix}[r]
   2 & -1 &  0 &  0 \\ 
  -1 & 2 & -1 &  0 \\  
   0 & -1 &  2 & -1 \\ 
   0 &  0 & -1 &  2 \\   
  \end{bmatrix} 
\]
\item \emph{Invertible}: Matrix $A$ is invertible if there exists a matrix $A^{-1}$ such that $A A^{-1} = I$.

\item \emph{Rank}: For an arbitrary $m\times n$ matrix $A$, the rank is the number of linearly independent columns or, alternatively, rows of $A$.\footnote{Whether you check rows or columns doesn't matter, those numbers will be equal.} The rank $r$ is such that $r \leq n$ and $r\leq m$.
\end{enumerate}
Now, one of the most important concepts for describing an $n\times n$ matrix $A$. The singular vs. nonsingular distinction entails a whole host of consequences, detailed below:
\begin{table}[h!]
\centering
\begin{tabular}{lll}
\textbf{Nonsingular}                     && \textbf{Singular} \\
$A$ invertible                           && $A$ not invertible \\
Columns independent                      && Columns dependent \\
Rows independent                         && Rows dependent \\
$\det(A)\neq0$                           && $\det(A)=0$ \\
$Ax = 0$ has one solution, $x = 0$       && $Ax=0$ has infinitely many solutions\\
$Ax = b$ has one solution, $x = A^{-1}b$ && $Ax = b$ has no solution or infinitely many \\
$A$ has $n$ (nonzero) pivots             && $A$ has $r<n$ pivots \\
$A$ has full rank                        && $A$ has rank $r<n$ \\
Reduced row echelon form is $R = I$      && $R$ has at least one zero row \\
Column space is all of $\mathbb{R}^n$    && Column space has dimension $r<n$ \\
Row space is all of $\mathbb{R}^n$       && Row space has dimension $r<n$ \\
All eigenvalues are non-zero             && Zero is an eigenvalue of $A$ \\
$A^T A$ is symmetric positive definite   && $A^T A$ is only semidefinite \\
$A$ has $n$ (positive) singular values   && $A$ has $r<n$ singular values
\end{tabular}
\end{table}

\section{Matrix Multiplication and the Nullspace}

{\sl Matrix multiplication $Ax$ is the combination of the columns of $A$. Each column scaled by the corresponding element of $x$.} This is, hands down, one of the simplest, most important, and most practical ways of thinking about matrix multiplication. 

Being more explicit, and letting $A_{\cdot i}$ denote the $i$th column of $A$ (an $m\times n$ matrix), and letting $x_j$ denote the $j$th element of $n\times 1$ vector $x$, we can express
\[
  A x = x_1 A_{\cdot 1}  + \cdots + x_n A_{\cdot n} 
\]
Since the result is a combination of the columns of $A$---each an $m\times 1$ vector---we can easily see that the result will also be an $m\times 1$ column vector.

This intuition also provides a convenient way to ``pick out'' the $i$th column of matrix $A$ via matrix multiplication: just right-multiply $A$ by an $n \times 1$ vector whose elements are all zero, except for the $i$th element, which equals 1. Example,
\[
  \begin{bmatrix} 
  8 & 1 & 6 \\
  3 & 5 & 7 \\
  4 & 9 & 2 
  \end{bmatrix} 
  \begin{bmatrix} 
  0 \\ 1 \\ 0 
  \end{bmatrix} 
  = 0  
  \begin{bmatrix} 8 \\ 3 \\ 4 \\ \end{bmatrix} 
  + 1  \begin{bmatrix} 1 \\ 5 \\ 9 \\ \end{bmatrix} 
  + 0  \begin{bmatrix} 6 \\ 7 \\ 2 \\ \end{bmatrix} 
  = 
  \begin{bmatrix} 1 \\ 5 \\ 9 \\ \end{bmatrix} 
\]


\begin{defn} 
The \emph{nullspace} of a matrix $A$ ($m\times n$), which is denoted $N(A)$, is the set of all $x \in \mathbb{R}^n$ such that $Ax=0$. Intuitively, the nullspace consists of all vectors that can linearly combine the columns of $A$ to get the zero vector. 

It is \emph{always} the case that the zero vector is in $N(A)$, so the set can never be empty. 
\end{defn}


\section{Canonical Problem: The System of Linear Equations under the Row and Column Interpretations}

Linear algebra's founding problem is likely the task of solving the simple system of linear equations:
\begin{equation}
    \label{canonical}
    Ax = b
\end{equation}
For now, assume $A$ is $p\times p$. There are two ways to think of this equation:
\begin{enumerate}
    \item {\sl Row Interpretation}: We think of each row in the equation as defining a line in $p$ dimensional place. The solution is where those lines intersect.
    \item {\sl Column Interpretation}: A solution to the equation exists if and only if $b$ is in the column space of $A$. More explicitly, there is a solution if and only if $b$ can be written as a linear combination of the $p$ vectors defined by the columns of $A$.
\end{enumerate}
The latter interpretation is very powerful. We see that the column space of $A$---the set of all linear combinations of the columns of $A$---is crucial to characterizing the matrix $A$. 

For example, if the columns of $A$ are linearly independent, then the matrix, in a sense, ``covers'' the entire $p$-dimensional space. We can thus solve \emph{any} equation like Equation \ref{canonical} because $b$ will certainly lie within the column space of $A$. Thus, we can find some $x$ such that $Ax=b$. 

We only potentially run into problems if the columns aren't linearly independent. So if the rank of the matrix is $k<p$, only a $k$-dimensional hyperplance embedded within $\mathbb{R}^p$ is covered by the columns of $A$. And $b$ might not lie in that subspace. This very possibility manifests itself in the non-invertability of $A$, which prevents us from trivially solving the system via $x = A^{-1}b$, for the very reasons just discussed.





%% APPPENDIX %%

% \appendix




\end{document}


%%%%%%%%%%%%%%%%%%%%%%%%%%%%%%%%%%%%%%%%%%%%%%%%%%%%%%%%%%%%%%%%%%%%%%%% 
%%%%%%%%%%%%%%%%%%%%%%%%%%%%%%%%%%%%%%%%%%%%%%%%%%%%%%%%%%%%%%%%%%%%%%%%
%%%%%%%%%%%%%%%%%%%%%%%%%%%%%%%%%%%%%%%%%%%%%%%%%%%%%%%%%%%%%%%%%%%%%%%% 

%%%% SAMPLE CODE %%%%%%%%%%%%%%%%%%%%%%%%%%%%%%%%%%%%%%

    %% BIBLIOGRAPHIES %%

        \cite{LabelInSourcesFile} 
        \citep{LabelInSourcesFile} Cites in parens
        \nocite{LabelInSourceFile} includes in refs w/o specific citation
        \bibliographystyle{apalike} 
        \bibliography{sources.bib} where sources.bib is file

    %% SPACING %%

        \vspace{1in}
        \hspace{1in}


    %% INCLUDING PDF PAGE %%

        \includepdf{file.pdf}


    %% INCLUDING CODE %%

        \verbatiminput{file.ext}    
            %---Includes verbatim text from the file
        \texttt{text}	  
            %---Renders text in courier, or code-like, font

        \matlabcode{file.m}	  
            %---Includes Matlab code with colors and line numbers


    %% INCLUDING FIGURES %%

        % Basic Figure with size scaling
            \begin{figure}[h!]
               \centering
               \includegraphics[scale=1]{file.pdf}
            \end{figure}

        % Basic Figure with specific height
            \begin{figure}[h!]
               \centering
               \includegraphics[height=5in, width=5in]{file.pdf}
            \end{figure}

        % Figure with cropping, where the order for trimming is  L, B, R, T
            \begin{figure}
               \centering
               \includegraphics[trim={1cm, 1cm, 1cm, 1cm}, clip]{file.pdf}
            \end{figure}


        % Side by Side figures
            \begin{figure}[h!]
                \centering
                \mbox{\subfigure{
                    \includegraphics[scale=1]{file1.pdf}
                }\quad\subfigure{
                    \includegraphics[scale=1]{file2.pdf} 
                }
                }
            \end{figure}
    

