\documentclass[12pt]{article}

\author{Matthew D. Cocci}
\title{Factor Models}
\date{\today}

%% Formatting & Spacing %%%%%%%%%%%%%%%%%%%%%%%%%%%%%%%%%%%%

%\usepackage[top=1in, bottom=1in, left=1in, right=1in]{geometry} % most detailed page formatting control
\usepackage{fullpage} % Simpler than using the geometry package; std effect
\usepackage{setspace}
%\onehalfspacing
\usepackage{microtype}

%% Formatting %%%%%%%%%%%%%%%%%%%%%%%%%%%%%%%%%%%%%%%%%%%%%

%\usepackage[margin=1in]{geometry}
    %   Adjust the margins with geometry package
%\usepackage{pdflscape}
    %   Allows landscape pages
%\usepackage{layout}
    %   Allows plotting of picture of formatting 



%% Header %%%%%%%%%%%%%%%%%%%%%%%%%%%%%%%%%%%%%%%%%%%%%%%%%

%\usepackage{fancyhdr} 
%\pagestyle{fancy} 
%\lhead{}
%\rhead{}
%\chead{}
%\setlength{\headheight}{15.2pt} 
    %   Make the header bigger to avoid overlap

%\fancyhf{}
    %   Erase header settings

%\renewcommand{\headrulewidth}{0.3pt} 
    %   Width of the line

%\setlength{\headsep}{0.2in}    
    %   Distance from line to text
            

%% Mathematics Related %%%%%%%%%%%%%%%%%%%%%%%%%%%%%%%%%%%

\usepackage{amsmath}
\usepackage{amsfonts}
\usepackage{mathrsfs}
\usepackage{amsthm} %allows for labeling of theorems
\theoremstyle{plain}
\newtheorem{thm}{Theorem}[section]
\newtheorem{lem}[thm]{Lemma}
\newtheorem{prop}[thm]{Proposition}
\newtheorem{cor}[thm]{Corollary}

\theoremstyle{definition}
\newtheorem{defn}[thm]{Definition}
\newtheorem{ex}[thm]{Example}

\theoremstyle{remark}
\newtheorem*{rem}{Remark}
\newtheorem*{note}{Note}

% Below supports left-right alignment in matrices so the negative
% signs don't look bad
\makeatletter
\renewcommand*\env@matrix[1][c]{\hskip -\arraycolsep
  \let\@ifnextchar\new@ifnextchar
  \array{*\c@MaxMatrixCols #1}}
\makeatother


%% Font Choices %%%%%%%%%%%%%%%%%%%%%%%%%%%%%%%%%%%%%%%%%

\usepackage[T1]{fontenc}
\usepackage{lmodern}
\usepackage[utf8]{inputenc}
%\usepackage{blindtext}


%% Figures %%%%%%%%%%%%%%%%%%%%%%%%%%%%%%%%%%%%%%%%%%%%%%

\usepackage{graphicx}
\usepackage{subfigure} 
    %   For plotting multiple figures at once
%\graphicspath{ {Directory/} }
    %   Set a directory for where to look for figures


%% Hyperlinks %%%%%%%%%%%%%%%%%%%%%%%%%%%%%%%%%%%%%%%%%%%%
\usepackage{hyperref} 
\hypersetup{%
    colorlinks,		
        %   This colors the links themselves, not boxes
    citecolor=black,	
        %   Everything here and below changes link colors
    filecolor=black,
    linkcolor=black,
    urlcolor=black
}

%% Including Code %%%%%%%%%%%%%%%%%%%%%%%%%%%%%%%%%%%%%%% 

\usepackage{verbatim} 
    %   For including verbatim code from files, no colors

\usepackage{listings}
\usepackage{color}
\definecolor{mygreen}{RGB}{28,172,0}
\definecolor{mylilas}{RGB}{170,55,241}
\newcommand{\matlabcode}[1]{%
    \lstset{language=Matlab,%
        basicstyle=\footnotesize,%
        breaklines=true,%
        morekeywords={matlab2tikz},%
        keywordstyle=\color{blue},%
        morekeywords=[2]{1}, keywordstyle=[2]{\color{black}},%
        identifierstyle=\color{black},%
        stringstyle=\color{mylilas},%
        commentstyle=\color{mygreen},%
        showstringspaces=false,%
            %   Without this there will be a symbol in 
            %   the places where there is a space
        numbers=left,%
        numberstyle={\tiny \color{black}},% 
            %   Size of the numbers
        numbersep=9pt,% 
            %   Defines how far the numbers are from the text
        emph=[1]{for,end,break,switch,case},emphstyle=[1]\color{red},%
            %   Some words to emphasise
    }%
    \lstinputlisting{#1}
}
    %   For including Matlab code from .m file with colors,
    %   line numbering, etc. 

%% Bibliographies %%%%%%%%%%%%%%%%%%%%%%%%%%%%%%%%%%%% 

%\usepackage{natbib} 
    %---For bibliographies
%\setlength{\bibsep}{3pt} % Set how far apart bibentries are

%% Misc %%%%%%%%%%%%%%%%%%%%%%%%%%%%%%%%%%%%%%%%%%%%%% 

\usepackage{enumitem} 
    %   Has to do with enumeration	
\usepackage{appendix}
%\usepackage{natbib} 
    %   For bibliographies
\usepackage{pdfpages}
    %   For including whole pdf pages as a page in doc


%% User Defined %%%%%%%%%%%%%%%%%%%%%%%%%%%%%%%%%%%%%%%%%% 

%\newcommand{\nameofcmd}{Text to display}
\newcommand*{\Chi}{\mbox{\large$\chi$}} %big chi
    %   Bigger Chi



%%%%%%%%%%%%%%%%%%%%%%%%%%%%%%%%%%%%%%%%%%%%%%%%%%%%%%%%%%%%%%%%%%%%%%%% 
%% BODY %%%%%%%%%%%%%%%%%%%%%%%%%%%%%%%%%%%%%%%%%%%%%%%%%%%%%%%%%%%%%%%%
%%%%%%%%%%%%%%%%%%%%%%%%%%%%%%%%%%%%%%%%%%%%%%%%%%%%%%%%%%%%%%%%%%%%%%%% 


\begin{document}
\maketitle

\tableofcontents 

\clearpage
\section{Key Result for Factor Models}

The conditional distribution of jointly multivariate normal random variables is
one of the most important results that underlies the estimation and application
of factor, regression, and state space models. It can be summarized as follows:
\begin{align}
  \text{Given} \qquad 
    \begin{bmatrix} X_1 \\ X_2 \end{bmatrix}
    &\sim 
    N\left(
    \begin{bmatrix} \mu_1 \\ \mu_2 \end{bmatrix},
    \begin{bmatrix} 
      \Sigma_{11} & \Sigma_{12} \\
      \Sigma_{21} & \Sigma_{22} 
    \end{bmatrix} 
    \right) \notag\\
    \label{reg} \\
  X_1 | X_2 &\sim N(\hat{\mu}, \hat{\Sigma})  \notag\\
  \text{where} \quad
  \hat{\mu} &= \mu_1 + \Sigma_{12} \Sigma^{-1}_{22} 
    (X_2-\mu_2) \notag\\
  \hat{\Sigma} &= \Sigma_{11} - \Sigma_{12} \Sigma^{-1}_{22} 
    \Sigma_{21}\notag
\end{align}
This will be used again and again to derive the results presented below.

\section{Factor Models without Dynamics}

To begin with the simplest case, this section considers factor models without
dynamics, ignoring any autocorrelation between observations even when dealing
with time series data.

First, suppose that we have data $\{X_{it}\}$ for $i=1,\ldots,n$ and
$t=1,\ldots,T$. In economics, this will often be a panel dataset, with
individuals or data series along the $i$ dimension and time across the
$t$ dimension. However, the model presented below could just as well
characterize any dataset that varies along two dimensions.\footnote{For
example, factor analysis was developed early on to extract certain
components of intelligence (general intelligence, verbal intelligence,
analytical intelligence, etc.) that explain the types of questions that
people answer correctly and how often they answer correctly. In that
case, you have $T$ individuals answering $n$ different questions that
test a range of abilities.  Then, the $X_{it}$ would be variables
denoting whether individual $t$ got question $i$ right, and you could
extract common factors across individuals which correspond to these
different types of intelligence.}

The data $\{X_{it}\}$ are assumed to be explained by $r$ factors,
$\{f_j\}_{j=1}^r$, plus random error:
\begin{equation}
  X_{it} = \mu_i + \lambda_{i1} f_{1t} + \cdots + \lambda_{ir} f_{rt} + e_{it}
  \qquad i = 1,\ldots,n
\end{equation}
In matrix notation, we can stack all ovservations for $t$ and write out the
additional distributional assumptions as:
\begin{align*}
  X_t = \Lambda F_t + e_t
\end{align*}




%% APPPENDIX %%

% \appendix




\end{document}


%%%%%%%%%%%%%%%%%%%%%%%%%%%%%%%%%%%%%%%%%%%%%%%%%%%%%%%%%%%%%%%%%%%%%%%% 
%%%%%%%%%%%%%%%%%%%%%%%%%%%%%%%%%%%%%%%%%%%%%%%%%%%%%%%%%%%%%%%%%%%%%%%%
%%%%%%%%%%%%%%%%%%%%%%%%%%%%%%%%%%%%%%%%%%%%%%%%%%%%%%%%%%%%%%%%%%%%%%%% 

%%%% SAMPLE CODE %%%%%%%%%%%%%%%%%%%%%%%%%%%%%%%%%%%%%%

    %% VIEW LAYOUT %%
    
        \layout

    %% LANDSCAPE PAGE %%

        \begin{landscape}
        \end{landscape}

    %% BIBLIOGRAPHIES %%

        \cite{LabelInSourcesFile}  %Use in text; cites
        \citep{LabelInSourcesFile} %Use in text; cites in parens

        \nocite{LabelInSourceFile} % Includes in refs w/o specific citation
        \bibliographystyle{apalike}  % Or some other style

        % To ditch the ``References'' header
        \begingroup
        \renewcommand{\section}[2]{} 
        \endgroup

        \bibliography{sources} % where sources.bib has all the citation info

    %% SPACING %%

        \vspace{1in}
        \hspace{1in}


    %% INCLUDING PDF PAGE %%

        \includepdf{file.pdf}


    %% INCLUDING CODE %%

        \verbatiminput{file.ext}    
            %   Includes verbatim text from the file
        \texttt{text}	  
            %   Renders text in courier, or code-like, font

        \matlabcode{file.m}	  
            %   Includes Matlab code with colors and line numbers


    %% INCLUDING FIGURES %%

        % Basic Figure with size scaling
            \begin{figure}[h!]
               \centering
               \includegraphics[scale=1]{file.pdf}
            \end{figure}

        % Basic Figure with specific height
            \begin{figure}[h!]
               \centering
               \includegraphics[height=5in, width=5in]{file.pdf}
            \end{figure}

        % Figure with cropping, where the order for trimming is  L, B, R, T
            \begin{figure}
               \centering
               \includegraphics[trim={1cm, 1cm, 1cm, 1cm}, clip]{file.pdf}
            \end{figure}


        % Side by Side figures
            \begin{figure}[h!]
                \centering
                \mbox{\subfigure{
                    \includegraphics[scale=1]{file1.pdf}
                }\quad\subfigure{
                    \includegraphics[scale=1]{file2.pdf} 
                }
                }
            \end{figure}
    

