\documentclass[12pt]{article}

\author{Matthew Cocci}
\title{Linear Algebra}
\date{\today}

%% Spacing %%%%%%%%%%%%%%%%%%%%%%%%%%%%%%%%%%%%%%%%%%%%%%%%

\usepackage{fullpage}
\usepackage{setspace}
%\onehalfspacing
\usepackage{microtype}


%% Header %%%%%%%%%%%%%%%%%%%%%%%%%%%%%%%%%%%%%%%%%%%%%%%%%

\usepackage{fancyhdr}
\pagestyle{fancy}
\rhead{Matthew D. Cocci}
\lhead{Singular Value Decomposition and PCA}
\chead{}
\setlength{\headheight}{15.2pt}
    %---Make the header bigger to avoid overlap

\renewcommand{\headrulewidth}{0.3pt}
    %---Width of the line

\setlength{\headsep}{0.2in}
    %---Distance from line to text


%% Mathematics Related %%%%%%%%%%%%%%%%%%%%%%%%%%%%%%%%%%%

\usepackage{amsmath}
\usepackage{amsfonts}
\usepackage{amsthm} %allows for labeling of theorems
\newtheorem{thm}{Theorem}[section]
\newtheorem{lem}[thm]{Lemma}
\newtheorem{prop}[thm]{Proposition}
\newtheorem{cor}[thm]{Corollary}
%\numberwithin{equation}{section}
    %---This labels the equations in relation to the sections
    %---rather than other equations
%\numberwithin{equation}{subsection}
    %---This labels relative to subsections


%% Font Choices %%%%%%%%%%%%%%%%%%%%%%%%%%%%%%%%%%%%%%%%%

\usepackage[T1]{fontenc}
\usepackage{lmodern}
\usepackage[utf8]{inputenc}
%\usepackage{blindtext}


%% Figures %%%%%%%%%%%%%%%%%%%%%%%%%%%%%%%%%%%%%%%%%%%%%%

\usepackage{graphicx}
\usepackage{subfigure}
    %---For plotting multiple figures at once
%\graphicspath{ {Directory/} }
    %---Set a directory for where to look for figures


%% Hyperlinks %%%%%%%%%%%%%%%%%%%%%%%%%%%%%%%%%%%%%%%%%%%%
\usepackage{hyperref}
\hypersetup{
    colorlinks,
        %---This colors the links themselves, not boxes
    citecolor=black,
        %---Everything here and below changes link colors
    filecolor=black,
    linkcolor=black,
    urlcolor=black
}

%% Including Code %%%%%%%%%%%%%%%%%%%%%%%%%%%%%%%%%%%%%%%

\usepackage{verbatim}
    %---For including verbatim code from files, no colors

\usepackage{listings}
\usepackage{color}
\definecolor{mygreen}{RGB}{28,172,0}
\definecolor{mylilas}{RGB}{170,55,241}
\newcommand{\matlabcode}[1]{%
    \lstset{language=Matlab,%
        basicstyle=\footnotesize,%
        breaklines=true,%
        morekeywords={matlab2tikz},%
        keywordstyle=\color{blue},%
        morekeywords=[2]{1}, keywordstyle=[2]{\color{black}},%
        identifierstyle=\color{black},%
        stringstyle=\color{mylilas},%
        commentstyle=\color{mygreen},%
        showstringspaces=false,%
            %---Without this there will be a symbol in
            %---the places where there is a space
        numbers=left,%
        numberstyle={\tiny \color{black}},%
            %---Size of the numbers
        numbersep=9pt,%
            %---Defines how far the numbers are from the text
        emph=[1]{for,end,break,switch,case},emphstyle=[1]\color{red},%
            %---Some words to emphasise
    }%
    \lstinputlisting{#1}
}
    %---For including Matlab code from .m file with colors,
    %---line numbering, etc.


%% Misc %%%%%%%%%%%%%%%%%%%%%%%%%%%%%%%%%%%%%%%%%%%%%%

\usepackage{enumitem}
    %---Has to do with enumeration
\usepackage{appendix}
%\usepackage{natbib}
    %---For bibliographies
\usepackage{pdfpages}
    %---For including whole pdf pages as a page in doc


%% User Defined %%%%%%%%%%%%%%%%%%%%%%%%%%%%%%%%%%%%%%%%%%

%\newcommand{\nameofcmd}{Text to display}



%%%%%%%%%%%%%%%%%%%%%%%%%%%%%%%%%%%%%%%%%%%%%%%%%%%%%%%%%%%%%%%%%%%%%%%%
%% BODY %%%%%%%%%%%%%%%%%%%%%%%%%%%%%%%%%%%%%%%%%%%%%%%%%%%%%%%%%%%%%%%%
%%%%%%%%%%%%%%%%%%%%%%%%%%%%%%%%%%%%%%%%%%%%%%%%%%%%%%%%%%%%%%%%%%%%%%%%


\begin{document}
%\maketitle

%\tableofcontents

\section{The Singular Value Decomposition}

The singular value decomposition is a useful decomposition that works for \emph{any} matrix, which is part of the reason for its popularity. At it's most fundamental, the singular value decomposition amounts to the following:
\begin{quotation}
\noindent
SVD breaks down any matrix into a three-matrix product:
  \[
    A = U\Sigma V^T
\]
Geometrically, $A$ is the mapping of an orthonormal basis, $V^T$, into another orthonormal basis $U$, where the different basis elements of $V$ are \emph{stretched} by the values in the diagonal matrix $\Sigma$.
\end{quotation}
In the following subsection, we'll build up intuition by imposing different conditions on the matrix $A$, and then seeing how we might find those orthonormal bases, $U$ and $V$.


\subsection{Intuition}

The intuition for the SVD comes down to simple notions like stretching and rotating. So we'll start with very simple examples, then consider increasingly general matrices.
\begin{enumerate}
\item {\sl Diagonal Matrices}: To begin, consider any diagonal matrix,
\[
    A =
    \begin{pmatrix}
        a_1 & 0 & \cdots & 0 \\
        0 & a_2 & \cdots & 0 \\
        \vdots & \vdots & \ddots & \vdots \\
        0 & 0 & \cdots & a_n \\
    \end{pmatrix}
    \quad
    \Rightarrow
    \quad
    Ax = \begin{pmatrix} a_1 x_1 \\ a_2 x_2 \\ \vdots \\
        a_n x_n \end{pmatrix}
\]
It's quite clear that if $x$ is vector, the result of the product $Ax$ is merely a \emph{stretched} version of $x$. The $i$th coordinate, $x_i$, is simply stretched by $a_i$ along the $i$th axis.

To see an example, consider the following matrix and picture
\[ \text{Figure} \]

\item {\sl Symmetric Matrices}: Next, suppose that instead of having $A$ diagonal, it is (more generally) just symmetric so that $A^T = A$. Then $A$ must have $n$ (not necessarily distinct) eigenvalues and $n$ distinct eigenvectors, where the eigenvectors are also orthogonal.\footnote{These are all properties of symmetric matrices.} Now why does this matter?

At first blush, since $A$ isn't diagonal, we seem to have lost the direct interpretation about $A$ stretching each coordinate of a vector along a Cartesian axis. \emph{However}, it so happens that the eigenvectors (by definition) define special axes along which vectors are stretched when multiplied by $A$.
\[ \text{Figure} \]
Just as a diagonal matrix \emph{stretches} each vector element in the direction of its corresponding Cartesian axis, so too does a symmetric matrix stretch all vector elements along \emph{special} axes defined the orthogonal eigenvectors. Note the degree of stretching is determined by the corresponding eigenvalue for an eigenvector axis.

\item {\sl General Matrices}: For any abitrary matrix, $A$, we might ask ourselves ``Can we define a special set of axes such that vectors lying along any of these special axes simply stretch in the direction of those axes.'' The answer: yes, as we'll see below.


\end{enumerate}
But first, before proving that such special axes exist for any arbitrary
matrix, $A$, let's fully develop the ideas from above.


\subsection{The SVD Representation}

Suppose that we can choose a set of orthogonal unit
vectors, $\{v_1,\ldots,v_n\}$ such that the mapped vectors,
$\{Av_1,\ldots,Av_n\}=:\{\sigma_1 u_1, \ldots, \sigma_n u_n\}$, are also
orthogonal, where the $u_i$ are all unit vectors and the $\sigma_i$
scale those unit vectors appropriately to match the $Av_i$.\footnote{In
subsequent subsections, we'll show how to actually find
$\{v_1,\ldots,v_n\}$.}

Next, recall that we can express any arbitrary vector $x\in\mathbb{R}^n$
in terms of any arbitrary orthogonal basis for $\mathbb{R}^n$ (like
$\{v_1,\ldots,v_n\}$ as follows:
\[
  x = \sum^n_{i=1} \langle v_i, x\rangle v_i
\]
That is, project $x$ onto a special axis defined by $v_i$, multiply the
resulting scalar by the $v_i$, and sum them up. As a result, we can
write
\begin{align*}
  Ax &= A\left(\sum^n_{i=1} \langle v_i, x\rangle v_i\right)
  = \sum^n_{i=1} \langle v_i, x\rangle A v_i
  = \sum^n_{i=1} \langle v_i, x\rangle \sigma_i u_i\\
  \Leftrightarrow \quad &= \sum^n_{i=1} u_i \sigma_i (v_i^T x)\\
\end{align*}
where we played fast-and-loose with the ordering of terms (though it all
checks out). Finally, let's remove $x$ from both sides to get
\begin{align*}
  A = \sum^n_{i=1} u_i \sigma_i v_i^T
  \qquad \Leftrightarrow \qquad
  A = U \Sigma V^T
\end{align*}
where $U$ has columns $\{u_1,\ldots,u_n\}$ and $V$ has columns
$\{v_1,\ldots,v_n\}$. The righthand express is the textbook
representation you often see for the SVD.
\\
\\
{\sl Interpretation}: We've just written arbitrary matrix $A$ as the
product of three matrices:
\begin{itemize}
  \item $V$: An orthonormal basis in the domain of $A$.
  \item $U$: An orthonormal basis in the range of $A$.
  \item $\Sigma$: How much to stretch the vectors in $V$ to get the
    vectors of $U$.
\end{itemize}
To be even more explicit, the SVD breaks down any arbitrary matrix $A$
into
\begin{itemize}
  \item An initial rotation, $V^T$.
  \item A scaling by $\Sigma$.
  \item A final rotation, $U$.
\end{itemize}







\section{Principal Component Analysis}
\subsection{Overview}

Principal Component Analysis is a type of \emph{feature extraction}
that takes a number of ``features'' or independent variables, then
uses them in combination to build and rank new variables which
explain the largest share of variability in some response (or
dependent) variable. This differs from \emph{feature selection}
which takes a list of features and selects a subset to use as is
(free of any combination or modification).

Principal Component Analysis can also be used to combat the
\emph{curse of dimensionality}.  Loosely speaking, this technical
problem arises when you have many features or independent variables
that might explain some dependent variable. As the number of
features grows, the data becomes ever sparser in a higher
dimensional space.\footnote{Say that an outcome $y$ is dependent
upon the roll of $N$ dice, $\{x_1, \ldots, x_N\}$. As $N$ increases,
there are exponentially more outcomes that might
result--$6^N$ to be precise. So as $N$ increases, the space within
which the data lives grows exponentially, which necessitates
exponentially larger samples to match the explanatory power you
would have in dimensions with lower $N$.}
Principal Component Analysis can be used to combine features, rank
them in order or importance, and throw out features that don't
explain much variance in the dependent variable.


\subsection{Intuition}

Principal Component Analysis really just finds new ways
to look at the data by changing the axes.  Imagine we had
the following data.

PCA highlights the fact that there is nothing sacred
about the axes I have drawn there.  We can express the
data points relative to a new set of axes so that the
variation along different dimensions is more explicit. In
linear algebra terminology, we can use another \emph{basis}.

\subsection{Implementation}




%% APPPENDIX %%

% \appendix




\end{document}


%%%%%%%%%%%%%%%%%%%%%%%%%%%%%%%%%%%%%%%%%%%%%%%%%%%%%%%%%%%%%%%%%%%%%%%%
%%%%%%%%%%%%%%%%%%%%%%%%%%%%%%%%%%%%%%%%%%%%%%%%%%%%%%%%%%%%%%%%%%%%%%%%
%%%%%%%%%%%%%%%%%%%%%%%%%%%%%%%%%%%%%%%%%%%%%%%%%%%%%%%%%%%%%%%%%%%%%%%%

%%%% SAMPLE CODE %%%%%%%%%%%%%%%%%%%%%%%%%%%%%%%%%%%%%%

    %% BIBLIOGRAPHIES %%

        \cite{LabelInSourcesFile}
        \citep{LabelInSourcesFile} Cites in parens
        \nocite{LabelInSourceFile} includes in refs w/o specific citation
        \bibliographystyle{apalike}
        \bibliography{sources.bib} where sources.bib is file

    %% SPACING %%

        \vspace{1in}
        \hspace{1in}


    %% INCLUDING PDF PAGE %%

        \includepdf{file.pdf}


    %% INCLUDING CODE %%

        \verbatiminput{file.ext}
            %---Includes verbatim text from the file
        \texttt{text}
            %---Renders text in courier, or code-like, font

        \matlabcode{file.m}
            %---Includes Matlab code with colors and line numbers


    %% INCLUDING FIGURES %%

        % Basic Figure with size scaling
            \begin{figure}[h!]
               \centering
               \includegraphics[scale=1]{file.pdf}
            \end{figure}

        % Basic Figure with specific height
            \begin{figure}[h!]
               \centering
               \includegraphics[height=5in, width=5in]{file.pdf}
            \end{figure}

        % Figure with cropping, where the order for trimming is  L, B, R, T
            \begin{figure}
               \centering
               \includegraphics[trim={1cm, 1cm, 1cm, 1cm}, clip]{file.pdf}
            \end{figure}


        % Side by Side figures
            \begin{figure}[h!]
                \centering
                \mbox{\subfigure{
                    \includegraphics[scale=1]{file1.pdf}
                }\quad\subfigure{
                    \includegraphics[scale=1]{file2.pdf}
                }
                }
            \end{figure}


