\documentclass[12pt]{article}

\author{Matthew D. Cocci}
\title{Differential Equations}
\date{\today}

%% Formatting & Spacing %%%%%%%%%%%%%%%%%%%%%%%%%%%%%%%%%%%%

%\usepackage[top=1in, bottom=1in, left=1in, right=1in]{geometry} % most detailed page formatting control
\usepackage{fullpage} % Simpler than using the geometry package; std effect
\usepackage{setspace}
%\onehalfspacing
\usepackage{microtype}

%% Formatting %%%%%%%%%%%%%%%%%%%%%%%%%%%%%%%%%%%%%%%%%%%%%

%\usepackage[margin=1in]{geometry}
    %   Adjust the margins with geometry package
%\usepackage{pdflscape}
    %   Allows landscape pages
%\usepackage{layout}
    %   Allows plotting of picture of formatting



%% Header %%%%%%%%%%%%%%%%%%%%%%%%%%%%%%%%%%%%%%%%%%%%%%%%%

%\usepackage{fancyhdr}
%\pagestyle{fancy}
%\lhead{}
%\rhead{}
%\chead{}
%\setlength{\headheight}{15.2pt}
    %   Make the header bigger to avoid overlap

%\fancyhf{}
    %   Erase header settings

%\renewcommand{\headrulewidth}{0.3pt}
    %   Width of the line

%\setlength{\headsep}{0.2in}
    %   Distance from line to text


%% Mathematics Related %%%%%%%%%%%%%%%%%%%%%%%%%%%%%%%%%%%

\usepackage{amsmath}
\usepackage{amsfonts}
\usepackage{mathrsfs}
\usepackage{amsthm} %allows for labeling of theorems
\theoremstyle{plain}
\newtheorem{thm}{Theorem}[section]
\newtheorem{lem}[thm]{Lemma}
\newtheorem{prop}[thm]{Proposition}
\newtheorem{cor}[thm]{Corollary}

\theoremstyle{definition}
\newtheorem{defn}[thm]{Definition}
\newtheorem{ex}[thm]{Example}

\theoremstyle{remark}
\newtheorem*{rem}{Remark}
\newtheorem*{note}{Note}

% Below supports left-right alignment in matrices so the negative
% signs don't look bad
\makeatletter
\renewcommand*\env@matrix[1][c]{\hskip -\arraycolsep
  \let\@ifnextchar\new@ifnextchar
  \array{*\c@MaxMatrixCols #1}}
\makeatother


%% Font Choices %%%%%%%%%%%%%%%%%%%%%%%%%%%%%%%%%%%%%%%%%

\usepackage[T1]{fontenc}
\usepackage{lmodern}
\usepackage[utf8]{inputenc}
%\usepackage{blindtext}


%% Figures %%%%%%%%%%%%%%%%%%%%%%%%%%%%%%%%%%%%%%%%%%%%%%

\usepackage{graphicx}
\usepackage{subfigure}
    %   For plotting multiple figures at once
%\graphicspath{ {Directory/} }
    %   Set a directory for where to look for figures


%% Hyperlinks %%%%%%%%%%%%%%%%%%%%%%%%%%%%%%%%%%%%%%%%%%%%
\usepackage{hyperref}
\hypersetup{%
    colorlinks,
        %   This colors the links themselves, not boxes
    citecolor=black,
        %   Everything here and below changes link colors
    filecolor=black,
    linkcolor=black,
    urlcolor=black
}

%% Colors %%%%%%%%%%%%%%%%%%%%%%%%%%%%%%%%%%%%%%%%%%%%%%%

\usepackage{color}
\definecolor{codegreen}{RGB}{28,172,0}
\definecolor{codelilas}{RGB}{170,55,241}


%% Including Code %%%%%%%%%%%%%%%%%%%%%%%%%%%%%%%%%%%%%%%

\usepackage{verbatim}
    %   For including verbatim code from files, no colors
\usepackage{listings}
    %   For including code snippets written directly in this doc

\lstdefinestyle{bash}{%
  language=bash,%
  basicstyle=\ttfamily,%
  showstringspaces=false,%
  commentstyle=\color{gray},%
  keywordstyle=\color{blue},%
  xleftmargin=0.25in,%
  xrightmargin=0.25in
}

\definecolor{codegreen}{RGB}{28,172,0}
\definecolor{codelilas}{RGB}{170,55,241}
\newcommand{\matlabcode}[1]{%
    \lstset{language=Matlab,%
        basicstyle=\footnotesize,%
        breaklines=true,%
        morekeywords={matlab2tikz},%
        keywordstyle=\color{blue},%
        morekeywords=[2]{1}, keywordstyle=[2]{\color{black}},%
        identifierstyle=\color{black},%
        stringstyle=\color{codelilas},%
        commentstyle=\color{codegreen},%
        showstringspaces=false,%
            %   Without this there will be a symbol in
            %   the places where there is a space
        numbers=left,%
        numberstyle={\tiny \color{black}},%
            %   Size of the numbers
        numbersep=9pt,%
            %   Defines how far the numbers are from the text
        emph=[1]{for,end,break,switch,case},emphstyle=[1]\color{red},%
            %   Some words to emphasise
    }%
    \lstinputlisting{#1}
}
    %   For including Matlab code from .m file with colors,
    %   line numbering, etc.

%% Bibliographies %%%%%%%%%%%%%%%%%%%%%%%%%%%%%%%%%%%%

%\usepackage{natbib}
    %---For bibliographies
%\setlength{\bibsep}{3pt} % Set how far apart bibentries are

%% Misc %%%%%%%%%%%%%%%%%%%%%%%%%%%%%%%%%%%%%%%%%%%%%%

\usepackage{enumitem}
    %   Has to do with enumeration
\usepackage{appendix}
%\usepackage{natbib}
    %   For bibliographies
\usepackage{pdfpages}
    %   For including whole pdf pages as a page in doc


%% User Defined %%%%%%%%%%%%%%%%%%%%%%%%%%%%%%%%%%%%%%%%%%

%\newcommand{\nameofcmd}{Text to display}
\newcommand*{\Chi}{\mbox{\large$\chi$}} %big chi
    %   Bigger Chi



%%%%%%%%%%%%%%%%%%%%%%%%%%%%%%%%%%%%%%%%%%%%%%%%%%%%%%%%%%%%%%%%%%%%%%%%
%% BODY %%%%%%%%%%%%%%%%%%%%%%%%%%%%%%%%%%%%%%%%%%%%%%%%%%%%%%%%%%%%%%%%
%%%%%%%%%%%%%%%%%%%%%%%%%%%%%%%%%%%%%%%%%%%%%%%%%%%%%%%%%%%%%%%%%%%%%%%%


\begin{document}
\maketitle

%\tableofcontents

\section{First Order Ordinary Differential Equations}

This section focuses on methods for solving and anlyzing first order
ordinary differential equations (ODEs). These equations relate the value
of a function $y = f(t)$ to expressions involving its derivatives and
the independing variable. Quite generally, they can be written
\begin{align}
  y' = f(y,t)
  \label{eq:ode1}
\end{align}
We might write Equation~\ref{eq:ode1} a bit differential (i.e.\ moving
parts of $f(y,t)$ to the lefthand side) or we might suppose $y'$ is only
a function of $y$ or $t$, but the representation above is sufficient
general to capture all of this.

A first order \emph{initial value problem} (IVP) is an ODE like
Equation~\ref{eq:ode1} paird with initial conditions on the function
such as
\begin{align*}
  y(0) = y_0
  \qquad
  y_0 \in \mathbb{R}
\end{align*}
This will help us pin down a \emph{specific} solution to a differential
equation, since solving the ODE leaves us with a function that is
parameterized by an arbitrary constant. An initial value \emph{implies}
a value for the constant and resolves this free parameter.

Most of the time, the goal is to \emph{solve} the ODE. There are two
general approaches
\begin{enumerate}
  \item Analytic: Manipulate the ODE and integrate to find a satisfying
    function, or make an educated guess and check that it satisfies the
    ODE. This is usually what we mean when we say ``solve a differential
    equation.''
  \item Geometric: Use pictures; draw a direction field and find
    integral curves.
\end{enumerate}

\subsection{Geometric Method}

For any point on $(t,y)$ on the plane, the ODE written as in
Equation~\ref{eq:ode1} tells you the slope at that point. Therefore, we
can visualize the ODE using the following steps:
\begin{enumerate}
  \item Choose a grid of points
  \item Compute the slope at each point on the grid.
  \item Draw tiny lines, centered at each point, with the appropriate
    slope.
\end{enumerate}
This is called a \emph{direction field}.

We graphically ``solve'' the ODE by finding an \emph{integral curve},
through the direction field. The defining feature of an integral curve
is that it is everywhere tangent to line elements in the direction field
(including those we didn't plot). The slope of the curve at every point
equals the slope of the direction field at that point.

A few things about integral curves:
\begin{enumerate}
  \item $y_1$ is a solution to the ODE in Equation~\ref{eq:ode1} if and
    only if it is an integral curve.
  \item They cannot cross at an angle (tangency case handled below).
    Crossing at an angle would imply two different slopes at the same
    point, which is clearly nuts.
  \item If $\frac{\partial f}{\partial t}$ is continuous in the
    neighborhood of a point, then two integral curves cannot even touch
    and be tangent  at that point because of the ``existence and
    uniqueness theorem'' (covered below). So an ODE must have one and
    only solution through that point
  \item In the same way that there are infinitely many solutions that
    differ by a constant in the ``analytic'' approach, there are
    infinitely many integral curves that we might draw on the graph.
    Only an initial value could pin them down.
\end{enumerate}
While I specified a sequence of (computer friendly) steps to draw a
direction field above, there's a more clever way.
\begin{enumerate}
  \item Pick a constant $c$.
  \item By Equation~\ref{eq:ode1}, the formula $c=f(y,t)$ defines an
    \emph{isocline}---a curve in the plane along which the line elements
    of the direction field all have the same slope (namely, $c$).
    Just rearrange $c=f(y,t)$ to get a function $y=g(c,t)$ for the
    isocline. You can also think about the isocline as a contour line
    for the function $f(y,t)$.
  \item Draw the isocline, and along that isocline, fill in line
    elements of slope $c$.
  \item Repeat this process for different choices of $c$. Often,
    choosing nice slopes like $-1,0,1,\ldots$ is a smart thing to do.
\end{enumerate}

\clearpage
\subsection{First Order Linear ODEs}

This is a special case of Representation~\ref{eq:ode1} where the ODE can
be written
\begin{align}
  y' + p(t) y = q(t)
  \label{eq:linear}
\end{align}
If there are no outside signals (if $q(t)=0$), then we call the ODE
\emph{homogeneous} and can be solved by separation of variables.  Of
course, not every first order ODE can be written this way, but for those
that can, we will develop a method to solve them very easily.

On top of that, ODEs in this form have a very nice ``system and
signals'' physical interpretation:
\begin{enumerate}
  \item The \emph{system} is on the LHS, and it's some combination of
    the dependent variable $y$ and its derivatives.
  \item A \emph{signal} is any dependent variable---any function of the
    independent variable $t$.
  \item The RHS describes the \emph{input signal}---outside influences
    \emph{driving} the system on the LHS.
  \item The solution $y$ is called the \emph{output signal} or
    \emph{system response}. It depends upon $t$ so it too is a signal.
\end{enumerate}

We solve ODEs in form Equation~\ref{eq:linear} by defining an
\emph{integrating factor} $u(t)$ that will multiply both sides of
Equation~\ref{eq:linear}:
\begin{align*}
  u(t) y' + u(t)p(t) y = u(t) q(t)
\end{align*}
The integrating factor should be such that the LHS is the derivative of
something:
\begin{align*}
  u(t) y' + u(t) p(t) y = \left( u(t) y\right)'
\end{align*}
In other words, to express the LHS as the derivative of $u(t)y$, it
should be the case that
\begin{align*}
  u' = u(t) p(t)
\end{align*}
which can be solved by separation of variables:
\begin{align*}
  u(t) = e^{\int p(t) \; dt}
\end{align*}
With this in hand, you can solve Equation~\ref{eq:linear} as follows:
\begin{align*}
  y' + p(t) y &= q(t) \\
  u(t) y' + u'(t) y &= q(t)u(t)\\
  \left(u(t) y\right)' &= q(t)u(t) \\
  \int \left(u(t) y\right)' &= \int q(t)u(t) \\
  u(t) y &= \int q(t)u(t) + c\\
\end{align*}


\clearpage
\section{Numerical Methods}

Often, we can't solve ODE's analytically or geometrically, so we resort
to numerical approximations.

\subsection{Euler's Method}

We have first order ODE
\begin{align}
  y' = f(y,t)
  \label{eq:euler}
\end{align}
Here's the basic scheme
\begin{enumerate}
  \item Start with the initial value $(t_0, y_0)$.
  \item Choose a step size $h$.
  \item Given starting point, $(t_n, y_n)$ compute $y'_n = f(t_n, y_n)$.
  \item Set $(t_{n+1}, y_{n+1}) = (t_n + h, t_n + y'_n*h)$
  \item Go back to Step 3 and repeat, starting with $(t_{n+1},y_{n+1}$.
\end{enumerate}
Of course, we want to evaluate the numerical method. That means asking
``How close is the numerical method to the actual?'' Well it's quite
easy to see that
\begin{enumerate}
  \item If the solution is convex ($y'' > 0$), the approximation will be
    too low
  \item If the solution is concave ($y'' < 0$), the approximation will
    be too high
\end{enumerate}
But this seems like we need to \emph{solve} the equation to determine if
$y$ is concave or convex, and the whole point of Euler is that we
\emph{couldn't} solve Equation~\ref{eq:euler}. But actually we can. Just
\emph{use} Equation~\ref{eq:euler} to analyze the higher derivatives.

   %1. You can determine if the solution is concave or convex by
   %differentiating the first-order ODE itself to get at second
   %derivatives
%1. Errors depend upon step size
   %1. error ~ c*h (on the order of h)
   %2. If you halve the step size, you approximately halve the error
%1. Heun’s Method, aka RK2 (Runge-Kutta)
   %1. This is a second order method, so e ~ c*(h^2)
   %2. Halve the step size, goes down by a factor of 4
   %3. Process
      %1. x_{n+1} = x_n + h still
      %2. Call a_{n} the slope at (x_n, y_n)
      %3. Call b_n the slope at (x_n+h, y_n + h*a_n)
      %4. Set y_{n+1} = y_n + h*(a_n + b_n)/2
   %1. Intuition: Like Euler’s method but you average the slope over the next two steps of the usual Euler method to get the slope for this jump
   %2. Requires computing the function twice
%1. RK4
   %1. Requires 4 evaluations of the function
   %2. Very accurate; standard method
   %3. Errors on the order e ~ c*(h^4); so you halve the step size, reduce the error by a factor of 16
   %4. (A_n + 2*B_n + 2*C_n + D_n)/6
%1. Pitfalls
   %1. must watch out for pitfalls when the function blows up
   %2. You often can’t predict singularities



%% APPPENDIX %%

% \appendix




\end{document}


%%%%%%%%%%%%%%%%%%%%%%%%%%%%%%%%%%%%%%%%%%%%%%%%%%%%%%%%%%%%%%%%%%%%%%%%
%%%%%%%%%%%%%%%%%%%%%%%%%%%%%%%%%%%%%%%%%%%%%%%%%%%%%%%%%%%%%%%%%%%%%%%%
%%%%%%%%%%%%%%%%%%%%%%%%%%%%%%%%%%%%%%%%%%%%%%%%%%%%%%%%%%%%%%%%%%%%%%%%

%%%% SAMPLE CODE %%%%%%%%%%%%%%%%%%%%%%%%%%%%%%%%%%%%%%

    %% VIEW LAYOUT %%

        \layout

    %% LANDSCAPE PAGE %%

        \begin{landscape}
        \end{landscape}

    %% BIBLIOGRAPHIES %%

        \cite{LabelInSourcesFile}  %Use in text; cites
        \citep{LabelInSourcesFile} %Use in text; cites in parens

        \nocite{LabelInSourceFile} % Includes in refs w/o specific citation
        \bibliographystyle{apalike}  % Or some other style

        % To ditch the ``References'' header
        \begingroup
        \renewcommand{\section}[2]{}
        \endgroup

        \bibliography{sources} % where sources.bib has all the citation info

    %% SPACING %%

        \vspace{1in}
        \hspace{1in}

    %% URLS, EMAIL, AND LOCAL FILES %%

      \url{url}
      \href{url}{name}
      \href{mailto:mcocci@raidenlovessusie.com}{name}
      \href{run:/path/to/file.pdf}{name}


    %% INCLUDING PDF PAGE %%

        \includepdf{file.pdf}


    %% INCLUDING CODE %%

        %\verbatiminput{file.ext}
            %   Includes verbatim text from the file

        \texttt{text}
            %   Renders text in courier, or code-like, font

        \matlabcode{file.m}
            %   Includes Matlab code with colors and line numbers

        \lstset{style=bash}
        \begin{lstlisting}
        \end{lstlisting}
            % Inline code rendering


    %% INCLUDING FIGURES %%

        % Basic Figure with size scaling
            \begin{figure}[h!]
               \centering
               \includegraphics[scale=1]{file.pdf}
            \end{figure}

        % Basic Figure with specific height
            \begin{figure}[h!]
               \centering
               \includegraphics[height=5in, width=5in]{file.pdf}
            \end{figure}

        % Figure with cropping, where the order for trimming is  L, B, R, T
            \begin{figure}
               \centering
               \includegraphics[trim={1cm, 1cm, 1cm, 1cm}, clip]{file.pdf}
            \end{figure}

        % Side by Side figures: Use the tabular environment


