\documentclass[12pt]{article}

\author{Matthew Cocci}
\title{\textbf{Homework 5}}
\date{\today}

%% Spacing %%%%%%%%%%%%%%%%%%%%%%%%%%%%%%%%%%%%%%%%%%%%%%%%

\usepackage{fullpage}
\usepackage{setspace}
%\onehalfspacing
\usepackage{microtype}


%% Header %%%%%%%%%%%%%%%%%%%%%%%%%%%%%%%%%%%%%%%%%%%%%%%%%

%\pagestyle{fancy} 
%\lhead{}
%\rhead{}
%\chead{}
%\setlength{\headheight}{15.2pt} 
    %---Make the header bigger to avoid overlap

%\renewcommand{\headrulewidth}{0.3pt} 
    %---Width of the line

%\setlength{\headsep}{0.2in}    
    %---Distance from line to text
            

%% Mathematics Related %%%%%%%%%%%%%%%%%%%%%%%%%%%%%%%%%%%

\usepackage{amsmath}
\usepackage{amsfonts}
\usepackage{mathrsfs}
\usepackage{amsthm} %allows for labeling of theorems
\theoremstyle{plain}
\newtheorem{thm}{Theorem}[section]
\newtheorem{lem}[thm]{Lemma}
\newtheorem{prop}[thm]{Proposition}
\newtheorem{cor}[thm]{Corollary}

\theoremstyle{definition}
\newtheorem{defn}[thm]{Definition}
\newtheorem{ex}[thm]{Example}

\theoremstyle{remark}
\newtheorem*{rem}{Remark}
\newtheorem*{note}{Note}


%% Font Choices %%%%%%%%%%%%%%%%%%%%%%%%%%%%%%%%%%%%%%%%%

\usepackage[T1]{fontenc}
\usepackage[utf8]{inputenc}
\usepackage{lmodern}
%\usepackage{blindtext}


%% Figures %%%%%%%%%%%%%%%%%%%%%%%%%%%%%%%%%%%%%%%%%%%%%%

\usepackage{graphicx}
\usepackage{subfigure} 
    %---For plotting multiple figures at once
%\graphicspath{ {Directory/} }
    %---Set a directory for where to look for figures


%% Hyperlinks %%%%%%%%%%%%%%%%%%%%%%%%%%%%%%%%%%%%%%%%%%%%
\usepackage{hyperref} 
\hypersetup{	
    colorlinks,		
        %---This colors the links themselves, not boxes
    citecolor=black,	
        %---Everything here and below changes link colors
    filecolor=black,
    linkcolor=black,
    urlcolor=black
}

%% Including Code %%%%%%%%%%%%%%%%%%%%%%%%%%%%%%%%%%%%%%% 

\usepackage{verbatim} 
    %---For including verbatim code from files, no colors

\usepackage{listings}
\usepackage{color}
\definecolor{mygreen}{RGB}{28,172,0}
\definecolor{mylilas}{RGB}{170,55,241}
\newcommand{\matlabcode}[1]{%
    \lstset{language=Matlab,%
        basicstyle=\footnotesize,%
        breaklines=true,%
        morekeywords={matlab2tikz},%
        keywordstyle=\color{blue},%
        morekeywords=[2]{1}, keywordstyle=[2]{\color{black}},%
        identifierstyle=\color{black},%
        stringstyle=\color{mylilas},%
        commentstyle=\color{mygreen},%
        showstringspaces=false,%
            %---Without this there will be a symbol in 
            %---the places where there is a space
        numbers=left,%
        numberstyle={\tiny \color{black}},% 
            %---Size of the numbers
        numbersep=9pt,% 
            %---Defines how far the numbers are from the text
        emph=[1]{for,end,break,switch,case},emphstyle=[1]\color{red},%
            %---Some words to emphasise
    }%
    \lstinputlisting{#1}
}
    %---For including Matlab code from .m file with colors,
    %---line numbering, etc. 


%% Misc %%%%%%%%%%%%%%%%%%%%%%%%%%%%%%%%%%%%%%%%%%%%%% 

\usepackage{enumitem} 
    %---Has to do with enumeration	
\usepackage{appendix}
%\usepackage{natbib} 
    %---For bibliographies
\usepackage{pdfpages}
    %---For including whole pdf pages as a page in doc


%% User Defined %%%%%%%%%%%%%%%%%%%%%%%%%%%%%%%%%%%%%%%%%% 

%\newcommand{\nameofcmd}{Text to display}
\newcommand*{\Chi}{\mbox{\large$\chi$}} %big chi



%%%%%%%%%%%%%%%%%%%%%%%%%%%%%%%%%%%%%%%%%%%%%%%%%%%%%%%%%%%%%%%%%%%%%%%% 
%% BODY %%%%%%%%%%%%%%%%%%%%%%%%%%%%%%%%%%%%%%%%%%%%%%%%%%%%%%%%%%%%%%%%
%%%%%%%%%%%%%%%%%%%%%%%%%%%%%%%%%%%%%%%%%%%%%%%%%%%%%%%%%%%%%%%%%%%%%%%% 


\begin{document}

\maketitle 

\begin{enumerate} 

% Question 1
\item \textbf{Exercise 88.5}: $(X,\mathscr{M},\mu)$ is a positive measure space, and we then define a relative measure space $(E,\mathscr{M}_E,\lambda)$ where $\lambda(A)=\mu(A\cap E)$.

\begin{enumerate}
% Question 1a
\item For $f\in\mathscr{L}(X,\mathscr{M},\mu)$, we want to show that $f|_E \in\mathscr{L}(E,\mathscr{M}_E,\lambda)$ and that
\[
    \int_E f|_E \; d\lambda = \int_X f \Chi_E\; d\mu
\]
\begin{enumerate}
    \item First, we know show that $f|_E$ is $\mathscr{M}_E$ measurable so that the integral of $f|_E$ makes sense. To do so, consider 
\[
    S_a = f|_E^{-1}([a,\infty]) = \{x \in E \; | \; f|_E(x)\in [a,\infty]\}
\]

\end{enumerate}

Then show $f|_E \leq f$ for all $E\subset X$, so that $\int_X f|_E \; d\mu$ will be integrable. (Start with nonneg functions, then generalize.) 

Now if we show $\int_E f|_E \; d\lambda$ equals that value, the integral is finite, so the result is achieved.

$(\leq)$ First show that $f\chi_E$ is measurable, and that the integral is finite. Then take any simple function for the $E$ case, unpack it, to get it into $X$ space, then take the sup.

$(\geq)$ Unpack the righthand side, sup the left.

% 1b 
\item Show that $g$ is measurable in $X$ space.  

\end{enumerate}

% Question 2
\item In both, we have a positive measure space, $(X,\mathscr{M},\mu)$, with $\mu(X)<\infty$.

\begin{enumerate} 

% Question 2a
\item \textbf{Exercise 88.6}: We have a sequence of measurable functions $\{f_n\}_1^\infty$ such that $f=\lim f_n$. We also know that the $f_n$ are bounded so that $|f_n(x)|\leq M$ for all $n$ and $x$. 

This setup lends itself nicely to using the Dominated Convergence Theorem.  Specifically, set $g(x)=M$. A constant function is clearly measurable, and because $\mu(X)<\infty$ we know from class and by Theorem 88.4 in the book that $\int_X g \; d\mu$ exists (and is finite) so that $g\in\mathscr{L}(X,\mathscr{M},\mu)$.

Thus we satisfy the conditions of the Dominated Convergence Theorem because $f_n$ is measurable, $\lim f_n$ exists for all $x$, and $|f_n|\leq g\in\mathscr{L}(X,\mathscr{M},\mu)$. Thus, 
\[
  \lim_{n\rightarrow\infty} \int_X f_n \; d\mu  
  = \int_X \lim_{n\rightarrow\infty} f_n \; d\mu
  = \int_X  f \; d\mu
\]

% Question 2b
\item \textbf{Exercise 88.10}: For this question, we have a sequence of bounded real measurable functions $\{f_n\}^\infty_1$ on $X$ such that $f_n\rightarrow f$ uniformly. We again want to show that 
\begin{equation}
    \lim_{n\rightarrow\infty} \int_X f_n \; d\mu
    = \int_X f \; d\mu
\end{equation}
    
($\leq$ Direction) First, we know that because $f_n\rightarrow f$ uniformly, $f$ is measurable, so we can integrate. We also know that for any $\varepsilon>0$, we can find an $N$ such that 
\begin{equation}
    \label{stillholds}
    f_n \in (f-\varepsilon, f+\varepsilon) 
    \qquad \text{For all $n>N$ and $x\in X$}
\end{equation}
And so we know that $f-\varepsilon\leq f_N\leq f+\varepsilon$, to which we can apply the monotonicity of the integral
\begin{equation}
    \label{q2.tolim}
    \int_X (f-\varepsilon) \; d\mu\leq
    \int_X f_N \; d\mu \leq \int_X (f+\varepsilon) \; d\mu
\end{equation}
Note, both $f-\varepsilon$ and $f+\varepsilon$ are measurable because they're the difference and sum of two measurable functions: $f$ and a constant, $\varepsilon$. 
\\
\\
Now take the limit of Inequality \ref{q2.tolim} with respect to $N$. 
\begin{align*}
    \lim_{N\rightarrow\infty} \int_X (f-\varepsilon) \; d\mu
    \leq \lim_{N\rightarrow\infty}
    \int_X f_N \; d\mu \leq 
    \lim_{N\rightarrow\infty}\int_X (f+\varepsilon) \; d\mu
\end{align*}
The inequality will still hold because Statement (\ref{stillholds}) holds for all $n>N$. And since outer integrals don't depending upon $N$, we gett
\begin{equation}
    \int_X (f-\varepsilon) \; d\mu \leq
    \lim_{N\rightarrow\infty}\int_X f_N \; d\mu \leq 
    \int_X (f+\varepsilon) \; d\mu
\end{equation}
But recall $\varepsilon$ was arbitrary, and the middle integral no longer depends upon $\varepsilon$, so take the limit as $\varepsilon\rightarrow0$ to get that 
\begin{align*}
    \int_X f \; d\mu \leq
    \lim_{N\rightarrow\infty}\int_X f_N \; d\mu &\leq 
    \int_X f \; d\mu \\
    \Rightarrow\qquad
    \lim_{N\rightarrow\infty}\int_X f_N \; d\mu &=
    \int_X f \; d\mu 
\end{align*}
Finally, these integrals are finite, which can be seen by the finiteness of the righthand side. Namely, if $\{f_n\}$ converges uniformly to $f$ and the $f_n$ are bounded, then $f$ must be bounded. That means that $|f|$ is a bounded nonnegative function on $X$ which, together with $\mu(X)<\infty$, implies $|f|\in\mathscr{L}(X,\mathscr{M},\mu)$. This, in turn, implies $f\in\mathscr{L}(X,\mathscr{M},\mu)$ as well.
\end{enumerate}

\textbf{Discussion}: At the surface, part (b) looks very similar to part (a); however, there is one key difference: The sequence is not necessarily bounded. We know the functions are bounded, so for each $f_n$, there exists a $M_n$ such that $|f_n|\leq M_n$. But we cannot necessarily find a \emph{common} $M$ such that all functions are bounded by it, which disallows the application of the Dominated Convergence Theorem in part (a).

% Question 3
\newpage
\item \textbf{Exercise 88.7} In the following problems, we replace within the theorems the condition that must hold ``for all $x$'' with ``for \emph{almost} all $x$'' with respect to $\mu$.
\begin{enumerate}
    \item \textbf{Theorem 88.6}: We want to show that the MTC holds if $f_n(x)\leq f_{n+1}(x)$ for \emph{almost} all $x$, rather than all $x$. So define the set
\begin{align*}
    E &= \{x \; | \; f_n > f_{n+1}\} \\
        &= \{x \; | \; f_n - f_{n+1} > 0\} \\
        &= \{x \; | \; g(x) > 0\} \qquad \qquad
        \text{where $g=f_n - f_{n+1}$} \\
    &= g^{-1}((0,\infty]) 
\end{align*}
Now because $f_n$ is measurable for all $n$, we know that $g=f_n-f_{n+1}$ will be measurable. And because $g$ is measurable, the set $E$ must be in the $\sigma$-algebra $\mathscr{M}$, which implies $X\setminus E\in\mathscr{M}$ as well.
\\
\\
Next, note that we can rewrite the integral of $f_n$ with respect to these two sets within the $\sigma$-algebra:
\begin{align*}
    \int_X f_n \; d\mu &= \int_{X\setminus E} f_n \; d\mu
    + \int_E f_n \; d\mu\\
    &= \int_X f_n \Chi_{X\setminus E} \; d\mu+ 0
\end{align*}
because, by assumption, $\mu(E)=0$, which implies the integral over $E$ is zero for all $f_n$ and, even more generally, for \emph{all} measurable functions, which will include $\lim f_n$ below. 

Next, notice that the integral of $f_n\cdot\Chi_{X\setminus E}$ over $X$ satisifies all the criteria for applying the Monotone Convergence Theorem: The $f_n\cdot\Chi_{X\setminus E}$ are measurable, they converge to $f\cdot\Chi_{X\setminus E}$, and $f_n\cdot\Chi_{X\setminus E}\leq f_{n+1}\cdot\Chi_{X\setminus E}$ for all $x\in X\setminus E$, by construction. 
\\
\\
So we can apply the Monotone Converge Theorem to this set to derive
\[
    \lim_{n\rightarrow\infty}\int_{X\setminus E} f_n \; d\mu=
    \int_{X\setminus E} \lim_{n\rightarrow\infty}f_n \; d\mu
\]
We then use this in the final steps of the proof:
\begin{align*}
    \lim_{n\rightarrow\infty}\int_{X} f_n \; d\mu&=
    \lim_{n\rightarrow\infty}\int_{X\setminus E} f_n \; d\mu
    +
    \lim_{n\rightarrow\infty}\int_{E} f_n \; d\mu \\
    &= 
    \int_{X\setminus E} \lim_{n\rightarrow\infty}f_n \; d\mu
    + 0 \\
    &= 
    \int_{X\setminus E} \lim_{n\rightarrow\infty}f_n \; d\mu
    + \int_{E} \lim_{n\rightarrow\infty}f_n \; d\mu\\
    &= \int_{X} \lim_{n\rightarrow\infty}f_n \; d\mu
\end{align*}



\end{enumerate}


% Question 4
\newpage
\item We have $f\in\mathscr{L}(X,\mathscr{M},\mu)$, and want to show that for all $\varepsilon>0$, there exists a $\delta>0$ such that 
\[
    \mu(E)<\delta \quad \text{for $E\in\mathscr{M}$ implies}
    \quad \int_E |f| \; d\mu <\varepsilon
\]
We will proceed via proof by contradiction.

As in the hint, take $\varepsilon>0$ and let $\{E_n\}^\infty_1 \subset \mathscr{M}$ such that $\lim_{n\rightarrow\infty} \mu(E_n)=0$. We know such a set exists because we always have $\emptyset \in \mathscr{M}$, where $\mu(\emptyset)$ always. Also, suppose by contradiction that for all $\delta$, there exists a $\varepsilon>0$ such that 
\begin{equation}
    \label{q4}
    \varepsilon\leq \int_{E_n} |f| \; d\mu :=
    \int_X |f|\Chi_{E_n} \; d\mu
    \qquad\text{for all $n$}
\end{equation}
Next, define a sequence $\{f_n\}=\left\{|f|\chi_{E_n}\right\}$. A couple facts about our sequence and about $|f|$:
\begin{itemize}
    \item $|f|\in\mathscr{L}(X,\mathscr{M},\mu)$. By Theorem 88.12, our assumption that $f\in\mathscr{L}(X,\mathscr{M},\mu)$ implies this result.
    \item $f_n$ is measurable, for all $n$. To see this, note that since $|f|\in\mathscr{L}(X,\mathscr{M},\mu)$, $|f|$ is measurable, and since $E_n\in\mathscr{M}$, $f_n = |f|\chi_{E_n}$ is also measurable, for all $n$.
    \item $f_n \leq |f|$: This is fairly obvious, after writing out the definition of $f_n$, 
\begin{equation}
    \label{dtcapply}
    f_n := |f|\Chi_{E_n}\leq |f|
\end{equation}
    \item $\lim_{n\rightarrow\infty}f_n=0$. Note that since $\mu(E_n)\rightarrow 0$, it must be that $\Chi_{E_n}\rightarrow 0$. 
\end{itemize}
So, let's use these facts to apply the Dominated Convergence Theorem. 

First, since $\{f_n\}$ is a measurable sequence of functions, bounded by $|f|\in\mathscr{L}(X,\mathscr{M},\mu)$ for all $n$, we know that we can apply the Dominated Convergence Theorem, and exchange the limit and integral:
\begin{align*}
    \lim_{n\rightarrow\infty} \int_X f_n \; d\mu&=
    \int_X \lim_{n\rightarrow\infty}f_n \; d\mu  \\ 
\end{align*}
Then, we can evaluate the limit within the integral, applying what was stated above:
\begin{align*}
    \lim_{n\rightarrow\infty} \int_X f_n \; d\mu &=
    \int_X 0 \; d\mu = 0 \\ 
    \Rightarrow\quad
    \lim_{n\rightarrow\infty} \int_X |f|\Chi_{E_n} \; 
    d\mu &= 0
\end{align*}
But this contradicts our assumption in Inequality \ref{q4}, which should hold for all $n$.


% Quesiton 6
\newpage 
\item \textbf{Exercise 88.15}: For a sequence of measurable functions $\{f_n\}$, we're told that $\sum^\infty_{n=1} |f_n(x)|$ converges for all $x\in X$, and that $\sum^\infty_{n=1}|f_n(x)|\in\mathscr{L}(X,\mathscr{M},\mu)$. 
\begin{enumerate}
    % Part 6a
    \item First, we want to prove that $\sum^\infty_{n=1} f_n(x)$ converges for all $x\in X$. Begin by noting that 
\begin{equation}
    \label{q6a.1}
    \sum^\infty_{n=1} f_n(x) \leq 
    \left\lvert\sum^\infty_{n=1} f_n(x)\right\rvert =
    c \sum^\infty_{n=1} f_n(x)
\end{equation}
where $c=1$ or $-1$. Then, it's clear that 
\begin{equation}
    \label{q6a.2}
    c \sum^\infty_{n=1} f_n(x) =
    \sum^\infty_{n=1} c f_n(x) \leq
    \sum^\infty_{n=1} \left\lvert f_n(x) \right\rvert
    <\infty
\end{equation}
where the final inequality follows from the assumption that $\sum^\infty_{n=1} |f_n(x)|$ converges for all $x$. Putting (\ref{q6a.1}) and (\ref{q6a.2}) together, we see
\begin{equation}
    \sum^\infty_{n=1} f_n(x) \leq 
    \sum^\infty_{n=1} \left\lvert f_n(x) \right\rvert
    <\infty
\end{equation}
so that $\sum^\infty_{n=1} f_n(x)$ converges for all $x$.

% Part 6b
\item Next, we want to show that $\sum^\infty_{n=1} f_n\in \mathscr{L}(X,\mathscr{M},\mu)$. But to show this, we just need to show that $\left\lvert\sum^\infty_{n=1}f_n\right\rvert\in\mathscr{L}(X,\mathscr{M},\mu)$, and then apply Theorem 88.12.
\\
\\
But first, a result from class states that because the $f_n$ are measurable, and because $\sum^\infty_{n=1} f_n$ converges (by part (a)), we know that the sum is measurable, which implies the absolute value $\left\lvert\sum^\infty_{n=1}f_n\right\rvert$ is measurable too. 
\\
\\
Next, note from (\ref{q6a.1}) and (\ref{q6a.2}) that 
\begin{align*}
    0\leq 
    \left\lvert \sum^\infty_{n=1} f_n(x)\right\rvert \leq
    \sum^\infty_{n=1} \left\lvert 
        f_n(x)\right\rvert 
\end{align*}
Then use monotonicity of the integral on these measurable functions to get
\begin{align*}
    \int_X 0\;d\mu &\leq 
    \int_X 
    \left\lvert \sum^\infty_{n=1} f_n(x)\right\rvert 
    \; d\mu
    \leq
    \int_X\sum^\infty_{n=1} \left\lvert 
        f_n(x)\right\rvert \;d\mu < \infty \\
    \Rightarrow \quad
    0 &\leq
    \int_X 
    \left\lvert \sum^\infty_{n=1} f_n(x)\right\rvert 
    \; d\mu <\infty
\end{align*}
And because the integral is finite, we know that $\left\lvert \sum^\infty_{n=1} f_n \right\rvert\in\mathscr{L}(X,\mathscr{M},\mu)$, which implies $\sum^\infty_{n=1} f_n \in\mathscr{L}(X,\mathscr{M},\mu)$ by Theorem 88.12.

% Q6c 
\item Finally, we want to prove that 
\begin{equation}
    \label{q6c.target}
    \int_X \left( \sum^\infty_{n=1} f_n\right) d\mu
    = \sum^\infty_{n=1} \int_X f_n \; d\mu
\end{equation}
So start by defining the partial sum, which is a measurable function as a finite sum of measurable functions:
\begin{equation}
    g_m = \sum^m_{n=1} f_n
\end{equation}
Note that, by what we showed in part (a), we have 
\begin{equation}
    \lim_{m\rightarrow\infty} g_m = 
    \lim_{m\rightarrow\infty} \sum^m_{n=1} <\infty
    \quad \Rightarrow\quad
    g_m\rightarrow \sum^\infty_{n=1} f_n(x)
\end{equation}
Now rewrite Equality \ref{q6c.target} in a more suggestive, but equivalent statement
\begin{align}
    \label{q6.crux}
    \int_X \lim_{m\rightarrow\infty} g_m \; d\mu =
    \lim_{m\rightarrow\infty} \int_X g_m \; d\mu
\end{align}
The first term is obviously equilvalent to that of Equality \ref{q6c.target}, and the second can be shown as equivalent because, for any $m$, we can apply Theorem 88.11 to write
\[
    \int_X g_m \; d\mu 
    = \int_X \sum^m_{n=1}f_n \; d\mu
    = \sum^m_{n=1}\int_X f_n \; d\mu
\]
And while the sequence $\{g_m\}$ is a sequence of measurable functions, we just need to show that we can bound the sequence for all $m$ and $x$, after which we can apply the Dominated Convergence Theorem.
\\
\\
So start by using facts from and simplifications similar to parts (a) and (b)
\[
    |g_m| =
    \left\lvert \sum^m_{n=1} f_n \right\rvert \leq 
    \sum^m_{n=1} \left\lvert f_n \right\rvert \leq 
    \sum^\infty_{n=1} \left\lvert f_n \right\rvert
    <\infty
\]
We now have a sequence of measurable functions $\{g_m\}$ whose limit exists for all $x\in X$, since $\sum^\infty_{n=1} f_n$ is convergent. We also have a function, $\sum^\infty_{n=1}|f_n|\in\mathscr{L}(X,\mathscr{M},\mu)$ bounding $\{g_m\}$. Then, we can apply the Dominated Convergence Theorem to say that Equation \ref{q6.crux} holds, a statement which is exactly equivalent to what we want to prove: Equation \ref{q6c.target}.





    


\end{enumerate}



\newpage
% Question 7
\item We start with the function $|f|>0$, which is measurable because $f$ is integrable and $|f|$ is a continuous transformation of $f$. We also know that $|f|\in\mathscr{L}(X,\mathscr{M},\mu)$ by Theorem 88.12. Now consider the set
\[
    A_1 = |f|^{-1}([t,\infty]) = \{x \;:\; |f(x)|\geq t\}
    \qquad t>0
\]
Next, let $A_2 = X\setminus A_1$. 

We now construct a simple function $s$:
\[
    s = t\cdot \Chi_{A_1} + 0 \cdot \Chi_{A_2}
\]
First, it's obvious that $0\leq s$ because $t>0$. Second, it's obvious that $s\leq |f|$, because we've partitioned the range of $|f|$ into two segments: the parts greater than $t$ (which are mapped from $A_1$) and everything else (which cannot be less than zero).

Now, by the definition of the integral, we have that 
\[
    \int_X |f| \; d\mu = \sup_{0\leq s \leq f} \int_X s\;d\mu
    \qquad\Rightarrow\qquad
    \int_X |f| \; d\mu \geq \int_X s\;d\mu
\]
where the $s$ in the second inequality is the $s$ defined above. So now, let's compute it:
\begin{align*}
    \int_X |f| \; d\mu \geq \int_X s\;d\mu
    &= t \mu(A_1) + 0 \mu(A_2) \\
    &= t \mu(A_1) 
\end{align*}
Rearranging, and substituting in for $A_1$, we see that
\begin{equation}
    \mu\left(\{x \;:\; |f(x)|\geq t\}\right) \leq
    \frac{1}{t}\int_X |f| \; d\mu 
\end{equation}




\end{enumerate}

\end{document}



%%%% INCLUDING FIGURES %%%%%%%%%%%%%%%%%%%%%%%%%%%%

   % H indicates here 
   %\begin{figure}[h!]
  %   \centering
   %   \includegraphics[scale=1]{file.pdf}
   %\end{figure}

%   \begin{figure}[h!]
%      \centering
%      \mbox{
%	 \subfigure{
%	    \includegraphics[scale=1]{file1.pdf}
%	 }\quad
%	 \subfigure{
%	    \includegraphics[scale=1]{file2.pdf} 
%	 }
%      }
%   \end{figure}
 

%%%%% Including Code %%%%%%%%%%%%%%%%%%%%%5
% \verbatiminput{file.ext}    % Includes verbatim text from the file
% \texttt{text}	  % includes text in courier, or code-like, font
