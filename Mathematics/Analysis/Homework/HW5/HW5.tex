\documentclass[12pt]{article}

\author{Matthew Cocci}
\title{\textbf{Homework 5}}
\date{\today}

%% Spacing %%%%%%%%%%%%%%%%%%%%%%%%%%%%%%%%%%%%%%%%%%%%%%%%

\usepackage{fullpage}
\usepackage{setspace}
%\onehalfspacing
\usepackage{microtype}


%% Header %%%%%%%%%%%%%%%%%%%%%%%%%%%%%%%%%%%%%%%%%%%%%%%%%

%\pagestyle{fancy} 
%\lhead{}
%\rhead{}
%\chead{}
%\setlength{\headheight}{15.2pt} 
    %---Make the header bigger to avoid overlap

%\renewcommand{\headrulewidth}{0.3pt} 
    %---Width of the line

%\setlength{\headsep}{0.2in}    
    %---Distance from line to text
            

%% Mathematics Related %%%%%%%%%%%%%%%%%%%%%%%%%%%%%%%%%%%

\usepackage{amsmath}
\usepackage{amsfonts}
\usepackage{mathrsfs}
\usepackage{amsthm} %allows for labeling of theorems
\theoremstyle{plain}
\newtheorem{thm}{Theorem}[section]
\newtheorem{lem}[thm]{Lemma}
\newtheorem{prop}[thm]{Proposition}
\newtheorem{cor}[thm]{Corollary}

\theoremstyle{definition}
\newtheorem{defn}[thm]{Definition}
\newtheorem{ex}[thm]{Example}

\theoremstyle{remark}
\newtheorem*{rem}{Remark}
\newtheorem*{note}{Note}


%% Font Choices %%%%%%%%%%%%%%%%%%%%%%%%%%%%%%%%%%%%%%%%%

\usepackage[T1]{fontenc}
\usepackage[utf8]{inputenc}
\usepackage{lmodern}
%\usepackage{blindtext}


%% Figures %%%%%%%%%%%%%%%%%%%%%%%%%%%%%%%%%%%%%%%%%%%%%%

\usepackage{graphicx}
\usepackage{subfigure} 
    %---For plotting multiple figures at once
%\graphicspath{ {Directory/} }
    %---Set a directory for where to look for figures


%% Hyperlinks %%%%%%%%%%%%%%%%%%%%%%%%%%%%%%%%%%%%%%%%%%%%
\usepackage{hyperref} 
\hypersetup{	
    colorlinks,		
        %---This colors the links themselves, not boxes
    citecolor=black,	
        %---Everything here and below changes link colors
    filecolor=black,
    linkcolor=black,
    urlcolor=black
}

%% Including Code %%%%%%%%%%%%%%%%%%%%%%%%%%%%%%%%%%%%%%% 

\usepackage{verbatim} 
    %---For including verbatim code from files, no colors

\usepackage{listings}
\usepackage{color}
\definecolor{mygreen}{RGB}{28,172,0}
\definecolor{mylilas}{RGB}{170,55,241}
\newcommand{\matlabcode}[1]{%
    \lstset{language=Matlab,%
        basicstyle=\footnotesize,%
        breaklines=true,%
        morekeywords={matlab2tikz},%
        keywordstyle=\color{blue},%
        morekeywords=[2]{1}, keywordstyle=[2]{\color{black}},%
        identifierstyle=\color{black},%
        stringstyle=\color{mylilas},%
        commentstyle=\color{mygreen},%
        showstringspaces=false,%
            %---Without this there will be a symbol in 
            %---the places where there is a space
        numbers=left,%
        numberstyle={\tiny \color{black}},% 
            %---Size of the numbers
        numbersep=9pt,% 
            %---Defines how far the numbers are from the text
        emph=[1]{for,end,break,switch,case},emphstyle=[1]\color{red},%
            %---Some words to emphasise
    }%
    \lstinputlisting{#1}
}
    %---For including Matlab code from .m file with colors,
    %---line numbering, etc. 


%% Misc %%%%%%%%%%%%%%%%%%%%%%%%%%%%%%%%%%%%%%%%%%%%%% 

\usepackage{enumitem} 
    %---Has to do with enumeration	
\usepackage{appendix}
%\usepackage{natbib} 
    %---For bibliographies
\usepackage{pdfpages}
    %---For including whole pdf pages as a page in doc


%% User Defined %%%%%%%%%%%%%%%%%%%%%%%%%%%%%%%%%%%%%%%%%% 

%\newcommand{\nameofcmd}{Text to display}
\newcommand*{\Chi}{\mbox{\large$\chi$}} %big chi



%%%%%%%%%%%%%%%%%%%%%%%%%%%%%%%%%%%%%%%%%%%%%%%%%%%%%%%%%%%%%%%%%%%%%%%% 
%% BODY %%%%%%%%%%%%%%%%%%%%%%%%%%%%%%%%%%%%%%%%%%%%%%%%%%%%%%%%%%%%%%%%
%%%%%%%%%%%%%%%%%%%%%%%%%%%%%%%%%%%%%%%%%%%%%%%%%%%%%%%%%%%%%%%%%%%%%%%% 


\begin{document}

\maketitle 

\begin{enumerate} 

% Question 1
\item \textbf{Exercise 88.5}: $(X,\mathscr{M},\mu)$ is a positive measure space, and we then define a relative measure space $(E,\mathscr{M}_E,\lambda)$ where $\lambda(A)=\mu(A\cap E)$.

\begin{enumerate}
% Question 1a
\item For $f\in\mathscr{L}(X,\mathscr{M},\mu)$, we want to show that $f|_E \in\mathscr{L}(E,\mathscr{M}_E,\lambda)$ and that
\[
    \int_E f|_E \; d\lambda = \int_X f \Chi_E\; d\mu
\]
First, show that $f|_E$ is $\mathscr{M}_E$ measurable (previous exercise), and then that $f\chi_E$ is $\mathscr{M}$ measurable.

Then show $f|_E \leq f$ for all $E\subset X$, so that $\int_X f|_E \; d\mu$ will be integrable. (Start with nonneg functions, then generalize.) 

Now if we show $\int_E f|_E \; d\lambda$ equals that value, the integral is finite, so the result is achieved.

$(\leq)$ First show that $f\chi_E$ is measurable, and that the integral is finite. Then take any simple function for the $E$ case, unpack it, to get it into $X$ space, then take the sup.

$(\geq)$ Unpack the righthand side, sup the left.

% 1b 
\item Show that $g$ is measurable in $X$ space.  

\end{enumerate}

% Question 2
\item 

\begin{enumerate} 
        
% Question 2a
\item \textbf{Exercise 88.6}: The constant function $M$ is measurable. Since $\mu(X)<\infty$, $g(x)=M$ is integrable. Then we can bound the limit of the integrals, exchange the limit and integral(DTC), so that the integral is finite, so lebesgue integrable.

Here we have that $f$ is bounded by pointwise convergence. Look VERY carefully at DTC to see if I can apply it.

% Question 2b
\item \textbf{Exercise 88.10}: Moving bump from class. Here we have that $f$ is not necessarily bounded, although the $f_n$ are
\end{enumerate}


\end{enumerate}

\end{document}



%%%% INCLUDING FIGURES %%%%%%%%%%%%%%%%%%%%%%%%%%%%

   % H indicates here 
   %\begin{figure}[h!]
  %   \centering
   %   \includegraphics[scale=1]{file.pdf}
   %\end{figure}

%   \begin{figure}[h!]
%      \centering
%      \mbox{
%	 \subfigure{
%	    \includegraphics[scale=1]{file1.pdf}
%	 }\quad
%	 \subfigure{
%	    \includegraphics[scale=1]{file2.pdf} 
%	 }
%      }
%   \end{figure}
 

%%%%% Including Code %%%%%%%%%%%%%%%%%%%%%5
% \verbatiminput{file.ext}    % Includes verbatim text from the file
% \texttt{text}	  % includes text in courier, or code-like, font
