\documentclass[12pt]{article}

\author{Matthew Cocci}
\title{\textbf{Homework 2}}
\date{\today}

%% Spacing %%%%%%%%%%%%%%%%%%%%%%%%%%%%%%%%%%%%%%%%%%%%%%%%

\usepackage{fullpage}
\usepackage{setspace}
%\onehalfspacing
\usepackage{microtype}


%% Header %%%%%%%%%%%%%%%%%%%%%%%%%%%%%%%%%%%%%%%%%%%%%%%%%

%\pagestyle{fancy} 
%\lhead{}
%\rhead{}
%\chead{}
%\setlength{\headheight}{15.2pt} 
    %---Make the header bigger to avoid overlap

%\renewcommand{\headrulewidth}{0.3pt} 
    %---Width of the line

%\setlength{\headsep}{0.2in}    
    %---Distance from line to text
            

%% Mathematics Related %%%%%%%%%%%%%%%%%%%%%%%%%%%%%%%%%%%

\usepackage{amsmath}
\usepackage{amsfonts}
\usepackage{mathrsfs}
\usepackage{amsthm} %allows for labeling of theorems
\theoremstyle{plain}
\newtheorem{thm}{Theorem}[section]
\newtheorem{lem}[thm]{Lemma}
\newtheorem{prop}[thm]{Proposition}
\newtheorem{cor}[thm]{Corollary}

\theoremstyle{definition}
\newtheorem{defn}[thm]{Definition}
\newtheorem{ex}[thm]{Example}

\theoremstyle{remark}
\newtheorem*{rem}{Remark}
\newtheorem*{note}{Note}


%% Font Choices %%%%%%%%%%%%%%%%%%%%%%%%%%%%%%%%%%%%%%%%%

\usepackage[T1]{fontenc}
\usepackage[utf8]{inputenc}
%\usepackage{blindtext}


%% Figures %%%%%%%%%%%%%%%%%%%%%%%%%%%%%%%%%%%%%%%%%%%%%%

\usepackage{graphicx}
\usepackage{subfigure} 
    %---For plotting multiple figures at once
%\graphicspath{ {Directory/} }
    %---Set a directory for where to look for figures


%% Hyperlinks %%%%%%%%%%%%%%%%%%%%%%%%%%%%%%%%%%%%%%%%%%%%
\usepackage{hyperref} 
\hypersetup{	
    colorlinks,		
        %---This colors the links themselves, not boxes
    citecolor=black,	
        %---Everything here and below changes link colors
    filecolor=black,
    linkcolor=black,
    urlcolor=black
}

%% Including Code %%%%%%%%%%%%%%%%%%%%%%%%%%%%%%%%%%%%%%% 

\usepackage{verbatim} 
    %---For including verbatim code from files, no colors

\usepackage{listings}
\usepackage{color}
\definecolor{mygreen}{RGB}{28,172,0}
\definecolor{mylilas}{RGB}{170,55,241}
\newcommand{\matlabcode}[1]{%
    \lstset{language=Matlab,%
        basicstyle=\footnotesize,%
        breaklines=true,%
        morekeywords={matlab2tikz},%
        keywordstyle=\color{blue},%
        morekeywords=[2]{1}, keywordstyle=[2]{\color{black}},%
        identifierstyle=\color{black},%
        stringstyle=\color{mylilas},%
        commentstyle=\color{mygreen},%
        showstringspaces=false,%
            %---Without this there will be a symbol in 
            %---the places where there is a space
        numbers=left,%
        numberstyle={\tiny \color{black}},% 
            %---Size of the numbers
        numbersep=9pt,% 
            %---Defines how far the numbers are from the text
        emph=[1]{for,end,break,switch,case},emphstyle=[1]\color{red},%
            %---Some words to emphasise
    }%
    \lstinputlisting{#1}
}
    %---For including Matlab code from .m file with colors,
    %---line numbering, etc. 


%% Misc %%%%%%%%%%%%%%%%%%%%%%%%%%%%%%%%%%%%%%%%%%%%%% 

\usepackage{enumitem} 
    %---Has to do with enumeration	
\usepackage{appendix}
%\usepackage{natbib} 
    %---For bibliographies
\usepackage{pdfpages}
    %---For including whole pdf pages as a page in doc


%% User Defined %%%%%%%%%%%%%%%%%%%%%%%%%%%%%%%%%%%%%%%%%% 

%\newcommand{\nameofcmd}{Text to display}



%%%%%%%%%%%%%%%%%%%%%%%%%%%%%%%%%%%%%%%%%%%%%%%%%%%%%%%%%%%%%%%%%%%%%%%% 
%% BODY %%%%%%%%%%%%%%%%%%%%%%%%%%%%%%%%%%%%%%%%%%%%%%%%%%%%%%%%%%%%%%%%
%%%%%%%%%%%%%%%%%%%%%%%%%%%%%%%%%%%%%%%%%%%%%%%%%%%%%%%%%%%%%%%%%%%%%%%% 


\begin{document}

\maketitle

\begin{enumerate}
\item \textbf{Exercise 51.17, FoMA}: We consider $f$, a positive continuous function where $f  \in \mathscr{R}([a,b])$. Letting 
    \[ M = \max_{x \in [a,b]} f(x) \]
we want to prove 
\begin{equation}
    \label{q1.toprove}
    M = 
    \lim_{n\rightarrow\infty} \left[\int^b_a \left[f(x)\right]^n
        \; dx\right]^{1/n}
\end{equation}
Also, let $x^*$ be the $x$ value such that $f(x^*)=M$. By the intermediate value theorem, we know such an $x^*$ exists.
\\
\\
Now because $f$ is continuous on a compact interval, $f$ is bounded, implying that $M$ does exist. And since $f$ is positive, $M>0$ as well. Given that, we can show an equivalent statement to Equation \ref{q1.toprove} by dividing through by $M$:
\begin{align}
    1 &= \lim_{n\rightarrow\infty} \left[\frac{1}{M^n} \int^b_a 
        \left[f(x)\right]^n \; dx\right]^{1/n} \notag \\
    1 &= \lim_{n\rightarrow\infty} \left[ \int^b_a 
        \left[\frac{f(x)}{M}\right]^n \; dx\right]^{1/n} 
        \label{q1.toprove2}
\end{align}
This will be our equivalent statement to prove, rather than Equation \ref{q1.toprove}.
\\
\\
So first, we know that that the integral in Equation \ref{q1.toprove2} over the function $h$ exists (i.e. $h \in \mathscr{R}([a,b])$---where
    \[ h(x) = \left[\frac{f(x)}{M}\right]^n 
        \quad \text{with} \quad 
        \begin{cases} h(x) = 1 & x=x^* \\ h(x) < 1 & x\neq x^* 
        \end{cases}\]
This integral exists because both $f(x)$ is continuous and $g(x) = \left(x/M\right)^n$ is continuous. And because $h = g \circ f$, a composition of continuous functions, $h \in \mathscr{R}_\alpha([a,b])$ by Theorem 51.12 in FoMA. 
\\
\\
It's also clear that we must have
\begin{equation}
    \lim_{n\rightarrow\infty} h(x) =  
    \lim_{n\rightarrow\infty}\left[\frac{f(x)}{M}\right]^n
    = \begin{cases} 1 & x = x^* \\ 0 & x \neq x^*
      \end{cases} 
\end{equation}
Finally, since $h$ is a continuous function in $\mathscr{R}_\alpha([a,b])$, we can use Theorem 52.5 in FoMA to assert that
    \[ \int^b_a h \; dx = \lim_{||P||\rightarrow 0}
        S(h,P,T) \]
for any $T$. Choosing $T$ so that $x^*$ is one of the evaluation points in $T$, we can then

\item \textbf{Exercise 52.1, FoMA}: We want to prove, for $f \in \mathscr{R}([0,1])$ that
\begin{equation}
    \label{q2.toprove}
    \lim_{n\rightarrow\infty} \sum^n_{k=1} f(k/n)\;
        \frac{1}{n} = \int^1_0 f(x) \; dx
\end{equation}
Note that we're taking $\alpha(x) = x$, a continuous function.
\\
\\
We construct the proof by treating the lefthand side as a Riemman Sum. Namely, we construct a partition
\begin{align*}
    P &= \left\{0, \frac{1}{n}, \frac{2}{n}, \ldots, 1\right\} 
    \qquad 
    \Rightarrow \quad \Delta x_i = \frac{1}{n}
\end{align*}
We also consider the evaluation points within each interval
\begin{equation}
    T = \left\{\frac{1}{n}, \frac{2}{n}, \ldots, 1\right\} 
\end{equation}
Combining our partition and evaluation points into one sum, we have a Riemann Sum
\begin{align*}
    S(f,P,T) = \sum^n_{k=1} f(t_i) \Delta x_i = 
        \sum^n_{k=1} f(k/n)\;
        \frac{1}{n}
\end{align*}
which is exactly the lefthand side of Equation \ref{q2.toprove}. Now, we can invoke Theorem 52.5 in FoMA, as we satisfy the conditions that
\begin{itemize}
    \item $f \in \mathscr{R}([0,1])$, which is assumed. 
    \item $\alpha$ continuous, as $\alpha(x)=x$.
\end{itemize}
Thus, we can assert, since $||P|| = 1/n$, which goes to zero as $n$ grows, that
\begin{align*}
    \int^1_0 f dx &= \lim_{||P||\rightarrow 0} S(f,P,T) 
    = \lim_{||P||\rightarrow 0} \sum^n_{k=1} f(t_i) \Delta x_i\\
    &= \sum^n_{k=1} f(k/n)\; \frac{1}{n}
\end{align*}
which is exactly what we wanted to show.

\item Recall the Lipshitz Condition: A function $f$ is Lipshitz at $x$ if for some $C, \delta >0$
\begin{equation}
    |x-y| \leq \delta \quad \Rightarrow \quad
    |f(x) - f(y)| \leq C|x-y|
\end{equation}
We want to show that for $f: [0,1] \rightarrow \mathbb{R}$
\begin{equation}
    \label{q3.torewrite}
    \left\lvert \int^1_0 f \; dx - \frac{1}{n} \sum^n_{k=1}
        f(k/n) \right\rvert \leq \frac{C}{n} \qquad \forall n
\end{equation}
Now let's consider the partition,
\begin{equation}
    P = \left\{ 0, \frac{1}{n},\frac{2}{n}, \ldots,\frac{n}{n}\right\}
    \qquad \Rightarrow \qquad
    \Delta x_i = \frac{1}{n}
\end{equation}
With this partition in mind, we can rewrite the integral
\begin{align*}
    \int^1_0 f \; dx &= \underline{\int}^1_0 f \; dx  \\
    &= L(f, P) + \epsilon
\end{align*}
where $\epsilon\geq 0$, since 
    \[ L(f, P) \leq \sup_Q L(f, Q) \]
And so we can rewrite the lefthand side of Inequality \ref{q3.torwrite} as
\begin{align*}
    \left\lvert \int^1_0 f \; dx - \frac{1}{n} \sum^n_{k=1}
        f(k/n) \right\rvert &= 
    \left\lvert  \left[ L(f, P) + \epsilon\right]  - \frac{1}{n} \sum^n_{k=1}
        f(k/n) \right\rvert \\
    &\leq \left\lvert  L(f, P) - \frac{1}{n} \sum^n_{k=1}
        f(k/n) \right\rvert \\
    &\leq \left\lvert  \sum^n_{i=1} m_i(f)\Delta x_i  - \frac{1}{n} \sum^n_{k=1}
        f(k/n) \right\rvert \\
    &\leq \left\lvert  \frac{1}{n} \sum^n_{i=1} 
        \left[ m_i(f) - f(k/n) \right] \right\rvert \\
\end{align*}


The integral is the lower sum at the same intervals as the sum on the right, plus some epsilon.

Remove the epsilon to get the inequality.

Use the lipshitz to bound.











\end{enumerate}


\end{document}



%%%% INCLUDING FIGURES %%%%%%%%%%%%%%%%%%%%%%%%%%%%

   % H indicates here 
   %\begin{figure}[h!]
   %   \centering
   %   \includegraphics[scale=1]{file.pdf}
   %\end{figure}

%   \begin{figure}[h!]
%      \centering
%      \mbox{
%	 \subfigure{
%	    \includegraphics[scale=1]{file1.pdf}
%	 }\quad
%	 \subfigure{
%	    \includegraphics[scale=1]{file2.pdf} 
%	 }
%      }
%   \end{figure}
 

%%%%% Including Code %%%%%%%%%%%%%%%%%%%%%5
% \verbatiminput{file.ext}    % Includes verbatim text from the file
% \texttt{text}	  % includes text in courier, or code-like, font
