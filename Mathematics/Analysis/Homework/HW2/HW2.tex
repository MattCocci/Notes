\documentclass[12pt]{article}

\author{Matthew Cocci}
\title{\textbf{Homework 2}}
\date{\today}

%% Spacing %%%%%%%%%%%%%%%%%%%%%%%%%%%%%%%%%%%%%%%%%%%%%%%%

\usepackage{fullpage}
\usepackage{setspace}
%\onehalfspacing
\usepackage{microtype}


%% Header %%%%%%%%%%%%%%%%%%%%%%%%%%%%%%%%%%%%%%%%%%%%%%%%%

%\pagestyle{fancy} 
%\lhead{}
%\rhead{}
%\chead{}
%\setlength{\headheight}{15.2pt} 
    %---Make the header bigger to avoid overlap

%\renewcommand{\headrulewidth}{0.3pt} 
    %---Width of the line

%\setlength{\headsep}{0.2in}    
    %---Distance from line to text
            

%% Mathematics Related %%%%%%%%%%%%%%%%%%%%%%%%%%%%%%%%%%%

\usepackage{amsmath}
\usepackage{amsfonts}
\usepackage{mathrsfs}
\usepackage{amsthm} %allows for labeling of theorems
\theoremstyle{plain}
\newtheorem{thm}{Theorem}[section]
\newtheorem{lem}[thm]{Lemma}
\newtheorem{prop}[thm]{Proposition}
\newtheorem{cor}[thm]{Corollary}

\theoremstyle{definition}
\newtheorem{defn}[thm]{Definition}
\newtheorem{ex}[thm]{Example}

\theoremstyle{remark}
\newtheorem*{rem}{Remark}
\newtheorem*{note}{Note}


%% Font Choices %%%%%%%%%%%%%%%%%%%%%%%%%%%%%%%%%%%%%%%%%

\usepackage[T1]{fontenc}
\usepackage[utf8]{inputenc}
%\usepackage{blindtext}


%% Figures %%%%%%%%%%%%%%%%%%%%%%%%%%%%%%%%%%%%%%%%%%%%%%

\usepackage{graphicx}
\usepackage{subfigure} 
    %---For plotting multiple figures at once
%\graphicspath{ {Directory/} }
    %---Set a directory for where to look for figures


%% Hyperlinks %%%%%%%%%%%%%%%%%%%%%%%%%%%%%%%%%%%%%%%%%%%%
\usepackage{hyperref} 
\hypersetup{	
    colorlinks,		
        %---This colors the links themselves, not boxes
    citecolor=black,	
        %---Everything here and below changes link colors
    filecolor=black,
    linkcolor=black,
    urlcolor=black
}

%% Including Code %%%%%%%%%%%%%%%%%%%%%%%%%%%%%%%%%%%%%%% 

\usepackage{verbatim} 
    %---For including verbatim code from files, no colors

\usepackage{listings}
\usepackage{color}
\definecolor{mygreen}{RGB}{28,172,0}
\definecolor{mylilas}{RGB}{170,55,241}
\newcommand{\matlabcode}[1]{%
    \lstset{language=Matlab,%
        basicstyle=\footnotesize,%
        breaklines=true,%
        morekeywords={matlab2tikz},%
        keywordstyle=\color{blue},%
        morekeywords=[2]{1}, keywordstyle=[2]{\color{black}},%
        identifierstyle=\color{black},%
        stringstyle=\color{mylilas},%
        commentstyle=\color{mygreen},%
        showstringspaces=false,%
            %---Without this there will be a symbol in 
            %---the places where there is a space
        numbers=left,%
        numberstyle={\tiny \color{black}},% 
            %---Size of the numbers
        numbersep=9pt,% 
            %---Defines how far the numbers are from the text
        emph=[1]{for,end,break,switch,case},emphstyle=[1]\color{red},%
            %---Some words to emphasise
    }%
    \lstinputlisting{#1}
}
    %---For including Matlab code from .m file with colors,
    %---line numbering, etc. 


%% Misc %%%%%%%%%%%%%%%%%%%%%%%%%%%%%%%%%%%%%%%%%%%%%% 

\usepackage{enumitem} 
    %---Has to do with enumeration	
\usepackage{appendix}
%\usepackage{natbib} 
    %---For bibliographies
\usepackage{pdfpages}
    %---For including whole pdf pages as a page in doc


%% User Defined %%%%%%%%%%%%%%%%%%%%%%%%%%%%%%%%%%%%%%%%%% 

%\newcommand{\nameofcmd}{Text to display}



%%%%%%%%%%%%%%%%%%%%%%%%%%%%%%%%%%%%%%%%%%%%%%%%%%%%%%%%%%%%%%%%%%%%%%%% 
%% BODY %%%%%%%%%%%%%%%%%%%%%%%%%%%%%%%%%%%%%%%%%%%%%%%%%%%%%%%%%%%%%%%%
%%%%%%%%%%%%%%%%%%%%%%%%%%%%%%%%%%%%%%%%%%%%%%%%%%%%%%%%%%%%%%%%%%%%%%%% 


\begin{document}

\maketitle

\begin{enumerate}
\item \textbf{Exercise 51.17, FoMA}: We consider $f$, a positive continuous function where $f  \in \mathscr{R}([a,b])$. Letting 
    \[ M = \max_{x \in [a,b]} f(x) \]
we want to prove 
\begin{equation}
    \label{q1.toprove}
    M = 
    \lim_{n\rightarrow\infty} \left[\int^b_a \left[f(x)\right]^n
        \; dx\right]^{1/n}
\end{equation}
Also, let $x^*$ be the $x$ value such that $f(x^*)=M$. By the intermediate value theorem, we know such an $x^*$ exists.
\\
\\
Now because $f$ is continuous on a compact interval, $f$ is bounded, implying that $M$ does exist. And since $f$ is positive, $M>0$ as well. Given that, we can show an equivalent statement to Equation \ref{q1.toprove} by dividing through by $M$:
\begin{align}
    1 &= \lim_{n\rightarrow\infty} \left[\frac{1}{M^n} \int^b_a 
        \left[f(x)\right]^n \; dx\right]^{1/n} \notag \\
    1 &= \lim_{n\rightarrow\infty} \left[ \int^b_a 
        \left[\frac{f(x)}{M}\right]^n \; dx\right]^{1/n} 
        \label{q1.toprove2}
\end{align}
This will be our equivalent statement to prove, rather than Equation \ref{q1.toprove}.
\\
\\
So first, we know that that the integral in Equation \ref{q1.toprove2} over the function $h$ exists (i.e. $h \in \mathscr{R}([a,b])$---where
    \[ h(x) = \left[\frac{f(x)}{M}\right]^n 
        \quad \text{with} \quad 
        \begin{cases} h(x) = 1 & x=x^* \\ 0< h(x) < 1 & x\neq x^* 
        \end{cases}\]
This integral exists because both $f(x)$ is continuous and $g(x) = \left(x/M\right)^n$ is continuous. And because $h = g \circ f$, a composition of continuous functions, $h \in \mathscr{R}_\alpha([a,b])$ by Theorem 51.12 in FoMA. 
\\
Now because $h \in \mathscr{R}_\alpha([a,b])$, we can write
\begin{align*}
    \int^b_a \left[\frac{f(x)}{M}\right]^n \; dx &=
    \overline{\int^b_a} \left[\frac{f(x)}{M}\right]^n \; dx =
    \inf_P U(h, P) \\
    &= \sum^m_{i=1} M_i(h) \Delta x_i  
\end{align*}
But over all intervals, $M_i(h)$ will be greater than 0 (since $h(x)$ is positive), but less than or equal to one, as mentioned above. Thus, we can put bounds on the integral by putting bounds on the sum:
\begin{align*}
    \sum^m_{i=1} 0 \Delta x_i  &< 
    \sum^m_{i=1} M_i(h) \Delta x_i  \leq
    \sum^m_{i=1} 1 \Delta x_i  \\
    \Leftrightarrow \qquad 
    \epsilon  &\leq 
    \sum^m_{i=1} M_i(h) \Delta x_i  
    \leq  b-a  \\
    \Rightarrow \qquad 
    \epsilon  &\leq
    \int^b_a h(x) \; dx   \leq
    b-a  
\end{align*}
for some small, non-zero value $\epsilon$. The value isn't strictly crucial---only that it's non-zero.
\\
\\
Now using these inequalities, we can squeeze the limit in Equation \ref{q1.toprove2}:
\begin{align*}
    \epsilon^{1/n}  &\leq
    \left[\int^b_a h(x) \; dx\right]^{1/n}   \leq
    (b-a)^{1/n} \\
    \lim_{n\rightarrow\infty} \epsilon^{1/n}  &\leq
    \lim_{n\rightarrow\infty} \left[\int^b_a h(x) \; dx\right]^{1/n}   \leq
    \lim_{n\rightarrow\infty} (b-a)^{1/n} \\
    1  &\leq
    \lim_{n\rightarrow\infty} \left[\int^b_a h(x) \; dx\right]^{1/n}   \leq 1 \\
    \Rightarrow \qquad
    1 &= \lim_{n\rightarrow\infty} \left[\int^b_a h(x) \; dx\right]^{1/n}
\end{align*}
which is what we wanted to prove.


\newpage
\item \textbf{Exercise 52.1, FoMA}: We want to prove, for $f \in \mathscr{R}([0,1])$ that
\begin{equation}
    \label{q2.toprove}
    \lim_{n\rightarrow\infty} \sum^n_{k=1} f(k/n)\;
        \frac{1}{n} = \int^1_0 f(x) \; dx
\end{equation}
Note that we're taking $\alpha(x) = x$, a continuous function.
\\
\\
We construct the proof by treating the lefthand side as a Riemann Sum. Namely, we construct a partition
\begin{align*}
    P &= \left\{0, \frac{1}{n}, \frac{2}{n}, \ldots, 1\right\} 
    \qquad 
    \Rightarrow \quad \Delta x_i = \frac{1}{n}
\end{align*}
We also consider the evaluation points within each interval
\begin{equation}
    T = \left\{\frac{1}{n}, \frac{2}{n}, \ldots, 1\right\} 
\end{equation}
Combining our partition and evaluation points into one sum, we have a Riemann Sum
\begin{align*}
    S(f,P,T) = \sum^n_{k=1} f(t_i) \Delta x_i = 
        \sum^n_{k=1} f(k/n)\;
        \frac{1}{n}
\end{align*}
which is exactly the lefthand side of Equation \ref{q2.toprove}. Now, we can invoke Theorem 52.5 in FoMA, as we satisfy the conditions that
\begin{itemize}
    \item $f \in \mathscr{R}([0,1])$, which is assumed. 
    \item $\alpha$ continuous, as $\alpha(x)=x$.
\end{itemize}
Thus, we can assert, since $||P|| = 1/n$, which goes to zero as $n$ grows, that
\begin{align*}
    \int^1_0 f dx &= \lim_{||P||\rightarrow 0} S(f,P,T) 
    = \lim_{||P||\rightarrow 0} \sum^n_{k=1} f(t_i) \Delta x_i\\
    &= \sum^n_{k=1} f(k/n)\; \frac{1}{n}
\end{align*}
which is exactly what we wanted to show.

\newpage
\item Recall the Lipshitz Condition: A function $f$ is Lipshitz at $x$ if for some $C, \delta >0$
\begin{equation}
    |x-y| \leq \delta \quad \Rightarrow \quad
    |f(x) - f(y)| \leq C|x-y|
\end{equation}
We want to show that for $f: [0,1] \rightarrow \mathbb{R}$
\begin{equation}
    \label{q3.torewrite}
    \left\lvert \int^1_0 f \; dx - \frac{1}{n} \sum^n_{k=1}
        f(k/n) \right\rvert \leq \frac{C}{n} \qquad \forall n
\end{equation}
Now let's define our partition, which we'll use throughout the problem:
\begin{equation}
    P = \left\{ 0, \frac{1}{n},\frac{2}{n}, \ldots,\frac{n}{n}\right\}
    \qquad \Rightarrow \qquad
    \Delta x_i = \frac{1}{n}
\end{equation}
We next use the modulus of continuity result from HW1 that 
\begin{equation}
    \label{q3.hw1}
    U(f,P) - L(f,P) \leq w_f\left(||P||\right) (b-a) 
\end{equation}
Given that $f$ is Lipshitz, we see
    \[ w_f\left(||P||\right) = \sup_{|x-y|\leq ||P||}
        |f(x) - f(y)| \leq C \cdot \frac{1}{n} \]
Substituting into Equation \ref{q3.hw1} for this particular function, $f$:
    \[ U(f,P) - L(f,P) \leq \frac{C}{n} \]
Now clearly, we must have 
\begin{equation}
    \label{comb1}
     L(f,P) \leq \int^1_0 f \; dx \leq U(f,P) 
\end{equation}
since the integral must lie between the upper and lower sums always.
\\
\\
Next, the evaluation points $k/n$ for each interval must fall in between the sup and the inf on the interval, which implies (by our choice of partition)
\begin{align}
    \sum^n_{k=1} m_k(f) \frac{1}{n} 
        &\leq \sum^n_{k=1} f\left(\frac{k}{n}
        \right) \frac{1}{n} \leq 
        \sum^n_{k=1} M_k(f) \frac{1}{n}  \notag \\
    L(f,P) &\leq \frac{1}{n} \sum^n_{k=1} f\left(\frac{k}{n}
        \right) \leq U(f,P) \label{comb2}
\end{align}
Now combining Inequalties \ref{comb1} and \ref{comb2}, it's clear that we must have
    \[ \left\lvert \int^1_0 f \; dx -  
        \frac{1}{n} \sum^n_{k=1} f\left(\frac{k}{n} \right) 
        \right\rvert\leq 
        U(f,P) - L(f,P) \leq \frac{C}{n} \]
since both terms on the left lie in between the upper and lower sums.

\newpage
\item We'll take a similar approach to the last problem to prove that $f: [0,\infty) \rightarrow \mathbb{R}$ is improperly Riemann integrable and Lipschitz. We first generalize by using the partition 
\begin{equation}
    P = \left\{ 0, \frac{L}{n},\frac{2L}{n}, \ldots,\frac{nL}{n}\right\}
    \qquad \Rightarrow \qquad
    \Delta x_i = \frac{L}{n}
\end{equation}
Since $f$ is Lipschitz, we get that result that 
\begin{equation}
    w_f\left(||P||\right) = \sup_{|x-y|\leq ||P||} |f(x) - f(y)| \leq C \cdot \frac{L}{n}
\end{equation}
Using our result from the last homework, we have that
\begin{equation}
    U(f, P) - L(f,P) \leq \frac{CL}{n}
\end{equation}
By exactly the same reasoning as in the last problem, we have that 
\begin{equation}
    \label{q4.key}
     \left\lvert \int^L_0 f \; dx -  
        \frac{L}{n} \sum^n_{k=1} f\left(\frac{kL}{n} \right) 
        \right\rvert\leq 
        U(f,P) - L(f,P) \leq \frac{CL}{n} 
\end{equation}
Now, recall that we want to show that 
\begin{equation}
    \int^\infty_0 f \; dx = \lim_{n\rightarrow\infty} \frac{1}{n^\alpha} 
        \sum^n_{k=1} f\left(\frac{k}{n^\alpha}\right)
\end{equation}
But the lefthand side is just 
\begin{equation}
    \int^\infty_0 f \; dx = \lim_{L\rightarrow\infty} \int^L_0 f\;dx
\end{equation}
And we can write this a bit differently, choosing $L = n^{1-\alpha}$, which is an equivalent formulation in the limit:
\begin{align*}
    \int^\infty_0 f \; dx &= \lim_{L\rightarrow\infty} \int^L_0 f\;dx \\
    &= \lim_{n\rightarrow\infty} \int^{n^{1-\alpha}}_0 f \; dx 
\end{align*}
It's clear that $n^{1-\alpha}$ still goes to infinity in the limit, since $\alpha \in (1/2, 1)$, but this let's us rewrite Condition \ref{q4.key}, substituting in for $L$, to say that 
\begin{align*}
     \left\lvert \int^{n^{1-\alpha}}_0 f \; dx -  
        \frac{{n^{1-\alpha}}}{n} \sum^n_{k=1} f\left(\frac{k{n^{1-\alpha}}}{n} \right) 
        \right\rvert
        &\leq \frac{C{n^{1-\alpha}}}{n}  \\
    \Leftrightarrow \qquad 
     \left\lvert \int^{n^{1-\alpha}}_0 f \; dx -  
        \frac{1}{n^\alpha} \sum^n_{k=1} f\left(\frac{k}{n^\alpha} \right) 
        \right\rvert
        &\leq \frac{C}{n^\alpha}  
\end{align*}
Now, we take the limit of this expression
\begin{align*}
     0\leq \lim_{n\rightarrow\infty} \left\lvert \int^{n^{1-\alpha}}_0 f \; dx -  
        \frac{1}{n^\alpha} \sum^n_{k=1} f\left(\frac{k}{n^\alpha} \right) 
        \right\rvert
        &\leq \lim_{n\rightarrow\infty} \frac{C}{n^\alpha}  \\
     \Rightarrow \qquad 
        0\leq \left\lvert \int^{n^{1-\alpha}}_0 f \; dx -  
        \frac{1}{n^\alpha} \sum^n_{k=1} f\left(\frac{k}{n^\alpha} \right) 
        \right\rvert
        &\leq 0  
\end{align*}
This implies that 
\begin{equation}
        \lim_{n\rightarrow\infty} \int^{n^{1-\alpha}}_0 f \; dx =  
        \lim_{n\rightarrow\infty} \frac{1}{n^\alpha} \sum^n_{k=1} f\left(\frac{k}{n^\alpha} \right) 
\end{equation}
But the lefthand side is just the improper integral (since the limit is assumed to exist). Thus
\begin{equation}
        \int^{\infty}_0 f \; dx =  
        \lim_{n\rightarrow\infty} \frac{1}{n^\alpha} \sum^n_{k=1} f\left(\frac{k}{n^\alpha} \right) 
\end{equation}


    





\newpage
\item We want to show that $f(x) = \cos(x^2)$ is improperly Riemmann integrable on $[0,\infty)$. We do so by first noting that over a finite interval $[a,b]$, $f \in \mathscr{R}_\alpha([a,b])$. This is because $g(x) = x^2$ is continuous (and thus integrable), and $h(x) = cos(x)$ is continuous (and thus integrable).  Since $f = h\circ g$, a composition of integrable functions, $f \in \mathscr{R}_\alpha([a,b])$ for finite $[a,b]$.
\\
\\
Next, to show that $f(x)$ is improperly integrable, we must show that the following limit exists:
\begin{equation}
    \lim_{b\rightarrow\infty} \int^b_0 \cos(x^2) \; dx 
\end{equation} 
Equivalently, for the sake of easier notation, I will show another statement, breaking up the integral into a proper and an improper component:
\begin{equation}
    \label{q5.equiv}
    \lim_{b\rightarrow\infty} \int^b_0 \cos(x^2) \; dx  = 
    \int^{\sqrt{\pi/2}}_0 \cos(x^2) \; dx  + 
    \lim_{b\rightarrow\infty} \int^b_{\sqrt{\pi/2}} \cos(x^2) \; dx  
\end{equation} 
Since the first term is over a finite interval, by what we said above, it's integrable.  To prove that the second term on the lefthand side exists, we introduce the following sequence:
\begin{equation}
    \left\{ a_k \right\}_{k=0}^\infty = 
    \left\{ \sqrt{\frac{\left(2k+1\right)\pi}{2}} \right\}_{k=0}^\infty = 
    \left\{ \sqrt{\frac{\pi}{2}}, 
            \sqrt{\frac{3\pi}{2}},
            \sqrt{\frac{5\pi}{2}}, \ldots
    \right\}
\end{equation}
This essentially uses the regular periods of the cosine function, defining those points at which the cosine function crosses from negative to positive (or vice versa). But it's a slight modification, taking the square root since the function itself will square it.
\\
\\
Now we can use this to break the integral up and rewrite the limit from Equation \ref{q5.equiv} as 
\begin{equation}
    \label{q5.equiv2}
    \lim_{b\rightarrow\infty} \int^b_{\sqrt{\pi/2}} \cos(x^2) \; dx   =
    \lim_{m\rightarrow\infty} \sum^m_{k=1} \int^{a_k}_{a_{k-1}} \cos(x^2) \; dx  
\end{equation}
Now let's consider each component interval
\begin{equation}
    \int^{a_k}_{a_{k-1}} \cos(x^2) \; dx  
\end{equation}
By the properties of the cosine function and the way we wrote the series $a_k$, we have that
\begin{align*}
    \cos(x^2) &< 0 \quad \forall x \in [a_{k-1}, a_k] \quad \text{if $k$ odd}\\
    \cos(x^2) &> 0 \quad \forall x \in [a_{k-1}, a_k] \quad \text{if $k$ even}
\end{align*}
Thus, the terms in the sum, which are $\int^{a_k}_{a_{k-1}} \cos(x^2) \; dx$, will be \emph{alternating} positive and negative. So we rewrite Equation \ref{q5.equiv2} a little more suggestively:
\begin{align*}
    \lim_{b\rightarrow\infty} \int^b_{\sqrt{\pi/2}} \cos(x^2) \; dx   &=
    \lim_{m\rightarrow\infty} \sum^m_{k=1} (-1)^k \left\lvert \int^{a_k}_{a_{k-1}} \cos(x^2) \; dx \right\rvert\\
    &=
    \lim_{m\rightarrow\infty} \sum^m_{k=1} (-1)^k b_k
\end{align*}
Now if we can show that $\lim_{k\rightarrow\infty} b_k = 0$, we can use the alternating series test to assert that the series converges and our limit exists. Let's show that.
\\
\\
Start with the fact that $b_k>0$ for all $k$. This is because we chose the bounds of the integral which determines $b_k$ such that the function is entirely positive or entirely negative over the interval.  There's no cancelling out, and it's not zero everywhere ont the interval. Thus,
\begin{align*}
    0 < \left\lvert \int^{a_k}_{a_{k-1}} \cos(x^2) \; dx \right\rvert \leq 
    \int^{a_k}_{a_{k-1}} \left\lvert \cos(x^2)\right\rvert \; dx 
\end{align*}
The second inequality holds because $cos(x^2)$ is Riemann integrable over a finite interval, and we just use a standard property of Riemann Integrable functions.
\\
\\
Now because $\cos(x^2)$ lies within $[-1,1]$, we can bound the terms $b_k$:
\begin{align*}
    0 \leq \left\lvert \int^{a_k}_{a_{k-1}} \cos(x^2) \; dx \right\rvert \leq 
    \int^{a_k}_{a_{k-1}} \left\lvert \cos(x^2)\right\rvert \; dx &\leq 
    \int^{a_k}_{a_{k-1}} 1 \cdot dx \\
    \Rightarrow \quad \left\lvert \int^{a_k}_{a_{k-1}} \cos(x^2) \; dx \right\rvert &\leq a_k - a_{k-1}
\end{align*}
So if we can prove that $a_k - a_{k-1} \rightarrow 0$ as $k\rightarrow \infty$, then we prove that $b_k\rightarrow0$ as well, permitting us to use the alternating series test
\\
\\
To do so, we note that if you take a fixed-length interval, $[x,x+\delta]$ for fixed $\delta$, the change in the function $f(x) = \sqrt{x}$ goes to 0 as $x\rightarrow\infty$. We can see this by noting that $f(x) = \sqrt{x}$ is continuous on $[0, \infty)$, implying that we have approximate
\begin{align*}
    (f(x+\delta) - f(x)) &= f'(c)(x+\delta - x)\\
\end{align*}
for some $c\in[x, x+\delta]$. And since we have
\begin{align*}
    f'(x) = \frac{d}{dx} \sqrt{x} = \frac{1}{2\sqrt{x}}
\end{align*}
we can sub in and say
\begin{align*}
    (f(x+\delta) - f(x)) &= f'(c)(x+\delta - x)\\
    &= \frac{1}{2\sqrt{c}} \delta
\end{align*}
Since $\delta$ is fixed and $c\rightarrow\infty$ along with $x$, we know that $f(x+\delta) - f(x)$ goes to 0 as well. Thus we have that
\begin{align*}
    \lim_{k\rightarrow\infty}\left\lvert \int^{a_k}_{a_{k-1}} \cos(x^2) \; dx \right\rvert \leq 
     \lim_{k\rightarrow\infty}   a_k - a_{k-1} 
        &\leq 0
\end{align*}
So we see that $b_k\rightarrow 0$, allowing us to apply the alternating series test.


\newpage
\item 
\begin{enumerate}

\item Take the same partition from Question 5 above.  Now consider the function 
    \[ f(x) = \begin{cases} -1 & \forall x \in [a_{k-1}, a_k] \quad \text{$k$ odd} \\
                             1 & \forall x \in [a_{k-1}, a_k] \quad \text{$k$ even} 
                \end{cases} \]
Now this function is improperly Riemann Integrable as we can apply everything we save in Question 5 above, just substituting our new function $f$ for $\cos(x^2)$.
\\
\\
However, $|f(x)| = 1$ for all $x$, so that 
    \[ \lim_{z\rightarrow 0} \sum^\infty_{k=0} f(kz) z \]
need not exist.

\item 
We want to show, for uniformly continuous $f$ on $[0,\infty)$ where $|f(x)| \leq 1/x^2$, that 
\begin{equation}
    \label{q5.toshow}
    \int^\infty_0 f \; dx = \lim_{z\rightarrow 0} \sum^\infty_{k=0} f(kz) z
\end{equation}
Now first, the improper integral for $f$ must exist, because 
\begin{align*}
    \int^\infty_0 f \; dx &= \int^1_0 f \; dx + \lim_{b\rightarrow\infty} \int^b_1 f \; dx
\end{align*}
On the righthand side, the first integral is over a finite interval for a continuous function, so it's integrable. And for the second term on the righthand side, we know that that $1/x^2$ is integrable over $[1,\infty)$, so by the comparison test
    \[ 0 \leq \int^\infty_1 |f(x)| \;dx \leq \int^\infty_1 \frac{1}{x^2} \;dx = L \]
Now since $|f(x)|$ converges and $f \in \mathscr{R}_\alpha([a,b])$ for all $b>a$ (since it's continuous), we know that $f$ is absolutely convergent, implying that for an increasing $\alpha$ the improper integral $f$ over $[1,\infty)$ converges.
\\
\\
Next, we can rewrite the Equality \ref{q5.toshow} above a little differently, using the fact that $f$ is integrable 
\begin{align*}
    \int^\infty_0 f \; dx &= \lim_{z\rightarrow 0} \sum^\infty_{k=0} f(kz) z \\
    \Leftrightarrow \qquad 
    \sum_{k=0}^\infty \int^{(k+1)z}_{kz} f \; dx &= \lim_{z\rightarrow 0} 
        \sum^\infty_{k=0} \int^{(k+1)z}_{kz} f(kz) \; dx 
\end{align*}
Each component in the sum on the lefthand side will be integrable, by the fact that $f$ is continuous.
\\
\\
Our goal will be to show the equality above, demonstrating that, for all $\epsilon>0$,
\begin{equation}
    \left\lvert\lim_{z\rightarrow 0} 
        \sum^\infty_{k=0} \int^{(k+1)z}_{kz} f(kz) \; dx 
        -\sum_{k=0}^\infty \int^{(k+1)z}_{kz} f \; dx \right\rvert \leq \epsilon
\end{equation}
We start by rewriting
\begin{equation}
    \label{q6.equiv}
    \left\lvert\lim_{z\rightarrow 0} 
        \sum^\infty_{k=0} \int^{(k+1)z}_{kz} f(kz) - f(x) \; dx 
        \right\rvert \leq \epsilon 
\end{equation}
Now for large $k$, the term within the integral is effectively small enough, and for small $k$, we can make $z$ small enough to counteract so that the sum converges to 0. So let's make that more concrete.
\\
\\
Given any any $\epsilon>0$, there's a value $K$ such that $k>K$ implies that $|f(kz) - f(x)|<\epsilon/2$ on the interval $[kz, (k+1)z]$.  This is because $|f(x)|\leq 1/x^2$, which means that we can bound the difference as follows:
    \[ \forall x \in [kz, kz+z] \qquad |f(kz) - f(x)| \leq 2\cdot\frac{1}{(kz)^2} \]
since $1/x^2$ attains it's highest value (in absolute terms) at the beginning of any interval. And so we can always choose a big enough $k$, denoted $K$, such that $|f(kz) - f(x)|$ is less than $\epsilon/2$.  But even more importantly, it's bounded by $2/(kz)^2$.
\\
\\
Next, we note that $f$ is uniformly continuous.  Then, for those ``small'' $k$, we use the uniform continuity of $f$ to say that given $\epsilon/2K$, we can choose a $\delta$ such that $|x-y| \leq \delta$ implies that $|f(x)-f(y)|\leq\epsilon/2K$. By setting $z=\delta$, we have what we want because $|x-kz|\leq z$ for all $k$ since $x \in [kz, kz+z]$.
\\
\\
So we rewrite Equation \ref{q6.equiv}
\begin{align*}
    \left\lvert\lim_{z\rightarrow 0} 
        \sum^\infty_{k=0} \int^{(k+1)z}_{kz} f(kz) - f(x) \; dx 
        \right\rvert &=
        \lvert\lim_{z\rightarrow 0} 
        \sum^K_{k=0} \int^{(k+1)z}_{kz} f(kz) - f(x) \; dx \\
   &\qquad
        + \sum^\infty_{k=K+1} \int^{(k+1)z}_{kz} f(kz) - f(x) \; dx 
       \rvert \\
    &\leq 
        \left\lvert \lim_{z\rightarrow 0}
        \sum^K_{k=0} \int^{(k+1)z}_{kz} \frac{\epsilon}{2K} \; dx +
       \lim_{z\rightarrow 0}  \sum^\infty_{k=K+1} \int^{(k+1)z}_{kz} \frac{\epsilon}{2} \; dx 
        \right\rvert \\
    &\leq \left\lvert \lim_{z\rightarrow 0} K \cdot \frac{\epsilon(K+1)z}{2K} 
	+ \lim_{z\rightarrow 0} \sum^\infty_{k=K+1} \frac{\epsilon}{2} z 
	\right\rvert \\
    &\leq \left\lvert 0
	+ \lim_{z\rightarrow 0} z \sum^\infty_{k=K+1} \frac{\epsilon}{2} 
	\right\rvert
\end{align*}
And here's where I got stuck. You can make the lefthand term small, but the righthand term diverges.  There's clearly something I'm missing.



\end{enumerate}

\end{enumerate}


\end{document}



%%%% INCLUDING FIGURES %%%%%%%%%%%%%%%%%%%%%%%%%%%%

   % H indicates here 
   %\begin{figure}[h!]
   %   \centering
   %   \includegraphics[scale=1]{file.pdf}
   %\end{figure}

%   \begin{figure}[h!]
%      \centering
%      \mbox{
%	 \subfigure{
%	    \includegraphics[scale=1]{file1.pdf}
%	 }\quad
%	 \subfigure{
%	    \includegraphics[scale=1]{file2.pdf} 
%	 }
%      }
%   \end{figure}
 

%%%%% Including Code %%%%%%%%%%%%%%%%%%%%%5
% \verbatiminput{file.ext}    % Includes verbatim text from the file
% \texttt{text}	  % includes text in courier, or code-like, font
