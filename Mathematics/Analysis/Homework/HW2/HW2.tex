\documentclass[12pt]{article}

\author{Matthew Cocci}
\title{\textbf{Homework 2}}
\date{\today}

%% Spacing %%%%%%%%%%%%%%%%%%%%%%%%%%%%%%%%%%%%%%%%%%%%%%%%

\usepackage{fullpage}
\usepackage{setspace}
%\onehalfspacing
\usepackage{microtype}


%% Header %%%%%%%%%%%%%%%%%%%%%%%%%%%%%%%%%%%%%%%%%%%%%%%%%

%\pagestyle{fancy} 
%\lhead{}
%\rhead{}
%\chead{}
%\setlength{\headheight}{15.2pt} 
    %---Make the header bigger to avoid overlap

%\renewcommand{\headrulewidth}{0.3pt} 
    %---Width of the line

%\setlength{\headsep}{0.2in}    
    %---Distance from line to text
            

%% Mathematics Related %%%%%%%%%%%%%%%%%%%%%%%%%%%%%%%%%%%

\usepackage{amsmath}
\usepackage{amsfonts}
\usepackage{mathrsfs}
\usepackage{amsthm} %allows for labeling of theorems
\theoremstyle{plain}
\newtheorem{thm}{Theorem}[section]
\newtheorem{lem}[thm]{Lemma}
\newtheorem{prop}[thm]{Proposition}
\newtheorem{cor}[thm]{Corollary}

\theoremstyle{definition}
\newtheorem{defn}[thm]{Definition}
\newtheorem{ex}[thm]{Example}

\theoremstyle{remark}
\newtheorem*{rem}{Remark}
\newtheorem*{note}{Note}


%% Font Choices %%%%%%%%%%%%%%%%%%%%%%%%%%%%%%%%%%%%%%%%%

\usepackage[T1]{fontenc}
\usepackage[utf8]{inputenc}
%\usepackage{blindtext}


%% Figures %%%%%%%%%%%%%%%%%%%%%%%%%%%%%%%%%%%%%%%%%%%%%%

\usepackage{graphicx}
\usepackage{subfigure} 
    %---For plotting multiple figures at once
%\graphicspath{ {Directory/} }
    %---Set a directory for where to look for figures


%% Hyperlinks %%%%%%%%%%%%%%%%%%%%%%%%%%%%%%%%%%%%%%%%%%%%
\usepackage{hyperref} 
\hypersetup{	
    colorlinks,		
        %---This colors the links themselves, not boxes
    citecolor=black,	
        %---Everything here and below changes link colors
    filecolor=black,
    linkcolor=black,
    urlcolor=black
}

%% Including Code %%%%%%%%%%%%%%%%%%%%%%%%%%%%%%%%%%%%%%% 

\usepackage{verbatim} 
    %---For including verbatim code from files, no colors

\usepackage{listings}
\usepackage{color}
\definecolor{mygreen}{RGB}{28,172,0}
\definecolor{mylilas}{RGB}{170,55,241}
\newcommand{\matlabcode}[1]{%
    \lstset{language=Matlab,%
        basicstyle=\footnotesize,%
        breaklines=true,%
        morekeywords={matlab2tikz},%
        keywordstyle=\color{blue},%
        morekeywords=[2]{1}, keywordstyle=[2]{\color{black}},%
        identifierstyle=\color{black},%
        stringstyle=\color{mylilas},%
        commentstyle=\color{mygreen},%
        showstringspaces=false,%
            %---Without this there will be a symbol in 
            %---the places where there is a space
        numbers=left,%
        numberstyle={\tiny \color{black}},% 
            %---Size of the numbers
        numbersep=9pt,% 
            %---Defines how far the numbers are from the text
        emph=[1]{for,end,break,switch,case},emphstyle=[1]\color{red},%
            %---Some words to emphasise
    }%
    \lstinputlisting{#1}
}
    %---For including Matlab code from .m file with colors,
    %---line numbering, etc. 


%% Misc %%%%%%%%%%%%%%%%%%%%%%%%%%%%%%%%%%%%%%%%%%%%%% 

\usepackage{enumitem} 
    %---Has to do with enumeration	
\usepackage{appendix}
%\usepackage{natbib} 
    %---For bibliographies
\usepackage{pdfpages}
    %---For including whole pdf pages as a page in doc


%% User Defined %%%%%%%%%%%%%%%%%%%%%%%%%%%%%%%%%%%%%%%%%% 

%\newcommand{\nameofcmd}{Text to display}



%%%%%%%%%%%%%%%%%%%%%%%%%%%%%%%%%%%%%%%%%%%%%%%%%%%%%%%%%%%%%%%%%%%%%%%% 
%% BODY %%%%%%%%%%%%%%%%%%%%%%%%%%%%%%%%%%%%%%%%%%%%%%%%%%%%%%%%%%%%%%%%
%%%%%%%%%%%%%%%%%%%%%%%%%%%%%%%%%%%%%%%%%%%%%%%%%%%%%%%%%%%%%%%%%%%%%%%% 


\begin{document}

\maketitle

\begin{enumerate}
\item \textbf{Exercise 51.17, FoMA}: We consider $f$, a positive continuous function where $f  \in \mathscr{R}([a,b])$. Letting 
    \[ M = \max_{x \in [a,b]} f(x) \]
we want to prove 
\begin{equation}
    \label{q1.toprove}
    M = 
    \lim_{n\rightarrow\infty} \left[\int^b_a \left[f(x)\right]^n
        \; dx\right]^{1/n}
\end{equation}
Also, let $x^*$ be the $x$ value such that $f(x^*)=M$. By the intermediate value theorem, we know such an $x^*$ exists.
\\
\\
Now because $f$ is continuous on a compact interval, $f$ is bounded, implying that $M$ does exist. And since $f$ is positive, $M>0$ as well. Given that, we can show an equivalent statement to Equation \ref{q1.toprove} by dividing through by $M$:
\begin{align}
    1 &= \lim_{n\rightarrow\infty} \left[\frac{1}{M^n} \int^b_a 
        \left[f(x)\right]^n \; dx\right]^{1/n} \notag \\
    1 &= \lim_{n\rightarrow\infty} \left[ \int^b_a 
        \left[\frac{f(x)}{M}\right]^n \; dx\right]^{1/n} 
        \label{q1.toprove2}
\end{align}
This will be our equivalent statement to prove, rather than Equation \ref{q1.toprove}.
\\
\\
So first, we know that that the integral in Equation \ref{q1.toprove2} over the function $h$ exists (i.e. $h \in \mathscr{R}([a,b])$---where
    \[ h(x) = \left[\frac{f(x)}{M}\right]^n 
        \quad \text{with} \quad 
        \begin{cases} h(x) = 1 & x=x^* \\ 0< h(x) < 1 & x\neq x^* 
        \end{cases}\]
This integral exists because both $f(x)$ is continuous and $g(x) = \left(x/M\right)^n$ is continuous. And because $h = g \circ f$, a composition of continuous functions, $h \in \mathscr{R}_\alpha([a,b])$ by Theorem 51.12 in FoMA. 
\\
Now because $h \in \mathscr{R}_\alpha([a,b])$, we can write
\begin{align*}
    \int^b_a \left[\frac{f(x)}{M}\right]^n \; dx &=
    \overline{\int^b_a} \left[\frac{f(x)}{M}\right]^n \; dx =
    \inf_P U(h, P) \\
    &= \sum^m_{i=1} M_i(h) \Delta x_i  
\end{align*}
But over all intervals, $M_i(h)$ will be greater than 0 (since $h(x)$ is positive), but less than or equal to one, as mentioned above. Thus, we can put bounds on the integral by putting bounds on the sum:
\begin{align*}
    \sum^m_{i=1} 0 \Delta x_i  &< 
    \sum^m_{i=1} M_i(h) \Delta x_i  \leq
    \sum^m_{i=1} 1 \Delta x_i  \\
    \Leftrightarrow \qquad 
    \epsilon  &\leq 
    \sum^m_{i=1} M_i(h) \Delta x_i  
    \leq  b-a  \\
    \Rightarrow \qquad 
    \epsilon  &\leq
    \int^b_a h(x) \; dx   \leq
    b-a  
\end{align*}
for some small, non-zero value $\epsilon$. The value isn't strictly crucial---only that it's non-zero.
\\
\\
Now using these inequalities, we can squeeze the limit in Equation \ref{q1.toprove2}:
\begin{align*}
    \epsilon^{1/n}  &\leq
    \left[\int^b_a h(x) \; dx\right]^{1/n}   \leq
    (b-a)^{1/n} \\
    \lim_{n\rightarrow\infty} \epsilon^{1/n}  &\leq
    \lim_{n\rightarrow\infty} \left[\int^b_a h(x) \; dx\right]^{1/n}   \leq
    \lim_{n\rightarrow\infty} (b-a)^{1/n} \\
    1  &\leq
    \lim_{n\rightarrow\infty} \left[\int^b_a h(x) \; dx\right]^{1/n}   \leq 1 \\
    \Rightarrow \qquad
    1 &= \lim_{n\rightarrow\infty} \left[\int^b_a h(x) \; dx\right]^{1/n}
\end{align*}
which is what we wanted to prove.


\newpage
\item \textbf{Exercise 52.1, FoMA}: We want to prove, for $f \in \mathscr{R}([0,1])$ that
\begin{equation}
    \label{q2.toprove}
    \lim_{n\rightarrow\infty} \sum^n_{k=1} f(k/n)\;
        \frac{1}{n} = \int^1_0 f(x) \; dx
\end{equation}
Note that we're taking $\alpha(x) = x$, a continuous function.
\\
\\
We construct the proof by treating the lefthand side as a Riemman Sum. Namely, we construct a partition
\begin{align*}
    P &= \left\{0, \frac{1}{n}, \frac{2}{n}, \ldots, 1\right\} 
    \qquad 
    \Rightarrow \quad \Delta x_i = \frac{1}{n}
\end{align*}
We also consider the evaluation points within each interval
\begin{equation}
    T = \left\{\frac{1}{n}, \frac{2}{n}, \ldots, 1\right\} 
\end{equation}
Combining our partition and evaluation points into one sum, we have a Riemann Sum
\begin{align*}
    S(f,P,T) = \sum^n_{k=1} f(t_i) \Delta x_i = 
        \sum^n_{k=1} f(k/n)\;
        \frac{1}{n}
\end{align*}
which is exactly the lefthand side of Equation \ref{q2.toprove}. Now, we can invoke Theorem 52.5 in FoMA, as we satisfy the conditions that
\begin{itemize}
    \item $f \in \mathscr{R}([0,1])$, which is assumed. 
    \item $\alpha$ continuous, as $\alpha(x)=x$.
\end{itemize}
Thus, we can assert, since $||P|| = 1/n$, which goes to zero as $n$ grows, that
\begin{align*}
    \int^1_0 f dx &= \lim_{||P||\rightarrow 0} S(f,P,T) 
    = \lim_{||P||\rightarrow 0} \sum^n_{k=1} f(t_i) \Delta x_i\\
    &= \sum^n_{k=1} f(k/n)\; \frac{1}{n}
\end{align*}
which is exactly what we wanted to show.

\newpage
\item Recall the Lipshitz Condition: A function $f$ is Lipshitz at $x$ if for some $C, \delta >0$
\begin{equation}
    |x-y| \leq \delta \quad \Rightarrow \quad
    |f(x) - f(y)| \leq C|x-y|
\end{equation}
We want to show that for $f: [0,1] \rightarrow \mathbb{R}$
\begin{equation}
    \label{q3.torewrite}
    \left\lvert \int^1_0 f \; dx - \frac{1}{n} \sum^n_{k=1}
        f(k/n) \right\rvert \leq \frac{C}{n} \qquad \forall n
\end{equation}
Now let's define our partition, which we'll use throughout the problem:
\begin{equation}
    P = \left\{ 0, \frac{1}{n},\frac{2}{n}, \ldots,\frac{n}{n}\right\}
    \qquad \Rightarrow \qquad
    \Delta x_i = \frac{1}{n}
\end{equation}
We next use the modulus of continuity result from HW1 that 
\begin{equation}
    \label{q3.hw1}
    U(f,P) - L(f,P) \leq w_f\left(||P||\right) (b-a) 
\end{equation}
Given that $f$ is Lipshitz, we see
    \[ w_f\left(||P||\right) = \sup_{|x-y|\leq ||P||}
        |f(x) - f(y)| \leq C \cdot \frac{1}{n} \]
Substituting into Equation \ref{q3.hw1} for this particular function, $f$:
    \[ U(f,P) - L(f,P) \leq \frac{C}{n} \]
Now clearly, we must have 
\begin{equation}
    \label{comb1}
     L(f,P) \leq \int^1_0 f \; dx \leq U(f,P) 
\end{equation}
since the integral must lie between the upper and lower sums always.
\\
\\
Next, the evaluation points $k/n$ for each interval must fall in between the sup and the inf on the interval, which implies (by our choice of partition)
\begin{align}
    \sum^n_{k=1} m_k(f) \frac{1}{n} 
        &\leq \sum^n_{k=1} f\left(\frac{k}{n}
        \right) \frac{1}{n} \leq 
        \sum^n_{k=1} M_k(f) \frac{1}{n}  \notag \\
    L(f,P) &\leq \frac{1}{n} \sum^n_{k=1} f\left(\frac{k}{n}
        \right) \leq U(f,P) \label{comb2}
\end{align}
Now combining Inequalties \ref{comb1} and \ref{comb2}, it's clear that we must have
    \[ \left\lvert \int^1_0 f \; dx -  
        \frac{1}{n} \sum^n_{k=1} f\left(\frac{k}{n} \right) 
        \right\rvert\leq 
        U(f,P) - L(f,P) \leq \frac{C}{n} \]
since both terms on the left lie in between the upper and lower sums.



\end{enumerate}


\end{document}



%%%% INCLUDING FIGURES %%%%%%%%%%%%%%%%%%%%%%%%%%%%

   % H indicates here 
   %\begin{figure}[h!]
   %   \centering
   %   \includegraphics[scale=1]{file.pdf}
   %\end{figure}

%   \begin{figure}[h!]
%      \centering
%      \mbox{
%	 \subfigure{
%	    \includegraphics[scale=1]{file1.pdf}
%	 }\quad
%	 \subfigure{
%	    \includegraphics[scale=1]{file2.pdf} 
%	 }
%      }
%   \end{figure}
 

%%%%% Including Code %%%%%%%%%%%%%%%%%%%%%5
% \verbatiminput{file.ext}    % Includes verbatim text from the file
% \texttt{text}	  % includes text in courier, or code-like, font
