\documentclass[12pt]{article}

\author{Matthew Cocci}
\title{\textbf{Homework 9}}
\date{\today}

%% Spacing %%%%%%%%%%%%%%%%%%%%%%%%%%%%%%%%%%%%%%%%%%%%%%%%

\usepackage{fullpage}
\usepackage{setspace}
%\onehalfspacing
\usepackage{microtype}


%% Header %%%%%%%%%%%%%%%%%%%%%%%%%%%%%%%%%%%%%%%%%%%%%%%%%

%\pagestyle{fancy} 
%\lhead{}
%\rhead{}
%\chead{}
%\setlength{\headheight}{15.2pt} 
    %---Make the header bigger to avoid overlap

%\renewcommand{\headrulewidth}{0.3pt} 
    %---Width of the line

%\setlength{\headsep}{0.2in}    
    %---Distance from line to text
            

%% Mathematics Related %%%%%%%%%%%%%%%%%%%%%%%%%%%%%%%%%%%

\usepackage{amsmath}
\usepackage{amsfonts}
\usepackage{mathtools}
\usepackage{mathrsfs}
\usepackage{amsthm} %allows for labeling of theorems
\theoremstyle{plain}
\newtheorem{thm}{Theorem}[section]
\newtheorem{lem}[thm]{Lemma}
\newtheorem{prop}[thm]{Proposition}
\newtheorem{cor}[thm]{Corollary}

\theoremstyle{definition}
\newtheorem{defn}[thm]{Definition}
\newtheorem{ex}[thm]{Example}

\theoremstyle{remark}
\newtheorem*{rem}{Remark}
\newtheorem*{note}{Note}


%% Font Choices %%%%%%%%%%%%%%%%%%%%%%%%%%%%%%%%%%%%%%%%%

\usepackage[T1]{fontenc}
\usepackage[utf8]{inputenc}
\usepackage{lmodern}
%\usepackage{blindtext}


%% Figures %%%%%%%%%%%%%%%%%%%%%%%%%%%%%%%%%%%%%%%%%%%%%%

\usepackage{graphicx}
\usepackage{subfigure} 
    %---For plotting multiple figures at once
%\graphicspath{ {Directory/} }
    %---Set a directory for where to look for figures


%% Hyperlinks %%%%%%%%%%%%%%%%%%%%%%%%%%%%%%%%%%%%%%%%%%%%
\usepackage{hyperref} 
\hypersetup{	
    colorlinks,		
        %---This colors the links themselves, not boxes
    citecolor=black,	
        %---Everything here and below changes link colors
    filecolor=black,
    linkcolor=black,
    urlcolor=black
}

%% Including Code %%%%%%%%%%%%%%%%%%%%%%%%%%%%%%%%%%%%%%% 

\usepackage{verbatim} 
    %---For including verbatim code from files, no colors

\usepackage{listings}
\usepackage{color}
\definecolor{mygreen}{RGB}{28,172,0}
\definecolor{mylilas}{RGB}{170,55,241}
\newcommand{\matlabcode}[1]{%
    \lstset{language=Matlab,%
        basicstyle=\footnotesize,%
        breaklines=true,%
        morekeywords={matlab2tikz},%
        keywordstyle=\color{blue},%
        morekeywords=[2]{1}, keywordstyle=[2]{\color{black}},%
        identifierstyle=\color{black},%
        stringstyle=\color{mylilas},%
        commentstyle=\color{mygreen},%
        showstringspaces=false,%
            %---Without this there will be a symbol in 
            %---the places where there is a space
        numbers=left,%
        numberstyle={\tiny \color{black}},% 
            %---Size of the numbers
        numbersep=9pt,% 
            %---Defines how far the numbers are from the text
        emph=[1]{for,end,break,switch,case},emphstyle=[1]\color{red},%
            %---Some words to emphasise
    }%
    \lstinputlisting{#1}
}
    %---For including Matlab code from .m file with colors,
    %---line numbering, etc. 


%% Misc %%%%%%%%%%%%%%%%%%%%%%%%%%%%%%%%%%%%%%%%%%%%%% 

\usepackage{enumitem} 
    %---Has to do with enumeration	
\usepackage{appendix}
%\usepackage{natbib} 
    %---For bibliographies
\usepackage{pdfpages}
    %---For including whole pdf pages as a page in doc


%% User Defined %%%%%%%%%%%%%%%%%%%%%%%%%%%%%%%%%%%%%%%%%% 

%\newcommand{\nameofcmd}{Text to display}
\newcommand*{\Chi}{\mbox{\large$\chi$}} %big chi



%%%%%%%%%%%%%%%%%%%%%%%%%%%%%%%%%%%%%%%%%%%%%%%%%%%%%%%%%%%%%%%%%%%%%%%% 
%% BODY %%%%%%%%%%%%%%%%%%%%%%%%%%%%%%%%%%%%%%%%%%%%%%%%%%%%%%%%%%%%%%%%
%%%%%%%%%%%%%%%%%%%%%%%%%%%%%%%%%%%%%%%%%%%%%%%%%%%%%%%%%%%%%%%%%%%%%%%% 


\begin{document}

\maketitle 

\begin{enumerate} 

% Question 1
\item Let $f$ be a function that is Lipschitz. We want to prove that it is absolutely continuous. So let $C$ and $\eta$ be non-negative constants such that that 
\begin{equation}
\label{lip}
    |x-y|<\eta
    \quad\Rightarrow\quad
    |f(x)-f(y)|<C|x-y|
\end{equation}
Then, fix $\varepsilon>0$ and choose $\delta$ such that
\begin{equation}
\label{q1.step}
    \sum_{i=1}^N|b_i-a_i| \leq 
    \delta := \frac{\min\{\varepsilon,\eta\}}{C}
\end{equation}
This choice of $\delta$ obviously implies that $(b_i-a_i)<\eta$ for all $i$. So 
now let's use that together with Condition \ref{lip} to bound the target sum by $\varepsilon$:
\begin{align*}
    \sum_{i=1}^N |f(b_i)-f(a_i)| \leq 
    \sum_{i=1}^N C|b_i-a_i| &=
    C \sum_{i=1}^N |b_i-a_i| \\
    \text{By (\ref{q1.step})} \qquad
    &\leq 
    C \cdot \frac{\min\{\varepsilon,\eta\}}{C}
    = \min\{\varepsilon,\eta\}
\end{align*}
Clearly, the sum is bounded by $\varepsilon$.




% Question 2
\item 
\begin{enumerate} 

% Part a
\item Suppose that we have a composition $f\circ g$. We want to show that, for fixed $\varepsilon>0$, there exists a $\delta$ such that 
\begin{equation}
    \label{q2a}
    \sum_{i=1}^N |b_i-a_i|<\delta 
    \quad\Rightarrow\quad
    \sum_{i=1}^N |f(g(b_i))-f(g(a_i))|<\varepsilon
\end{equation}
The ``obvious'' attempt says to, fixing $\varepsilon$, find the relevant $\delta_f$ for the function $f$ such that
\[
    \sum_{i=1}^N |b_i-a_i|<\delta_f
    \quad\Rightarrow\quad
    \sum_{i=1}^N |f(b_i)-f(a_i)|<\varepsilon
\]
Then, let this $\delta_f$ be the new $\varepsilon'$, and find $\delta_g$ for the function $g$ such that
\[
    \sum_{i=1}^N |b_i-a_i|<\delta_g
    \quad\Rightarrow\quad
    \sum_{i=1}^N |g(b_i)-g(a_i)|<\varepsilon' := \delta_f
\]
In a sense, this approach tries to control the size of the second sum in (\ref{q2a}) in two stages. We want the intervals to be sufficiently small such that, when mapped by $g$, the result will still be sufficiently small for when $f$ maps them.

% Part b
\item ($f$ Absolutely Continuous) Let $\varepsilon$ and $N$ (the number of disjoint intervals) be given. Then choose $\delta$ such that 
\begin{equation}
    \label{q2b}
    \delta:= \frac{\varepsilon^2}{N^2}
\end{equation}
For any individual open interval $(b_i-a_i)\subset[0,1]$, we see that $|b_i-a_i|<\delta$, so that
\begin{align*}
    |f({b_i})-f({a_i})|^2 &\leq
    \left(\sqrt{b_i}-\sqrt{a_i}\right)\left(\sqrt{b_i}+\sqrt{a_i}\right) 
    = b_i - a_i < \delta := \frac{\varepsilon^2}{N^2} \\
    \Rightarrow\quad
    |f({b_i})-f({a_i})|^2 &\leq \frac{\varepsilon^2}{N^2} \\
    \Leftrightarrow\quad
    |f({b_i})-f({a_i})| &\leq \frac{\varepsilon}{N} 
\end{align*}
Since this is true for all $i$, we can sum over the $N$ intervals to get 
\[
  \sum_{i=1}^N  |f({b_i})-f({a_i})| 
  \leq \sum_{i=1}^N\frac{\varepsilon}{N} = \varepsilon
\]
($g$ Absolutely Continuous) We will use Question 1 and show that $g$ is Lipschitz, which implies absolutely continuous. So, start $x,y\in[0,1]$:
\begin{align*}
    |f({x})-f({y})| &= \left\lvert 
    x^2\left(2+\sin\frac{1}{x}\right) -
    y^2\left(2+\sin\frac{1}{y}\right) \right\rvert \\
    &= \left\lvert 2(x^2 - y^2) + x^2\sin\frac{1}{x}
    - y^2\sin\frac{1}{y}\right\rvert \\
    &\leq \left\lvert 2(x^2 - y^2)\right\rvert 
    + \left\lvert x^2\sin\frac{1}{x}
    - y^2\sin\frac{1}{y}\right\rvert 
\end{align*}
Now, note that the maximum of the sin function is 1 and the minimum -1, so the righthand term in the sum can't possibly be larger than $|x^2 + y^2|$.

% Q3c
\item The flaw in (a)'s reasoning comes in neglecting the possibility that the function $g$ in $f\circ g$ might turn disjoint intervals into non-disjoint intervals. As hinted at, the definition is for disjoint intervals, and that is crucial.

To give an example, take $f$ and $g$ in part (b). Then note that the $\sin\frac{1}{x}$ has its maxima and minima at $\left\{\frac{2}{(4k+1)\pi}\right\}_0^\infty$ and $\left\{\frac{2}{(4k+3)\pi}\right\}_0^\infty$, respectively. So let's take disjoint intervals in $[0,1]$ to be
\[
    (a_1,b_1) = \left(\frac{2}{3\pi}, \frac{2}{\pi}\right)
    \qquad
    (a_2,b_2) = \left(\frac{2}{7\pi}, \frac{2}{5\pi}\right)
\]
Then, mapping these intervals with $g$, we see that we get overlapping intervals as a result:
\begin{align*}
    g((a_1,b_1)) = \left(\frac{4}{9\pi^2}, \frac{12}{\pi^2}\right)
    \approx (0.045, 1.215)\\
    g((a_2,b_2)) = \left(\frac{4}{49\pi^2}, 
    \frac{12}{25\pi^2}\right)
    \approx (0.008, 0.049)
\end{align*}
So we see that $g$ doesn't preserve the disjointness of the intervals when it's time to apply $f$ to get $f\circ g$.


% Q3d
\item 
\begin{enumerate} 

% Q3d,i
\item Suppose that the functions $f$ and $g$ are strongly absolutely continuous. Letting $\varepsilon>0$ be given, we want to show that $f\circ g$ is stongly absolutely continuous, i.e. we can choose a $\delta$ such that 
\begin{equation}
    \sum_{i=1}^N |b_i-a_i|<\delta_{f\circ g}
    \quad\Rightarrow\quad
    \sum_{i=1}^N |f(g(b_i))-f(g(a_i))|<\varepsilon
\end{equation}
So, we'll take the ``obvious'' approach from part (a), which will work here for the reasons explained in part (c). So, since $f$ is absolutely continuous, there exists a $\delta_f$ such that
\begin{equation}
    \sum_{i=1}^N |p_i-q_i|<\delta_f
    \quad\Rightarrow\quad
    \sum_{i=1}^N |f(p_i)-f(q_i)|<\varepsilon
\end{equation}
where $\{(q_i,p_i)\}_1^N$ is \emph{any} finite collection of intervals. Next, we can use the fact that $g$ is strongly absolutely continuous to choose a $\delta_g$ such that
\begin{equation}
    \sum_{i=1}^N |b_i-a_i|<\delta_g
    \quad\Rightarrow\quad
    \sum_{i=1}^N |g(b_i)-g(a_i)|<\delta_f
\end{equation}
Now setting our overall $\delta_{g\circ f}:=\delta_g$, we see that for any set of intervals $\{(a_i, b_i)\}$
\begin{align*}
    \sum_{i=1}^N |b_i-a_i|<\delta_g
    &\quad\Rightarrow\quad
    \sum_{i=1}^N |g(b_i)-g(a_i)|<\delta_f \\
    &\quad\Rightarrow\quad
    \sum_{i=1}^N |f(g(b_i))-f(g(a_i))|<\varepsilon
\end{align*}
where $\{(g(a_i),g(b_i)\}^N_1$ is a finite collection of open intervals for $f$.

% Q3d,ii 
\item We want to show $f(x)=\sqrt{x}$ is not strongly absolutely continuous. To do so, suppose by contridiction that $f$ is strongly absolutely continuous. And without loss of generality, also suppose that $\varepsilon<1/2$ and that $\delta$ is chosen accordingly. We will now find a finite collection such that
\[
    \sum_{i=1}^N (b_i-a_i) < \delta
    \quad\text{but}\quad
    \sum_{i=1}^N |f(b_i)-f(a_i)| \geq \varepsilon 
\]
So, given $\delta$, construct a set of overlapping intervals 
\[
    \left\{\left(0,\frac{1}{(N+k)^2}\right)\right\}^N_{k=1}
\]
Now given this set, to choose $N$, we note that
\begin{align*}
    \frac{1}{(N+k)^2} <  \frac{1}{N^2}  
    \quad\Rightarrow\quad
    \sum_{k=1}^N |b_i-a_i| &=
    \sum_{k=1}^N \left\lvert\frac{1}{(N+k)^2}-0\right\rvert \\
    &<\sum_{k=1}^N \frac{1}{N^2}
     =N\cdot \frac{1}{N^2} = \frac{1}{N}
\end{align*}
So then just choose $N$ such that $1/N < \delta$. We now have our finite set of intervals. 

Next, let's look at result, using the fact that $\frac{1}{N+k} \geq \frac{1}{2N} $ for $k\leq N$. 
\begin{align*}
    \sum_{k=1}^N |f(b_i)-f(a_i)| = \sum_{k=1}^N 
    \left\lvert \sqrt{\frac{1}{(N+k)^2}} - \sqrt{0}\right\rvert
    &= \sum_{k=1}^N {\frac{1}{N+k}}  \\
    &\geq \sum_{k=1}^N {\frac{1}{2N}}  = N\cdot \frac{1}{2N} = \frac{1}{2}
\end{align*}
And we assumed that $\varepsilon<1/2$, so this is a contradiction. 


% Part q3d, iii
\item To show that Lipshitz implies strongly absolutely continuous, note that the proof given in Question 1 in now way depended upon the disjointness of the intervals. It was entirely formulated in terms of the \emph{lengths} of the intervals being sufficiently small. 

So as long as we set $\delta$ as in Equation \ref{q1.step}, we get the desired result.

% Part q3d, iv
\item To show strongly absolutely continuous implies Lipschitz, let $\varepsilon=1$ and let $\delta_0$ be the corresponding $\delta$. Now if $|u-v|<\delta_0$, let $N=\left\lfloor{\frac{\delta_0}{|u-v|}}\right\rfloor$





\end{enumerate} 


\end{enumerate} 


\end{enumerate} 
\end{document}



%%%% INCLUDING FIGURES %%%%%%%%%%%%%%%%%%%%%%%%%%%%

   % H indicates here 
   %\begin{figure}[h!]
  %   \centering
   %   \includegraphics[scale=1]{file.pdf}
   %\end{figure}

%   \begin{figure}[h!]
%      \centering
%      \mbox{
%	 \subfigure{
%	    \includegraphics[scale=1]{file1.pdf}
%	 }\quad
%	 \subfigure{
%	    \includegraphics[scale=1]{file2.pdf} 
%	 }
%      }
%   \end{figure}
 

%%%%% Including Code %%%%%%%%%%%%%%%%%%%%%5
% \verbatiminput{file.ext}    % Includes verbatim text from the file
% \texttt{text}	  % includes text in courier, or code-like, font
