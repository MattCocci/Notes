\documentclass[12pt]{article}

\author{Matthew Cocci}
\title{\textbf{Homework 3}}
\date{\today}

%% Spacing %%%%%%%%%%%%%%%%%%%%%%%%%%%%%%%%%%%%%%%%%%%%%%%%

\usepackage{fullpage}
\usepackage{setspace}
%\onehalfspacing
\usepackage{microtype}


%% Header %%%%%%%%%%%%%%%%%%%%%%%%%%%%%%%%%%%%%%%%%%%%%%%%%

%\pagestyle{fancy} 
%\lhead{}
%\rhead{}
%\chead{}
%\setlength{\headheight}{15.2pt} 
    %---Make the header bigger to avoid overlap

%\renewcommand{\headrulewidth}{0.3pt} 
    %---Width of the line

%\setlength{\headsep}{0.2in}    
    %---Distance from line to text
            

%% Mathematics Related %%%%%%%%%%%%%%%%%%%%%%%%%%%%%%%%%%%

\usepackage{amsmath}
\usepackage{amsfonts}
\usepackage{mathrsfs}
\usepackage{amsthm} %allows for labeling of theorems
\theoremstyle{plain}
\newtheorem{thm}{Theorem}[section]
\newtheorem{lem}[thm]{Lemma}
\newtheorem{prop}[thm]{Proposition}
\newtheorem{cor}[thm]{Corollary}

\theoremstyle{definition}
\newtheorem{defn}[thm]{Definition}
\newtheorem{ex}[thm]{Example}

\theoremstyle{remark}
\newtheorem*{rem}{Remark}
\newtheorem*{note}{Note}


%% Font Choices %%%%%%%%%%%%%%%%%%%%%%%%%%%%%%%%%%%%%%%%%

\usepackage[T1]{fontenc}
\usepackage[utf8]{inputenc}
\usepackage{lmodern}
%\usepackage{blindtext}


%% Figures %%%%%%%%%%%%%%%%%%%%%%%%%%%%%%%%%%%%%%%%%%%%%%

\usepackage{graphicx}
\usepackage{subfigure} 
    %---For plotting multiple figures at once
%\graphicspath{ {Directory/} }
    %---Set a directory for where to look for figures


%% Hyperlinks %%%%%%%%%%%%%%%%%%%%%%%%%%%%%%%%%%%%%%%%%%%%
\usepackage{hyperref} 
\hypersetup{	
    colorlinks,		
        %---This colors the links themselves, not boxes
    citecolor=black,	
        %---Everything here and below changes link colors
    filecolor=black,
    linkcolor=black,
    urlcolor=black
}

%% Including Code %%%%%%%%%%%%%%%%%%%%%%%%%%%%%%%%%%%%%%% 

\usepackage{verbatim} 
    %---For including verbatim code from files, no colors

\usepackage{listings}
\usepackage{color}
\definecolor{mygreen}{RGB}{28,172,0}
\definecolor{mylilas}{RGB}{170,55,241}
\newcommand{\matlabcode}[1]{%
    \lstset{language=Matlab,%
        basicstyle=\footnotesize,%
        breaklines=true,%
        morekeywords={matlab2tikz},%
        keywordstyle=\color{blue},%
        morekeywords=[2]{1}, keywordstyle=[2]{\color{black}},%
        identifierstyle=\color{black},%
        stringstyle=\color{mylilas},%
        commentstyle=\color{mygreen},%
        showstringspaces=false,%
            %---Without this there will be a symbol in 
            %---the places where there is a space
        numbers=left,%
        numberstyle={\tiny \color{black}},% 
            %---Size of the numbers
        numbersep=9pt,% 
            %---Defines how far the numbers are from the text
        emph=[1]{for,end,break,switch,case},emphstyle=[1]\color{red},%
            %---Some words to emphasise
    }%
    \lstinputlisting{#1}
}
    %---For including Matlab code from .m file with colors,
    %---line numbering, etc. 


%% Misc %%%%%%%%%%%%%%%%%%%%%%%%%%%%%%%%%%%%%%%%%%%%%% 

\usepackage{enumitem} 
    %---Has to do with enumeration	
\usepackage{appendix}
%\usepackage{natbib} 
    %---For bibliographies
\usepackage{pdfpages}
    %---For including whole pdf pages as a page in doc


%% User Defined %%%%%%%%%%%%%%%%%%%%%%%%%%%%%%%%%%%%%%%%%% 

%\newcommand{\nameofcmd}{Text to display}



%%%%%%%%%%%%%%%%%%%%%%%%%%%%%%%%%%%%%%%%%%%%%%%%%%%%%%%%%%%%%%%%%%%%%%%% 
%% BODY %%%%%%%%%%%%%%%%%%%%%%%%%%%%%%%%%%%%%%%%%%%%%%%%%%%%%%%%%%%%%%%%
%%%%%%%%%%%%%%%%%%%%%%%%%%%%%%%%%%%%%%%%%%%%%%%%%%%%%%%%%%%%%%%%%%%%%%%% 


\begin{document}

\maketitle 

\begin{enumerate}
    
\item We want to show that for differentiable $f$ where $f'\in\mathscr{R}([a,b])$, that $f\in BV([a,b])$ and $V^b_a(f) = \int^b_a |f'| \; dx$. 
\\
\\
We start by noting that since $f$ is differentiable, we know that $f$ is continuous. This will allow us to rewrite the components of any approximating sums we use, via the Mean Value Theorem. That is, for all subintervals, 
\[ 
    \exists c_i \in [x_{i-1}, x_i] \quad \text{s.t.} \quad
    f(x_i) - f(x_{i-1}) = f'(c_i) (x_{i} - x_{i-1})
\]
Thus, let's write the expression for the total variation, suitably modified to take advantage of the MVT:
\begin{align*}
    V^b_a(f) &= \sup_P \sum_{i=1}^n |f(x_i) - f(x_{i-1})| \qquad \forall P \\
    &= \sup_P \sum_{i=1}^n |f'(c_i)[x_i-x_{i-1}]| =\sup_P
    \sum_{i=1}^n |f'(c_i)| \Delta x_i
\end{align*}
Now let's consider the lower sum of $|f'|$, for any partition $P$:
\begin{align*}
    L(|f'|,P) = \sum_{i=1}^n m_i(|f'|) \Delta x_i 
\end{align*}
Now clearly, sicne $m_i(|f'|)$ is the minimum over the $i$th subinterval, we have that 
\begin{align*}
    m_i(|f'|) \Delta x_i  \leq |f'(c_i)| \Delta x_i \quad \forall \; i
    \quad \Rightarrow \quad L(|f'|,P) \leq V^b_a(f)
\end{align*}
Next, for any partition $P$, we can write
\begin{align}
    \label{q1.sum}
    \sum^n_{i=1} |f(x_i) - f(x_{i-1})| &= 
        \sum^n_{i=1} \left\lvert \int^{x_i}_{x_{i-1}} df 
        \right\rvert 
\end{align}
Now we know that these integrals within the sum exist from a theorem we proved in class. Specifically, $f'$ is bounded since $f'\in BV([a,b])$ (and we're working on a closed interval, not with improperly Riemann integrable functions). Bounded variation implies bounded. Next, we also know that $f$ is differentiable, which implies it is also continuous. Thus, we can apply the theorem we proved (Theorem 54.3 in FoMA) to assert that $f\in BV([a,b])$. Finally, since our integrator, $f$, is of bounded variation and our function equals $1$ everywhere (which is continuous), we know that the integrals in the sum exist.

Now back to the sum in Equation \ref{q1.sum}. Since $f'$ is differentiable, we can imply another theorem to express the sum in yet another form:
\begin{align*}
    \sum^n_{i=1} |f(x_i) - f(x_{i-1})| = 
        \sum^n_{i=1} \left\lvert \int^{x_i}_{x_{i-1}} df 
        \right\rvert = 
        \sum^n_{i=1} \left\lvert \int^{x_i}_{x_{i-1}} f'(x) \; dx
        \right\rvert 
\end{align*}
Applying some rules we know about absolute value with integrals, we can further say that:
\begin{align*}
    \sum^n_{i=1} |f(x_i) - f(x_{i-1})| &= 
        \sum^n_{i=1} \left\lvert \int^{x_i}_{x_{i-1}} f'(x) \; dx
        \right\rvert \\
    &\leq \sum^n_{i=1} \int^{x_i}_{x_{i-1}} \left\lvert f'(x) \;
        \right\rvert dx  = \int^{b}_{a} \left\lvert f'(x) 
        \right\rvert \; dx  
\end{align*}
Now since this holds for all $P$, we know that this must also hold if we take the sup of the lefthand side over $P$. Thus
\[
    V_a^b(f) = \sup_P
    \sum^n_{i=1} |f(x_i) - f(x_{i-1})|  
    \leq 
    \int^{b}_{a} \left\lvert f'(x) \;
    \right\rvert dx  
\]
Thus, we have established that 
\begin{align}
    L_\alpha(|f'|,P) \leq V_a^b(f) \leq \int^b_a |f'|\;d\alpha
    \label{q1.cond}
\end{align}
\textbf{Conclusion}: This allows us to conclude
\begin{itemize}
    \item $V_a^b(f)  = \int^b_a |f'|\;d\alpha$. We can do this because Relation \ref{q1.cond} holds for all $P$.  And so, for any $\varepsilon>0$, there exists a $P$ such that 
\[ 
    \left\lvert L_\alpha(|f'|,P) - \int^b_a |f'|\;dx 
    \right\rvert \leq \varepsilon 
\]
And so for any $\varepsilon>0$, by \ref{q1.cond}, we must also have
\[ 
    \left\lvert V_a^b(f)  - \int^b_a |f'|\;dx 
    \right\rvert \leq \varepsilon 
\]

\item Next, we can also conclude that $f\in BV([a,b])$ because $|f'|\in\mathscr{R}_\alpha([a,b])$, which is over a closed interval $[a,b]$; therefore, the integral must be finite. And because $V_a^b(f)= \int^b_a |f'|\;dx$, we know that $V_a^b(f)$ must be finite as well.
\end{itemize}

\newpage
\item Given $\alpha, \beta >0$ we have
\begin{align*}
    f_{\alpha, \beta}(x) = 
        \begin{cases}
            x^\alpha \sin(x^{-\beta}) & x\in (0, 1] \\
            0 & x=0 \\
        \end{cases}
\end{align*}
We want to show that $f_{\alpha,\beta}\in BV([0,1])$ if and only if $\alpha>\beta$.
\\
\\
Recall the definition for the total variation:
\[
    V_a^b(f) = \sup_P \sum^n_{i=1} |f(x_{i}) - f(x_{i-1})|
\]
Also recall that over an increasing interval, the total variation equals $f(b)-f(a)$ (absolute value if decreasing). This means we should try to pin down those points between which $f_{\alpha, \beta}(x)$ is increasing or decreasing. Then, we know that the total variation of the intervals will just be the difference between $f_{\alpha,\beta}$ evaluated at the beginning and the end of the interval.

Conveniently, the peaks and troughs come at regular intervals for the sine function, and we will use these points to define the intervals between which the function is increasing or decreasing:
\[ 
    \text{Peaks and Troughs of sin at} \quad
    \left\{ \frac{(2k+1)\pi}{2} \right\}^\infty_{k=0} 
\]
Then, for our function $f_{\alpha,\beta}$, we have the peaks and troughs at 
\[ 
    \text{Peaks and Troughs of $f_{\alpha,\beta}$ at} \quad
    \left\{ \left(\frac{(2k+1)\pi}{2}\right)^{-1/\beta} 
    \right\}^\infty_{k=0}  =
    \left\{ \left(\frac{2}{(2k+1)\pi}\right)^{1/\beta} 
    \right\}^\infty_{k=0} 
\]
Note that the multiplicative factor $x^\alpha$ won't change the locations of the peaks and troughs, just the magnitude. 

And so using the points defined above as our partition of the interval, we get the total variation:
\begin{align}
    V_a^b(f) &= \sup_P \sum^n_{i=1} |f(x_i) - f(x_{i-1})|\notag\\
        &\leq \sum^n_{k=0} 
        \left(\frac{2}{(2k+1)\pi}\right)^{\alpha/\beta}
        \cdot 2
        \label{q2.pain}
\end{align}
This inequality is obtained by setting one of the leading $x_i^\alpha$ or $x_{i-1}^\alpha$ equal to the other. Then, after factoring that term out, we use the fact that the maximum gap from peak to trough of the sine function is 2 to bound the sum.

Finally, given above is a harmonic series like
\[
    \sum^n_{i=1} \frac{1}{n^p}
\]
which only converges if and only if $p>1$. Thus the sum in Inequality \ref{q2.pain} converges if and only if $\alpha>\beta$.

\newpage
\item We suppose that $f\in BV([a,b])$ and want to show that
    \[ \sup_{x\in[0,1]} |f(x)| \leq \int^1_0 |f(x)|\;dx
        + V^1_0(f) \]
    So suppose that this is not true.  That is, suppose there's an $x_1\in[0,1]$ such that
\begin{align*}
    |f(x_1)| &> \int^1_0 |f(x)|\;dx + V_1^0(f)  
\end{align*}
Then it's also true if we replace the term inside the integral with a constant equal to the smallest value of $|f|$ (since we're dealing with absolute values):
\begin{align*}
    |f(x_1)| &> \int^1_0 \inf_{t\in[0,1]} |f(t)|\;dx + V_1^0(f)  
        = \inf_{t\in[0,1]} |f(t)| \int^1_0 \;dx + V_1^0(f)  \\
    \Rightarrow \quad
        |f(x_1)| &> 
        \inf_{t\in[0,1]} |f(t)| + V_1^0(f) 
\end{align*}
Since it's a strict inequality, that means there exists an $\varepsilon>0$ such that
\begin{align}
    \label{q3.eps}
    |f(x_1)|  - \inf_{t\in[0,1]} |f(t)| - V_1^0(f) = \varepsilon  
\end{align}
Now let's use this $\varepsilon$ to choose another point, $x_2\in[0,1]$, such that
\[ 
    |f(x_2)| - \inf_{t\in[0,1]} |f(t)|  < \varepsilon
\]
Next, substitute into the previous equation Equation \ref{q3.eps}'s representation of $\varepsilon$: 
\begin{align}
    |f(x_2)| - \inf_{t\in[0,1]} |f(t)| &< 
        |f(x_1)|  - \inf_{t\in[0,1]} |f(t)| - V_1^0(f) \notag \\
    \Rightarrow \quad
        V_0^1(f) &< |f(x_1)| - |f(x_2)| 
        \label{q3.almost}
\end{align}
But if we apply the properties of the absolute value again, we note that
\begin{align*}
    |f(x_1)| - |f(x_2)| \quad \leq \quad
        \left\lvert\; |f(x_1)| - |f(x_2)|\; \right\rvert 
        \quad\leq\quad 
        \left\lvert\; f(x_1) - f(x_2)\; \right\rvert 
\end{align*}
Substituting this result into Inequality \ref{q3.almost}, we get that
\begin{align}
        V_0^1(f) &<
        \left\lvert\; f(x_1) - f(x_2)\; \right\rvert 
\end{align}
However, this leads to a contradiction since we assumed that $V_1^0(f)$ maximized the variation over the interval $[0,1]$.  But this last line tells us that if we choose a partition spanning $x_1$ to $x_2$ (or vice versa; the ordering doesn't matter), we can construct a higher value for the total variation. Thus, we must have 
\[ 
    |f(x)| \leq \int^1_0 |f(x)|\;dx
    + V^1_0(f) \quad \forall\; x
    \qquad \Rightarrow \qquad
    \sup_{x\in[0,1]} |f(x)| \leq \int^1_0 |f(x)|\;dx
    + V^1_0(f) 
\]


\item 
\begin{enumerate} 
\item \textbf{Exercise 54.4}: We want to show that $\alpha\in BV([a,b]) \Rightarrow |\alpha|\in BV([a,b])$. We do that by noting for any partition $P$, 
\[ 
    \sum^n_{i=1} \left\lvert \;|\alpha(x_{i})| - |\alpha(x_{i-1})|
    \;\right\rvert 
\]
But, we also know by a property of $|\cdot|$ that we will have
\[ 
    \left\lvert \;|\alpha(x_{i})| - |\alpha(x_{i-1})|\;
    \right\rvert \leq 
    \left\lvert \;\alpha(x_{i}) - \alpha(x_{i-1})\;
    \right\rvert  \qquad \forall\; i
\]
So this allows us to write
\[ 
    \sum^n_{i=1} \left\lvert \;|\alpha(x_{i})| - |\alpha(x_{i-1})|
    \;\right\rvert 
    \leq 
    \sum^n_{i=1} 
    \left\lvert \;\alpha(x_{i}) - \alpha(x_{i-1})\;
    \right\rvert  \qquad \forall \; P
\]
And since this holds for all $P$, this must also hold when we consider the total variation of $\alpha$.  Thus, it is bounded by $V_a^b(\alpha)$, implying
\[ 
    V_a^b(|\alpha|) = \sup_P
    \sum^n_{i=1} \left\lvert \;|\alpha(x_{i})| - |\alpha(x_{i-1})|
    \;\right\rvert 
    \leq 
    \sum^n_{i=1} 
    \left\lvert \;\alpha(x_{i}) - \alpha(x_{i-1})\;
    \right\rvert \leq V_a^b(\alpha) < \infty
\]
\[ 
    \Rightarrow |\alpha|\in BV([a,b])
\]

\item 
    \textbf{Exercise 54.6}: Next, we want to show that if $\alpha, \beta \in BV([a,b])$, then $\max\{\alpha, \beta\} \in BV([a,b])$ and $\min\{\alpha, \beta\} \in BV([a,b])$.

\end{enumerate} 

\item We can use the function from question 2:
    \[ 
        f_{\alpha, \beta}(x) = 
        \begin{cases}
            x^\alpha \sin(x^{-\beta}) & x\in (0, 1] \\
            0 & x=0 \\
        \end{cases}
    \]
It's clearly differentiable on $(a,b)$, and that derivative equals
\begin{equation}
        {f'}_{\alpha, \beta}(x) = \alpha x^{\alpha-1} 
            \sin(x^{-\beta}) - \beta x^{\alpha-\beta-1} \cos(x^{-\beta})
\end{equation}
This gives us an easy hint.  If we take $\alpha = 0.5$ and $\beta=0.25$, we have that $\alpha>\beta$.  So by the answer to Question 2, we know that $f\in BV([a,b])$. However, because the exponents are negative in front of the sin and cos terms, we know that $f'$ is unbounded. 
    

\item 
\begin{enumerate}
\item \textbf{Exercise 55.5}: We want to show that for some $c\in(a,b)$,
\begin{align*}
    \int^b_a f\;d\alpha &= f(a)\int^c_a d\alpha 
        + f(b) \int^b_c d\alpha 
\end{align*}
So let's rewrite the integral $\int^b_a f\;d\alpha$ using the integration by parts theorem:
\begin{align}
    \label{q6.ibp}
    \int^b_a f\;d\alpha  
        &= f(b) \alpha(b) - f(a)\alpha(a) - \int^b_a\alpha \; df
\end{align}
But, since $\alpha$ is continuous and $f$ is assumed increasing, we can apply the Mean Value Theorem to assert that there exists some $c^*\in(a,b)$ such that 
\begin{equation}
    \int^b_a \alpha\;df = \alpha(c^*)[f(b)-f(a)]
\end{equation}
Subbing this into Equation \ref{q6.ibp}, we get that
\begin{align*}
    \int^b_a f\;d\alpha  
        &= f(b) \alpha(b)-f(a)\alpha(a)-\alpha(c^*)[f(b)-f(a)]\\
        &= f(b) \left[\alpha(b)-\alpha(c^*)\right]
            + f(a)\left[\alpha(c^*)-\alpha(a)\right]\\
        &= f(b) \int^b_{c^*} d\alpha
            + f(a)\int^{c^*}_a d\alpha
\end{align*}
Letting $c=c^*$ gives us exactly what we want.


\item 
\textbf{Exercise 56.4}: Let $g$ be continuous on $[a,b]$ and suppose $f$ is increasing on $[a,b]$.
\begin{enumerate}
\item First, we want to show that there exists a $c$ in $[a,b]$ such that 
\begin{align}
    \label{q6.2.toprove}
    \int^b_a f(x)g(x)\;dx = f(a) \cdot \int^c_a g(x)\;dx
        + f(b) \cdot \int^b_c g(x)\;dx
\end{align}
We do so by noting that $g(x)$ is continuous, which implies $g\in\mathscr{R}([a,b])$. With that, we can use the Fundamental Theorem of Calculus to define the function
\begin{align*}
    G(x) = \int^x_a f(t)\;dt \qquad
\end{align*}
By the Theorem, we know both $G$ is continuous on $[a,b]$, and we know that $G$ is differentiable, wherever $g$ is continuous (which is everywhere on $[a,b]$). And because of that 
\begin{align*}
    G'(x) = g(x) \qquad \forall \; x\in[a,b] 
\end{align*}
Next, we also know by another theorem (55.7 in FoMA) that when $\alpha$ is continuous and differentiable on $[a,b]$, we have (taking $f(x)=1$ in the theorem)
\[ 
    \int^b_a d\alpha = \int^b_a \alpha' \; dx 
\]
Now by what we said above, it's clear that $G$ is differentiable and continuous on $[a,b]$, so we know that 
\begin{align}
    \int^b_a dG &= \int^b_a G'(x) \;dx\notag\\
    &= \int^b_a g(x) \;dx
    \label{q6.bigresult}
\end{align}
And so, we substitute this result from Equation \ref{q6.bigresult} into Equation \ref{q6.2.toprove}, rewriting it in a more familiar form:
\begin{align*}
    \int^b_a f(x)g(x)\;dx &= f(a) \cdot \int^c_a g(x)\;dx
        + f(b) \int^b_c g(x)\;dx\\
    \Leftrightarrow
    \int^b_a f(x)\;dG 
    &= f(a) \cdot \int^c_a dG
        + f(b) \cdot \int^b_c dG
\end{align*}
And now this is exactly what we had in the first part of this question, except $\alpha$ is $G$ instead. And, as in the first part, we know that $f$ is increasing and $\alpha=G$ is increasing.  So there must be a $c$ such that the statement holds.

\item Next, we want to prove Bonnet's Theorem, which says that if $f$ is non-negative, in addition to our previous assumptions, then there exists a $c\in[a,b]$ such that
\[
    \int^b_a f(x)g(x)\;dx = f(b)\cdot\int^b_c g(x)\;dx
\]
    
    
\end{enumerate}


\end{enumerate}



\newpage    
\item  
\begin{enumerate}
\item
We want to show that for $f', g'\in\mathscr{R}([a,b])$
\begin{equation}
    \int^b_a f'g = f(b)g(b) - f(a)g(a) - \int^b_a f g'
\end{equation}
We use a result similar to what we used above. Namely, since $f'$ exists, $f$ is continuous since differentiability implies continuity. Then, since $f'\in\mathscr{R}_\alpha([a,b])$ by assumption, we know we can represent the integral on the left by
\[ 
    \int^b_a f' g = \int^b_a g \; df 
\]
And if we expand out the integrl on the right of the last line by using the traditional integration by parts, we get that
\begin{align*}
    \int^b_a f' g &= \int^b_a g \; df \\
    &= f(b)g(b) - f(a)g(a) - \int^b_a f\;dg
\end{align*}
But, since we know that $g$ is continuous and differentiable (for the same reasons as $f$ given above) and that $g'\in\mathscr{R}_\alpha([a,b])$, we can rewrite the last line by taking the derivative of the integrator $g$ to get
\begin{align*}
    \int^b_a f' g &= 
     f(b)g(b) - f(a)g(a) - \int^b_a fg'
\end{align*}
which is exactly what we wanted to prove.

\end{enumerate}
        
        

\end{enumerate}

\end{document}



%%%% INCLUDING FIGURES %%%%%%%%%%%%%%%%%%%%%%%%%%%%

   % H indicates here 
   %\begin{figure}[h!]
   %   \centering
   %   \includegraphics[scale=1]{file.pdf}
   %\end{figure}

%   \begin{figure}[h!]
%      \centering
%      \mbox{
%	 \subfigure{
%	    \includegraphics[scale=1]{file1.pdf}
%	 }\quad
%	 \subfigure{
%	    \includegraphics[scale=1]{file2.pdf} 
%	 }
%      }
%   \end{figure}
 

%%%%% Including Code %%%%%%%%%%%%%%%%%%%%%5
% \verbatiminput{file.ext}    % Includes verbatim text from the file
% \texttt{text}	  % includes text in courier, or code-like, font
