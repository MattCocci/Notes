\documentclass[12pt]{article}

\author{Matthew Cocci}
\title{\textbf{Homework 3}}
\date{\today}

%% Spacing %%%%%%%%%%%%%%%%%%%%%%%%%%%%%%%%%%%%%%%%%%%%%%%%

\usepackage{fullpage}
\usepackage{setspace}
%\onehalfspacing
\usepackage{microtype}


%% Header %%%%%%%%%%%%%%%%%%%%%%%%%%%%%%%%%%%%%%%%%%%%%%%%%

%\pagestyle{fancy} 
%\lhead{}
%\rhead{}
%\chead{}
%\setlength{\headheight}{15.2pt} 
    %---Make the header bigger to avoid overlap

%\renewcommand{\headrulewidth}{0.3pt} 
    %---Width of the line

%\setlength{\headsep}{0.2in}    
    %---Distance from line to text
            

%% Mathematics Related %%%%%%%%%%%%%%%%%%%%%%%%%%%%%%%%%%%

\usepackage{amsmath}
\usepackage{amsfonts}
\usepackage{mathrsfs}
\usepackage{amsthm} %allows for labeling of theorems
\theoremstyle{plain}
\newtheorem{thm}{Theorem}[section]
\newtheorem{lem}[thm]{Lemma}
\newtheorem{prop}[thm]{Proposition}
\newtheorem{cor}[thm]{Corollary}

\theoremstyle{definition}
\newtheorem{defn}[thm]{Definition}
\newtheorem{ex}[thm]{Example}

\theoremstyle{remark}
\newtheorem*{rem}{Remark}
\newtheorem*{note}{Note}


%% Font Choices %%%%%%%%%%%%%%%%%%%%%%%%%%%%%%%%%%%%%%%%%

\usepackage[T1]{fontenc}
\usepackage[utf8]{inputenc}
%\usepackage{blindtext}


%% Figures %%%%%%%%%%%%%%%%%%%%%%%%%%%%%%%%%%%%%%%%%%%%%%

\usepackage{graphicx}
\usepackage{subfigure} 
    %---For plotting multiple figures at once
%\graphicspath{ {Directory/} }
    %---Set a directory for where to look for figures


%% Hyperlinks %%%%%%%%%%%%%%%%%%%%%%%%%%%%%%%%%%%%%%%%%%%%
\usepackage{hyperref} 
\hypersetup{	
    colorlinks,		
        %---This colors the links themselves, not boxes
    citecolor=black,	
        %---Everything here and below changes link colors
    filecolor=black,
    linkcolor=black,
    urlcolor=black
}

%% Including Code %%%%%%%%%%%%%%%%%%%%%%%%%%%%%%%%%%%%%%% 

\usepackage{verbatim} 
    %---For including verbatim code from files, no colors

\usepackage{listings}
\usepackage{color}
\definecolor{mygreen}{RGB}{28,172,0}
\definecolor{mylilas}{RGB}{170,55,241}
\newcommand{\matlabcode}[1]{%
    \lstset{language=Matlab,%
        basicstyle=\footnotesize,%
        breaklines=true,%
        morekeywords={matlab2tikz},%
        keywordstyle=\color{blue},%
        morekeywords=[2]{1}, keywordstyle=[2]{\color{black}},%
        identifierstyle=\color{black},%
        stringstyle=\color{mylilas},%
        commentstyle=\color{mygreen},%
        showstringspaces=false,%
            %---Without this there will be a symbol in 
            %---the places where there is a space
        numbers=left,%
        numberstyle={\tiny \color{black}},% 
            %---Size of the numbers
        numbersep=9pt,% 
            %---Defines how far the numbers are from the text
        emph=[1]{for,end,break,switch,case},emphstyle=[1]\color{red},%
            %---Some words to emphasise
    }%
    \lstinputlisting{#1}
}
    %---For including Matlab code from .m file with colors,
    %---line numbering, etc. 


%% Misc %%%%%%%%%%%%%%%%%%%%%%%%%%%%%%%%%%%%%%%%%%%%%% 

\usepackage{enumitem} 
    %---Has to do with enumeration	
\usepackage{appendix}
%\usepackage{natbib} 
    %---For bibliographies
\usepackage{pdfpages}
    %---For including whole pdf pages as a page in doc


%% User Defined %%%%%%%%%%%%%%%%%%%%%%%%%%%%%%%%%%%%%%%%%% 

%\newcommand{\nameofcmd}{Text to display}



%%%%%%%%%%%%%%%%%%%%%%%%%%%%%%%%%%%%%%%%%%%%%%%%%%%%%%%%%%%%%%%%%%%%%%%% 
%% BODY %%%%%%%%%%%%%%%%%%%%%%%%%%%%%%%%%%%%%%%%%%%%%%%%%%%%%%%%%%%%%%%%
%%%%%%%%%%%%%%%%%%%%%%%%%%%%%%%%%%%%%%%%%%%%%%%%%%%%%%%%%%%%%%%%%%%%%%%% 


\begin{document}

\maketitle 

\begin{enumerate}
    
\item We want to show that for differentiable $f$ where $f'\in\mathscr{R}([a,b])$, that $f\in BV([a,b])$ and $V^b_a(f) = \int^b_a |f'| \; dx$. 
\\
\\
We start by noting that since $f$ is differentiable, we know that $f$ is continuous. This will allow us to rewrite the components of any approximating sums we use, via the Mean Value Theorem. That is, for all subintervals, 
\[ 
    \exists c_i \in [x_{i-1}, x_i] \quad \text{s.t.} \quad
    f(x_i) - f(x_{i-1}) = f'(c_i) (x_{i} - x_{i-1})
\]
Thus, let's write the expression for the total variation, suitably modified to take advantage of the MVT:
\begin{align*}
    V^b_a(f) &= \sup_P \sum_{i=1}^n |f(x_i) - f(x_{i-1})| \qquad \forall P \\
    &= \sup_P \sum_{i=1}^n |f'(c_i)[x_i-x_{i-1}]| =\sup_P
    \sum_{i=1}^n |f'(c_i)| \Delta x_i
\end{align*}
Now let's consider the lower sum of $|f'|$, for any partition $P$:
\begin{align*}
    L(|f'|,P) = \sum_{i=1}^n m_i(|f'|) \Delta x_i 
\end{align*}
Now clearly, sicne $m_i(|f'|)$ is the minimum over the $i$th subinterval, we have that 
\begin{align*}
    m_i(|f'|) \Delta x_i  \leq |f'(c_i)| \Delta x_i \quad \forall \; i
    \quad \Rightarrow \quad L(|f'|,P) \leq V^b_a(f)
\end{align*}
Next, we know that the since $f'$ is Riemann integral, we must also have $|f'| \in \mathscr{R}_\alpha([a,b])$ as well. And so we can say that
    \[ \int^b_a f\;d\alpha = \sup_P \]
TIGHTEN THIS UP


\item Given $\alpha, \beta >0$ we have
\begin{align*}
    f_{\alpha, \beta}(x) = 
        \begin{cases}
            x^\alpha \sin(x^{-\beta}) & x\in (0, 1] \\
            0 & x=0 \\
        \end{cases}
\end{align*}
We want to show that $f_{\alpha,\beta}\in BV([0,1])$ if and only if $\alpha>\beta$.
\\
\\
Suppose that $\beta\geq\alpha$ and that 

Start by taking the derivative:
\begin{align*}
    f'_{\alpha, \beta}(x) = 
        \begin{cases} 
            \alpha x^{\alpha-1}\sin(x^{-\beta}) -
            \beta x^{\alpha-\beta-1}\cos(x^{-\beta}) & 
            x\in(0,1] \\
            0 & x=0
        \end{cases}
\end{align*}
Choose the partition that would give the sup
        

\item We suppose that $f\in BV([a,b])$ and want to show that
    \[ \sup_{x\in[0,1]} |f(x)| \leq \int^1_0 |f(x)|\;dx
        + V^1_0(f) \]
We start with the fact that since $f\in BV([a,b])$ and since, $\alpha(x)=x$, We start by writing
\begin{align*}
    f(x) = 
\end{align*}

\item 54.4: Any var is less than total var, but greater than if you have the abs value

54.6: Riemann integrable, flip around; then max is riemann integrable; flip again

\item We can use the function from question 2:
    \[ 
        f_{\alpha, \beta}(x) = 
        \begin{cases}
            x^\alpha \sin(x^{-\beta}) & x\in (0, 1] \\
            0 & x=0 \\
        \end{cases}
    \]
It's clearly differentiable on $(a,b)$, and that derivative equals
\begin{equation}
        {f'}_{\alpha, \beta}(x) = \alpha x^{\alpha-1} 
            \sin(x^{-\beta}) - \beta x^{\alpha-\beta-1} \cos(x^{-\beta})
\end{equation}
This gives us an easy hint.  If we take $\alpha = 0.5$ and $\beta=0.25$, we have that $\alpha>\beta$.  So by the answer to Question 2, we know that $f\in BV([a,b])$. However, because the exponents are negative in front of the sin and cos terms, we know that $f'$ is unbounded. 
    

\item 
\begin{enumerate}
\item \textbf{Exercise 55.5}: We want to show that for some $c\in(a,b)$,
\begin{align*}
    \int^b_a f\;d\alpha &= f(a)\int^c_a d\alpha 
        + f(b) \int^b_c d\alpha 
\end{align*}
So let's rewrite the integral $\int^b_a f\;d\alpha$ using the integration by parts theorem:
\begin{align}
    \label{q6.ibp}
    \int^b_a f\;d\alpha  
        &= f(b) \alpha(b) - f(a)\alpha(a) - \int^b_a\alpha \; df
\end{align}
But, since $\alpha$ is continuous and $f$ is assumed increasing, we can apply the Mean Value Theorem to assert that there exists some $c^*\in(a,b)$ such that 
\begin{equation}
    \int^b_a \alpha\;df = \alpha(c^*)[f(b)-f(a)]
\end{equation}
Subbing this into Equation \ref{q6.ibp}, we get that
\begin{align*}
    \int^b_a f\;d\alpha  
        &= f(b) \alpha(b)-f(a)\alpha(a)-\alpha(c^*)[f(b)-f(a)]\\
        &= f(b) \left[\alpha(b)-\alpha(c^*)\right]
            + f(a)\left[\alpha(c^*)-\alpha(a)\right]\\
        &= f(b) \int^b_{c^*} d\alpha
            + f(a)\int^{c^*}_a d\alpha
\end{align*}
Letting $c=c^*$ gives us exactly what we want.
\end{enumerate}



    
        
        

\end{enumerate}

\end{document}



%%%% INCLUDING FIGURES %%%%%%%%%%%%%%%%%%%%%%%%%%%%

   % H indicates here 
   %\begin{figure}[h!]
   %   \centering
   %   \includegraphics[scale=1]{file.pdf}
   %\end{figure}

%   \begin{figure}[h!]
%      \centering
%      \mbox{
%	 \subfigure{
%	    \includegraphics[scale=1]{file1.pdf}
%	 }\quad
%	 \subfigure{
%	    \includegraphics[scale=1]{file2.pdf} 
%	 }
%      }
%   \end{figure}
 

%%%%% Including Code %%%%%%%%%%%%%%%%%%%%%5
% \verbatiminput{file.ext}    % Includes verbatim text from the file
% \texttt{text}	  % includes text in courier, or code-like, font
