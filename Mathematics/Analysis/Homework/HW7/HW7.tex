\documentclass[12pt]{article}

\author{Matthew Cocci}
\title{\textbf{Homework 7}}
\date{\today}

%% Spacing %%%%%%%%%%%%%%%%%%%%%%%%%%%%%%%%%%%%%%%%%%%%%%%%

\usepackage{fullpage}
\usepackage{setspace}
%\onehalfspacing
\usepackage{microtype}


%% Header %%%%%%%%%%%%%%%%%%%%%%%%%%%%%%%%%%%%%%%%%%%%%%%%%

%\pagestyle{fancy} 
%\lhead{}
%\rhead{}
%\chead{}
%\setlength{\headheight}{15.2pt} 
    %---Make the header bigger to avoid overlap

%\renewcommand{\headrulewidth}{0.3pt} 
    %---Width of the line

%\setlength{\headsep}{0.2in}    
    %---Distance from line to text
            

%% Mathematics Related %%%%%%%%%%%%%%%%%%%%%%%%%%%%%%%%%%%

\usepackage{amsmath}
\usepackage{amsfonts}
\usepackage{mathrsfs}
\usepackage{amsthm} %allows for labeling of theorems
\theoremstyle{plain}
\newtheorem{thm}{Theorem}[section]
\newtheorem{lem}[thm]{Lemma}
\newtheorem{prop}[thm]{Proposition}
\newtheorem{cor}[thm]{Corollary}

\theoremstyle{definition}
\newtheorem{defn}[thm]{Definition}
\newtheorem{ex}[thm]{Example}

\theoremstyle{remark}
\newtheorem*{rem}{Remark}
\newtheorem*{note}{Note}


%% Font Choices %%%%%%%%%%%%%%%%%%%%%%%%%%%%%%%%%%%%%%%%%

\usepackage[T1]{fontenc}
\usepackage[utf8]{inputenc}
\usepackage{lmodern}
%\usepackage{blindtext}


%% Figures %%%%%%%%%%%%%%%%%%%%%%%%%%%%%%%%%%%%%%%%%%%%%%

\usepackage{graphicx}
\usepackage{subfigure} 
    %---For plotting multiple figures at once
%\graphicspath{ {Directory/} }
    %---Set a directory for where to look for figures


%% Hyperlinks %%%%%%%%%%%%%%%%%%%%%%%%%%%%%%%%%%%%%%%%%%%%
\usepackage{hyperref} 
\hypersetup{	
    colorlinks,		
        %---This colors the links themselves, not boxes
    citecolor=black,	
        %---Everything here and below changes link colors
    filecolor=black,
    linkcolor=black,
    urlcolor=black
}

%% Including Code %%%%%%%%%%%%%%%%%%%%%%%%%%%%%%%%%%%%%%% 

\usepackage{verbatim} 
    %---For including verbatim code from files, no colors

\usepackage{listings}
\usepackage{color}
\definecolor{mygreen}{RGB}{28,172,0}
\definecolor{mylilas}{RGB}{170,55,241}
\newcommand{\matlabcode}[1]{%
    \lstset{language=Matlab,%
        basicstyle=\footnotesize,%
        breaklines=true,%
        morekeywords={matlab2tikz},%
        keywordstyle=\color{blue},%
        morekeywords=[2]{1}, keywordstyle=[2]{\color{black}},%
        identifierstyle=\color{black},%
        stringstyle=\color{mylilas},%
        commentstyle=\color{mygreen},%
        showstringspaces=false,%
            %---Without this there will be a symbol in 
            %---the places where there is a space
        numbers=left,%
        numberstyle={\tiny \color{black}},% 
            %---Size of the numbers
        numbersep=9pt,% 
            %---Defines how far the numbers are from the text
        emph=[1]{for,end,break,switch,case},emphstyle=[1]\color{red},%
            %---Some words to emphasise
    }%
    \lstinputlisting{#1}
}
    %---For including Matlab code from .m file with colors,
    %---line numbering, etc. 


%% Misc %%%%%%%%%%%%%%%%%%%%%%%%%%%%%%%%%%%%%%%%%%%%%% 

\usepackage{enumitem} 
    %---Has to do with enumeration	
\usepackage{appendix}
%\usepackage{natbib} 
    %---For bibliographies
\usepackage{pdfpages}
    %---For including whole pdf pages as a page in doc


%% User Defined %%%%%%%%%%%%%%%%%%%%%%%%%%%%%%%%%%%%%%%%%% 

%\newcommand{\nameofcmd}{Text to display}
\newcommand*{\Chi}{\mbox{\large$\chi$}} %big chi



%%%%%%%%%%%%%%%%%%%%%%%%%%%%%%%%%%%%%%%%%%%%%%%%%%%%%%%%%%%%%%%%%%%%%%%% 
%% BODY %%%%%%%%%%%%%%%%%%%%%%%%%%%%%%%%%%%%%%%%%%%%%%%%%%%%%%%%%%%%%%%%
%%%%%%%%%%%%%%%%%%%%%%%%%%%%%%%%%%%%%%%%%%%%%%%%%%%%%%%%%%%%%%%%%%%%%%%% 


\begin{document}

\maketitle 

\begin{enumerate} 

% Question 1
\item 
    
\begin{enumerate} 

% Gunturk supplement
\item 
First, let $A\in\mathscr{M}$ and $r\in\mathbb{R}$. We want to show that 
\[
    m(r+A) = m(A) \qquad m(rA) = |r|m(A)
\]
\textbf{Translation Invariance}: Let $\{I_n\}$ be a countable set of open intervals covering that cover $A$ such that
\[
    m(A) + \varepsilon = \sum^\infty_{n=1} \ell(I_n) := 
    \sum^\infty_{n=1} \ell((a_n, b_n))
\]
which we know to exist, since $m(A)$ is the infimum over all such open interval covers. Since $\{I_n\}$ covers $A$, it's clear that
\[
    \bigcup^\infty_{n=1} (I_n + r) \supset A+r
\]
This is true because for all $p\in A+r$, we know that $p=q+r$ for some $q\in A$. In that case, there is a $n$ such that $I_n:=(a_n,b_n)\ni q$, which implies $I_n + r := (a_n+r, b_n+r) \ni p$.
\\
\\
And so we now have an open interval cover of $A+r$, so that 
\begin{align*}
    m(A+r) = \inf_{\{J_n\}}
        \left\{\sum^\infty_{n=1} \ell(J_n)\right\}
        &\leq \sum^\infty_{n=1} \ell(I_n + r) 
        = \sum^\infty_{n=1} [(b_n+r)-(a_n+r)] \\
    \Leftrightarrow \qquad
        &\leq \sum^\infty_{n=1} (b_n-a_n) 
        = \sum^\infty_{n=1} \ell(I_n) \\
    \Leftrightarrow \qquad
        &\leq m(A) + \varepsilon
\end{align*}
But $\varepsilon>0$ was arbitrary, so that 
\begin{equation}
    m(A+r) \leq m(A) 
\end{equation}
Now, if we proceed in exactly the same way as above, but take $\{I_n\}$ to be a cover of $m(A+r)$ such that 
\[
    m(A + r) + \varepsilon = \sum^\infty_{n=1} \ell(I_n) 
\]
Then, $\{J_n\}$ (where $J_n:=I_n-r$) will be an open cover of $A$ and the logic of the steps above can be adapted to get
\begin{equation}
    m(A) \leq m(A+r)
\end{equation}
Implying, together with the other direction above, that 
\[
    m(A) = m(A+r)
\]
\textbf{Scaling}: First, suppose $r=0$. Then $rA=\{0\}$ impliying $m(rA)=0$ since $(-1/n, 1/n)$ will cover $\{0\}$ for all $n$, so that the smallest open interval cover can be made arbitrarily tiny. Thus, $m(rA)=0=|0|\cdot m(A)$ for all $A$.
\\
\\
Next, suppose that $r\neq 0$. Again, let $\{I_n\}$ be a countable collection of intervals covering $A$. In exactly the same way as above, $\{r I_n\} := \{(r a_n,r  b_n)\}$ will cover $rA$, since
\[
    \sum_{n=1}^\infty \ell(rI_n) = |r| \sum_{n=1}^\infty\ell(I_n)
\]
Then everything follows like we saw above with the additive case.

% Book exercies
\item     

\textbf{Exercise 89.1}: We want to show that for $A\in\mathscr{M}_m$ (the collection of Lebesgue-measurable sets) and $r\in\mathbb{R}$, that 
\begin{equation}
    \label{q1.given}
    \{x+r\;|\;x\in A\}, 
    \{-x\;|\;x\in A\}, 
    \{xr\;|\;x\in A\}\in \mathscr{M}_m
\end{equation}
To show this, first note that the part (a) was unduly restrictive.  We could have worked, more generally, with the Lebesgue outer measure $m^*$, and the same logic would still have applied, since the fact that $A\in\mathscr{M}$ entered nowhere in the proof. $A$ could just as well have been any arbitrary set in the power set, $\mathscr{P}(\mathbb{R})$, since the outer measure works with open-interval coverings as well.
\\
\\
So it's clear that we also have
\[
    m^*(A + r) = m^*(A) \qquad m^*(rA) = |r|m^*(A)
\]
So let's start by showing that if $A\in \mathscr{M}_m$, then $A_r:=A+r \in \mathscr{M}_m$. To do so, suppose by contradiction that $A_r \not\in \mathscr{M}_m$. Then there exists a set (suggestively) called $S_r := S + r$ such that 
\begin{align*}
    \label{q1.contra}
    m^*(S_r) &\neq m^*(S_r \cap A_r) +m^*(S_r \cap A_r^c) \\
    m^*(S+r) &\neq m^*((S + r) \cap (A+r)) +m^*((S+r) \cap (A+r)^c) \\
    m^*(S+r) &\neq m^*((S\cap A)+ r) +m^*((S+r) \cap (A^c+r)) \\
    m^*(S+r) &\neq m^*((S\cap A)+ r) +m^*((S\cap A^c)+r) 
\end{align*}
Applying the fact that $m^*$ is translation invariant:
\begin{align*}
    m^*(S) &\neq m^*(S\cap A) +m^*(S\cap A^c) 
\end{align*}
This is clearly a contradiction, since we assumed $A\in\mathscr{M}_m$, implying the last line holds for arbitrary $S$
\\
\\
Next, suppose that $A\in \mathscr{M}_m$, and by contradiction, that $A_r:=rA\not\in \mathscr{M}_m$. If $r=0$, then the result is trivial, as $A_r = \{0\}$, which is a closed set that will be in $\mathscr{M}_m$ since it contains the Borel $\sigma$-algebra (and, hence, all closed sets).
\\
\\
Now suppose $r\neq 0$. Then since $A_r\not\in\mathscr{M}_m$, there exists a set $S_r:=rS$ such that 
\begin{align*}
    m^*(S_r) &\neq m^*(S_r \cap A_r) +m^*(S_r \cap A_r^c) \\
    m^*(rS) &\neq m^*(rS \cap rA) +m^*(rS \cap (rA)^c) \\
    m^*(rS) &\neq m^*(rS \cap rA) +m^*(rS \cap rA^c) \\
    m^*(rS) &\neq m^*(r(S \cap A)) +m^*(r(S \cap A^c)) 
\end{align*}
Now, applying the result above for scaling, 
\begin{align*}
    |r|m^*(S) &\neq |r|m^*(S \cap A) + |r| m^*(S \cap A^c)
\end{align*}
Since $r\neq 0$, we can divide through to derive a contradiction to the assumption that $A\in\mathscr{M}_m$:
\begin{align*}
    m^*(S) &\neq m^*(S \cap A) + m^*(S \cap A^c)
\end{align*}

\end{enumerate}

% Question 2
\item 
\begin{enumerate}

% Q2.a
\item \textbf{Exercise 89.6}: We want to prove that $E\subset \mathbb{R}$ is $m^*$-measurable if and only if, for every $\varepsilon>0$, there exists a closed set $F\subset E$ such that $m^*(E\setminus F)< \varepsilon$.
\\
\\
($\Rightarrow$ Direction) Suppose that $E\subset \mathscr{M}$ is $m^*$-measurable. By Theorem 89.13, we can approximate $E^c$ from the outside, i.e. there exists a open set (suggestively) named $F^c\supset E^c$ for all $\varepsilon>0$ such that
\begin{equation}
    \label{q2a.approx}
    m(F^c) \leq m(E^c) + \varepsilon 
\end{equation}
Then, since $F^c$ is open and $F^c\supset E^c$, $F$ will be closed and $F \subset E$. So now we have our candidate for $F$; let's show the final result.
\\
\\
Since $E$ is measurable, we can use Caratheodory's Criterion, along with the fact that $F^c \supset E^c$ (by construction) and $F^c \cap E = E\setminus F$ (by simple set theory):
\begin{align*}
    m(F^c) &= m(F^c \cap E) + m(F^c \cap E^c)\\
    &= m(E\setminus F) + m(E^c)
\end{align*}
Then we can substitute in for $m(F^c)$ using the inequality in (\ref{q2a.approx}):
\begin{align*}
    m(E^c) + \varepsilon &\geq m(E\setminus F) + m(E^c)
\end{align*}
Now, supposing that $m(E^c)$ is finite, we can immediately cancel to get the desired result:
\begin{align*}
    \varepsilon &\geq m(E\setminus F) 
\end{align*}
If, on the other hand, $m(E^c)$ is infinite, we have to take a different approach, working with another representation of $\mathbb{R}$:
\[
    \mathbb{R} = \bigcup^\infty_{n=1} I_n = \bigcup^\infty_{n=1} [-n,n]
\]
Then, working with the sets
\[
    E_n = E \cap I_n 
\]
we can apply what we did above to these finite sets.
\\
\\
($\Leftarrow$ Direction) Suppose that for all $\varepsilon>0$, there exists a closed set $F\subset E$ such that $m^*(E\setminus F)<\varepsilon$. We want to show $E$ is $m^*$-measurable. But first, let's prove a result that we will use later on. Namely, we want to show for any $T$ and $F\subset E$
\begin{align}  
    (T\cap E) \setminus (E\setminus F) 
        &= (T\cap E) \cap (E \cap F^c)^c \notag\\
        &= T\cap E \cap (E^c \cup F) \notag\\
        &= T\cap ((E \cap E^c) \cup (E\cap F)) \notag\\
        &= T\cap ((\emptyset)\cup F) \notag\\
        &= T\cap F
        \label{q2.preproof}
\end{align}
Now, we can move onto the main proof. Let $T$ be an arbitrary subset of $\mathbb{R}$, and form the basis for Caratheodory's criterion. Then we can split up $T\cap E$:
\begin{align*}
    m^*(T\cap E^c) +  m^*(T\cap E) = 
    m^*(T\cap E^c) +  m^*(( (T\cap E)\setminus (E\setminus F) )  
    \cup (E\setminus F))
\end{align*}
Next, apply countable additivity 
\begin{align*}
    m^*(T\cap E^c) +  m^*(T\cap E) &\leq 
    m^*(T\cap E^c) + m^*(E\setminus F) \\
    &\qquad   + m^*( (T\cap E)\setminus (E\setminus F) )  
\end{align*}
Next, use the result we derived about in \ref{q2.preproof}:
\begin{align}
    m^*(T\cap E^c) +  m^*(T\cap E) &\leq 
    m^*(T\cap E^c) + m^*(E\setminus F) 
    + m^*(T\cap F)
    \label{q2.almost}
\end{align}
Then, invoke monontonicity:
\[
    F\subset E 
    \quad \Rightarrow \quad
    F^c\supset E^c
    \quad \Rightarrow \quad
    (T\cap F^c)\supset (T\cap E^c)
    \quad \Rightarrow \quad
    m(T \cap F^c) \geq m(T \cap E^c)
\]
Now use this result to sub in for $m^*(T\cap E^c)$ in \ref{q2.almost}:
\begin{align}
    m^*(T\cap E^c) +  m^*(T\cap E) &\leq 
    m^*(T\cap F^c) + m^*(E\setminus F) 
    + m^*(T\cap F)
\end{align}
Finally, we can sub in for $m^*(E\setminus F)\leq \varepsilon$, while also using the fact that $F$ is $m^*$-measurable (since closed) to collapse things via Caratheodory's Criterion
\begin{align*}
    m^*(T\cap E^c) +  m^*(T\cap E) &\leq 
    [m^*(T\cap F^c) + 
     m^*(T\cap F)] 
    +\varepsilon \\
    &\leq 
    m(T)
    +\varepsilon 
\end{align*}
But since $\varepsilon$ was arbitrary, we have 
\begin{align*}
    m^*(T\cap E^c) +  m^*(T\cap E) &\leq 
    m(T)
\end{align*}
This implies that $E$ is $m^*$-measurable by Caratheodory's criterion. The other direction for the inequality sign is implied automatically by countable sub-additivity, which---togther with the result derived here---gives equality.

% Q2b
\item \textbf{Exercise 89.7d}: For $E\subset\mathbb{R}$, we want to show that $E$ is measurable if and only if $E = B\cup S$ where $B\in\mathscr{B}_\mathbb{R}$ and $m(S)=0$.
\\
\\
($\Rightarrow$ Direction) Suppose that $E$ is measurable. Let $1/n>0$ be given. Then according to the last problem, there exists a closed set $F_n\subset E$ such that $m(E\setminus F_n) \leq 1/n$. Letting
\[
    F = \bigcup^\infty_{n=1} F_n 
\]
it's clear that since all $F_n\subset E$, $F\subset E$ as well. Moreover, we note that
\begin{align*}
    E\setminus F &= E\cap \left(\bigcup^\infty_{n=1} F_n\right)^c  
    = E\cap \left(\bigcap^\infty_{n=1} F_n^c\right) \\
    &= \lim_{N\rightarrow\infty}
        E\cap \left(\bigcap^N_{n=1} F_n^c\right) 
\end{align*}
Moreover, we note that for all $N$,
\begin{align*}
    E\cap \left(\bigcap^N_{n=1} F_n^c\right) 
    \subset E \cap F^c_N
\end{align*}
so that we can use monotonicity of $m^*$
\begin{align*}
    \lim_{N\rightarrow\infty} 
    m^*\left(E\cap \left(\bigcap^N_{n=1} F_n^c\right) 
   \right)  \leq
    \lim_{N\rightarrow\infty} 
    m^*\left(\subset E \cap E_N\right) 
    \leq \lim_{N\rightarrow\infty} 1/N = 0
\end{align*}
Thus, we have our candidates for $B$ and $S$:
\begin{align*}
    E = B \cup S &=: (E \cap F) \cup (E\cap F^c)  \\
    \text{Since $F\subset E$} \quad &= F \cup (E\cap F^c)
\end{align*}
To finish up, not that $F$ is a countable union of closed sets. The Borel $\sigma$-algebra contains all open and closed sets, plus countable unions of them since it is a $\sigma$-algebra, implying $F\in\mathscr{B}_\mathbb{R}$. So $B=F$ works.
\\
\\
Next, we showed above that $m^*(E\cap F^c)=0$, so that works for $S$.




\end{enumerate}


% Question 3
\item We want to show that for all measurable sets $E\subset \mathbb{R}$, 
\[
    m(E) = \sup \{ m(K) \; | \; K \subset E, \; 
    \text{ $K$ compact}\}  
    = \sup \{ m(F) \; | \; F \subset E, \; 
    \text{ $F$ closed}\}  
\]
We start with the first equality. So suppose for now that $E$ is finite. Then let $A$ be a compact subset of $\mathbb{R}$ such that $A \supset E$. Then, by Theorem 89.13(i), we can approximate the set $A\setminus E$ by an open set $B$ from the outside (i.e. $B\supset A\setminus E$) for all $\varepsilon>0$ such that
\begin{equation}
    \label{q3.approx1}
    m(B) \leq m^*(A\setminus E) + \varepsilon
    \qquad \text{$B$ open}
\end{equation}
But since $B$ is open, $B^c$ is closed, so that 
\[
    K = A \setminus B = A \cap B^c 
\]
where the set representation of $K$ as the intersection of two closed sets (with $A$ compact), implies that $K$ is closed as well. On top of that, since $B$ is larger
\begin{align*}
    B \supset (A\setminus E) 
    \quad \Rightarrow\quad
    K &= (A\setminus B) \subset A\setminus (A\setminus E) \\
    &\subset A\cap (A \cap E^c)^c = A\cap (A^c \cup E)  \\
    &\subset (A \cap A^c) \cup (A\cap E)\\
    \Rightarrow\quad
    K &\subset E
\end{align*}
Finally, we use the fact that $E$ is measurable to say that
\begin{align*}
    m(A) &= m(A \cap E) + m(A \cap E^c) \\
        &= m(E) + m(A\setminus E ) \\
    \Rightarrow \quad
    m(E) &= m(A) - m(A \cap E^c) 
\end{align*}
Next, we use the fact that $A \subset (K\cup B)$ along with countable additivity:
\begin{align*}
    m(A) \leq m(K) + m(B)
\end{align*}
Now use our approximation \ref{q3.approx1} with these previous expressions:
\begin{align*}
    m(E) &= m(A) - m(A\cap E^c)  \\
    &\leq m(K) + m(B) - m(A\setminus E)\\
    &\leq m(K) + [m(A \setminus E)+\varepsilon] - m(A\setminus E)\\
    &\leq m(K) + \varepsilon
\end{align*}
But since $\varepsilon$ is arbitrary, we have $m(E) \leq m(K)$. The other direction follows from monotonicity, i.e.
\[
    K \subset E 
    \quad \Rightarrow \quad
    m(K) \leq m(E)
\]
Since the proof no way dependend upon compactness (only the fact that $K$ was closed), the second equality follows as well.


% Question 4
\item Let $\varepsilon>0$ be given. For a measurable $A\supset E$ and arbitrary $E$, we have by countable sub-additivity and the fact that $A$ is measurable,
\begin{align*}
    m(A) = m^*(A)&= m^*((A\cap E) \cup (A\cap E^c)) \\
    &\leq m^*(A\cap E) + m^*(A\cap E^c) 
    = m^*(E) + m^*(A\setminus E) 
\end{align*}
Though I wasn't able to get this, the proof will likely use the fact that for a measurable set $A$, clearly,
\[
    m_*(A) = m(A) = m^*(A)
\]
It will also make use of approximating sets by supersets that are open.



% Question 5
\item \textbf{Exercise 89.9}: Supposing that $f$ is differentiable on $[a,b]$. To show that $f'$ is $m$-measurable in $[a,b]$, note first that $f$ will be continuous since it is differentiable. Then, since $\mathscr{M}$, the set of all Lebesgue- or $m$-measurable  sets, contains the Borel $\sigma$-algebra, we know that $f$ will be $m$-measurable. This follows from the definition of a measurable function, since for any open $U\subset \bar{\mathbb{R}}$, we know that 
\[
    f^{-1}(U) \text{ open, by continuity} 
    \quad \Rightarrow \quad
    f^{-1}(U) \in \mathscr{B}_{\bar{\mathbb{R}}} \subset \mathscr{M}
\]
so that $f$ is $m$-measurable. Next define the following function:
\[
    f_n = \frac{f(x+1/n) - f(x)}{1/n}
\]
Since $f(x)$ is $m$-measurable, $f(x+1/n)$ will be as well, since we can use an alternative representation of measurability
\[
    f(x) \text{ measurable}
    \quad \Rightarrow \quad
    \{x \; | \; f(x)> a \} \in \mathscr{M} \quad 
    \text{for all $a\in\mathbb{R}$}
\]
But since $a$ is arbitrary, it's clear that by changing $a$, we can establish a one-to-one correspondence between 
\[
    \{x \; | \; f(x)> a_1 \} \quad
    \{x + 1/n \; | \; f(x)> a_2 \}
\]
And so the difference of two-measurable functions, $f(x)$ and $f(x+1/n)$ will be measurable. Moreover, the constant function $n = 1/(1/n)$ is certainly $m$-measurable for all $n$.
\\
\\ 
Putting everything together, we see that $f(x + 1/n) - f(x)$, as the difference of two $m$-measurable functions, is $m$-measurable. Then, multiplied by another $m$-measurable function $1/(1/n)$, the result will again be $m$-measurable. So $f_n$ itself is $m$-measurable. Finally, pointwise limits of measurable functions are measurable, and the fact that $f$ is differentiable let's us say
\[
    \lim_{n\rightarrow\infty} 
    \frac{f(x+1/n) - f(x)}{1/n} = f'(x)
\]
which will be $m$-measurable.

% Question 6
\item \textbf{Exercise 89.10}: For $X,Y\subset\mathbb{R}$, we want to show that  
\begin{align*}
    \inf\{|x-y| \; |\; x\in X, y\in Y\} 
    \quad \Rightarrow\quad
    m(X\cup Y) = m(X) + m(Y)
\end{align*}
To do so, note that the condition on the left effectively says that $X$ and $Y$ are disjoint. Otherwise, there would be a $p\in X, Y$, allowing $|p-p|=0$. Also, let 
\[
    q =
    \inf\{|x-y| \; |\; x\in X, y\in Y\}  > 0
\]
Next, let $U$ be an open cover of $X \cup Y$ such that 
\begin{align}
    \label{q6.approx}
    m(U) \leq m^*(X \cup Y) + \varepsilon
\end{align}
Then, we can construct separate open covers of $X$ and $Y$: 
\begin{align*}
    X \subset U_1 &:= U \cap \{x \in \mathbb{R} \; | \; \inf_{p\in X} |x-p| < q/2 \} \\
    Y \subset U_2 &:= U \cap \{y \in \mathbb{R} \; | \; \inf_{p\in Y} |y-p| < q/2 \}
\end{align*}
It's clear that if $x\in X$, $x\in U_1$. It's also clear that $U_1$ and $U_2$ are disjoint. So now, use monotonicity to write out
\begin{align*}
    m^*(X) + m^*(Y) \leq 
    m^*(U_1) + m^*(U_2)
\end{align*}
Next, use the fact that $U_1$ and $U_2$ are open, implying they are $m^*$-measurable, permitting countable additivity on the last expression:
\begin{align*}
    m^*(X) + m^*(Y) \leq 
    m^*(U_1) + m^*(U_2)= 
    m(U_1) + m(U_2)= m(U_1\cup U_2)
\end{align*}
Next, use the monotonicity again, since $U\cap (U_1 \cup U_2)$ and use our approximation (\ref{q6.approx}), to write
\begin{align*}
    m^*(X) + m^*(Y) &\leq 
    m^*(U_1) + m^*(U_2)= 
    m(U_1) + m(U_2)= m(U_1\cup U_2)\\
    &\leq m(U) \leq m^*(X \cup Y) + \varepsilon
\end{align*}
But $\varepsilon$ was arbitrary, implying that
\begin{align*}
    m^*(X) + m^*(Y) &\leq 
    m^*(X \cup Y)
\end{align*}
The other direction, 
\begin{align*}
    m^*(X) + m^*(Y) &\geq 
    m^*(X \cup Y)
\end{align*}
follows from countable sub-additivity, allowing us to conclude equivalence. 





\end{enumerate}
\end{document}



%%%% INCLUDING FIGURES %%%%%%%%%%%%%%%%%%%%%%%%%%%%

   % H indicates here 
   %\begin{figure}[h!]
  %   \centering
   %   \includegraphics[scale=1]{file.pdf}
   %\end{figure}

%   \begin{figure}[h!]
%      \centering
%      \mbox{
%	 \subfigure{
%	    \includegraphics[scale=1]{file1.pdf}
%	 }\quad
%	 \subfigure{
%	    \includegraphics[scale=1]{file2.pdf} 
%	 }
%      }
%   \end{figure}
 

%%%%% Including Code %%%%%%%%%%%%%%%%%%%%%5
% \verbatiminput{file.ext}    % Includes verbatim text from the file
% \texttt{text}	  % includes text in courier, or code-like, font
