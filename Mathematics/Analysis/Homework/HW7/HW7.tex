\documentclass[12pt]{article}

\author{Matthew Cocci}
\title{\textbf{Homework 7}}
\date{\today}

%% Spacing %%%%%%%%%%%%%%%%%%%%%%%%%%%%%%%%%%%%%%%%%%%%%%%%

\usepackage{fullpage}
\usepackage{setspace}
%\onehalfspacing
\usepackage{microtype}


%% Header %%%%%%%%%%%%%%%%%%%%%%%%%%%%%%%%%%%%%%%%%%%%%%%%%

%\pagestyle{fancy} 
%\lhead{}
%\rhead{}
%\chead{}
%\setlength{\headheight}{15.2pt} 
    %---Make the header bigger to avoid overlap

%\renewcommand{\headrulewidth}{0.3pt} 
    %---Width of the line

%\setlength{\headsep}{0.2in}    
    %---Distance from line to text
            

%% Mathematics Related %%%%%%%%%%%%%%%%%%%%%%%%%%%%%%%%%%%

\usepackage{amsmath}
\usepackage{amsfonts}
\usepackage{mathrsfs}
\usepackage{amsthm} %allows for labeling of theorems
\theoremstyle{plain}
\newtheorem{thm}{Theorem}[section]
\newtheorem{lem}[thm]{Lemma}
\newtheorem{prop}[thm]{Proposition}
\newtheorem{cor}[thm]{Corollary}

\theoremstyle{definition}
\newtheorem{defn}[thm]{Definition}
\newtheorem{ex}[thm]{Example}

\theoremstyle{remark}
\newtheorem*{rem}{Remark}
\newtheorem*{note}{Note}


%% Font Choices %%%%%%%%%%%%%%%%%%%%%%%%%%%%%%%%%%%%%%%%%

\usepackage[T1]{fontenc}
\usepackage[utf8]{inputenc}
\usepackage{lmodern}
%\usepackage{blindtext}


%% Figures %%%%%%%%%%%%%%%%%%%%%%%%%%%%%%%%%%%%%%%%%%%%%%

\usepackage{graphicx}
\usepackage{subfigure} 
    %---For plotting multiple figures at once
%\graphicspath{ {Directory/} }
    %---Set a directory for where to look for figures


%% Hyperlinks %%%%%%%%%%%%%%%%%%%%%%%%%%%%%%%%%%%%%%%%%%%%
\usepackage{hyperref} 
\hypersetup{	
    colorlinks,		
        %---This colors the links themselves, not boxes
    citecolor=black,	
        %---Everything here and below changes link colors
    filecolor=black,
    linkcolor=black,
    urlcolor=black
}

%% Including Code %%%%%%%%%%%%%%%%%%%%%%%%%%%%%%%%%%%%%%% 

\usepackage{verbatim} 
    %---For including verbatim code from files, no colors

\usepackage{listings}
\usepackage{color}
\definecolor{mygreen}{RGB}{28,172,0}
\definecolor{mylilas}{RGB}{170,55,241}
\newcommand{\matlabcode}[1]{%
    \lstset{language=Matlab,%
        basicstyle=\footnotesize,%
        breaklines=true,%
        morekeywords={matlab2tikz},%
        keywordstyle=\color{blue},%
        morekeywords=[2]{1}, keywordstyle=[2]{\color{black}},%
        identifierstyle=\color{black},%
        stringstyle=\color{mylilas},%
        commentstyle=\color{mygreen},%
        showstringspaces=false,%
            %---Without this there will be a symbol in 
            %---the places where there is a space
        numbers=left,%
        numberstyle={\tiny \color{black}},% 
            %---Size of the numbers
        numbersep=9pt,% 
            %---Defines how far the numbers are from the text
        emph=[1]{for,end,break,switch,case},emphstyle=[1]\color{red},%
            %---Some words to emphasise
    }%
    \lstinputlisting{#1}
}
    %---For including Matlab code from .m file with colors,
    %---line numbering, etc. 


%% Misc %%%%%%%%%%%%%%%%%%%%%%%%%%%%%%%%%%%%%%%%%%%%%% 

\usepackage{enumitem} 
    %---Has to do with enumeration	
\usepackage{appendix}
%\usepackage{natbib} 
    %---For bibliographies
\usepackage{pdfpages}
    %---For including whole pdf pages as a page in doc


%% User Defined %%%%%%%%%%%%%%%%%%%%%%%%%%%%%%%%%%%%%%%%%% 

%\newcommand{\nameofcmd}{Text to display}
\newcommand*{\Chi}{\mbox{\large$\chi$}} %big chi



%%%%%%%%%%%%%%%%%%%%%%%%%%%%%%%%%%%%%%%%%%%%%%%%%%%%%%%%%%%%%%%%%%%%%%%% 
%% BODY %%%%%%%%%%%%%%%%%%%%%%%%%%%%%%%%%%%%%%%%%%%%%%%%%%%%%%%%%%%%%%%%
%%%%%%%%%%%%%%%%%%%%%%%%%%%%%%%%%%%%%%%%%%%%%%%%%%%%%%%%%%%%%%%%%%%%%%%% 


\begin{document}

\maketitle 

\begin{enumerate} 

% Question 1
\item 
    
\begin{enumerate} 

% Gunturk supplement
\item 
First, let $A\in\mathscr{M}$ and $r\in\mathbb{R}$. We want to show that 
\[
    m(r+A) = m(A) \qquad m(rA) = |r|m(A)
\]
\textbf{Translation Invariance}: Let $\{I_n\}$ be a countable set of open intervals covering that cover $A$ such that
\[
    m(A) + \varepsilon = \sum^\infty_{n=1} \ell(I_n) := 
    \sum^\infty_{n=1} \ell((a_n, b_n))
\]
which we know to exist, since $m(A)$ is the infimum over all such open interval covers. Since $\{I_n\}$ covers $A$, it's clear that
\[
    \bigcup^\infty_{n=1} (I_n + r) \supset A+r
\]
This is true because for all $p\in A+r$, we know that $p=q+r$ for some $q\in A$. In that case, there is a $n$ such that $I_n:=(a_n,b_n)\ni q$, which implies $I_n + r := (a_n+r, b_n+r) \ni p$.
\\
\\
And so we now have an open interval cover of $A+r$, so that 
\begin{align*}
    m(A+r) = \inf_{\{J_n\}}
        \left\{\sum^\infty_{n=1} \ell(J_n)\right\}
        &\leq \sum^\infty_{n=1} \ell(I_n + r) 
        = \sum^\infty_{n=1} [(b_n+r)-(a_n+r)] \\
    \Leftrightarrow \qquad
        &\leq \sum^\infty_{n=1} (b_n-a_n) 
        = \sum^\infty_{n=1} \ell(I_n) \\
    \Leftrightarrow \qquad
        &\leq m(A) + \varepsilon
\end{align*}
But $\varepsilon>0$ was arbitrary, so that 
\begin{equation}
    m(A+r) \leq m(A) 
\end{equation}
Now, if we proceed in exactly the same way as above, but take $\{I_n\}$ to be a cover of $m(A+r)$ such that 
\[
    m(A + r) + \varepsilon = \sum^\infty_{n=1} \ell(I_n) 
\]
Then, $\{J_n\}$ (where $J_n:=I_n-r$) will be an open cover of $A$ and the logic of the steps above can be adapted to get
\begin{equation}
    m(A) \leq m(A+r)
\end{equation}
Implying, together with the other direction above, that 
\[
    m(A) = m(A+r)
\]
\textbf{Scaling}: First, suppose $r=0$. Then $rA=\{0\}$ impliying $m(rA)=0$ since $(-1/n, 1/n)$ will cover $\{0\}$ for all $n$, so that the smallest open interval cover can be made arbitrarily tiny. Thus, $m(rA)=0=|0|\cdot m(A)$ for all $A$.
\\
\\
Next, suppose that $r\neq 0$. Again, let $\{I_n\}$ be a countable collection of intervals covering $A$. Since 
\\
\\
Easy, finish this up

% Book exercies
\item     

\textbf{Exercise 89.1}: We want to show that for $A\in\mathscr{M}_m$ (the collection of Lebesgue-measurable sets) and $r\in\mathbb{R}$, that 
\begin{equation}
    \label{q1.given}
    \{x+r\;|\;x\in A\}, 
    \{-x\;|\;x\in A\}, 
    \{xr\;|\;x\in A\}\in \mathscr{M}_m
\end{equation}
To show this, first note that the part (a) was unduly restrictive.  We could have worked, more generally, with the Lebesgue outer measure $m^*$, and the same logic would still have applied, since the fact that $A\in\mathscr{M}$ entered nowhere in the proof. $A$ could just as well have been any arbitrary set in the power set, $\mathscr{P}(\mathbb{R})$, since the outer measure works with open-interval coverings as well.
\\
\\
So it's clear that we also have
\[
    m^*(A + r) = m^*(A) \qquad m^*(rA) = |r|m^*(A)
\]
To show this, we will show an equivalent statement, for all $a,b\in\mathbb{R}$:
\begin{equation}
    \label{q1.equiv}
    B = \{ax+b\;|\;x\in A\}\in\mathscr{M}_m
\end{equation}
Then, we can get each set in Statement \ref{q1.given} by taking $a=1$, $b=r$; $a=-1$, $b=0$; $a=r$, $b=0$, respectively.

\end{enumerate}

% Question 2
\item 
\begin{enumerate}

% Q2.a
\item \textbf{Exercise 89.6}: We want to prove that $E\subset \mathbb{R}$ is $m^*$-measurable if and only if, for every $\varepsilon>0$, there exists a closed set $F\subset E$ such that $m^*(E\setminus F)< \varepsilon$.
\\
\\
($\Rightarrow$ Direction) Suppose that $E\subset \mathscr{M}$ is $m^*$-measurable. By Theorem 89.13, we can approximate $E^c$ from the outside, i.e. there exists a open set (suggestively) named $F^c\supset E^c$ for all $\varepsilon>0$ such that
\begin{equation}
    \label{q2a.approx}
    m(F^c) \leq m(E^c) + \varepsilon 
\end{equation}
Then, since $F^c$ is open and $F^c\supset E^c$, $F$ will be closed and $F \subset E$. So now we have our candidate for $F$; let's show the final result.
\\
\\
Since $E$ is measurable, we can use Caratheodory's Criterion, along with the fact that $F^c \supset E^c$ (by construction) and $F^c \cap E = E\setminus F$ (by simple set theory):
\begin{align*}
    m(F^c) &= m(F^c \cap E) + m(F^c \cap E^c)\\
    &= m(E\setminus F) + m(E^c)
\end{align*}
Then we can substitute in for $m(F^c)$ using the inequality in (\ref{q2a.approx}):
\begin{align*}
    m(E^c) + \varepsilon &\geq m(E\setminus F) + m(E^c)
\end{align*}
Now, supposing that $m(E^c)$ is finite, we can immediately cancel to get the desired result:
\begin{align*}
    \varepsilon &\geq m(E\setminus F) 
\end{align*}
If, on the other hand, $m(E^c)$ is infinite, we have to take a different approach.
\\
\\
($\Leftarrow$ Direction) Suppose that for all $\varepsilon>0$, there exists a closed set $F\subset E$ such that $m^*(E\setminus F)<\varepsilon$. We want to show $E$ is $m^*$-measurable. But first, let's prove a result that we will use later on. Namely, we want to show for any $T$ and $F\subset E$
\begin{align}  
    (T\cap E) \setminus (E\setminus F) 
        &= (T\cap E) \cap (E \cap F^c)^c \notag\\
        &= T\cap E \cap (E^c \cup F) \notag\\
        &= T\cap ((E \cap E^c) \cup (E\cap F)) \notag\\
        &= T\cap ((\emptyset)\cup F) \notag\\
        &= T\cap F
        \label{q2.preproof}
\end{align}
Now, we can move onto the main proof. Let $T$ be an arbitrary subset of $\mathbb{R}$, and form the basis for Caratheodory's criterion. Then we can split up $T\cap E$:
\begin{align*}
    m^*(T\cap E^c) +  m^*(T\cap E) = 
    m^*(T\cap E^c) +  m^*(( (T\cap E)\setminus (E\setminus F) )  
    \cup (E\setminus F))
\end{align*}
Next, apply countable additivity 
\begin{align*}
    m^*(T\cap E^c) +  m^*(T\cap E) &\leq 
    m^*(T\cap E^c) + m^*(E\setminus F) \\
    &\qquad   + m^*( (T\cap E)\setminus (E\setminus F) )  
\end{align*}
Next, use the result we derived about in \ref{q2.preproof}:
\begin{align}
    m^*(T\cap E^c) +  m^*(T\cap E) &\leq 
    m^*(T\cap E^c) + m^*(E\setminus F) 
    + m^*(T\cap F)
    \label{q2.almost}
\end{align}
Then, invoke monontonicity:
\[
    F\subset E 
    \quad \Rightarrow \quad
    F^c\supset E^c
    \quad \Rightarrow \quad
    (T\cap F^c)\supset (T\cap E^c)
    \quad \Rightarrow \quad
    m(T \cap F^c) \geq m(T \cap E^c)
\]
Now use this result to sub in for $m^*(T\cap E^c)$ in \ref{q2.almost}:
\begin{align}
    m^*(T\cap E^c) +  m^*(T\cap E) &\leq 
    m^*(T\cap F^c) + m^*(E\setminus F) 
    + m^*(T\cap F)
\end{align}
Finally, we can sub in for $m^*(E\setminus F)\leq \varepsilon$, while also using the fact that $F$ is $m^*$-measurable (since closed) to collapse things via Caratheodory's Criterion
\begin{align*}
    m^*(T\cap E^c) +  m^*(T\cap E) &\leq 
    [m^*(T\cap F^c) + 
     m^*(T\cap F)] 
    +\varepsilon \\
    &\leq 
    m(T)
    +\varepsilon 
\end{align*}
But since $\varepsilon$ was arbitrary, we have 
\begin{align*}
    m^*(T\cap E^c) +  m^*(T\cap E) &\leq 
    m(T)
\end{align*}
This implies that $E$ is $m^*$-measurable by Caratheodory's criterion. The other direction for the inequality sign is implied automatically by countable sub-additivity, which---togther with the result derived here---gives equality.

% Q2b
\item \textbf{Exercise 89.7d}:


\end{enumerate}


% Question 3
\item We want to show that for all measurable sets $E\subset \mathbb{R}$, 
\[
    m(E) = \sup \{ m(K) \; | \; K \subset E, \; 
    \text{ $K$ compact}\}  
    = \sup \{ m(F) \; | \; F \subset E, \; 
    \text{ $F$ closed}\}  
\]
We start with the first equality. So suppose for now that $E$ is finite. Then let $A$ be a compact subset of $\mathbb{R}$ such that $A \supset E$. Then, by Theorem 89.13(i), we can approximate the set $A\setminus E$ by an open set $B$ from the outside (i.e. $B\supset A\setminus E$) for all $\varepsilon>0$ such that
\begin{equation}
    \label{q3.approx1}
    m(B) \leq m^*(A\setminus E) + \varepsilon
    \qquad \text{$B$ open}
\end{equation}
But since $B$ is open, $B^c$ is closed, so that 
\[
    K = A \setminus B = A \cap B^c 
\]
where the set representation of $K$ as the intersection of two closed sets (with $A$ compact), implies that $K$ is compact as well. On top of that, since $B$ is larger
\begin{align*}
    B \supset (A\setminus E) 
    \quad \Rightarrow\quad
    K &= (A\setminus B) \subset A\setminus (A\setminus E) \\
    &\subset A\cap (A \cap E^c)^c = A\cap (A^c \cup E)  \\
    &\subset (A \cap A^c) \cup (A\cap E)\\
    \Rightarrow\quad
    K &\subset E
\end{align*}
Finally, we can apply Caratheodory's Criterion to $E$ since it's measurable, then use the approximation (\ref{q3.approx1}):
\begin{align*}
    m(A) = m(A\cap E) 
\end{align*}




\end{enumerate}
\end{document}



%%%% INCLUDING FIGURES %%%%%%%%%%%%%%%%%%%%%%%%%%%%

   % H indicates here 
   %\begin{figure}[h!]
  %   \centering
   %   \includegraphics[scale=1]{file.pdf}
   %\end{figure}

%   \begin{figure}[h!]
%      \centering
%      \mbox{
%	 \subfigure{
%	    \includegraphics[scale=1]{file1.pdf}
%	 }\quad
%	 \subfigure{
%	    \includegraphics[scale=1]{file2.pdf} 
%	 }
%      }
%   \end{figure}
 

%%%%% Including Code %%%%%%%%%%%%%%%%%%%%%5
% \verbatiminput{file.ext}    % Includes verbatim text from the file
% \texttt{text}	  % includes text in courier, or code-like, font
