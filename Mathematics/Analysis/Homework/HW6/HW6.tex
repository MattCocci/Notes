\documentclass[12pt]{article}

\author{Matthew Cocci}
\title{\textbf{Homework 6}}
\date{\today}

%% Spacing %%%%%%%%%%%%%%%%%%%%%%%%%%%%%%%%%%%%%%%%%%%%%%%%

\usepackage{fullpage}
\usepackage{setspace}
%\onehalfspacing
\usepackage{microtype}


%% Header %%%%%%%%%%%%%%%%%%%%%%%%%%%%%%%%%%%%%%%%%%%%%%%%%

%\pagestyle{fancy} 
%\lhead{}
%\rhead{}
%\chead{}
%\setlength{\headheight}{15.2pt} 
    %---Make the header bigger to avoid overlap

%\renewcommand{\headrulewidth}{0.3pt} 
    %---Width of the line

%\setlength{\headsep}{0.2in}    
    %---Distance from line to text
            

%% Mathematics Related %%%%%%%%%%%%%%%%%%%%%%%%%%%%%%%%%%%

\usepackage{amsmath}
\usepackage{amsfonts}
\usepackage{mathrsfs}
\usepackage{amsthm} %allows for labeling of theorems
\theoremstyle{plain}
\newtheorem{thm}{Theorem}[section]
\newtheorem{lem}[thm]{Lemma}
\newtheorem{prop}[thm]{Proposition}
\newtheorem{cor}[thm]{Corollary}

\theoremstyle{definition}
\newtheorem{defn}[thm]{Definition}
\newtheorem{ex}[thm]{Example}

\theoremstyle{remark}
\newtheorem*{rem}{Remark}
\newtheorem*{note}{Note}


%% Font Choices %%%%%%%%%%%%%%%%%%%%%%%%%%%%%%%%%%%%%%%%%

\usepackage[T1]{fontenc}
\usepackage[utf8]{inputenc}
\usepackage{lmodern}
%\usepackage{blindtext}


%% Figures %%%%%%%%%%%%%%%%%%%%%%%%%%%%%%%%%%%%%%%%%%%%%%

\usepackage{graphicx}
\usepackage{subfigure} 
    %---For plotting multiple figures at once
%\graphicspath{ {Directory/} }
    %---Set a directory for where to look for figures


%% Hyperlinks %%%%%%%%%%%%%%%%%%%%%%%%%%%%%%%%%%%%%%%%%%%%
\usepackage{hyperref} 
\hypersetup{	
    colorlinks,		
        %---This colors the links themselves, not boxes
    citecolor=black,	
        %---Everything here and below changes link colors
    filecolor=black,
    linkcolor=black,
    urlcolor=black
}

%% Including Code %%%%%%%%%%%%%%%%%%%%%%%%%%%%%%%%%%%%%%% 

\usepackage{verbatim} 
    %---For including verbatim code from files, no colors

\usepackage{listings}
\usepackage{color}
\definecolor{mygreen}{RGB}{28,172,0}
\definecolor{mylilas}{RGB}{170,55,241}
\newcommand{\matlabcode}[1]{%
    \lstset{language=Matlab,%
        basicstyle=\footnotesize,%
        breaklines=true,%
        morekeywords={matlab2tikz},%
        keywordstyle=\color{blue},%
        morekeywords=[2]{1}, keywordstyle=[2]{\color{black}},%
        identifierstyle=\color{black},%
        stringstyle=\color{mylilas},%
        commentstyle=\color{mygreen},%
        showstringspaces=false,%
            %---Without this there will be a symbol in 
            %---the places where there is a space
        numbers=left,%
        numberstyle={\tiny \color{black}},% 
            %---Size of the numbers
        numbersep=9pt,% 
            %---Defines how far the numbers are from the text
        emph=[1]{for,end,break,switch,case},emphstyle=[1]\color{red},%
            %---Some words to emphasise
    }%
    \lstinputlisting{#1}
}
    %---For including Matlab code from .m file with colors,
    %---line numbering, etc. 


%% Misc %%%%%%%%%%%%%%%%%%%%%%%%%%%%%%%%%%%%%%%%%%%%%% 

\usepackage{enumitem} 
    %---Has to do with enumeration	
\usepackage{appendix}
%\usepackage{natbib} 
    %---For bibliographies
\usepackage{pdfpages}
    %---For including whole pdf pages as a page in doc


%% User Defined %%%%%%%%%%%%%%%%%%%%%%%%%%%%%%%%%%%%%%%%%% 

%\newcommand{\nameofcmd}{Text to display}
\newcommand*{\Chi}{\mbox{\large$\chi$}} %big chi



%%%%%%%%%%%%%%%%%%%%%%%%%%%%%%%%%%%%%%%%%%%%%%%%%%%%%%%%%%%%%%%%%%%%%%%% 
%% BODY %%%%%%%%%%%%%%%%%%%%%%%%%%%%%%%%%%%%%%%%%%%%%%%%%%%%%%%%%%%%%%%%
%%%%%%%%%%%%%%%%%%%%%%%%%%%%%%%%%%%%%%%%%%%%%%%%%%%%%%%%%%%%%%%%%%%%%%%% 


\begin{document}

\maketitle 

\begin{enumerate} 

% Question 1
\item We want to prove that if $N$ open intervals in $\mathbb{R}$ cover $[a,b]$, denoted $I_1, \ldots,I_N$, then $\sum^N_{n=1} \ell(I_n)\geq b-a$.

So without loss of generality, let $\{I_n\}_1^N$ be any such collection, where $I_n=(a_n, b_n)$ and $a_1\leq a_2\leq\cdots\leq a_N$.\footnote{If they are not ordered in such a way, just reorder and relabel so that they are.} Then, rearrange the $b_n$ and define 
\[
    b_1^* \leq \cdots \leq b_N^* \qquad
    I_n^* := (a_n, b^*_n)
\]
We now want to show two main points to prove the desired result.
\begin{enumerate}

\item First, we want to show that
\begin{equation}
    \label{q1.eq}
    \sum^N_{n=1} \ell(I_n^*) = \sum^N_{n=1} \ell(I_n) 
\end{equation}
To do so, use the definition of the length of the interval, and also make use of the fact that we're working with finite sums (so splitting things up works nicely):
\begin{align*}
    \sum^N_{n=1} \ell(I_n^*) &= \sum^N_{n=1} (b_n^* - a_n )
    = \sum^N_{n=1} b_n^* -  \sum^N_{n=1}a_n 
\end{align*}
But we can reorder the terms in the first sum back to 
\begin{align*}
    \sum^N_{n=1} \ell(I_n^*) &= 
        \sum^N_{n=1} b_n -  \sum^N_{n=1}a_n 
    = \sum^N_{n=1} (b_n -  a_n) \\
    &= \sum^N_{n=1} \ell(I_n) 
\end{align*}

\item Next, we want to show that $\{I_n^*\}^N_1$ also covers $[a,b]$. To do so, we suppose that there exists an $x\in[a,b]$ that is not in $I^*_n$ for all $n$, and then derive a contradiction.
    
Begin by noting that we will have $a_n \leq b^*_n$ for all $n$. To see this, consider
\begin{enumerate}
    \item If $N=1$, this is clearly true since $b_1^*=b_1>a_1$.
    \item Next, suppose the statement is true for all $n=1,\ldots,N-1$. Then $a_{N}\leq b^*_{N}$ as well, else 
\[ 
    a_{N}>b^*_{N} \quad \Rightarrow \quad 
    a_{N}>b^*_{N} \geq b^*_{m} \qquad \text{for all $m=1, \ldots, N$}
\]
since we ordered the $b^*_n$ so that there were monotonically increasing. But this is a clear contradiction, since we have $b_{N} > a_{N}$ in $I_{N}$.
\end{enumerate}
So by induction, we have $a_n\leq b_n^*$ for all $n=1,\ldots,N$.
\\
\\
But this statement we just proved will give us our contradiction. To see that, note that if $x\in[a,b]$, then $x\in I_n$ for some $n$ since $\{I_n\}$ is an open cover of $[a,b]$. 

So putting together our assumptions so far,
\[
    x\in I_n \supset I^*_n \quad \text{but}\quad
    x\not\in I^*_n
\]
where the fact that $I^*_n\subset I_n$ follows from the fact that $x\not\in I^*_m$ for any $m$.

Now we can say that to have $x\not\in I_n^*$, it must be that $b^*_n < b_n$, which means there exists an index $m$ such that $b_m < b_n$, where $m < n$. 

\end{enumerate}

Okay, now we're ready to prove the final result.



\end{enumerate}

\begin{enumerate}
% Question 3
\item ($I_y\subseteq I_x$) To show this, note that we have
\[
    I_y := (a_y, b_y) \quad \Rightarrow \quad
    I_y\subseteq (a_y, b_y) \cup (a_x, b_x) 
    = (\min\{a_x,a_y\}, \max\{b_x, b_y\})
\]
The fact that the union above results in an interval stems from the fact that $y$ is in both $I_y$ and $I_x$; therefore, the intervals must overlap. Finally, this final interval is clearly an interval containing $x$. And since $I_x$ is the \emph{largest} interval containing $x$, it follows that $I_x\supset I_y$.
\\
\\
Now that we have $I_y = I_x$ for all $y\in I_x$, we can dramatically reduce in size the number of open intervals needed to cover $U$. 
\\
\\
Next, we consider the sequence the 
\[
    U\supset \bigcup_{x\in U} I_x 
    =\bigcup_{x\in U} (a_x, b_x)
\]
Now we saw above that if $y\in I_x$ then $I_y = I_x$, so suppose we have two points $x\neq y$ such that $I_y\neq I_x$. Then it must be that
\[
    I_y \cap I_x = \emptyset \quad \Rightarrow
    \quad b_y > a_y \geq b_x > a_x
\]
This now allows us to order the intervals, based on the unique values of the lefthand side of each interval. So for each interval $I_x$, let $I_y$ denote the next interval in the ordering and choose
\[
    q_x \in [a_x, a_y] \qquad q_x \in \mathbb{Q}
\]
which we can do since the rationals are dense in the reals. For the final set (if the set of intervals is finite), any rational larger than the last will do.
\\
\\
So we now have a one-to-one correspondence between our sets and the rational numbers, implying that the set $\{I_x\}_{x\in U}$ reduces to a countable set of disjoint intervals.


\item We have a finite sequence of disjoint intervals, $\{I_n\}_1^N$. We can use the result from class that the outer measure value $m^*(I_n)$ is independent of whether of whether $I_n$ is open, closed, or half-open.
\\
\\
So first, denote the endpoints of $I_n$ as $a_n$ on the left and $b_n$ on the right. With that, for an positive $\varepsilon$, we know that we can cover $I_n$ by open sets
\[
    I_n \subset (a_n-\varepsilon, b_n+\varepsilon)
    \quad \Rightarrow\quad
    \bigcup^\infty_{n=1} I_n \subset 
    \bigcup^\infty_{n=1}(a_n-\varepsilon, b_n+\varepsilon)
\]
Thus, if we return to the definition of the outer measure, we see that 
\begin{align*}
    m^*\left(\bigcup^N_{n=1} I_n\right) &\leq 
    \sum^N_{n=1} \ell\left((a_n-\varepsilon, b_n+\varepsilon)
    \right) \\
    &= \sum^N_{n=1} b_n - a_n + 2\varepsilon 
    \quad \text{for all $\varepsilon>0$}\\
    \Rightarrow\quad
    m^*\left(\bigcup^N_{n=1} I_n\right) &\leq 
    \sum^N_{n=1} \ell(I_n) 
\end{align*}
where the last line follows by taking the limit as $\varepsilon\rightarrow 0$.

Next, for any open cover of the interval $I_n$, denoted $\{J_{n,m}\}_{m=1}^\infty$, we know by compactness that there exists a finite set of open intervals, $\{J_{n,1}, J_{n,2}, \ldots,J_{n,M_n}\}$ such that 
\[
    \bigcup^{M_n}_{m=1} J_{n,m} \supset I_n
\]
And from Question 1, we know that 
\begin{equation}
    \sum^{M_n}_{m=1} \ell(J_{n,m}) \geq b_n - a_n
\end{equation}
Proceeding in this way for all $n$, we see that 
\begin{equation}
    \sum^N_{n=1} 
    \sum^{\infty}_{m=1} \ell(J_{n,m}) \geq 
    \sum^N_{n=1} 
    \sum^{M_n}_{m=1} \ell(J_{n,m})
    \geq \sum^N_{n=1} (b_n - a_n)
    =\sum^N_{n=1} \ell\left(I_n\right)
\end{equation}
Taking the infimum over all open covers (which affects the lefthand side only), we see that 
\begin{align*}
    m^*\left(\bigcup^N_{n=1} I_n\right) &\geq 
    \sum^N_{n=1} \ell(I_n) 
\end{align*}
And so we have equality.






\end{enumerate}

\end{document}



%%%% INCLUDING FIGURES %%%%%%%%%%%%%%%%%%%%%%%%%%%%

   % H indicates here 
   %\begin{figure}[h!]
  %   \centering
   %   \includegraphics[scale=1]{file.pdf}
   %\end{figure}

%   \begin{figure}[h!]
%      \centering
%      \mbox{
%	 \subfigure{
%	    \includegraphics[scale=1]{file1.pdf}
%	 }\quad
%	 \subfigure{
%	    \includegraphics[scale=1]{file2.pdf} 
%	 }
%      }
%   \end{figure}
 

%%%%% Including Code %%%%%%%%%%%%%%%%%%%%%5
% \verbatiminput{file.ext}    % Includes verbatim text from the file
% \texttt{text}	  % includes text in courier, or code-like, font
