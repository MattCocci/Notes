\documentclass[12pt]{article}

\author{Matthew Cocci}
\title{\textbf{Homework 6}}
\date{\today}

%% Spacing %%%%%%%%%%%%%%%%%%%%%%%%%%%%%%%%%%%%%%%%%%%%%%%%

\usepackage{fullpage}
\usepackage{setspace}
%\onehalfspacing
\usepackage{microtype}


%% Header %%%%%%%%%%%%%%%%%%%%%%%%%%%%%%%%%%%%%%%%%%%%%%%%%

%\pagestyle{fancy} 
%\lhead{}
%\rhead{}
%\chead{}
%\setlength{\headheight}{15.2pt} 
    %---Make the header bigger to avoid overlap

%\renewcommand{\headrulewidth}{0.3pt} 
    %---Width of the line

%\setlength{\headsep}{0.2in}    
    %---Distance from line to text
            

%% Mathematics Related %%%%%%%%%%%%%%%%%%%%%%%%%%%%%%%%%%%

\usepackage{amsmath}
\usepackage{amsfonts}
\usepackage{mathrsfs}
\usepackage{amsthm} %allows for labeling of theorems
\theoremstyle{plain}
\newtheorem{thm}{Theorem}[section]
\newtheorem{lem}[thm]{Lemma}
\newtheorem{prop}[thm]{Proposition}
\newtheorem{cor}[thm]{Corollary}

\theoremstyle{definition}
\newtheorem{defn}[thm]{Definition}
\newtheorem{ex}[thm]{Example}

\theoremstyle{remark}
\newtheorem*{rem}{Remark}
\newtheorem*{note}{Note}


%% Font Choices %%%%%%%%%%%%%%%%%%%%%%%%%%%%%%%%%%%%%%%%%

\usepackage[T1]{fontenc}
\usepackage[utf8]{inputenc}
\usepackage{lmodern}
%\usepackage{blindtext}


%% Figures %%%%%%%%%%%%%%%%%%%%%%%%%%%%%%%%%%%%%%%%%%%%%%

\usepackage{graphicx}
\usepackage{subfigure} 
    %---For plotting multiple figures at once
%\graphicspath{ {Directory/} }
    %---Set a directory for where to look for figures


%% Hyperlinks %%%%%%%%%%%%%%%%%%%%%%%%%%%%%%%%%%%%%%%%%%%%
\usepackage{hyperref} 
\hypersetup{	
    colorlinks,		
        %---This colors the links themselves, not boxes
    citecolor=black,	
        %---Everything here and below changes link colors
    filecolor=black,
    linkcolor=black,
    urlcolor=black
}

%% Including Code %%%%%%%%%%%%%%%%%%%%%%%%%%%%%%%%%%%%%%% 

\usepackage{verbatim} 
    %---For including verbatim code from files, no colors

\usepackage{listings}
\usepackage{color}
\definecolor{mygreen}{RGB}{28,172,0}
\definecolor{mylilas}{RGB}{170,55,241}
\newcommand{\matlabcode}[1]{%
    \lstset{language=Matlab,%
        basicstyle=\footnotesize,%
        breaklines=true,%
        morekeywords={matlab2tikz},%
        keywordstyle=\color{blue},%
        morekeywords=[2]{1}, keywordstyle=[2]{\color{black}},%
        identifierstyle=\color{black},%
        stringstyle=\color{mylilas},%
        commentstyle=\color{mygreen},%
        showstringspaces=false,%
            %---Without this there will be a symbol in 
            %---the places where there is a space
        numbers=left,%
        numberstyle={\tiny \color{black}},% 
            %---Size of the numbers
        numbersep=9pt,% 
            %---Defines how far the numbers are from the text
        emph=[1]{for,end,break,switch,case},emphstyle=[1]\color{red},%
            %---Some words to emphasise
    }%
    \lstinputlisting{#1}
}
    %---For including Matlab code from .m file with colors,
    %---line numbering, etc. 


%% Misc %%%%%%%%%%%%%%%%%%%%%%%%%%%%%%%%%%%%%%%%%%%%%% 

\usepackage{enumitem} 
    %---Has to do with enumeration	
\usepackage{appendix}
%\usepackage{natbib} 
    %---For bibliographies
\usepackage{pdfpages}
    %---For including whole pdf pages as a page in doc


%% User Defined %%%%%%%%%%%%%%%%%%%%%%%%%%%%%%%%%%%%%%%%%% 

%\newcommand{\nameofcmd}{Text to display}
\newcommand*{\Chi}{\mbox{\large$\chi$}} %big chi



%%%%%%%%%%%%%%%%%%%%%%%%%%%%%%%%%%%%%%%%%%%%%%%%%%%%%%%%%%%%%%%%%%%%%%%% 
%% BODY %%%%%%%%%%%%%%%%%%%%%%%%%%%%%%%%%%%%%%%%%%%%%%%%%%%%%%%%%%%%%%%%
%%%%%%%%%%%%%%%%%%%%%%%%%%%%%%%%%%%%%%%%%%%%%%%%%%%%%%%%%%%%%%%%%%%%%%%% 


\begin{document}

\maketitle 

\begin{enumerate} 

% Question 1
\item We want to prove that if $N$ open intervals in $\mathbb{R}$ cover $[a,b]$, denoted $I_1, \ldots,I_N$, then $\sum^N_{n=1} \ell(I_n)\geq b-a$. 

So without loss of generality, let $\{I_n\}_1^N$ be any such collection, where $I_n=(a_n, b_n)$ and $a_1\leq a_2\leq\cdots\leq a_N$.\footnote{If they are not ordered in such a way, just reorder and relabel so that they are.} We will assume that $b_n>a_n$, otherwise $(a_n,b_n)=\emptyset$, in which case that interval is irrelevant to covering $[a,b]$. 
\\
\\
Now, rearrange the $b_n$ and define 
\[
    b_1^* \leq \cdots \leq b_N^* \qquad
    I_n^* := (a_n, b^*_n)
\]
We now want to show two main points to prove the desired result.
\begin{enumerate}

\item First, we want to show that
\begin{equation}
    \label{q1.eq}
    \sum^N_{n=1} \ell(I_n^*) = \sum^N_{n=1} \ell(I_n) 
\end{equation}
To do so, use the definition of the length of the interval, and also make use of the fact that we're working with finite sums (so splitting things up works nicely):
\begin{align*}
    \sum^N_{n=1} \ell(I_n^*) &= \sum^N_{n=1} (b_n^* - a_n )
    = \sum^N_{n=1} b_n^* -  \sum^N_{n=1}a_n 
\end{align*}
But we can reorder the terms in the first sum, which won't change its value, and get back to 
\begin{align*}
    \sum^N_{n=1} \ell(I_n^*) &= 
        \sum^N_{n=1} b_n -  \sum^N_{n=1}a_n 
    = \sum^N_{n=1} (b_n -  a_n) \\
    &= \sum^N_{n=1} \ell(I_n) 
\end{align*}

\item Next, we want to show that $\{I_n^*\}^N_1$ also covers $[a,b]$. To do so, suppose there is a point $x\in [a,b]$ that's not covered by $\{I_n^*\}^N_1$, which will lead to a contradiction. 
\\
\\
Now, because $\{I^*_n\}$ does not cover $x$, we must have 
\begin{equation}
    \label{q1.crux}
    \text{for all $a_n < x$} \qquad b_n^*\leq x
\end{equation}
Otherwise, there would be some open interval $(a_n, b_n^*)$ covering $x$.
\\
\\
Now suppose there are $K$ left endpoints such that $a_n < x$. Then, by the fact that $\{a_n\}$ is ordered, the set of such points must be 
\begin{equation}
\label{q1.a}
    A = \{a_1, \ldots, a_K\} = \{ a_n \; | \; a_n < x\}
\end{equation}
Next, by Statement \ref{q1.crux}, we must have $K$ intervals where $b_n^*\leq x$. But because the set $\{b_n^*\}$ is just a reordering of $\{b_n\}$, the set of right endpoints, we can define the following
\begin{equation}
\label{q1.b}
    B = \{b_n \; | \; b_n \leq x\}
\end{equation}
and say that it must be of size $K$ too.
\\
\\
But this will give us our contradiction. To see this, note that 
\begin{equation}
    \label{q1.ind}
    b_n \leq x \quad \Rightarrow \quad n \leq K
\end{equation}
This is because if $n>K$, we know from set $A$ that $a_n\geq x$. So $b_n\leq x$ couldn't be paired with $a_n \geq x$ to form a non-empty or ill-formed interval.
\\
\\
And so we've split up the original collection of intervals, $\{I_n\}$, into two classes:
\begin{align*}
    S_1 &= \{I_n \; | \; n \leq K \} = \{ I_n \; | \; b_n \leq x \} 
        \quad \text{From \ref{q1.b} and \ref{q1.ind}} \\
    S_2 &= \{I_n \; | \; n > K \} = \{ I_n \; | \; a_n \geq x \} 
        \quad \text{From \ref{q1.a}}
\end{align*}
It's clear that neither $S_1$ nor $S_2$ can cover $x$; therefore, $\{I_n\}$ is not an open cover, contradicting our assumption.
\end{enumerate}

Now to the final proof. By part (a), we can show an equivalent statement:
\begin{equation}
    \label{q1.toshow}
    \sum_{n=1}^N I^*_n \geq b - a
\end{equation}
So suppose that we have a cover where $N=1$ of $[a,b]$. Then clearly $I_1 = I_1^*$ so that if $I_1$ covers $[a,b]$, then $I_1^*$ will too. It's then clear that 
\[
    I_1 = I_1^* = (a_1, b_1^*) = (a-\varepsilon_1, b+\varepsilon_2)
\]
where $\varepsilon_1, \varepsilon_2>0$, so that 
\[
    \ell(I_1^*) = b_1^* - a_1 = b-a + \varepsilon_2 + \varepsilon_1 \geq b -a
\]
Next, suppose that it holds when $N$ open intervals cover a closed interval. To prove it for the $N+1$ case, divide $[a,b]$ into 
\begin{equation}
    [a,b] = [a,c] \cup [c,b]
\end{equation}
where $c \in (a_{N+1}, b_N^*)$. We know such a $c$ exists because we know that $a_{n+1} < b_n^*$ for all $n=1,\ldots,N-1$. Otherwise, there would be a set $[b_n^*, a_{n+1}]\neq \emptyset$ not covered by $\{I_n^*\}_1^{N+1}$. We know $[b_n^*, a_{n+1}]$ wouldn't be covered by $\{I_n^*\}$ because there'd be no $a_m$ or $b_k$ in between $b_n^*$ and $a_{n+1}$ by the way we ordered the endpoints above. But this is a contradiction, because we know $\{I_n^*\}$ covers $[a,b]$.
\\
\\
As a result, the set $\{I^*_n\}^N_1$ covers $[a,c]$ and $I_{N+1}^*$ covers $[c,b]$. Since we assumed the $N$ case holds, we can write
\begin{align*}
    \sum_{n=1}^{N+1} \ell(I_n^*)  
    &= \ell(I_{N+1}^*) + \sum_{n=1}^{N} \ell(I_n^*) \\ 
    &\geq (b - c) + (c-a) = b-a
\end{align*}
The last line followed from the previous because, again, we assumed the $N$ case to hold (which got us $c-a$) and because we proved the case where $N=1$ above (which got us $b-c$). And so the statement holds for all $N$.


% Question 2
\item We want to show that every open set $U\subset\mathbb{R}$ can be written as the union of countably many disjoint open intervals, and that this decomposition is unique.

So start with $x\in U$. Since $U$ is open, we know there exists an $\varepsilon>0$ such that the interval
\begin{equation}
    (x-\varepsilon, x+\varepsilon) \subset U
\end{equation}
Then, we can define $I_x:=(a_x, b_x)$ to be the \emph{largest} interval such that $x\in I_x \subset U$.

We want to show that for all $y\in I_x$, we have $I_y = I_x$. To do so, we show each is a subset of the other.
\begin{enumerate}
    \item ($I_y\supseteq I_x$): First, note that $y\in I_x$ and $I_x$ is an open interval in $U$. But since $I_y$ is the \emph{largest} open interval containing $y$ such that $I_y\subset U$, we see that clearly $I_y\supseteq I_x$. 
    \item ($I_y\subseteq I_x$): To show this, note that we have
\[
    I_y := (a_y, b_y) \quad \Rightarrow \quad
    I_y\subseteq (a_y, b_y) \cup (a_x, b_x) 
    = (\min\{a_x,a_y\}, \max\{b_x, b_y\})
\]
The fact that the union above results in an interval stems from the fact that $y$ is in both $I_y$ and $I_x$; therefore, the intervals must overlap since they are connected. Finally, this interval $(\min\{a_x,a_y\}, \max\{b_x, b_y\})$ is clearly an interval containing $x$ in $U$. And since $I_x$ is the \emph{largest} interval containing $x$ such that $I_x\subset U$, it follows that $I_x\supset I_y$.
\end{enumerate}
Now that we have $I_y = I_x$ for all $y\in I_x$, we can dramatically reduce in size the number of open intervals needed to construct $U$. 
\\
\\
So consider the union
\[
    U = \bigcup_{x\in U} I_x 
    =\bigcup_{x\in U} (a_x, b_x) \qquad I_x \subset U
\]
Now we saw above that if $y\in I_x$ then $I_y = I_x$, so suppose we have two points $x\neq y$ such that $I_y\neq I_x$. Then it must be that
\[
    I_y \cap I_x = \emptyset \quad \Rightarrow
    \quad b_y > a_y \geq b_x > a_x
\]
This now allows us to order the disjoint intervals, based on the unique values of the left endpoint of each interval. So for each unique interval $I_x$, let $I_y$ denote the next unique, disjoint interval in the ordering and choose
\[
    q_x \in [a_x, a_y] \qquad q_x \in \mathbb{Q}
\]
which we can do since the rationals are dense in the reals and $a_x < a_y$. For the final set (if the set of intervals is finite), any rational larger than the previous rational will do.
\\
\\
So we now have a one-to-one correspondence between our disjoint intervals and the rational numbers, implying that the set covering $U$, written $\{I_x\}_{x\in U}$, reduces to a countable set of disjoint intervals. 
\\
\\
To show uniqueness, suppose by contradiction that there is another such countable, pairwise-disjoint union of open intervals, $\{J_n\}$ with
\[
    U = \bigcup^\infty_{n=1} I_n = 
    \bigcup^\infty_{n=1} J_n \qquad \{I_n\} \neq \{J_n\}
\]
Then, suppose there is an $I \in \{I_n\}$ such that $I \not\in \{J_n\}$. But because $I \subset \cup I_n = U$, we know that that
\begin{align*}
    I \subset U = 
    \bigcup^\infty_{n=1} I_n
    = \bigcup^\infty_{n=1} J_n
\end{align*}
Thus, there must be a $J \in \{J_n\}$ where 
\[
    J \cap I \neq \emptyset
\]
Since we're assuming by contradiction that $J$ and $I$ are distinct, we can say, without loss of generality, that there's an endpoint of $J$, call it $a_j$, in $I$.\footnote{For $J$ and $I$ to have a non-empty intersection, each must have an endpoint in the other, and we could handle the right endpoint just as easily as the left with minor modifications.} So now, because $a_{j}\in I$, it is in $U = \cup I_n = \cup J_n $; therefore, there must be \emph{another} interval $J' \in \{J_n\}$ containing $a_{j}$. 
\\
\\
But this gives us a contradiction because $a_j$ being an endpoint of $J$ contained within $J'$ implies
\[
    J \cap J' \neq \emptyset
\]
since the right endpoint of $J'$ will have to be strictly larger than the left endpoint of $J$ to have $a_j \in J'$. But this contradicts the assumption of $\{J_n\}$ being pairwise disjoint and distinct.





% Question 3
\item We have a finite sequence of disjoint intervals, $\{I_n\}_1^N$. We can use the result from class that the outer measure value $m^*(I_n)$ is independent of whether $I_n$ is open, closed, or half-open.
\\
\\
So first, denote the endpoints of $I_n$ as $a_n$ on the left and $b_n$ on the right. With that, for any positive $\varepsilon$, we know that we can cover $I_n$ by open sets
\[
    I_n \subset (a_n-\varepsilon, b_n+\varepsilon)
    \quad \Rightarrow\quad
    \bigcup^\infty_{n=1} I_n \subset 
    \bigcup^\infty_{n=1}(a_n-\varepsilon, b_n+\varepsilon)
\]
Thus, if we return to the definition of the outer measure, we see that 
\begin{align*}
    m^*\left(\bigcup^N_{n=1} I_n\right) &\leq 
    \sum^N_{n=1} \ell\left((a_n-\varepsilon, b_n+\varepsilon)
    \right) \\
    &= \sum^N_{n=1} b_n - a_n + 2\varepsilon 
    \quad \text{for all $\varepsilon>0$}\\
    \Rightarrow\quad
    m^*\left(\bigcup^N_{n=1} I_n\right) &\leq 
    \sum^N_{n=1} \ell(I_n) 
\end{align*}
where the last line follows by taking the limit as $\varepsilon\rightarrow 0$.

Next, for any open cover of the interval $I_n$ consisting of intervals, denoted $\{J_{n,m}\}_{m=1}^\infty$, we know by compactness that there exists a finite set of open intervals, $\{J_{n,1}, J_{n,2}, \ldots,J_{n,M_n}\}$ such that 
\[
    \bigcup^{M_n}_{m=1} J_{n,m} \supset I_n
\]
And from Question 1, we know that 
\begin{equation}
    \sum^{M_n}_{m=1} \ell(J_{n,m}) \geq b_n - a_n
\end{equation}
Proceeding in this way for all $n$, we see that 
\begin{equation}
    \sum^N_{n=1} 
    \sum^{\infty}_{m=1} \ell(J_{n,m}) \geq 
    \sum^N_{n=1} 
    \sum^{M_n}_{m=1} \ell(J_{n,m})
    \geq \sum^N_{n=1} (b_n - a_n)
    =\sum^N_{n=1} \ell\left(I_n\right)
\end{equation}
Taking the infimum over all open covers consisting of intervals (which affects the lefthand side only), we see that 
\begin{align*}
    m^*\left(\bigcup^N_{n=1} I_n\right) &\geq 
    \sum^N_{n=1} \ell(I_n) 
\end{align*}
And so, combined with the earlier result, we have equality.






\end{enumerate}

\end{document}



%%%% INCLUDING FIGURES %%%%%%%%%%%%%%%%%%%%%%%%%%%%

   % H indicates here 
   %\begin{figure}[h!]
  %   \centering
   %   \includegraphics[scale=1]{file.pdf}
   %\end{figure}

%   \begin{figure}[h!]
%      \centering
%      \mbox{
%	 \subfigure{
%	    \includegraphics[scale=1]{file1.pdf}
%	 }\quad
%	 \subfigure{
%	    \includegraphics[scale=1]{file2.pdf} 
%	 }
%      }
%   \end{figure}
 

%%%%% Including Code %%%%%%%%%%%%%%%%%%%%%5
% \verbatiminput{file.ext}    % Includes verbatim text from the file
% \texttt{text}	  % includes text in courier, or code-like, font
