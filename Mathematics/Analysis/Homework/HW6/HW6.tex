\documentclass[12pt]{article}

\author{Matthew Cocci}
\title{\textbf{Homework 6}}
\date{\today}

%% Spacing %%%%%%%%%%%%%%%%%%%%%%%%%%%%%%%%%%%%%%%%%%%%%%%%

\usepackage{fullpage}
\usepackage{setspace}
%\onehalfspacing
\usepackage{microtype}


%% Header %%%%%%%%%%%%%%%%%%%%%%%%%%%%%%%%%%%%%%%%%%%%%%%%%

%\pagestyle{fancy} 
%\lhead{}
%\rhead{}
%\chead{}
%\setlength{\headheight}{15.2pt} 
    %---Make the header bigger to avoid overlap

%\renewcommand{\headrulewidth}{0.3pt} 
    %---Width of the line

%\setlength{\headsep}{0.2in}    
    %---Distance from line to text
            

%% Mathematics Related %%%%%%%%%%%%%%%%%%%%%%%%%%%%%%%%%%%

\usepackage{amsmath}
\usepackage{amsfonts}
\usepackage{mathrsfs}
\usepackage{amsthm} %allows for labeling of theorems
\theoremstyle{plain}
\newtheorem{thm}{Theorem}[section]
\newtheorem{lem}[thm]{Lemma}
\newtheorem{prop}[thm]{Proposition}
\newtheorem{cor}[thm]{Corollary}

\theoremstyle{definition}
\newtheorem{defn}[thm]{Definition}
\newtheorem{ex}[thm]{Example}

\theoremstyle{remark}
\newtheorem*{rem}{Remark}
\newtheorem*{note}{Note}


%% Font Choices %%%%%%%%%%%%%%%%%%%%%%%%%%%%%%%%%%%%%%%%%

\usepackage[T1]{fontenc}
\usepackage[utf8]{inputenc}
\usepackage{lmodern}
%\usepackage{blindtext}


%% Figures %%%%%%%%%%%%%%%%%%%%%%%%%%%%%%%%%%%%%%%%%%%%%%

\usepackage{graphicx}
\usepackage{subfigure} 
    %---For plotting multiple figures at once
%\graphicspath{ {Directory/} }
    %---Set a directory for where to look for figures


%% Hyperlinks %%%%%%%%%%%%%%%%%%%%%%%%%%%%%%%%%%%%%%%%%%%%
\usepackage{hyperref} 
\hypersetup{	
    colorlinks,		
        %---This colors the links themselves, not boxes
    citecolor=black,	
        %---Everything here and below changes link colors
    filecolor=black,
    linkcolor=black,
    urlcolor=black
}

%% Including Code %%%%%%%%%%%%%%%%%%%%%%%%%%%%%%%%%%%%%%% 

\usepackage{verbatim} 
    %---For including verbatim code from files, no colors

\usepackage{listings}
\usepackage{color}
\definecolor{mygreen}{RGB}{28,172,0}
\definecolor{mylilas}{RGB}{170,55,241}
\newcommand{\matlabcode}[1]{%
    \lstset{language=Matlab,%
        basicstyle=\footnotesize,%
        breaklines=true,%
        morekeywords={matlab2tikz},%
        keywordstyle=\color{blue},%
        morekeywords=[2]{1}, keywordstyle=[2]{\color{black}},%
        identifierstyle=\color{black},%
        stringstyle=\color{mylilas},%
        commentstyle=\color{mygreen},%
        showstringspaces=false,%
            %---Without this there will be a symbol in 
            %---the places where there is a space
        numbers=left,%
        numberstyle={\tiny \color{black}},% 
            %---Size of the numbers
        numbersep=9pt,% 
            %---Defines how far the numbers are from the text
        emph=[1]{for,end,break,switch,case},emphstyle=[1]\color{red},%
            %---Some words to emphasise
    }%
    \lstinputlisting{#1}
}
    %---For including Matlab code from .m file with colors,
    %---line numbering, etc. 


%% Misc %%%%%%%%%%%%%%%%%%%%%%%%%%%%%%%%%%%%%%%%%%%%%% 

\usepackage{enumitem} 
    %---Has to do with enumeration	
\usepackage{appendix}
%\usepackage{natbib} 
    %---For bibliographies
\usepackage{pdfpages}
    %---For including whole pdf pages as a page in doc


%% User Defined %%%%%%%%%%%%%%%%%%%%%%%%%%%%%%%%%%%%%%%%%% 

%\newcommand{\nameofcmd}{Text to display}
\newcommand*{\Chi}{\mbox{\large$\chi$}} %big chi



%%%%%%%%%%%%%%%%%%%%%%%%%%%%%%%%%%%%%%%%%%%%%%%%%%%%%%%%%%%%%%%%%%%%%%%% 
%% BODY %%%%%%%%%%%%%%%%%%%%%%%%%%%%%%%%%%%%%%%%%%%%%%%%%%%%%%%%%%%%%%%%
%%%%%%%%%%%%%%%%%%%%%%%%%%%%%%%%%%%%%%%%%%%%%%%%%%%%%%%%%%%%%%%%%%%%%%%% 


\begin{document}

\maketitle 

\begin{enumerate} 

% Question 1
\item We want to prove that if $N$ open intervals in $\mathbb{R}$ cover $[a,b]$, denoted $I_1, \ldots,I_N$, then $\sum^N_{n=1} \ell(I_n)\geq b-a$.

So without loss of generality, let $\{I_n\}_1^N$ be any such collection, where $I_n=(a_n, b_n)$ and $a_1\leq a_2\leq\cdots\leq a_N$.\footnote{If they are not ordered in such a way, just reorder and relabel so that they are.} Then, rearrange the $b_n$ and define 
\[
    b_1^* \leq \cdots \leq b_N^* \qquad
    I_n^* := (a_n, b^*_n)
\]
We now want to show two main points to prove the desired result.
\begin{enumerate}

\item First, we want to show that
\begin{equation}
    \label{q1.eq}
    \sum^N_{n=1} \ell(I_n^*) = \sum^N_{n=1} \ell(I_n) 
\end{equation}
To do so, use the definition of the length of the interval, and also make use of the fact that we're working with finite sums (so splitting things up works nicely):
\begin{align*}
    \sum^N_{n=1} \ell(I_n^*) &= \sum^N_{n=1} (b_n^* - a_n )
    = \sum^N_{n=1} b_n^* -  \sum^N_{n=1}a_n 
\end{align*}
But we can reorder the terms in the first sum, which won't change its value, and get back to 
\begin{align*}
    \sum^N_{n=1} \ell(I_n^*) &= 
        \sum^N_{n=1} b_n -  \sum^N_{n=1}a_n 
    = \sum^N_{n=1} (b_n -  a_n) \\
    &= \sum^N_{n=1} \ell(I_n) 
\end{align*}

\item Next, we want to show that $\{I_n^*\}^N_1$ also covers $[a,b]$. But first, assume without loss of generality that $I_n\cap [a,b] \neq \emptset$ for all $n$. Then, note that for $\{I_n\}$ to cover $[a,b]$, we must have 
\begin{equation}
    \label{q1.reason}
    a_{n+1} < b_n 
    \qquad \text{for all $n=1,\ldots,N-1$}
\end{equation}
Otherwise, since the $\{a_n\}$ are ordered and monotonically increasing, there would be an interval $[b_n, a_{n+1}]\neq\emptyset$ not covered by $\{I_n\}$, contradicting the assumption that it covers $[a,b]$.
\\
\\
Next, we will assume we have an $x\in[a,b]$ such that $x$ is not in any $I_n^*$ for any $n$, then draw a contradiction. So let $I_n$ be the interval containing $x$, which we know to exist since $\{I_n\}$ covers $[a,b]$ and $x\in[a,b]$. Then, since $x\not\in \{I_n^*\}$, it must be the case that
\[
    x \in I_n \setminus I_n^* \quad \Rightarrow \quad
    x \in [b_n^*, b_n)
\]
But since we just reordered to get the $\{b_n^*\}$, we know $b_n^*$ is just $b_m$ for some $m=1,\ldots,N$. Now note that we can't have $m=N$, because 

So by Statement \ref{q1.reason}, we can find an $a_{m+1}$ such that 

\end{enumerate}

Now to the final proof. By part (a), we can show an equivalent statment:
\begin{equation}
    \label{q1.toshow}
    \sum_{n=1}^N I^*_n \geq b - a
\end{equation}
So suppose that we have a cover where $N=1$ of $[a,b]$. 


% Question 2
\item We want to show that every open set $U\subset\mathbb{R}$ can be written as the union of countably many disjoint open intervals, and that this decomposition is unique.

So start with $x\in U$. Since $U$ is open, we know there exists an $\varepsilon>0$ such that the interval
\begin{equation}
    (x-\varepsilon, x+\varepsilon) \subset U
\end{equation}
Then, we can define $I_x:=(a_x, b_x)$ to be the \emph{largest} interval such that $x\in I_x \subset U$.

We want to show that for all $y\in I_x$, we have $I_y = I_x$. To do so, we show each is a subset of the other.
\begin{enumerate}
    \item ($I_y\supseteq I_x$): First, note that $y\in I_x$ and $I_x$ is an open interval. But since $I_y$ is the \emph{largest} open interval containing $y$, we see that clearly $I_y\supseteq I_x$. 
    \item ($I_y\subseteq I_x$): Next, note that since 
    
\end{enumerate}





\end{enumerate}

\end{document}



%%%% INCLUDING FIGURES %%%%%%%%%%%%%%%%%%%%%%%%%%%%

   % H indicates here 
   %\begin{figure}[h!]
  %   \centering
   %   \includegraphics[scale=1]{file.pdf}
   %\end{figure}

%   \begin{figure}[h!]
%      \centering
%      \mbox{
%	 \subfigure{
%	    \includegraphics[scale=1]{file1.pdf}
%	 }\quad
%	 \subfigure{
%	    \includegraphics[scale=1]{file2.pdf} 
%	 }
%      }
%   \end{figure}
 

%%%%% Including Code %%%%%%%%%%%%%%%%%%%%%5
% \verbatiminput{file.ext}    % Includes verbatim text from the file
% \texttt{text}	  % includes text in courier, or code-like, font
