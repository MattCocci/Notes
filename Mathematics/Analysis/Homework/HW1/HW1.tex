\documentclass[12pt]{article}

\author{Matthew Cocci}
\title{\textbf{Homework 1}}
\date{\today}
\usepackage{fullpage}
\usepackage{enumitem} %Has to do with enumeration	
\usepackage{amsfonts}
\usepackage{amsmath}
\usepackage[T1]{fontenc}
\usepackage[utf8]{inputenc}
%\usepackage{blindtext}
\usepackage{graphicx}
\usepackage{hyperref} 
\hypersetup{	
    colorlinks,		% This colors the links themselves, not boxes
    citecolor=black,	% Everything below changes the link colors
    filecolor=black,
    linkcolor=black,
    urlcolor=black
}
\usepackage{mathrsfs} %allows for labeling of theorems
\usepackage{amsthm} %allows for labeling of theorems
\theoremstyle{plain}
\newtheorem{thm}{Theorem}[section]
\newtheorem{lem}[thm]{Lemma}
\newtheorem{prop}[thm]{Proposition}
\newtheorem{cor}[thm]{Corollary}

\theoremstyle{definition}
\newtheorem{defn}[thm]{Definition}
\newtheorem{ex}[thm]{Example}

\theoremstyle{remark}
\newtheorem*{rem}{Remark}
\newtheorem*{note}{Note}
\usepackage{appendix}
\usepackage{subfigure} % For plotting multiple figures at once
\usepackage{verbatim} % for including verbatim code from a file
\usepackage{natbib} % for bibliographies

\begin{document}

\maketitle

%\tableofcontents %adds it here

\begin{enumerate}

\item 
    \begin{enumerate}
        \item We want to show that, for all $P$, 
        \begin{align*}
            U_\alpha(f, P) - L_\alpha(f,P) &\leq 
                w_f\left(||P||\right) (\alpha(b) - \alpha(a)) 
        \end{align*}
        By the definition of the Upper and Lower Darboux sums,
        \begin{align}
            U_\alpha(f, P) - L_\alpha(f,P) &= 
            \sum^n_{i=1} M_i(f)\Delta\alpha_i - 
            \sum^n_{i=1} m_i(f)\Delta\alpha_i \notag \\
            &= \sum^n_{i=1} \left[M_i(f) - m_i(f)\right]
            \Delta\alpha_i \label{q1a.1}
        \end{align}
        Now let $\overline{x}_i$ and $\underline{x}_i$ be the 
        points such that 
        \begin{align*}
            f(\underline{x}_i) &= m_i(f) 
                = \inf_{x\in [x_i, x_{i-1}]} f(x)\\
            f(\overline{x}_i) &= M_i(f)
                = \sup_{x\in [x_i, x_{i-1}]} f(x)
        \end{align*}
        Now since $\underline{x}_i$ and $\overline{x}_i$ are
        both in $[x_{i-1}, x_i]$, we can combine this with
        the definition of $||P||$ to get
            \[ |\underline{x}_i - \overline{x}_i| \leq
                |x_i - x_{i-1}| \leq ||P|| \]
        By the definition of the modulus of continuity, 
        \begin{align*}
            |\underline{x}_i - \overline{x}_i| \leq
                ||P|| \quad \Rightarrow \quad
                &|f(\underline{x}_i) - f(\overline{x}_i)| 
                \leq w_f\left(||P||\right)  \\
            \Leftrightarrow \quad &|M_i(f) - m_i(f)| 
                \leq w_f\left(||P||\right)  
        \end{align*}
        Substituting this fact back into \ref{q1a.1}, and 
        using the properties of telescoping sums, we get that
        \begin{align*}
            \sum^n_{i=1} \left[M_i(f) - m_i(f)\right]
            \Delta\alpha_i 
            &\leq  \sum^n_{i=1} w_f\left(||P||\right) 
            \Delta\alpha_i = 
            w_f\left(||P||\right) \sum^n_{i=1} \Delta\alpha_i\\
            &\leq w_f\left(||P||\right) (\alpha(b)-\alpha(a)) 
        \end{align*}
                

    \item We know $f$ is monotone on $[a,b]$. Suppose that 
        $f$ is monotone increasing.\footnote{
        Note that I'll lose no generality in assuming
        that the function $f$ was monotone increasing.  If it's
        monotone decreasing instead, swap the values of $M_i$
        and $m_i$ in Equation \ref{q1b.1}, and proceed in 
        exactly the same way.}Then given any partition,
        $P$,
        \[ f(x_{i-1}) \leq f(c) \leq f(x_i) \quad \forall i, 
            c \in [x_{i-1}, x_i]\]
        This implies that 
        \begin{equation}
            M_i = f(x_i) \quad m_i=f(x_{i-1})\qquad\forall i
            \label{q1b.1}
        \end{equation}
        Writing $U_\alpha - L_\alpha$ as above in Equation 
        \ref{q1a.1}, we see that the expression reduces to
        \begin{equation}
            \label{q1b.2}
            \sum^n_{i=1} \left[M_i(f) - m_i(f)\right]
            \Delta\alpha_i = 
            \sum^n_{i=1} \left[f(x_{i}) - f(x_{i-1})\right]
            \Delta\alpha_i 
        \end{equation}
        Now let's consider $w_\alpha\left(||P||\right)$.
        By definition, it is
            \[ w_\alpha\left(||P||\right) := 
                \sup_{|x-y|\leq ||P||} |\alpha(x)-\alpha(y)|
                \]
        It's clear that for each sub-interval, $[x_{i-1}, x_i]$,
        we have
            \[ |x_{i}-x_{i-1}| \leq ||P|| \quad \forall i\]
        This implies that
            \[ |\Delta \alpha_i| = 
                |\alpha(x_i) - \alpha(x_{i-1})| \leq
                w_\alpha\left(||P||\right) \qquad \forall i\]
        Subbing this into the right-hand side of equation 
        \ref{q1b.2}, we can simplify using the properties 
        of telescoping sums to get
        \begin{align}
            \sum^n_{i=1} \left[f(x_{i}) - f(x_{i-1})\right]
            \Delta\alpha_i  &\leq 
            \sum^n_{i=1} \left[f(x_{i}) - f(x_{i-1})\right]
             w_\alpha\left(||P||\right) \notag\\
            &\leq w_\alpha\left(||P||\right) 
            \sum^n_{i=1} \left[f(x_{i}) - f(x_{i-1})\right]
             \notag \\
            \Leftrightarrow U_\alpha(f,P) - L_\alpha(f,P) 
            &\leq w_\alpha\left(||P||\right) [f(b) -f(a)]
            \label{q1b.3}
        \end{align}
        \\
        Finally, since $\alpha$ is assumed continuous on the
        compact interval $[a,b]$, $\alpha$ is uniformly 
        continuous on $[a,b]$.  That means
        \[ \forall \epsilon > 0 \quad \exists \delta >0 
            \quad \text{s.t.} \quad |x-y| \leq \delta \quad
            \Rightarrow \quad
            |\alpha(x) -\alpha(y)| \leq \epsilon \]
        So if we can make $w_\alpha\left(||P||\right) < \epsilon$,
        by choosing our partition $P$ such that 
        $||P|| \leq \delta$.
        With Equation \ref{q1b.3}, we therefore ensure that
        \begin{equation}
             U_\alpha(f,P) - L_\alpha(f,P) 
            \leq \epsilon [f(b) -f(a)]
        \end{equation}
        for any arbitrary $\epsilon$.
        And so by Riemann's Condition, $f \in 
        \mathscr{R}_\alpha([a,b])$ because for any
        $\epsilon>0$, we can ensure the upper and lower sums 
        are within that distance of each other by choosing
        a sufficiently fine partition.

    \end{enumerate}

\item \textbf{Exercise 51.15}: First, let's establish some
    basic building blocks for the proof. 
    We have $\alpha$ increasing on
    $[a,b]$ and $f,g \in \mathscr{R}_\alpha([a,b])$. By
    Riemann's condition, it's clear that for all $\delta>0$,
    there exist partitions $P_1$ and $P_2$ such that 
    \begin{align}
        U_\alpha(f,P_1) - L_\alpha(f,P_1) < \delta 
            \label{q2p1} \\
        U_\alpha(g,P_2) - L_\alpha(g,P_2) < \delta 
            \label{q2p2} 
    \end{align}
    Now take the common refinement, $P^* = P_1 \cup P_2$.
    By Lemma 51.5 and Corollary 51.6 (in FoMA), we know
    that 
    \begin{align*}
         L_\alpha(f,P_1) \leq L_\alpha(f,P^*) \leq
            U_\alpha(f,P^*) \leq U_\alpha(f,P_1) \\
         L_\alpha(g,P_2) \leq L_\alpha(g,P^*) \leq
            U_\alpha(g,P^*) \leq U_\alpha(g,P_2) 
    \end{align*}
    Combining this with the result from Riemann's condition
    above (and rewriting as in Question 1), 
    we see that we must also have
    \begin{align*}
        \sum^n_{i=1} [M_i(f) - m_i(f)] \Delta\alpha_i =
        U_\alpha(f,P^*) - L_\alpha(f,P^*) \leq
        U_\alpha(f,P_1) - L_\alpha(f,P_1) < \delta \\
        \sum^n_{i=1} [M_i(g) - m_i(g)] \Delta\alpha_i =
        U_\alpha(g,P^*) - L_\alpha(g,P^*) \leq
        U_\alpha(g,P_2) - L_\alpha(g,P_2) < \delta 
    \end{align*}
    Now since $\alpha$ is assumed increasing, we know
    that $\Delta\alpha_i \geq 0$ for all $i$. Also,
    $M_i(\cdot) \geq m_i(\cdot)$ 
    for all $i$. Therefore, we can conclude
    \begin{align}
        0 \leq M_i(f) - m_i(f) < \delta \qquad
        \forall i \label{ems1} \\
        0 \leq M_i(g) - m_i(g) < \delta \qquad
        \forall i \label{ems2} 
    \end{align}
    Now, using all of this as groundwork, let's show the main
    result.
    \newpage
    \begin{enumerate}
        \item We'll start by showing $h(x) = \max \{f, g\} \in
            \mathscr{R}_\alpha([a,b])$ using 
            Riemann's Condition. So for all $\epsilon>0$, we need
            to find a partitition $P$ such that 
            \begin{equation}
                \label{toprove}
                U_\alpha(h,P) - L_\alpha(h,P) < \epsilon
            \end{equation}
            To do so, take $\delta = 
            \epsilon/[\alpha(b)-\alpha(a)]$ and use 
            Riemann's condition to find $P_1$ and $P_2$ 
            as above, in Equations \ref{q2p1} and \ref{q2p2}.
            Then take their common refinement to find $P^*$. 
            \textbf{This will be our partition $P$ such that 
            Equation \ref{toprove} holds}.
            \\
            \\
            To formally show this, consider any arbitrary 
            interval defined by the partition $P^*$. Over any 
            interval $[x_{i-1}, x_i]$ in $P^*$, 
            \begin{align*}
                M_i(h) = \sup_{x\in[x_{i-1}, x_i]} 
                    \max\{f(x), g(x)\} \\
                m_i(h) = \inf_{x\in[x_{i-1}, x_i]} 
                    \max\{f(x), g(x)\} 
            \end{align*}
            It's clear that we can narrow the list of 
            candidates for $M_i(h)$ and $m_i(h)$:
            \begin{align*}
                M_i(h) =  \max\{M_i(f), M_i(g)\} \\
                m_i(h) = \max\{m_i(f), m_i(g)\} 
            \end{align*}
            Let's consider the cases:
            \begin{enumerate}
                \item Suppose $M_i(h) = M_i(f)$ and
                    $m_i(h) = m_i(f)$. Then by Equation
                    \ref{ems1} and our choice of $\delta$,
                    we have
                    \[ 0  \leq \leq M_i(h) - m_i(h) =
                        M_i(f) - m_i(f)  =
                        \leq\frac{\epsilon}{\alpha(b)-\alpha(a)}
                        \]
                \item Similarly, if $M_i(h) = M_i(g)$ and
                    $m_i(h) = m_i(g)$. Then by Equation
                    \ref{ems2} and our choice of $\delta$,
                    we have
                    \[ 0 \leq
                        M_i(h) - m_i(h) =  M_i(g) - m_i(g) 
                        \leq\frac{\epsilon}{\alpha(b)-\alpha(a)}
                        \]
                \item Next, suppose that $M_i(h) = M_i(f)$ and
                    $m_i(h) = m_i(g)$. In that case, 
                        \[ m_i(g) > m_i(f) \quad
                           \Rightarrow \quad 
                           M_i(f) - m_i(g) < M_i(f) - m_i(f) \]
                    We also know that $M_i(f) - m_i(g)$ is 
                    bounded below by zero becuase if not, then
                    $m_i(g)>M_i(f)$, implying that  we didn't   
                    choose $M_i(h)$
                    correctly, as $M_i(g)$ would certainly
                    have been larger than $m_i(g)$ and, thus also
                    $M_i(f)$.
                    \\
                    \\
                    So by this fact, Equation \ref{ems1}, and our
                    choice of $\delta$:
                    \[ 0 \leq M_i(h) - m_i(h) 
                        =  M_i(f) - m_i(g)  \leq
                        M_i(f) - m_i(f) 
                        \leq\frac{\epsilon}{\alpha(b)-\alpha(a)}
                        \]
                \item Finally, suppose that $M_i(h) = M_i(g)$ and
                    $m_i(h) = m_i(f)$. In that case, 
                        \[ m_i(f) > m_i(g) \quad
                           \Rightarrow \quad 
                           M_i(g) - m_i(f) < M_i(g) - m_i(g) \]
                    We also know that $M_i(g) - m_i(f)$ is 
                    bounded below by zero becuase if not, then
                    $m_i(f)>M_i(g)$, implying that  we didn't   
                    choose $M_i(h)$
                    correctly, as $M_i(f)$ would certainly
                    have been larger than $m_i(f)$ and, thus also
                    $M_i(g)$.
                    \\
                    \\
                    So by this fact, Equation \ref{ems2}, and our
                    choice of $\delta$:
                    \[ 0 \leq M_i(h) - m_i(h) 
                        =  M_i(g) - m_i(f)  \leq
                        M_i(g) - m_i(g) 
                        \leq\frac{\epsilon}{\alpha(b)-\alpha(a)}
                        \]
            \end{enumerate}
            Putting it all together, we managed to bound
                \[ M_i(h) - m_i(h) 
                    \leq\frac{\epsilon}{\alpha(b)-\alpha(a)}
                    \qquad \forall i \]
            This implies that 
            \begin{align*}
                U_\alpha(h,P^*) - L_\alpha(h,P^*) &=
                \sum^n_{i=1} [M_i(h) - m_i(h)] \Delta\alpha_i 
                \leq \sum^n_{i=1} 
                    \frac{\epsilon}{\alpha(b)-\alpha(a)}
                    \Delta\alpha_i \\
                &\leq  \frac{\epsilon}{\alpha(b)-\alpha(a)}
                    \sum^n_{i=1} \Delta\alpha_i \\
                &\leq \frac{\epsilon}{\alpha(b)-\alpha(a)} \cdot
                    (\alpha(b) - \alpha(a)) 
                = \epsilon
            \end{align*}
            So we have a process to get the partition 
            to satisfy Riemann's Condition for integrability
            for any $\epsilon$ implying $h\in 
            \mathscr{R}_\alpha([a,b])$.
            
        \newpage
        \item Next, we show that $h(x) = \min \{f, g\} \in
            \mathscr{R}_\alpha([a,b])$ using 
            Riemann's Condition. So for all $\epsilon>0$, we need
            to again find a sufficient partitition $P$.
            \\
            \\
            Also again, take $\delta = 
            \epsilon/[\alpha(b)-\alpha(a)]$ and use 
            Riemann's condition to find $P_1$ and $P_2$ 
            as above, in Equations \ref{q2p1} and \ref{q2p2}.
            Then take their common refinement to find $P^*$. 
            Again, 
            \textbf{this will be our partition $P$ such that 
            Equation \ref{toprove} holds}.
            \\
            \\
            To formally show this, consider any arbitrary 
            interval defined by the partition $P^*$. Over any 
            interval $[x_{i-1}, x_i]$ in $P^*$, 
            \begin{align*}
                M_i(h) = \sup_{x\in[x_{i-1}, x_i]} 
                    \min\{f(x), g(x)\} \\
                m_i(h) = \inf_{x\in[x_{i-1}, x_i]} 
                    \min\{f(x), g(x)\} 
            \end{align*}
            Narrowing down the list of 
            candidates for $M_i(h)$ and $m_i(h)$:
            \begin{align*}
                M_i(h) =  \min\{M_i(f), M_i(g)\} \\
                m_i(h) = \min\{m_i(f), m_i(g)\} 
            \end{align*}
            The cases are then \emph{exactly analogous}
            to what we saw for the case of the max.
            The result is that we again managed to bound
                \[ M_i(h) - m_i(h) 
                    \leq\frac{\epsilon}{\alpha(b)-\alpha(a)}
                    \qquad \forall i \]
            This implies that 
            \begin{align*}
                U_\alpha(h,P^*) - L_\alpha(h,P^*) &=
                \sum^n_{i=1} [M_i(h) - m_i(h)] \Delta\alpha_i 
                \leq \sum^n_{i=1} 
                    \frac{\epsilon}{\alpha(b)-\alpha(a)}
                    \Delta\alpha_i \\
                &\leq  \frac{\epsilon}{\alpha(b)-\alpha(a)}
                    \sum^n_{i=1} \Delta\alpha_i \\
                &\leq \frac{\epsilon}{\alpha(b)-\alpha(a)} \cdot
                    (\alpha(b) - \alpha(a)) 
                = \epsilon
            \end{align*}
            So we have a process to get the partition 
            to satisfy Riemann's Condition for integrability
            for any $\epsilon$ implying $h\in 
            \mathscr{R}_\alpha([a,b])$.
    \end{enumerate}

\newpage
\item \textbf{Exercise 51.18}: We want an example of an increasing
    $\alpha$ on $[a,b]$ and a bounded function $f$ such that
    $|f| \in \mathscr{R}_\alpha([a,b])$ but 
    $f \not\in \mathscr{R}_\alpha([a,b])$. 
    \\
    \\
    Consider the interval $[0,1]$ and the functions
    \begin{equation}
        f(x) = 
            \begin{cases} 
                -1 & x \in \mathbb{Q} \subset \mathbb{R} \\
                1 &  x \in \mathbb{R} \setminus \mathbb{Q}
            \end{cases} 
        \qquad 
        \alpha(x) = x
    \end{equation}
    First, for $|f| \in \mathscr{R}_\alpha([0,1])$, we
    a situation similar to Example 51.9 in the book, as 
        \[ |f(x)| = 1 \]
    a constant. Then for any partition $P$, we have
        \[ M_i(|f|) = m_i(|f|) = 1 \]
    which implies (after telescoping the sum) that 
    \begin{align*}
        U_\alpha(|f|, P) &= \sum^n_{i=1} M_i(|f|) \Delta\alpha(i)  =
            \sum^n_{i=1} 1\cdot \Delta\alpha(i) \\
            &= \alpha(b) -\alpha(a) = 1 - 0 = 1\\
        L_\alpha(|f|, P) &= \sum^n_{i=1} m_i(|f|) \Delta\alpha(i) =
            \sum^n_{i=1} 1\cdot \Delta\alpha(i) \\
            &= \alpha(b) -\alpha(a) = 1 - 0  = 1
    \end{align*}
    And so we satisfy Riemann's Condition as
        \[ U_\alpha(|f|, P) - L_\alpha(|f|, P) = 0 \leq \epsilon \]
    for all $\epsilon>0$, which implies $|f| \in \mathscr{R}_\alpha([a,b])$
    \\
    \\
    \textbf{Show $f\not\in \mathscr{R}_\alpha([0,1])$}: For this, we want
    to show that there does not exist a partition such that Riemann's
    condition holds. 
    \\
    \\
    So start with the fact that for any partition $P$, all
    intervals $[x_{i-1}, x_i]$  
    will contain a rational and an irrational number. Thus
    by our definition of $f$, 
        \[ M_i(f) = 1, \quad m_i(f) = -1 \qquad \forall i \]
    Therefore, for all $P$
    \begin{align*}
        U_\alpha(f,P) - L_\alpha(f,P) &= 
            \sum^n_{i=1} [M_i(f) - m_i(f)] \Delta\alpha_i =
            \sum^n_{i=1} [1 - (-1)] \Delta\alpha_i \\
        &= 2 \sum^n_{i=1} \Delta\alpha_i = 
            2(\alpha(1) - \alpha(0)) = 2 (1-0)\\
        &= 2 \not\leq \epsilon \qquad \forall \epsilon > 0
    \end{align*}
    Thus, by Riemann's condition $f \not\in 
    \mathscr{R}_\alpha([0,1])$

            
        
\item We want to show 
    \begin{equation}
        \label{q4.1}
        U_\alpha(|f|, P) - L_\alpha(|f|, P) \leq
        U_\alpha(f, P) - L_\alpha(f, P) 
    \end{equation}
    and  then give an alternate proof that 
    \begin{equation}
        \label{q4.2}
         f \in \mathscr{R}_\alpha([a,b]) \quad \Rightarrow
            \quad  |f| \in \mathscr{R}_\alpha([a,b]) 
    \end{equation}
    Now it's clear that if we can show Relation \ref{q4.1},
    then Relation \ref{q4.2} follows immediately. This is 
    because if $f \in \mathscr{R}_\alpha([a,b])$, then
    (by Riemann's condition) there exists a partition 
    $P^*$ such that 
        \[ U_\alpha(f, P^*) - L_\alpha(f, P^*) \leq \epsilon \]
    And if this is true, by Relation \ref{q4.1}
        \[ U_\alpha(|f|, P^*) - L_\alpha(|f|, P^*) \leq \epsilon \]
    and so by Riemann's condition in reverse, we have
    that $|f| \in \mathscr{R}_\alpha([a,b])$.
    \\
    \\
    So now, we just need to prove Relation \ref{q4.1}.
    Rewriting as we did in the problems above, we 
    need to show that
    \begin{align}
        \sum^n_{i=1} [M_i(|f|) - m(|f|)]\Delta \alpha_i
        &\leq \sum^n_{i=1} [M_i(f) - m(f)]\Delta \alpha_i 
        \notag \\
        \text{By $\alpha$ increasing }
        \Leftrightarrow \quad
            M_i(|f|) - m(|f|) &\leq M_i(f) - m(f)
            \qquad \forall i
            \label{equiv}
    \end{align}
    Now first, by our definition of $M_i(f)$ and
    $m_i(f)$, we have that
    \begin{equation}
        \label{q4.ineq}
        f(x) - f(y) \leq M_i(f) - m_i(f) 
        \qquad \forall x,y \in [x_{i-1}, x_i]
    \end{equation}
    This is true because $[m_i(f), M_i(f)]$ bound 
    the range of $f$ on $[x_{i-1}, x_i]$ such that
    $f(x) \in [m_i(f), M_i(f)]$ for all $x$.\footnote{If
    that weren't the case, then either 
    $f(x)>M_i(f)$ or $f(x)<m_i(f)$, in which case
    we clearly chose the wrong $m_i(f)$ or $M_i(f)$.}
    Since $M_i(f) - m_i(f)$ denotes the maximum distance 
    between points in the image of $f$ on the interval
    $[x_{i-1}, x_i]$, Relation \ref{q4.ineq} must 
    hold.
    \\
    \\
    Next, we'll use a simple property of norms:
    \begin{equation}
        \label{q4.3}
        \left\lvert \;|p| - |q| \;\right\rvert 
            \leq 
        \left\lvert p - q \right\rvert 
    \end{equation}
    Putting Relations \ref{q4.ineq} and 
    \ref{q4.3} together, we see that 
    \begin{align*}
        \left\lvert \; |f(x)| - |f(y)| \;\right\rvert 
        \leq 
        \left\lvert f(x) - f(y)\right\rvert 
        \leq 
        \left\lvert M_i(f) - m_i(f)\right\rvert 
        \qquad x, y \in[x_{i-1}, x_i]
    \end{align*}
    This proves Relation \ref{equiv} because
    \begin{align*}
        M_i(|f|) = |f(x)|\qquad 
            \text{for some $x \in [x_{i-1}, x_i]$} \\
        m_i(|f|) = |f(y)|\qquad 
            \text{for some $y \in [x_{i-1}, x_i]$} 
    \end{align*}




\item Without loss of generality, suppose the set $\Omega$ 
    has one accumulation point. Let 
    $a \in [0,1]$ denote this set 
    accumulation point. Our goal is to find a $P$ such
    that
        \[ U_\alpha(\chi_\Omega, P) - L_\alpha(\chi_\Omega, P) 
            \leq \epsilon \quad \Rightarrow \quad 
            \chi_\Omega \in \mathscr{R}_\alpha([a,b])\]
    Do do so, we'll make the intervals $[x_{i-1}, x_i]$ 
    around the points in $\Omega$ (the jumps where 
    $M_i(\chi_\Omega) - m_i(\chi_\Omega)= 1$ all 
    very small.  
    And recall that this problem assumes $\alpha(x) = x$
    \\
    \\
    So consider the accumulation point $a$ and 
    take $\delta=\epsilon/4$ (where $\epsilon$ is 
    as above). By definition,
    there exists a sequence in $\{a_m\}$ in $\Omega$
    such that 
        \[ \lim_{m\rightarrow \infty} a_m = a \]
    Then, by the definition of limit points, there is
    a number $M$ such that 
        \[ m>M \Rightarrow |a_m - a|\leq \delta \]
    Thus there are only finitely many points 
    outside of the interval 
    \begin{equation} 
        \label{aint}
        [a-\delta, a+\delta]
    \end{equation} 
    ---$M$ points outside, to be exact.
    Now let 
        \[ B = \{b_1, \ldots, b_M\} \]
    be that set of points. 
    \\
    \\
    So given this setup, let's choose the intervals 
    for our partition, $P$.  Around the point $a$, 
    let, 
    \begin{equation}
        \label{istar}
         [y_0, y_1] = [a-\delta, a + \delta]
            \qquad \text{from \ref{aint}} 
    \end{equation}
    Next, for each $b_k \in B$, set 
    \begin{equation}
        \label{bigI}
         [y_{2\cdot k }, y_{2\cdot k+1}] = 
            [b_k - \delta/M, b_k + \delta/M ]
    \end{equation}
    Now we have a set of points, $\{y_0, \ldots, 
    y_{2\cdot M+1}\}$ on the 
    interval $[0,1]$ which can be sorted,
    relabeled as $x$'s, and re-indexed to
    form our partition $\{x_0, \ldots, x_N\}$.
    \\
    \\
    Given this partition, $P$, let $i^*$ denote
    the index for the interval formed in 
    Equation \ref{istar}. Let $I$ denote the indices 
    for the intervals from Equation
    \ref{bigI}. Let $J$ denote the other interval 
    indices. Note that the intervals with index $i^*$
    or $i\in I$ will have the characteristic-function 
    jumps where $M_i - m_i = 1$. Other intervals will
    be flat at 0, by design.
    \\
    \\
    Now, finally, let's rewrite our difference
    of Darboux sums, noting that the problem assumes
    $\alpha(x)=x$:
    \begin{align*}
        U_\alpha(\chi_\Omega, P) - 
            L_\alpha(\chi_\Omega, P) &=
            \sum^n_{i=1} [M_i(\chi_\Omega) 
            - m_i(\chi_\Omega)] \Delta x_i \\
        &=  [M_{i^*}(\chi_\Omega) - m_{i^*}(\chi_\Omega)]
            \Delta x_{i^*} \\
        &\quad +
            \sum_{i\in I}[M_{i}(\chi_\Omega) - m_{i}(\chi_\Omega)]
                \Delta x_i \\
        &\quad +
            \sum_{j\in J}[M_{j}(\chi_\Omega) - m_{j}(\chi_\Omega)]
                \Delta x_j \\
        &\leq [1 - 0]
            \left(2\cdot\delta\right) 
         +
            \sum_{i\in I}[1-0]
            \left(2\cdot\delta/M\right) \\
        &\quad +
            \sum_{j\in J} 0
                (x_j - x_{j-1}) \\
        &\leq 2\delta + M (2\delta/M) = 4\delta\\
        \Rightarrow U_\alpha(\chi_\Omega, P) - 
            L_\alpha(\chi_\Omega, P) &\leq
            4\delta = 4(\epsilon/4) = \epsilon
    \end{align*}
    So by Riemann's condition $\chi_\Omega \in 
    \mathscr{R}_\alpha([a,b])$.
    \\
    \\
    Finally, I didn't lose any generality by assuming that
    there was only one accumulation point.  If there
    were $N_a$ accumualtion points, divide $\delta$ 
    above by $N_a$ and proceed in exactly the same way
    for each accumulation point.  





  
    




\end{enumerate}



%%%% APPPENDIX %%%%%%%%%%%

% \appendix

%\cite{LabelInSourcesFile} 
%\citep{LabelInSourcesFile} Cites in parens
%\nocite{LabelInSourceFile} includes in refs w/o specific citation
%\bibliographystyle{apalike} 
%\bibliography{sources.bib} where sources.bib is file




\end{document}



%%%% INCLUDING FIGURES %%%%%%%%%%%%%%%%%%%%%%%%%%%%

   % H indicates here 
   %\begin{figure}[h!]
   %   \centering
   %   \includegraphics[scale=1]{file.pdf}
   %\end{figure}

%   \begin{figure}[h!]
%      \centering
%      \mbox{
%	 \subfigure{
%	    \includegraphics[scale=1]{file1.pdf}
%	 }\quad
%	 \subfigure{
%	    \includegraphics[scale=1]{file2.pdf} 
%	 }
%      }
%   \end{figure}
 

%%%%% Including Code %%%%%%%%%%%%%%%%%%%%%5
% \verbatiminput{file.ext}    % Includes verbatim text from the file
% \texttt{text}	  % includes text in courier, or code-like, font
