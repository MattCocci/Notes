\documentclass[12pt]{article}

\author{Matthew Cocci}
\title{\textbf{Homework 1}}
\date{\today}
\usepackage{fullpage}
\usepackage{enumitem} %Has to do with enumeration	
\usepackage{amsfonts}
\usepackage{amsmath}
\usepackage[T1]{fontenc}
\usepackage[utf8]{inputenc}
\usepackage{blindtext}
\usepackage{graphicx}
\usepackage{hyperref} 
\hypersetup{	
    colorlinks,		% This colors the links themselves, not boxes
    citecolor=black,	% Everything below changes the link colors
    filecolor=black,
    linkcolor=black,
    urlcolor=black
}
\usepackage{mathrsfs} %allows for labeling of theorems
\usepackage{amsthm} %allows for labeling of theorems
\theoremstyle{plain}
\newtheorem{thm}{Theorem}[section]
\newtheorem{lem}[thm]{Lemma}
\newtheorem{prop}[thm]{Proposition}
\newtheorem{cor}[thm]{Corollary}

\theoremstyle{definition}
\newtheorem{defn}[thm]{Definition}
\newtheorem{ex}[thm]{Example}

\theoremstyle{remark}
\newtheorem*{rem}{Remark}
\newtheorem*{note}{Note}
\usepackage{appendix}
\usepackage{subfigure} % For plotting multiple figures at once
\usepackage{verbatim} % for including verbatim code from a file
\usepackage{natbib} % for bibliographies

\begin{document}

\maketitle

%\tableofcontents %adds it here

\begin{enumerate}

\item 
    \begin{enumerate}
        \item We want to show that, for all $P$, 
        \begin{align*}
            U_\alpha(f, P) - L_\alpha(f,P) &\leq 
                w_f\left(||P||\right) (\alpha(b) - \alpha(a)) 
        \end{align*}
        By the definition of the Upper and Lower Darboux sums,
        \begin{align}
            U_\alpha(f, P) - L_\alpha(f,P) &= 
            \sum^n_{i=1} M_i(f)\Delta\alpha_i - 
            \sum^n_{i=1} m_i(f)\Delta\alpha_i \notag \\
            &= \sum^n_{i=1} \left[M_i(f) - m_i(f)\right]
            \Delta\alpha_i \label{q1a.1}
        \end{align}
        Now let $\overline{x}_i$ and $\underline{x}_i$ be the 
        points such that 
        \begin{align*}
            f(\underline{x}_i) &= m_i(f) 
                = \inf_{x\in [x_i, x_{i-1}]} f(x)\\
            f(\overline{x}_i) &= M_i(f)
                = \sup_{x\in [x_i, x_{i-1}]} f(x)
        \end{align*}
        Now since $\underline{x}_i$ and $\overline{x}_i$ are
        both in $[x_{i-1}, x_i]$, we can combine this with
        the definition of $||P||$ to get
            \[ |\underline{x}_i - \overline{x}_i| \leq
                |x_i - x_{i-1}| \leq ||P|| \]
        By the definition of the modulus of continuity, 
        \begin{align*}
            |\underline{x}_i - \overline{x}_i| \leq
                ||P|| \quad \Rightarrow \quad
                &|f(\underline{x}_i) - f(\overline{x}_i)| 
                \leq w_f\left(||P||\right)  \\
            \Leftrightarrow \quad &|M_i(f) - m_i(f)| 
                \leq w_f\left(||P||\right)  \\
        \end{align*}
        Substituting this fact back into \ref{q1a.1}, and 
        using the properties of telescoping sums, we get that
        \begin{align*}
            \sum^n_{i=1} \left[M_i(f) - m_i(f)\right]
            \Delta\alpha_i 
            &\leq  \sum^n_{i=1} w_f\left(||P||\right) 
            \Delta\alpha_i = 
            w_f\left(||P||\right) \sum^n_{i=1} \Delta\alpha_i\\
            &\leq w_f\left(||P||\right) (\alpha(b)-\alpha(a)) 
        \end{align*}
                

    \item We know $f$ is monotone on $[a,b]$. Suppose that 
        $f$ is monotone increasing.\footnote{
        Note that I'll lose no generality in assuming
        that the function $f$ was monotone increasing.  If it's
        monotone decreasing instead, swap the values of $M_i$
        and $m_i$ in Equation \ref{q1b.1}, and proceed in 
        exactly the same way.}Then given any partition,
        $P$,
        \[ f(x_{i-1}) \leq f(c) \leq f(x_i) \quad \forall i, 
            c \in [x_{i-1}, x_i]\]
        This implies that 
        \begin{equation}
            M_i = f(x_i) \quad m_i=f(x_{i-1})\qquad\forall i
            \label{q1b.1}
        \end{equation}
        Writing $U_\alpha - L_\alpha$ as above in Equation 
        \ref{q1a.1}, we see that the expression reduces to
        \begin{equation}
            \label{q1b.2}
            \sum^n_{i=1} \left[M_i(f) - m_i(f)\right]
            \Delta\alpha_i = 
            \sum^n_{i=1} \left[f(x_{i}) - f(x_{i-1})\right]
            \Delta\alpha_i 
        \end{equation}
        Now let's consider $w_\alpha\left(||P||\right)$.
        By definition, it is
            \[ w_\alpha\left(||P||\right) := 
                \sup_{|x-y|\leq ||P||} |\alpha(x)-\alpha(y)|
                \]
        It's clear that for each sub-interval, $[x_{i-1}, x_i]$,
        we have
            \[ |x_{i}-x_{i-1}| \leq ||P|| \quad \forall i\]
        This implies that
            \[ |\Delta \alpha_i| = 
                |\alpha(x_i) - \alpha(x_{i-1})| \leq
                w_\alpha\left(||P||\right) \qquad \forall i\]
        Subbing this into the right-hand side of equation 
        \ref{q1b.2}, we can simplify using the properties 
        of telescoping sums to get
        \begin{align}
            \sum^n_{i=1} \left[f(x_{i}) - f(x_{i-1})\right]
            \Delta\alpha_i  &\leq 
            \sum^n_{i=1} \left[f(x_{i}) - f(x_{i-1})\right]
             w_\alpha\left(||P||\right) \notag\\
            &\leq w_\alpha\left(||P||\right) 
            \sum^n_{i=1} \left[f(x_{i}) - f(x_{i-1})\right]
             \notag \\
            \Leftrightarrow U_\alpha(f,P) - L_\alpha(f,P) 
            &\leq w_\alpha\left(||P||\right) [f(b) -f(a)]
            \label{q1b.3}
        \end{align}
        \\
        Finally, since $\alpha$ is assumed continuous on the
        compact interval $[a,b]$, $\alpha$ is uniformly 
        continuous on $[a,b]$.  That means
        \[ \forall \epsilon > 0 \quad \exists \delta >0 
            \quad \text{s.t.} \quad |x-y| \leq \delta \quad
            \Rightarrow \quad
            |\alpha(x) -\alpha(y)| \leq \epsilon \]
        So if we can make $w_\alpha\left(||P||\right) < \epsilon$,
        by choosing our partition $P$ such that 
        $||P|| \leq \delta$.
        With Equation \ref{q1b.3}, we therefore ensure that
        \begin{equation}
             U_\alpha(f,P) - L_\alpha(f,P) 
            \leq \epsilon [f(b) -f(a)]
        \end{equation}
        for any arbitrary $\epsilon$.
        And so by Riemann's Condition, $f \in 
        \mathscr{R}_\alpha([a,b])$ because for any
        $\epsilon>0$, we can ensure the upper and lower sums 
        are within that distance of each other by choosing
        a sufficiently fine partition.

    \end{enumerate}

\item \textbf{Exercise 51.15}: First, let's establish some
    basic building blocks for the proof. 
    We have $\alpha$ increasing on
    $[a,b]$ and $f,g \in \mathscr{R}_\alpha([a,b])$. By
    Riemann's condition, it's clear that for all $\delta>0$,
    there exist partitions $P_1$ and $P_2$ such that 
    \begin{align}
        U_\alpha(f,P_1) - L_\alpha(f,P_1) < \delta 
            \label{q2p1} \\
        U_\alpha(g,P_2) - L_\alpha(g,P_2) < \delta 
            \label{q2p2} 
    \end{align}
    Now take the common refinement, $P^* = P_1 \cup P_2$.
    By Lemma 51.5 and Corollary 51.6 (in FoMA), we know
    that 
    \begin{align*}
         L_\alpha(f,P_1) \leq L_\alpha(f,P^*) \leq
            U_\alpha(f,P^*) \leq U_\alpha(f,P_1) \\
         L_\alpha(g,P_2) \leq L_\alpha(g,P^*) \leq
            U_\alpha(g,P^*) \leq U_\alpha(g,P_2) 
    \end{align*}
    Combining this with the result from Riemann's condition
    above (and rewriting as in Question 1), 
    we see that we must also have
    \begin{align*}
        \sum^n_{i=1} [M_i(f) - m_i(f)] \Delta\alpha_i =
        U_\alpha(f,P^*) - L_\alpha(f,P^*) \leq
        U_\alpha(f,P_1) - L_\alpha(f,P_1) < \delta \\
        \sum^n_{i=1} [M_i(g) - m_i(g)] \Delta\alpha_i =
        U_\alpha(g,P^*) - L_\alpha(g,P^*) \leq
        U_\alpha(g,P_2) - L_\alpha(g,P_2) < \delta 
    \end{align*}
    Now since $\alpha$ is assumed increasing, we know
    that $\Delta\alpha_i \geq 0$ for all $i$. Also,
    $M_i(\cdot) \geq m_i(\cdot)$ 
    for all $i$. Therefore, we can conclude
    \begin{align}
        0 \leq M_i(f) - m_i(f) < \delta \qquad
        \forall i \label{ems1} \\
        0 \leq M_i(g) - m_i(g) < \delta \qquad
        \forall i \label{ems2} 
    \end{align}
    Now, using all of this as groundwork, let's show the main
    result.
    \newpage
    \begin{enumerate}
        \item We'll start by showing $h(x) = \max \{f, g\} \in
            \mathscr{R}_\alpha([a,b])$ using 
            Riemann's Condition. So for all $\epsilon>0$, we need
            to find a partitition $P$ such that 
            \begin{equation}
                \label{toprove}
                U_\alpha(h,P) - L_\alpha(h,P) < \epsilon
            \end{equation}
            To do so, take $\delta = 
            \epsilon/[\alpha(b)-\alpha(a)]$ and use 
            Riemann's condition to find $P_1$ and $P_2$ 
            as above, in Equations \ref{q2p1} and \ref{q2p2}.
            Then take their common refinement to find $P^*$. 
            \textbf{This will be our partition $P$ such that 
            Equation \ref{toprove} holds}.
            \\
            \\
            To formally show this, consider any arbitrary 
            interval defined by the partition $P^*$. Over any 
            interval $[x_{i-1}, x_i]$ in $P^*$, 
            \begin{align*}
                M_i(h) = \sup_{x\in[x_{i-1}, x_i]} 
                    \max\{f(x), g(x)\} \\
                m_i(h) = \inf_{x\in[x_{i-1}, x_i]} 
                    \max\{f(x), g(x)\} 
            \end{align*}
            It's clear that we can narrow the list of 
            candidates for $M_i(h)$ and $m_i(h)$:
            \begin{align*}
                M_i(h) =  \max\{M_i(f), M_i(g)\} \\
                m_i(h) = \max\{m_i(f), m_i(g)\} 
            \end{align*}
            Let's consider the cases:
            \begin{enumerate}
                \item Suppose $M_i(h) = M_i(f)$ and
                    $m_i(h) = m_i(f)$. Then by Equation
                    \ref{ems1} and our choice of $\delta$,
                    we have
                    \[ 0  \leq \leq M_i(h) - m_i(h) =
                        M_i(f) - m_i(f)  =
                        \leq\frac{\epsilon}{\alpha(b)-\alpha(a)}
                        \]
                \item Similarly, if $M_i(h) = M_i(g)$ and
                    $m_i(h) = m_i(g)$. Then by Equation
                    \ref{ems2} and our choice of $\delta$,
                    we have
                    \[ 0 \leq
                        M_i(h) - m_i(h) =  M_i(g) - m_i(g) 
                        \leq\frac{\epsilon}{\alpha(b)-\alpha(a)}
                        \]
                \item Next, suppose that $M_i(h) = M_i(f)$ and
                    $m_i(h) = m_i(g)$. In that case, 
                        \[ m_i(g) > m_i(f) \quad
                           \Rightarrow \quad 
                           M_i(f) - m_i(g) < M_i(f) - m_i(f) \]
                    We also know that $M_i(f) - m_i(g)$ is 
                    bounded below by zero becuase if not, then
                    $m_i(g)>M_i(f)$, implying that  we didn't   
                    choose $M_i(h)$
                    correctly, as $M_i(g)$ would certainly
                    have been larger than $m_i(g)$ and, thus also
                    $M_i(f)$.
                    \\
                    \\
                    So by this fact, Equation \ref{ems1}, and our
                    choice of $\delta$:
                    \[ 0 \leq M_i(h) - m_i(h) 
                        =  M_i(f) - m_i(g)  \leq
                        M_i(f) - m_i(f) 
                        \leq\frac{\epsilon}{\alpha(b)-\alpha(a)}
                        \]
                \item Finally, suppose that $M_i(h) = M_i(g)$ and
                    $m_i(h) = m_i(f)$. In that case, 
                        \[ m_i(f) > m_i(g) \quad
                           \Rightarrow \quad 
                           M_i(g) - m_i(f) < M_i(g) - m_i(g) \]
                    We also know that $M_i(g) - m_i(f)$ is 
                    bounded below by zero becuase if not, then
                    $m_i(f)>M_i(g)$, implying that  we didn't   
                    choose $M_i(h)$
                    correctly, as $M_i(f)$ would certainly
                    have been larger than $m_i(f)$ and, thus also
                    $M_i(g)$.
                    \\
                    \\
                    So by this fact, Equation \ref{ems2}, and our
                    choice of $\delta$:
                    \[ 0 \leq M_i(h) - m_i(h) 
                        =  M_i(g) - m_i(f)  \leq
                        M_i(g) - m_i(g) 
                        \leq\frac{\epsilon}{\alpha(b)-\alpha(a)}
                        \]
            \end{enumerate}
            Putting it all together, we managed to bound
                \[ M_i(h) - m_i(h) 
                    \leq\frac{\epsilon}{\alpha(b)-\alpha(a)}
                    \qquad \forall i \]
            This implies that 
            \begin{align*}
                U_\alpha(h,P^*) - L_\alpha(h,P^*) &=
                \sum^n_{i=1} [M_i(h) - m_i(h)] \Delta\alpha_i 
                \leq \sum^n_{i=1} 
                    \frac{\epsilon}{\alpha(b)-\alpha(a)}
                    \Delta\alpha_i \\
                &\leq  \frac{\epsilon}{\alpha(b)-\alpha(a)}
                    \sum^n_{i=1} \Delta\alpha_i \\
                &\leq \frac{\epsilon}{\alpha(b)-\alpha(a)} \cdot
                    (\alpha(b) - \alpha(a)) 
                = \epsilon
            \end{align*}
            So we have a process to get the partition 
            to satisfy Riemann's Condition for integrability
            for any $\epsilon$ implying $h\in 
            \mathscr{R}_\alpha([a,b])$.
            
        \newpage
        \item Next, we show that $h(x) = \min \{f, g\} \in
            \mathscr{R}_\alpha([a,b])$ using 
            Riemann's Condition. So for all $\epsilon>0$, we need
            to again find a sufficient partitition $P$.
            \\
            \\
            Also again, take $\delta = 
            \epsilon/[\alpha(b)-\alpha(a)]$ and use 
            Riemann's condition to find $P_1$ and $P_2$ 
            as above, in Equations \ref{q2p1} and \ref{q2p2}.
            Then take their common refinement to find $P^*$. 
            Again, 
            \textbf{this will be our partition $P$ such that 
            Equation \ref{toprove} holds}.
            \\
            \\
            To formally show this, consider any arbitrary 
            interval defined by the partition $P^*$. Over any 
            interval $[x_{i-1}, x_i]$ in $P^*$, 
            \begin{align*}
                M_i(h) = \sup_{x\in[x_{i-1}, x_i]} 
                    \min\{f(x), g(x)\} \\
                m_i(h) = \inf_{x\in[x_{i-1}, x_i]} 
                    \min\{f(x), g(x)\} 
            \end{align*}
            Narrowing down the list of 
            candidates for $M_i(h)$ and $m_i(h)$:
            \begin{align*}
                M_i(h) =  \min\{M_i(f), M_i(g)\} \\
                m_i(h) = \min\{m_i(f), m_i(g)\} 
            \end{align*}
            The cases are then \emph{exactly analogous}
            to what we saw for the case of the max.
            The result is that we again managed to bound
                \[ M_i(h) - m_i(h) 
                    \leq\frac{\epsilon}{\alpha(b)-\alpha(a)}
                    \qquad \forall i \]
            This implies that 
            \begin{align*}
                U_\alpha(h,P^*) - L_\alpha(h,P^*) &=
                \sum^n_{i=1} [M_i(h) - m_i(h)] \Delta\alpha_i 
                \leq \sum^n_{i=1} 
                    \frac{\epsilon}{\alpha(b)-\alpha(a)}
                    \Delta\alpha_i \\
                &\leq  \frac{\epsilon}{\alpha(b)-\alpha(a)}
                    \sum^n_{i=1} \Delta\alpha_i \\
                &\leq \frac{\epsilon}{\alpha(b)-\alpha(a)} \cdot
                    (\alpha(b) - \alpha(a)) 
                = \epsilon
            \end{align*}
            So we have a process to get the partition 
            to satisfy Riemann's Condition for integrability
            for any $\epsilon$ implying $h\in 
            \mathscr{R}_\alpha([a,b])$.
    \end{enumerate}

\item \textbf{Exercise 51.18}: 

\item 
\item 


\end{enumerate}



%%%% APPPENDIX %%%%%%%%%%%

% \appendix

%\cite{LabelInSourcesFile} 
%\citep{LabelInSourcesFile} Cites in parens
%\nocite{LabelInSourceFile} includes in refs w/o specific citation
%\bibliographystyle{apalike} 
%\bibliography{sources.bib} where sources.bib is file




\end{document}



%%%% INCLUDING FIGURES %%%%%%%%%%%%%%%%%%%%%%%%%%%%

   % H indicates here 
   %\begin{figure}[h!]
   %   \centering
   %   \includegraphics[scale=1]{file.pdf}
   %\end{figure}

%   \begin{figure}[h!]
%      \centering
%      \mbox{
%	 \subfigure{
%	    \includegraphics[scale=1]{file1.pdf}
%	 }\quad
%	 \subfigure{
%	    \includegraphics[scale=1]{file2.pdf} 
%	 }
%      }
%   \end{figure}
 

%%%%% Including Code %%%%%%%%%%%%%%%%%%%%%5
% \verbatiminput{file.ext}    % Includes verbatim text from the file
% \texttt{text}	  % includes text in courier, or code-like, font
