\documentclass[12pt]{article}

\author{Matthew Cocci}
\title{\textbf{Homework 8}}
\date{\today}

%% Spacing %%%%%%%%%%%%%%%%%%%%%%%%%%%%%%%%%%%%%%%%%%%%%%%%

\usepackage{fullpage}
\usepackage{setspace}
%\onehalfspacing
\usepackage{microtype}


%% Header %%%%%%%%%%%%%%%%%%%%%%%%%%%%%%%%%%%%%%%%%%%%%%%%%

%\pagestyle{fancy} 
%\lhead{}
%\rhead{}
%\chead{}
%\setlength{\headheight}{15.2pt} 
    %---Make the header bigger to avoid overlap

%\renewcommand{\headrulewidth}{0.3pt} 
    %---Width of the line

%\setlength{\headsep}{0.2in}    
    %---Distance from line to text
            

%% Mathematics Related %%%%%%%%%%%%%%%%%%%%%%%%%%%%%%%%%%%

\usepackage{amsmath}
\usepackage{amsfonts}
\usepackage{mathrsfs}
\usepackage{amsthm} %allows for labeling of theorems
\theoremstyle{plain}
\newtheorem{thm}{Theorem}[section]
\newtheorem{lem}[thm]{Lemma}
\newtheorem{prop}[thm]{Proposition}
\newtheorem{cor}[thm]{Corollary}

\theoremstyle{definition}
\newtheorem{defn}[thm]{Definition}
\newtheorem{ex}[thm]{Example}

\theoremstyle{remark}
\newtheorem*{rem}{Remark}
\newtheorem*{note}{Note}


%% Font Choices %%%%%%%%%%%%%%%%%%%%%%%%%%%%%%%%%%%%%%%%%

\usepackage[T1]{fontenc}
\usepackage[utf8]{inputenc}
\usepackage{lmodern}
%\usepackage{blindtext}


%% Figures %%%%%%%%%%%%%%%%%%%%%%%%%%%%%%%%%%%%%%%%%%%%%%

\usepackage{graphicx}
\usepackage{subfigure} 
    %---For plotting multiple figures at once
%\graphicspath{ {Directory/} }
    %---Set a directory for where to look for figures


%% Hyperlinks %%%%%%%%%%%%%%%%%%%%%%%%%%%%%%%%%%%%%%%%%%%%
\usepackage{hyperref} 
\hypersetup{	
    colorlinks,		
        %---This colors the links themselves, not boxes
    citecolor=black,	
        %---Everything here and below changes link colors
    filecolor=black,
    linkcolor=black,
    urlcolor=black
}

%% Including Code %%%%%%%%%%%%%%%%%%%%%%%%%%%%%%%%%%%%%%% 

\usepackage{verbatim} 
    %---For including verbatim code from files, no colors

\usepackage{listings}
\usepackage{color}
\definecolor{mygreen}{RGB}{28,172,0}
\definecolor{mylilas}{RGB}{170,55,241}
\newcommand{\matlabcode}[1]{%
    \lstset{language=Matlab,%
        basicstyle=\footnotesize,%
        breaklines=true,%
        morekeywords={matlab2tikz},%
        keywordstyle=\color{blue},%
        morekeywords=[2]{1}, keywordstyle=[2]{\color{black}},%
        identifierstyle=\color{black},%
        stringstyle=\color{mylilas},%
        commentstyle=\color{mygreen},%
        showstringspaces=false,%
            %---Without this there will be a symbol in 
            %---the places where there is a space
        numbers=left,%
        numberstyle={\tiny \color{black}},% 
            %---Size of the numbers
        numbersep=9pt,% 
            %---Defines how far the numbers are from the text
        emph=[1]{for,end,break,switch,case},emphstyle=[1]\color{red},%
            %---Some words to emphasise
    }%
    \lstinputlisting{#1}
}
    %---For including Matlab code from .m file with colors,
    %---line numbering, etc. 


%% Misc %%%%%%%%%%%%%%%%%%%%%%%%%%%%%%%%%%%%%%%%%%%%%% 

\usepackage{enumitem} 
    %---Has to do with enumeration	
\usepackage{appendix}
%\usepackage{natbib} 
    %---For bibliographies
\usepackage{pdfpages}
    %---For including whole pdf pages as a page in doc


%% User Defined %%%%%%%%%%%%%%%%%%%%%%%%%%%%%%%%%%%%%%%%%% 

%\newcommand{\nameofcmd}{Text to display}
\newcommand*{\Chi}{\mbox{\large$\chi$}} %big chi



%%%%%%%%%%%%%%%%%%%%%%%%%%%%%%%%%%%%%%%%%%%%%%%%%%%%%%%%%%%%%%%%%%%%%%%% 
%% BODY %%%%%%%%%%%%%%%%%%%%%%%%%%%%%%%%%%%%%%%%%%%%%%%%%%%%%%%%%%%%%%%%
%%%%%%%%%%%%%%%%%%%%%%%%%%%%%%%%%%%%%%%%%%%%%%%%%%%%%%%%%%%%%%%%%%%%%%%% 


\begin{document}

\maketitle 

\begin{enumerate} 

% Question 1
\item \textbf{Exercise 90.1}: We define a sequence of functions $\{f_n\}$ by
\[
    f_n(x) = n \Chi_{(0,1/n)}
    \qquad x \in [0,1]
\]
(Prove $f_n$ converges to $f(x)=0$ p.w.) Start by supposing that $x\in(0,1]$; we will consider $x=0$ later. Then fixing $\varepsilon>0$, we need to find an $N$ such that
\[
    n>N
    \quad \Rightarrow \quad
    |f_n(x) - 0 | < \varepsilon
\]
But by the Archimedean Property, we can find an $N$ such that 
\[
    \frac{1}{N} < x
    \quad \Rightarrow \quad
    \Chi_{(0,1/N)}(x) = 0
    \quad \Rightarrow \quad
    f_n(x) = N\cdot 0 = 0
\]
This is true for all $x\in(0,1]$, so it converges pointwise on that interval.

Next, suppose that $x=0$. Then $\Chi_{(0,1/n)} = 0$ for all $n$, so that $f_n(0)\rightarrow 0$.
\\
\\
(Show each $f_n, f\in\mathscr{L}([0,1])$ for all $n$) First, note that $f_n$ is a simple function taking on only two values:
\begin{align*}
    f_n = 
    \begin{cases} n & x\in(0,1/n)
        \\ 0 & x\in[0,1]\setminus (0,1/n)
    \end{cases}
\end{align*}
So then we can integrate according to the very straightforward approach we know for simple functions:
\begin{align*}
    \int_{[a,b]} f_n \; dm &= 
    n \cdot m\left((0,1/n)\right)
    + 0 \cdot m\left([0,1]\setminus (0,1/n)\right) \\
    &= n \cdot \left( 1/n - 0\right) + 0= 1
\end{align*}
Now this is true for all $f_n$, so that the limit of the integrals is the limit of a sequence of ones:
\begin{align*}
    \lim_{n\rightarrow\infty}
    \int_{[a,b]} f_n \; dm &= 
    \lim_{n\rightarrow\infty} 1 = 1
\end{align*}
Moving on to $f$, we know that $f(x)=0$ on $[0,1]$, therefore
\[
    \int_{[a,b]} f \; dm = 0
    \qquad\Rightarrow\qquad
    1 = \lim_{n\rightarrow\infty}
    \int_{[a,b]} f_n \; dm \neq
    \int_{[a,b]} f \; dm  = 0
\]

% Question 2
\item For $f\in\mathscr{L}([a,b])$, we want to show that $F(x)$, defined
\begin{align*}
    F(x) = \int_{[a,x]} f \; dm \qquad x\in[a,b]
\end{align*}
is of bounded variation on $[a,b]$.


% Question 3
\item \textbf{Exercise 90.5}
\begin{enumerate} 

% Part a
\item We have $f,|f|\in\mathscr{R}([c,b])$ for all $c\in(a,b)$ and that the improper Riemann integral
\begin{equation}
    \int^b_{a^+} |f| \; dx
\end{equation}
converges. Then we want to show that 
\[
    \int_{[a,b]} f \; dm = 
    \int^b_{a^+} f \; dx
\]
But first, we want to show, more simply, that $f\in\mathscr{L}([a,b])$. To do so, we define a sequence of functions
\[
    f_n(x) = f(x)\cdot\Chi_{[1/n,1]} 
\]
Now it's clear that as $n\rightarrow\infty$, 



\end{enumerate} 
    



\end{enumerate} 
\end{document}



%%%% INCLUDING FIGURES %%%%%%%%%%%%%%%%%%%%%%%%%%%%

   % H indicates here 
   %\begin{figure}[h!]
  %   \centering
   %   \includegraphics[scale=1]{file.pdf}
   %\end{figure}

%   \begin{figure}[h!]
%      \centering
%      \mbox{
%	 \subfigure{
%	    \includegraphics[scale=1]{file1.pdf}
%	 }\quad
%	 \subfigure{
%	    \includegraphics[scale=1]{file2.pdf} 
%	 }
%      }
%   \end{figure}
 

%%%%% Including Code %%%%%%%%%%%%%%%%%%%%%5
% \verbatiminput{file.ext}    % Includes verbatim text from the file
% \texttt{text}	  % includes text in courier, or code-like, font
