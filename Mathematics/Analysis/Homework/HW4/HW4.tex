\documentclass[12pt]{article}

\author{Matthew Cocci}
\title{\textbf{Homework 4}}
\date{\today}

%% Spacing %%%%%%%%%%%%%%%%%%%%%%%%%%%%%%%%%%%%%%%%%%%%%%%%

\usepackage{fullpage}
\usepackage{setspace}
%\onehalfspacing
\usepackage{microtype}


%% Header %%%%%%%%%%%%%%%%%%%%%%%%%%%%%%%%%%%%%%%%%%%%%%%%%

%\pagestyle{fancy} 
%\lhead{}
%\rhead{}
%\chead{}
%\setlength{\headheight}{15.2pt} 
    %---Make the header bigger to avoid overlap

%\renewcommand{\headrulewidth}{0.3pt} 
    %---Width of the line

%\setlength{\headsep}{0.2in}    
    %---Distance from line to text
            

%% Mathematics Related %%%%%%%%%%%%%%%%%%%%%%%%%%%%%%%%%%%

\usepackage{amsmath}
\usepackage{amsfonts}
\usepackage{mathrsfs}
\usepackage{amsthm} %allows for labeling of theorems
\theoremstyle{plain}
\newtheorem{thm}{Theorem}[section]
\newtheorem{lem}[thm]{Lemma}
\newtheorem{prop}[thm]{Proposition}
\newtheorem{cor}[thm]{Corollary}

\theoremstyle{definition}
\newtheorem{defn}[thm]{Definition}
\newtheorem{ex}[thm]{Example}

\theoremstyle{remark}
\newtheorem*{rem}{Remark}
\newtheorem*{note}{Note}


%% Font Choices %%%%%%%%%%%%%%%%%%%%%%%%%%%%%%%%%%%%%%%%%

\usepackage[T1]{fontenc}
\usepackage[utf8]{inputenc}
\usepackage{lmodern}
%\usepackage{blindtext}


%% Figures %%%%%%%%%%%%%%%%%%%%%%%%%%%%%%%%%%%%%%%%%%%%%%

\usepackage{graphicx}
\usepackage{subfigure} 
    %---For plotting multiple figures at once
%\graphicspath{ {Directory/} }
    %---Set a directory for where to look for figures


%% Hyperlinks %%%%%%%%%%%%%%%%%%%%%%%%%%%%%%%%%%%%%%%%%%%%
\usepackage{hyperref} 
\hypersetup{	
    colorlinks,		
        %---This colors the links themselves, not boxes
    citecolor=black,	
        %---Everything here and below changes link colors
    filecolor=black,
    linkcolor=black,
    urlcolor=black
}

%% Including Code %%%%%%%%%%%%%%%%%%%%%%%%%%%%%%%%%%%%%%% 

\usepackage{verbatim} 
    %---For including verbatim code from files, no colors

\usepackage{listings}
\usepackage{color}
\definecolor{mygreen}{RGB}{28,172,0}
\definecolor{mylilas}{RGB}{170,55,241}
\newcommand{\matlabcode}[1]{%
    \lstset{language=Matlab,%
        basicstyle=\footnotesize,%
        breaklines=true,%
        morekeywords={matlab2tikz},%
        keywordstyle=\color{blue},%
        morekeywords=[2]{1}, keywordstyle=[2]{\color{black}},%
        identifierstyle=\color{black},%
        stringstyle=\color{mylilas},%
        commentstyle=\color{mygreen},%
        showstringspaces=false,%
            %---Without this there will be a symbol in 
            %---the places where there is a space
        numbers=left,%
        numberstyle={\tiny \color{black}},% 
            %---Size of the numbers
        numbersep=9pt,% 
            %---Defines how far the numbers are from the text
        emph=[1]{for,end,break,switch,case},emphstyle=[1]\color{red},%
            %---Some words to emphasise
    }%
    \lstinputlisting{#1}
}
    %---For including Matlab code from .m file with colors,
    %---line numbering, etc. 


%% Misc %%%%%%%%%%%%%%%%%%%%%%%%%%%%%%%%%%%%%%%%%%%%%% 

\usepackage{enumitem} 
    %---Has to do with enumeration	
\usepackage{appendix}
%\usepackage{natbib} 
    %---For bibliographies
\usepackage{pdfpages}
    %---For including whole pdf pages as a page in doc


%% User Defined %%%%%%%%%%%%%%%%%%%%%%%%%%%%%%%%%%%%%%%%%% 

%\newcommand{\nameofcmd}{Text to display}



%%%%%%%%%%%%%%%%%%%%%%%%%%%%%%%%%%%%%%%%%%%%%%%%%%%%%%%%%%%%%%%%%%%%%%%% 
%% BODY %%%%%%%%%%%%%%%%%%%%%%%%%%%%%%%%%%%%%%%%%%%%%%%%%%%%%%%%%%%%%%%%
%%%%%%%%%%%%%%%%%%%%%%%%%%%%%%%%%%%%%%%%%%%%%%%%%%%%%%%%%%%%%%%%%%%%%%%% 


\begin{document}

\maketitle 

\begin{enumerate}

\item 
\begin{enumerate}
\item \textbf{Exercise 57.7}: We want to show that $f$ continuous on $[a,b]$ with $f(x)=0$ almost everywhere on $[a,b]$ implies that $f(x)=0$ for all $x\in[a,b]$. We'll do so by contradiction.

So suppose that $f(x)=0$ almost everywhere, but there exists an $x\in[a,b]$ such that $f(x)\neq 0$. That means that 
\[ 
    |f(x)| > \varepsilon \qquad \text{for some $\varepsilon>0$}
\]
Then, using this $\varepsilon$, by the continuity of $f$, we know that there exists a $\delta>0$ such that
\[
    |x-y|<\delta \quad \Rightarrow \quad |f(x)-f(y)|<\varepsilon
\]
Now since this is the case, we can say that
\[
    y\in(x-\delta, x+\delta) \quad \Rightarrow \quad 
    f(y) \in \left(f(x)-\varepsilon, f(x)+\varepsilon\right)
\]
Supposing, without loss of generality, that $f(x) > 0$, we have that 
\[
    f(x) > \varepsilon \quad \Rightarrow \quad
    f(x) - \varepsilon > 0
\]
Thus, we can conclude that, $f(y) > 0$ for all $y\in(x-\delta, x+\delta)$.\footnote{A similar result holds if $f(x)<0$, just flip the direction of the inequality.}

\item \textbf{Exercise 57.8}: We want to show that if $\alpha\in BV([a,b])$, then $\alpha$ is continuous almost everywhere in $[a,b]$.  

Now if we want a cheap proof, we can say that $\alpha$ is of bounded variation, and so it's Riemann Integrable. In that case, by Lebesgue's Theorem, $\alpha$ is continuous almost everywhere, and we're done.

\emph{But}, that's kind of cheating. So let's use the hint, which says to prove that a monotonic function has at most countably many points of discontinuity. If we can prove this, then we're done. Here's the steps and results from class that motivate that statement:
\begin{enumerate}
    \item We know from class that our function $\alpha$ of bounded variation can be represented as the difference of monotonically increasing functions, $u$ and $v$.
    \item By the hint we will prove, each monotonic function will have at most countably many points of discontinuity. Call those sets of points $D_u$ and $D_v$.
    \item Since these two sets $D_u$ and $D_v$ are at most countable, each set will be of measure zero by a result we saw in class. 
    \item Since the total set of discontinuities for $f$ is the union $D_u \cup D_v$, we know that this finite union of at most countable sets will be at most countable, and so measure zero.
    \item Finally, we now have that the set of discountinuities of $f$ is measure zero, implying $f$ is continuous almost everywhere.
\end{enumerate}
Now let's prove that monontonic functions have at most countable many points of discontinuities, so that whole line of reasoning doesn't fall apart.

Let $A$ equal the set of discontinuities of $f$. Denote the righthand limit at $a\in A$ as $f(a^+)$, and lefthand limit as $f(a^-)$. Now since $f$ is monotonic, we know that $f(a^-) < f(a^+)$. Now because we're on the real line, we know that there exists a $q_a\in\mathbb{Q}$ such that
\[
    f(a^-) < q_a < f(a^+)   \qquad \forall a \in A
\]
And because $f$ is monotonic, each $q_a$ will be unique for each point of discontinuity. That's because if $b$ is a point of discontinuity greater than $a$, then $f(b^-)>f(a^+)$ by the monotonicity of $f$. 
\\
\\
And so this establishes a one to one correspondence between $A$ and $\mathbb{Q}$ (or a subset of $\mathbb{Q}$ if $A$ finite). So $A$ is countable.

\end{enumerate}

\item \textbf{Exercise 58.1}: Let $f\in\mathscr{R}([a,b])$ and suppose $\int^x_a f\;dx = 0$ for all $x\in(a,b]$. We want to show that $f(x)=0$ almost everywhere in $[a,b]$. We'll do this by contradiction, in a manner similar to Question (1a).

Start with the fact that $f\in\mathscr{R}([a,b])$. First, by the Fundamental Theorem of calculus, $F(x)=\int^x_a f\;dx$ is continuous and differentiable wherever $f$ is continuous. Then by Lebesgue's theorem, $f$ is continuous almost everywhere. 

So let $x^*$ be a point where $f$ is continuous in $(a,b]$. Then, because $f$ is continuous, we know that for any $\varepsilon>0$, there exists a $\delta>0$ such that 
\[
    |x^*-x|<\delta \quad \Rightarrow \quad |f(x^*)-f(x)|<\varepsilon
\]
Thus, we can write $F(x^*)$ as 
\[
    F(x^*) = \int^{x^*}_a f\;dx = 
\]
Use Lebesgue, think of question one, split up integral from $[a,x-\delta]$ and $[x-\delta, x]$, the latter half of which will be greater than 0.
    
\item 
\begin{enumerate}
\item 
Let $S$ equal the set of all finite unions of intervals in $\mathbb{R}$:
\[
    S = \left\{ \bigcup^n_{i=1} I_i \quad | \quad
    I_i \text{ is an interval in }  \mathbb{R} 
    \right\}
\]
First, we show that this is an algebra. To do so, we need to show that the properties of an algebra are satisfied:
\begin{itemize}
    \item To show $A\in{S}\Rightarrow A^c\in S$, write out $A$:
    \[ 
        A = \bigcup^n_{i=1} I_i  
    \]
    By DeMorgan's laws, we can rewrite
    \begin{align*}
        A^c &= \left(\bigcup^n_{i=1} I_i \right)^c 
        = \bigcap^n_{i=1} I^c_i  \\
    \end{align*}
    But for an interval 
    
\end{itemize}

\item To show that a union of two algebras need not be an algebra, consider the following two algebras:
\begin{align*}
    \mathscr{A}_1 &= \{ A \subset \mathbb{Z} \; | \;
        A \text{ or } A^c \text{ odd} \} \\
    \mathscr{A}_2 &= \{ A \subset \mathbb{Z} \; | \;| 
        A \text{ or } A^c \text{ even} \}
\end{align*}
Both are clearly algebras. First, a union of odd numbers will be odd, and the same for even. Second, by the criterion that the set \emph{or} it's complement is odd, we guarantee that $A\in\mathscr{A}_1$ implies $A^c\in\mathscr{A}_1$. Same for even.

Now, we can also easily show that $\mathscr{A}_1 \cup \mathscr{A}_2$ is not an algebra. We do this by taking the sets
\[
    A_1 \cup A_2 = \{1\} \cup \{2\} = \{1,2\}
    \qquad A_1 \subset \mathscr{A}_1
    \quad A_2 \subset \mathscr{A}_2
\]
Now since $\{1,2\}$ has both odd and even numbers (and so does its compliment) $A_1 \cup A_2$ will be in neither $\mathscr{A}_1 \cup \mathscr{A}_2$. Therefore, it won't be in $\mathscr{A}_1 \cup \mathscr{A}_2$, violating a basic property of algebras.

\end{enumerate}

\item 
\item 
\item 
\item Use Theorem 4.6.4 from the notes. Draw the function to get intuition.

For other direction, show contrapositive.


\end{enumerate}

\end{document}



%%%% INCLUDING FIGURES %%%%%%%%%%%%%%%%%%%%%%%%%%%%

   % H indicates here 
   %\begin{figure}[h!]
   %   \centering
   %   \includegraphics[scale=1]{file.pdf}
   %\end{figure}

%   \begin{figure}[h!]
%      \centering
%      \mbox{
%	 \subfigure{
%	    \includegraphics[scale=1]{file1.pdf}
%	 }\quad
%	 \subfigure{
%	    \includegraphics[scale=1]{file2.pdf} 
%	 }
%      }
%   \end{figure}
 

%%%%% Including Code %%%%%%%%%%%%%%%%%%%%%5
% \verbatiminput{file.ext}    % Includes verbatim text from the file
% \texttt{text}	  % includes text in courier, or code-like, font
