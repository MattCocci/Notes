\documentclass[12pt]{article}

\author{Matthew Cocci}
\title{\textbf{Analysis}}
\date{\today}

%% Spacing %%%%%%%%%%%%%%%%%%%%%%%%%%%%%%%%%%%%%%%%%%%%%%%%

\usepackage{fullpage}
\usepackage{setspace}
%\onehalfspacing
\usepackage{microtype}


%% Header %%%%%%%%%%%%%%%%%%%%%%%%%%%%%%%%%%%%%%%%%%%%%%%%%

%\pagestyle{fancy} 
%\lhead{}
%\rhead{}
%\chead{}
%\setlength{\headheight}{15.2pt} 
    %---Make the header bigger to avoid overlap

%\renewcommand{\headrulewidth}{0.3pt} 
    %---Width of the line

%\setlength{\headsep}{0.2in}    
    %---Distance from line to text
            

%% Mathematics Related %%%%%%%%%%%%%%%%%%%%%%%%%%%%%%%%%%%

\usepackage{amsmath}
\usepackage{amsfonts}
\usepackage{mathrsfs}
\usepackage{amsthm} %allows for labeling of theorems
\theoremstyle{plain}
\newtheorem{thm}{Theorem}[section]
\newtheorem{lem}[thm]{Lemma}
\newtheorem{prop}[thm]{Proposition}
\newtheorem{cor}[thm]{Corollary}

\theoremstyle{definition}
\newtheorem{defn}[thm]{Definition}
\newtheorem{ex}[thm]{Example}

\theoremstyle{remark}
\newtheorem*{rmk}{Remark}
\newtheorem*{note}{Note}


%% Font Choices %%%%%%%%%%%%%%%%%%%%%%%%%%%%%%%%%%%%%%%%%

\usepackage[T1]{fontenc}
\usepackage[utf8]{inputenc}
\usepackage{lmodern}
%\usepackage{blindtext}


%% Figures %%%%%%%%%%%%%%%%%%%%%%%%%%%%%%%%%%%%%%%%%%%%%%

\usepackage{graphicx}
\usepackage{subfigure} 
    %---For plotting multiple figures at once
%\graphicspath{ {Directory/} }
    %---Set a directory for where to look for figures


%% Hyperlinks %%%%%%%%%%%%%%%%%%%%%%%%%%%%%%%%%%%%%%%%%%%%
\usepackage{hyperref} 
\hypersetup{	
    colorlinks,		
        %---This colors the links themselves, not boxes
    citecolor=black,	
        %---Everything here and below changes link colors
    filecolor=black,
    linkcolor=black,
    urlcolor=black
}

%% Including Code %%%%%%%%%%%%%%%%%%%%%%%%%%%%%%%%%%%%%%% 

\usepackage{verbatim} 
    %---For including verbatim code from files, no colors

\usepackage{listings}
\usepackage{color}
\definecolor{mygreen}{RGB}{28,172,0}
\definecolor{mylilas}{RGB}{170,55,241}
\newcommand{\matlabcode}[1]{%
    \lstset{language=Matlab,%
        basicstyle=\footnotesize,%
        breaklines=true,%
        morekeywords={matlab2tikz},%
        keywordstyle=\color{blue},%
        morekeywords=[2]{1}, keywordstyle=[2]{\color{black}},%
        identifierstyle=\color{black},%
        stringstyle=\color{mylilas},%
        commentstyle=\color{mygreen},%
        showstringspaces=false,%
            %---Without this there will be a symbol in 
            %---the places where there is a space
        numbers=left,%
        numberstyle={\tiny \color{black}},% 
            %---Size of the numbers
        numbersep=9pt,% 
            %---Defines how far the numbers are from the text
        emph=[1]{for,end,break,switch,case},emphstyle=[1]\color{red},%
            %---Some words to emphasise
    }%
    \lstinputlisting{#1}
}
    %---For including Matlab code from .m file with colors,
    %---line numbering, etc. 


%% Misc %%%%%%%%%%%%%%%%%%%%%%%%%%%%%%%%%%%%%%%%%%%%%% 

\usepackage{enumitem} 
    %---Has to do with enumeration	
\usepackage{appendix}
%\usepackage{natbib} 
    %---For bibliographies
\usepackage{pdfpages}
    %---For including whole pdf pages as a page in doc


%% User Defined %%%%%%%%%%%%%%%%%%%%%%%%%%%%%%%%%%%%%%%%%% 

%\newcommand{\nameofcmd}{Text to display}



%%%%%%%%%%%%%%%%%%%%%%%%%%%%%%%%%%%%%%%%%%%%%%%%%%%%%%%%%%%%%%%%%%%%%%%% 
%% BODY %%%%%%%%%%%%%%%%%%%%%%%%%%%%%%%%%%%%%%%%%%%%%%%%%%%%%%%%%%%%%%%%
%%%%%%%%%%%%%%%%%%%%%%%%%%%%%%%%%%%%%%%%%%%%%%%%%%%%%%%%%%%%%%%%%%%%%%%% 


\begin{document}

\maketitle

\tableofcontents %adds it here

\section{The Riemann-Stieltjes Integral}

This definition of the integral was made rigorous in the 1800s by Riemann, Darboux, and Stieltjes.  It's an intuitive way to define the area under a curve, and it works well with numerical integration (approximations). \emph{However}, it is incomplete in the sense that there are functions of interest that we cannot integrate in a Riemann sense but can in a Lebesgue sense. 

Throughout this section, unless otherwise noted, we'll largely stick to bounded functions that are univariate from a compact interval to $\mathbb{R}$: 

    \[ f: [a, b] \rightarrow \mathbb{R} \]

    We'll begin by discussing \emph{partitions} of that interval $[a,b]$ into smaller pieces, from which we'll construct sums that approximate the area under the curve.  This will lead us to a definition of the Riemann Integral.  Then, we'll generalize and allow the \emph{weight} we place on the sub-intervals (when summing over the entire interval) to vary, which will give us the Riemann-Stieltjes integral. From there, we discuss the relationships between the approximating sums and the integral.

\subsection{Partitions}

\begin{defn} A \emph{partition}, $P$, is an ordered tuple representing a finite sequence on the interval $[a,b]$,
    \[ a = x_0 < x_1 < \cdots < x_n = b
        \qquad \text{with} \quad \Delta x_i := x_i - x_{i-1}
    \]
\end{defn}

\begin{defn} The \emph{norm} of a partition $P$, sometimes called ``mesh $P$'' represents
    \[ || P || = \text{norm}(P) := \max_i |x_i - x_{i-1}| =
        \max_i |\Delta x_i| \] 
\end{defn}

\begin{defn} $Q$ is a \emph{refinement} of $P$ if $Q \supset P$ where $Q$ and $P$ are both partitions of $[a,b]$. $Q$ the intervals \emph{finer}.
\end{defn}

\begin{defn} For two partitions, $P_1$ and $P_2$, their \emph{common refinement} is $P_1 \cup P_2$.
\end{defn}

\begin{defn} A \emph{tagged partition} is a couplet $(P,T)$, where $P$ is some partition $\{x_0, \ldots, x_n\}$ and $T$ is a set of evaluation points, $\{t_1, \ldots, t_n\}$, for the function $f$  such that
    \[ x_{i-1} \leq t_i \leq x_i \]
\end{defn}

\begin{note} We will now generalize to allow weighting of the sub-intervals within the partition, defined for an \emph{increasing} function $\alpha: [a,b] \rightarrow \mathbb{R}$, where 
    \[ \Delta \alpha_i = \alpha(x_i) - \alpha(x_{i-1}) > 0 \]
This is the main difference between the plain Riemann sum and integral, versus the Riemann-Stieltjes (RS) sum and integral.  The latter retains the former as a special case by taking $\alpha(x) = x$.  Therefore, the RS version is just a generalization of Riemann, weighting the contribution of the sub-intervals to the total sum/integral by the function $\alpha$, \emph{not} by the length of the sub-interval.
\end{note}

\subsection{Sum Definitions}

We now define the various sums approximating the Riemann and RS integrals.

\begin{defn} We define the upper and lower \emph{Darboux Sums}, respectively, as follows 
    \begin{align*}
        U(f,P) &:= \sum^n_{i=1} M_i(f)(x_i - x_{i-1}) 
            \quad\text{where} \quad 
            M_i(f) := \sup_{x \in [x_i, x_{i-1}]} f(x)\\
        L(f,P) &:= \sum^n_{i=1} m_i(f)(x_i - x_{i-1})
            \quad\text{where} \quad 
            m_i(f) := \inf_{x \in [x_i, x_{i-1}]} f(x) 
    \end{align*}
\end{defn}

\begin{defn} Given $f$ (bounded) and tagged partition $(P,T)$ we define the \emph{Riemann Sum} as 
    \begin{equation}
        S(f,P,T) := \sum^n_{i=1} f(t_i) (x_i - x_{i-1})
    \end{equation}
\end{defn}

\begin{defn} 
\label{RSD}
We define the upper and lower \emph{RS-Darboux Sums}, respectively, as follows 
    \begin{align*}
        U_\alpha(f,P) &:= \sum^n_{i=1} M_i(f) \Delta \alpha_i
            \quad\text{where} \quad 
            M_i(f) := \sup_{x \in [x_i, x_{i-1}]} f(x) \\
        L_\alpha(f,P) &:= \sum^n_{i=1} m_i(f)\Delta \alpha_i
            \quad\text{where} \quad 
            m_i(f) := \inf_{x \in [x_i, x_{i-1}]} f(x) 
    \end{align*}
\end{defn}

\begin{defn} 
\label{RSS}
Given $f$ (bounded) and tagged partition $(P,T)$ we define the \emph{Riemann-Stieltjes Sum} as 
    \begin{equation}
        S_\alpha(f,P,T) := \sum^n_{i=1} f(t_i) \Delta \alpha_i
    \end{equation}
\end{defn}

\subsection{Sum Relations}

\begin{rmk} Clearly, by Definitions \ref{RSD} and \ref{RSS}, for all $T$ associated with $P$
    \[ L_\alpha(f,P) \leq S_\alpha(f,P,T) \leq U_\alpha(f,P) \]
\end{rmk}

\begin{thm} 
\label{sumineq}
    If $Q \supset P$, i.e. if $Q$ refines $P$, then
    \[ L_\alpha(f,P) \leq L_\alpha(f,Q) \leq U_\alpha(f,Q) 
        \leq U_\alpha(f,P) \]
\end{thm}
\begin{proof} The proof proceeds by induction. Assume that $Q = P \cup \{x^*\}$, a single point. Then $x^*\in [x_{i-1}, x_i]$ for some interval, and it's easy show the relation from there.
\end{proof}

\begin{thm} 
\label{pineq}
    For all partitions $P_1, P_2$, 
    \[ L_\alpha(f,P_1) \leq U_\alpha(f,P_2) \]
\end{thm}
\begin{proof} Let $Q = P_1 \cup P_2$. Then by Theorem \ref{sumineq}, 
    \[ L_\alpha(f,P_1) \leq L_\alpha(f,Q) \leq U_\alpha(f,Q) 
        \leq U_\alpha(f,P_2) \]
\end{proof}


\subsection{Definition of $\mathscr{R}_\alpha([a,b])$}


\begin{defn} We define \emph{the upper and lower Riemann-Stieltjes integrals}, respectively, in terms of the RS-Darboux Sums
    \begin{align*} 
        \overline{\int^b_a} f d\alpha &:= \inf_P U_\alpha(f,P) \\
        \underline{\int^b_a} f d\alpha &:= \sup_P L_\alpha(f,P)
    \end{align*}
We can assert that the sup and inf exist because $f$ is bounded. From Theorem \ref{pineq}, it's clear that $\underline{\int} f d\alpha \leq \overline{\int} f d\alpha$.
\\
\\
\emph{Note}: This means the value of the upper (lower) integral might not be \emph{in} the set of upper (lower) sums, just like a real number will not be in a sequence of approaching rationals. 
\end{defn}

\begin{defn} We say  $f$ is \emph{Riemann-Stieltjes integrable} on $[a,b]$---i.e. $f \in \mathscr{R}_\alpha([a,b])$---if 
    \[ \overline{\int^b_a} f d\alpha  =
        \underline{\int^b_a} f d\alpha 
        := {\int^b_a} f d\alpha
        \]
\end{defn}

\begin{ex} A case where $f \notin \mathscr{R}_\alpha([a,b])$ is where 
    \[ f(x) = \begin{cases} 1 & $x$\text{ rational} \\
            0 & $x$\text{ irrational} \end{cases}\]
for $x\in[0,1]$. In this case, the upper integral is always 1, while the lower integral is always zero.
\end{ex}

\begin{thm}
\emph{(Riemann's Condition)}
\label{riemcond}
$f \in \mathscr{R}_\alpha([a,b])$ if and only if there exists a partition $P$ such that the upper and lower RS-Darboux sums can be made arbitrarily close given that $P$, i.e.
    \[  U_\alpha(f,P) - L_\alpha(f,P) \leq \varepsilon \]
\end{thm}
\begin{proof} First, the $\Leftarrow$ direction. Use Theorems \ref{sumineq} and \ref{pineq}. It's obvious. Next, for the $\Rightarrow$ direction. By the definition of the RS integral and the RS-Darboux sums, 
\begin{equation}
    \label{p1}
    U_\alpha(f,P_1) < \int^b_a f d\alpha + \varepsilon/2 \qquad
    L_\alpha(f,P_2) > \int^b_a f d\alpha - \varepsilon/2
\end{equation}
Taking the common refinement, and using Theorem \ref{sumineq}, we get that 
\begin{align*}
    U_\alpha(f,P_1 \cup P_2) - L_\alpha(f,P_1 \cup P_2) &\leq 
    U_\alpha(f,P_1) - L_\alpha(f,P_2) \\
    &= \left(\int^b_a f d\alpha + \varepsilon/2\right) - 
        \left(\int^b_a f d\alpha -\varepsilon/2  \right) \\
    \text{By Expression \ref{p1}}\qquad &\leq \varepsilon/2 + \varepsilon/2
\end{align*}
\end{proof}

\subsection{Properties of $\mathscr{R}_\alpha([a,b])$}

\begin{thm} The set of all continuous functions on $[a,b]$, denoted ${C}([a,b])$, is a subset of $\mathscr{R}([a,b])$.
\end{thm}
\begin{proof} 
By Theorem \ref{riemcond}, we want to show that, for all $\epsilon>0$, there exists a partition $P$ such that 
\begin{align*}
    U_\alpha(f,P) - L_\alpha(f,P) < \epsilon  \\
    \Leftrightarrow
    \sum^n_{i=1} (M_i(f) - m_i(f)) \Delta\alpha_i < \epsilon 
\end{align*}
Now since $f$ is continuous on a compact interval, $[a,b]$, $f$ is \emph{uniformly continuous} on $[a,b]$. That means, given our $\epsilon$ from above, 
    \[ \exists \; \delta >0 \quad \text{ s.t. } \quad
        |x_{i} - x_{i-1}| < \delta \quad \Rightarrow \quad
        |f(x_{i}) - f(x_{i-1})| < \frac{\epsilon}{\alpha(b)-\alpha(a)} \]
So we can choose $P$ such that that $||P|| < \delta$.  This means that 
\begin{align*}
    \sum^n_{i=1} [M_i(f) - m_i(f)] \Delta\alpha_i &\leq
    \sum^n_{i=1} \frac{\epsilon}{\alpha(b)-\alpha(a)} 
        \Delta\alpha_i  
    =\frac{\epsilon}{\alpha(b)-\alpha(a)}\sum^n_{i=1}  
        \Delta\alpha_i \\
    &\leq\frac{\epsilon}{\alpha(b)-\alpha(a)} \cdot [\alpha(b)-\alpha(a)] = \epsilon
\end{align*}
\end{proof}
Now for some useful properties of the set of Riemann-Stieltjes integrable functions. Consider $f,g \in \mathscr{R}_\alpha([a,b])$
and $c \in \mathbb{R}$.
\begin{itemize}
    \item \textbf{Linearity}: $f+g \in \mathscr{R}_\alpha([a,b])$
        and $cf \in \mathscr{R}_\alpha([a,b])$, with 
        \[ \int^b_a cf \; d\alpha = c \int^b_a f\; d\alpha  
            \qquad \text{and} \qquad 
            \int^b_a f+g\; d\alpha = \int^b_a f \;d\alpha + 
            \int^b_a g\; d\alpha 
        \]
    \item \textbf{Subsets}: If $[c,d]\subset[a,b]$, then $f \in \mathscr{R}_\alpha([c,d])$.
    \item \textbf{Splitting the Interval}: If $c \in [a,b]$, then 
        \[ \int^b_a f d\alpha = \int^c_a f d\alpha 
            + \int^b_c f d\alpha \]
    \item \textbf{Monotonicity}: If $f,g,h \in 
        \mathscr{R}_\alpha([a,b])$, $f\geq0$, and
        $g \leq h$ on $[a,b]$, then
            \[ \int^b_a f d\alpha \geq 0 \qquad
                \int^b_a g \; d\alpha \leq 
                \int^b_a h \; d\alpha  \]
    \item \textbf{Compositions}: Suppose that $f \in 
        \mathscr{R}_\alpha([a,b])$ and
        $g: \mathbb{R}\rightarrow\mathbb{R}$ is 
        continuous. Then $g \circ f\in\mathscr{R}_\alpha([a,b])$
        \begin{proof}
            Since $f$ is bounded, 
            let $m:=\inf_{[a,b]}f$ and $M:=\sup_{[a,b]}f$.
            Then since $g$ is continuous on the on the compact 
            interval $[m,M]$, $g$ is uniformly continuous.
            As a result, for all $\varepsilon>0$
            there then exists $\delta>0$ such that 
            \begin{equation}
                \label{cont}
                 |u- v| \leq \delta \quad \Rightarrow
                    \quad |g(u) - g(v)| \leq \varepsilon
            \end{equation}
            Finally, $g$ also happens to be bounded because it
            continuous on a compact interval, thus $|g(x)| \leq K$
            for some $K$ and all $x$.
            \\
            \\
            Next, since we know that $f \in 
            \mathscr{R}_\alpha([a,b])$, 
            there exists a partition $P$ such that
            \begin{equation}
                \label{forf}
                 U_\alpha(f,P)- L_\alpha(f,P) \leq \epsilon\cdot
                    \delta
            \end{equation}
            And, considering the building blocks $M_i(f)$ and 
            $m_i(f)$, by the uniform continuity of $g$, we
            also have for any $i$, 
            \begin{equation}
                \label{compcont}
                M_i(f) - m_i(f) < \delta \quad \Rightarrow \quad 
                    M_i(g\circ f) - m_i(g\circ f) \leq \epsilon
            \end{equation}
            Now with everything finally spelled out, 
            we're interested in showing Riemann 
            integrability for $g\circ f$, which requires us
            to show that the upper and lower RS-Darboux sums
            are arbitrarily close. We will do so by breaking
            the sums into two parts based on the following
            characteristics of the original function $f$:
            \begin{align*}
                A = \{ i \; | \; M_i(f) - m_i(f)\leq\delta\}\\
                B = \{ i \; | \; M_i(f) - m_i(f)>\delta\}
            \end{align*}
            Now let's work out the sums for $g\circ f$:
            \begin{align}
                U_\alpha(g\circ f,P) - L_\alpha(g\circ f,P)
                    &= \sum^n_{i=1} \left[
                    M_i(g\circ f) - m_i(g\circ f)\right]
                    \Delta\alpha_i \notag\\
                &= \sum_{i\in A} \quad + \quad
                    \sum_{i\in B} \notag\\
                \text{By Equation \ref{compcont}} \qquad
                &\leq \varepsilon \sum_{i\in A} \Delta\alpha_i+
                    K \sum_{i\in B} \Delta\alpha_i
                    \label{final}
            \end{align}
            where the result from the sum for $B$ comes from $g$
            being bounded.
            \\
            \\
            Now let's consider the righthand term from the last
            line, and relate it back to Expression \ref{forf}.
            We know that by our definition of $B$:
            \begin{align*}
                \delta \sum_{i\in B} \Delta\alpha_i &\leq    
                \sum_{i\in B} \left[M_i(f) - m_i(f) \right]
                \Delta\alpha_i \leq \varepsilon\delta \\
                &\Rightarrow \qquad \sum_{i\in B} \Delta\alpha_i
                    \leq \epsilon
            \end{align*}
                
        TO FINISH
        \end{proof}
    \item \textbf{Multiplication}: If we have 
        $f,g\in\mathscr{R}_\alpha([a,b])$, then 
        $fg\in\mathscr{R}_\alpha([a,b])$.
        \begin{proof}
            We know that $f - g$ and $f+g \in
            \mathscr{R}_\alpha([a,b])$. We also know that 
            $h(x) = x^2$ is continuous, so then we can say that
            \[ fg = \frac{1}{4} \left[(f+g)^2 - (f-g)^2 \right]
                \in\mathscr{R}_\alpha([a,b])
                \]
        \end{proof}
    \item \textbf{Absolute Value Relations}: If we have
        $f \in \mathscr{R}_\alpha([a,b])$, then both
        \[ |f| \in \mathscr{R}_\alpha([a,b]) \qquad
            \left\lvert\int^b_a f d\alpha \right\rvert
            \leq \int^b_a |f| \; d\alpha \]
        \begin{proof}
            The first expression is clearly true because $g(x)=|x|$
            is a continuous function, so $g\circ f = |f|$ is
            RS integrable.
            \\
            \\
            Next, let $c$ be either 1 or $-1$ such that 
                \[ c\int^b_a f\; d\alpha = \left\lvert 
                \int^b_a f\;d\alpha \right\rvert \]
            Then we have that $cf \leq |f|$ and by the monotonicity
            property, we have that
            \begin{align*}
                \int^b_a |f| \; d\alpha &\geq 
                    \int^b_a cf \; d\alpha \\
                \Rightarrow \qquad 
                \int^b_a |f| \; d\alpha &\geq
                    \int^b_a cf \; d\alpha =
                    c \int^b_a f \; d\alpha = 
                    \left\lvert
                    \int^b_a f \; d\alpha \right\rvert 
            \end{align*}
        \end{proof}
        
\end{itemize}

\newpage
\subsection{Mean Value Theorem}

The classical Mean-Value Theorem states that if $f$ is continous on $[a,b]$ with $\alpha$ increasing, there's some $c\in[a,b]$ such that 
    \[ \int^b_a f\;d\alpha = f(c)[\alpha(b)-\alpha(a)] \]
We'll prove a more general result that retains this as a special case.
\begin{thm}
\label{mvt}
Suppose that $f,g\in\mathscr{R}_\alpha([a,b])$ and assume $g\geq0$. Then there exists a $\mu\in[m,M]$, i.e. in between the upper and lower bounds of $f$, such that 
        \[ \int^b_a fg\;d\alpha = \mu \int^b_a g\;d\alpha \]
Moreover, if $f$ is continous, then there exists a $c\in[a,b]$ such that $\mu = f(c)$.
\end{thm}
\begin{proof}
Since $g\geq0$, we can conclude that 
\begin{align*}
    m &\leq f \leq M \\
    mg &\leq fg \leq Mg \\
    \Rightarrow \qquad m\int^b_a g\;d\alpha &\leq 
        \int^b_a fg\;d\alpha \leq  M \int^b_a g\;d\alpha
\end{align*}
If we consider the case where $\int^b_a g\;d\alpha= 0$, then any old $\mu$ will work since we'll have that $\int^b_a fg\;d\alpha=0$. And if it's not 0, then we can set 
\[ \mu = \frac{\int^b_a fg\;d\alpha}{\int^b_a g\;d\alpha} \]
We know the numerator and denominator exist because both $f$ and $g$ were assumed integrable, so their product is as well.
\end{proof}
\begin{rmk}
If we take $g(x)=1$, then we get the plain vanilla MVT we opened up the subsection with.  And in the case where $f$ is continuous, we know from the intermediate value theorem that there exists a $c$ such that $\mu=f(c)$. 
\end{rmk}


\newpage
\subsection{Convergence of Riemann-Stieltjes Sums, $S_\alpha(f,P,T)$}

Basic calculus makes us all think that we approximate a Riemann-Sieltjes integral by taking finer and finer partitions, which we use to weight the value of the function at an increasing number of points.  So it would be very nice and very convenient if Riemann-Stieltjes sums converged to the value of the integral whenever we construct just that sort of limiting sequnce. 

\emph{However}, this will not always be the case.  That is, there will be some functions that are RS-integrable, which won't \emph{necessarily} converge if you take finer and finer partitions. The theorems in this section will illustrate when that will and won't apply. In certain cases, arbitrarily fine partitions might converge to different values, none of which must equal the value of the integral (if it even exists).

\begin{thm}
\label{weaker}
Let $f$ be a bounded function on $[a,b]$. Then $f\in\mathscr{R}_\alpha([a,b])$ if and only if there exists a number $I$such that for every $\varepsilon>0$, there exists a corresponding partition $P$ such that for all $P^* \supset P$, we have that
    \[ |S_\alpha(f,P*,T) - I | \leq \varepsilon \qquad \forall T\]
And if $f\in\mathscr{R}_\alpha([a,b])$, then $I = \int^b_a f\;d\alpha$.
\end{thm}
\begin{rmk}
We could have written a weaker theorem, leaving out the part about $P^*\supset P$, which is a special case of the above theorem.  But going forward, we'll want to consider the idea of taking finer and finer partitions. This will be particularly true when $\alpha$ might not be increasing.
\end{rmk}
\begin{proof}
First, we prove the $\Rightarrow$ direction. Since $f\in\mathscr{R}_\alpha([a,b])$, we know that there's a partition $P$ such that
    \[ U_\alpha(f,P) - L_\alpha(f,P)\varepsilon \qquad 
    \forall \varepsilon \]
Considering $P^*\supset P$, we also have
\begin{align*}
    L_\alpha(f,P^*)&\leq S_\alpha(f,P^*,T) \leq U_\alpha(f,P^*)\\
    L_\alpha(f,P^*)&\leq \int^b_a f\;d\alpha \leq U_\alpha(f,P^*)
\end{align*}
Thus, we can conclude that 
\begin{align*}
    \left\lvert S_\alpha(f,P^*,T) - \int^b_a f\;d\alpha 
    \right\rvert \leq \varepsilon \qquad \forall T
\end{align*}
It's clear that $I=\int^b_a f\;d\alpha$,
\\
\\
Next, we want to show the $\Leftarrow$ direction. We begin by noting that given any $P$, there exists a $T_1$ and $T_2$ such that
\begin{align}
    S_\alpha(f,P,T_1) &\geq U_\alpha(f,P) - \varepsilon\notag\\
    S_\alpha(f,P,T_2) &\leq L_\alpha(f,P) + \varepsilon
    \label{proof1.17}
\end{align}
This works because given any partition, we can always choose a $T_1$ and $T_2$ to satisfy by taking $T_1$ ($T_2$) as the set of all $M_i$ ($m_i$). 
\\
\\
So now, let's take $P^*$ to be the partition such that the RS sum is $\varepsilon$ close to $I$ for all $T$; isnote, this $P^*$ is taken to exist in this direction of the proof. Then we have from the inequalities from \ref{proof1.17} that
\begin{align*}
    U_\alpha(f,P) - L_\alpha(f,P) &\leq 
        \left[ S_\alpha(f,P,T_1) + \varepsilon\right] -
        \left[ S_\alpha(f,P,T_2) - \varepsilon\right] \\
    &\leq \left[ S_\alpha(f,P,T_1) - I + \varepsilon\right] -
        \left[ S_\alpha(f,P,T_2) - I - \varepsilon\right] \\
    &\leq \left\lvert S_\alpha(f,P,T_1) - I \right\rvert+       
        \left\lvert S_\alpha(f,P,T_2) - I \right\rvert
        + \varepsilon + \varepsilon \\
    \text{From Inequalities \ref{proof1.17}} \quad
        &\leq 4\varepsilon
\end{align*}
Since the Upper and Lower sums can be made arbitrarily close, we then know that $f\in\mathscr{R}_\alpha([a,b])$.
\end{proof}
\begin{rmk}
Before moving on, let's recap what the last theorem said in words---or at least what it didn't say.  It didn't say that if you keep taking smaller partitions, you will \emph{always} force your RS-Sum to converge to the value of an integral.  What it did say was twofold:
\begin{enumerate}
    \item First, if your function \emph{is} Riemann-Integrable, then and you can find \emph{some} partition (though it might not be true for all of them) where the RS-Sums given that partition and \emph{all finer} partitions get arbitrarily close to this integral that exists.
    \item Next, suppose you know an $I$ that the sums get arbitrarily close to.  More specifically, that means if you're handed any $\varepsilon>0$, you can find a partition that gets the RS-Sums arbitrarily close to that $I$, no matter the $T$ tagging the partition $P$. Now we're not saying that the sums \emph{must} converge to some $I$ when you take finer partitions. (We only kind of have that if we consider the other direction of the proof where $f\in\mathscr{R}_\alpha([a,b])$ is assumed). Instead, we're just saying it's possible, and if $I$ \emph{does} exist, $f$ is RS integrable.
\end{enumerate}
Now we can move on to consider that thorny case of ever-shrinking paritions.
\end{rmk}
\begin{defn}
\label{strongest}
For a bounded function $f$ on $[a,b]$, we say 
    \[ \lim_{||P||\rightarrow 0} S_\alpha(f,P,T) = I \]
if for every $\varepsilon>0$, there exists a $\delta>0$ such that $||P||\leq\delta$ implies that
    \[ |S_\alpha(f,P,T) - I | \leq \varepsilon \qquad \forall T \]
    Now here's the big difference between this and what we've seen earlier: the sums might converge, but I never said anything about the equivalence of this statement to RS-integrability, like we had in Theorem \ref{weaker}.  In fact, we can't.
\\
\\
There might be functions where this condition does \emph{not} hold (i.e. the limit does not exist), but still the function is integrable. (See the next example.) But we do have at least one direction intact, as shown in the next theorem.
\end{defn}
\begin{thm}
\label{oneway}
Suppose that $f$ is bounded on $[a,b]$. If $\lim_{||P||\rightarrow 0}$ exists, then $f\in\mathscr{R}_\alpha([a,b])$ and
    \[ \int^b_a f\;d\alpha = \lim_{||P||\rightarrow 0} 
    S_\alpha(f,P,T) \]
\end{thm}
\begin{proof}
\end{proof}
\begin{ex}
Now we give an example of why the other direction in Theorem \ref{oneway} wouldn't work.  Consider
\[ f(x) = \begin{cases} 0 & x \in [0, 1/2) \\ 1 & x \in [1/2, 1] 
        \end{cases}  \qquad
    \alpha(x) = \begin{cases} 0 & x \in [0,1/2] \\ 1& x\in(1/2, 1] 
        \end{cases} \]
Depending on whether the point $1/2$ is included in the partition, we can construct ever-finer partitions that make the RS-Sums converge to 0 sometimes, and to 1 at other times.
\end{ex}
\begin{rmk}
Thus we have that the strongest statement is in Definition \ref{strongest}. If we can show the limit exists as $||P||\rightarrow 0$, then we can show that $f$ is integrable. If we can't use that, we should think a little harder and try to find a $P$ as in Theorem \ref{weaker} before concluding that $f$ isn't RS-integrable. However, if we're will to put some tougher conditions on our function $f$, we can get more powerful results, as in the next theorem.
\end{rmk}
\begin{thm}
Let $f\in\mathscr{R}_\alpha([a,b])$. If either $f$ or $\alpha$ is continuous on $[a,b]$, then 
    \[ \int^b_a f\;d\alpha = \lim_{||P||\rightarrow 0} 
    S_\alpha(f,P,T) \]
\end{thm}
\begin{rmk}
This previous definition is important, because if we just consider \emph{Riemann} integrability, then $\alpha(x) = x$, which is continuous. Thus, Riemann-integrability and the existence of the limit of Riemann sums are equivalent notions.
\end{rmk}
\begin{proof}
    TO FINISH
\end{proof}








\newpage
\subsection{Integration for Arbitrary Integrators, $\alpha$}

\newpage
\subsection{Integration by Parts}

\newpage
\subsection{Improper Integrals}

\newpage
\section{To Do Yet}
Finish up the proof for compositions


%%%% APPPENDIX %%%%%%%%%%%

\newpage
\appendix
\section{Additional Definitions}

Modulus of Continuity


%\cite{LabelInSourcesFile} 
%\citep{LabelInSourcesFile} Cites in parens
%\nocite{LabelInSourceFile} includes in refs w/o specific citation
%\bibliographystyle{apalike} 
%\bibliography{sources.bib} where sources.bib is file




\end{document}



%%%% INCLUDING FIGURES %%%%%%%%%%%%%%%%%%%%%%%%%%%%

   % H indicates here 
   %\begin{figure}[h!]
   %   \centering
   %   \includegraphics[scale=1]{file.pdf}
   %\end{figure}

%   \begin{figure}[h!]
%      \centering
%      \mbox{
%	 \subfigure{
%	    \includegraphics[scale=1]{file1.pdf}
%	 }\quad
%	 \subfigure{
%	    \includegraphics[scale=1]{file2.pdf} 
%	 }
%      }
%   \end{figure}
 

%%%%% Including Code %%%%%%%%%%%%%%%%%%%%%5
% \verbatiminput{file.ext}    % Includes verbatim text from the file
% \texttt{text}	  % includes text in courier, or code-like, font
