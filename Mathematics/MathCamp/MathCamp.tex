\documentclass[12pt]{article}

\author{Matthew D. Cocci}
\title{Math Camp}
\date{\today}

%% Formatting & Spacing %%%%%%%%%%%%%%%%%%%%%%%%%%%%%%%%%%%%

%\usepackage[top=1in, bottom=1in, left=1in, right=1in]{geometry} % most detailed page formatting control
\usepackage{fullpage} % Simpler than using the geometry package; std effect
\usepackage{setspace}
%\onehalfspacing
\usepackage{microtype}

%% Formatting %%%%%%%%%%%%%%%%%%%%%%%%%%%%%%%%%%%%%%%%%%%%%

%\usepackage[margin=1in]{geometry}
    %   Adjust the margins with geometry package
%\usepackage{pdflscape}
    %   Allows landscape pages
%\usepackage{layout}
    %   Allows plotting of picture of formatting



%% Header %%%%%%%%%%%%%%%%%%%%%%%%%%%%%%%%%%%%%%%%%%%%%%%%%

%\usepackage{fancyhdr}
%\pagestyle{fancy}
%\lhead{}
%\rhead{}
%\chead{}
%\setlength{\headheight}{15.2pt}
    %   Make the header bigger to avoid overlap

%\fancyhf{}
    %   Erase header settings

%\renewcommand{\headrulewidth}{0.3pt}
    %   Width of the line

%\setlength{\headsep}{0.2in}
    %   Distance from line to text


%% Mathematics Related %%%%%%%%%%%%%%%%%%%%%%%%%%%%%%%%%%%

\usepackage{amsmath}
\usepackage{amsfonts}
\usepackage{mathrsfs}
\usepackage{amsthm} %allows for labeling of theorems
\theoremstyle{plain}
\newtheorem{thm}{Theorem}[section]
\newtheorem{lem}[thm]{Lemma}
\newtheorem{prop}[thm]{Proposition}
\newtheorem{cor}[thm]{Corollary}

\theoremstyle{definition}
\newtheorem{ax}[thm]{Axiom}
\newtheorem{defn}[thm]{Definition}
\newtheorem{ex}[thm]{Example}

\theoremstyle{remark}
\newtheorem*{rmk}{Remark}
\newtheorem*{note}{Note}

% Below supports left-right alignment in matrices so the negative
% signs don't look bad
\makeatletter
\renewcommand*\env@matrix[1][c]{\hskip -\arraycolsep
  \let\@ifnextchar\new@ifnextchar
  \array{*\c@MaxMatrixCols #1}}
\makeatother

% Make it impossible to break inline math
\relpenalty=10000
\binoppenalty=10000

%% Font Choices %%%%%%%%%%%%%%%%%%%%%%%%%%%%%%%%%%%%%%%%%

\usepackage[T1]{fontenc}
\usepackage{lmodern}
\usepackage[utf8]{inputenc}
%\usepackage{blindtext}
\usepackage{courier}


%% Figures %%%%%%%%%%%%%%%%%%%%%%%%%%%%%%%%%%%%%%%%%%%%%%

\usepackage{graphicx}
\usepackage{subfigure}
    %   For plotting multiple figures at once
%\graphicspath{ {Directory/} }
    %   Set a directory for where to look for figures


%% Hyperlinks %%%%%%%%%%%%%%%%%%%%%%%%%%%%%%%%%%%%%%%%%%%%
\usepackage{hyperref}
\hypersetup{%
    colorlinks,
        %   This colors the links themselves, not boxes
    citecolor=black,
        %   Everything here and below changes link colors
    filecolor=black,
    linkcolor=black,
    urlcolor=black
}

%% Colors %%%%%%%%%%%%%%%%%%%%%%%%%%%%%%%%%%%%%%%%%%%%%%%

\usepackage{color}
\definecolor{codegreen}{RGB}{28,172,0}
\definecolor{codelilas}{RGB}{170,55,241}


%% Including Code %%%%%%%%%%%%%%%%%%%%%%%%%%%%%%%%%%%%%%%

\usepackage{verbatim}
    %   For including verbatim code from files, no colors
\usepackage{listings}
    %   For including code snippets written directly in this doc

\lstdefinestyle{bash}{%
  language=bash,%
  basicstyle=\footnotesize\ttfamily,%
  showstringspaces=false,%
  commentstyle=\color{gray},%
  keywordstyle=\color{blue},%
  xleftmargin=0.25in,%
  xrightmargin=0.25in
}

\lstdefinestyle{matlab}{%
  language=Matlab,%
  basicstyle=\footnotesize\ttfamily,%
  breaklines=true,%
  morekeywords={matlab2tikz},%
  keywordstyle=\color{blue},%
  morekeywords=[2]{1}, keywordstyle=[2]{\color{black}},%
  identifierstyle=\color{black},%
  stringstyle=\color{codelilas},%
  commentstyle=\color{codegreen},%
  showstringspaces=false,%
    %   Without this there will be a symbol in
    %   the places where there is a space
  numbers=left,%
  numberstyle={\tiny \color{black}},%
    %   Size of the numbers
  numbersep=9pt,%
    %   Defines how far the numbers are from the text
  emph=[1]{for,end,break,switch,case},emphstyle=[1]\color{red},%
    %   Some words to emphasise
}

\newcommand{\matlabcode}[1]{%
    \lstset{style=matlab}%
    \lstinputlisting{#1}
}
    %   For including Matlab code from .m file with colors,
    %   line numbering, etc.

%% Bibliographies %%%%%%%%%%%%%%%%%%%%%%%%%%%%%%%%%%%%

%\usepackage{natbib}
    %---For bibliographies
%\setlength{\bibsep}{3pt} % Set how far apart bibentries are

%% Misc %%%%%%%%%%%%%%%%%%%%%%%%%%%%%%%%%%%%%%%%%%%%%%

\usepackage{enumitem}
    %   Has to do with enumeration
\usepackage{appendix}
%\usepackage{natbib}
    %   For bibliographies
\usepackage{pdfpages}
    %   For including whole pdf pages as a page in doc


%% User Defined %%%%%%%%%%%%%%%%%%%%%%%%%%%%%%%%%%%%%%%%%%

%\newcommand{\nameofcmd}{Text to display}
\newcommand*{\Chi}{\mbox{\large$\chi$}} %big chi
    %   Bigger Chi

% In math mode, Use this instead of \munderbar, since that changes the
% font from math to regular
\makeatletter
\def\munderbar#1{\underline{\sbox\tw@{$#1$}\dp\tw@\z@\box\tw@}}
\makeatother

% My additions for math shorthand
\newcommand{\limn}{\lim_{n\rightarrow\infty}}
\newcommand{\R}{\mathbb{R}}
\newcommand{\E}{\mathbb{E}}
\newcommand{\N}{\mathbb{N}}



%%%%%%%%%%%%%%%%%%%%%%%%%%%%%%%%%%%%%%%%%%%%%%%%%%%%%%%%%%%%%%%%%%%%%%%%
%% BODY %%%%%%%%%%%%%%%%%%%%%%%%%%%%%%%%%%%%%%%%%%%%%%%%%%%%%%%%%%%%%%%%
%%%%%%%%%%%%%%%%%%%%%%%%%%%%%%%%%%%%%%%%%%%%%%%%%%%%%%%%%%%%%%%%%%%%%%%%


\begin{document}
\maketitle

\tableofcontents

\clearpage
\section{Basic Building Blocks}

\begin{defn}{(Supremum)}
\label{defn:supdef1}
Given a set $A\subseteq X$, the \emph{supremum} denoted $\sup(A)$ is the
smallest possible upper bound for $A$.
More precisely, the supremum satisfies the following two properties
\begin{enumerate}
  \item $a\leq \sup(A)$ for all $a\in A$.
  \item If $x<\sup(A)$, then $x\in X$ is not an upper bound for $A$.
    Otherwise, we would have just taken $x$ as the supremum.
\end{enumerate}
It might be in $A$, in which case $\sup A = \max A$, or it might be
$X\setminus A$, in which case $\max A$ is undefined.
\end{defn}

\begin{ax}{(Axiom of Completeness)}
\label{ax:completeness}
Suppose that $A\subset \R^n$ is bounded above and $A\neq
\emptyset$. Then $\sup A$ exists and is finite.
\end{ax}

\begin{defn}{(Space of Bounded Functions)}
Here's a more general space than simple $\R^n$: the set of all
bounded functions in $\R^n$,
\begin{align*}
  \mathscr{B}(A,\R^n)
  :=
  \left\{
    f:A\rightarrow \R^n \; \big| \; \sup_{x\in A} |f(x)|<\infty
  \right\}
\end{align*}
Notice that this is a lot tougher to visualize than $X=\R^n$.
Here, elements of $X$ are functions, not vectors.

Notice also that this is a vector space, as
\begin{align*}
  (f+g)(x) &= f(x) + g(x)\\
  (cf)(x) &= cf(x)
\end{align*}
\end{defn}



\section{Metric Spaces and Sequences}

Take any arbitrary set $X$. Metric spaces are simply some pair $(X,d)$
of an arbitrary set and some distance function $d$ that tells you ``how
close'' any two points in $X$ are. By defining the concept of
``closeness'', we can then talk about convergence of sequences within
$X$.

The most common space is $\R^k$; however, the treatment below
will be sufficiently general to capture any set $X$ for which a distance
function or \emph{metric} (satisfying certain properties) can be
defined.


\subsection{Metrics, Norms, and Metric Spaces}

\begin{defn}{(Metric Space)}
\label{defn:metric}
A \emph{metric space} is a pair $\mathscr{M}=(X,d)$ where $X$ is some
space and $d:X\times X\rightarrow \R$ is a function
called a \emph{metric} that satisfies the following conditions for all
``points'' $x,y,z\in X$
\begin{enumerate}
  \item \emph{Non-negativity}: $d(x,y)\geq 0$
  \item \emph{Identification}: $d(x,y) = 0 \quad \Leftrightarrow \quad x=y$
  \item \emph{Symmetry}: $d(x,y)=d(y,x)$
  \item \emph{Triangle Inequality}: $d(x,y) \leq d(x,z) + d(z,y)$.
\end{enumerate}
Identification requires that distinct points have positive distance.
The triangle inequality enforces the idea that the distance function
represents the \emph{shortest path} between points, i.e.\ you can't have
an intermediate pit stop somewhere that reduces the total trip length.
\end{defn}
\begin{defn}{(Metric Subspace)}
We say that $(Y,d_Y)$ is a metric subspace of $(X,d_x)$ if and only if
\begin{enumerate}
  \item $Y\subseteq X$
  \item $d_Y(x,y) = d_X(x,y)$ for all $x,y$.
\end{enumerate}
\end{defn}

\begin{ex}{(Distance Functions in $\R^n$)}
There are some standard distance functions in $\R^n$ that show
up a lot and have a nice geometric representation. So for $x,y\in
\R^n$, define
\begin{enumerate}
  \item Manhattan Taxi Distance
    \begin{align*}
      d_1(x,y) = \sum^n_{i=1} |x_i - y_i|
    \end{align*}

  \item Standard Euclidean Distance
    \begin{align*}
      d_2(x,y) = \sqrt{\sum^n_{i=1} (x_i - y_i)^2}
    \end{align*}

  \item $L_p$ distance
    \begin{align*}
      d_p(x,y) = \left(\sum^n_{i=1} |x_i - y_i|^p\right)^{1/p}
    \end{align*}

  \item ``max'' distance
    \begin{align*}
      d_\infty(x,y) = \max_{i} |x_i-y_i|
    \end{align*}

\end{enumerate}
\end{ex}
\begin{rmk}
The ``max'' distance is useful when $x\in\R^n$ doesn't represent
real location. Rather, $n$-dimensional $x$ might represent $n$ different
attributes or characteristics of some object (like height, weight, hair
length, etc.) that aren't directly comparable. In that case, to compare
individuals, you might just define the ``distance'' between two people
by looking at all these attributes and seeing where the differ most
sharply.
\end{rmk}

Distance functions are commonly built up from another special type of
function called a \emph{norm}. Often, we first encounter \emph{norms} as
functions that measure the length of vectors in $\R^n$, although
the definition permits much more general functions over much more
general spaces (as we'll see). But we really care about them here
because given a norm, you can always construct a distance function from
that norm. So let's define them.

\begin{defn}
\label{defn:norm}
Given a space $X$, a function $N:X\rightarrow \R$ is called a
\emph{norm} if it satisfies the following properties:
\begin{enumerate}
  \item $N(x)\geq 0$
  \item $N(x)=0 \quad \Leftrightarrow \quad x=0$
  \item $N(\alpha x)= |\alpha| N(x)$
  \item $N(x+y) \leq N(x) + N(y)$
\end{enumerate}
Often, you see the notation $||x||$ instead of $N(x)$ to denote the norm
of $x$.
\end{defn}

\begin{ex}
One common norm for $x\in \R^n$ is
\begin{align*}
  N_1(x) = \sum^n_{i=1} |x_i|
\end{align*}
To show this is a norm, one must show that it satisfies the four
properties in Definition~\ref{defn:norm}. Since that's pretty easy, the
proof is omitted.
\end{ex}


\begin{prop}
If $N$ is a norm, then $d:X \times X\rightarrow \R$ defined as
$d(x,y):=N(x-y)$ is a metric for space $X$.
\end{prop}
\begin{proof}
To show that $d$ is a valid metric, we need to show that it satisfies
the four properties of Definition~\ref{defn:metric}:
\begin{enumerate}
  \item Non-negativity: Trivial by Property 1 of
    Definition~\ref{defn:norm}.
  \item Identification: ($\Leftarrow$) If $x=y$, then
    $d(x,y)=N(x-y)=N(0)=0$ by Property 2 of Definition~\ref{defn:norm}.

    ($\Rightarrow$) Suppose $0 = d(x,y) =: N(x-y)$. Property 2 of
    Definition~\ref{defn:norm} tells us that $x-y=0$, hence $x=y$.
  \item Symmetry: Take
    \begin{align*}
      d(x,y)=N(x-y) = N(-(y-x))
    \end{align*}
    By Property 3 of
    Definition~\ref{defn:norm}, we can pull out the negative sign to get
    \begin{align*}
      d(x,y)=N(-(y-x))=|-1|N(y-x) = d(y,x)
    \end{align*}
  \item Triangle Inequality: Take some $z$, then by Property 4 of
    Definition~\ref{defn:norm}
    \begin{align*}
      d(x,y) = N(x-y) = N((x-z)+(z-y))
      &\leq N(x-z) + N(z-y)  \\
      \Leftrightarrow\qquad
      &\leq d(x,z) + d(z,y)
    \end{align*}
\end{enumerate}
\end{proof}

\begin{ex}{(Norms and Metrics for $\mathscr{B}(A,\R^n)$)}
Start by defining a norm $N$ for any bounded function $f:A
\rightarrow \R^n$
\begin{align*}
  N(f) = \sup_{x\in A} |f(x)|
\end{align*}
By ``bounded function'', I mean that $N(f)<\infty$.  Since
Axiom~\ref{ax:completeness} states that the sup always exists for any
function bounded in $\R^n$, this norm is well
defined, and it is called the \emph{sup norm}.

%With this norm, we can define our space $X$ as the set of all bounded
%functions
%\begin{align*}
  %X = \left\{
      %f: A \rightarrow \R^n
      %\; \big|\; N(f) < \infty
      %\right\}
%\end{align*}
Given the space $X=\mathscr{B}(A,\R^n)$, we just need a distance
function to construct the metric space $(X,d)$. As is always the case
with norms, we use the norm and define a metric:
\begin{align*}
  d_\infty(f,g) = N(f-g)
\end{align*}
for any $f,g\in X$. Intuitively, this distance functions looks at how
far apart $f$ and $g$ are at each $x\in A$, and then sets the distance
equal to the gap between the functions at the point of ``maximal gap.''

Of course, we need to show that $N$ is actually a norm for all of the
above to hold.
\end{ex}

\begin{ex}{($L_p$ Space)}
Consider the space of $L_p$ integrable functions:
\begin{align*}
  X = \left\{
    f:A\rightarrow \R \; \big|\;
    \int_{x\in A} |f(x)|^p dx <\infty
  \right\}
\end{align*}
On this space, we can define the metric
\begin{align*}
  d_p(f,g) = \left(\int_{x\in A} |f-g|^p \; dx\right)^{1/p}
\end{align*}
\end{ex}

\subsection{Sequences, Subsequences, Limits, and Convergence}

Now that we have defined the notions of metrics and metric spaces, which
let us talk about ``closeness'' in sets, we present a general treatment
of sequences and convergence within a metric space.

Now for some conventions that this section will use unless othwerwise
noted.  When dealing with sequences, this section will assume that the
sequence $\{x_n\}$ lives in a metric space $(X,d)$. Any functions in
this section are assumed to be of the form $f:(X,d_X)\rightarrow
(Y,d_Y)$.

\begin{defn}{(Sequence)}
A \emph{sequence} in space $X$ is an enumeration of points in $X$,
denoted $\{x_n\}_{n=1}^\infty$ and often abbreviated $\{x_n\}$.
Equivalently, it is a function from the natural numbers to points in the
space $X$, i.e.  $x:\mathbb{N}\rightarrow X$. There can be repetitions.
\end{defn}

\begin{defn}{(Subsequence)}
Given a sequence $\{x_n\}_{n=1}^\infty \subseteq X$ and an increasing
sequence $\{n_k\}_{k=1}^\infty \subseteq \mathbb{N}$, we call the
sequence $\{x_{n_k}\}_{k=1}^\infty$ a \emph{subsequence} of
$\{x_n\}_{n=1}^\infty$.
\end{defn}

\begin{defn}{(Convergence)}
\label{defn:convergence}
A sequence $\{x_n\}\subseteq X$ \emph{coverges}  if there is a point
$x\in X$ (emphasis on ``in $X$'') such that, for any $\varepsilon>0$,
there exists an $N$ where
\begin{align*}
  n > N \quad \Rightarrow\quad
  d(x_n,x) < \varepsilon
\end{align*}
Convergence is often abbreviated $x_n\rightarrow x$ or
$\{x_n\}\rightarrow x$ and $x$ is called the \emph{limit point}.
\end{defn}

\begin{ex}
Importantly, the definition of convergence requires that the
limit point $x$ is in the space $X$. So if we consider metric space
$\mathscr{M}=((0,2),d_1)$, the sequence
\begin{align*}
  \{x_n\} = \{1/n\}
\end{align*}
does not converge since the limit $0$ is not in the space $(0,2)$.
\end{ex}

\begin{note}
Here's a general recipe for proving that a sequence $\{x_n\}$ converges
to some point $x$.

Given $\varepsilon$ and limit $x$, simply write out write out $d(x_n,x)$
for arbitrary $x_n$. Try to bound that expression for the distance as
\begin{align*}
  d(x_n,x) < G(n,x)
\end{align*}
where $G$ is some function that depends only upon $x$ and $n$. If that's
possible, then equate $G(n,x)=\varepsilon$ and solve for $n$. Set the
result as your $N$.

This recipe emphasizes that our choice of $N$ should depend only upon
$\varepsilon$ and $x$, the two ``givens'' within the problem.
\end{note}

\begin{defn}{(Pointwise and Uniform Convergence of Functions)}
Consider the space of functions $X=\{f:A\rightarrow B\}$.  Consider a
sequence of functions $\{f_n\}\subseteq X$.  We say that
$\{f_n\}\rightarrow f^* \in X$ \emph{pointwise} if
\begin{align*}
  \lim_{n\rightarrow \infty} f_n(x) = f^*(x)
  \qquad \forall x\in A
\end{align*}
We say that $\{f_n\}\rightarrow f^*$ \emph{uniformly} if
\begin{align*}
  \lim_{n\rightarrow \infty} \sup_{x\in A}
  |f_n(x)-f(x)| = 0
\end{align*}
\end{defn}
\begin{prop}
\label{prop:pwunif}
Uniform convergence of functions implies pointwise convergence of
functions (though the converse does not hold in general).
\end{prop}
\begin{rmk}
This proposition implicitly contains some practical advice.
Specifically, if a sequence of functions does converge, it converges to
the pointwise limit. So when checking uniform convergence, your
candidate function will \emph{always} be the pointwise limit. You only
have to check if a sequence converges uniformly to this single function
rather than the infinite number of functions that exist.
\end{rmk}

\begin{prop}{\emph{(Uniqueness of Limits)}}
If a sequence $\{x_n\}\rightarrow x$ and $\{x_n\}\rightarrow x'$,
then $x=x'$.
\end{prop}
\begin{proof}
Suppose that $\{x_n\}\rightarrow x$ and $\{x_n\} \rightarrow x'$ such
that $x\neq x'$. Then given $\varepsilon>0$, there exists a $N_x$ and
$N_{x'}$ such that
\begin{align*}
  n > N_x
  &\quad\Rightarrow\quad
  d(x_n,x) \leq \frac{\varepsilon}{2} \\
  n > N_{x'}
  &\quad\Rightarrow\quad
  d(x_n,x') \leq \frac{\varepsilon}{2}
\end{align*}
Now take $n>\max\{N_x, N_{x'}\}$ and use the triangle inequality to get
\begin{align*}
  d(x,x') \leq d(x,x_n) + d(x_n, x')
  < \frac{\varepsilon}{2} + \frac{\varepsilon}{2} = \varepsilon
\end{align*}
Since this is true for all $\varepsilon$, we must have $d(x,x')=0$ which
means $x=x'$.
\end{proof}

\begin{prop}{\emph{(Close Together Sequences)}}
Consider two sequences $\{x_n\}$ and $\{y_n\}$.
Suppose that one sequence converges to some limit and that the two
sequences get arbitrarily close together, i.e.
\begin{align*}
  x_n\rightarrow x
  \quad\text{and}\quad
  d(x_n,y_n)\rightarrow 0
\end{align*}
Then $y_n\rightarrow x$ as well.
\end{prop}
\begin{proof}
Suppose that $\varepsilon>0$ is given. Since $x_n\rightarrow x$, there
exists an $N_x$ such that
\begin{align*}
  n > N_x
  \quad\Rightarrow\quad
  d(x_n,x) < \frac{\varepsilon}{2}
\end{align*}
Next, since $d(x_n,y_n)\rightarrow 0$, there exists an $N_y$ such that
\begin{align*}
  n > N_y
  \quad\Rightarrow\quad
  d(x_n,y_n) < \frac{\varepsilon}{2}
\end{align*}
By the triangle inequality
\begin{align*}
  n>\max\{N_x,N_y\}
  \quad\Rightarrow\quad
  d(y_n,x) \leq
  d(y_n,x_n) + d(x_n,x)
  < \frac{\varepsilon}{2} + \frac{\varepsilon}{2} = \varepsilon
\end{align*}
Hence $\{y_n\}\rightarrow x$ as well.

\end{proof}

\begin{prop}{\emph{(Limits of Subsequences for Convergent Sequences)}}
\label{prop:subseq}
The sequence $\{x_n\}$ converges to $x$ if and only if every subsequence
also converges to $x$.
\end{prop}


\subsection{Sequences in $\R$: Supremum, Infimum, Limit Superior, and
Limit Inferior}

This section provides further tools for dealing with suprema and infima,
along with definition of the \emph{lim sup} and \emph{lim inf}, which
are extremely useful in the study of sequences and limits.
I restrict this section to $\R$ space since the Axiom of Completeness
(Axiom~\ref{ax:completeness}) furnishes the existence of sups and infs
for bounded sequences. Much of what follows wouldn't be true or sensible
without completeness of the space.

But first, I reproduce the definition of the supremum and provide some
more ways to use and think about it, since it will be extremely
important from here on.

\begin{defn}{(Supremum)}
\label{defn:supdef2}
Given a set $A\subseteq X$, the \emph{supremum} denoted $\sup(A)$ is the
smallest possible upper bound for $A$.
More precisely, the supremum satisfies the following two properties
\begin{enumerate}
  \item $a\leq \sup(A)$ for all $a\in A$.
  \item If $x<\sup(A)$, then $x\in X$ is not an upper bound for $A$.
    Otherwise, we would have just taken $x$ as the supremum.
\end{enumerate}
The sup might be in $A$, in which case $\sup(A) = \max(A)$, or it might
be in $X\setminus A$, in which case $\max(A)$ is undefined.
\end{defn}

In practice, we often use the supremum in proofs by writing it in
special representations that are equivalent to
Definition~\ref{defn:supdef2}. We now lay those out.
\begin{prop}
Given some $A\subseteq X$, we have the following results for the
supremum:
\begin{enumerate}
  \item Given any $a\in A$, there exists an $\varepsilon\geq 0$ such
    that
    \begin{align*}
      a = \sup(A) - \varepsilon
    \end{align*}
    Moreover, $\varepsilon=0$ if and only if $\sup(A) \in A$, which is
    not always the case.
  \item Given any $\varepsilon>0$, there exists an $a\in A$ such that
    \begin{align*}
      a > \sup(A) -\varepsilon
    \end{align*}
\end{enumerate}
\end{prop}
\begin{proof}
We just want to show that the two representations above match
Definition~\ref{defn:supdef2}.

To prove (1), suppose that no such $\varepsilon\geq 0$ existed. Then $a
> \sup(A)$, which violates the definition of $\sup(A)$ as the biggest
$x\in X$ such that $x\geq a$ for all $a\in A$.

To prove(2), suppose $\varepsilon>0$ is given and that there is no $a$
such that $a>\sup(A)-\varepsilon$. Then $\sup(A)-\varepsilon$ is an
upper bound of $A$ smaller than $\sup(A)$, which violates the definition
of $\sup(A)$ as the \emph{smallest} upper bound.
\end{proof}
\begin{rmk}
This proposition is very useful. We often use (2) by taking
$\varepsilon=1/n$ and choosing elements of $A$ to build up a sequence,
as we'll do in the next proposition.
\end{rmk}

\begin{prop}
\label{prop:supseq}
Suppose that $A\subseteq \R$. We have the following results:
\begin{enumerate}
  \item If $A$ is bounded from above, then
    $\exists \; \{\bar{a}_n\} \subseteq A$ such that
    $\{\bar{a}_n\} \rightarrow \sup(A)$
  \item If $A$ is bounded from below, then
    $\exists \; \{\munderbar{a}_n\} \subseteq A$ such that
    $\{\munderbar{a}_n\} \rightarrow \inf(A)$
\end{enumerate}
\end{prop}
\begin{proof}
Since $A$ is bounded from above, $\sup A$ exists and is well defined by
Axiom~\ref{ax:completeness}.  Now use the existence of $\sup A$ to
define a particular sequence, $\{\bar{a}_n\}$. We choose the elements by
recalling the definition of $\sup A$, which says
\begin{align*}
  \forall \varepsilon = \frac{1}{n} >0
  \quad
  \exists \; \bar{a}_n\in A \;\; \text{s.t.} \;\;
  \sup A - \frac{1}{n} \leq \bar{a}_n \leq \sup A
\end{align*}
Clearly, if we choose elements this way, all the $\bar{a}_n$ terms are
in $A$ and
\begin{align*}
  \lim_{n\rightarrow \infty} \bar{a}_n &= \sup A
\end{align*}
The proof for $\inf A$ is similar.
\end{proof}

\begin{prop}
Suppose that $\{x_n\}$ is a nondecreasing sequence in $\R$, $x_n \leq
x_{n+1}$. If $\{x_n\}$ is bounded then
\begin{align*}
  \lim_{n\rightarrow \infty} x_n = \sup A
\end{align*}
where $A=\{x_1,x_2,\ldots\}$ is the image of the sequence.
\end{prop}
\begin{proof}
By Axiom~\ref{ax:completeness}, $\sup A$ exists and is well defined
since $A$ is bounded.

Now, given $\varepsilon>0$, we want an $N$ such that
\begin{align*}
  n>N\quad\Rightarrow\quad
  \sup A - x_n < \varepsilon
\end{align*}
Now by the definition of $\sup A$, given $\varepsilon>0$, there exists a
\begin{align*}
  x_m \in A
  \quad \text{s.t.} \quad
  \sup A - \varepsilon &< x_m\\
  \Leftrightarrow\quad
  \sup A - x_m&< \varepsilon
\end{align*}
Next, note that
\begin{align*}
  x_{m+i} &\geq x_m
  \qquad \forall i \geq 0 \\
  \Leftrightarrow\quad
  -x_{m+i} &\leq -x_m \\
  \sup A -x_{m+i} &\leq \sup A-x_m \\
  \Rightarrow \qquad
  \sup A -x_{m+i} &\leq \sup A-x_m < \varepsilon
  \qquad \forall i \geq 0
\end{align*}
So if we take $N=m$, we ensure that $\sup A - x_n< \varepsilon$ for all
$n>N$.
\end{proof}

\begin{cor}
\label{cor:nonincreasing}
Suppose that $\{x_n\}$ is a nonincreasing sequence in $\R$. If
$\{x_n\}$ is bounded then
\begin{align*}
  \lim_{n\rightarrow \infty} x_n = \inf A
\end{align*}
where $A=\{x_1,x_2,\ldots\}$ is the image of the sequence.
\end{cor}
\begin{proof}
Consider $\{y_n\}$ where $y_n = -x_n$ for all $n$. Then $\{y_n\}$ is a
nondecreasing sequence, so
\begin{align*}
  \lim_{n\rightarrow \infty} y_n = \sup -A
\end{align*}
But $\sup -A = \inf A$, so $\lim_{n\rightarrow \infty} x_n = \inf A$
\end{proof}

\begin{ex}
Next, we'll see a definition that can help make sense of sequences that
look like the following
\begin{align*}
  x_n = (-1)^n + \frac{1}{n}
\end{align*}
By taking $n\rightarrow\infty$, you can kind of see that the series has
two ``limits'', -1 and 1. Of course, the series doesn't converge since
there is no single number that $x_n$ converges to, but it does seem to
accumulate around the points 1 and -1 in the tail. The next definitions
will help us talk about and make sense of that idea.
\end{ex}

\begin{defn}
Given a bounded sequence $\{x_n\}$, define the limit superior and limit
inferior as
\begin{align*}
  \bar{x}=
  \limsup_{n\rightarrow \infty} x_n
  &= \lim_{n\rightarrow \infty} \bar{x}_n
  \qquad \text{where} \; \bar{x}_n = \sup_{k\geq n} x_k
  = \sup \{x_n, x_{n+1}, \ldots \}\\
  \munderbar{x}=
  \liminf_{n\rightarrow \infty} x_n
  &= \lim_{n\rightarrow \infty} \munderbar{x}_n
  \qquad \text{where} \; \munderbar{x}_n = \inf_{k\geq n} x_k
  = \inf \{x_n, x_{n+1}, \ldots \}
\end{align*}
\end{defn}
\begin{rmk}
It's important to note that elements of the sequences $\{\bar{x}_n\}$
and $\{\munderbar{x}_n\}$ are not necessarily elements of the original
sequence $\{x_n\}$. Since we're taking sup's and inf's of sets, it's
always possible that the sup's and inf's are outside the set.

For example, consider
\begin{align*}
  \{x_n\} =
  \left\{
    \frac{1}{n}
  \right\}=
  \left\{
    1, \frac{1}{2}, \frac{1}{3}, \ldots
  \right\}
\end{align*}
In that case,
\begin{align*}
  \bar{x}_n &= \sup \left\{ \frac{1}{n}, \frac{1}{n+1},\ldots \right\}
  = \frac{1}{n} \qquad \forall n\\
  \munderbar{x}_n &= \inf \left\{ \frac{1}{n}, \frac{1}{n+1},\ldots \right\}
  = 0\qquad \forall n
\end{align*}
Though $\{\bar{x}_n\}$ consists of elements in the series,
$\{\munderbar{x}_n\}$ does not since the series is constant at zero,
something not in $\{x_n\}$.
\end{rmk}

\begin{prop}
Given a bounded sequence $\{x_n\}$,
both the limit superior and limit inferior always have well defined
limits (i.e.\ they always exist), and
\begin{align*}
  \limsup_{n\rightarrow \infty} x_n
  &= \inf_{n\in \mathbb{N}} \left(\sup_{k\geq n} x_k\right)\\
  \liminf_{n\rightarrow \infty} x_n
  &= \sup_{n\in \mathbb{N}} \left(\inf_{k\geq n} x_k\right)
\end{align*}
In a sense, the $\limsup$ is the ``largest limit'' of the sequence,
while the $\liminf$ is the ``smallest limit'' of the sequence.
\end{prop}
\begin{proof}
Given a sequence $\{x_n\}$, define $\bar{x}_n$ as
\begin{align*}
  \bar{x}_n = \sup_{k\geq n} x_k
\end{align*}
Since $\{x_n\}$ is bounded, the sup taken to form $\bar{x}_n$ exists for
each $n$ by Axiom~\ref{ax:completeness}. Moreover, $\{\bar{x}_n\}$ is a
nonincreasing sequence because, in moving from $\bar{x}_n$ to
$\bar{x}_{n+1}$, you lose a candidate ($x_n$ specifically) for the
supremum. So $\bar{x}_n$ either gets smaller or stays the same. You can
only do ``worse'' with the additional constraint of $k\geq n+1$ versus
$k\geq n$. Hence by Corollary~\ref{cor:nonincreasing}, the $\bar{x}_n$
converges to the following well-defined object
\begin{align*}
  \lim_{n\rightarrow \infty}\bar{x}_n
  = \inf_{n\in \mathbb{N}} \bar{x}_n
  = \inf_{n\in \mathbb{N}} \left( \sup_{k\geq n} x_k\right)
\end{align*}
\end{proof}


\subsection{Cauchy Sequences and Complete Metric Spaces}

\begin{defn}{(Cauchy Sequence)}
Suppose we have a sequence $\{x_n\}\subseteq X$. That sequence is a
\emph{Cauchy sequence} if and only if, for all $\varepsilon>0$, there
exists an $N$ such that
\begin{align*}
  n,m>N \quad\Rightarrow\quad
  d(x_n,x_m) <\varepsilon
\end{align*}
Here's a general recipe for proving that a sequence $\{x_n\}$ is
Cauchy.

Given $\varepsilon$, simply write out write out $d(x_n,x_m)$
for arbitrary $x_n$ and $x_m$. Try to bound that expression for the
distance as
\begin{align*}
  d(x_n,x_m) < G(\max\{n,m\})
\end{align*}
where $G$ is some function that depends only upon the max of $n$ and
$m$. If that's possible, then equate $G(N)=\varepsilon$ and solve for
$N$.
\end{defn}
\begin{rmk}
Notice the difference from Definition~\ref{defn:convergence} of
convergence. For Cauchy sequences, you don't have to know the limit
point, while traditional convergence a la
Defintion~\ref{defn:convergence} requires you to know limit point.
Cauchy sequences require simply that points in the tail are
``close together.''

In fact, the main use of Cauchy sequences is precisely this. We might be
able to say very easily that points are close together in the tail,
though we have no idea what the limit point is or if one exists. But if
we're in a special type of metric space (a \emph{complete} metric space,
described below), Cauchy will be ``good enough'' in some sense---we'll
be able to say that a Cauchy sequence will converge to some limit in a
complete metric space, though we might not know what that limit is.

So we care about Cauchy sequences because \emph{sometimes} Cauchy
implies convergent. The next theorem concerns the other direction:
whether a convergent series is Cauchy.
\end{rmk}

\begin{thm}{\emph{(Convergent $\Rightarrow$ Cauchy)}}
Though Cauchy does \textbf{not} imply convergent,
a convergent sequence is \textbf{always} Cauchy.
\end{thm}
\begin{proof}
Suppose that we are given $\varepsilon>0$ and convergent sequence
$\{x_n\}\rightarrow x$.  By the triangle inequality, we know that for
any $n$ and $m$,
\begin{align*}
  d(x_n,x_m) \leq d(x_n,x) + d(x,x_m)
\end{align*}
Since $\{x_n\}$ is convergent, we can choose $N$ such that
\begin{align*}
  n>N
  \quad\Rightarrow\quad
  d(x,x_n) < \varepsilon/2
  %\begin{cases}
    %d(x,x_n) < \varepsilon/2\\
    %d(x,x_m) < \varepsilon/2
  %\end{cases}
\end{align*}
If we use this as our Cauchy $N$, that implies that
\begin{align*}
  n,m>N
  \quad \Rightarrow\quad
  d(x_n,x_m) \leq d(x_n,x) + d(x,x_m) < \frac{\varepsilon}{2} +
    \frac{\varepsilon}{2}
  =\varepsilon
\end{align*}
\end{proof}

\begin{prop}{\emph{(Cauchy $\Rightarrow$ Bounded)}}
\label{prop:cauchy-bounded}
Any Cauchy sequence is bounded.
\end{prop}
\begin{proof}
Let $\{x_n\}$ be a Cauchy sequence. Set $\varepsilon=1$. Then there is
an $N$ such that
\begin{align}
  \label{cauchybddN}
  n,m>N \quad\Rightarrow\quad
  d(x_n,x_m) < 1
\end{align}
Next, taking advantage of the fact that $\{x_n\}_{n=1}^{N+1}$ is a
finite set, let
\begin{align}
  \label{cauchybddK}
  K = \max_{\hat{n},\hat{m} \in \{1,\ldots,N+1\}}
  d(x_{\hat{n}},x_{\hat{m}})
\end{align}
Just to emphasize, $K$ exists and is finite because
$\{x_n\}_{n=1}^{N+1}$ is a finite set.

Therefore, we can conclude that for any $n,m$
\begin{align*}
  d(x_n,x_m) \leq 1+K
\end{align*}
To see this, consider all of the cases that might arise for $n,m$
\begin{enumerate}
  \item If both $n,m\leq N$, then by the definition of $K$ in
    (\ref{cauchybddK}), you have $d(x_n,x_m)<K<1+K$.
  \item If both $n,m>N$, then by (\ref{cauchybddN}), you have
    $d(x_n,x_m)<1<1+K$.
  \item If $n\leq N$ and $m> N$ (or vice versa), then by triangle
    inequality,
    \begin{align*}
      d(x_n,x_m)
      &\leq d(x_n,x_{N+1}) + d(x_{N+1},x_m)\\
      &< K + d(x_N,x_m) \qquad \text{By~(\ref{cauchybddK})}\\
      &< K + 1 \quad\qquad\qquad \text{By~(\ref{cauchybddN})}\\
    \end{align*}
\end{enumerate}
Hence, under any case $d(x_n,x_m)<1+K$ for all $n,m$.
\end{proof}
\begin{cor}
Convergent sequences are bounded.
\end{cor}

\begin{prop}
\label{prop:cauchy-subseq}
Suppose that $\{x_n\}$ is a Cauchy sequence and which has a subsequence
$\{x_{n_k}\}\rightarrow x$. Then the original sequence
$\{x_n\}\rightarrow x$ as well.

In other words, if you can find a convergent subsequence for some Cauchy
sequence, then we know the limit of the original sequence. You might not
always be able to do that, but if you can, that's good news.
\end{prop}
\begin{proof}
Suppose $\varepsilon>0$ is given.  Since $\{x_n\}$ is Cauchy, there is
an $N_{n,m}$ such that
\begin{align*}
  n,m>N_{n,m}
  \quad\Rightarrow\quad
  d(x_n,x_m) <\varepsilon/2
\end{align*}
Next, since $\{x_{n_k}\}\rightarrow x$, we can choose an $N_k \geq
N_{n,m}$ such that
\begin{align*}
  n > N_k
  \quad\Rightarrow\quad
  d(x_n, x) < \varepsilon/2
\end{align*}
Therefore, by choosing $n>N_k$ and using the triangle inequality, we can
say that
\begin{align*}
  d(x_n,x) \leq d(x_n, x_{n_k}) + d(x_{n_k},x)
  d(x_n,x) \leq \varepsilon/2 + \varepsilon/2 = \varepsilon
\end{align*}
\end{proof}

\begin{defn}{(Complete Metric Space)}
Metric space $(X,d)$ is called a \emph{complete} metric space if and
only if all Cauchy sequences are convergent.

In $\R^n$, this is a statement that can be proved by starting
with Axiom~\ref{ax:completeness}.
\end{defn}

\begin{ex}
The most common example of a compelte metric space is
$(\R,d_2)$.
\end{ex}

\begin{ex}
\label{ex:Ncomplete}
Metric space $(\N,d_1)$ is a complete metric space.
\end{ex}

\begin{ex}
Given any set $X$ and the \emph{discrete} metric
\begin{align*}
  d(x,y) =
  \begin{cases}
    1 & x\neq y \\
    0 & x= y \\
  \end{cases}
\end{align*}
the metric space $(X,d)$ is complete.
Moreover, this is ``equivalent'' to Example~\ref{ex:Ncomplete} in some
since since in that example and in this one, all convergent sequences
must be constant in the tail.
\end{ex}

\begin{ex}
The metric space $\left(\mathscr{B}(A,\R^n), d_\infty\right)$,
where $\mathscr{B}(A,\R^n)$ is the set of all bounded functions
$f:A\rightarrow \R^n$, is a complete metric space.
\end{ex}

\begin{prop}
\label{prop:bwproperty}
In metric space, $(X,d)$,
suppose that the \emph{Bolzano-Weierstrass} property is true:
\begin{align}
  \label{bwproperty}
  \{x_n\} \;\text{bounded}
  \quad\Rightarrow\quad
  \exists \{x_{n_k}\}\rightarrow x \in X
\end{align}
where $\{x_{n_k}\}$ is a convergent subsequence of $\{x_n\}$.  In other
words, any bounded sequence has a convergent subsequence. If this is
true, then $(X,d)$ is a complete metric space.
\end{prop}
\begin{proof}
Suppose that we have a Cauchy sequence $\{x_n\}$. By
Theorem~\ref{prop:cauchy-bounded}, $\{x_n\}$ is bounded. Since we assume
Relationship~\ref{bwproperty}, there is a convergent subsequence
$\{x_{n_k}\}\rightarrow x \in X$.

Since there is a convergent subsequence,
Proposition~\ref{prop:cauchy-subseq} states that $\{x_n\}\rightarrow
x\in X$. Hence, the sequence has a limit implying $(X,d)$ is complete.
\end{proof}

\begin{note}
%(Relation to Proposition~\ref{prop:cauchy-subseq})
On the surface, this looks very similar to
Proposition~\ref{prop:cauchy-subseq}, but they are quite different.

First, Proposition~\ref{prop:cauchy-subseq} holds for all metric spaces
no matter what. It simply says ``If you can find a convergent
subsequence for some Cauchy sequence, then we know the limit of the
original sequence.'' It doesn't say you'll be able to do this for every
Cauchy sequence. It just says ``if you can\ldots''

Contrast that with Proposition~\ref{prop:bwproperty} and note the
differences
\begin{enumerate}
  %\item As shown in the proof, it basically does say ``you'll be able to
    %find a convergent subsequence for every Cauchy sequence.'' It just
    %says it indirectly by saying ``you'll be able to do that for any
    %bounded sequence.'' And since Cauchy sequences are bounded, it
    %follows directly.
  \item This proposition doesn't work in all metric spaces, and requires
    the metric space to abide by the special condition laid out in
    Property~\ref{bwproperty}. That's restrictive for the metric space,
    while Proposition~\ref{prop:cauchy-subseq} is general for all metric
    spaces.
  \item Proposition~\ref{bwproperty} assumes that boundedness
    will \emph{generate} convergent subsequences; whereas,
    Proposition~\ref{prop:cauchy-subseq} assumed you brought a
    convergent subsequence to the table straightaway.
  \item Proposition~\ref{prop:bwproperty} makes a statement about the
    metric space---namely that it's \emph{complete}---which is hugely
    important. Property~\ref{prop:bwproperty} didn't say anything about
    the metric space.
\end{enumerate}
\end{note}

\begin{rmk}
(Relation to Traditional Bolzano-Weierstrass Theorem) This is different
than Theorem~\ref{thm:bolzano-weierstrass}, the Bolzano-Weierstrass
Theorem we will see in a bit. That specifically concerns $\R^n$,
which just so happens satisfy the Bolzano-Weierstrass property. This
statement is more general and just says that \emph{if} a metric space
satisfies the property, then the space is complete.
\end{rmk}


\section{Metric Topology}

Topology is the study of open sets and continuous functions in arbitrary
spaces. Metric topology is the same thing, but when you also have a
\emph{metric} or \emph{distance function} on those spaces, which can be
used to aid in defining or generating open sets and continuous
functions.

Unless otherwise noted, it assumed below that any function
$f:X\rightarrow Y$ is implicitly a function between metric spaces
$(X,d_X)$ and $(Y,d_Y)$, where the metric for each space $X$ and $Y$
might differ, hence the subscripted $d_X$ and $d_Y$.
I will sometimes write the function's metric spaces explicitly as
$f:(X,d_X)\rightarrow(Y,d_Y)$.

\subsection{Primitives}

\begin{defn}{(Open Ball)}
Given space $X$ and real number $r>0$, we define an \emph{open ball}
about point $x$ as
\begin{align*}
  B_r(x) := \{ y \in X \;|\; d(x,y)<r\}
\end{align*}
\end{defn}

\begin{defn}{(Open Set)}
Set $A\subseteq X$ is \emph{open} if, for any $x\in A$, there exists a
$r$ such that $B_r(x)\subseteq A$.
\end{defn}

\begin{prop}
Open balls are open sets.
\end{prop}
\begin{proof}
Within metric space $(X,d)$, take the open ball $B_{r_x}(x)$.  To prove
that $B_{r_x}(x)$ is open, we need to be able to do the following:
\begin{align*}
  \text{Given any $y\in B_{r_x}(x)$, choose $r_y$ such that $B_{r_y}(y)\subseteq B_{r_x}(x)$}
\end{align*}
By the definition of open balls, the statement $B_{r_y}(y)\subseteq
B_{r_x}(x)$ is equivalent to
\begin{align*}
  d(z,y) < r_y \quad\Rightarrow\quad
  d(z,x) < r_x
  \quad\text{for any $z\in X$}
\end{align*}
So how can we choose $r_y$ such that the above statement holds?
Well, let's expand our target for bounding---i.e.\ $d(z,x)$--- via the
triangle inequality so that we get something that is a function of
$r_y$:
\begin{align*}
  d(z,x) &\leq d(z,y) + d(y,x)\\
  \Rightarrow \quad
  d(z,x) &< r_y + d(y,x) \qquad
  \text{since $z\in B_{r_y}(y) \; \Rightarrow\; d(z,y)<r_y$}
\end{align*}
Therefore, since we want our choice of $r_y$ to have the side effect of
bounding $d(z,x)$ by $r_x$ it's enough to set $r_y = r_x - d(y,x)$ and
that will do the trick.
\end{proof}

\begin{defn}{(Interior)}
Given a set $A\subseteq X$, we define the \emph{interior} of $A$ as
\begin{align*}
  \mathring{A}
  := \{x \in A \; | \; \exists\; r>0 \;\; \text{s.t.} \;\; B_r(x) \subseteq A\}
\end{align*}
Note that $r$ can be different for each $x\in A$. It need not be a fixed
number for all $x\in A$.
\end{defn}

\begin{prop}
A set $A\subseteq X$ is open if and only if $\mathring{A} = A$.
\end{prop}

\begin{defn}{(Closure)}
Given a set $A\subseteq X$, we define the \emph{closure} of $A$ as
\begin{align*}
  \bar{A}:= \{
    x\in X \;|\;
    \forall r>0 \quad B_r(x) \cap A \neq \emptyset
  \}
\end{align*}
\end{defn}

\begin{defn}{(Closed Sets, Topology)}
\label{defn:closed}
A set $A\subseteq X$ is \emph{closed} if and only if $\bar{A} = A$.
\end{defn}

\begin{prop}{\emph{(Closed Sets, Limit Points)}}
\label{prop:closed}
Equivalently, $C\subseteq X$ is \emph{closed} if and only if $C$
contains its limit points. More explicitly, for any sequence
$\{x_n\}\subseteq C$ where $x_n\rightarrow x$, the set $C$ is closed if
and only if $x\in C$.
\end{prop}
\begin{rmk}
In Proposition~\ref{prop:closed}, We see the origin of the term
``closed'', a word that is typically shorthand in math for ``closed
under some operation.'' In vector spaces, elements of the space are
``closed'' under the operations of addition and scalar
multiplication---if you arbitrarily add or multiply elements in the
space, you still get back an element in that space.

Here, the operation is \emph{taking limits}. If you have a sequence in
some set and take its limit, the limit point must also be in the set (if
the limit exists, of course).
\end{rmk}
\begin{proof}{(Proposition~\ref{prop:closed})}
($\Rightarrow$) Suppose that $\{x_n\}\subseteq C$ converges to $x$ and
$C$ is closed. We want to show that $x\in C$.

So start by supposing that $x\not\in C$. Then $x\in C^C$, which is an
open set. Since $C^C$ is open, there exists an $\varepsilon>0$ such that
the open ball $B_\varepsilon(x) \subseteq C^C$ and does not intersect
$C$. Now since $x_n\in C$ for all $n$, it is the case that
$d(x_n,x)\geq\varepsilon$ for all $n$. But this violates the assumption
that $x_n\rightarrow x$.

($\Leftarrow$) Now assume that for any sequence $\{x_n\}\subseteq C$
converging to some $x$, it is the case that $x\in C$. We want to show
that $C$ is closed according to Definition~\ref{defn:closed}.  To do so,
we basically need to show that \emph{any} $p\in X$ satsifying the
criterion for being in the closure $\bar{C}$, i.e.\
\begin{align}
  \label{pcond}
  p \quad \text{satisfying}\quad
  \forall r>0 \quad B_r(p) \cap C \neq \emptyset
\end{align}
can be written as the limit of some sequence in $C$. Then, by assumption
of this direction of the proof, that limit is in $C$, implying
$\bar{C}=C$.

So let $p$ satisfy Condition~\ref{pcond}. Then form the sequence
$\{p_n\}$ where element $p_n$ satisfies
\begin{align*}
  p_n \in B_{1/n}(p)\cap C
\end{align*}
That is, form a sequence of shrinking open balls, each with radius $1/n$
for $n=1,2,\ldots$. From the $n$th ball, pick some element in
$B_{1/n}(p)\cap C$ to serve as $p_n$. Assuming Condition~\ref{pcond}
guarantees that there is at least one such element.

Now that's a sequence converging to $p$, since for any $\varepsilon>0$
it is the case that
\begin{align*}
  n>\frac{1}{\varepsilon}
  \quad\Rightarrow\quad
  d(p_n,p) < \varepsilon
\end{align*}
Since we assumed that $p\in C$ when it is the limit of some sequence, we
can conclude $\bar{C}=C$.
\end{proof}

\begin{ex}{(Set of Bounded Nondecreasing Functions)}
The Proposition~\ref{prop:closed} characterization of closed sets is
very useful for function spaces in particular. In fact, this proof will
use it twice: once for a sequence of functions and another time for a
numerical sequence of points. Here it goes.

Consider the set of bounded nondecreasing functions:
\begin{align*}
  C = \{
    f:A\rightarrow \R \; | \;
    x\geq y
    \;\Rightarrow\;
    f(x) \geq f(y)
  \}\subset \mathscr{B}(A,\R)
\end{align*}
Suppose that we take some sequence $\{f_n\}\subseteq C$ such that
$f_n\rightarrow f$ uniformly.  By Proposition~\ref{prop:closed}, we want
to show that $f\in C$ to show that $C$ is closed.

So to draw out a contradiction, suppose instead that $f\not \in C$. Then
there exists a pair $\hat{x},\hat{y}\in A$ such that $\hat{x}\leq
\hat{y}$ and $f(\hat{y})<f(\hat{x})$ (or equivalently,
$f(\hat{y})-f(\hat{x})=a<0$).

For that pair $\hat{x},\hat{y}$, define the numerical sequence
\begin{align*}
  a_n = f_n(\hat{y})-f_n(\hat{x})
  %\qquad
  %\text{for some pair}
  %\;
  %\hat{x} \leq \hat{y}
\end{align*}
By assumption, $f_n$ is increasing for all $n$, so $a_n\geq 0$ for all $n$.
In other words $\{a_n\} \subseteq[0,\infty)$.
Since $[0,\infty)$ is a closed set, Proposition~\ref{prop:closed} again
tells us that the limit $a = \lim_{n\rightarrow \infty} a_n$ must be in
$[0,\infty)$ if the sequence converges.

Now first, it does converge. Recall Proposition~\ref{prop:pwunif}, which
says that uniform $f_n\rightarrow f$ implies pointwise
$f_n(x)\rightarrow f(x)$ for all $x\in A$. So
\begin{align*}
  \lim_{n\rightarrow \infty} a_n
  = \lim_{n\rightarrow \infty} [f_n(\hat{y}) - f_n(\hat{x})]
  = f(\hat{y}) - f(\hat{x})
  =a
\end{align*}
Hence $\{a_n\}$ converges to $a$.

However, we assumed $f\not\in C$ with $f(\hat{y})-f(\hat{x})=a<0$.
But this can't be. $a\not\in[0,\infty)$ and
Proposition~\ref{prop:closed} guaranteed it would be. A contradiction,
implying $f\in C$, implying $C$ is closed.

Again, we used Proposition~\ref{prop:closed} twice, and in both
directions. In particular, we used the fact that the inclusion
$\lim_{n\rightarrow \infty} f_n=f\in C$ \emph{implied} $C$ was closed.
Along the way, we also used the fact that $\{a_n\}\subseteq [0,\infty)$
(a set we \emph{knew} to be closed) implied the \emph{inclusion}
$\lim_{n\rightarrow \infty} a_n =a\in [0,\infty)$.
Proposition~\ref{prop:closed} did crucial double duty in this proof.
\end{ex}

\begin{prop}{\emph{(Closed and Open Relationship\footnote{Not what you think})}}
$A\subseteq X$ is open if and only if $A^C\subseteq X$ is closed.
\end{prop}
\begin{proof}
($\Rightarrow$) Suppose that the set $A\subseteq X$ is open.  Suppose
also that the set $A^C$ is not closed. THat means $A^C$ is missing a
limit point, i.e.\ there is a point $x\notin A^C$ such that
\begin{align}
\forall r>0 \qquad B_r(x)\cap A^C\neq \emptyset
\label{contra.open}
\end{align}
But $x\not\in A^c$ implies that $x\in A$. And
Statement~\ref{contra.open} implies that any ball you put around $x\in
A$ will intersect $A^C$. This is a violation of $A$ being open, since
$A$ open means that you can put some ball around $x$ that lies entirely
within $A$. Hence Satement~\ref{contra.open} implies a contradiction, so
$A^C$ is closed.

($\Leftarrow$) Now suppose that $A^C$ is closed. By a similar argument
to the other direction if $x\in A$, then it is the case that it is not a
limit point of $A^C$, i.e.\ there's some ball $B_r(x)$ that does not
interset $A^C$ (otherwise $x$ would be in $A^C$ by definition). Thus
that ball lies entirely in $A$. Since this is true for all $x\in A$, the
set $A$ is open.
\end{proof}

\begin{lem}
Given any set $A\subseteq X$, we have $\mathring{A} \subseteq A
\subseteq \bar{A}$.
\end{lem}

\begin{proof}
Given $A\subseteq X$, the set $A$ is open if and only if $A^c$ is
closed.
\end{proof}



\begin{defn}{(Metric Topology)}
A \emph{metric topology} starts with a metric space $\mathscr{M}=(X,d)$
and uses the metric to define the family of open sets $\mathscr{T} = \{A
\subseteq X\;|\; A \text{ open}\}$. The metric topology has the
following characteristics:
\begin{enumerate}
  \item $X,\emptyset$ are open sets open
  \item $A^* = \bigcup_{A\in \mathscr{T}} A$ is open, for any arbitrary
    finite or infinite union of sets in $\mathscr{T}$.
  \item $A_* = \bigcap_{i=1}^n A_i$ is open, for any arbitrary but
    finite intersection of sets in $\mathscr{T}$.
\end{enumerate}
\end{defn}

\begin{cor}
Finite unions of closed sets are closed.
Infinite intersections of closed sets are closed.
\end{cor}
\begin{proof}
De Morgan's law. To do.
\end{proof}

\subsection{Topologically Equivalent Metric Spaces}

This section can be skipped entirely, and all other sections will still
make sense. It's included simply for reference.

\begin{defn}
Two metric spaces $(X,d_1)$ and $(X,d_2)$ using different metrics for
the same underlying space $X$, we say that $d_1$ is \emph{finer} than
$d_2$ if $(X,d_1)$ has all of the open sets of $(X,d_2)$ and possibly
more. Mathematically, we write
\begin{align*}
  d_1 \succeq d_2
  \quad \Leftrightarrow \quad
  \left(
  \text{$A$ open in $(X,d_2)$}
  ; \Rightarrow\;
  \text{$A$ open in $(X,d_1)$}
  \right)
\end{align*}
\end{defn}

\begin{defn}
Metric space $(X,d_1)$ is \emph{topologically equivalent} to $(X,d_2)$
if and only if $d_1 \succeq d_2$ and $d_1 \preceq d_2$. This can be
written more compactly as $d_1\sim d_2$.
\end{defn}

\begin{prop}
If $d_1 \sim d_2$, $C$ is closed in $(X,d_1)$ if and only if $C$ is
closed in $(X,d_2)$.
\end{prop}


\begin{prop}{\emph{(Equivalence of Distance Functions from Norms)}}
For space $X$, suppose that $\hat{N}$ and $\tilde{N}$ are two norms and
we define corresponding distance functions
\begin{align*}
  \hat{d}(x,y) = \hat{N}(x-y)
  \qquad
  \tilde{d}(x,y) = \tilde{N}(x-y)
  \qquad \forall x,y\in X
\end{align*}
Then the open sets of $(X,\hat{d})$ are the same as the open sets of
$(X,\tilde{d})$. The distance functions are equivalent.
\end{prop}

\subsection{Compactness}

\begin{defn}{(Open Cover)}
Given a metric space $(X,d)$ and $K\subseteq X$, we say that the family
of open sets $\{C_\alpha\}_{\alpha \in J}$ is an \emph{open cover} of
$K$ if
\begin{align*}
  K \subseteq \bigcup_{\alpha \in J} C_\alpha
\end{align*}
\end{defn}

\begin{defn}{(Compactness, Topological)}
A set $K$ is said to be compact if and only if \emph{every} arbitrary
open cover $\{C_\alpha\}_{\alpha \in J}$ has a finite subcover, i.e.\
there exist a finite number of indices $\{\alpha_i\}_{i=1}^N$ such that
\begin{align*}
  K \subseteq \bigcup_{i=1}^N C_{\alpha_i}
\end{align*}
\end{defn}

\begin{thm}{\emph{(Compactness, Sequential)}}
\label{thm:compact}
A set $K\subseteq X$ is compact if and only if every sequence
$\{x_n\}\subseteq K$ has a convergent subsequence $\{x_{n_k}\}
\rightarrow x \in K$ (emphasis on $x\in K$).
\end{thm}

\begin{prop}{\emph{(Compact $\Rightarrow$ Closed)}}
\label{prop:compact-closed}
If a set $K$ is compact, then $K$ is also closed.
\end{prop}
\begin{proof}
We'll prove this Proposition~\ref{prop:closed}. So suppose that
$\{x_n\}$ is some convergent sequence in $K$, i.e.
\begin{align*}
  \{x_n\} \subseteq K
  \quad \text{and}\quad
  \lim_{n\rightarrow \infty} x_n=x.
\end{align*}
Take any subsequence $\{x_{n_k}\}$ of $\{x_n\}$.
By Theorem~\ref{prop:subseq}, since $\{x_n\}\rightarrow x$, all
subsequences, including $\{x_{n_k}\}$, must also converge to $x$. By
Theorem~\ref{thm:compact}, $K$ compact implies $x\in K$.  Finally,
Proposition~\ref{prop:closed} tells us that $x\in K$ implies $K$ is
closed.
\end{proof}

\begin{prop}
Closed subsets of compact sets are also compact. Mathematically, for
compact $K\subseteq X$ and closed $C \subseteq K$, the set $C$ is
closed.
\end{prop}

\begin{thm}{\emph{(Bolzano-Weierstrass)}}
\label{thm:bolzano-weierstrass}
Every bounded sequence in Euclidean space $(\R^n,d_2)$ has at
least one convergent subsequence.
\end{thm}

\begin{thm}{\emph{(Heine-Borel)}}
\label{thm:heine-borel}
Within metric space $(\R,d_1)$, a set $K\subseteq \R$ is
compact if and only if $K$ is closed and bounded.
\end{thm}
\begin{proof}
($\Rightarrow$)
Closed is easy. Proposition~\ref{prop:compact-closed} tells us that
compact implies closed. Now for ``compact implies bounded.''

The following is certainly an open cover for compact
$K\subseteq \R$:
\begin{align*}
  \{ A_n &= B_n(0) \; | \; n \in \mathbb{N}\}\\
  \Rightarrow\quad
   \R &= \bigcup_{n\in \mathbb{N}} A_n
\end{align*}
By compactness, there then exists a finite subcover
$\{A_{n_k}\}_{k=1}^N$ such that
\begin{align*}
  K \subseteq \bigcup_{k=1}^N A_{n_k}
  = B_r(0)
  \quad \text{where} \;r=\max_{k=1,\ldots,N} n_k
\end{align*}
Therefore, the set $B_r(0)$ bounds $K$.

($\Leftarrow$) Suppose $K\subseteq \R$ closed and bounded. Let's
show $K$ is compact.

Let $\{x_n\}$ be any sequence in $K$. Since $K$ is bounded,
Theorem~\ref{thm:bolzano-weierstrass} says that there is at least one
convergent subsequence of $\{x_n\}$; call it $\{x_{n_k}\}$.  Since
$\{x_{n_k}\}$ converges in closed $K$, we know that the limit is also in
$K$. By Theorem~\ref{thm:compact}, $K$ is closed.
\end{proof}


\subsection{Continuous Functions}

\begin{defn}{(Continuity, $\varepsilon$-$\delta$)}
\label{defn:cts}
%Suppose that we have two metric spaces $(X,d_X)$ and $(Y,d_Y)$.
A function $f:X\rightarrow Y$ is \emph{continuous} at point $x\in X$ if,
given any $\varepsilon>0$, there exists a $\delta$ such that
\begin{align*}
  d_X(x,p) < \delta \quad \Rightarrow\quad
  d_Y(f(x),f(p))<\varepsilon
\end{align*}
But that's a local definition that defines continuity only at some point
$x\in X$. However, if $f$ is continuous at each point $x\in X$, then we
say that ``the function $f$ is continuous'' (without any ``at $x$''
qualifier).
\end{defn}

\begin{rmk}
You'll notice that Definition~\ref{defn:cts} has a similar flavor to the
definition of convergence in Definition~\ref{defn:convergence}. Well,
there's also a similar recipe for showing that a function is continuous,
and it will look a lot like the general recipe for showing convergence.

Given $f:X\rightarrow Y$, fix $x$ and suppose that $\varepsilon$ is
given. For arbitrary $p$, try to bound the expresion for the distance in
$Y$ as
\begin{align*}
  d_Y(f(x), f(p)) < G\left(d_X(x,p), x\right)
\end{align*}
where $G$ is a function of $d_X(x,p)$ and $x$.

In other words, simply write out $d_Y(f(x),f(p))$ for given $x$ and
arbitrary $p$.  Then simplify the expression to get something that only
depends upon $d_X(x,p)$ and $x$. Finally, substitute $\delta$ for
$d_X(x,p)$ in $G$ and solve $G(\delta,x)=\varepsilon$ for $\delta$.

This recipe emphasizes that our choice of $\delta$ should depend only
upon $\varepsilon$ and $x$, two ``givens'' within the problem.
\end{rmk}

\begin{ex}
Consider an affine function $f:(\R,d_1) \rightarrow
(\R,d_1)$ written as follows:
\begin{align*}
  f(x) = ax + b
\end{align*}
Using the recipe above, let's write out $d_Y(f(x),f(p))$:
\begin{align*}
  d_Y(f(x),f(p))
  &= |f(x)-f(p)|\\
  &= |ax+b-ap+b|\\
  &= |a||x-p|\\
  &= |a|d_X(x,p)
\end{align*}
Now set this guy equal to $\varepsilon$, sub in $\delta$ for $d_X(x,p)$,
and solve for $\delta$:
\begin{align*}
  \varepsilon = |a| \delta
  \quad \Rightarrow\quad
  \delta = \frac{\varepsilon}{|a|}
\end{align*}
Note that in this case, the choice of $\delta$ is independent of $x$,
though this isn't always the case. We see this in the next example.
\end{ex}

\begin{ex}
Consider $f: (\R,d_1)\rightarrow(\R,d_1)$ where
$f=\sqrt{x}$. Before getting into it, we establish the following fact:
\begin{align}
  (a-b)(a+b)
  &= a^2 - b^2\notag\\
  \Leftrightarrow\quad
  (a-b)
  &= \frac{a^2 - b^2}{(a+b)}
  \qquad \text{for $a,b>0$}
  \label{eq:sqrtxhelper}
\end{align}
With that, again follow the recipe:
\begin{align*}
  d_Y(f(x),f(p))
  &= |f(x)-f(p)|\\
  &= |\sqrt{x}-\sqrt{p}|\\
  \text{By Equation~\ref{eq:sqrtxhelper}}\qquad
  &= \frac{|x-p|}{\left\lvert \sqrt{x}+\sqrt{p}\right\rvert}\\
  &\leq \frac{d_X(x,p)}{\left\lvert \sqrt{x}\right\rvert}
\end{align*}
So now, equate that to $\varepsilon$, sub in $\delta$, and solve
\begin{align*}
  \varepsilon &= \frac{\delta}{\left\lvert \sqrt{x}\right\rvert}\\
  \Leftrightarrow\quad
  \delta &= \varepsilon\sqrt{x}
\end{align*}
\end{ex}

There's another more general (but perfectly equivalent) way to define
continuity that doesn't make use of the $\varepsilon$-$\delta$ method
introduced in Definition~\ref{defn:cts}. Instead, we can work directly
with open sets.

\begin{thm}{\emph{(Continuity, Topology)}}
\label{defn:topcts}
Function $f:X\rightarrow Y$ is continuous if and only if
\begin{align*}
  A \subseteq Y \; \text{open}
  \quad \Rightarrow \quad
  f^{-1}(A) \subseteq X \; \text{open}
\end{align*}
\end{thm}
\begin{rmk}
Unlike Definition~\ref{defn:cts}, Definition~\ref{defn:topcts} is super
useless for proving a particular function that you are given is
continuous. Instead, it's useful for proving general properties
\emph{about} continuous functions.
\end{rmk}
\begin{proof}
(Theorem~\ref{defn:topcts})
First, the $\Rightarrow$ direction, assuming
Definition~\ref{defn:cts}-type continuity.

To prove openness of $f^{-1}(A)$,  we want to show that for any $x\in
f^{-1}(A)$ we can find a $\delta$ such that $B_\delta(x)\subseteq
f^{-1}(A)$.  To do so, note that $f(x) \in A$, and $A$ is open.
Therefore, there exists an $\varepsilon$ such that
$B_\varepsilon(f(x))\subseteq A$.  Since $f$ is continuous, there is a
corresponding $\delta$ such that $y \in B_\delta(x)$ implies $f(y) \in
B_\varepsilon(f(x))$. In other words
\begin{align*}
  f\left(B_\delta(y)\right) \subseteq
  B_\varepsilon(f(x))\subseteq A
\end{align*}
Taking the preimage
\begin{align*}
  B_\delta(y) \subseteq f^{-1}(A)
\end{align*}

($\Leftarrow$) Suppose that $f^{-1}(A)$ is open. We need to show the
$\varepsilon$-$\delta$ definition. So suppose that $x\in X$ and
$\varepsilon>0$ are given, and our task is finding $\delta$.

The set $B_\varepsilon(y)=B_\varepsilon(f(x))$ is open. Therefore, the
preimage $f^{-1}(B_\varepsilon(y))$ will be open by assumption. That
means we can find some $\delta$ such that $B_\delta(x)\subseteq
f^{-1}(B_\varepsilon(y))$. Mapping this subset relationship back into
$Y$, we ensure
\begin{align*}
  f(B_\delta(x)) \subseteq B_\varepsilon(y) = B_\varepsilon(f(x))
\end{align*}
In other words,
\begin{align*}
\left( p\in B_\delta(x)
\;\Leftrightarrow\;
d(p,x)<\delta\right)
\quad
\Rightarrow
\quad
\left( f(p)\in B_\varepsilon(f(x))
\;\Leftrightarrow\;
d(f(p),f(x))<\varepsilon\right)
\end{align*}
That is precisely the $\varepsilon$-$\delta$ definition of continuity.
\end{proof}

Below, we again reformulate the definition of continuity from the
$\varepsilon$-$\delta$ or topological definition into one concerning
convergence of sequences.
\begin{thm}{\emph{(Continuity, Sequential)}}
\label{thm:cts-sequential}
Suppose that we have a function $f:X\rightarrow Y$ and a sequence
$\{x_n\}\subset X$ where $x_n\rightarrow x$.  A function is continuous
if and only if
\begin{align*}
  \lim_{n\rightarrow \infty} f(x_n)
  &= f(x)\\
  \Leftrightarrow\qquad
  \lim_{n\rightarrow \infty} f(x_n)
  &= f(\; \lim_{n\rightarrow \infty} x_n \;)
\end{align*}
Note that $\{x_n\}$ is a sequence in metric space $(X,d_X)$, while
$\{f(x_n)\}$ is a sequence in metric space $(Y,d_Y)$.
\end{thm}


\begin{prop}
\label{prop:XtoR}
Given continuous function $f:(X,d_X)\rightarrow (\R,d_1)$, for
any $a\in \R$,
\begin{align*}
  A &= \{x\in X \; |\; f(x) > a\} \; \text{\emph{is open}} \\
  C &= \{x\in X \; |\; f(x) \geq a\} \; \text{\emph{is closed}}
\end{align*}
Similar for $<$ and $\leq$.
\end{prop}
\begin{proof}
Set $A$ is simply $f^{-1}((a,\infty))$. The interval $(a,\infty)$ is an
open set. Since $f$ is continuous, the preimage of $(a,\infty)$ is open.

$C$ can be written as $f^{-1}([a,\infty))$ or, equivalently,
$(f^{-1}((-\infty,a)))^C$, which is the representation we actually
care about. Now since $(-\infty,a)$ is open, its preimage
$f^{-1}((-\infty,a))$ is also open. Since the complement of an open
set is closed, $(f^{-1}((-\infty,a)))^C$ is closed.
\end{proof}
\begin{rmk}
Proposition~\ref{prop:XtoR} is great since continuous functions that
look like $f$ in the proposition come up a lot. We can then use this
proposition, our knowledge of open sets in $\R$, and
Definition~\ref{defn:topcts} to generate open sets in arbitrary space
$X$.\footnote{Copyright Elon Musk.}

For example, suppose that $X$ is the space of bounded functions and
$d_X$ is the sup norm. Tricky space, and I don't know what the hell open
sets look like in that.  But if $f$ is a continuous integral operator
which maps bounded functions in $X$ to real numbers---something covered
by Proposition~\ref{prop:XtoR})---we can recover sets of functions in
$X$ that are ``open'' by doing $f^{-1}((a,\infty))$.  I challenge you to
dream up open sets of bounded functions without this trick.
\end{rmk}

\begin{prop}
Given continuous functions $f:(X,d_X)\rightarrow (Y,d_Y)$ and
$g:(Y,d_Y)\rightarrow (Z,d_Z)$, the composition
\begin{align*}
  h(x) := (g\circ f)(x) = g(f(x))
\end{align*}
is also continuous.
\end{prop}

\begin{thm}{\emph{(Compactness and Continuous Functions)}}
\label{thm:contcompact}
If the function $f:(X,d_X)\rightarrow (Y,d_Y)$ is continuous, then
\begin{align*}
  K\subseteq X \; \text{compact}
  \quad \Rightarrow\quad
  f(K)\subseteq Y \; \text{compact as well}
\end{align*}
\end{thm}
\begin{proof}
We will use the topological definition of compactness, showing that any
open cover of $f(K)$ has a finite subcover. So start with an open cover
of $f(K)$:
\begin{align*}
  f(K) \subseteq \bigcup_{\alpha\in J} C_\alpha
\end{align*}
Since $f$ is continuous, we know that $f^{-1}(C_\alpha)$ is open for any
$\alpha \in J$. Moreover,
\begin{align*}
  K \subseteq \bigcup_{\alpha \in J} f^{-1}(C_\alpha)
\end{align*}
In other words, $\{f^{-1}(C_\alpha)\}$ is an open cover of $K$.
But $K$ is compact. So it has a finite subcover:
\begin{align*}
  \exists \{\alpha_i\}_{i=1}^N
  \quad\text{s.t.}\quad
  K \subseteq \bigcup_{i=1}^N f^{-1}(C_{\alpha_i})
\end{align*}
Now apply $f$ to both sides of the inclusion statement:
\begin{align*}
  f(K) \subseteq
  f\left(\bigcup_{i=1}^N f^{-1}(C_{\alpha_i})\right)
  =\bigcup_{i=1}^N f\left(f^{-1}(C_{\alpha_i})\right)
  =\bigcup_{i=1}^N C_{\alpha_i}
\end{align*}
\end{proof}

\begin{thm}
Given continuous function $f:(X,d_X) \rightarrow (Y,d_Y)$ and
compact $K\subseteq X$, the function $f$ is \emph{uniformly} continuous
on $K$.
\end{thm}

\begin{thm}{\emph{(Limit of Sequence of Bounded Continuous Real Functions)}}
\label{thm:ctslimit}
Consider the sequence $\{f_n\}$ of bounded continuous functions
$f_n:A\rightarrow \R$. If $f_n\rightarrow f$, then $f$ is continuous as
well (where $\R$ is equipped with one of the usual metrics).
\end{thm}
\begin{proof}
We want to show that $f$ is continuous according to the usual
$\varepsilon$-$\delta$ definition, so take $x\in A$ and $\varepsilon>0$
as given. Let $d$ denote one of the usual distance functions for $\R$.
We now use the triangle inequality to bound the desired distance
$d(f(x),f(y))$ by objects we know more about
\begin{align}
 \forall n \qquad d(f(x),f(y))
  &\leq d(f(x),f_n(x)) + d(f_n(x),f(y)) \notag \\
  &\leq d(f(x),f_n(x)) + d(f_n(x),f_n(y)) + d(f_n(y),f(y))\notag\\
  &\leq 2 \; d_\infty(f,f_n) + d(f_n(x),f_n(y))
  \label{Cclosed}
\end{align}
where the last line follows because $d_\infty$ is defined as $\sup_{z\in
A} d(f_n(z), f(z))$.

Now Inequality~\ref{Cclosed} holds for all $n$. The trick will
be to find a particular $N$ so that we can then bound $d_\infty(f,f_N)$
and $d(f_N(x),f_N(y))$. That will then also allow us to use the
continuity of $f_N$ to find a suitable $\delta$ for $f$. Let's do it.

Since $f_n\rightarrow f$ uniformly (or in the $d_\infty$ metric), there
exists an $N$ such that
\begin{align*}
  d_\infty(f_N, f) < \frac{\varepsilon}{4}
\end{align*}
Using that $N$, consider $f_N$. Since it is continuous, there is a
$\delta_N$ such that
\begin{align*}
  d(x,y) < \delta_N
  \quad\Rightarrow\quad
  d(f_N(x),f_N(y)) < \frac{\varepsilon}{2}
\end{align*}
If we use $\delta = \delta_N$ for $f$, then we can use
Inequality~\ref{Cclosed} to conclude
\begin{align*}
  d(x,y) < \delta_N
  \quad\Rightarrow\quad
  d(f(x),f(y))
  &\leq 2 \; d_\infty(f,f_N) + d(f_N(x),f_N(y))\\
  &\leq 2 \cdot \frac{\varepsilon}{4} + \frac{\varepsilon}{2}
  =\varepsilon
\end{align*}
Therefore $f$ is continuous.
\end{proof}

\begin{cor}{\emph{(The Space of Bounded Continuous Functions)}}
The set of bounded continuous functions $\mathscr{C}(A,\R)\subset
\mathscr{B}(A,\R)$ is a closed set in metric space $(\mathscr{B}(A,\R),
d_\infty)$.
\end{cor}

\begin{proof}
To prove that this set is closed, we again use
Proposition~\ref{prop:closed}.

Suppose that we have some sequence of bounded continuous functions
$\{f_n\}\subseteq \mathscr{C}(A,\R)$ converging uniformly to $f$.
Theorem~\ref{thm:ctslimit} says that $f$ is continuous, i.e.  $f\in
\mathscr{C}(A,\R)$ as well. Since $\mathscr{C}(A,\R)$ contains its limit
points, the set is closed by Proposition~\ref{prop:closed}.
\end{proof}



\begin{thm}{\emph{(Weierstrass)}}
\label{thm:weierstrass}
Given metric space $(X,d_X)$, function $f:X\rightarrow \R$, and
compact $K\subseteq X$, there exist maximum and minimum points for
$f$ on $K$
\begin{align*}
  \bar{x} &= \arg \max_{x\in K} f(x) \\
  \munderbar{x} &= \arg \min_{x\in K} f(x) \\
\end{align*}
\end{thm}
i.e.\ $\bar{x}$ and $\munderbar{x}$ are well-defined.
\begin{rmk}
Compactness is so important because it ensures that the set is ``finite
enough'' is some sense. We know that the max always exists for finite
sets. Now we know that the max (not just the sup) always exists when all
open covers of some set have finite subcovers. It's in the same spirit.
\end{rmk}
\begin{proof}
(Theorem~\ref{thm:weierstrass})
Since $K\subseteq X$ is compact, Theorem~\ref{thm:contcompact} says that
the set $f(K)$ is compact in $\R$.  By
Theorem~\ref{thm:heine-borel}, the set $f(K)$ is therefore closed and
bounded.

Since $f(K)\subseteq \R$ is bounded, we can apply
Proposition~\ref{prop:subseq}, which furnishes a sequence
$\{x_n\}\subseteq f(K)$ that converges to $\sup f(K)$.

Since $f(K)$ is closed and $\{x_n\}\subseteq f(K)$, the limit $\sup
f(K)$ must be in $f(K)$.  Therefore, the set $f(K)$ has a max, and we
can find $\bar{x}$ such that
\begin{align*}
  f(\bar{x}) = \sup f(K) = \sup_{x\in K} f(x)
\end{align*}
\end{proof}


\section{Optimization}

\subsection{Contraction Mapping Theorem and Brouwer's Fixed Point Theorem}
\begin{defn}{(Operator)}
If $X$ is some function space
\begin{align*}
  X = \{f:A\rightarrow B\}
\end{align*}
then an \emph{operator}, denoted $T$, is a function
\begin{align*}
  T:X&\rightarrow X \\
  f&\mapsto Tf
  \qquad \forall f\in X
\end{align*}
It is a ``function of functions'', taking elements $f\in X$ and mapping
them to some other element $Tf\in X$. Though of course $T$ is a
function, the word ``operator'' is used to distinguish $T$ from the
functions it acts on.

To avoid overly cumbersome notation, I follow convention and write $Tf$
rather than $T(f)$ when applying operator $T$ to function $f$.
\end{defn}

\begin{ex}
Here's a simple example of an operator $T:X\rightarrow \R$. It acts on
the space of bounded and integrable functions
\begin{align*}
  X = \{f: A\subseteq \R \rightarrow \R\}
\end{align*}
where $A$ is bounded.
The operator is defined
\begin{align*}
  Tf = \int_A f(x) \; dx
\end{align*}
It takes a function $f$, integrates it, and returns a number.
Therefore, $T$ is an operator.
\end{ex}

\begin{defn}{(Contraction Map)}
In metric space $(X,d)$, the function $T: X\rightarrow X$ is a
\emph{contraction map} if and only there exists a $\beta \in (0,1)$ such
that
\begin{align*}
  d(T(x), T(y)) < \beta \; d(x,y)
  \qquad x,y\in X
\end{align*}
$\beta$ is called the \emph{modulus} of $T$.
\end{defn}

\begin{prop}{\emph{(Contraction Maps are Continuous)}}
If $T$ is a contraction map, then $T$ is continuous.
\end{prop}
\begin{proof}
Suppose $\varepsilon>0$ given. Then
\begin{align*}
  d(T(x),T(y)) < \beta \; d(x,y)
\end{align*}
Sub in $\delta$ for $d(x,y)$ and solve for $\delta$
\begin{align*}
  \beta \delta = \varepsilon
  \quad\Rightarrow\quad
  \delta = \frac{\varepsilon}{\beta}
\end{align*}
\end{proof}

\begin{thm}{\emph{(Banach Fixed Point, Contraction Mapping Theorem)}}
\label{thm:banach}
Suppose that $T:X\rightarrow X$ is a contraction map and $(X,d)$ is a
complete metric space. Then
\begin{enumerate}
  \item $\exists! \; x^*$ such that $T(x^*)=x^*$.\footnote{%
      $\exists!$ means ``there exists a unique''}
  \item For all $x_0 \in X$,
    \begin{align}
      \label{thm:fixed-bound1}
      d(T^n(x_0), x^*) \leq \beta^n d(x_0,x^*)
    \end{align}
    where $T^n$ is the contraction map applied $n$ times.

    So applying the contraction map many, many times to any starting
    point $x_0$ will get you closer and closer to the fixed point at
    each iteration.
\end{enumerate}
\end{thm}

\begin{proof}
First, given $x_0$, construct sequence $\{x_n\}$ by defining the $n$th
term to be
\begin{align}
  \label{contraction-map-seq}
  x_n = T^n(x_0)
\end{align}
Now for the proof. It's long, so we do it in parts
\begin{enumerate}
\item
\emph{Show $\{x_n\}$ Converges}:
Let's first prove that $x_n$ converges to \emph{something}. Since the
space $(X,d)$ is complete, it's enough to show that the sequence is
Cauchy.  And to show that $\{x_n\}$ is Cauchy we want to find a bound of
the form
\begin{align*}
  d(x_n,x_m) < G(\min\{m,n\})
\end{align*}
Without loss of generality, suppose that $m\geq n$, or equivalently
$m=n+p$ for some integer $p\geq 0$. Then we can use the triangle
inequality $p$ times to write
\begin{align}
  d(x_n,x_m) = d(x_n,x_{n+p})
  &\leq d(x_n,x_{n+1}) + d(x_{n+1},x_{n+p}) \notag \\
  &\leq d(x_n,x_{n+1}) + d(x_{n+1},x_{n+2}) + d(x_{n+2},x_{n+p})\notag\\
  &\;\; \vdots\notag\\
  &\leq \sum^{p-1}_{i=0} d(x_{n+i},x_{n+i+1})
  \label{fixedpt-series}
\end{align}
Now let's take a look at each of the constituent terms in the
Series~\ref{fixedpt-series} for $i>0$, using the fact that $T$ is a
contraction map:
\begin{alignat*}{3}
  d(x_{n+1},x_{n+2}) &=
  d(T(x_{n}),T(x_{n+1}))
  &&\leq
  \beta d(x_{n},x_{n+1}) \\
  d(x_{n+2},x_{n+3}) &=
  d(T(x_{n+1}),T(x_{n+2}))
  &&\leq
  \beta d(x_{n+1},x_{n+2}) \\
  \vdots \quad & \quad\qquad \vdots
\end{alignat*}
In general, the inequalities above can be summarized as follows:
\begin{align*}
  d(x_{n+i},x_{n+i+1}) &=
  d(T(x_{n+i-1}),T(x_{n+i}))
  \leq
  \beta d(x_{n+i-1},x_{n+i})
  \qquad i = 1,\ldots,p-1
\end{align*}
By recursive substitution for each $i$, we can even bound each term by a
function of $\beta$ and $d(x_n,x_{n+1})$, the first term in
Series~\ref{fixedpt-series}:
\begin{align*}
  d(x_{n+i},x_{n+i+1})
  &\leq
  \beta^i d(x_{n},x_{n+1})
  \qquad i = 1,\ldots,p-1
\end{align*}
This allows us to rewrite Series~\ref{fixedpt-series} and the inequality
for $d(x_n,x_{n+p})$ as
\begin{align}
  d(x_n,x_{n+p})
  \leq \sum^{p-1}_{i=0} d(x_{n+i},x_{n+i+1})
  &\leq \sum^{p-1}_{i=0} \beta^i d(x_{n},x_{n+1})\notag\\
  &\leq  d(x_{n},x_{n+1})\sum^{\infty}_{i=0} \beta^i\notag \\
  \Rightarrow
  d(x_n,x_{n+p})
  &\leq  d(x_{n},x_{n+1})\times \frac{1}{1-\beta}
  \label{fixedpt-ineq}
\end{align}
Okay, that's pretty good. Let's just bound $d(x_n,x_{n+1})$, again using
recursive substitution
\begin{align*}
  d(x_n,x_{n+1})
  = d(T(x_{n-1}), T(x_n))
  &\leq \beta d(x_{n-1}, x_n)\\
  &\leq \beta \cdot \beta d(x_{n-2}, x_{n-1})\\
  &\;\; \vdots \\
  \Rightarrow\quad
  d(x_n,x_{n+1})
  &\leq \beta^n d(x_{0}, x_1)
\end{align*}
Substituting this back into Inequality~\ref{fixedpt-ineq}, we get
\begin{align*}
  d(x_n,x_{n+1})
  &\leq \frac{\beta^n}{1-\beta} d(x_0,x_1)
\end{align*}
Since we want to bound the LHS of the last inequality by $\varepsilon$,
equate and solve for $N$:
\begin{align*}
  \varepsilon &= \frac{\beta^N}{1-\beta} d(x_0,x_1) \\
  \Rightarrow\quad
  N &=
  \frac{1}{\ln \beta}
  \left[
  \ln(1-\beta)+ \ln \varepsilon - \ln(d(x_0,x_1))
  \right]
\end{align*}
This is the choice of $N$ with $\varepsilon$ given.
Therefore $\{x_n\}$ is Cauchy and, since $X$ is complete, convergent.

\item
\emph{Show $x^*$ Exists}: Call $x$ the limit of $\{x_n\}$, which we just
showed exists. In notation, that means
\begin{align*}
  x = \limn x_n = \limn T(x_{n-1})
\end{align*}
By Theorem~\ref{thm:cts-sequential}, the continuity of $T$ means we can
exchange the limit and the function:
\begin{align*}
  x = \limn T(x_{n-1})
  =
  T\left(\limn x_{n-1}\right)
  =
  T\left(x\right)
\end{align*}
Hence, the limit of the sequence $\{x_n\}$ is the fixed point, i.e.\
$x^*=x = \limn x_n$.

\item
\emph{Show Uniqueness}:
Suppose that there were two fixed points $x^*$ and $y^*$. Then
\begin{align*}
  d(x^*, y^*) &= d(T(x^*), T(y^*))
  \quad\text{Since fixed points} \\
  \Rightarrow\quad
  d(x^*, y^*) &\leq \beta d(x^*, y^*)
  \; \quad
  \qquad\text{Since $d(T(x^*), T(y^*)) \leq \beta d(x^*, y^*)$}
\end{align*}
But since $\beta\in (0,1)$, this is impossible unless $d(x^*,y^*)=0$.

\item
\emph{Bound $d(T^n(x_0),x^*)$}:
Similar to what we did above
\begin{align*}
  d(x_n, x^*) = d(T(x_{n-1},T(x^*))
  &\leq \beta d(x_{n-1},x^*)\\
  & \;\; \vdots\\
  &\leq \beta^n d(x_0,x^*)
\end{align*}
\end{enumerate}
\end{proof}

The last theorem is a very useful bit of theory, but we can do a bit
better than the bound given by Inequality~\ref{thm:fixed-bound1}.
Specifically, to be able to tell how far $T^n(x_0)$ is from the fixed
point $x^*$, Theorem~\ref{thm:banach} requires that we know
$d(x_0,x^*)$. Well, a lot of times we don't know. And a lot of times,
we use the contraction mapping theorem to motivate forming a sequence
$T^n(x_0)$ that converges to some fixed point we don't know.
So can we bound $d(T^n(x_0),x^*)$ with stuff we do know?

In fact, we can bound $d(T^n(x_0), x^*)$ by a function of $d(T^n(x_0),
T^{n+1}(x_0))$ instead, which is the change in the sequence
$\{T^n(x_0)\}$. In practical terms, that means ``Keep iterating on the
sequence $\{T^n(x_0)\}$, and if you see that the change in the sequence
from element to element is pretty small, you can be confident that
you're close to the fixed point.'' You don't have to know what the fixed
point is. You just need to compute $T^{n+1}(x_0) - T^n(x_0)$, which is
always feasible, in general. This is the next result.
\begin{cor}
Consider contraction map $T:X\rightarrow X$ with modulus $\beta$ in
complete metric space $(X,d)$. Given any starting point $x_0\in X$,
\begin{align*}
  d(T^n(x_0), x^*)
  \leq \frac{d(T^n(x_0),T^{n+1}(x_0))}{1-\beta}
\end{align*}
\end{cor}
\begin{proof}
Again, form the sequence $\{x_n\}$ as we did above in
(\ref{contraction-map-seq}).
\begin{align*}
  x_n = T^n(x_0)
\end{align*}
The goal is to bound the distance between $T^n(x_0)$ and $x^*$ using
only information about the change
\begin{align*}
  d(T^n(x_0), x^*)
  &\leq
  G\left(\;d(T^n(x_0), T^{n+1}(x_0))\;\right)\\
  \Leftrightarrow\qquad
  d(x_n, x^*)
  &\leq
  G\left(\;d(x_n, x_{n+1})\;\right)
\end{align*}
Using the triangle inequality, the fact that $T$ is a contraction map,
and the fact that $x^*$ is a fixed point,
\begin{align*}
  d(x_n, x^*)
  &\leq d(x_n,x_{n+1}) + d(x_{n+1},x^*) \\
  &\leq d(x_n,x_{n+1}) + d(T(x_n),T(x^*)) \\
  &\leq d(x_n,x_{n+1}) + \beta d(x_n,x^*) \\
  \Rightarrow\quad
  d(x_n, x^*)
  &\leq \frac{d(x_n,x_{n+1})}{1-\beta}
\end{align*}
Since $d(x_n,x^*)=d(T^n(x_0),x^*)$, we have the desired result.
\end{proof}

\begin{defn}{($T$-Invariant Sets)}
Given an operator/function $T:X\rightarrow X$ and some set $C\subseteq
X$, we say that the set $C$ is \emph{$T$-invariant} if
\begin{align*}
  T(C)\subseteq C
\end{align*}
\end{defn}

\begin{prop}
Suppose that $T: X\rightarrow X$ is a contraction map for complete space
$X$. Let $x^*$ denote the unique fixed point. Then
\begin{enumerate}
  \item If $C\subseteq X$ is a closed, nonempty, $T$-invariant set, then
    $x^* \in C$.
  \item If $C' \subseteq C \subseteq X$ where $C$ is closed but $C'$ is
    an arbitrary subset such that $T(C) \subseteq C'$, then $x^* \in
    C'$.
\end{enumerate}
\end{prop}
\begin{proof}
(Part 1) Start with some $x_0\in C$. Since $C$ is $T$-invariant,
$T^n(x_0) \in C$ for all $n$. Moreover, by Theorem~\ref{thm:banach},
the sequence $\{T^n(x_0)\}\subseteq C$ converges to $x^*$. Since $C$ is
closed, the set must contain its limit point by
Proposition~\ref{prop:closed}, hence $x^*\in C$.

(Part 2) Again, the assumptions of this part imply that $C$ is
$T$-invariant since $T(C)\subseteq C' \subseteq C$.  And since $C$ is
closed, $x^*\in C$ by Part 1.

Finally, by assumption $T(x) \in C'$ for all $x \in C$, and we just said
$x^*\in C$. So $T(x^*)\in C'$ too.
\end{proof}




%% APPPENDIX %%

% \appendix




\end{document}


%%%%%%%%%%%%%%%%%%%%%%%%%%%%%%%%%%%%%%%%%%%%%%%%%%%%%%%%%%%%%%%%%%%%%%%%
%%%%%%%%%%%%%%%%%%%%%%%%%%%%%%%%%%%%%%%%%%%%%%%%%%%%%%%%%%%%%%%%%%%%%%%%
%%%%%%%%%%%%%%%%%%%%%%%%%%%%%%%%%%%%%%%%%%%%%%%%%%%%%%%%%%%%%%%%%%%%%%%%

%%%% SAMPLE CODE %%%%%%%%%%%%%%%%%%%%%%%%%%%%%%%%%%%%%%

    %% VIEW LAYOUT %%

        \layout

    %% LANDSCAPE PAGE %%

        \begin{landscape}
        \end{landscape}

    %% BIBLIOGRAPHIES %%

        \cite{LabelInSourcesFile}  %Use in text; cites
        \citep{LabelInSourcesFile} %Use in text; cites in parens

        \nocite{LabelInSourceFile} % Includes in refs w/o specific citation
        \bibliographystyle{apalike}  % Or some other style

        % To ditch the ``References'' header
        \begingroup
        \renewcommand{\section}[2]{}
        \endgroup

        \bibliography{sources} % where sources.bib has all the citation info

    %% SPACING %%

        \vspace{1in}
        \hspace{1in}

    %% URLS, EMAIL, AND LOCAL FILES %%

      \url{url}
      \href{url}{name}
      \href{mailto:mcocci@raidenlovessusie.com}{name}
      \href{run:/path/to/file.pdf}{name}


    %% INCLUDING PDF PAGE %%

        \includepdf{file.pdf}


    %% INCLUDING CODE %%

        %\verbatiminput{file.ext}
            %   Includes verbatim text from the file

        \texttt{text}
            %   Renders text in courier, or code-like, font

        \matlabcode{file.m}
            %   Includes Matlab code with colors and line numbers

        \lstset{style=bash}
        \begin{lstlisting}
        \end{lstlisting}
            % Inline code rendering


    %% INCLUDING FIGURES %%

        % Basic Figure with size scaling
            \begin{figure}[h!]
               \centering
               \includegraphics[scale=1]{file.pdf}
            \end{figure}

        % Basic Figure with specific height
            \begin{figure}[h!]
               \centering
               \includegraphics[height=5in, width=5in]{file.pdf}
            \end{figure}

        % Figure with cropping, where the order for trimming is  L, B, R, T
            \begin{figure}
               \centering
               \includegraphics[trim={1cm, 1cm, 1cm, 1cm}, clip]{file.pdf}
            \end{figure}

        % Side by Side figures: Use the tabular environment


